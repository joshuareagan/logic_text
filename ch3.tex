
%%%%%%%%%%%%%%%%%%%%%%%%%%%%%%%%%%%%%%%%%%%%%%%%%%
\chapter{Quantifier Language I}\label{quantifierlogic1}
%%%%%%%%%%%%%%%%%%%%%%%%%%%%%%%%%%%%%%%%%%%%%%%%%%
% \AddToShipoutPicture*{\BackgroundPicB}

%%%%%%%%%%%%%%%%%%%%%%%%%%%%%%%%%%%%%%%%%%%%%%%%%%
\section{The Language \GQL{}1}
%%%%%%%%%%%%%%%%%%%%%%%%%%%%%%%%%%%%%%%%%%%%%%%%%%

%\setcounter{DefThm}{0}

\subsection{Sentences of \GQL{}1}\label{Sec:GQLSymbols1}
\GSL{} allows us to investigate certain aspects of logical consequence, but leaves out a great deal of interest.  The \mention{atoms} of \GSL{} are sentence letters, which are each assigned either $\TrueB$ or $\FalseB$ by a model.  Sentence letters can only represent declarative sentences of the English language, which means that \GSL{} is too coarse-grained to capture the logical meaning of certain \emph{parts} of a sentence.  For example:

\begin{RESTARTmenumerate}
\item All women are mortal.
\item Ophelia is a woman.

Therefore,

\item Ophelia is mortal.
\end{RESTARTmenumerate}

\noindent{}The third sentence is a logical consequence of the first two, but we cannot use \GSL{} to represent this as an entailment.  Let sentence letter $\Al$ stand for \mention{All women are mortal,} let $\Bl$ stand for \mention{Ophelia is a woman,} and let $\Cl$ stand for \mention{Ophelia is mortal}.  
The entailment claim $\Al, \Bl \:\sdtstile{}{}\: \Cl$ does not hold, because there is a model $\IntA$ that assigns $\TrueB$ to $\Al$ and $\Bl$, but assigns $\FalseB$ to $\Cl$.  We need a formal language more expressive than \GSL{}.

This new language needs symbols that represent named objects, including people like Ophelia.  It also needs symbols that stand for the predicates of English.  Predicates correspond roughly to what you get if you take an English sentence and remove the subject, leaving a blank, e.g.: 

\begin{menumerate}
	\item \mention{John is tall} $\Rightarrow$ \mention{\_\_\_\_\_\_ is tall}
	\item \mention{Ophelia is a woman} $\Rightarrow$ \mention{\_\_\_\_\_\_ is a woman}
	\item \mention{Ophelia is mortal} $\Rightarrow$ \mention{\_\_\_\_\_\_ is mortal}	
\end{menumerate}

\noindent{}To apply a predicate without invoking a name we use a variable, which functions somewhat like a pronoun in English.  And to account for the word \mention{all} in our new language, we use a \emph{quantifier}.  (We discuss quantifiers a bit later.)  In this chapter we outline a formal language with these features, and thus which can capture the kind of logical consequence exhibited above.

Many logic texts call the following language \PL{}, for \idf{predicate language}. 
We use \mention{\QL{}} for \idf{quantifier language}, because the quantifiers are more important than the predicates. However, before turning to the full language of \GQL{} (in the next chapter), we first consider a simpler sublanguage which we call \mention{\GQL{}1}.  We call it \mention{\GQL{}1} because the predicates will all be 1-place.\footnote{In \GQL{} there will be many-place predicates.  The basics of a less formal version of \GQL{}1 were developed by Aristotle over 2,000 years before Gottlob Frege and others developed the full language we're calling \GQL{}.  \GQL{} was a big step for mankind.}
 
\GQL{}1 has all the basic symbols of \GSL{}, plus a few more. 
\begin{majorILnc}{\LnpDC{Symbols of GQL1}}
The \df{basic symbols} of \GQL{}1 are:
\begin{cenumerate}
\item Logical Connectives: those of \GSL{}, plus $\forall$ and $\exists$
\item Punctuation Symbols: those of \GSL{}
\item Sentence Letters: those of \GSL{}
\item Individual Constants: $\constant{a}$, $\constant{b}$, $\constant{c}$, $\constant{d}$, $\ldots$, $\constant{p}$, $\constant{a}_1$, $\constant{b}_1$, $\constant{c}_1$, $\ldots$, $\constant{p}_1$, $\constant{a}_2$, $\ldots$
\item Individual Variables:\index{variables!individual (GQL1)|textbf} $\variable{u}$, $\variable{v}$, $\variable{w}$, $\variable{x}$, $\variable{y}$, $\variable{z}$, $\variable{u}_1$, $\variable{v}_1$, $\ldots$, $\variable{z}_1$, $\variable{u}_2$, $\ldots$
\item 1-Place Predicates: $\Ap{'}$, $\Bp{'}$, $\ldots$, $\Tp{'}$, $\Ap{'}_1$, $\Bp{'}_1$, $\ldots$, $\Tp{'}_1$, $\Ap{'}_2$, $\Bp{'}_2$, $\ldots$
\end{cenumerate}
\end{majorILnc}
\noindent{}The most prominent additions are the new logical connectives, the two quantifiers.  The first, \mention{$\forall$}, is called the \idf{universal quantifier}. 
It corresponds to the words \mention{all} or \mention{every} in English. 
The second, \mention{$\exists$}, is called the \idf{existential quantifier}. 
It corresponds to \mention{there exists}, \mention{there is}, or \mention{some}, as in \mention{Some elephants live a long time.}\footnote{Although Frege, Peirce, and Mitchell first introduced quantifiers, the notation used here comes from Russell, who Church \citeyearpar[288]{Church1956} says modified Peano's notation.}

The individual constants of \GQL{} correspond roughly to names in English. 
They are lowercase Roman letters that start at \mention{$\constant{a}$} and stop at \mention{$\constant{p}$}, and then they start at \mention{$\constant{a}$} again with subscripted integers. So we have, for example, $\constant{a_1}$, $\constant{a_2}$, $\constant{a_3}$, and so on. 

Next, the individual variables correspond roughly to pronouns in English. 
They are Roman lowercase letters that go from \mention{$\variable{u}$} to \mention{$\variable{z}$} and then start at \mention{$\variable{u}$} again with subscripted positive integers, e.g. \mention{$\variable{u}_1$}. 

We also have 1-place predicates in \GQL{}1. The one-place predicates are capital Roman letters going from \mention{$\Ap{'}$} to \mention{$\Tp{'}$} and then starting from \mention{$\Ap{'}$} again with subscripted integers, e.g. \mention{$\Ap{'}_{1}$}.  There are an infinite number of each of the individual constants, variables, and 1-place predicates. 
This is so we never run out of \GQL{}1 symbols when analyzing some sentence or argument, no matter how complex or long it is.

\subsection{Formulas of \GQL{}1}\label{Formulas of GQL1}
Before we can define \emph{sentences} of \GQL{}1 we must first define a larger set of strings called \mention{formulas.}\index{formulas} 
But before that we must introduce another kind of metavariable to MathEnglish, one that ranges over for \GQL{}1 individual variables.\index{variables!MathEnglish}\footnote{MathEnglish variables are symbols of the metalanguage while individual variables of \GQL{}1 are symbols of the object language.} 
We use lowercase Greek letters to stand for variables of \GQL{}1, usually but not necessarily always from the beginning of the Greek alphabet (e.g., \mention{$\ALPHA$} and \mention{$\BETA$}). Although we also used lowercase Greek letters as variables for \GSL{} sentences and will continue to do so for \GQL{}1 sentences and formulas, confusion shouldn't arise as we typically use \mention{$\CAPPHI$}, \mention{$\CAPPSI$}, and \mention{$\CAPTHETA$} for \GSL{} and \GQL{}1 sentences and use \mention{$\ALPHA$} and \mention{$\BETA$} for \GQL{}1 variables. 
\begin{majorILnc}{\LnpDC{Definition of Formula of GQL1}} The \nidf{formulas} \underdf{of \GQL{}1}{formulas} are given by the following recursive definition:
\begin{description}
\item[Base Clauses:] \hfill{}
\begin{cenumerate}
\item A sentence letter is a formula.
\item\label{atomic pred} A 1-place predicate followed by one individual constant or variable is a formula.
\end{cenumerate}
\item[Generating Clauses:] \hfill{}
\begin{cenumerate}
\item If $\CAPPHI$ is a formula then so is $\negation{\CAPPHI}$.
\item If $\CAPPHI$ and $\CAPTHETA$ are formulas then so are $\parhorseshoe{\CAPPHI}{\CAPTHETA}$ and $\partriplebar{\CAPPHI}{\CAPTHETA}$.
\item\label{GQL conj disj} If all of $\CAPPHI_1,\CAPPHI_2,\CAPPHI_3,\CAPPHI_4,\ldots,\CAPPHI_{\integer{n}}$ are formulas (where $n$ is an integer $\geq 2$) then so are $\parconjunction{\CAPPHI_1}{\conjunction{\CAPPHI_2}{\conjunction{\CAPPHI_3}{\conjunction{\CAPPHI_4}{\conjunction{\ldots}{\CAPPHI_{\integer{n}}}}}}}$ and $\pardisjunction{\CAPPHI_1}{\disjunction{\CAPPHI_2}{\disjunction{\CAPPHI_3}{\disjunction{\CAPPHI_4}{\disjunction{\ldots}{\CAPPHI_{\integer{n}}}}}}}$.
\item\label{GQL quant} If $\ALPHA$ is a \GQL{}1 variable and $\CAPPHI$ is a formula that does not contain an expression of the form $\universal{\ALPHA}$ or $\existential{\ALPHA}$, then $\universal{\ALPHA}\CAPPHI$ and $\existential{\ALPHA}\CAPPHI$ are formulas.
\end{cenumerate}
\item[Closure Clause:] A string of symbols is a formula only if it can be generated by the clauses above.
\end{description}
\end{majorILnc}
\noindent{}For example, $\App{'}{\constant{b}}$ is a formula, and so is $\Dpp{'}{\variable{x}_4}$. Each of these formulas is atomic.  Formulas of the form $\universal{\ALPHA}\CAPPHI$ are called \underidf{universal}{formulas} formulas and formulas of the form $\existential{\ALPHA}\CAPPHI$ are called \underidf{existential}{formulas}. 
Here are some additional examples of \GQL{}1 formulas. 
\begin{multicols}{2}
\begin{enumerate}
\item $\universal{\variable{x}}\Jpp{'}{\variable{x}}$ 
\item $\negation{\existential{\variable{y}}\Kpp{'}{\variable{x}}}$ 
%\item $\universal{\variable{y}}\Gppp{''}{\variable{x}}{\variable{y}}$ 
%\item $\existential{\variable{y}}\Gppp{''}{\variable{x}}{\variable{y}}$ 
%\item $\universal{\variable{z}}\Gppp{''}{\variable{x}}{\variable{y}}$ 
\item $\existential{\variable{z}}\Lpp{'_{12}}{\constant{b}}$
\item $\existential{\variable{y}}\negation{\universal{\variable{x}}\Gpp{'}{\variable{y}}}$ 
%\item $\existential{\variable{x}}\universal{\variable{y}}\Gppp{''}{\variable{x}}{\variable{y}}$ 
%\item $\universal{\variable{x}}\universal{\variable{y}}\Gppp{''}{\variable{x}}{\variable{y}}$ 
\item $\universal{\variable{x}}\existential{\variable{y}}\Hpp{'}{\variable{z}}$ 
\item $\parhorseshoe{\universal{\variable{x}}\universal{\variable{z}}\Ppp{'}{\variable{z}_{176}}}{\universal{\variable{y}}\Gpp{'}{\variable{y}}}$ 
%\item $\universal{\variable{x}}\existential{\variable{z}}\Gppp{''}{\variable{x}}{\variable{y}}$
\end{enumerate}
\end{multicols}
\noindent{}Contrarily, $\universal{\variable{x}}\universal{\x}\Gpp{'}{\variable{x}}$ is \emph{not} a formula.  That's because it's of the form $\universal{\variable{x}}\CAPPHI$ where $\CAPPHI$ is a formula that contains the expression $\universal{\variable{x}}$.\footnote{Continuing the practice started in section \ref{use mention comment}, we do not always put symbols, expressions, and sentences of \GQL{}1 that are mentioned (instead of used) in single quotes. 
For example, the tokens of the universal and existential quantifiers in definition \mvref{Bound Variable} should, strictly speaking, be in quotes because they \emph{mention} the symbols. 
Being stringent in the use of single quotes---and overly sensitive to the \distinction{use}{mention} distinction in general---can cloud what are relatively clear and straightforward concepts.
Wherever it may be helpful, we provide footnotes with more rigorous and detailed explanation.} Neither is $\universal{\constant{a}}\Gp{'}\constant{a}$, because $\forall$ \emph{must} be paired with a variable, but \mention{$\constant{a}$} is a constant.

Finally, we have unofficial formulas, just as we had unofficial sentences in \GSL{} (compare with def. \pmvref{Unofficial Sentence of GSL}).
\begin{majorILnc}{\LnpDC{Unofficial Formula of GQL1}}
A string of symbols is an \nidf{unofficial} formula\index{formulas!unofficial|textbf} \Iff we can obtain it from an official formula by
\begin{cenumerate}
\item deleting outer parentheses,
\item replacing official parentheses ( ) with square brackets [ ] or curly brackets \{ \}, or
\item omitting primes $'$ on a predicate letter.
\end{cenumerate}
\end{majorILnc}
\noindent{}From an unofficial formula we can unambiguously reconstruct the corresponding official formula.
\subsection{Other Properties of Formulas}\label{Other Properties of Formulas1} 
As in section \ref{Other Properties of GSL Sentences} for sentences of \GSL{}, we define the concepts of subformula, order, main connective, and construction tree for formulas of \GQL{}1. 
\begin{majorILnc}{\LnpDC{GQL Subformulas}}
The following clauses define when one formula is a \df{subformula} of another:
\begin{cenumerate}
	\item Every formula is a subformula of itself.
	\item $\CAPPHI$ is a subformula of $\negation{\CAPPHI}$.
	\item $\CAPPHI$ and $\CAPTHETA$ are subformulas of $\parhorseshoe{\CAPPHI}{\CAPTHETA}$ and $\partriplebar{\CAPPHI}{\CAPTHETA}$.
	\item Each of $\CAPPHI_1,\CAPPHI_2,\ldots,\CAPPHI_{\integer{n}}$ is a subformula of $\parconjunction{\CAPPHI_1}{\conjunction{\CAPPHI_2}{\conjunction{\ldots}{\CAPPHI_{\integer{n}}}}}$\\ and $\pardisjunction{\CAPPHI_1}{\disjunction{\CAPPHI_2}{\disjunction{\ldots}{\CAPPHI_{\integer{n}}}}}$.
	\item $\CAPPHI$ is a subformula of $\universal{\ALPHA}\CAPPHI$ and $\existential{\ALPHA}\CAPPHI$.
	\item (Transitivity) If $\CAPPHI$ is a subformula of $\CAPTHETA$ and $\CAPTHETA$ is a subformula of $\CAPPSI$, then $\CAPPHI$ is a subformula of $\CAPPSI$.
	\item That's all. 
\end{cenumerate}

\end{majorILnc}
\noindent{}Importantly, the quantifier phrase is \emph{not} a subformula. 
Thus, $\universal{\variable{x}}\universal{\variable{z}}\Gpp{'}{\variable{x}}$ has three subformulas: $\universal{\variable{x}}\universal{\variable{z}}\Gpp{'}{\variable{x}}$, $\universal{\variable{z}}\Gpp{'}{\variable{x}}$, and $\Gpp{'}{\variable{x}}$. 
Neither $\universal{\variable{x}}$ nor $\universal{\variable{x}}\universal{\variable{z}}$ is a subformula.
Middle parts of the formula cannot be removed to form a subformula. So $\universal{\variable{x}}\Gpp{'}{\variable{x}}$ isn't one either.
\begin{majorILnc}{\LnpDC{GQL Order}}
The \df{order} of a formula is defined parallel to that of an \GSL{} sentence (see def. \pmvref{Order}) with extra clauses specifying that adding a quantifier increases the order by one.
The following clauses define the \df{order} of a formula. Let $\ORD{\CAPPHI}$ be the order of $\CAPPHI$. Then: 
\begin{cenumerate}
\item If $\CAPPHI$ is a sentence letter then $\ORD{\CAPPHI}=1$.
\item If $\Ppp{'}{}$ is a predicate letter and $\ALPHA$ is a constant or variable then $\ORD{\Ppp{'}{\ALPHA}}=1$.
\item For any formula $\CAPPHI$, $\ORD{\negation{\CAPPHI}}=\ORD{\CAPPHI}+1$.
\item For any formulas $\CAPPHI$ and $\CAPTHETA$, $\ORD{\parhorseshoe{\CAPPHI}{\CAPTHETA}}$ is one greater than the max of $\ORD{\CAPPHI}$ and $\ORD{\CAPTHETA}$. Likewise, $\ORD{\partriplebar{\CAPPHI}{\CAPTHETA}}$ is one greater than the max of $\ORD{\CAPPHI}$ and $\ORD{\CAPTHETA}$.
\item For any formulas $\CAPPHI_1,\ldots,\CAPPHI_\integer{n}$, $\ORD{\parconjunction{\CAPPHI_1}{\conjunction{\ldots}{\CAPPHI_\integer{n}}}}$ is one greater than the max of $\ORD{\CAPPHI_1}$, $\ldots$, $\ORD{\CAPPHI_\integer{n}}$.
\item For any formulas $\CAPPHI_1,\ldots,\CAPPHI_\integer{n}$, $\ORD{\pardisjunction{\CAPPHI_1}{\disjunction{\ldots}{\CAPPHI_\integer{n}}}}$ is one greater than the max of $\ORD{\CAPPHI_1}$, $\ldots$, $\ORD{\CAPPHI_\integer{n}}$. 
\item For formulas $\universal{\ALPHA}\CAPPHI$ and $\existential{\ALPHA}\CAPPHI$, $\ORD{\universal{\ALPHA}\CAPPHI}=\ORD{\existential{\ALPHA}\CAPPHI}=\ORD{\CAPPHI}+1$.
\item That's all.
\end{cenumerate}
\end{majorILnc}
\begin{majorILnc}{\LnpDC{GQL Main Connective}}
The \nidf{main connective}\index{main connective!of GQL|textbf} of a formula is the connective token (or tokens) that occur(s) in the formula but in no proper subformula.
\end{majorILnc}
\begin{majorILnc}{\LnpDC{GQL Construction Tree}}
The \df{construction tree} of a formula is defined as with \GSL{} (def. \pmvref{Construction Tree}) but with the obvious extension for quantifiers. 
The order of a formula is the height of the construction tree's longest branch, measured by counting nodes.  Each node in the tree (including the bottom node) is a subformula, and the main connective is the connective added at the very bottom of the tree. 
\end{majorILnc}
\begin{majorILnc}{\LnpEC{GQL1SubformulaPropertiesExampleA}}
Consider the (unofficial) formula $\horseshoe{\universal{\variable{x}}\Hp{'\variable{x}}}{\universal{\variable{x}}\Gp{'\variable{x}}}$. 
Its construction tree is:
\begin{center}
	\begin{tikzpicture}[grow=up]
	\tikzset{level distance=50pt}
	\tikzset{sibling distance=40pt}
	\tikzset{every tree node/.style={align=center,anchor=north}}
	\Tree%http://angasm.org/papers/qtree/    http://www.ling.upenn.edu/advice/latex/qtree/qtreenotes.pdf
	[.{$\horseshoe{\universal{\variable{x}}\Hp{'\variable{x}}}{\universal{\variable{x}}\Gp{'\variable{x}}}$}
	[.{$\universal{\variable{x}}\Gp{'\variable{x}}$} %!{\qsetw{3in}} 
	  [.{$\Gp{'\variable{x}}$}
	  ]
	]
	[.{$\universal{\variable{x}}\Hp{'\variable{x}}$}
	  [.{$\Hp{'\variable{x}}$}
	  ]
	]
	]%
	%\caption{Example formula tree}
	%\label{fig:ExampleFormulaTree}
	\end{tikzpicture}
\end{center}
The height of the construction tree is $3$, so that's the order of the formula.  It has five subformulas:
\begin{enumerate}[label=(\arabic*), leftmargin=1.85\parindent,
labelindent=.35\parindent, labelsep=*, itemsep=0pt]%,start=1
\item $\horseshoe{\universal{\variable{x}}\Hp{'\variable{x}}}{\universal{\variable{x}}\Gp{'\variable{x}}}$
\end{enumerate}
\vspace*{-.5cm}
\begin{multicols}{2}
\begin{enumerate}[label=(\arabic*), leftmargin=1.85\parindent,
labelindent=.35\parindent, labelsep=*, itemsep=0pt, start=2]
\item $\universal{\variable{x}}\Hp{'\variable{x}}$
\item $\universal{\variable{x}}\Gp{'\variable{x}}$
\item $\Hp{'\variable{x}}$
\item $\Gp{'\variable{x}}$
\end{enumerate}
\end{multicols}
\end{majorILnc}
\begin{majorILnc}{\LnpEC{GQL1SubformulaPropertiesExampleB}}
Next consider the formula $\universal{\variable{x}}\parhorseshoe{\Hp{'\variable{x}}}{\Gp{'\variable{x}}}$.
It is not to be confused with the formula from example \ref{GQL1SubformulaPropertiesExampleA}. 
Consider carefully the differences between the two.
The formula in \ref{GQL1SubformulaPropertiesExampleA} has \mention{$\HORSESHOE$} as its main connective, but the main connective of this formula is \mention{$\forall$}.
Its construction tree is:
\begin{center}
	\begin{tikzpicture}[grow=up]
	\tikzset{level distance=50pt}
	\tikzset{sibling distance=40pt}
	\tikzset{every tree node/.style={align=center,anchor=north}}
	\Tree%http://angasm.org/papers/qtree/    http://www.ling.upenn.edu/advice/latex/qtree/qtreenotes.pdf
	[.{$\universal{\variable{x}}\parhorseshoe{\Hp{'\variable{x}}}{\Gp{'\variable{x}}}$} %!{\qsetw{3in}}
	[.{$\horseshoe{\Hp{'\variable{x}}}{\Gp{'\variable{x}}}$}
	  [.{$\Gp{'\variable{x}}$}
	  ] 
	  [.{$\Hp{'\variable{x}}$}
	  ] 
	]
	]%
	%\caption{Example formula tree}
	%\label{fig:ExampleFormulaTree}
	\end{tikzpicture}
\end{center}
The height of the tree is $3$ so the order of the formula is $3$. It has four subformulas:
\begin{multicols}{2}
\begin{cenumerate}
\item $\universal{\variable{x}}\parhorseshoe{\Hp{'\variable{x}}}{\Gp{'\variable{x}}}$
\item $\parhorseshoe{\Hp{'\variable{x}}}{\Gp{'\variable{x}}}$
\item $\Hp{'\variable{x}}$
\item $\Gp{'\variable{x}}$
\end{cenumerate}
\end{multicols}
\end{majorILnc}
\begin{majorILnc}{\LnpEC{GQL1SubformulaPropertiesExampleC}}
Consider the formula $\disjunction{\existential{\variable{x}}\parconjunction{\universal{y}\Epp{'}{\variable{y}}}{\App{'}{\variable{x}}}}{\universal{\variable{z}}\parhorseshoe{\existential{\variable{y}}\Hp{'\constant{a}}}{\Gp{'\variable{x}}}}$.
Its construction tree is:
\begin{center}
	\begin{tikzpicture}[grow=up]
	\tikzset{level distance=50pt}
	\tikzset{level 1/.style={level distance=65pt}}
	\tikzset{sibling distance=40pt}
	\tikzset{every tree node/.style={align=center,anchor=north}}
	\Tree%http://angasm.org/papers/qtree/    http://www.ling.upenn.edu/advice/latex/qtree/qtreenotes.pdf
	[.{$\disjunction{\existential{\variable{x}}\parconjunction{\universal{y}\Epp{'}{\variable{y}}}{\Ap{'\variable{x}}}}{\universal{\variable{z}}\parhorseshoe{\existential{\variable{y}}\Hp{'\constant{a}}}{\Gp{'\variable{x}}}}$}
	[.{$\universal{\variable{z}}\parhorseshoe{\existential{\variable{y}}\Hp{'\constant{a}}}{\Gp{'\variable{x}}}$}
	[.{$\horseshoe{\existential{\variable{y}}\Hp{'\constant{a}}}{\Gp{'\variable{x}}}$}
	[.{$\Gp{'\variable{x}}$}
	]    
	[.{$\existential{\variable{y}}\Hp{'\constant{a}}$}
	  [.{$\Hp{'\constant{a}}$}		
	  ]
	]
	]
	]
	[.{$\existential{\variable{x}}\parconjunction{\universal{y}\Epp{'}{\variable{y}}}{\Ap{'\variable{x}}}$} %!{\qsetw{3in}}
	[.{$\conjunction{\universal{y}\Epp{'}{\variable{y}}}{\Ap{'\variable{x}}}$}
	[.{$\Ap{'\variable{x}}$}
	]    
	[.{$\universal{y}\Epp{'}{\variable{y}}$}
	  [.{$\Epp{'}{\variable{y}}$}		
	  ]
	] 
	]
	]
	]%
	%\caption{Example formula tree}
	%\label{fig:ExampleFormulaTree}
	\end{tikzpicture}
\end{center}
The height of the construction tree is $5$, so that's the order of the formula. It has eleven subformulas:
\begin{enumerate}[label=(\arabic*), leftmargin=1.85\parindent,
labelindent=.35\parindent, labelsep=*, itemsep=0pt]%,start=1
\item $\disjunction{\existential{\variable{x}}\parconjunction{\universal{y}\Epp{'}{\variable{y}}}{\Ap{'\variable{x}}}}{\universal{\variable{z}}\parhorseshoe{\existential{\variable{y}}\Hp{'\constant{a}}}{\Gp{'\variable{x}}}}$
\end{enumerate}
\vspace*{-.5cm}
\begin{multicols}{2}
\begin{enumerate}[label=(\arabic*), leftmargin=1.85\parindent,
labelindent=.35\parindent, labelsep=*, itemsep=0pt, start=2]%,start=1
\item $\existential{\variable{x}}\parconjunction{\universal{y}\Epp{'}{\variable{y}}}{\Ap{'\variable{x}}}$
\item $\conjunction{\universal{y}\Epp{'}{\variable{y}}}{\Ap{'\variable{x}}}$
\item $\universal{y}\Epp{'}{\variable{y}}$
\item $\Ap{'\variable{x}}$
\item $\Epp{'}{\variable{y}}$
\item $\universal{\variable{z}}\parhorseshoe{\existential{\variable{y}}\Hp{'\constant{a}}}{\Gp{'\variable{x}}}$
\item $\horseshoe{\existential{\variable{y}}\Hp{'\constant{a}}}{\Gp{'\variable{x}}}$
\item $\existential{\variable{y}}\Hp{'\constant{a}}$
\item $\Gp{'\variable{x}}$
\item $\Hp{'\constant{a}}$
\end{enumerate}
\end{multicols}
\end{majorILnc}

\subsection{Sentences of \GQL{}1}\label{Sentences of GQL1} 
\GQL{}1 sentences are defined in terms of \GQL{}1 formulas, but we need a few other definitions before we can proceed: \emph{scope}, and \emph{free} vs. \emph{bound} variables.

\begin{majorILnc}{\LnpDC{Scope Definition}}
	In a formula $\existential{\alpha}\CAPPHI$ or $\universal{\alpha}\CAPPHI$, we say that $\CAPPHI$ is the \df{scope} of the quantifier $\existential{\alpha}$ or $\universal{\alpha}$. 
\end{majorILnc}
\noindent{}For example, in the formula $\universal{y}\Epp{'}{\variable{y}}$, $\Epp{'}{\variable{y}}$ is the scope of $\universal{y}$.
Note that the scope of a quantifier may be only a small subformula of a much larger whole.
Consider $\existential{\variable{x}}\parconjunction{\universal{y}\Epp{'}{\variable{y}}}{\Ap{'\variable{x}}}$.
The scope of $\universal{y}$ is just $\Epp{'}{\variable{y}}$, but the scope of $\existential{\variable{x}}$ is $\parconjunction{\universal{y}\Epp{'}{\variable{y}}}{\Ap{'\variable{x}}}$.

\begin{majorILnc}{\LnpDC{Bound Variable}}
	A variable token $\alpha$ in a formula $\CAPPHI$ is \underdf{bound}{variables} \Iff either (i) it's part of a quantifier expression, $\existential{\alpha}$ or $\universal{\alpha}$; or (ii) it occurs within the scope of a quantifier expression with variable $\alpha$. 
\end{majorILnc}

\begin{majorILnc}{\LnpDC{Free Variable}}
	A variable token in a formula $\CAPPHI$ is \underdf{free}{variables} \Iff it is not bound.
\end{majorILnc}

\noindent{}In the formula $\existential{\variable{x}}\parconjunction{\Gp{'\variable{x}}}{\Hp{'\variable{x}}}$, the first token of $\variable{x}$ is bound because it's part of the quantifier expression \mention{$\existential{\variable{x}}$}.  The second and third tokens of $\variable{x}$ are bound because they are within the scope of \mention{$\existential{\variable{x}}$}.\footnote{
	We can also determine whether a variable is bound by thinking of a tree of the formula. A variable token is bound \Iff a quantifier with the same variable appears below or at the same level as (but still on the same branch as) that token. 
	The quantifier that binds a variable is the \emph{first} quantifier that appears below or at the same level as (but still on the same branch as) the variable. 
}

Now we are ready for the definition of a \GQL{}1 sentence:

\begin{majorILnc}{\LnpDC{GQL1 Sentence}}
A string of \GQL{}1 symbols is a \nidf{sentence}\index{sentence!of \GQL{}1|textbf} \Iff it is a formula that contains no free variables.
\end{majorILnc}

\begin{majorILnc}{\LnpDC{Atomic Sentence of GQL}}
An \underdf{atomic}{sentence} \nidf{sentence} of \GQL{}1 is an atomic formula of \GQL{}1 that has no free variable.
\end{majorILnc}
\noindent{}We have unofficial sentences too, just as we have unofficial formulas.
\begin{majorILnc}{\LnpDC{Unofficial Sentence of GQL}}
A string of symbols is an \nidf{unofficial} sentence\index{sentence!unofficial (of \GQL{})|textbf} \Iff it's an unofficial formula that contains no free variables. In other words, we can get an unofficial sentence from an official one by
\begin{cenumerate}
\item deleting outer parentheses,
\item replacing official parentheses ( ) with square brackets [ ] or curly brackets \{ \}, or
\item omitting primes $'$ on the predicate letters.
\end{cenumerate}
\end{majorILnc}
\begin{majorILnc}{\LnpEC{GQLSentenceFreeVariableExampleA}}
Both formulas $\horseshoe{\universal{\variable{x}}\Hp{'\variable{x}}}{\universal{\variable{x}}\Gp{'\variable{x}}}$ and $\universal{\variable{x}}\parhorseshoe{\Hp{'\variable{x}}}{\Gp{'\variable{x}}}$ from examples \ref{GQL1SubformulaPropertiesExampleA} and \ref{GQL1SubformulaPropertiesExampleB} are sentences, because in each all variables are bound.
Below we have arrows pointing from each variable token to the quantifier that binds it.  (There are no arrows for the variables in quantifier expressions since it's obvious which quantifiers they belong to.)
\begin{cenumerate}
\item $\horseshoe{\universal{\variable{x}}\Hp{'\variable{x}}\NextLineRef[black, distance=6, out=-70, in=-50]}{\universal{\variable{x}}\Gp{'\variable{x}}}\NextLineRef[black, distance=6, out=-70, in=-50]$
\item $\universal{\variable{x}}\parhorseshoe{\Hp{'\variable{x}}\NextLineRefJ[black, out=-50, in=-50]}{\Gp{'\variable{x}}\NextLineRefB[black, distance=9, out=205, in=315]}$\footnote{
	You can also use trees to show that these variables are all bound. Look at the construction trees of the formulas from examples \ref{GQL1SubformulaPropertiesExampleA} and \ref{GQL1SubformulaPropertiesExampleB}. You will see that in each case the indicated quantifier is the first that appears below, but still on the same branch as, the token variable.
}
\end{cenumerate}

\end{majorILnc}
\begin{majorILnc}{\LnpEC{GQLSentenceFreeVariableExampleB}}
The formula $\disjunction{\existential{\variable{x}}\parconjunction{\universal{y}\Epp{'}{\variable{x}}}{\Ap{'\variable{x}}}}{\universal{\variable{z}}\parhorseshoe{\existential{\variable{y}}\Hp{'\constant{a}}}{\Gp{'\variable{x}}}}$ is not a sentence. There is a free variable in it. 
The free variable is underlined, while arrows point from each bound variable to the quantifier that binds it (again, ignoring variables in quantifier expressions). 

\smallskip
\begin{cenumerate}
\item $\disjunction{\existential{\variable{x}}\parconjunction{\universal{y}\Epp{'}{\variable{x}\NextLineRefC[black, distance=15, out=135, in=45]}}{\Ap{'\variable{x}}\NextLineRefH[black, distance=13, out=295, in=305]}}{\universal{\variable{z}}\parhorseshoe{\existential{\variable{y}}\Hp{'\constant{a}}}{\Gp{'\underline{\variable{x}}}}}$
\end{cenumerate}

\medskip
\noindent{}The two quantifiers on the \CAPS{rhs} of the disjunction are not binding any token variables, besides the ones that appear in the quantifier expressions themselves.
\end{majorILnc}
\begin{majorILnc}{\LnpEC{GQLSentenceFreeVariableExampleC}}
In each of the following three formulas the free variable tokens are underlined, and arrows go from variable tokens to the quantifiers that bind them.

\smallskip
\begin{enumerate}[label=(\arabic*), leftmargin=1.85\parindent,
labelindent=.35\parindent, labelsep=*, itemsep=8pt]
\item $\universal{\variable{x}}\horseshoe{\parconjunction{\Bl}{\existential{\variable{z}}\Kpp{'}{\variable{x}\NextLineRefF[black, distance=17, out=135, in=145]}}}{\existential{\variable{x}}\Np{\variable{x}}\NextLineRefK[black, out=-50, in=-50]}$
\item $\conjunction{\Dl}{\pardisjunction{\existential{\variable{u}}\universal{\variable{w}}\Ppp{'}{\variable{u}\NextLineRefC[black, distance=12, out=135, in=45]}}{\Cpp{'}{\underline{\variable{y}}}}}$
\item $\existential{\variable{z}}\partriplebar{\parconjunction{\Hpp{'}{\underline{\variable{x}}}}{\Gl}}{\Dp{\variable{z}}\NextLineRefI[black, distance=25, out=205, in=315]}$
\end{enumerate}

\medskip
\noindent{}Formula (1) has no free variables but (2) and (3) do, so (1) is a sentence but (2) and (3) aren't. 
\end{majorILnc}

%%%%%%%%%%%%%%%%%%%%%%%%%%%%%%%%%%%%%%%%%%%%%%%%%%
\section{Models}\label{GQL1 Interpretations}
%%%%%%%%%%%%%%%%%%%%%%%%%%%%%%%%%%%%%%%%%%%%%%%%%%

As with \GSL{}, sentences of \GQL{}1 have no inherent meaning.  
But, also as with \GSL{}, we give \mention{models} for sentences of \GQL{}1.
These models allow us to define and investigate entailment for \GQL{}1.

Our goal is to define \mention{model} such that each \GQL{}1 sentence has a determinate truth value on each model.
Note that only sentences of \GQL{}1 have true values, and not formulas.
A formula that isn't a sentence has some free variable.
Accordingly, it corresponds to a grammatical sentence of English that doesn't have a determinate truth value, because it contains one or more pronouns.
For example, the sentence \mention{He is the author of Waverley,} may be either true or false, depending on who \mention{he} is.
In ordinary conversation we use context to determine the referent of a pronoun.
Someone discussing the English author Sir Walter Scott may read the above sentence and evaluate it as true.
However, in a conversation in which \mention{he} refers to Aristotle, the sentence would be evaluated as false.
The sentence cannot, however, be evaluated without a referent specified.
Analogously, the unbound variables of \GQL{}1 formulas don't have any context-independent \mention{referent} (i.e., assignment).
Therefore non-sentence formulas of \GQL{}1 are not to have determinate truth values on the models of \GQL{}1.  

\subsection{Models in \GQL{}1}\label{Interpretations in GQL1}
An \GSL{} model for $\CAPPHI$ assigns a truth value to each sentence letter in $\CAPPHI$, and that's it. 
\GQL{}1 has predicate letters and constants, which will require different kinds of assignments. 
Furthermore, \GQL{}1 has quantifiers.
The quantifiers roughly correspond to English words such as \emph{all} or \emph{some}, so each model must specify a set of objects that the quantifiers range over.
In other words, a \GQL{}1 model must fix a domain of objects over which to quantify. This is sometimes called a \mention{universe of discourse}.
\begin{majorILnc}{\LnpDC{GQL1 Interpretation}} 
A \df{model} for $\CAPPHI$, $\IntA$, consists of:
\begin{cenumerate}
\item an assignment of a truth value $\TrueB$ or $\FalseB$ to each sentence letter in $\CAPPHI$; 
\item a single, non-empty set $\integer{U}$, called the \df{universe} or \df{domain};
\item an assignment of a subset of $\integer{U}$ to each 1-place predicate in $\CAPPHI$;
\item an assignment of an element from $\integer{U}$ to each individual constant in $\CAPPHI$.
\end{cenumerate}
\end{majorILnc}
\noindent{}We use the following notational conventions: 

\begin{cenumerate}
	\item Given some sentence letter, like $\PP$, $\IntA(\PP)$ is the truth value $\IntA$ assigns to $\PP$.
	\item $\IntA(\integer{U})$ is the set $\m$ assigns to $\integer{U}$.
	\item Given a 1-place predicate, like $\Gp{'}$, $\IntA(\Gp{'})$ is the subset of $\integer{U}$ assigned to $\Gp{'}$ by $\IntA$.
	\item Given an individual constant, like $\constant{a}$, $\IntA(\constant{a})$ is the element from $\integer{U}$ assigned by $\IntA$ to $\constant{a}$.\footnote{%
		We pause here to make two points for those keeping careful score:
		\begin{enumerate*}[label=(\arabic*)]
		\item We can think of models as functions from the set of basic symbols of \GQL{}1 (less the logical operators, variables, and parentheses) to the kinds of objects mentioned in definition \mvref{GQL1 Interpretations} (objects or subsets of $\integer{U}$). 
		\item\label{pointtwo} Those trying to keep careful track of the \distinction{use}{mention} distinction\index{\distinction{use}{mention} distinction}\index{single quotes} should note that here we've been especially loose. 
		We justify our laxity on the grounds that strict adherence to the distinction would clutter up our notation with confusing layers of quotes. 
		\end{enumerate*}
		\label{Int Footnote}
	} 
\end{cenumerate}

\noindent{}Earlier we said that the individual constants are roughly similar to proper names in English.
One difference is that, in \GQL{}1, each individual constant in $\CAPPHI$ corresponds to exactly one object in the domain.
In English, on the other hand, some proper names---e.g., \mention{John Smith}---correspond to more than one person, and some---e.g., \mention{Mordecai Alonzo Frazzle III}---do not correspond to any person.
While each constant in $\CAPPHI$ is assigned an object from the domain, it is not required that different constants be assigned different objects. 

We distinguish different models by affixing integers as subscripts to the symbol \mention{$\As{}{}$}.  So, for example, $\As{}{1}$, $\As{}{2}$, $\As{}{3}$, \ldots, $\As{}{316}$, etc., are each different models.

As with \GSL{}, we have \GQL{}1 models for sets of sentences:

\begin{majorILnc}{\LnpDC{Definition of Model for QL1 Set}}
	$\IntA$ is a \df{model for a set of \GQL{}1 sentences $\Delta$} \Iff $\IntA$ is a model for each sentence in $\Delta$.
\end{majorILnc}

There are also models that make assignments to all the sentence letters, constants, and 1-place predicates of \GQL{}1.
Any such model is a model for every \GQL{}1 sentence.
Let's call these \emph{models for \GQL{}1}:

\begin{majorILnc}{\LnpDC{Definition of Model for QL1}}
	$\IntA$ is a \df{model for \GQL{}1} \Iff $\IntA$ is a model for every sentence of \GQL{}1.
\end{majorILnc}

We define truth in \GQL{}1 so that every model for $\CAPPHI$ fixes a unique truth value for $\CAPPHI$.
But there is a price we must pay for \GQL{}1's superior models.
\GQL{}1 is more complicated than \GSL{}, so its definition of truth requires some additional metalinguistic tools.
We must first define \mention{terms} and \mention{model variants}.

\begin{majorILnc}{\LnpDC{Terms}}
The \idf{individual terms} of \GQL{}1 are the constants and variables of \GQL{}1.
\end{majorILnc}
\noindent{}We typically use \mention{$\variable{q}$},\mention{$\variable{r}$}, \mention{$\variable{s}$}, and \mention{$\variable{t}$}  (along with subscripts) as MathEnglish metavariables for terms. This means that italic Roman \mention{$\variable{q}$}, \mention{$\variable{r}$}, \mention{$\variable{s}$}, \mention{$\variable{t}$}, and the Greek \mention{$\ALPHA$}, \mention{$\BETA$}, etc., can all be MathEnglish variables for \GQL{} variables.
However, while \mention{$\ALPHA$}, \mention{$\BETA$}, etc. stand \emph{only} for variables, the italic Roman letters can also range over \GQL{}1 constants.
The use of metavariables for individual terms in the object language simplifies our notation considerably.

At this point we could define truth for sentences without quantifiers, though for now we offer only a short sketch.
The sentence $\Gp{\constant{a}}$ is true on $\IntA$ \Iff the element $\IntA$ assigns to \mention{$\constant{a}$} is in the set $\IntA$ assigns to \mention{$\GG$}; i.e., \Iff $\IntA(\constant{a})\in\IntA(\GG)$.
If the element $\IntA$ assigns to \mention{$\constant{a}$} isn't in the set assigned to \mention{$\GG$}, $\Gp{\constant{a}}$ is false on $\IntA$.

Quantifiers require more complexity.
A sentence like $\universal{\variable{x}}\Ep{\variable{x}}$ is true \Iff every element in the domain, $\integer{U}$, is an element of $\IntA(\EE)$.
So if $\IntA$ assigns $\integer{U}$ the set of even integers and $\IntA(\EE)=\integer{U}$, $\universal{\variable{x}}\Ep{\variable{x}}$ is true.
But if instead $\IntA(\EE)=\{2, 4, 6\}$, then there are objects in the domain (e.g. $8$) not in $\IntA(\EE)$, and so $\universal{\variable{x}}\Ep{\variable{x}}$ is false.
For simple quantified sentences this quick definition is good enough; but it won't work for more complex sentences, such as $\universal{\variable{x}}\existential{\variable{y}}\parhorseshoe{\Hp{\variable{y}}}{\Gp{\variable{x}}}$.

To define truth for quantified sentences more precisely, we need a convenient notation for two models that make identical assignments everywhere except at one constant.

\begin{majorILnc}{\LnpDC{Variant}}
	Let $\m$ and $\m^{\variable{t}}$ be \GQL{}1 models and $\variable{t}$ be a term that is given some assignment by $\m^{\variable{t}}$. Then $\m^{\variable{t}}$ is a $\variable{t}$-\nidf{variant}\index{model!$\variable{t}$-variant} of $\m$ \Iff for every term $\variable{r}$ such that $\variable{r}\not=\variable{t}$, $\m(r)=\m^{\variable{t}}(r)$.
\end{majorILnc}
\noindent{}Put another way, a $\variable{t}$-variant of $\As{}{}$ is a model that makes all the same assignments as $\As{}{}$ except possibly at constant $\variable{t}$.
One result of this definition is that any model $\As{}{}$ that assigns something to $\variable{t}$ is a $\variable{t}$-variant of itself.
But what if $\As{}{}$ doesn't assign anything to $\variable{t}$? In that case, the $\variable{t}$-variants are all the models that are identical with $\As{}{}$ except with an additional assignment to $\variable{t}$.

We generally denote $\variable{t}$-variants of a model $\As{}{}$ by affixing $\variable{t}$ as a superscript to \mention{$\As{}{}$}. For example, $\As{\constant{c}}{}$ is a $\constant{c}$-variant of $\As{}{}$.
Then $\As{\constant{c}}{}$ and $\As{}{}$ make all the same assignments, except possibly to the constant $\constant{c}$.
We extend this notation when the symbol denoting the original model is itself complex.
So, given an $\constant{c}$-variant of model $\As{}{}$, i.e., $\As{\constant{c}}{}$, $\As{\constant{cd}}{}$ is a $\constant{d}$-variant of $\As{\constant{c}}{}$.  For another example, if $\As{\constant{e}}{4}$ is an $\constant{e}$-variant of $\As{}{4}$, then $\As{\constant{e}}{4}$ and $\As{}{4}$ make identical assignments to everything except maybe $\constant{e}$. 

We need one more piece of notation before we get to the definition of truth.
\begin{majorILnc}{\LnpDC{MathEnglishVariableSub1}}
	If $\CAPPHI$ is a \GQL{}1 formula and $\variable{t}$ and $\variable{s}$ are terms, then $\CAPPHI\variable{s}/\variable{t}$ is the formula you get by replacing each unbound token of $\variable{t}$ in $\CAPPHI$ with a token of $\variable{s}$.
\end{majorILnc}

\begin{majorILnc}{\LnpEC{MathEnglishVariableSubEx1}}
	\begin{cenumerate}
		\item If $\CAPPHI$ is $\Al$, then $\CAPPHI\variable{y}/\variable{x}$ is $\Al$.
		\item If $\CAPPHI$ is $\Bp{\variable{x}}$, then $\CAPPHI\variable{y}/\variable{x}$ is $\Bp{\variable{y}}$.
		\item If $\CAPPHI$ is $\Bp{\variable{y}}$, then $\CAPPHI\variable{y}/\variable{x}$ is $\Bp{\variable{y}}$.
		\item If $\CAPPHI$ is $\Bp{\variable{x}}$, then $\CAPPHI\variable{y}/\variable{w}$ is $\Bp{\variable{x}}$.
		\item If $\CAPPHI$ is $\universal{x}\Bp{\variable{x}}$, then $\CAPPHI\variable{y}/\variable{x}$ is $\universal{x}\Bp{\variable{x}}$.
		\item If $\CAPPHI$ is $\conjunction{\Cp{\variable{x}}}{\universal{x}\Bp{\variable{x}}}$, then $\CAPPHI\variable{y}/\variable{x}$ is $\conjunction{\Cp{\variable{y}}}{\universal{x}\Bp{\variable{x}}}$.
		\item If $\CAPPHI$ is $\existential{\variable{y}}\parconjunction{\Cp{\variable{x}}}{\universal{x}\Bp{\variable{x}}}$, then $\CAPPHI\variable{y}/\variable{x}$ is $\existential{\variable{y}}\parconjunction{\Cp{\variable{y}}}{\universal{x}\Bp{\variable{x}}}$.
		\item If $\CAPPHI$ is $\existential{\variable{y}}\parconjunction{\Cp{\variable{x}}}{\universal{x}\Bp{\variable{x}}}$, then $\CAPPHI\constant{a}/\variable{x}$ is $\existential{\variable{y}}\parconjunction{\Cp{\constant{a}}}{\universal{x}\Bp{\variable{x}}}$.
		\item If $\CAPPHI$ is $\existential{\variable{y}}\parconjunction{\Cp{\variable{x}}}{\Bp{\variable{x}}}$, then $\CAPPHI\constant{a}/\variable{x}$ is $\existential{\variable{y}}\parconjunction{\Cp{\constant{a}}}{\Bp{\constant{a}}}$.
	\end{cenumerate}
\end{majorILnc}

%\subsection{Truth in a Model: Preliminary Ideas}\label{GQL Truth in an Interpretation Prelims}

\subsection{Truth in a Model}\label{GQL1 Truth in an Interpretation}
We now define truth in a \GQL{}1 model.

\begin{majorILnc}{\LnpDC{Truth for GQL1 Sentence}}
The following clauses fix when a \GQL{}1 sentence $\CAPTHETA$ is \nidf{$\True$} (or \nidf{$\False$}) on a model for $\CAPTHETA$, $\IntA$:
\begin{cenumerate}
\item A sentence letter $\CAPPHI$ is $\True$ on $\IntA$ \Iff $\As{}{}(\CAPPHI)=\TrueB$.
\item An atomic sentence $\Pp{\variable{t}}$ with a 1-place predicate $\PP$ and an individual term $\variable{t}$ is $\True$ on $\IntA$ \Iff $\IntA(\variable{t})\in\IntA(\PP)$.
\item A negation $\negation{\CAPPHI}$ is $\True$ on $\IntA$ \Iff $\CAPPHI$ is $\False$ on $\IntA$.
\item A conjunction $\parconjunction{\CAPPHI_1}{\conjunction{\ldots}{\CAPPHI_{\integer{n}}}}$ is $\True$ on $\IntA$ \Iff all of $\CAPPHI_1,\ldots,\CAPPHI_{\integer{n}}$ are $\True$ on $\IntA$.
\item A disjunction $\pardisjunction{\CAPPHI_1}{\disjunction{\ldots}{\CAPPHI_{\integer{n}}}}$ is $\True$ on $\IntA$ \Iff at least one of $\CAPPHI_1,\ldots,\CAPPHI_{\integer{n}}$ is $\True$ on $\IntA$.
\item A conditional $\parhorseshoe{\CAPPSI}{\CAPPHI}$ is $\True$ on $\IntA$ \Iff the \CAPS{lhs} $\CAPPSI$ is $\False$ or the \CAPS{rhs} $\CAPPHI$ is $\True$ on $\IntA$, or both.
\item A biconditional $\partriplebar{\CAPPSI}{\CAPPHI}$ is $\True$ on $\IntA$ \Iff $\CAPPSI$ and $\CAPPHI$ have the same truth value on $\IntA$.
\item\label{GQL1TruthUnvQuant} A universal quantification $\universal{\ALPHA}\CAPPHI$ is $\True$ on $\IntA$ \Iff $\CAPPHI\variable{t}/\ALPHA$ is $\True$ on \emph{all} $\variable{t}$-variants of $\IntA$, where $\variable{t}$ is the first constant not in $\CAPPHI$.
\item An existential quantification $\existential{\ALPHA}\CAPPHI$ is $\True$ on $\IntA$ \Iff $\CAPPHI\variable{t}/\ALPHA$ is $\True$ on \emph{some} $\variable{t}$-variant of $\IntA$, where $\variable{t}$ is the first constant not in $\CAPPHI$.
\item A sentence $\CAPPHI$ is $\False$ on $\IntA$ \Iff $\CAPPHI$ is not $\True$ on $\IntA$.
\end{cenumerate}
\end{majorILnc}

\noindent{}For the following examples consult the models \emph{Pos Int} and \emph{States}, given in figure \mvref{table:Partial Models}.

\begin{figure}
\begin{longtable}[c]{ l l l l } %p{2.2in} p{2in}
	\toprule
	&\textbf{Symbol} & \multicolumn{2}{c}{\textbf{Model}} \\ \cmidrule(l){3-4}
	& & \textbf{Pos Int} & \textbf{States} \\
	\midrule 
	\endfirsthead
	\multicolumn{4}{c}{\emph{Continued from Previous Page}}\\
	\toprule
	&\textbf{Symbol} & \multicolumn{2}{c}{\textbf{Model}} \\ \cmidrule(l){3-4}
	& & \textbf{Pos Int} & \textbf{States} \\
	\midrule 
	\endhead
	\bottomrule
	\caption{Example Models}\\[-.15in]
	\multicolumn{4}{c}{\emph{Continued next Page}}\\
	\endfoot
	\bottomrule
	\caption{Example Models}\\%
	\endlastfoot%
	\label{table:Partial Models}%
	%\begin{tabular}{ l l l l } %p{2in} p{2in} %\begin{tabular}{ p{1in} l l } %p{2.2in} p{2in}
	%\toprule
	%&\textbf{Symbol} & \multicolumn{2}{c}{\textbf{Interpretation}} \\ \cmidrule(l){3-4}
	%& & \textbf{Pos Int} & \textbf{States} \\
	%\midrule 
	{Universe:} & & The set of positive integers & The set of US states (2024) \\ \addlinespace[.25cm]
	{Sent. Let.:}& A&$\True$&$\False$\\
	& B&$\True$&$\False$\\
	& C&$\False$&$\True$\\
	& D&$\True$&$\False$\\
	& E&$\True$&$\False$\\
	& G&$\False$&$\True$\\ \addlinespace[.25cm]
	{Constants:}&$\constant{a}$&1&Louisiana\\
	&$\constant{b}$&9&Maine\\
	&$\constant{c}$&72&Georgia\\
	&$\constant{d}$&3&Nebraska\\
	&$\constant{e}$&1&New Mexico\\
	&$\constant{f}$&2&Texas\\ \addlinespace[.25cm]
	{1-place:}&$\Ap{'}$&all pos int&Midwestern\\
	&$\Bp{'}$&empty set&name with $>5$ letters\\
	&$\Cp{'}$&even&Coastal\\
	&$\Dp{'}$&odd&on the Pacific coast\\
	&$\Ep{'}$&prime&\{Ohio\}\\
	&$\Gp{'}$&multiple of 7&\{Ohio,Alabama\}\\ \addlinespace[.25cm]
	%\bottomrule
\end{longtable}
\caption{Two \GQL{}1 models}
\end{figure}

\begin{majorILnc}{\LnpEC{GQL1TruthExamplePA}}
	The sentence $\negation{\Gp{\constant{b}}}$ is true on the model \emph{Pos Int}. 
\end{majorILnc}
\begin{PROOF}
	The model \emph{Pos Int} assigns $9$ to $\constant{b}$, and the set of multiples of $7$ to $\GG$.
	The number $9$ is not a multiple of $7$, so $\emph{Pos Int}(\constant{b})\notin\emph{Pos Int}(\GG)$.
	So, by the definition of truth, $\Gp{\constant{b}}$ is false on \emph{Pos Int}.
	Therefore, by the definition of truth for $\NEGATION$, \emph{Pos Int} makes $\negation{\Gp{\constant{b}}}$ true.

\end{PROOF}
\begin{commentary}
	$\emph{Pos Int}(\constant{b})\notin\emph{Pos Int}(\GG)$ asserts that what \emph{Pos Int} assigns to $\constant{b}$, $9$, is not an element of the set that \emph{Pos Int} assigns to $\GG$.
	To check this, we need only to look at the set assigned to $\GG$ and see whether $7$ is a member; it's not, since $9$ is not a multiple of $7$.
\end{commentary}

\begin{majorILnc}{\LnpEC{GQL1TruthExampleA}}
The sentence ${\parhorseshoe{\Gp{\constant{d}}}{\Dp{\constant{e}}}}$ is true on the model \emph{Pos Int} (\ref{table:Partial Models}). 
\end{majorILnc}
\begin{PROOF}
$\parhorseshoe{\Gp{\constant{d}}}{\Dp{\constant{e}}}$ is true on \emph{Pos Int} \Iff the \CAPS{LHS} is false or the \CAPS{RHS} is true.
\emph{Pos Int} assigns the number $3$ to $\constant{d}$ and the set of multiples of $7$ to $\GG$.
But $3$ isn't a multiple of $7$, so $\emph{Pos Int}(\constant{d})\notin\emph{Pos Int}(\GG)$.
It follows that $\Gp{\constant{d}}$ is false on \emph{Pos Int}.
Therefore $\parhorseshoe{\Gp{\constant{d}}}{\Dp{\constant{e}}}$ is true on \emph{Pos Int}.
\end{PROOF}

\begin{majorILnc}{\LnpEC{GQL1TruthExampleB}}
The sentence $\disjunction{\Ap{\constant{a}}}{\Gp{\constant{c}}}$ is false on the model \emph{States} (\ref{table:Partial Models}).
\end{majorILnc}
\begin{PROOF}
The model \emph{States} assigns Louisiana to $\constant{a}$, Georgia to $\constant{c}$, the set of Midwestern states to $\AA$, and the set $\{$Ohio, Alabama$\}$ to $\GG$.

$\Ap{\constant{a}}$ is true on \emph{States} \Iff $\emph{States}(\constant{a})\in\emph{States}(\AA)$.
But Louisiana isn't a Midwestern state.
So $\Ap{\constant{a}}$ is false on \emph{States}.
$\Gp{\constant{c}}$ is true on \emph{States} \Iff $\emph{States}(\constant{c})\in\emph{States}(\GG)$.
But \emph{States} assigns $\{$Ohio, Alabama$\}$ to $\GG$.
Georgia isn't in that set, so \emph{States} makes $\Gp{\constant{c}}$ false.

$\Ap{\constant{a}}$ and $\Gp{\constant{c}}$ are false on \emph{States}, so \emph{States} makes $\disjunction{\Ap{\constant{a}}}{\Gp{\constant{c}}}$ false.
\end{PROOF}
\noindent{}Pay attention to the full structure of a sentence when assessing its truth value.
The sentences $\horseshoe{\universal{\variable{x}}\Dp{\variable{x}}}{\universal{\variable{x}}\Gp{\variable{x}}}$ and $\universal{\variable{x}}\parhorseshoe{\Dp{\variable{x}}}{\Gp{\variable{x}}}$ may look similar, but they mean very different things.
Note that the first has \mention{$\HORSESHOE$} as its main connective, whereas the main connective of the second is \mention{$\forall$}.
We show that their truth values come apart on \emph{Pos Int} in the next two examples.

\begin{majorILnc}{\LnpEC{GQL1TruthExampleC}}
$\horseshoe{\universal{\variable{x}}\Dp{\variable{x}}}{\universal{\variable{x}}\Gp{\variable{x}}}$ is true on the model \emph{Pos Int} (\ref{table:Partial Models}).
\end{majorILnc}
\begin{PROOF}
	$\horseshoe{\universal{\variable{x}}\Dp{\variable{x}}}{\universal{\variable{x}}\Gp{\variable{x}}}$ is true on \emph{Pos Int} \Iff \emph{Pos Int} either makes the \CAPS{lhs} is false or the \CAPS{RHS} true.
	By the definition of truth for $\forall$, $\universal{\variable{x}}\Dp{\variable{x}}$ is true on \emph{Pos Int} \Iff $\Dp{\constant{a}}$ is true on all $\constant{a}$-variants of \emph{Pos Int}.
	\begin{commentary}
		The definition of truth for $\forall$ has us substitute the first constant not in $\Dp{\variable{x}}$.
		Since it has no constant we use the first one: $\constant{a}$.
		The result of the substitution is $\Dp{\constant{a}}$.
	\end{commentary}
	\noindent{}\emph{Pos Int} assigns the set of odd numbers to $\DD$.
	Consider the $\constant{a}$-variant of \emph{Pos Int}, \emph{Pos Int}\textsuperscript{a}, such that \emph{Pos Int}\textsuperscript{a}($\constant{a}$) is $2$.
	Clearly $2$ is not an odd number.
	It follows that $\Dp{\constant{a}}$ is false on \emph{Pos Int}\textsuperscript{a}, and so $\universal{\variable{x}}\Dp{\variable{x}}$ is false on \emph{Pos Int}.
	Therefore $\horseshoe{\universal{\variable{x}}\Dp{\variable{x}}}{\universal{\variable{x}}\Gp{\variable{x}}}$ is true on \emph{Pos Int}.
\end{PROOF}

\begin{majorILnc}{\LnpEC{GQL1TruthExampleC2}}
	$\universal{\variable{x}}\parhorseshoe{\Dp{\variable{x}}}{\Gp{\variable{x}}}$ is false on \emph{Pos Int} (\ref{table:Partial Models}).
\end{majorILnc}
\begin{PROOF}
	By the definition of truth for $\forall$, $\universal{\variable{x}}\parhorseshoe{\Dp{\variable{x}}}{\Gp{\variable{x}}}$ is true on \emph{Pos Int} \Iff every $\constant{a}$-variant of \emph{Pos Int} makes $\horseshoe{\Dp{\constant{a}}}{\Gp{\constant{a}}}$ true.
	\emph{Pos Int} assigns the set of odd numbers to $\DD$ and the set of multiples of $7$ to $\GG$.
	Let there be an $\constant{a}$-variant of \emph{Pos Int}, $\emph{Pos Int}^{\constant{a}}$, such that $\emph{Pos Int}^{\constant{a}}(\constant{a})=3$.
	The number $3$ is odd and not a multiple of $7$.
	So $\emph{Pos Int}^{\constant{a}}(\constant{a})\in\emph{Pos Int}^{\constant{a}}(\DD)$ and $\emph{Pos Int}^{\constant{a}}(\constant{a})\notin\emph{Pos Int}^{\constant{a}}(\GG)$.
	Thus, $\emph{Pos Int}^{\constant{a}}$ makes $\Dp{\constant{a}}$ true and $\Gp{\constant{a}}$ false, which in turn makes $\horseshoe{\Dp{\constant{a}}}{\Gp{\constant{a}}}$ false.
	Therefore $\universal{\variable{x}}\parhorseshoe{\Dp{\variable{x}}}{\Gp{\variable{x}}}$ is false on \emph{Pos Int}.
\end{PROOF}

\noindent{}Let's compare another pair of superficially similar sentences, this time with existential quantifiers: $\existential{\variable{x}}\parconjunction{\Cp{\variable{x}}}{\Dp{\variable{x}}}$ and $\parconjunction{\existential{\variable{x}}\Cp{\variable{x}}}{\existential{\variable{x}}\Dp{\variable{x}}}$.

\begin{majorILnc}{\LnpEC{GQLTruthExampleD}}
	$\existential{\variable{x}}\parconjunction{\Cp{\variable{x}}}{\Dp{\variable{x}}}$ is false on \emph{Pos Int} (\ref{table:Partial Models}).
\end{majorILnc}
\begin{PROOF}
By the definition of truth for $\exists$, $\existential{\variable{x}}\parconjunction{\Cp{\variable{x}}}{\Dp{\variable{x}}}$ is true on \emph{Pos Int} \Iff there is an $\constant{a}$-variant of \emph{Pos Int} that makes $\conjunction{\Cp{\constant{a}}}{\Dp{\constant{a}}}$ true.  
\emph{Pos Int} assigns the even numbers to $\CC$ and the odd numbers to $\DD$.
There is no element in the domain of \emph{Pos Int} that is both odd and even.
So there is no $\constant{a}$-variant of \emph{Pos Int} that makes both $\Cp{\constant{a}}$ and $\Dp{\constant{a}}$ true.
Hence $\conjunction{\Cp{\constant{a}}}{\Dp{\constant{a}}}$ is false on all $\constant{a}$-variants.
Therefore $\existential{\variable{x}}\parconjunction{\Cp{\variable{x}}}{\Dp{\variable{x}}}$ is false on \emph{Pos Int}.
\end{PROOF}	

\begin{majorILnc}{\LnpEC{GQLTruthExampleD2}}
	$\parconjunction{\existential{\variable{x}}\Cp{\variable{x}}}{\existential{\variable{x}}\Dp{\variable{x}}}$ is true on \emph{Pos Int} (\ref{table:Partial Models}).
\end{majorILnc}
\begin{PROOF}
$\parconjunction{\existential{\variable{x}}\Cp{\variable{x}}}{\existential{\variable{x}}\Dp{\variable{x}}}$ is true on \emph{Pos Int} \Iff $\existential{\variable{x}}\Cp{\variable{x}}$ and $\existential{\variable{x}}\Dp{\variable{x}}$ are true on \emph{Pos Int}.

By the definition of truth for $\exists$, $\existential{\variable{x}}\Cp{\variable{x}}$ is true on \emph{Pos Int} \Iff there is an $\constant{a}$-variant of \emph{Pos Int} that makes $\Cp{\constant{a}}$ true.
$\emph{Pos Int}$ assigns the even numbers to $\CC$.
Let $\emph{Pos Int}^{\constant{a}}$ be an $\constant{a}$-variant that assigns $4$ to $\constant{a}$.
The number $4$ is even, so $\emph{Pos Int}^{\constant{a}}(\constant{a})\in\emph{Pos Int}^{\constant{a}}(\CC)$.
Thus, $\emph{Pos Int}^{\constant{a}}$ makes $\Cp{\constant{a}}$ true, and so \emph{Pos Int} makes $\existential{\variable{x}}\Cp{\variable{x}}$ true.

By the definition of truth for $\exists$, $\existential{\variable{x}}\Dp{\variable{x}}$ is true on \emph{Pos Int} \Iff there is an $\constant{a}$-variant of \emph{Pos Int} that makes $\Dp{\constant{a}}$ true.
$\emph{Pos Int}$ assigns the odd numbers to $\DD$.
Let $\emph{Pos Int}^{\constant{a}}$ be an $\constant{a}$-variant that assigns $3$ to $\constant{a}$.
The number $3$ is odd, so $\emph{Pos Int}^{\constant{a}}(\constant{a})\in\emph{Pos Int}^{\constant{a}}(\DD)$.
Thus, $\emph{Pos Int}^{\constant{a}}$ makes $\Dp{\constant{a}}$ true.
It follows that \emph{Pos Int} makes $\existential{\variable{x}}\Dp{\variable{x}}$ true.
	
Therefore \emph{Pos Int} makes $\parconjunction{\existential{\variable{x}}\Cp{\variable{x}}}{\existential{\variable{x}}\Dp{\variable{x}}}$ true.
\end{PROOF}

\noindent{}On model \emph{Pos Int} we can interpret $\existential{\variable{x}}\parconjunction{\Cp{\variable{x}}}{\Dp{\variable{x}}}$ as saying, roughly that there is some positive integer that is both even and odd.
That's clearly false.
By contrast, $\parconjunction{\existential{\variable{x}}\Cp{\variable{x}}}{\existential{\variable{x}}\Dp{\variable{x}}}$ means, roughly, that there is some even positive integer and there is some odd positive integer, which is true.

We can calculate the truth value of a \GQL{}1 sentence $\CAPPHI$ on some $\IntA$ \Iff $\IntA$ is a model for $\CAPPHI$.  If $\IntA$ \emph{isn't} a model for $\CAPPHI$, then $\CAPPHI$ has no truth value on it.  Be careful: if $\As{}{1}$ is a model for $\CAPPHI$ and $\As{}{2}$ is a model for $\CAPPSI$, it doesn't follow that either $\As{}{1}$ or $\As{}{2}$ is a model for, say, $\horseshoe{\CAPPHI}{\CAPPSI}$.

\subsection{Minimal Models in \GQL{}1}\label{Minimal Models in GQL1}

For the examples above we used models with assignments that are not referenced in the sentences we evaluated.
The Minimal Model theorem below proves that assignments to irrelevant predicates, constants, etc. don't matter for the truth value of a sentence.

\begin{majorILnc}{\LnpDC{Definition of Minimal QL1 Model}}
	Model $\IntA$ is a \df{minimal model for $\CAPPHI$} \Iff $\IntA$ makes the minimum assignments necessary for $\IntA$ to be a model for $\CAPPHI$.
\end{majorILnc}

\noindent{}That is, a minimal model makes assignments to the universe $\integer{U}$, to each sentence letter, constant, and 1-place predicate in $\CAPPHI$, but to nothing else.
Consider the sentence $\parconjunction{\existential{\variable{x}}\Cp{\variable{x}}}{\existential{\variable{x}}\Dp{\variable{x}}}$.
This sentence has two 1-place predicates, $\CC$ and $\DD$, and no sentence letters or constants.
A minimal model for this sentence makes no assignment to any sentence letter or constant.
It assigns subsets of $\integer{U}$ to $\CC$ and $\DD$, but makes no assignment to any other 1-place predicate.
Most logic texts do not define minimal models but implicitly use them.
We prefer to define them explicitly.

When calculating the value of a sentence $\CAPPHI$ on a model, we only need to worry about the assignments to the symbols in $\CAPPHI$ and the universe.
We can ignore the other assignments.
The following theorem demonstrates this.
This is the \GQL{}1 version of theorem \ref{thm:localityoftruth} in chapter \ref{sententiallogic}.

\begin{THEOREM}{\LnpTC{Two Models}} 
	Let $\CAPPHI$ be any \GQL{}1 sentence.  If there are two models for $\CAPPHI$, $\As{}{1}$ and $\As{}{2}$, that have the same domain, $\integer{U}$, and make the same assignments for all the sentence letters, individual constants, and 1-place predicates contained in $\CAPPHI$, then $\CAPPHI$ is true on $\As{}{1}$ \Iff $\CAPPHI$ is true on $\As{}{2}$.
\end{THEOREM}	
\begin{PROOF}
	\begin{description}
		\item[Base Step:]  Let $\CAPTHETA$ be a sentence of order 1.  $\CAPTHETA$ is either (i) a sentence letter, or (ii), a 1-place predicate followed by a constant.
		
		(i) $\CAPTHETA$ is a sentence letter.
		We assume that $\As{}{1}$ and $\As{}{2}$ make the same assignments to all the sentence letters.
		Then $\CAPTHETA$ is true on $\As{}{1}$ \Iff $\CAPTHETA$ is true on $\As{}{2}$.
		
		(ii) $\CAPTHETA=\PP\variable{t}$, where $\PP$ is a 1-place predicate and $\variable{t}$ is a constant.
		We assume that $\As{}{1}$ and $\As{}{2}$ make the same assignments to all the constants and 1-place predicates.
		Then $\As{}{1}(\PP)=\As{}{2}(\PP)$ and $\As{}{1}(\variable{t})=\As{}{2}(\variable{t})$.
		It follows that $\As{}{1}(\variable{t})\in\As{}{1}(\PP)$ \Iff $\As{}{2}(\variable{t})\in\As{}{2}(\PP)$.
		Thus, $\CAPTHETA$ is true on $\As{}{1}$ \Iff $\CAPTHETA$ is true on $\As{}{2}$.
		
		\item[Inheritance Step:]  \hfill{}
		\begin{description}
			\item[Recursive Assumption:] Assume that $\CAPTHETA$ is a \GQL{}1 sentence of order $n$.
			Assume also that $\As{}{1}$ and $\As{}{2}$ are any two models for $\CAPTHETA$ that make the same assignments to all the sentence letters, individual constants, and 1-place predicates contained in $\CAPTHETA$.
			Then let $\CAPTHETA$ be true on $\As{}{1}$ \Iff $\CAPTHETA$ is true on $\As{}{2}$.
			
			\item[Negation, Conditional, Biconditional, Disjunction, Conjunction:]  The reasoning for these is the same as in the corresponding clauses of theorem \ref{thm:localityoftruth} in chapter \ref{sententiallogic}.
			
			\item[Quantifier Preface:] Let $\universal{\ALPHA}\CAPTHETA$ and $\existential{\ALPHA}\CAPTHETA$ be sentences of order $n+1$.
			By RA, $\model{}{1}$ and $\model{}{2}$ have the same universe, $\integer{U}$.
			It follows that for each element $u\in\integer{U}$, there is a $\variable{t}$-variant of $\model{}{1}$, $\model{\variable{t}}{1}$, and a $\variable{t}$-variant of $\model{}{2}$, $\model{\variable{t}}{2}$, such that $\model{\variable{t}}{1}(t)=\model{\variable{t}}{2}(t)=u$.
			Since $\As{}{1}$ is a model for $\universal{\ALPHA}\CAPTHETA$ and $\existential{\ALPHA}\CAPTHETA$, each $\model{\variable{t}}{1}$ is a model for $\CAPTHETA\variable{t}/\ALPHA$.
			By similar reasoning each $\model{\variable{t}}{2}$ is a model for $\CAPTHETA\variable{t}/\ALPHA$.
			$\CAPTHETA\variable{t}/\ALPHA$ is of order $n$.
			So by RA, for each pair of $\variable{t}$-variants such that $\model{\variable{t}}{1}(t)=\model{\variable{t}}{2}(t)$, $\CAPTHETA\variable{t}/\ALPHA$ is true on $\model{\variable{t}}{1}$ \Iff it's true on $\model{\variable{t}}{2}$.

			\begin{description}

				\item[Universal Quantification:] ($\Rightarrow$) Let $\universal{\ALPHA}\CAPTHETA$ be true on $\As{}{1}$.
				Then, by the definition of truth of $\forall$, all $\variable{t}$-variants of $\model{}{1}$ make $\CAPTHETA\variable{t}/\ALPHA$ true.
				Since by the quantifier preface there is a corresponding $\variable{t}$-variant of $\model{}{2}$ for each $\variable{t}$-variant of $\model{}{1}$, it follows that all $\variable{t}$-variants of $\model{}{2}$ make $\CAPTHETA\variable{t}/\ALPHA$ true.
				So, by the definition of truth for $\forall$, $\universal{\ALPHA}\CAPTHETA$ is true on $\As{}{2}$.

				($\Leftarrow$) Let $\universal{\ALPHA}\CAPTHETA$ be true on $\As{}{2}$.
				Then, by the definition of truth of $\forall$, all $\variable{t}$-variants of $\model{}{2}$ make $\CAPTHETA\variable{t}/\ALPHA$ true.
				Since by the quantifier preface there is a corresponding $\variable{t}$-variant of $\model{}{1}$ for each $\variable{t}$-variant of $\model{}{2}$, it follows that all $\variable{t}$-variants of $\model{}{1}$ make $\CAPTHETA\variable{t}/\ALPHA$ true.
				So, by the definition of truth for $\forall$, $\universal{\ALPHA}\CAPTHETA$ is true on $\As{}{1}$.

				Thus $\universal{\ALPHA}\CAPTHETA$ is true on $\As{}{1}$ \Iff $\universal{\ALPHA}\CAPTHETA$ is true on $\As{}{2}$.

				\item[Existential Quantification:] ($\Rightarrow$) Let $\existential{\ALPHA}\CAPTHETA$ be true on $\As{}{1}$.
				Then, by the definition of truth of $\exists$, there is some $\variable{t}$-variant of $\model{}{1}$ that makes $\CAPTHETA\variable{t}/\ALPHA$ true.
				Since by the quantifier preface there is a corresponding $\variable{t}$-variant of $\model{}{2}$ for each $\variable{t}$-variant of $\model{}{1}$, it follows that there is some $\variable{t}$-variant of $\model{}{2}$ that makes $\CAPTHETA\variable{t}/\ALPHA$ true.
				So, by the definition of truth for $\exists$, $\existential{\ALPHA}\CAPTHETA$ is true on $\As{}{2}$.

				($\Leftarrow$) Let $\existential{\ALPHA}\CAPTHETA$ be true on $\As{}{2}$.
				Then, by the definition of truth of $\exists$, there is some $\variable{t}$-variant of $\model{}{2}$ that makes $\CAPTHETA\variable{t}/\ALPHA$ true.
				Since by the quantifier preface there is a corresponding $\variable{t}$-variant of $\model{}{1}$ for each $\variable{t}$-variant of $\model{}{2}$, it follows that there is some $\variable{t}$-variant of $\model{}{1}$ that makes $\CAPTHETA\variable{t}/\ALPHA$ true.
				So, by the definition of truth for $\exists$, $\existential{\ALPHA}\CAPTHETA$ is true on $\As{}{1}$.

				Thus $\existential{\ALPHA}\CAPTHETA$ is true on $\As{}{1}$ \Iff $\existential{\ALPHA}\CAPTHETA$ is true on $\As{}{2}$.
			\end{description}
			
		\end{description}
		\item[Closure Step:] There is no other way to form a \GQL{}1 sentence $\CAPPHI$, so the above clauses are sufficient to show that if $\As{}{1}$ and $\As{}{2}$ have the same domain and make the same assignments, then $\CAPPHI$ is true on $\As{}{1}$ \Iff $\CAPPHI$ is true on $\As{}{2}$.
	\end{description}
\end{PROOF}

\subsection{Logical Truth: QT, QF, \& QC}\label{QT QT QI}
Just as we have the notions of \emph{logical truth}, \emph{falsity}, and \emph{contingency} for \GSL{} (see section \ref{TFT TFF TFI}), we have analogous notions for \GQL{}1.
\begin{majorILnc}{\LnpDC{QT}}
A sentence $\CAPPHI$ of \GQL{} is \nidf{quantificationally true}\index{truth!quantificational|textbf} (\CAPS{qt})\index{QT|see{truth, quantificational}} iff it is true on every model for $\CAPPHI$.
\end{majorILnc} 
\begin{majorILnc}{\LnpDC{QF}}
A sentence $\CAPPHI$ of \GQL{} is \nidf{quantificationally false}\index{falsehood!quantificational|textbf} (\CAPS{qf})\index{QF|see{falsehood, quantificational}} iff it is false on every model for $\CAPPHI$.
\end{majorILnc} 
\begin{majorILnc}{\LnpDC{QI}}
A sentence $\CAPPHI$ of \GQL{} is \nidf{quantificationally contingent}\index{indeterminate!quantificational|textbf} (\CAPS{qc})\index{QI|see{indeterminate, quantificational}} iff there's at least one model $\As{}{1}$ on which it's true and at least one model $\As{}{2}$ on which it's false.
\end{majorILnc} 
\begin{majorILnc}{\LnpEC{GQL1LogicallTruthExampleA}}
	Prove that $\universal{\variable{x}}\pardisjunction{\Bp{\variable{x}}}{\negation{\Bp{\variable{x}}}}$ is \CAPS{qt}.
\end{majorILnc}
\begin{PROOF}
	$\universal{\variable{x}}\pardisjunction{\Bp{\variable{x}}}{\negation{\Bp{\variable{x}}}}$ is true on $\IntA$ \Iff every $\constant{a}$-variant of $\IntA$, $\As{\constant{a}}{}$, makes $\disjunction{\Bp{\constant{a}}}{\negation{\Bp{\constant{a}}}}$ true.
	One of two things must be true of each $\As{\constant{a}}{}$.
	Either $\As{\constant{a}}{}(\constant{a})\in\As{\constant{a}}{}(\BB)$ or $\As{\constant{a}}{}(\constant{a})\notin\As{\constant{a}}{}(\BB)$.
	If $\As{\constant{a}}{}(\constant{a})\in\As{\constant{a}}{}(\BB)$, then $\Bp{\constant{a}}$ is true on $\As{\constant{a}}{}$, and so is $\disjunction{\Bp{\constant{a}}}{\negation{\Bp{\constant{a}}}}$.
	If $\As{\constant{a}}{}(\constant{a})\notin\As{\constant{a}}{}(\BB)$, then $\negation{\Bp{\constant{a}}}$ is true on $\As{\constant{a}}{}$, and so is $\disjunction{\Bp{\constant{a}}}{\negation{\Bp{\constant{a}}}}$.
	So, $\disjunction{\Bp{\constant{a}}}{\negation{\Bp{\constant{a}}}}$ is true on every $\As{\constant{a}}{}$.
	It follows that $\universal{\variable{x}}\pardisjunction{\Bp{\variable{x}}}{\negation{\Bp{\variable{x}}}}$ is true on $\IntA$.
	Nothing particular was assumed about $\IntA$, so $\universal{\variable{x}}\pardisjunction{\Bp{\variable{x}}}{\negation{\Bp{\variable{x}}}}$ is true on all models.
	Therefore, $\universal{\variable{x}}\pardisjunction{\Bp{\variable{x}}}{\negation{\Bp{\variable{x}}}}$ is \CAPS{qt}.
\end{PROOF}

\begin{majorILnc}{\LnpEC{GQL1LogicallTruthExampleA2}}
	Prove that $\universal{\variable{x}}\parconjunction{\Bp{\variable{x}}}{\negation{\Bp{\variable{x}}}}$ is \CAPS{qf}.
\end{majorILnc}
\begin{PROOF}
	$\universal{\variable{x}}\parconjunction{\Bp{\variable{x}}}{\negation{\Bp{\variable{x}}}}$ is true on $\IntA$ \Iff every $\constant{a}$-variant of $\IntA$, $\As{\constant{a}}{}$, makes $\conjunction{\Bp{\constant{a}}}{\negation{\Bp{\constant{a}}}}$ true.
	There is some $\constant{a}$-variant $\As{\constant{a}}{}$ such that either $\As{\constant{a}}{}(\constant{a})\in\As{\constant{a}}{}(\BB)$ or $\As{\constant{a}}{}(\constant{a})\notin\As{\constant{a}}{}(\BB)$.
	If $\As{\constant{a}}{}(\constant{a})\in\As{\constant{a}}{}(\BB)$ then $\negation{\Bp{\constant{a}}}$ is false on $\As{\constant{a}}{}$, and so is $\conjunction{\Bp{\constant{a}}}{\negation{\Bp{\constant{a}}}}$.
	Otherwise $\As{\constant{a}}{}(\constant{a})\notin\As{\constant{a}}{}(\BB)$, in which case $\Bp{\constant{a}}$ is false on $\As{\constant{a}}{}$, and so is $\conjunction{\Bp{\constant{a}}}{\negation{\Bp{\constant{a}}}}$.
	In both cases it follows that $\universal{\variable{x}}\parconjunction{\Bp{\variable{x}}}{\negation{\Bp{\variable{x}}}}$ is false on $\IntA$.
	Nothing particular was assumed about $\IntA$, so $\universal{\variable{x}}\parconjunction{\Bp{\variable{x}}}{\negation{\Bp{\variable{x}}}}$ is false on all models.
	Therefore, $\universal{\variable{x}}\parconjunction{\Bp{\variable{x}}}{\negation{\Bp{\variable{x}}}}$ is \CAPS{qf}.
\end{PROOF}

\noindent{}In the last two examples we give proofs that the sentences in question are either \CAPS{qt} or \CAPS{qf}.
To show that a \GQL{}1 sentence $\CAPPHI$ is \CAPS{qc}, we use a different strategy: provide one model on which $\CAPPHI$ is true and another on which $\CAPPHI$ is false.

For example, we determined earlier that $\horseshoe{\universal{\variable{x}}\Dp{\variable{x}}}{\universal{\variable{x}}\Gp{\variable{x}}}$ is true on the model \emph{Pos Int}.
To show that it's \CAPS{qc} we must provide another model in which it's false.
We can modify \emph{Pos Int} to make a new model, \emph{Pos Int$^*$}.  Let \emph{Pos Int$^*$} assign the entire domain to $\DD$ so that it makes $\universal{\variable{x}}\Dp{\variable{x}}$ true.  We can keep the assignment \emph{Pos Int} makes to $\GG$, the multiples of 7.  So, on \emph{Pos Int$^*$}, $\universal{\variable{x}}\Gp{\variable{x}}$ is false.  Thus, $\horseshoe{\universal{\variable{x}}\Dp{\variable{x}}}{\universal{\variable{x}}\Gp{\variable{x}}}$ is false on \emph{Pos Int$^*$}.

The definition of model only requires that predicates be assigned a subset of the domain.
The assigned subset can be the entire domain or the empty set.
You will find that using the empty set or the entire domain as assignments is often helpful in constructing models for a desired outcome.

\begin{majorILnc}{\LnpEC{GQL1LogicallTruthExampleB}}
	The sentence $\disjunction{\universal{\variable{x}}\Bp{\variable{x}}}{\universal{\variable{x}}\negation{\Bp{\variable{x}}}}$ is \CAPS{qc}.
\end{majorILnc}
\begin{PROOF}
Let $\IntA_1$ be a model such that the universe $\integer{U}=\{1\}$ and $\IntA_1(\BB)=\{1\}$. 
$\IntA_1$ makes $\universal{\variable{x}}\Bp{\variable{x}}$ true, which in turn makes $\disjunction{\universal{\variable{x}}\Bp{\variable{x}}}{\universal{\variable{x}}\negation{\Bp{\variable{x}}}}$ true.
\begin{commentary}
	When we remove the quantifier from $\universal{\variable{x}}\Bp{\variable{x}}$ and replace the variable with a constant, the result is $\Bp{\constant{a}}$.
	Everything in the universe is also in the set assigned to $\BB$.
	So, no matter what an $\constant{a}$-variant assigns to the constant, the result is true.
\end{commentary}
\noindent{}Let $\IntA_2$ be a model such that the universe $\integer{U}=\{1,2\}$ and $\IntA_2(\BB)=\{1\}$.
$\universal{\variable{x}}\Bp{\variable{x}}$ is false on $\IntA_2$ because not everything in the domain is in the set assigned to $\BB$.
And $\universal{\variable{x}}\negation{\Bp{\variable{x}}}$ is false on $\IntA_2$ because the model doesn't make $\BB$ the empty set.
Then $\disjunction{\universal{\variable{x}}\Bp{\variable{x}}}{\universal{\variable{x}}\negation{\Bp{\variable{x}}}}$ is false in $\IntA_2$, because both disjuncts are false.  
Therefore, the sentence $\disjunction{\universal{\variable{x}}\Bp{\variable{x}}}{\universal{\variable{x}}\negation{\Bp{\variable{x}}}}$ is \CAPS{qc}.
\end{PROOF}

\begin{majorILnc}{\LnpEC{GQL1LogicallTruthExampleC}}
Prove that $\horseshoe{\universal{\variable{x}}\Dp{\variable{x}}}{\negation{\existential{\variable{x}}\negation{\Dp{\variable{x}}}}}$ is \CAPS{qt}.
\end{majorILnc}
\begin{PROOF}
Assume for indirect proof there is some model $\IntA$ that makes $\horseshoe{\universal{\variable{x}}\Dp{\variable{x}}}{\negation{\existential{\variable{x}}\negation{\Dp{\variable{x}}}}}$ false.
Thus, $\universal{\variable{x}}\Dp{\variable{x}}$ is true on $\IntA$ and $\negation{\existential{\variable{x}}\negation{\Dp{\variable{x}}}}$ is false on $\IntA$. 
From the truth of $\universal{\variable{x}}\Dp{\variable{x}}$ it follows that every $\constant{a}$-variant of $\IntA$ makes $\Dp{\constant{a}}$ true.
Because $\IntA$ makes $\negation{\existential{\variable{x}}\negation{\Dp{\variable{x}}}}$ false, it follows that $\existential{\variable{x}}\negation{\Dp{\variable{x}}}$ is true.  Given that $\IntA$ makes $\existential{\variable{x}}\negation{\Dp{\variable{x}}}$ true, there must be some $\constant{a}$-variant, $\As{\constant{a}}{}$, that makes $\negation{\Dp{\constant{a}}}$ true.
On $\As{\constant{a}}{}$, $\Dp{\constant{a}}$ must be false.
But it was already shown that every $\constant{a}$-variant of $\IntA$ makes $\Dp{\constant{a}}$ true.
Thus there is no model such that $\horseshoe{\universal{\variable{x}}\Dp{\variable{x}}}{\negation{\existential{\variable{x}}\negation{\Dp{\variable{x}}}}}$ is false.
It is therefore true on all models, and so is \CAPS{qt}.
\end{PROOF}


%%%%%%%%%%%%%%%%%%%%%%%%%%%%%%%%%%%%%%%%%%%%%%%%%%
\section{Entailment and other Relations}\label{GQL1 Entailment and other Relations}
%%%%%%%%%%%%%%%%%%%%%%%%%%%%%%%%%%%%%%%%%%%%%%%%%%

We defined the following terms in \GSL{}: entailment, equivalence, contradictory, contrary, subcontrary, and logical independence. 
Now we define these for \GQL{}1 sentences. 
While the \GSL{} definitions refer to models of \GSL{} sentences, their \GQL{}1 counterparts refer to models of \GQL{}1 sentences. 
To mark this difference, we talk of \emph{truth functional} entailment, \emph{truth functional} equivalence, etc., for \GSL{} sentences, and \emph{quantificational} entailment, \emph{quantificational} equivalence, etc., for \GQL{}1. 
We don't want to overstate the difference; the concepts underlying the definitions are essentially the same.

\begin{majorILnc}{\LnpDC{GQL1 Definition of Entailment}}
 A set $\Delta$ of \GQL{}1 sentences \df{quantificationally entails} a \GQL{}1 sentence $\CAPTHETA$ \Iff every model for $\Delta$ and $\CAPTHETA$ either makes at least one sentence in $\Delta$ $\False$ or makes $\CAPTHETA$ $\True$.
\end{majorILnc}

\noindent{}Put another way, $\Delta$ entails $\CAPTHETA$ \Iff every model that makes all sentences in $\Delta$ $\True$ also makes $\CAPTHETA$ $\True$.
As we did with \GSL{} (section \ref{Entailment}), we give two narrower consequences of this definition. 

\begin{cenumerate}
\item A finite set of \GQL{}1 sentences $\CAPPHI_1,\ldots,\CAPPHI_{\integer{n}}$ quantificationally entails another \GQL{}1 sentence $\CAPTHETA$ \Iff every model for $\CAPPHI_1,\ldots,\CAPPHI_{\integer{n}}$, and $\CAPTHETA$ either makes at least one of $\CAPPHI_1,\ldots,\CAPPHI_{\integer{n}}$ $\False$ or makes $\CAPTHETA$ $\True$.
\item A sentence $\CAPPHI$ of \GQL{} quantificationally entails another sentence $\CAPTHETA$ of \GQL{}1 \Iff every model for $\CAPPHI$ and $\CAPTHETA$ either makes $\CAPPHI$ $\False$ or makes $\CAPTHETA$ $\True$.
\end{cenumerate}

\noindent{}We continue to use the double turnstile to represent the entailment relation. 
So if $\CAPPHI$ entails $\CAPTHETA$, we write \mention{$\CAPPHI\sdtstile{}{}\CAPTHETA$}. 
If the sentences $\CAPPHI_1,\ldots,\CAPPHI_{\integer{n}}$ entail $\CAPTHETA$, we write \mention{$\CAPPHI_1,\ldots,\CAPPHI_{\integer{n}}\sdtstile{}{}\CAPTHETA$}.
And if a set $\Delta$ of sentences entails $\CAPTHETA$ we write \mention{$\Delta\sdtstile{}{}\CAPTHETA$}.

\begin{majorILnc}{\LnpEC{GQL Entailment Example}}
	Show whether the following holds: $\universal{\variable{x}}\Gp{\variable{x}}\sdtstile{}{}\Gp{\constant{a}}$.
\end{majorILnc}
\begin{PROOF}
	Assume a model $\IntA$ such that $\universal{\variable{x}}\Gp{\variable{x}}$ is true.
	By the definition of truth for $\forall$, it follows that $\Gp{\constant{a}}$ is true on all $\constant{a}$-variants of $\IntA$.
	The model $\IntA$ is an $\constant{a}$-variant of itself, so $\Gp{\constant{a}}$ is true on $\IntA$.
	Nothing was assumed about the assignments $\IntA$ makes, so for each model on which the \CAPS{lhs} is true, the \CAPS{rhs} is also true.
	Therefore $\universal{\variable{x}}\Gp{\variable{x}}\sdtstile{}{}\Gp{\constant{a}}$.
\end{PROOF}
\begin{majorILnc}{\LnpEC{GQL Entailment Example 2}}
	Show that $\universal{\variable{x}}\Gp{\variable{x}}\sdtstile{}{}\Gp{\constant{b}}$.
	\begin{commentary}
		This entailment is slightly harder to prove than the last.
		It's a quirk of our definition of truth that makes the last so easy to establish.
		By changing the constant in the sentence on the \CAPS{rhs} from \mention{$\constant{a}$} to \mention{$\constant{b}$}, we add a few steps to our proof.
	\end{commentary}
\end{majorILnc}
\begin{PROOF}
	Assume a model $\IntA$ such that $\universal{\variable{x}}\Gp{\variable{x}}$ is true.
	Then $\Gp{\constant{a}}$ is true on all $\constant{a}$-variants of $\IntA$.
	All $\constant{a}$-variants of $\IntA$ have the same universe as $\IntA$.
	So there is some $\constant{a}$-variant, $\As{\constant{a}}{}$, such that $\As{\constant{a}}{}(a)=\IntA(\constant{b})$.
	It follows that $\As{\constant{a}}{}(\constant{a})\in\As{\constant{a}}{}(\GG)$.
	$\As{\constant{a}}{}$ and $\IntA$ assign the same set to $\GG$, so $\As{\constant{a}}{}(\constant{a})\in\As{}{}(\GG)$.
	And because $\As{\constant{a}}{}(\constant{a})$ is the same element as $\IntA(\constant{b})$, it follows that $\As{}{}(\constant{b})\in\As{}{}(\GG)$.
	And so $\Gp{\constant{b}}$ is true on $\IntA$.
	Nothing was assumed about the assignments $\IntA$ makes, so for each model on which the \CAPS{lhs} is true, the \CAPS{rhs} is also true.
	Therefore $\universal{\variable{x}}\Gp{\variable{x}}\sdtstile{}{}\Gp{\constant{b}}$.
\end{PROOF}

\begin{majorILnc}{\LnpEC{GQL1Entailment}}
	Show whether $\universal{\variable{x}}\parhorseshoe{\Cp{\variable{x}}}{\Dp{\variable{x}}}, \Cp{\constant{o}}\sdtstile{}{}\Dp{\constant{o}}$.
\end{majorILnc}
\begin{PROOF}
	The entailment holds.
	Assume some model $\IntA$ such that $\universal{\variable{x}}\parhorseshoe{\Cp{\variable{x}}}{\Dp{\variable{x}}}$ and $\Cp{\constant{o}}$ are true.
	By the definition of truth for $\forall$, it follows that $\parhorseshoe{\Cp{\constant{a}}}{\Dp{\constant{a}}}$ is true on all $\constant{a}$-variants of $\IntA$.
	All $\constant{a}$-variants of $\IntA$ have the same universe as $\IntA$.
	So there is an $\constant{a}$-variant such that $\As{\constant{a}}{}(\constant{a})=\IntA(\constant{o})$.
	
	The model $\IntA$ makes $\Cp{\constant{o}}$ true, so $\IntA(\constant{o})\in\IntA(\Cp{})$.
	Because $\IntA(\constant{o})=\As{\constant{a}}{}(\constant{a})$ and $\As{\constant{a}}{}(\Cp{})=\IntA(\Cp{})$, it follows by substitution that $\As{\constant{a}}{}(\constant{a})\in\As{\constant{a}}{}(\Cp{})$.
	Hence, $\As{\constant{a}}{}$ makes $\Cp{\constant{a}}$ true.
	And because $\parhorseshoe{\Cp{\constant{a}}}{\Dp{\constant{a}}}$ is true on $\As{\constant{a}}{}$, it follows that $\As{\constant{a}}{}$ makes $\Dp{\constant{a}}$ true.
	From this it follows that $\As{\constant{a}}{}(\constant{a})\in\As{\constant{a}}{}(\Dp{})$.  $\As{\constant{a}}{}(\constant{a})=\IntA(\constant{o})$ and $\As{\constant{a}}{}(\Dp{})=\IntA(\Dp{})$, so by substitution we get: $\IntA(\constant{o})\in\IntA(\Dp{})$.
	Thus, $\IntA$ makes $\Dp{\constant{o}}$ true.

	Any model that makes the LHS true also makes the RHS true.  Therefore the entailment holds.
\end{PROOF}	
\begin{commentary}
	This entailment resembles the argument discussed at the beginning of the chapter: (1) All women are mortal, (2) Ophelia is a woman, therefore (3) Ophelia is mortal.
	To see this, interpret $\constant{o}$ as Ophelia, $\CC$ as the set of women, and $\DD$ as the set of mortals.
\end{commentary}

Consider an entailment $\Delta\sdtstile{}{}\CAPPHI$ that holds when the set $\Delta$ is empty, i.e. such that $\sdtstile{}{}\CAPPHI$.
By the definition of $\sdtstile{}{}$ every model must either make a sentence on the left false or the sentence on the right true.
But in this case there are no sentences on the \CAPS{lhs}.
So every model must make the \CAPS{rhs}, $\CAPPHI$, true.
Therefore, as was the case with \GSL{} in chapter \ref{sententiallogic}, $\sdtstile{}{}\CAPPHI$ \Iff $\CAPPHI$ is \CAPS{qt}.

\begin{majorILnc}{\LnpDC{GQL1 TFE}}
Two \GQL{}1 sentences $\CAPTHETA$ and $\CAPPHI$ are \nidf{quantificationally equivalent}\index{equivalent sentences!quantificational|textbf} \Iff all models for $\CAPTHETA$ and $\CAPPHI$ assign them the same truth value.
\end{majorILnc}
\begin{majorILnc}{\LnpDC{GQL1 contradictory}}
Two \GQL{}1 sentences $\CAPTHETA$ and $\CAPPHI$ are \nidf{quantificationally contradictory}\index{contradictory!quantificational|textbf} \Iff all models for $\CAPTHETA$ and $\CAPPHI$ assign them opposite truth values.
\end{majorILnc}
\begin{majorILnc}{\LnpDC{GQL1 contrary}}
Two \GQL{}1 sentences $\CAPTHETA$ and $\CAPPHI$ are \nidf{quantificationally contrary}\index{contraries!quantificational|textbf} \Iff they cannot both be $\True$ in the same model $\IntA$.
\end{majorILnc}
\begin{majorILnc}{\LnpDC{GQL1 subcontrary}}
Two \GQL{}1 sentences $\CAPTHETA$ and $\CAPPHI$ are \nidf{quantificationally subcontrary}\index{subcontraries!quantificational|textbf} \Iff they cannot both be $\False$ in the same model $\IntA$.
\end{majorILnc}
\begin{majorILnc}{\LnpDC{GQL1 Independent}}
Two \GQL{}1 sentences $\CAPTHETA$ and $\CAPPHI$ are \nidf{quantificationally independent}\index{independent sentences!quantificational|textbf} \Iff there are four models:
\begin{cenumerate}
	\item A model in which both $\CAPTHETA$ and $\CAPPHI$ are $\True$; 
	\item A model in which both $\CAPTHETA$ and $\CAPPHI$ are $\False$;
	\item A model in which $\CAPTHETA$ is $\True$ and $\CAPPHI$ is $\False$; and
	\item A model in which $\CAPTHETA$ is $\False$ and $\CAPPHI$ is $\True$.
\end{cenumerate}
\end{majorILnc}

First, a minor note.
These definitions only make sense for \GQL{}1 sentences.
We do not assess formulas that \emph{aren't} sentences for truth value, so none of the definitions above apply to them. 

Except for the fact that these definitions apply to sentences of \GQL{}1 instead of \GSL{}, and models for \GQL{}1 sentences instead of models for \GSL{} sentences, they are the same as the corresponding ones for \GSL{}. 
We might say that these definitions have the same \sq{structure}. 
The \emph{ideas} of equivalence, being contradictory, etc., haven't changed, even though the details of the definitions are slightly different.

The following four facts from \GSL{} also hold in \GQL{}1 (compare with the examples in section \ref{Other Relations}):

\begin{cenumerate}
	\item Contradictory sentences are also contrary, but sentences can be contrary without being contradictory: e.g. $\conjunction{\Cl}{\Dl}$ and $\conjunction{\Cl}{\negation{\Dl}}$.
	\item Contradictory sentences are also subcontrary, but sentences can be subcontrary without being contradictory: e.g. $\Dl$ and $\disjunction{\Cl}{\negation{\Dl}}$.
	\item If two sentences are both contrary and subcontrary, they are contradictory.
	\item Any two atomic sentences are independent of each other.
\end{cenumerate}

\noindent{}Because every sentence of \GSL{} is also a sentence of \GQL{}1 we can reuse the examples from the previous chapter. 

Finally, recall that in \GSL{} we have theorem \pmvref{Exponentiation of Entailment}: for all \GSL{} sentences $\CAPPHI$ and $\CAPTHETA$, $\CAPPHI\sdtstile{}{}\CAPTHETA$ \Iff $\sdtstile{}{}\parhorseshoe{\CAPPHI}{\CAPTHETA}$. 
The same theorem holds for \GQL{}1 sentences and the proof is essentially the same. 
\begin{THEOREM}{\LnpTC{Exponentiation of Entailment GQL} \GQL{}1 Exportation Theorem:} For all \GQL{}1 sentences $\CAPPHI$ and $\CAPTHETA$, $\CAPPHI\sdtstile{}{}\CAPTHETA$ \Iff $\:\sdtstile{}{}\parhorseshoe{\CAPPHI}{\CAPTHETA}$.
\end{THEOREM}
\noindent{}In addition, all the generalizations of this theorem (\ref{expo generalizations}) also hold for \GQL{}1 sentences, and again the proofs are essentially same.


%%%%%%%%%%%%%%%%%%%%%%%%%%%%%%%%%%%%%%%%%%%%%%%%%%
\section{Exercises}
%%%%%%%%%%%%%%%%%%%%%%%%%%%%%%%%%%%%%%%%%%%%%%%%%%

\notocsubsection{Formulas, Order, and Subformulas}{ex:Formulas, Order, and Subformulas1} Which of the following are \GQL{}1 \emph{formulas}? 
For those that are formulas, what is their order? 
How many subformulas does each have?
\begin{multicols}{2}
\begin{enumerate}
\item {$\universal{\variable{x}}\parhorseshoe{\Hpp{'}{\variable{x}}}{\Gpp{'}{\variable{x}}}$}
\item {$\universal{\variable{x}}\parhorseshoe{\Hpp{'}{\variable{x}}}{\Gpp{''}{\variable{x}}}$}
\item {$\universal{\variable{x}}\parhorseshoe{\Hpp{'}{\variable{x}}}{\Gpp{'_7}{\variable{x}}}$}
\item {$\universal{\variable{x}}\universal{\variable{z}}\parhorseshoe{\Hpp{'}{\variable{x}}}{\Gppp{''}{\variable{x}}{\variable{y}}}$}
\item {$\existential{\variable{y}}\universal{\variable{x}}\parhorseshoe{\Hpp{'}{\variable{x}}}{\Gpp{'}{\variable{x}}}$}
\item {$\universal{\variable{t}}\parhorseshoe{\Hpp{'}{\variable{x}}}{\Gpp{'}{\variable{x}}}$}
\item {$\disjunction{\Hpp{'}{\variable{x}}}{\Gpp{'}{\variable{x}}}$}
\item {$\universal{\variable{x}}\parconjunction{\Hpp{'}{\variable{y}}}{\Gpp{'}{\variable{z}}}$}
\end{enumerate}
\end{multicols}


\begin{longtable}[c]{ l l l l } %p{2.2in} p{2in}
	\toprule
	&\textbf{Symbol} & \multicolumn{2}{c}{\textbf{Model}} \\ \cmidrule(l){3-4}
	& & \textbf{Pos Int} & \textbf{States} \\
	\midrule 
	\endfirsthead
	\multicolumn{4}{c}{\emph{Continued from Previous Page}}\\
	\toprule
	&\textbf{Symbol} & \multicolumn{2}{c}{\textbf{Model}} \\ \cmidrule(l){3-4}
	& & \textbf{Pos Int} & \textbf{States} \\
	\midrule 
	\endhead
	\bottomrule
	\caption{Example Models}\\[-.15in]
	\multicolumn{4}{c}{\emph{Continued next Page}}\\
	\endfoot
	\bottomrule
	\caption{Example Models}\\%
	\endlastfoot%
	\label{table:Partial Models Again}%
	%\begin{tabular}{ l l l l } %p{2in} p{2in} %\begin{tabular}{ p{1in} l l } %p{2.2in} p{2in}
	%\toprule
	%&\textbf{Symbol} & \multicolumn{2}{c}{\textbf{Interpretation}} \\ \cmidrule(l){3-4}
	%& & \textbf{Pos Int} & \textbf{States} \\
	%\midrule 
	{Universe:} & & The set of positive integers & The set of US states (2024) \\ \addlinespace[.25cm]
	{Sent. Let.:}& A&$\True$&$\False$\\
	& B&$\True$&$\False$\\
	& C&$\False$&$\True$\\
	& D&$\True$&$\False$\\
	& E&$\True$&$\False$\\
	& G&$\False$&$\True$\\ \addlinespace[.25cm]
	{Constants:}&$\constant{a}$&1&Louisiana\\
	&$\constant{b}$&9&Maine\\
	&$\constant{c}$&72&Georgia\\
	&$\constant{d}$&3&Nebraska\\
	&$\constant{e}$&1&New Mexico\\
	&$\constant{f}$&2&Texas\\ \addlinespace[.25cm]
	{1-place:}&$\Ap{'}$&all pos int&Midwestern\\
	&$\Bp{'}$&empty set&name with $>5$ letters\\
	&$\Cp{'}$&even&Coastal\\
	&$\Dp{'}$&odd&on the Pacific coast\\
	&$\Ep{'}$&prime&\{Ohio\}\\
	&$\Gp{'}$&multiple of 7&\{Ohio,Alabama\}\\ \addlinespace[.25cm]
	%\bottomrule
\end{longtable}

\notocsubsection{Truth in a Model}{ex:Truth in an Interpretation1} Give the truth value of each of the following sentences on both of the models found in table \mvref{table:Partial Models Again}. 
\begin{multicols}{2}
\begin{enumerate}
\item $\existential{\variable{x}}\Gp{\variable{x}}$
\item $\negation{\existential{\variable{x}}\Gp{\variable{x}}}$
\item $\existential{\variable{x}}\negation{\Gp{\variable{x}}}$
\item $\universal{\variable{x}}\Gp{\variable{x}}$
\item $\negation{\universal{\variable{x}}\Gp{\variable{x}}}$
\item $\universal{\variable{x}}\negation{\Gp{\variable{x}}}$
\item $\conjunction{\existential{\variable{x}}\Cp{\variable{x}}}{\existential{\variable{x}}\Dp{\variable{x}}}$
\item $\existential{\variable{x}}\parconjunction{\Cp{\variable{x}}}{\Dp{\variable{x}}}$
\item $\negation{\existential{\variable{x}}\parconjunction{\Cp{\variable{x}}}{\Dp{\variable{x}}}}$
\item $\universal{\variable{x}}\parconjunction{\Cp{\variable{x}}}{\Dp{\variable{x}}}$
\item $\universal{\variable{x}}\parhorseshoe{\Cp{\variable{x}}}{\Dp{\variable{x}}}$
\item $\horseshoe{\universal{\variable{x}}\Cp{\variable{x}}}{\universal{\variable{x}}\Dp{\variable{x}}}$
\item $\negation{\universal{\variable{x}}\parhorseshoe{\Cp{\variable{x}}}{\Dp{\variable{x}}}}$
\item $\existential{\variable{x}}\parhorseshoe{\Cp{\variable{x}}}{\Dp{\variable{x}}}$
\end{enumerate}
\end{multicols}

\notocsubsection{Quantificational Truth Problems}{ex:Quantificational Truth Problems} 
For each sentence below, say whether it's a quantificational truth. 
If so, prove it. 
If not, give a model $\IntA$ that makes it false.
\begin{multicols}{2}
\begin{enumerate}
\item {$\disjunction{\universal{\variable{y}}\bparhorseshoe{\Ap{\variable{y}}}{\Bp{\variable{y}}}}{\universal{\variable{y}}\bparhorseshoe{\Bp{\variable{y}}}{\Ap{\variable{y}}}}$}
\item {$\disjunction{\existential{\variable{y}}\bparhorseshoe{\Ap{\variable{y}}}{\Bp{\variable{y}}}}{\existential{\variable{y}}\bparhorseshoe{\Bp{\variable{y}}}{\Ap{\variable{y}}}}$}
\item {$\horseshoe{\universal{\variable{y}}\bparhorseshoe{\Ap{\variable{y}}}{\Bp{\variable{y}}}}{\bparhorseshoe{\existential{\variable{y}}\Ap{\variable{y}}}{\existential{\variable{y}}\Bp{\variable{y}}}}$}
\item {$\horseshoe{\existential{\variable{y}}\bparhorseshoe{\Ap{\variable{y}}}{\Bp{\variable{y}}}}{\bparhorseshoe{\existential{\variable{y}}\Ap{\variable{y}}}{\existential{\variable{y}}\Bp{\variable{y}}}}$}
\item {$\horseshoe{\existential{\variable{y}}\bparhorseshoe{\Ap{\variable{y}}}{\Bp{\variable{y}}}}{\bparhorseshoe{\universal{\variable{y}}\Ap{\variable{y}}}{\universal{\variable{y}}\Bp{\variable{y}}}}$}
\item {$\horseshoe{\universal{\variable{y}}\negation{\Ap{\variable{y}}}}{\negation{\existential{\variable{y}}\Ap{\variable{y}}}}$}
\item {$\horseshoe{\negation{\existential{\variable{y}}\Ap{\variable{y}}}}{\universal{\variable{y}}\negation{\Ap{\variable{y}}}}$}
\item {$\horseshoe{\negation{\universal{\variable{y}}\Ap{\variable{y}}}}{\existential{\variable{y}}\negation{\Ap{\variable{y}}}}$}
\end{enumerate}
\end{multicols}
\begin{enumerate}[start=9]
\item {$\horseshoe{\universal{\variable{y}}\bparhorseshoe{\Ap{\variable{y}}}{\Bp{\variable{y}}}}{\bparhorseshoe{\universal{\variable{y}}\Ap{\variable{y}}}{\universal{\variable{y}}\Bp{\variable{y}}}}$}
\item {$\horseshoe{\universal{\variable{z}}\bparhorseshoe{\Ap{\variable{z}}}{\pardisjunction{\Bp{\variable{z}}}{\Cp{\variable{z}}}}}{\cpardisjunction{\universal{\variable{z}}\bparhorseshoe{\Ap{\variable{z}}}{\Bp{\variable{z}}}}{\universal{\variable{z}}\bparhorseshoe{\Ap{\variable{z}}}{\Cp{\variable{z}}}}}$}
\item {$\horseshoe{\universal{\variable{y}}\bparhorseshoe{\Ap{\variable{y}}}{\Bp{\variable{y}}}}{\cparhorseshoe{\universal{\variable{y}}\bparhorseshoe{\Bp{\variable{y}}}{\Cp{\variable{y}}}}{\universal{\variable{y}}\bparhorseshoe{\Ap{\variable{y}}}{\Cp{\variable{y}}}}}$}
\item {$\horseshoe{\universal{\variable{y}}\bparhorseshoe{\Ap{\variable{y}}}{\Bp{\variable{y}}}}{\cparhorseshoe{\universal{\variable{y}}\bparhorseshoe{\Cp{\variable{y}}}{\Bp{\variable{y}}}}{\universal{\variable{y}}\bparhorseshoe{\Ap{\variable{y}}}{\Cp{\variable{y}}}}}$}
\item {$\horseshoe{\universal{\variable{y}}\bparhorseshoe{\Ap{\variable{y}}}{\Bp{\variable{y}}}}{\cparhorseshoe{\existential{\variable{y}}\bparhorseshoe{\Bp{\variable{y}}}{\Cp{\variable{y}}}}{\universal{\variable{y}}\bparhorseshoe{\Ap{\variable{y}}}{\Cp{\variable{y}}}}}$}
\item {$\horseshoe{\universal{\variable{y}}\bparhorseshoe{\Ap{\variable{y}}}{\Bp{\variable{y}}}}{\cparhorseshoe{\existential{\variable{y}}\bparhorseshoe{\Bp{\variable{y}}}{\Cp{\variable{y}}}}{\existential{\variable{y}}\bparhorseshoe{\Ap{\variable{y}}}{\Cp{\variable{y}}}}}$}
\end{enumerate}


\notocsubsection{Entailment Problems for \GQL{}1}{Entailment Problems for GQL1} For each entailment below, either prove that it holds or show that it doesn't hold by giving a model that make the sentences on the \CAPS{lhs} of the turnstile true and the sentence on the \CAPS{rhs} false.
\begin{multicols}{2}
\begin{enumerate}
\item {$\universal{\variable{y}}\parhorseshoe{\Ap{\variable{y}}}{\Bp{\variable{y}}}\text{, }\universal{\variable{y}}\Ap{\variable{y}}\sdtstile{}{}\universal{\variable{y}}\Bp{\variable{y}}$}
\item {$\universal{\variable{y}}\parhorseshoe{\Ap{\variable{y}}}{\Bp{\variable{y}}}\text{, }\existential{\variable{y}}\Ap{\variable{y}}\sdtstile{}{}\existential{\variable{y}}\Bp{\variable{y}}$}
\item {$\existential{\variable{y}}\parhorseshoe{\Ap{\variable{y}}}{\Bp{\variable{y}}}\text{, }\existential{\variable{y}}\Ap{\variable{y}}\sdtstile{}{}\existential{\variable{y}}\Bp{\variable{y}}$}
\item {$\horseshoe{\universal{\variable{y}}\Ap{\variable{y}}}{\universal{\variable{y}}\Bp{\variable{y}}}\sdtstile{}{}\universal{\variable{y}}\parhorseshoe{\Ap{\variable{y}}}{\Bp{\variable{y}}}$}
\item {$\existential{\variable{y}}\pardisjunction{\Ap{\variable{y}}}{\Bp{\variable{y}}}\sdtstile{}{}\disjunction{\existential{\variable{y}}\Ap{\variable{y}}}{\existential{\variable{y}}\Bp{\variable{y}}}$}
\item {$\existential{\variable{y}}\parhorseshoe{\Ap{\variable{y}}}{\Bp{\variable{y}}}\text{, }\universal{\variable{y}}\Ap{\variable{y}}\sdtstile{}{}\universal{\variable{y}}\Bp{\variable{y}}$}
\end{enumerate}
\end{multicols}
\begin{enumerate}[start=7]
\item {$\universal{\variable{z}}\bparhorseshoe{\Ap{\variable{z}}}{\pardisjunction{\Bp{\variable{z}}}{\Cp{\variable{z}}}}\sdtstile{}{}\cpardisjunction{\universal{\variable{z}}\bparhorseshoe{\Ap{\variable{z}}}{\Bp{\variable{z}}}}{\universal{\variable{z}}\bparhorseshoe{\Ap{\variable{z}}}{\Cp{\variable{z}}}}$}
\item {$\universal{\variable{y}}\parhorseshoe{\Ap{\variable{y}}}{\Bp{\variable{y}}}\text{, }\existential{\variable{y}}\parhorseshoe{\Bp{\variable{y}}}{\Cp{\variable{y}}}\sdtstile{}{}\existential{\variable{y}}\parhorseshoe{\Ap{\variable{y}}}{\Cp{\variable{y}}}$}
\end{enumerate}

\notocsubsection{Relations Between \GQL{}1 Sentences}{ex:Relations Between GQL1 Sentences} For each sentence below, say whether it entails, is entailed by, is equivalent to, contradicts, is contrary to, is subcontrary to, or is independent from each of the other sentences. 
\begin{multicols}{2}
\begin{enumerate}
\item {$\universal{\variable{z}}\parhorseshoe{\Gp{\variable{z}}}{\Dp{\variable{z}}}$}
\item {$\horseshoe{\universal{\variable{z}}\Gp{\variable{z}}}{\universal{\variable{z}}\Dp{\variable{z}}}$}
\item {$\existential{\variable{z}}\parconjunction{\Gp{\variable{z}}}{\negation{\Dp{\variable{z}}}}$}
\item {$\existential{\variable{z}}\parconjunction{\Gp{\variable{z}}}{\Dp{\variable{z}}}$}
\item {$\universal{\variable{z}}\parconjunction{\Gp{\variable{z}}}{\Dp{\variable{z}}}$}
\item {$\existential{\variable{z}}\parhorseshoe{\Gp{\variable{z}}}{\Dp{\variable{z}}}$}
\end{enumerate}
\end{multicols}
\begin{enumerate}[start=7] 
\item {$\universal{\variable{z}}\parhorseshoe{\Gp{\variable{z}}}{\negation{\Dp{\variable{z}}}}$}
\end{enumerate}



%\theendnotes

