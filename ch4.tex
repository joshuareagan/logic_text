
%%%%%%%%%%%%%%%%%%%%%%%%%%%%%%%%%%%%%%%%%%%%%%%%%%
\chapter{Quantifier Language II}\label{quantifierlogic}
%%%%%%%%%%%%%%%%%%%%%%%%%%%%%%%%%%%%%%%%%%%%%%%%%%
% \AddToShipoutPicture*{\BackgroundPicB}

%%%%%%%%%%%%%%%%%%%%%%%%%%%%%%%%%%%%%%%%%%%%%%%%%%
\section{The Language \GQL{}}
%%%%%%%%%%%%%%%%%%%%%%%%%%%%%%%%%%%%%%%%%%%%%%%%%%

%\setcounter{DefThm}{0}

\subsection{Symbols}\label{Sec:GQLSymbols}
In this chapter we add many-place predicates to \GQL{}1.
The resulting language is \GQL{}, and its development was a significant event in the history of logic.\footnote{%
	The development of \GQL{} goes back to Gottlob Frege \citeyearpar{Frege1879,Frege1891,Frege1893}, O. H. Mitchell \citeyearpar{Mitchell1883} and Charles S. Peirce \citeyearpar{Peirce1883}, with Frege's work being independent of and unknown to the latter two. See \citealp[288]{Church1956} and \citealp[34]{Hodges2001}.  %
 It's probably safe to say that Frege and Peirce/Mitchell developed quantificational logic independently, but the extent to which Peirce and his students (like Mitchell) knew of Frege's work is a matter of debate.
It's clear they at least knew of Frege.
E.g., Ladd-Franklin \citeyearpar{LaddFranklin1883} cites Frege's \citeyearpar{Frege1879} through a review of it by Ernst Schr\"oder. 
See \citep{Dipert1984} for a brief discussion of this history.}
The 2-place predicates correspond roughly to what you get if you take an English sentence and remove two names, leaving blanks, e.g.:

\begin{RESTARTmenumerate}
	\item \mention{Goliath is taller than David} $\Rightarrow$ \mention{\_\_\_\_\_\_ is taller than \_\_\_\_\_\_}
	\item \mention{Juliet Capulet loves Romeo Montague} $\Rightarrow$ \mention{\_\_\_\_\_\_ loves \_\_\_\_\_\_}
\end{RESTARTmenumerate}

\noindent{}We may think of 2-place predicates as representing a 2-place \emph{relation}.
The \mention{taller than} relation holds between two objects when one is taller than the other.
The \mention{loves} relation holds when one person---or object of whatever kind---loves another.
We may understand 3-place predicates in a similar way.
For example:

\begin{menumerate}
	\item \mention{Three is between two and four} $\Rightarrow$ \mention{\_\_\_\_\_\_ is between \_\_\_\_\_\_ and \_\_\_\_\_\_}
\end{menumerate}

For any $n\geq2$, an $n$-place predicate represents an $n$-place relation. 
The introduction of many-place predicates significantly increases the power of our formal language. 
Consider the following argument:

\begin{menumerate}
\item All horses are animals.

Therefore,

\item All horses' tails are animals' tails.
\end{menumerate}

\noindent{}The argument is a good one, but it cannot be expressed as an entailment in either \GSL{} or \GQL{}1. 
\GQL{} can handle such arguments because the \mention{is the tail of} relation can be represented with a 2-place predicate. 

\GQL{} has all the basic symbols of \GQL{}1, plus predicate letters for $n$-placed predicates, for every integer $n$ such that $n\geq2$. 
\begin{majorILnc}{\LnpDC{Symbols of GQL}}
The \df{basic symbols} of \GQL{} are:
\begin{cenumerate}
\item Logical Connectives, Punctuation Symbols, Sentence Letters, Individual Constants, Individual Variables: same as \GQL{}1
\item 1-Place Predicates: $\Ap{'}$, $\Bp{'}$, $\ldots$, $\Tp{'}$, $\Ap{'}_1$, $\Bp{'}_1$, $\ldots$, $\Tp{'}_1$, $\Ap{'}_2$, $\Bp{'}_2$, $\ldots$
\item 2-Place Predicates: $\Ap{''}$, $\Bp{''}$, $\ldots$, $\Tp{''}$, $\Ap{''}_1$, $\Bp{''}_1$, $\ldots$, $\Tp{''}_1$, $\Ap{''}_2$, $\Bp{''}_2$, $\ldots$
\item 3-Place Predicates: $\Ap{'''}$, $\Bp{'''}$, $\ldots$
\item[] \hspace{.5in} . . . and so on for all positive integers.
\end{cenumerate}
\end{majorILnc}

\noindent{}The superscripted prime marks express the arity, or number of places, of the predicate.
The subscripted integers allow for an endless supply of predicates.
For every integer $\integer{n}$, \GQL{} contains an infinite number of $\integer{n}$-place predicates. 

\subsection{Formulas of \GQL{}}\label{Formulas of GQL}
As with \GQL{}1 we must define \GQL{} formulas\index{formulas} before getting to \GQL{} sentences.
The second base clause is the only one that differs from those of the \GQL{}1 definition of formula.
\begin{majorILnc}{\LnpDC{Definition of Formula of GQL}} The \nidf{formulas} \underdf{of \GQL{}}{formulas} are given by the following recursive definition:
\begin{description}
\item[Base Clauses:] \hfill{}
\begin{cenumerate}
\item A sentence letter is a formula.
\item An $\integer{n}$-place predicate followed by $\integer{n}$ tokens of individual constants or variables is a formula.
\end{cenumerate}
\item[Generating Clauses:] \hfill{}
\begin{cenumerate}
\item If $\CAPPHI$ is a formula then so is $\negation{\CAPPHI}$.
\item If $\CAPPHI$ and $\CAPTHETA$ are formulas then so are $\parhorseshoe{\CAPPHI}{\CAPTHETA}$ and $\partriplebar{\CAPPHI}{\CAPTHETA}$.
\item If all of $\CAPPHI_1,\CAPPHI_2,\CAPPHI_3,\CAPPHI_4,\ldots,\CAPPHI_{\integer{n}}$ are formulas (the list must include at least two formulas and be finite) then so are $\parconjunction{\CAPPHI_1}{\conjunction{\CAPPHI_2}{\conjunction{\CAPPHI_3}{\conjunction{\CAPPHI_4}{\conjunction{\ldots}{\CAPPHI_{\integer{n}}}}}}}$ and $\pardisjunction{\CAPPHI_1}{\disjunction{\CAPPHI_2}{\disjunction{\CAPPHI_3}{\disjunction{\CAPPHI_4}{\disjunction{\ldots}{\CAPPHI_{\integer{n}}}}}}}$.
\item If $\CAPPHI$ is a formula and it does not contain an expression of the form $\universal{\ALPHA}$ or $\existential{\ALPHA}$ for some \GQL{} variable $\ALPHA$, then $\universal{\ALPHA}\CAPPHI$ and $\existential{\ALPHA}\CAPPHI$ are formulas.
\end{cenumerate}
\item[Closure Clause:] A string of symbols is a formula \Iff it can be generated by the clauses above.
\end{description}
\end{majorILnc}
\noindent{}Formulas that can be constructed from the base clauses are \emph{atomic}.
$\App{'}{\constant{b}}$ is an atomic formula, as is $\Appp{''}{\variable{x}}{\constant{a}}$. 
$\Bpp{''}{\variable{x}}$ is not a formula because it has a 2-place predicate followed only by one individual variable. 
To determine whether some string is a formula, we must count \emph{tokens} of variables and constants. 
For example, $\Cppp{'''}{\variable{x}}{\variable{x}\variable{x}}$ is a formula because it has a 3-place predicate followed by three variable tokens.

$\Gppp{''}{\variable{x}}{\variable{y}}$ is a formula, so by clause 4 it follows that the following are also formulas.
This list is not exhaustive.
\begin{multicols}{2}
\begin{menumerate}
\item $\universal{\variable{x}}\Gppp{''}{\variable{x}}{\variable{y}}$ 
\item $\existential{\variable{x}}\Gppp{''}{\variable{x}}{\variable{y}}$ 
\item $\existential{\variable{z}}\Gppp{''}{\variable{x}}{\variable{y}}$
\item $\existential{\variable{y}}\universal{\variable{x}}\Gppp{''}{\variable{x}}{\variable{y}}$ 
\item $\universal{\variable{x}}\existential{\variable{y}}\Gppp{''}{\variable{x}}{\variable{y}}$ 
\item $\universal{\variable{x}}\universal{\variable{z}}\Gppp{''}{\variable{x}}{\variable{y}}$ 
\end{menumerate}
\end{multicols}
\noindent{}$\universal{\variable{x}}\universal{\x}\Gppp{''}{\variable{x}}{\variable{y}}$ is \emph{not} a formula, because it's of the form $\universal{\variable{x}}\CAPPHI$ where $\CAPPHI$ is a formula that contains the expression $\universal{\variable{x}}$.

As in \GSL{} and \GQL{}1 there are unofficial formulas.
\begin{majorILnc}{\LnpDC{Unofficial Formula of GQL}}
A string of symbols is an \nidf{unofficial} formula\index{formulas!unofficial|textbf} \Iff we can obtain it from an official formula by
\begin{cenumerate}
\item deleting outer parentheses,
\item replacing official parentheses ( ) with square brackets [ ] or curly brackets \{ \}, or
\item omitting primes $'$ on a predicate letter.
\end{cenumerate}
\end{majorILnc}

\noindent{}A unique official formula can always be reconstructed from an unofficial formula.

\subsection{Other Properties of Formulas}\label{Other Properties of Formulas} 
The concepts of subformula, order, main connective, and construction tree for formulas of \GQL{} are the same as in \GQL{}1.\footnote{See section \ref{Other Properties of Formulas1} of the last chapter.} 
\begin{majorILnc}{\LnpEC{GQLSubformulaPropertiesExampleC}}
Consider the formula $\disjunction{\existential{\variable{x}}\parconjunction{\universal{y}\Eppp{''}{\variable{x}}{\variable{y}}}{\App{'}{\variable{x}}}}{\universal{\variable{z}}\parhorseshoe{\existential{\variable{y}}\Hp{'\constant{a}}}{\Gp{'\variable{x}}}}$.
This is a disjunction; its main connective is vee, $\VEE$.
It has eleven subformulas:
\begin{enumerate}[label=(\arabic*), leftmargin=1.85\parindent,
labelindent=.35\parindent, labelsep=*, itemsep=0pt]%,start=1
\item $\disjunction{\existential{\variable{x}}\parconjunction{\universal{y}\Eppp{''}{\variable{x}}{\variable{y}}}{\Ap{'\variable{x}}}}{\universal{\variable{z}}\parhorseshoe{\existential{\variable{y}}\Hp{'\constant{a}}}{\Gp{'\variable{x}}}}$
\end{enumerate}
\vspace*{-.5cm}
\begin{multicols}{2}
\begin{enumerate}[label=(\arabic*), leftmargin=1.85\parindent,
labelindent=.35\parindent, labelsep=*, itemsep=0pt, start=2]%,start=1
\item $\existential{\variable{x}}\parconjunction{\universal{y}\Eppp{''}{\variable{x}}{\variable{y}}}{\Ap{'\variable{x}}}$
\item $\conjunction{\universal{y}\Eppp{''}{\variable{x}}{\variable{y}}}{\Ap{'\variable{x}}}$
\item $\universal{y}\Eppp{''}{\variable{x}}{\variable{y}}$
\item $\Ap{'\variable{x}}$
\item $\Eppp{''}{\variable{x}}{\variable{y}}$
\item $\universal{\variable{z}}\parhorseshoe{\existential{\variable{y}}\Hp{'\constant{a}}}{\Gp{'\variable{x}}}$
\item $\horseshoe{\existential{\variable{y}}\Hp{'\constant{a}}}{\Gp{'\variable{x}}}$
\item $\existential{\variable{y}}\Hp{'\constant{a}}$
\item $\Gp{'\variable{x}}$
\item $\Hp{'\constant{a}}$
\end{enumerate}
\end{multicols}
The construction tree of the formula is:
\begin{center}
\begin{tikzpicture}[grow=up]
\tikzset{level distance=50pt}
\tikzset{level 1/.style={level distance=65pt}}
\tikzset{sibling distance=40pt}
\tikzset{every tree node/.style={align=center,anchor=north}}
	\Tree%http://angasm.org/papers/qtree/    http://www.ling.upenn.edu/advice/latex/qtree/qtreenotes.pdf
[.{$\disjunction{\existential{\variable{x}}\parconjunction{\universal{y}\Eppp{''}{\variable{x}}{\variable{y}}}{\Ap{'\variable{x}}}}{\universal{\variable{z}}\parhorseshoe{\existential{\variable{y}}\Hp{'\constant{a}}}{\Gp{'\variable{x}}}}$}
  [.{$\universal{\variable{z}}\parhorseshoe{\existential{\variable{y}}\Hp{'\constant{a}}}{\Gp{'\variable{x}}}$}
    [.{$\horseshoe{\existential{\variable{y}}\Hp{'\constant{a}}}{\Gp{'\variable{x}}}$}
       [.{$\text{ }$\\ $\Gp{'\variable{x}}$}
       ]    
       [.{$\existential{\variable{y}}\Hp{'\constant{a}}$}
         [.{$\Hp{'\constant{a}}$}
		 ]
       ]
	]
  ]
  [.{$\existential{\variable{x}}\parconjunction{\universal{y}\Eppp{''}{\variable{x}}{\variable{y}}}{\Ap{'\variable{x}}}$} %!{\qsetw{3in}}
    [.{$\conjunction{\universal{y}\Eppp{''}{\variable{x}}{\variable{y}}}{\Ap{'\variable{x}}}$}
       [.{$\text{ }$\\ $\Ap{'\variable{x}}$}
       ]    
       [.{$\universal{y}\Eppp{''}{\variable{x}}{\variable{y}}$}
	     [.{$\Eppp{''}{\variable{x}}{\variable{y}}$}
		 ]
       ] 
	]
  ]
]%
	%\caption{Example formula tree}
	%\label{fig:ExampleFormulaTree}
\end{tikzpicture}
\end{center}
As you can see from the construction tree, the order of the formula is 5. 
\end{majorILnc}

\subsection{Sentences of \GQL{}}\label{Sentences of GQL} 
Sentences, atomic sentences, and unofficial sentences of \GQL{} are defined exactly as in \GQL{}1.\footnote{See section \ref{Sentences of GQL1} of the last chapter.} 


%%%%%%%%%%%%%%%%%%%%%%%%%%%%%%%%%%%%%%%%%%%%%%%%%%
\section{Models}\label{GQL Interpretations}
%%%%%%%%%%%%%%%%%%%%%%%%%%%%%%%%%%%%%%%%%%%%%%%%%%

\subsection{Models in \GQL{}}\label{Interpretations in GQL}
\GQL{} models are essentially the same as models of \GQL{}1 except that they accommodate many-place predicates.

\begin{majorILnc}{\LnpDC{GQL Interpretation}} 
	A \df{model} for $\CAPPHI$, $\IntA$, consists of:
	\begin{cenumerate}
		\item an assignment of a truth value $\TrueB$ or $\FalseB$ to each sentence letter in $\CAPPHI$; 
		\item a non-empty set $\integer{U}$, called the \df{universe} or \df{domain};
		\item an assignment of an element from $\integer{U}$ to each individual constant in $\CAPPHI$;
		\item an assignment of a subset of $\integer{U}$ to each 1-place predicate in $\CAPPHI$;
		\item an assignment of a set of ordered $\integer{n}$-tuples to each $\integer{n}$-place predicate in $\CAPPHI$.
		The elements in each $\integer{n}$-tuple are members of $\integer{U}$.
	\end{cenumerate}
\end{majorILnc}

\noindent{} Clause (5) is the only new part of the definition.

To illustrate \GQL{} models consider the $\integer{3}$-place predicate $\Bp{'''}$.
Let there be a model $\IntA$ on which $\IntA(U)=\set{1,2,3,4}$ and $\Bp{'''}$ stands for the \mention{between} relation for positive integers.
Then $\IntA(\Bp{'''})$ is a set of $3$-tuples $\langle x, y, z\rangle$ such that $y<x<z$:

\bigskip
\noindent{}$\IntA(\Bp{'''})=\{\langle 2, 1, 3\rangle, \langle 2, 1, 4\rangle, \langle 3, 1, 4\rangle, \langle 3, 2, 4\rangle\}$ 
\bigskip

For a second illustration, consider a model $\IntA_2$ whose domain contains exactly three people: Jack, Jill, and Bill.
Let Jack be taller than Jill and Jill be taller than Bill.
Let $\IntA_2(\Ap{''})$ is the \mention{taller than} relation.
The assignment to $\Ap{''}$ is the following set of ordered pairs:

\bigskip
\noindent{}$\IntA_2(\Ap{''})=\{\langle$Jack, Jill$\rangle, \langle$Jill, Bill$\rangle, \langle$Jack, Bill$\rangle\}$ 
\bigskip

As with \GSL{} and \GQL{}1, there are \GQL{} models for sets of sentences, models of \GQL{}, and \GQL{} minimal models:

\begin{majorILnc}{\LnpDC{Definition of Model for QL Set}}
	$\IntA$ is a \df{model for a set of \GQL{} sentences $\Delta$} \Iff $\IntA$ is a model for each sentence in $\Delta$.
\end{majorILnc}

\begin{majorILnc}{\LnpDC{Definition of Model for QL}}
	$\IntA$ is a \df{model for \GQL{}} \Iff $\IntA$ is a model for every sentence of \GQL{}.
\end{majorILnc}

\begin{majorILnc}{\LnpDC{Definition of Minimal QL Model}}
	$\IntA$ is a \df{minimal model for $\CAPPHI$} \Iff $\IntA$ makes the minimum assignments necessary for $\IntA$ to be a model for $\CAPPHI$.
\end{majorILnc}

\noindent{}The proof of the minimal model theorem for \GQL{} is almost precisely the same as \GQL{}1 proof (\ref{Two Models}) and is left as an exercise for the reader.

\subsection{Truth in a Model}\label{GQL Truth in an Interpretation}

The definition of truth in a model for \GQL{} is exactly the same as in \GQL{}1 except with an additional clause for many-place predicates.

\begin{majorILnc}{\LnpDC{Truth for GQL Sentence}}
The following clauses fix when a \GQL{} sentence $\CAPTHETA$ is \nidf{$\True$} (or \nidf{$\False$}) on a model for $\CAPTHETA$, $\IntA$:
\begin{cenumerate}
	\item A sentence letter $\CAPPHI$ is $\True$ on $\IntA$ \Iff $\As{}{}(\CAPPHI)=\TrueB$.
	\item An atomic sentence $\Pp{\variable{t}}$ with a 1-place predicate $\PP$ and an individual term $\variable{t}$ is $\True$ on $\IntA$ \Iff $\IntA(\variable{t})\in\IntA(\PP)$.
	\item\label{formtruthatomicn} An atomic sentence $\Pp{\variable{t}_1\ldots\variable{t}_{\integer{n}}}$ with an $\integer{n}$-place predicate $\PP$ is $\True$ on $\IntA$ \Iff $\langle \As{}{}(\variable{t}_1),\As{}{}(\variable{t}_2),\ldots,\As{}{}(\variable{t}_{\integer{n}}) \rangle \in \As{}{}(\PP)$. 
	\item A negation $\negation{\CAPPHI}$ is $\True$ on $\IntA$ \Iff $\CAPPHI$ is $\False$ on $\IntA$.
	\item A conjunction $\parconjunction{\CAPPHI_1}{\conjunction{\ldots}{\CAPPHI_{\integer{n}}}}$ is $\True$ on $\IntA$ \Iff all of $\CAPPHI_1,\ldots,\CAPPHI_{\integer{n}}$ are $\True$ on $\IntA$.
	\item A disjunction $\pardisjunction{\CAPPHI_1}{\disjunction{\ldots}{\CAPPHI_{\integer{n}}}}$ is $\True$ on $\IntA$ \Iff at least one of $\CAPPHI_1,\ldots,\CAPPHI_{\integer{n}}$ is $\True$ on $\IntA$.
	\item A conditional $\parhorseshoe{\CAPPSI}{\CAPPHI}$ is $\True$ on $\IntA$ \Iff the \CAPS{lhs} $\CAPPSI$ is $\False$ or the \CAPS{rhs} $\CAPPHI$ is $\True$ on $\IntA$.
	\item A biconditional $\partriplebar{\CAPPSI}{\CAPPHI}$ is $\True$ on $\IntA$ \Iff $\CAPPSI$ and $\CAPPHI$ have the same truth value on $\IntA$.
	\item\label{GQLTruthUnvQuant} A universal quantification $\universal{\ALPHA}\CAPPHI$ is $\True$ on $\IntA$ \Iff $\CAPPHI\variable{t}/\ALPHA$ is $\True$ on \emph{all} $\variable{t}$-variants of $\IntA$, where $\variable{t}$ is the first constant not in $\CAPPHI$.
	\item An existential quantification $\existential{\ALPHA}\CAPPHI$ is $\True$ on $\IntA$ \Iff $\CAPPHI\variable{t}/\ALPHA$ is $\True$ on \emph{some} $\variable{t}$-variant of $\IntA$, where $\variable{t}$ is the first constant not in $\CAPPHI$.
	\item A sentence $\CAPPHI$ is $\False$ on $\IntA$ \Iff $\CAPPHI$ is not $\True$ on $\IntA$.
\end{cenumerate}
\end{majorILnc}

\noindent{}Clause (3) is the only new one. To see how \GQL{} truth works for a two-place predicate, consider a model $\IntA$ with the following assignments:\\

\noindent{}$\IntA(\constant{j})=$ Jack\\
\noindent{}$\IntA(\constant{i})=$ Jill\\
\noindent{}$\IntA(\constant{b})=$ Bill\\
\noindent{}$\IntA(\Ap{''})=\{\langle$Jack, Jill$\rangle, \langle$Jill, Bill$\rangle, \langle$Jack, Bill$\rangle\}$\\

\begin{majorILnc}{\LnpEC{GQLTruthEasyExampleA}}
Show that the sentence $\Ap{''\constant{j}\constant{b}}$ is true on $\IntA$.
\end{majorILnc}
\begin{commentary}
	To determine the truth value of $\Ap{''\constant{j}\constant{b}}$ we must check whether $\langle\IntA(\constant{j}),\IntA(\constant{b})\rangle$ is a member of the set $\IntA(\Ap{''})$. 
\end{commentary}
\begin{PROOF}
	According to $\IntA$, $\langle\IntA(\constant{j}),\IntA(\constant{b})\rangle=\langle$Jack, Bill$\rangle$. 
	We find that $\langle$Jack, Bill$\rangle\in\IntA(\Ap{''})$, so $\Ap{''\constant{j}\constant{b}}$ is true on $\IntA$.
\end{PROOF}

\begin{majorILnc}{\LnpEC{GQLTruthEasyExampleB}}
Show that the sentence $\Ap{''\constant{b}\constant{i}}$ is false on $\IntA$.
\end{majorILnc}
\begin{PROOF}
	Since $\langle\IntA(\constant{b}),\IntA(\constant{i})\rangle=\langle$Bill, Jill$\rangle$ and $\langle$Bill, Jill$\rangle\notin\IntA(\Ap{''})$, it follows that $\Ap{''\constant{b}\constant{i}}$ is false on $\IntA$.
\end{PROOF}
\begin{commentary}
	Remember that $\langle$Bill, Jill$\rangle$ is not the same as $\langle$Jill, Bill$\rangle$.  Order matters!
\end{commentary}

\noindent{}Let's try more complicated examples using the models provided in figure \mvref{table:Example Interpretations}.

\begin{figure}
\begin{longtable}[c]{ l l l l } %p{2.2in} p{2in}
	\toprule
	&\textbf{Symbol} & \multicolumn{2}{c}{\textbf{Model}} \\ \cmidrule(l){3-4}
	& & \textbf{Pos Int} & \textbf{States} \\
	\midrule 
	\endfirsthead
	\multicolumn{4}{c}{\emph{Continued from Previous Page}}\\
	\toprule
	&\textbf{Symbol} & \multicolumn{2}{c}{\textbf{Model}} \\ \cmidrule(l){3-4}
	& & \textbf{Pos Int} & \textbf{States} \\
	\midrule 
	\endhead
	\bottomrule
	\caption{Example Models}\\[-.15in]
	\multicolumn{4}{c}{\emph{Continued next Page}}\\
	\endfoot
	\bottomrule
	\caption{Example Models}\\%
	\endlastfoot%
	\label{table:Example Interpretations}%
	%\begin{tabular}{ l l l l } %p{2in} p{2in} %\begin{tabular}{ p{1in} l l } %p{2.2in} p{2in}
	%\toprule
	%&\textbf{Symbol} & \multicolumn{2}{c}{\textbf{Interpretation}} \\ \cmidrule(l){3-4}
	%& & \textbf{Pos Int} & \textbf{States} \\
	%\midrule 
	{Universe:} & & The set of positive integers & The set of US states (2024) \\ \addlinespace[.25cm]
	{Sent. Let.:}& A&$\True$&$\False$\\
	& B&$\True$&$\False$\\
	& C&$\False$&$\True$\\
	& D&$\True$&$\False$\\
	& E&$\True$&$\False$\\
	& G&$\False$&$\True$\\ \addlinespace[.25cm]
	{Constants:}&$\constant{a}$&1&Louisiana\\
	&$\constant{b}$&9&Maine\\
	&$\constant{c}$&72&Georgia\\
	&$\constant{d}$&3&Nebraska\\
	&$\constant{e}$&1&New Mexico\\
	&$\constant{f}$&2&Texas\\ \addlinespace[.25cm]
	{1-place:}&$\Ap{'}$&all pos int&Midwestern\\
	&$\Bp{'}$&empty set&name with $>5$ letters\\
	&$\Cp{'}$&even&Coastal\\
	&$\Dp{'}$&odd&on the Pacific Coast\\
	&$\Ep{'}$&prime&\{Ohio\}\\
	&$\Gp{'}$&multiple of 7&\{Ohio, Alabama\}\\ \addlinespace[.25cm]
	{2-place:}&$\Ap{''}$&first $>$ second&share a border\\
	&$\Bp{''}$&are equal&first is north of second\\
	&$\Cp{''}$&first = 2 times second&first $>$ second (area)\\
	&$\Dp{''}$&sum of them equals 7&first $>$ second (population)\\
	&$\Ep{''}$&first $<$ second&first is west of second\\
	&$\Gp{''}$&are relatively prime&both coastal, or neither\\ \addlinespace[.25cm]
	{3-place:}&$\Ap{'''}$&all equal&all same population\\
	&$\Bp{'''}$&first $<$ second $<$ third&first is north of others\\
	&$\Cp{'''}$&all odd or all even&first $>$ second $>$ third (area)\\
	&$\Dp{'''}$&first + second = third&first + second $>$ third (area)\\
	&$\Ep{'''}$&first $\times$ second = third&first is west of the others\\
	&$\Gp{'''}$&are all relatively prime& at least two coastal \\
	%\bottomrule
\end{longtable}
\caption{Two \GQL{} models}
\end{figure}

If you look over the many-place predicates in figure \ref{table:Example Interpretations} you'll notice that we don't list explicit sets of $n$-tuples. 
Instead we provide a brief description of some relation which corresponds to some such set. 
This is our usual practice. 
Writing out all $n$-tuples explicitly would be tedious for the model \emph{States} and impossible for the model \emph{Pos Int}.

\begin{majorILnc}{\LnpEC{GQLTruthExampleA1}}
Determine the truth value of $\universal\variable{y}\existential\variable{x}\App{''\variable{x}}{\variable{y}}$ on model \emph{Pos Int} (figure \ref{table:Example Interpretations}).
\end{majorILnc}
\begin{commentary}
	Intuitively, this sentence can be read as \mention{For each $\variable{y}$, there is some $\variable{x}$ such that $\variable{x}>\variable{y}$.}
	That is, for each positive integer there is another that is greater.
	Thus, we may expect that $\universal\variable{y}\existential\variable{x}\App{''\variable{x}}{\variable{y}}$ is true on \emph{Pos Int}.
	This insight cannot serve as a proof that the sentence is true, but it can guide our efforts as we construct a proof according to the proper definitions.
\end{commentary}
\begin{PROOF}
	$\universal\variable{y}\existential\variable{x}\App{''\variable{x}}{\variable{y}}$ is true on \emph{Pos Int} \Iff $\existential\variable{x}\App{''\variable{x}}{\constant{a}}$ is true on all $\constant{a}$-variants of \emph{Pos Int} (definition of truth, $\forall$). 
	Let there be an $\constant{a}$-variant of \emph{Pos Int}, $\emph{Pos Int}^{\constant{a}}$, such that $\emph{Pos Int}^{\constant{a}}(a)=n$, where $n$ is an arbitrary positive integer.
	%So, to establish the truth of the sentence we can show that all assignments to $\constant{a}$ make $\existential\variable{x}\App{''\variable{x}}{\constant{a}}$ true.
	The sentence $\existential\variable{x}\App{''\variable{x}}{\constant{a}}$ is true on $\emph{Pos Int}^{\constant{a}}$ \Iff $\App{''\constant{b}}{\constant{a}}$ is true on some $\constant{b}$-variant of $\emph{Pos Int}^{\constant{a}}$ (definition of truth, $\exists$).
	%So, to show that an $\constant{a}$-assignment makes $\existential\variable{x}\App{''\variable{x}}{\constant{a}}$ true, we can provide any $\constant{b}$-assignment that makes $\App{''\constant{b}}{\constant{a}}$ true.
	$\emph{Pos Int}(\App{''}{})$ is the set of ordered pairs $\langle x, y \rangle$ such that $x>y$.
	Let $\emph{Pos Int}^{\constant{a}\constant{b}}$ be a $\constant{b}$-variant of $\emph{Pos Int}^{\constant{a}}$ such that $\emph{Pos Int}^{\constant{a}\constant{b}}(b)=n+1$.
	Clearly $\langle n+1, n \rangle\in\emph{Pos Int}(\App{''}{})$.
	Hence, regardless of what $n$ is, $\existential\variable{x}\App{''\variable{x}}{\constant{a}}$ is true on $\emph{Pos Int}^{\constant{a}}$.
	Because nothing about this argument depends upon a specific assignment to $\variable{a}$, it holds for all $\constant{a}$-variants of \emph{Pos Int}. 
	Therefore $\universal\variable{y}\existential\variable{x}\App{''\variable{x}}{\variable{y}}$ is true on \emph{Pos Int}.
\end{PROOF}

\begin{majorILnc}{\LnpEC{GQLTruthExampleA2}}
	Determine the truth value of $\existential\variable{x}\universal\variable{y}\App{''\variable{x}}{\variable{y}}$ on model \emph{Pos Int} (figure \ref{table:Example Interpretations}).
\end{majorILnc}
\begin{commentary}
	Intuitively, this sentence can be read as \mention{There is some $\variable{x}$ such that for all $\variable{y}$, $\variable{x}>\variable{y}$.}
	That is, there is some positive integer that is greater than all others.
	This is false, so we expect that $\existential\variable{x}\universal\variable{y}\App{''\variable{x}}{\variable{y}}$ is false on \emph{Pos Int}.
	But, as before, intuition is no substitute for proof.
\end{commentary}
\begin{PROOF}
	$\existential\variable{x}\universal\variable{y}\App{''\variable{x}}{\variable{y}}$ is true on \emph{Pos Int} \Iff $\universal\variable{y}\App{''\constant{a}}{\variable{y}}$ is true on some $\constant{a}$-variant of \emph{Pos Int}. 
	Let there be an $\constant{a}$-variant of \emph{Pos Int}, $\emph{Pos Int}^{\constant{a}}$, such that $\emph{Pos Int}^{\constant{a}}(a)=n$, where $n$ is an arbitrary positive integer.
	%So, to show that it is false we must find an assignment to $\constant{a}$ such that $\universal\variable{y}\App{''\constant{a}}{\variable{y}}$ is false.
	The sentence $\universal\variable{y}\App{''\constant{a}}{\variable{y}}$ is true on $\emph{Pos Int}^{\constant{a}}$ \Iff $\App{''\constant{a}}{\constant{b}}$ is true on all $\constant{b}$-variants of $\emph{Pos Int}^{\constant{a}}$.
	%So, to show that $\universal\variable{y}\App{''\constant{a}}{\variable{y}}$ is false on some $\constant{a}$-assignment, we need some additional $\constant{b}$-assignment that makes $\App{''\constant{a}}{\constant{b}}$ false.
	$\emph{Pos Int}(\App{''}{})$ is the set of ordered pairs $\langle x, y \rangle$ such that $x>y$.
	Let $\emph{Pos Int}^{\constant{a}\constant{b}}$ be a $\constant{b}$-variant of $\emph{Pos Int}^{\constant{a}}$ such that $\emph{Pos Int}^{\constant{a}\constant{b}}(b)=n+1$.
	Clearly $\langle n, n+1 \rangle\notin\emph{Pos Int}(\App{''}{})$.
	Hence, regardless of what $n$ is, $\App{''\constant{a}}{\constant{b}}$ is false on $\emph{Pos Int}^{\constant{a}\constant{b}}$.
	Thus, $\universal\variable{y}\App{''\constant{a}}{\variable{y}}$ is false on all $\constant{a}$-variants of \emph{Pos Int}. 
	Therefore, $\existential\variable{x}\universal\variable{y}\App{''\variable{x}}{\variable{y}}$ is false on \emph{Pos Int}.
\end{PROOF}

\noindent{}The only difference between $\universal\variable{y}\existential\variable{x}\App{''\variable{x}}{\variable{y}}$ and $\existential\variable{x}\universal\variable{y}\App{''\variable{x}}{\variable{y}}$ is the order of their quantifiers. 
That change is enough to transform the meaning of the sentence completely. 
Quantifier order matters!

\begin{majorILnc}{\LnpEC{GQLTruthExampleB}}
Show that $\universal{\variable{x}}\universal{\variable{y}}\universal{\variable{z}}\parhorseshoe{\parconjunction{\Cpp{\variable{x}}{\variable{y}}}{\Dppp{\variable{x}}{\variable{y}}{\variable{z}}}}{\Bppp{\variable{y}}{\variable{x}}{\variable{z}}}$ is true on the model \emph{Pos Int}. 
\end{majorILnc}
\begin{PROOF}
\raggedright$\universal{\variable{x}}\universal{\variable{y}}\universal{\variable{z}}\parhorseshoe{\parconjunction{\Cpp{\variable{x}}{\variable{y}}}{\Dppp{\variable{x}}{\variable{y}}{\variable{z}}}}{\Bppp{\variable{y}}{\variable{x}}{\variable{z}}}$ is true on \emph{Pos Int} \Iff $\universal{\variable{y}}\universal{\variable{z}}\parhorseshoe{\parconjunction{\Cpp{\constant{a}}{\variable{y}}}{\Dppp{\constant{a}}{\variable{y}}{\variable{z}}}}{\Bppp{\variable{y}}{\constant{a}}{\variable{z}}}$ is true on all $\constant{a}$-variants of \emph{Pos Int}, $\emph{Pos Int}^{\constant{a}}$.
$\universal{\variable{y}}\universal{\variable{z}}\parhorseshoe{\parconjunction{\Cpp{\constant{a}}{\variable{y}}}{\Dppp{\constant{a}}{\variable{y}}{\variable{z}}}}{\Bppp{\variable{y}}{\constant{a}}{\variable{z}}}$ is true on some $\emph{Pos Int}^{\constant{a}}$ \Iff $\universal{\variable{z}}\parhorseshoe{\parconjunction{\Cpp{\constant{a}}{\constant{b}}}{\Dppp{\constant{a}}{\constant{b}}{\variable{z}}}}{\Bppp{\constant{b}}{\constant{a}}{\variable{z}}}$ is true on all $\constant{b}$-variants of $\emph{Pos Int}^{\constant{a}}$, $\emph{Pos Int}^{\constant{a}\constant{b}}$.
$\universal{\variable{z}}\parhorseshoe{\parconjunction{\Cpp{\constant{a}}{\constant{b}}}{\Dppp{\constant{a}}{\constant{b}}{\variable{z}}}}{\Bppp{\constant{b}}{\constant{a}}{\variable{z}}}$ is true on some $\emph{Pos Int}^{\constant{a}\constant{b}}$ \Iff $\parhorseshoe{\parconjunction{\Cpp{\constant{a}}{\constant{b}}}{\Dppp{\constant{a}}{\constant{b}}{\constant{c}}}}{\Bppp{\constant{b}}{\constant{a}}{\constant{c}}}$ is true on all $\constant{c}$-variants of $\emph{Pos Int}^{\constant{a}\constant{b}}$, $\emph{Pos Int}^{\constant{a}\constant{b}\constant{c}}$.
So, if there is some $\emph{Pos Int}^{\constant{a}\constant{b}\constant{c}}$ such that $\parhorseshoe{\parconjunction{\Cpp{\constant{a}}{\constant{b}}}{\Dppp{\constant{a}}{\constant{b}}{\constant{c}}}}{\Bppp{\constant{b}}{\constant{a}}{\constant{c}}}$ is false, then $\universal{\variable{x}}\universal{\variable{y}}\universal{\variable{z}}\parhorseshoe{\parconjunction{\Cpp{\variable{x}}{\variable{y}}}{\Dppp{\variable{x}}{\variable{y}}{\variable{z}}}}{\Bppp{\variable{y}}{\variable{x}}{\variable{z}}}$ is false on \emph{Pos Int}; otherwise, it's true.
Assume for indirect proof that there are assignments to $\constant{a}$, $\constant{b}$, and $\constant{c}$ such that $\Cpp{\constant{a}}{\constant{b}}$ and $\Dppp{\constant{a}}{\constant{b}}{\constant{c}}$ are true and $\Bppp{\constant{b}}{\constant{a}}{\constant{c}}$ is false.
Let $\emph{Pos Int}^{\constant{a}\constant{b}\constant{c}}(b)=n$ for some arbitrary positive integer $n$.
Since $\emph{Pos Int}(\CC)$ is the set of positive integer pairs $\langle \variable{u},\variable{v}\rangle$ such that $\variable{u}=2\variable{v}$, and $\Cpp{\constant{a}}{\constant{b}}$ is true on $\emph{Pos Int}^{\constant{a}\constant{b}\constant{c}}$, it follows that $\emph{Pos Int}^{\constant{a}\constant{b}\constant{c}}(a)=2n$.
And since $\emph{Pos Int}(\DD)$ is the set of positive integer triples $\langle \variable{u},\variable{v},\variable{w}\rangle$ such that $\variable{u}+\variable{v}=\variable{w}$, and $\Dppp{\constant{a}}{\constant{b}}{\constant{c}}$ is true on $\emph{Pos Int}^{\constant{a}\constant{b}\constant{c}}$, it follows that $\emph{Pos Int}^{\constant{a}\constant{b}\constant{c}}(c)=n+2n=3n$.
Since $\emph{Pos Int}(\BB)$ is the set of positive integer triples $\langle \variable{v},\variable{u},\variable{w}\rangle$ such that $\variable{v}<\variable{u}<\variable{w}$, then $\langle \constant{b},\constant{a},\constant{c}\rangle=\langle n, 2n, 3n \rangle$.
And clearly $\langle n, 2n, 3n \rangle\in\emph{Pos Int}(\BB)$.
It follows that $\Bppp{\constant{b}}{\constant{a}}{\constant{c}}$ is true on $\emph{Pos Int}^{\constant{a}\constant{b}\constant{c}}$.
But we had assumed that $\Bppp{\constant{b}}{\constant{a}}{\constant{c}}$ is false on $\emph{Pos Int}^{\constant{a}\constant{b}\constant{c}}$. $\bot$

So there is no $\emph{Pos Int}^{\constant{a}\constant{b}\constant{c}}$ such that $\Cpp{\constant{a}}{\constant{b}}$ and $\Dppp{\constant{a}}{\constant{b}}{\constant{c}}$ are true and $\Bppp{\constant{b}}{\constant{a}}{\constant{c}}$ is false.
Therefore $\universal{\variable{x}}\universal{\variable{y}}\universal{\variable{z}}\parhorseshoe{\parconjunction{\Cpp{\variable{x}}{\variable{y}}}{\Dppp{\variable{x}}{\variable{y}}{\variable{z}}}}{\Bppp{\variable{y}}{\variable{x}}{\variable{z}}}$ is true on \emph{Pos Int}.
\end{PROOF}

\begin{majorILnc}{\LnpEC{GQLTruthExampleB2}}
	Show that $\universal{\variable{x}}\universal{\variable{y}}\universal{\variable{z}}\parhorseshoe{\parconjunction{\Cpp{\variable{x}}{\variable{y}}}{\Dppp{\variable{x}}{\variable{y}}{\variable{z}}}}{\Bppp{\variable{y}}{\variable{x}}{\variable{z}}}$ is false on the model \emph{States}. 
\end{majorILnc}
\begin{PROOF}
	As in the last example, assume for indirect proof that there are assignments to $\constant{a}$, $\constant{b}$, and $\constant{c}$ such that $\Cpp{\constant{a}}{\constant{b}}$ and $\Dppp{\constant{a}}{\constant{b}}{\constant{c}}$ are true and $\Bppp{\constant{b}}{\constant{a}}{\constant{c}}$ is false.
	Let $\emph{States}^{\constant{a}\constant{b}\constant{c}}(a)=\text{Alaska}$, $\emph{States}^{\constant{a}\constant{b}\constant{c}}(b)=\text{Delaware}$, and $\emph{States}^{\constant{a}\constant{b}\constant{c}}(c)=\text{Rhode Island}$.
	Alaska has an area of approximately $1.7\times{}10^6\text{ km}^2$, Delaware an area of approximately $2.5\times{}10^3\text{ km}^2$, and Rhode Island an area of approximately $1.5\times{}10^3\text{ km}^2$.
	Hence $\emph{States}^{\constant{a}\constant{b}\constant{c}}(\constant{a})>\emph{States}^{\constant{a}\constant{b}\constant{c}}(\constant{b})$ (area), $\emph{States}^{\constant{a}\constant{b}\constant{c}}(\constant{a})+\emph{States}^{\constant{a}\constant{b}\constant{c}}(\constant{b})>\emph{States}^{\constant{a}\constant{b}\constant{c}}(\constant{c})$ (area), but $\emph{States}^{\constant{a}\constant{b}\constant{c}}(\constant{b})$ is not north of both $\emph{States}^{\constant{a}\constant{b}\constant{c}}(\constant{a})$ and $\emph{States}^{\constant{a}\constant{b}\constant{c}}(\constant{c})$; that is, Delaware is not north of Alaska. 
	So, by clause (3) of the definition of truth, \ref{Truth for GQL Sentence}, $\Cpp{\constant{a}}{\constant{b}}$ and $\Dppp{\constant{a}}{\constant{b}}{\constant{c}}$ are true on $\emph{States}^{\constant{a}\constant{b}\constant{c}}$, while $\Bppp{\constant{b}}{\constant{a}}{\constant{c}}$ is false on $\emph{States}^{\constant{a}\constant{b}\constant{c}}$. 
	So by the definition of truth ($\WEDGE$ and $\HORSESHOE$), $\parhorseshoe{\parconjunction{\Cpp{\constant{a}}{\constant{b}}}{\Dppp{\constant{a}}{\constant{b}}{\constant{c}}}}{\Bppp{\constant{b}}{\constant{a}}{\constant{c}}}$ is false on $\emph{States}^{\constant{a}\constant{b}\constant{c}}$. 
	
	There is a $\constant{c}$-variant of $\emph{States}^{\constant{a}\constant{b}}$ on which $\parhorseshoe{\parconjunction{\Cpp{\constant{a}}{\constant{b}}}{\Dppp{\constant{a}}{\constant{b}}{\constant{c}}}}{\Bppp{\constant{b}}{\constant{a}}{\constant{c}}}$ is false.  It follows (by the definition of truth for $\forall$) that $\universal{\variable{z}}\parhorseshoe{\parconjunction{\Cpp{\constant{a}}{\constant{b}}}{\Dppp{\constant{a}}{\constant{b}}{\variable{z}}}}{\Bppp{\constant{b}}{\constant{a}}{\variable{z}}}$ is false on $\emph{States}^{\constant{a}\constant{b}}$.  So, there is, in turn, a $\constant{b}$-variant of $\emph{States}^{\constant{a}}$ on which $\universal{\variable{z}}\parhorseshoe{\parconjunction{\Cpp{\constant{a}}{\constant{b}}}{\Dppp{\constant{a}}{\constant{b}}{\variable{z}}}}{\Bppp{\constant{b}}{\constant{a}}{\variable{z}}}$ is false.  So, again, by the definition of truth for $\forall$, that $\universal{\variable{y}}\universal{\variable{z}}\parhorseshoe{\parconjunction{\Cpp{\constant{a}}{\variable{y}}}{\Dppp{\constant{a}}{\variable{y}}{\variable{z}}}}{\Bppp{\variable{y}}{\constant{a}}{\variable{z}}}$ is false on $\emph{States}^{\constant{a}}$.
	
	Finally, because there is an $\constant{a}$-variant of \emph{States} on which $\universal{\variable{y}}\universal{\variable{z}}\parhorseshoe{\parconjunction{\Cpp{\constant{a}}{\variable{y}}}{\Dppp{\constant{a}}{\variable{y}}{\variable{z}}}}{\Bppp{\variable{y}}{\constant{a}}{\variable{z}}}$ is false, $\universal{\variable{x}}\universal{\variable{y}}\universal{\variable{z}}\parhorseshoe{\parconjunction{\Cpp{\variable{x}}{\variable{y}}}{\Dppp{\variable{x}}{\variable{y}}{\variable{z}}}}{\Bppp{\variable{y}}{\variable{x}}{\variable{z}}}$ is false on \emph{States} (definition of truth, $\forall$).
\end{PROOF}

\subsection{Logical Truth: QT, QF, \& QC}\label{QT QF QI GQL}
The concepts of quantificational truth (\CAPS{qt}), quantificational falsehood (\CAPS{qf}), and quantificational contingency (\CAPS{qc}) are defined for \GQL{} exactly as in \GQL{}1.\footnote{See section \vref{QT QT QI}.} 

%%%%%%%%%%%%%%%%%%%%%%%%%%%%%%%%%%%%%%%%%%%%%%%%%%
\subsection{Entailment and other Relations}\label{GQL Entailment and other Relations}
%%%%%%%%%%%%%%%%%%%%%%%%%%%%%%%%%%%%%%%%%%%%%%%%%%

The concepts for entailment and the other logical relations are also defined for \GQL{} exactly as in \GQL{}1.\footnote{See section \vref{GQL1 Entailment and other Relations}.} 


%%%%%%%%%%%%%%%%%%%%%%%%%%%%%%%%%%%%%%%%%%%%%%%%%%
\section{The Dragnet Theorem}\label{Dragnet Theorem}
%%%%%%%%%%%%%%%%%%%%%%%%%%%%%%%%%%%%%%%%%%%%%%%%%%

In example \ref{GQL Entailment Example 2} we established that $\universal{\variable{x}}\Gp{\variable{x}}\sdtstile{}{}\Gp{\constant{b}}$ holds by reasoning as follows:

\begin{enumerate}[label=(\roman*)]
	\item Any model $\IntA$ that makes $\universal{\variable{x}}\Gp{\variable{x}}$ true makes $\Gp{\constant{a}}$ true on all $\constant{a}$-variants of $\IntA$.
	\item All $\constant{a}$-variants of $\IntA$ share the same domain as $\m$, so there is some $\constant{a}$-variant of $\IntA$, $\As{\constant{a}}{}$, such that $\As{\constant{a}}{}(\constant{a})=\IntA(\constant{b})$.
	\item Because $\As{\constant{a}}{}(\constant{a})\in\As{\constant{a}}{}(\GG)$ and $\As{\constant{a}}{}(\GG)=\IntA(\GG)$, $\As{\constant{a}}{}(\constant{a})\in\As{}{}(\GG)$.
	\item Because $\As{\constant{a}}{}(\constant{a})\in\As{}{}(\GG)$ and $\As{\constant{a}}{}(\constant{a})=\IntA(\constant{b})$, $\As{}{}(\constant{b})\in\As{}{}(\GG)$.
	Thus, $\Gp{\constant{b}}$ is true on $\IntA$.
\end{enumerate}

\noindent{}By analogous reasoning we can show that $\universal{\variable{x}}\Gp{\variable{x}}$ entails $\Gp{\constant{c}}$, $\Gp{\constant{d}}$, $\Gp{\constant{e}}$, and so on.
But it seems as if we should be able to save ourselves some work by proving a more general result.
Can we show that the entailment holds regardless of which constant follows $\GG$?
Yes, but demonstrating this is easier if we first develop \GQL{} metatheory a bit more.
For such proofs we use a metavariable, in this case $\variable{t}$: $\universal{\variable{x}}\Gp{\variable{x}}\sdtstile{}{}\Gp{\variable{x}}(\variable{t}/\variable{x})$, where $\variable{t}$ is any constant.
Remember from section \pmvref{MathEnglishVariableSubEx1} that $\CAPPHI\variable{t}/\variable{x}$ is the result of replacing every unbound token of $\variable{x}$ with a token of $\variable{t}$ in $\CAPPHI$.

We also want to prove stronger entailments, such as: $\universal{\variable{x}}\CAPPHI\sdtstile{}{}\CAPPHI\constant{b}/\variable{x}$, where $\CAPPHI$ is some \GQL{} formula with only $\variable{x}$ free.

To simplify the task of proving such general theorems we use the \mention{Dragnet Theorem}.
Informally, the theorem says: ``If you have a true sentence and replace the constants in it, but keep the same elements assigned to the new constants, you get another true sentence.''
Dragnet is extraordinarily helpful for proving many significant results about \GQL{} (e.g., in theorem \pmvref{Soundness of Quantifier Logic}, and \pmvref{MethodLemmaC}).
Although the claim behind Dragnet is fairly straightforward, its proof is long and complicated.
\\
\begin{commentary}
	The name ``Dragnet'' comes from the Dragnet radio and TV show.
    The theorem bears a close resemblance to the show's opening narration:
	``The story you are about to see is true. Only the names have been changed to protect the innocent.''
	-- Sergeant Joe Friday, LAPD.
\end{commentary}

\begin{THEOREM}{\LnpTC{The Dragnet Theorem} The Dragnet Theorem:} If
\begin{cenumerate}
\item there exists: \\(a) a list of distinct constants $\variable{t}_1$, $\variable{t}_2$, $\ldots$, $\variable{t}_{\variable{i}}$, $\variable{s}_1$, $\variable{s}_2$, $\ldots$, $\variable{s}_{\variable{i}}$, \\(b) a \GQL{} sentence $\CAPPHI$ that contains $\variable{t}_1$, $\variable{t}_2$, $\ldots$, $\variable{t}_{\variable{i}}$ but not $\variable{s}_1$, $\variable{s}_2$, $\ldots$, $\variable{s}_{\variable{i}}$, \\(c) a second sentence $\CAPPHI^*=\CAPPHI\variable{s}_1/\variable{t}_1,\variable{s}_2/\variable{t}_2,\ldots,\variable{s}_{\integer{i}}/\variable{t}_{\integer{i}}$, and
\item (a) $\As{}{1}$ and $\As{}{2}$ are \GQL{} models that have the same domain $\integer{U}$ and make the same assignments to everything in $\CAPPHI$ except $\variable{t}_1$, $\variable{t}_2$, $\ldots$, $\variable{t}_{\variable{i}}$, and \\(b) $\As{}{1}(\variable{t}_1)=\As{}{2}(\variable{s}_1)$, $\As{}{1}(\variable{t}_2)=\As{}{2}(\variable{s}_2)$, $\ldots$, $\As{}{1}(\variable{t}_{\integer{i}})=\As{}{2}(\variable{s}_{\integer{i}})$,
\end{cenumerate} 
Then: $\CAPPHI$ is true on $\As{}{1}$ iff $\CAPPHI^*$ is true on $\As{}{2}$.\footnote{
	See \citealt[66]{Mates1972} and \citealt[577]{Bergmann2003} for different versions of essentially the same theorem.
}

\end{THEOREM}

\begin{PROOF}

\begin{commentary}
In this proof we treat $^*$ as a function that takes a \GQL{} formula $\CAPPHI$ with constants $\variable{t}_1$, $\variable{t}_2$, $\ldots$, $\variable{t}_{\variable{i}}$ and returns an otherwise identical formula in which those constants are replaced with ones not in $\CAPPHI$, $\variable{s}_1$, $\variable{s}_2$, $\ldots$, $\variable{s}_{\variable{i}}$, respectively: $\CAPPHI^*=\CAPPHI\variable{s}_1/\variable{t}_1,\variable{s}_2/\variable{t}_2,\ldots,\variable{s}_{\integer{i}}/\variable{t}_{\integer{i}}$.
This stipulation allows us to take for granted that any pair of sentences $\CAPPSI$ and $\CAPPSI^*$ satisfies Dragnet condition (1) above. 
Additionally, throughout the proof we assume that $\As{}{1}$ and $\As{}{2}$ are two arbitrary models that satisfy Dragnet condition (2) above.
\end{commentary}

\begin{description}
\item[Base Step:] $\CAPPHI$ is atomic. There are three cases:
\begin{cenumerate}
\item $\CAPPHI$ is a sentence letter. Then there are no constants and $\CAPPHI$ and $\CAPPHI^*$ are be identical.
Clearly $\CAPPHI$ is true on $\As{}{1}$ iff $\CAPPHI^*$ is true on $\As{}{2}$.

\item $\CAPPHI$ is a predicate letter $\PP$ followed by $n$ constants, $\Pp{\variable{q}_{\integer{1}}\variable{q}_{\integer{2}}\ldots\variable{q}_{\integer{n}}}$.
Some or all of these constants are to be replaced in $\CAPPHI^*$.
We label the constants to be replaced $\variable{t}_1,\variable{t}_2,\ldots,\variable{t}_\integer{i}$, and their replacements $\variable{s}_1,\variable{s}_2,\ldots,\variable{s}_\integer{i}$.
Then $\CAPPHI^*=\CAPPHI\variable{s}_1/\variable{t}_1,\variable{s}_2/\variable{t}_2,\ldots,\variable{s}_{\integer{i}}/\variable{t}_{\integer{i}}$.
Let's assume, without loss of generality, that $\CAPPHI=\Pp{\variable{q}_{\integer{1}}\ldots\variable{t}_1\ldots\variable{t}_2\ldots\variable{t}_{\integer{i}}\ldots\variable{q}_{\integer{n}}}$.
So $\CAPPHI^*$ is $\Pp{\variable{q}_{\integer{1}}\ldots\variable{s}_1\ldots\variable{s}_2\ldots\variable{s}_{\integer{i}}\ldots\variable{q}_{\integer{n}}}$.
By the definition of truth, 

\begin{center}
$\CAPPHI$ is true on $\As{}{1}$ iff $\langle\As{}{1}(\variable{q}_{\integer{1}}),\ldots,\As{}{1}(\variable{t}_1),\ldots,\As{}{1}(\variable{t}_{\integer{i}}),\ldots,\As{}{1}(\variable{q}_{\integer{n}})\rangle\in\As{}{1}(\PP)$.
\end{center}

By Dragnet condition (2), $\As{}{1}(\PP)=\As{}{2}(\PP)$, so we substitute:
\begin{center}
	$\CAPPHI$ is true on $\As{}{1}$ iff $\langle\As{}{1}(\variable{q}_{\integer{1}}),\ldots,\As{}{1}(\variable{t}_1),\ldots,\As{}{1}(\variable{t}_{\integer{i}}),\ldots,\As{}{1}(\variable{q}_{\integer{n}})\rangle\in\As{}{2}(\PP)$.
\end{center}

By Dragnet condition (2), $\As{}{1}(\variable{t}_1)=\As{}{2}(\variable{s}_1)$, $\ldots$, $\As{}{1}(\variable{t}_{\integer{i}})=\As{}{2}(\variable{s}_{\integer{i}})$, so we again substitute:
\begin{center}
$\CAPPHI$ is true on $\As{}{1}$ iff $\langle\As{}{1}(\variable{q}_{\integer{1}}),\ldots,\As{}{2}(\variable{s}_1),\ldots,\As{}{2}(\variable{s}_{\integer{i}}),\ldots,\As{}{1}(\variable{q}_{\integer{n}})\rangle\in\As{}{2}(\PP)$.
\end{center}

$\As{}{1}$ and $\As{}{2}$ otherwise make all the same assignments.
So for each constant $\variable{q_k}$ in $\CAPPHI$ that \emph{isn't} replaced in $\CAPPHI^*$, $\As{}{1}(\variable{q}_{\integer{k}})=\As{}{2}(\variable{q}_{\integer{k}})$: 
\begin{center}
$\CAPPHI$ is true on $\As{}{1}$ iff $\langle\As{}{2}(\variable{q}_{\integer{1}}),\ldots,\As{}{2}(\variable{s}_1),\ldots,\As{}{2}(\variable{s}_{\integer{i}}),\ldots,\As{}{2}(\variable{q}_{\integer{n}})\rangle\in\As{}{2}(\PP)$.
\end{center}

Thus, by the definition of truth:
\begin{center}
    $\CAPPHI$ is true on $\As{}{1}$ iff $\CAPPHI^*$ is true on $\As{}{2}$.
\end{center}
\end{cenumerate}
\item[Inheritance Step:] \hfill 
\begin{description}
\item[Recursive Assumption] Assume that the Dragnet property holds of all \GQL{} sentences of order $k$ or less and that $\CAPPHI$ is an \GQL{} sentence of order $k+1$.
Consider all the ways to construct a sentence of order $k+1$:

\item[Negation:] $\CAPPHI$ is of the form $\negation{\CAPPSI}$.
Since the symbol \mention{$\NEGATION$} is not a constant it is not replaced in $\CAPPHI^*$.
Then $\CAPPHI^*=\negation{\CAPPSI^*}$.
Since $\CAPPSI$ is of order $k$, by recursive assumption (RA):

\begin{center}
$\CAPPSI$ is true on $\As{}{1}$ iff $\CAPPSI^*$ is true on $\As{}{2}$,
\end{center}

It follows that:

\begin{center}
$\CAPPSI$ is false on $\As{}{1}$ iff $\CAPPSI^*$ is false on $\As{}{2}$.
\end{center}

$\CAPPSI$ is false on $\As{}{1}$ \Iff $\negation{\CAPPSI}$ is true on $\As{}{1}$.
The same holds of $\CAPPSI^*$.
Thus:

\begin{center}
$\negation{\CAPPSI}$ is true on $\As{}{1}$ iff $\negation{\CAPPSI^*}$ is true on $\As{}{2}$.
\end{center}

\item[Conjunction:] $\CAPPHI$ is of the form $\parconjunction{\CAPPSI_1}{\conjunction{\CAPPSI_2}{\conjunction{\ldots}{\CAPPSI_{\integer{n}}}}}$. 
Since the symbol \mention{$\WEDGE$} is not a constant these tokens are not replaced in $\CAPPHI^*$.
Then $\CAPPHI^*=\parconjunction{\CAPPSI_1^*}{\conjunction{\CAPPSI_2^*}{\conjunction{\ldots}{\CAPPSI_{\integer{n}}^*}}}$. 
Let $\CAPPSI_{\integer{j}}$ be the $\integer{j}^{th}$ conjunct of $\CAPPHI$, where $1\leq j\leq n$.
Each $\CAPPSI_{\integer{j}}$ is of order $k$ or lower.
So, by RA:

\begin{center}
For all $\CAPPSI_{\integer{j}}$, $\CAPPSI_j$ is true on $\As{}{1}$ iff $\CAPPSI_j^*$ is true on $\As{}{2}$.
\end{center}

From these \mention{iff} clauses it follows that:

\begin{center}
All of $\CAPPSI_1$, $\CAPPSI_2$, $\ldots$, $\CAPPSI_{\integer{n}}$ are true on $\As{}{1}$ iff all of $\CAPPSI_1^*$, $\CAPPSI_2^*$, $\ldots$, $\CAPPSI_{\integer{n}}^*$ are true on $\As{}{2}$.
\end{center}

Thus, by the definition of truth for $\WEDGE$:

\begin{center}
$\parconjunction{\CAPPSI_1}{\conjunction{\CAPPSI_2}{\conjunction{\ldots}{\CAPPSI_{\integer{n}}}}}$ is true on $\As{}{1}$ iff  $\parconjunction{\CAPPSI_1^*}{\conjunction{\CAPPSI_2^*}{\conjunction{\ldots}{\CAPPSI_{\integer{n}}^*}}}$ is true on $\As{}{2}$.
\end{center}

\item[Disjunction:] We leave this case for the reader as an exercise.

\item[Conditional:] $\CAPPHI$ is of the form $\parhorseshoe{\CAPPSI}{\CAPTHETA}$.
Since the symbol \mention{$\HORSESHOE$} is not a constant it is not replaced in $\CAPPHI^*$.
Then $\parhorseshoe{\CAPPSI}{\CAPTHETA}^*=\parhorseshoe{\CAPPSI^*}{\CAPTHETA^*}$.
By the definition of truth for $\HORSESHOE$:

\begin{center}
$\parhorseshoe{\CAPPSI}{\CAPTHETA}$ is true on $\As{}{1}$
  iff  either
   (i) $\CAPPSI$ is false on $\As{}{1}$, or (ii)
   $\CAPTHETA$ is true on $\As{}{1}$.
\end{center}

The sentence $\CAPPSI$ is of order $k$ or lower.
Then, by RA, $\CAPPSI$ is true on $\As{}{1}$ iff $\CAPPSI^*$ is true on $\As{}{2}$.
It follows that $\CAPPSI$ is false on $\As{}{1}$ iff $\CAPPSI^*$ is false on $\As{}{2}$.
So we substitute into part (i):

\begin{center}
$\parhorseshoe{\CAPPSI}{\CAPTHETA}$ is true on $\As{}{1}$
  iff  either  
	  (i) $\CAPPSI^*$ is false on $\As{}{2}$,
        or (ii) $\CAPTHETA$ is true on $\As{}{1}$.
\end{center}

Similarly, $\CAPTHETA$ is of order $k$ or lower.
Then by RA, $\CAPTHETA$ is true on $\As{}{1}$ iff $\CAPTHETA^*$ is true on $\As{}{2}$.
So we substitute into part (ii):

\begin{center}
$\parhorseshoe{\CAPPSI}{\CAPTHETA}$ is true on $\As{}{1}$
 iff either 
  (i) $\CAPPSI^*$ is false on $\As{}{2}$,
 or (ii) $\CAPTHETA^*$ is true on $\As{}{2}$.
\end{center}

By the definition of truth for $\HORSESHOE$:

\begin{center}
$\parhorseshoe{\CAPPSI}{\CAPTHETA}$ is true on $\As{}{1}$
 iff $\parhorseshoe{\CAPPSI^*}{\CAPTHETA^*}$ is true on $\As{}{2}$.
\end{center}

And since $\parhorseshoe{\CAPPSI}{\CAPTHETA}^*=\parhorseshoe{\CAPPSI^*}{\CAPTHETA^*}$:

\begin{center}
$\parhorseshoe{\CAPPSI}{\CAPTHETA}$ is true on $\As{}{1}$ iff $\parhorseshoe{\CAPPSI}{\CAPTHETA}^*$ is true on $\As{}{2}$.
\end{center}

\item[Biconditional:] We leave this case for the reader as an exercise.

\item[Quantifier Preface:] $\CAPPHI$ is of the form $\universal{\BETA}\CAPPSI$ or $\existential{\BETA}\CAPPSI$.
Then formula $\CAPPSI$ is of order $k$. To reduce the work needed for the quantifier clauses, we first prove an intermediate result.
Let $\variable{q}$ be the first constant not in $\CAPPSI$ and let $\variable{r}$ be the first constant not in $\CAPPSI^*$.

\begin{commentary}
	We cannot assume that $\variable{q}=\variable{r}$.
	Consider the case such that $\CAPPSI=\parhorseshoe{\Dp{\variable{x}}}{\Gpp{\variable{x}}{\constant{b}}}$ and $\CAPPSI^*=\CAPPSI\constant{a}/\constant{b}=\parhorseshoe{\Dp{\variable{x}}}{\Gpp{\variable{x}}{\constant{a}}}$.
	The first constant not in $\parhorseshoe{\Dp{\variable{x}}}{\Gpp{\variable{x}}{\constant{b}}}$ is $\constant{a}$, and the first constant not in $\parhorseshoe{\Dp{\variable{x}}}{\Gpp{\variable{x}}{\constant{a}}}$ is $\constant{b}$.
\end{commentary}

Since $\m_1$ and $\m_2$ satisfy Dragnet condition (2), $\m_1(U)=\m_2(U)$.
Let $\Delta$ be the domain shared by $\m_1$ and $\m_2$.
Then:

\begin{center}
	For each $\delta\in\Delta$ there is a $\variable{q}$-variant of $\As{}{1}$, $\model{\variable{q}}{1}(\variable{q})=\delta$, and an $\variable{r}$-variant of $\As{}{2}$, $\model{\variable{r}}{2}(\variable{r})=\delta$, such that $\CAPPSI\variable{q}/\BETA$ is true on $\model{\variable{q}}{1}$ \Iff $\CAPPSI^*\variable{r}/\BETA$ is true on $\model{\variable{r}}{2}$.
\end{center}

\begin{commentary}
	The purpose of this claim may not be obvious.
	If you are willing to accept on faith that it's a useful result, feel free to continue.
	If not, then skip down to the \mention{Universal Quantification} clause and read through that until you come to the step that cites the Quantifier Preface.
	Once you understand why this Quantifier Preface is helpful, then come back and pick up where you left off.
\end{commentary}

\begin{SUBPROOF}

	\begin{commentary}
		It may be tempting to try to derive this result by applying the recursive assumption directly to $\CAPPSI\variable{q}/\BETA$ and $\CAPPSI^*\variable{r}/\BETA$.
		However, it's possible that $\variable{q}$ is in $\CAPPSI\variable{r}/\BETA$ or $\variable{r}$ is in $\CAPPSI\variable{q}/\BETA$.
		Either one violates Dragnet condition (1).
		\commentaryspace
		Our strategy is to use a third sentence, $\CAPPSI^*\variable{u}/\BETA$, where $\variable{u}$ is a variable that is not in either $\CAPPSI\variable{q}/\BETA$ or $\CAPPSI^*\variable{r}/\BETA$.
		We first prove the desired result with $\CAPPSI^*\variable{r}/\BETA$ and $\CAPPSI^*\variable{u}/\BETA$ (i.e. that there are variants of $\m_1$/$\m_2$ such that one sentence is true \Iff the other is).
		Then we do the same with $\CAPPSI\variable{q}/\BETA$ and $\CAPPSI^*\variable{u}/\BETA$.
		Finally, we can \mention{factor out} $\CAPPSI^*\variable{u}/\BETA$ as a middle term, and prove the desired result for $\CAPPSI\variable{q}/\BETA$ and $\CAPPSI^*\variable{r}/\BETA$.
	\end{commentary}

	% \begin{commentary}
	% 	We proceed in two parts.
	% 	First, we show that $\CAPPSI\variable{q}/\BETA$ and $\CAPPSI^*\variable{r}/\BETA$ satisfy Dragnet condition (1).
	% 	Second, we show that each corresponding $\model{\variable{q}}{1}$ and $\model{\variable{r}}{2}$ pair satisfy Dragnet condition (2).
	% 	Since $\CAPPSI$ is of order $k$ we are then justified in using the recursive assumption to complete the subproof.		
	% \end{commentary}

	(Part 1:) Let $\variable{u}$ be a constant not in either $\CAPPSI^*\variable{r}/\BETA$ or $\CAPPSI^*\variable{u}/\BETA$.
	Then $\CAPPSI^*\variable{r}/\BETA$ and $\CAPPSI^*\variable{u}/\BETA$ are the same except that the latter sentence has $\variable{u}$ in place of $\variable{r}$.
	These sentences satisfy Dragnet condition (1).

	Let $\delta$ be an arbitrary element of $\Delta$.
	Since $\Delta=\m_2(U)$, $\delta\in\m_2(U)$, and so there is an $\variable{r}$-variant of $\m_2$, $\model{\variable{r}}{2}$, such that $\model{\variable{r}}{2}(r)=\delta$.
	And there is a $\variable{u}$-variant of $\m_2$, $\model{\variable{u}}{2}$, such that $\model{\variable{u}}{2}(u)=\delta$.
	So, $\model{\variable{r}}{2}$ and $\model{\variable{u}}{2}$ make the same assignments to everything in $\CAPPSI^*\variable{r}/\BETA$ besides $\variable{r}$, and $\model{\variable{u}}{2}=\model{\variable{r}}{2}$.
	These models satisfy Dragnet condition (2).
	
	$\CAPPSI^*\variable{r}/\BETA$ and $\CAPPSI^*\variable{u}/\BETA$ are of order $k$.
    Thus, by RA:

	\begin{center}
		For each $\delta\in\Delta$ there is a $\variable{r}$-variant of $\As{}{2}$, $\model{\variable{r}}{2}(\variable{q})=\delta$, and a $\variable{u}$-variant of $\As{}{2}$, $\model{\variable{u}}{2}(\variable{u})=\delta$, such that $\CAPPSI^*\variable{r}/\BETA$ is true on $\model{\variable{r}}{2}$ \Iff $\CAPPSI^*\variable{u}/\BETA$ is true on $\model{\variable{u}}{2}$.
	\end{center}
	
	(Part 2:) The formula $\CAPPSI$ contains the terms $\variable{t}_1,\variable{t}_2,\ldots,\variable{t}_{\variable{i}},\BETA$.
	Then $\CAPPSI\variable{q}/\BETA$ is a sentence that contains the constants $\variable{t}_1,\variable{t}_2,\ldots,\variable{t}_{\variable{i}},\variable{q}$.
	And $\CAPPSI^*\variable{u}/\BETA$ is a sentence that contains the constants $\variable{s}_1,\variable{s}_2,\ldots,\variable{s}_{\variable{i}},\variable{u}$.
	It follows that $\CAPPSI\variable{q}/\BETA$ and $\CAPPSI^*\variable{u}/\BETA$ are exactly alike, except that $\variable{t}_1,\variable{t}_2,\ldots,\variable{t}_{\variable{i}},\variable{q}$ in $\CAPPSI\variable{q}/\BETA$ are replaced with $\variable{s}_1,\variable{s}_2,\ldots,\variable{s}_{\variable{i}},\variable{u}$ in $\CAPPSI^*\variable{u}/\BETA$.
	So the sentences $\CAPPSI\variable{q}/\BETA$ and $\CAPPSI^*\variable{u}/\BETA$ satisfy Dragnet condition (1).
	
	Let $\delta$ be an arbitrary element of $\Delta$.
	Since $\Delta=\m_1(U)$, $\delta\in\m_1(U)$, and so there is a $\variable{q}$-variant of $\m_1$, $\model{\variable{q}}{1}$, such that $\model{\variable{q}}{1}(q)=\delta$.
	And since $\Delta=\m_2(U)$, $\delta\in\m_2(U)$, and so there is a $\variable{u}$-variant of $\m_2$, $\model{\variable{u}}{2}$, such that $\model{\variable{u}}{2}(u)=\delta$.
	Hence $\As{\variable{q}}{1}(\variable{q})=\As{\variable{u}}{2}(\variable{u})$.

	It was assumed that $\As{}{1}(\variable{t}_1)=\As{}{2}(\variable{s}_1)$, $\As{}{1}(\variable{t}_2)=\As{}{2}(\variable{s}_2)$, $\ldots$, $\As{}{1}(\variable{t}_{\integer{i}})=\As{}{2}(\variable{s}_{\integer{i}})$.
	The model $\As{}{1}$ differs from $\As{\variable{q}}{1}$ only on the assignment to $\variable{q}$.
	But $\variable{q}$ is the first constant not in $\CAPPSI$, so $\As{}{1}$ and $\As{\variable{q}}{1}$ make the same assignments to each of $\variable{t}_1$, $\ldots$, $\variable{t}_\integer{i}$.
	Analogous reasoning shows that $\As{}{2}$ and $\As{\variable{u}}{2}$ make the same assignments to each of $\variable{s}_1$, $\ldots$, $\variable{s}_\integer{i}$.
	Therefore, $\As{q}{1}(\variable{t}_1)=\As{u}{2}(\variable{s}_1)$, $\As{q}{1}(\variable{t}_2)=\As{u}{2}(\variable{s}_2)$, $\ldots$, $\As{q}{1}(\variable{t}_{\integer{i}})=\As{u}{2}(\variable{s}_{\integer{i}})$, and $\As{\variable{q}}{1}(\variable{q})=\As{\variable{u}}{2}(\variable{u})$.
	
	And since $\As{\variable{q}}{1}$ and $\As{\variable{u}}{2}$ are variants of $\m_1$ and $\m_2$, respectively, they make the same assignments to everything in $\CAPPSI\variable{q}/\BETA$ apart from $\variable{t}_1,\variable{t}_2,\ldots,\variable{t}_{\variable{i}},\variable{q}$.
	Thus, $\As{\variable{q}}{1}$ and $\As{\variable{u}}{2}$ satisfy Dragnet condition (2).
	The sentences $\CAPPSI\variable{q}/\BETA$ and $\CAPPSI^*\variable{u}/\BETA$ are each of order $k$, so, by RA,
	
	\begin{center}
		For each $\delta\in\Delta$ there is a $\variable{q}$-variant of $\As{}{1}$, $\model{\variable{q}}{1}=\delta$, and a $\variable{u}$-variant of $\As{}{2}$,\\$\model{\variable{u}}{2}=\delta$, such that $\CAPPSI\variable{q}/\BETA$ is true on $\model{\variable{q}}{1}$ \Iff $\CAPPSI^*\variable{u}/\BETA$ is true on $\model{\variable{u}}{2}$.
	\end{center}

	From the conclusions of parts 1 and 2, it therefore follows that:

	\begin{center}
		For each $\delta\in\Delta$ there is a $\variable{q}$-variant of $\As{}{1}$, $\model{\variable{q}}{1}=\delta$, and an $\variable{r}$-variant of $\As{}{2}$,\\$\model{\variable{r}}{2}=\delta$, such that $\CAPPSI\variable{q}/\BETA$ is true on $\model{\variable{q}}{1}$ \Iff $\CAPPSI^*\variable{r}/\BETA$ is true on $\model{\variable{r}}{2}$.
	\end{center}

\end{SUBPROOF}

\begin{description}

	\item[Universal Quantification:] $\CAPPHI$ is of the form $\universal{\BETA}\CAPPSI$, where $\CAPPSI$ is a formula that has exactly one free variable, $\BETA$.
	Since neither \mention{$\forall$} nor \mention{$\BETA$} is a constant, $\CAPPHI^*=\universal{\BETA}\CAPPSI^*$.
	Since $\m_1$ and $\m_2$ satisfy Dragnet condition (2), $\m_1(U)=\m_2(U)$.
	Call this shared domain $\Delta$.
	By the Quantifier Preface, for each $\delta\in\Delta$ there is a $\variable{q}$-variant of $\As{}{1}$, $\model{\variable{q}}{1}$, and an $\variable{r}$-variant of $\As{}{2}$, $\model{\variable{r}}{2}$, such that $\CAPPSI\variable{q}/\BETA$ is true on $\model{\variable{q}}{1}$ \Iff $\CAPPSI^*\variable{r}/\BETA$ is true on $\model{\variable{r}}{2}$.
	From this it follows that:

	\begin{center}
		$\CAPPSI\variable{q}/\BETA$ is true on every $\variable{q}$-variant of $\As{}{1}$ iff\\
		$\CAPPSI^*\variable{r}/\BETA$ is true on every $\variable{r}$-variant of $\As{}{2}$.
	\end{center}

	By the definition of truth of $\forall$:

	\begin{center}
		$\universal{\BETA}\CAPPSI$ is true on $\As{}{1}$ iff
		$\CAPPSI\variable{q}/\BETA$ is true on every $\variable{q}$-variant of $\As{}{1}$.
	\end{center}

	And:

	\begin{center}
		$\universal{\BETA}\CAPPSI^*$ is true on $\As{}{2}$ iff
		$\CAPPSI^*\variable{r}/\BETA$ is true on every $\variable{r}$-variant of $\As{}{2}$.
	\end{center}

	Substitute these two results into the first to get:

	\begin{center}
		$\universal{\BETA}\CAPPSI$ is true on $\As{}{1}$
		iff $\universal{\BETA}\CAPPSI^*$ is true on $\As{}{2}$.
	\end{center}

	\item[Existential Quantification:] $\CAPPHI$ is of the form $\existential{\BETA}\CAPPSI$, where $\CAPPSI$ is a formula that has exactly one free variable, $\BETA$.
	Since neither \mention{$\exists$} nor \mention{$\BETA$} is a constant, $\CAPPHI^*=\existential{\BETA}\CAPPSI^*$.
	Since $\m_1$ and $\m_2$ satisfy Dragnet condition (2), $\m_1(U)=\m_2(U)$.
	Call this shared domain $\Delta$.
	By the Quantifier Preface, for each $\delta\in\Delta$ there is a $\variable{q}$-variant of $\As{}{1}$, $\model{\variable{q}}{1}$, and an $\variable{r}$-variant of $\As{}{2}$, $\model{\variable{r}}{2}$, such that $\CAPPSI\variable{q}/\BETA$ is true on $\model{\variable{q}}{1}$ \Iff $\CAPPSI^*\variable{r}/\BETA$ is true on $\model{\variable{r}}{2}$.
	From this it follows that:

	\begin{center}
		$\CAPPSI\variable{q}/\BETA$ is true on some $\variable{q}$-variant of $\As{}{1}$ iff\\
		$\CAPPSI^*\variable{r}/\BETA$ is true on some $\variable{r}$-variant of $\As{}{2}$.
	\end{center}

	By the definition of truth of $\exists$:

	\begin{center}
		$\existential{\BETA}\CAPPSI$ is true on $\As{}{1}$ iff
		$\CAPPSI\variable{q}/\BETA$ is true on some $\variable{q}$-variant of $\As{}{1}$.
	\end{center}

	And:

	\begin{center}
		$\existential{\BETA}\CAPPSI^*$ is true on $\As{}{2}$ iff
		$\CAPPSI^*\variable{r}/\BETA$ is true on some $\variable{r}$-variant of $\As{}{2}$.
	\end{center}

	Substitute these two results into the first to get:

	\begin{center}
		$\existential{\BETA}\CAPPSI$ is true on $\As{}{1}$
		iff $\existential{\BETA}\CAPPSI^*$ is true on $\As{}{2}$.
	\end{center}

\end{description}

\end{description}

\item[Closure Step:] There is no other way to construct a \GQL{} sentence, so we have shown that the Dragnet property holds of all \GQL{} sentences.
\end{description}
\end{PROOF}

Dragnet is particularly useful when proving theorems about sentence schemas, i.e., sentences of a general form but with inessential details left unspecified.
Most logic texts do not prove Dragnet but prove results using specific cases of Dragnet as needed.
In effect, we have done that work once and for all by proving Dragnet in fully general form.
We prefer to state the full proof to make its pattern evident.

\begin{majorILnc}{\LnpEC{DragnetExampleTwo}}
Let $\CAPPHI$ be a formula whose only free variable is $\variable{x}$.
Prove that $\universal{\variable{x}}\CAPPHI\sdtstile{}{}\CAPPHI\constant{b}/\variable{x}$.  
\end{majorILnc} 
 
\begin{PROOF}
Suppose $\universal{\variable{x}}\CAPPHI$ is true on $\IntA$. 
Then every $\variable{t}$-variant of $\IntA$ makes $\CAPPHI\variable{t}/\variable{x}$ true.
There is a $\variable{t}$-variant, $\As{\variable{t}}{}$, that makes the same assignment to $\variable{t}$ that $\IntA$ assigns to $\constant{b}$.
So $\CAPPHI\variable{t}/\variable{x}$ is true on $\As{\variable{t}}{}$.
The sentences $\CAPPHI\variable{t}/\variable{x}$ and $\CAPPHI\constant{b}/\variable{x}$ are the same, except that $\variable{t}$ is replaced by $\constant{b}$ in the latter.
So they satisfy Dragnet condition (1).
The models $\IntA$ and $\As{\variable{t}}{}$ make the same assignments to everything in $\CAPPHI\variable{t}/\variable{x}$ besides $\variable{t}$, and $\As{\variable{t}}{}(\variable{t})=\IntA(\constant{b})$.
So they satisfy Dragnet condition (2).
Thus, by Dragnet, $\IntA$ makes $\CAPPHI\constant{b}/\variable{x}$ true.
Therefore $\universal{\variable{x}}\CAPPHI\sdtstile{}{}\CAPPHI\constant{b}/\variable{x}$.
\end{PROOF} 

When might the Dragnet theorem be useful? 
Pay attention to the two Dragnet conditions.
When you can show that those hold there is a decent chance Dragnet can help.
In some cases you will need to be strategic in your selection of a model variant.

While the Dragnet theorem is helpful and illustrates an important general pattern it can be a bit unwieldy.
So we prove another, the \mention{Free Choice} theorem, which builds on Dragnet and saves us some hassle.

\begin{THEOREM}{\LnpTC{The Free Choice Theorem} The Free Choice Theorem:}
(i) A \GQL{} sentence of the form $\universal\ALPHA\CAPPHI$ is true on some model $\IntA$ \Iff all $\variable{s}$-variants of $\IntA$ make $\CAPPHI\variable{s}/\ALPHA$ true, where $\variable{s}$ is any constant not in $\CAPPHI$; and (ii) a \GQL{} sentence of the form $\existential\ALPHA\CAPPHI$ is true on some model $\IntA$ \Iff some $\variable{s}$-variant of $\IntA$ makes $\CAPPHI\variable{s}/\ALPHA$ true, where $\variable{s}$ is any constant not in $\CAPPHI$.
\end{THEOREM}

\begin{PROOF}
(i) Let $\CAPPHI$ be a formula with only $\ALPHA$ free.
Let $\variable{t}$ be the first constant not in $\CAPPHI$ and let $\variable{s}$ be an arbitrary constant not in $\CAPPHI$.
There are two cases.
(Case: 1) $\variable{s}=\variable{t}$. By the definition of truth for $\forall$, $\universal\ALPHA\CAPPHI$ is true on $\IntA$ \Iff all $\variable{s}$-variants of $\IntA$ make $\CAPPHI\variable{s}/\ALPHA$ true.

(Case: 2) $\variable{s}\not=\variable{t}$.
The sentences $\CAPPHI\variable{t}/\ALPHA$ and $\CAPPHI\variable{s}/\ALPHA$ are the same, except $\CAPPHI\variable{s}/\ALPHA$ has $\variable{s}$ where $\CAPPHI\variable{t}/\ALPHA$ has $\variable{t}$.
So they satisfy condition (1) of Dragnet.

Let $\m$ be some model for $\universal\ALPHA\CAPPHI$.
All variants of $\m$ share a domain.
So for each element of $\m(U)$, there is a $\variable{t}$-variant of $\IntA$ and a corresponding $\variable{s}$-variant of $\IntA$ that assign this element to $\variable{t}$ and $\variable{s}$, respectively.
Let $\As{\variable{t}}{}$ and $\As{\variable{s}}{}$ be variants such that $\As{\variable{t}}{}(\variable{t})=\As{\variable{s}}{}(\variable{s})$.
They're both variants of $\m$, so they make the same assignments to everything in $\CAPPHI\variable{t}/\ALPHA$ besides $\variable{t}$.
Thus $\As{\variable{t}}{}$ and $\As{\variable{s}}{}$ satisfy condition (2) of Dragnet.

By Dragnet,	$\CAPPHI\variable{t}/\ALPHA$ is true on $\IntA$ \Iff $\CAPPHI\variable{s}/\ALPHA$ is true on $\As{\variable{s}}{}$.
We assumed nothing special about $\variable{s}$ except that it's a constant not in $\CAPPHI$.
So, all $\variable{t}$-variants of $\IntA$ make $\CAPPHI\variable{t}/\ALPHA$ true \Iff all $\variable{s}$-variants of $\IntA$ make $\CAPPHI\variable{s}/\ALPHA$ true.
Therefore, by the definition of truth of $\forall$, $\universal\ALPHA\CAPPHI$ is true on $\m$ \Iff all $\variable{s}$-variants of $\IntA$ make $\CAPPHI\variable{s}/\ALPHA$ true.

(ii) We leave this part as an exercise for the reader.
\end{PROOF}

\begin{majorILnc}{\LnpEC{FreeChoiceExampleOne}}
	Let $\CAPPHI$ and $\CAPPSI$ be formulas whose only free variable is $\variable{x}$.  Prove that $\universal{\variable{x}}\parhorseshoe{\CAPPHI}{\CAPPSI}\sdtstile{}{}\horseshoe{\universal{\variable{x}}\CAPPHI}{\universal{\variable{x}}\CAPPSI}$.  
\end{majorILnc} 
 
\begin{PROOF}
	Let $\IntA$ be some model that makes $\universal{\variable{x}}\parhorseshoe{\CAPPHI}{\CAPPSI}$ true.
	Then, by the definition of truth for $\forall$, $\parhorseshoe{\CAPPHI}{\CAPPSI}\variable{t}/\variable{x}$ is true on all $\variable{t}$-variants of $\IntA$.
	Since $\CAPPHI$ and $\CAPPSI$ are parts of $\horseshoe{\CAPPHI}{\CAPPSI}$, and the first constant not in $\horseshoe{\CAPPHI}{\CAPPSI}$ is $\variable{t}$, $\variable{t}$ is not in $\CAPPHI$ or $\CAPPSI$.
	There are two cases.

	(Case 1:) $\CAPPHI\variable{t}/\variable{x}$ is false on at least one $\variable{t}$-variant of $\IntA$.
	Then, by the Free Choice theorem, $\universal{\variable{x}}\CAPPHI$ is false on $\IntA$, and thus, by the definition of truth for $\HORSESHOE$, $\horseshoe{\universal{\variable{x}}\CAPPHI}{\universal{\variable{x}}\CAPPSI}$ is true on $\IntA$.

	\begin{commentary}
		We cannot directly use the definition of truth to conclude that $\universal{\variable{x}}\CAPPHI$ is false on $\IntA$ because we cannot prove that $\variable{t}$ is the first constant not in $\CAPPHI$.
		For all we know, it isn't.
		Working around this with Dragnet would be a bit of work, but Free Choice makes it easy.
	\end{commentary}

	(Case 2:) $\CAPPHI\variable{t}/\variable{x}$ is not false on at any $\variable{t}$-variant of $\IntA$.
	So it's true on all $\variable{t}$-variants.
	Since $\parhorseshoe{\CAPPHI}{\CAPPSI}\variable{t}/\variable{x}$ is also true on all $\variable{t}$-variants, then by the definition of truth for $\HORSESHOE$, $\CAPPSI\variable{t}/\variable{x}$ is true on all $\variable{t}$-variants.
	By the Free Choice theorem, $\universal{\variable{x}}\CAPPSI$ is true on $\IntA$, and thus, by the definition of truth for $\HORSESHOE$, $\horseshoe{\universal{\variable{x}}\CAPPHI}{\universal{\variable{x}}\CAPPSI}$ is true on $\IntA$.

	Nothing particular was assumed about $\IntA$.
	Any model that makes $\universal{\variable{x}}\parhorseshoe{\CAPPHI}{\CAPPSI}$ true also makes $\horseshoe{\universal{\variable{x}}\CAPPHI}{\universal{\variable{x}}\CAPPSI}$ true.
	Therefore, $\universal{\variable{x}}\parhorseshoe{\CAPPHI}{\CAPPSI}\sdtstile{}{}\horseshoe{\universal{\variable{x}}\CAPPHI}{\universal{\variable{x}}\CAPPSI}$.	
\end{PROOF}

%%%%%%%%%%%%%%%%%%%%%%%%%%%%%%%%%%%%%%%%%%%%%%%%%%
\section{Exercises}
%%%%%%%%%%%%%%%%%%%%%%%%%%%%%%%%%%%%%%%%%%%%%%%%%%

\notocsubsection{Formulas, Order, and Subformulas}{ex:Formulas, Order, and Subformulas} Which of the following are \emph{formulas}? 
For those that are formulas, what is their order? 
How many subformulas does each have?
\begin{multicols}{2}
\begin{enumerate}
\item {$\universal{\variable{x}}\parhorseshoe{\Hpp{'}{\variable{x}}}{\Gpp{'}{\variable{x}}}$}
\item {$\universal{\variable{x}}\parhorseshoe{\Hpp{'}{\variable{x}}}{\Gpp{''}{\variable{x}}}$}
\item {$\universal{\variable{x}}\parhorseshoe{\Hpp{'}{\variable{x}}}{\Gpp{'_7}{\variable{x}}}$}
\item {$\universal{\variable{x}}\universal{\variable{z}}\parhorseshoe{\Hpp{'}{\variable{x}}}{\Gppp{''}{\variable{x}}{\variable{y}}}$}
\item {$\universal{\variable{x}}\universal{\variable{y}}\parhorseshoe{\Hpp{'}{\variable{x}}}{\Gppp{''}{\variable{x}}{\variable{y}}}$}
\item {$\universal{\variable{x}}\universal{\variable{z}}\parhorseshoe{\Hpp{'}{\constant{a}}}{\Gppp{''}{\variable{x}}{\variable{y}}}$}
\item {$\universal{\variable{x}}\universal{\variable{z}}\parhorseshoe{\Hp{\variable{x}}}{\Gppp{''}{\variable{x}}{\variable{y}}}$}
\item {$\universal{\variable{x}}\universal{\variable{z}}\parhorseshoe{\Wpp{'}{\variable{x}}}{\Gppp{''}{\variable{x}}{\variable{y}}}$}
\item {$\universal{\variable{x}}\universal{\variable{z}}\bparhorseshoe{\Hpp{'}{\variable{x}}}{\Gppp{''}{\variable{x}}{\variable{y}}}$}
\item {$\universal{\variable{x}_9}\universal{\variable{z}}\parhorseshoe{\Hpp{'}{\variable{x}_9}}{\Gppp{''}{\variable{x}}{\variable{y}}}$}
\item {$\negation{\universal{\variable{x}}\universal{\variable{y}}\parhorseshoe{\Hpp{'}{\variable{x}}}{\Gppp{''}{\variable{x}}{\variable{y}}}}$}
\item {$\universal{\constant{b}}\universal{\variable{y}}\parhorseshoe{\Hpp{'}{\constant{b}}}{\Gppp{''}{\variable{x}}{\variable{y}}}$}
\item {$\universal{\variable{x}}\parhorseshoe{\universal{\variable{x}}\Hpp{'}{x}}{\Gppp{''}{\variable{x}}{\variable{y}}}$}
\item {$\universal{\variable{x}}\parhorseshoe{\universal{\variable{z}}\Hpp{'}{x}}{\Gppp{''}{\variable{x}}{\variable{y}}}$}
\item {$\universal{\variable{y}}\parhorseshoe{\universal{\variable{x}}\Hpp{'}{x}}{\universal{\variable{x}}\Gppp{''}{\variable{x}}{\variable{y}}}$}
\end{enumerate}
\end{multicols}

\notocsubsection{Sentences and Order}{ex:Sentences and Order} For each of the following say whether it is an official sentence, an unofficial sentence, an official formula but not a sentence, an unofficial formula but not a sentence or none of the above. 
If it is a formula or sentence (official or unofficial) say what its order is and how many sub\emph{formulas} it has.
\begin{multicols}{2}
\begin{enumerate}
\item {$\universal{\variable{x}}\universal{\variable{y}}\parhorseshoe{\Hpp{'}{\variable{x}}}{\Gppp{''}{\variable{x}}{\variable{y}}}$}
\item {$\universal{\variable{x}}\existential{\variable{z}}\parhorseshoe{\Hpp{'}{\variable{x}}}{\Gppp{''}{\variable{x}}{\variable{y}}}$}
\item {$\universal{\variable{x}}\universal{\variable{y}}\parconjunction{\Hp{\variable{x}}}{\conjunction{\Gpp{\variable{x}}{\variable{y}}}{\Kp{\variable{y}}}}$}
\item {$\universal{\variable{x}}\universal{\variable{y}}\parhorseshoe{\Hp{\variable{x}}}{\horseshoe{\Gpp{\variable{x}}{\variable{y}}}{\Kp{\variable{y}}}}$}
\item {$\universal{\variable{x}}\universal{\variable{y}}\parconjunction{\negation{\Hp{\variable{x}}}}{\conjunction{\Gpp{\variable{z}}{\variable{y}}}{\Kp{\variable{y}}}}$}
\item {$\universal{\variable{x}}\universal{\variable{y}}\parconjunction{\Hp{\variable{x}}}{\parconjunction{\Gpp{\variable{z}}{\variable{y}}}{\Kp{\variable{y}}}}$}
\item {$\universal{\variable{x}}\universal{\variable{y}}\parconjunction{\Hpp{'}{\variable{x}}}{\parconjunction{\Gppp{''}{\variable{z}}{\variable{y}}}{\Kpp{''}{\variable{y}}}}$}
\item {$\universal{\variable{x}}\universal{\variable{y}}\conjunction{\Hp{\variable{x}}}{\parconjunction{\Gpp{\variable{z}}{\variable{y}}}{\Kp{\variable{y}}}}$}
\item {$\universal{\variable{x}}\existential{\variable{z}}\universal{\variable{y}}\parconjunction{\Hp{\variable{x}}}{\conjunction{\Gpp{\variable{x}}{\variable{y}}}{\Kp{\variable{y}}}}$}
\item {$\universal{\variable{x}}\existential{\variable{y}}\universal{\variable{z}}\parconjunction{\Hp{\variable{x}}}{\conjunction{\universal{\variable{y}}\Gpp{\variable{x}}{\variable{y}}}{\Kp{\variable{y}}}}$}
\end{enumerate}
\end{multicols}

\notocsubsection{Truth in a Model}{ex:Truth in an Interpretation} Give the truth value of each of the following sentences on both of the models found in figure \mvref{table:Example Interpretations Exercise}. 
\begin{multicols}{2}
\begin{enumerate}
\item {$\universal{\variable{x}}\universal{\variable{y}}\cparhorseshoe{\parconjunction{\Ap{\variable{x}}}{\Bp{\variable{y}}}}{\App{\variable{x}}{\variable{y}}}$}
\item {$\universal{\variable{x}}\universal{\variable{y}}\cparhorseshoe{\parconjunction{\Cp{\variable{x}}}{\Dp{\variable{y}}}}{\Dpp{\variable{x}}{\variable{y}}}$}
\item {$\universal{\variable{x}}\cparhorseshoe{\parconjunction{\Cp{\variable{x}}}{\Ep{\variable{x}}}}{\Cpp{\variable{x}}{\constant{a}}}$}
\item {$\universal{\variable{x}}\universal{\variable{y}}\cparhorseshoe{\Cpp{\variable{x}}{\variable{y}}}{\App{\variable{x}}{\variable{y}}}$}
\item {$\horseshoe{\universal{\variable{x}}\Cp{\variable{x}}}{\universal{\variable{y}}\Dp{\variable{y}}}$}
\item {$\universal{\variable{z}}\universal{\variable{w}}\cparhorseshoe{\parconjunction{\Gp{\variable{z}}}{\Gp{\variable{w}}}}{\negation{\Gpp{\variable{z}}{\variable{w}}}}$}
\item {$\universal{\variable{z}}\universal{\variable{w}}\universal{\variable{x}}\cparhorseshoe{\Cppp{\variable{x}}{\variable{z}}{\variable{w}}}{\bpartriplebar{\Cp{\variable{x}}}{\Cp{\variable{w}}}}$}
\item {$\universal{\variable{x}}\cparhorseshoe{\Ap{\variable{x}}}{\existential{\variable{y}}\parconjunction{\Cp{\variable{y}}}{\Bpp{\variable{y}}{\variable{x}}}}$}
\item {$\existential{\variable{y}}\existential{\variable{x}}\parconjunction{\Cpp{\variable{x}}{\variable{y}}}{\Dpp{\variable{x}}{\variable{y}}}$}
\item {$\existential{\variable{y}}\existential{\variable{x}}\parconjunction{\Epp{\variable{x}}{\variable{y}}}{\Gpp{\variable{x}}{\variable{y}}}$}
\item {$\universal{\variable{x}}\cparhorseshoe{\Gp{\variable{x}}}{\universal{\variable{w}}\bparhorseshoe{\App{\variable{x}}{\variable{w}}}{\pardisjunction{\Ap{\variable{w}}}{\Cp{\variable{w}}}}}$}
\item {$\existential{\variable{x}}\cparhorseshoe{\Cp{\variable{x}}}{\universal{\variable{y}}\Cp{\variable{y}}}$}
\item {$\universal{\variable{z}}\universal{\variable{w}}\universal{\variable{x}}\cparhorseshoe{\Dppp{\variable{x}}{\variable{z}}{\variable{w}}}{\App{\variable{x}}{\variable{w}}}$}
\item {$\universal{\variable{y}}\bparhorseshoe{\Ap{\variable{y}}}{\existential{\variable{x}}\parconjunction{\Ep{\variable{x}}}{\App{\variable{x}}{\variable{y}}}}$}
\item {$\universal{\variable{x}}\universal{\variable{y}}\parhorseshoe{\Gpp{\variable{x}}{\variable{y}}}{\Gpp{\variable{y}}{\variable{x}}}$}
\end{enumerate}
\end{multicols}

\begin{figure}
\begin{longtable}[c]{ l l l l } %p{2.2in} p{2in}
	\toprule
	&\textbf{Symbol} & \multicolumn{2}{c}{\textbf{Model}} \\ \cmidrule(l){3-4}
	& & \textbf{Pos Int} & \textbf{States} \\
	\midrule 
	\endfirsthead
	\multicolumn{4}{c}{\emph{Continued from Previous Page}}\\
	\toprule
	&\textbf{Symbol} & \multicolumn{2}{c}{\textbf{Model}} \\ \cmidrule(l){3-4}
	& & \textbf{Pos Int} & \textbf{States} \\
	\midrule 
	\endhead
	\bottomrule
	\caption{Example Models}\\[-.15in]
	\multicolumn{4}{c}{\emph{Continued next Page}}\\
	\endfoot
	\bottomrule
	\caption{Example Models}\\%
	\endlastfoot%
	\label{table:Example Interpretations Exercise}%
	%\begin{tabular}{ l l l l } %p{2in} p{2in} %\begin{tabular}{ p{1in} l l } %p{2.2in} p{2in}
	%\toprule
	%&\textbf{Symbol} & \multicolumn{2}{c}{\textbf{Interpretation}} \\ \cmidrule(l){3-4}
	%& & \textbf{Pos Int} & \textbf{States} \\
	%\midrule 
	{Universe:} & & The set of positive integers & The set of US states (2024) \\ \addlinespace[.25cm]
	{Sent. Let.:}& A&$\True$&$\False$\\
	& B&$\True$&$\False$\\
	& C&$\False$&$\True$\\
	& D&$\True$&$\False$\\
	& E&$\True$&$\False$\\
	& G&$\False$&$\True$\\ \addlinespace[.25cm]
	{Constants:}&$\constant{a}$&1&Louisiana\\
	&$\constant{b}$&9&Maine\\
	&$\constant{c}$&72&Georgia\\
	&$\constant{d}$&3&Nebraska\\
	&$\constant{e}$&1&New Mexico\\
	&$\constant{f}$&2&Texas\\ \addlinespace[.25cm]
	{1-place:}&$\Ap{'}$&all pos int&Midwestern\\
	&$\Bp{'}$&empty set&name with $>5$ letters\\
	&$\Cp{'}$&even&Coastal\\
	&$\Dp{'}$&odd&one of original 13\\
	&$\Ep{'}$&prime&\{Ohio\}\\
	&$\Gp{'}$&multiple of 7&\{Ohio, Alabama\}\\ \addlinespace[.25cm]
	{2-place:}&$\Ap{''}$&first $>$ second&share a border\\
	&$\Bp{''}$&are equal&first is north of second\\
	&$\Cp{''}$&first = 2 times second&first $>$ second (area)\\
	&$\Dp{''}$&sum of them equals 7&first $>$ second (population)\\
	&$\Ep{''}$&first $<$ second&first is west of second\\
	&$\Gp{''}$&are relatively prime&both coastal, or neither\\ \addlinespace[.25cm]
	{3-place:}&$\Ap{'''}$&all equal&all same population\\
	&$\Bp{'''}$&first $<$ second $<$ third&first is north of others\\
	&$\Cp{'''}$&all odd or all even&first $>$ second $>$ third (area)\\
	&$\Dp{'''}$&first + second = third&first + second $>$ third (area)\\
	&$\Ep{'''}$&first $\times$ second = third&first is west of the others\\
	&$\Gp{'''}$&are all relatively prime& at least two coastal \\
	%\bottomrule
\end{longtable}
\caption{Two \GQL{} models}
\end{figure}

\notocsubsection{Quantificational Truth Problems}{ex:More Quantificational Truth Problems} For each sentence below say whether it's a quantificational truth. 
If so, prove it. 
If not, show it by giving a model $\IntA$ that makes it false.
\begin{multicols}{2} 
\begin{enumerate}
\item {$\horseshoe{\universal{\variable{x}}\universal{\variable{y}}\Hpp{\variable{x}}{\variable{y}}}{\universal{\variable{y}}\universal{\variable{x}}\Hpp{\variable{x}}{\variable{y}}}$}
\item {$\horseshoe{\existential{\variable{x}}\existential{\variable{y}}\Hpp{\variable{x}}{\variable{y}}}{\existential{\variable{y}}\existential{\variable{x}}\Hpp{\variable{x}}{\variable{y}}}$}
\item {$\horseshoe{\universal{\variable{x}}\existential{\variable{y}}\Hpp{\variable{x}}{\variable{y}}}{\existential{\variable{y}}\universal{\variable{x}}\Hpp{\variable{x}}{\variable{y}}}$}
\item {$\horseshoe{\existential{\variable{y}}\universal{\variable{x}}\Hpp{\variable{x}}{\variable{y}}}{\universal{\variable{x}}\existential{\variable{y}}\Hpp{\variable{x}}{\variable{y}}}$}
\item {$\universal{\variable{x}}\bparhorseshoe{\Ap{\variable{x}}}{\existential{\variable{y}}\parconjunction{\Hpp{\variable{x}}{\variable{y}}}{\Bp{\variable{y}}}}$}
\item {$\existential{\variable{y}}\bparconjunction{\Ap{\variable{y}}}{\universal{\variable{z}}\parhorseshoe{\Bp{\variable{z}}}{\Hpp{\variable{y}}{\variable{z}}}}$}
\end{enumerate}
\end{multicols}
\begin{enumerate}[start=7]
\item {$\universal{\variable{x}}\bparconjunction{\existential{\variable{y}}\Hpp{\variable{x}}{\variable{y}}}{\conjunction{\negation{\Hpp{\variable{x}}{\variable{x}}}}{\universal{\variable{y}}\universal{z}\cparhorseshoe{\parconjunction{\Hpp{\variable{x}}{\variable{y}}}{\Hpp{\variable{y}}{\variable{z}}}}{\Hpp{\variable{x}}{\variable{z}}}}}$}
\end{enumerate}

\notocsubsection{Preliminary Dragnet Practice Problems}{ex:Preliminary Dragnet Practice Problems} 
\begin{enumerate}
\item If $\CAPPHI$ is $\horseshoe{\universal{\variable{z_{\integer{3}}}}\parconjunction{\Gpp{\variable{z_{\integer{1}}}}{\variable{z_{\integer{3}}}}}{\existential{\variable{x}}\Bppp{\variable{x}}{\variable{z_{\integer{3}}}}{\variable{y_{\integer{2}}}}}}{\parhorseshoe{\Al}{\universal{\variable{y_{\integer{2}}}}\Dpp{\variable{z_{\integer{3}}}}{\variable{y_{\integer{2}}}}}}$, what is
\begin{enumerate}
\item $\CAPPHI\constant{a}/\variable{z_{\integer{3}}}$
\item $\CAPPHI\constant{a}/\variable{y_{\integer{2}}}$
\item $\CAPPHI\variable{z_{\integer{3}}}/\variable{x}$
\item $\CAPPHI\constant{a}/\variable{x}$
\end{enumerate} 
\item If $\CAPPHI\variable{x}/\variable{y}$ is $\Dpp{\variable{x}}{\variable{x}}$, can you determine what $\CAPPHI$ is? If so, what is it? If not, why not?
\end{enumerate}

\notocsubsection{Dragnet Practice}{ex:Dragnet Practice} For each of the following statements, show whether it is true or false. 
Use Dragnet or Free Choice whenever it is helpful to do so.

Each of $\CAPTHETA$, $\CAPPSI$, and $\CAPPHI$ is a formula of \GQL{} whose only free variable is $\variable{x}$, and none contain the variables $\variable{y}$, $\variable{z}$ or $\variable{w}$. 

\begin{enumerate}
\item {$\universal{\variable{w}}\parconjunction{\CAPPHI{\variable{w}/\variable{x}}}{\CAPPSI{\variable{w}/\variable{x}}}\sdtstile{}{}\horseshoe{\universal{\variable{x}}\CAPPHI}{\universal{\variable{x}}\CAPPSI}$}
\item {$\horseshoe{\universal{\variable{x}}\CAPPHI}{\universal{\variable{x}}\CAPPSI}\sdtstile{}{}\horseshoe{\universal{\variable{y}}\CAPPHI{\variable{y}/\variable{x}}}{\universal{\variable{z}}\CAPPSI{\variable{z}/\variable{x}}}$}
\item {$\horseshoe{\universal{\variable{x}}\CAPPHI}{\universal{\variable{x}}\CAPPSI}\sdtstile{}{}\existential{\variable{x}}\parconjunction{\CAPPHI}{\CAPPSI}$}
\item {$\universal{\variable{x}}\parhorseshoe{\CAPPHI}{\CAPPSI}\sdtstile{}{}\horseshoe{\universal{\variable{y}}\CAPPHI{\variable{y}/\variable{x}}}{\universal{\variable{z}}\CAPPSI{\variable{z}/\variable{x}}}$}
\item {$\horseshoe{\universal{\variable{y}}\CAPPHI{\variable{y}/\variable{x}}}{\universal{\variable{z}}\CAPPSI{\variable{z}/\variable{x}}}\sdtstile{}{}\universal{\variable{x}}\parhorseshoe{\CAPPHI}{\CAPPSI}$}
\item {$\horseshoe{\universal{\variable{y}}\CAPPHI{\variable{y}/\variable{x}}}{\universal{\variable{z}}\CAPPSI{\variable{z}/\variable{x}}}\text{, }\existential{\variable{x}}\parconjunction{\CAPPHI}{\CAPPSI}\sdtstile{}{}\universal{\variable{x}}\parhorseshoe{\CAPPHI}{\CAPPSI}$}
\item {$\text{If }\sdtstile{}{}\universal{\variable{x}}\parhorseshoe{\CAPPHI}{\CAPPSI}\text{, then }\sdtstile{}{}\universal{\variable{x}}\cparhorseshoe{\parconjunction{\CAPPHI}{\CAPTHETA}}{\parconjunction{\CAPPSI}{\CAPTHETA}}$}
\item {$\existential{\variable{y}}\parconjunction{\CAPPHI{\variable{y}/\variable{x}}}{\negation{\CAPPSI{\variable{y}/\variable{x}}}}\sdtstile{}{}\universal{\variable{y}}\parhorseshoe{\CAPPHI{\variable{y}/\variable{x}}}{\negation{\CAPPSI{\variable{y}/\variable{x}}}}$}
\item {$\negation{\universal{\variable{x}}\CAPPHI}\sdtstile{}{}\existential{\variable{x}}\negation{\CAPPHI}$}
\item {$\sdtstile{}{}\disjunction{\universal{\variable{x}}\parhorseshoe{\CAPPHI}{\CAPPSI}}{\universal{\variable{y}}\parhorseshoe{\CAPPHI{\variable{y}/\variable{x}}}{\negation{\CAPPSI{\variable{y}/\variable{x}}}}}$}
\end{enumerate}

%\theendnotes

