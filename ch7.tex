
%%%%%%%%%%%%%%%%%%%%%%%%%%%%%%%%%%%%%%%%%%%%%%%%%%
\chapter{Quantificational Derivations}\label{Quantificational Derivations}
%%%%%%%%%%%%%%%%%%%%%%%%%%%%%%%%%%%%%%%%%%%%%%%%%%
%\AddToShipoutPicture*{\BackgroundPicC}

%%%%%%%%%%%%%%%%%%%%%%%%%%%%%%%%%%%%%%%%%%%%%%%%%%
\section{\GQD{}}\label{Section GQD}
%%%%%%%%%%%%%%%%%%%%%%%%%%%%%%%%%%%%%%%%%%%%%%%%%%

Now our goal is to extend \GSD{} to a natural deduction system for \GQL{}.
This system, which we call \idf{Quantificational Derivation System}, or \GQD{}, consists of all the rules of \GSD{}, plus an introduction and elimination rule for each quantifier (given in table \ref{GQD}).

\subsection{Introduction and Elimination Rules}

Like \GSD{}, we extend \GQD{} by adding shortcut rules. 
First, we can extend \GQD{} by adding all the shortcut rules we gave for \GSD{}. 
Second, we can extend \GQD{} by adding shortcut rules specifically for the quantifiers (see table \ref{GQDplus}).
The system with all the shortcut rules from \GSD{} (tables \ref{GSDplus1} and \ref{GSDplus1}) and the new shortcut rules for the quantifiers (table \ref{GQDplus}) is called \GQDP{}.
Note that some of the rules for \GQD{} have special restrictions.
We explain these in the examples below. 
%Since there are only two new connectives, introduce rules, strategies and do examples all at once.

%\begin{table}[!ht]
%\renewcommand{\arraystretch}{1.5}
%\begin{center}
%\begin{tabular}{ p{1in} l l } %p{2.2in} p{2in}
%\toprule
%\textbf{Name} & \textbf{Given} & \textbf{May Add} \\ 
%\midrule 
\renewcommand{\arraystretch}{1.5}
\begin{longtable}[c]{ p{1in} l l } %p{2.2in} p{2in}
\toprule
\textbf{Name} & \textbf{Given} & \textbf{May Add} \\ 
\midrule
\endfirsthead
\multicolumn{3}{c}{\emph{Continued from Previous Page}}\\
\toprule
\textbf{Name} & \textbf{Given} & \textbf{May Add} \\ 
\midrule
\endhead
\bottomrule
\caption{(New) Basic Rules for \GQD{}}\\[-.15in]
\multicolumn{3}{c}{\emph{Continued next Page}}\\
\endfoot
\bottomrule
\caption{(New) Basic Rules for \GQD{}}\\
\endlastfoot
\label{GQD}\Rule{$\forall$-Elim} & $\universal{\BETA}\CAPPHI$ & $\CAPPHI\constant{a}/\BETA$, for \mention{a} any  \\[-.25cm]
\nopagebreak
 &   &   individual constant \\
\Rule{$\forall$-Intro} & $\CAPPHI\constant{a}/\BETA$ & $\universal{\BETA}\CAPPHI$, iff \mention{a} does  \\[-.25cm]
\nopagebreak
 &  &  not occur in $\CAPPHI$  \\[-.25cm]
 \nopagebreak
 &  & nor in any unboxed line \\[-.25cm]
 \nopagebreak
& &  justified by Assumption\\
\Rule{$\exists$-Intro} & $\CAPPHI\constant{a}/\BETA$ & $\existential{\BETA}\CAPPHI$ \\
\Rule{$\exists$-Elim} & $\existential{\BETA}\CAPPHI$, $\horseshoe{\CAPPHI{\constant{a}/\BETA}}{\CAPTHETA}$ & $\CAPTHETA$, \Iff \mention{a} does \\[-.25cm]
\nopagebreak
 &  &  not occur in $\CAPPHI$ or $\CAPTHETA$, \\[-.25cm]
\nopagebreak
 & &  nor in any unboxed line\\[-.25cm]
 \nopagebreak
 & &  justified by Assumption\\
\end{longtable}
%\bottomrule
%\end{tabular}
%\end{center}
%\caption{(New) Basic Rules for \GQD{}}
%\label{GQD}
%\end{table}

The mechanics of writing proofs in \GQD{} are no different than the mechanics of writing proofs in \GSD{}, but we do have to slightly adapt the definition for sanctioning (Def. \pmvref{RuleSanctioning}).
\begin{majorILnc}{\LnpDC{GQDRuleSanctioning}}
A rule \Rule{R}, applied to unboxed lines $\integer{m}_1,\ldots,\integer{m}_{\integer{j}}$ with, respectively, sentences $\CAPPSI_1,\ldots,\CAPPSI_{\integer{j}}$, \df{sanctions} writing the sentence $\CAPPHI$ \Iff there's some substitution of \GQL{} \emph{formulas} that, for the given schemas of \Rule{R}, results in $\CAPPSI_1,\ldots,\CAPPSI_{\integer{j}}$ and, for the may-add schema, results in $\CAPPHI$. 
\end{majorILnc}
Another difference is that there are four new rules available. 
We now look at four examples which highlight each rule.

Our first example demonstrates \Rule{$\forall$-Elim}. 
Say we want to show that $\universal{\variable{x}}\parhorseshoe{\Ap{\variable{x}}}{\Bp{\variable{x}}},\Ap{\constant{d}}\sststile{}{}\Bp{\constant{d}}$.
We start by setting the two sentences on the \CAPS{lhs} of the turnstile as assumptions:
\begin{gproof}[\label{GQDExampleA}]
\galineNC{1}{$\universal{\variable{x}}\parhorseshoe{\Ap{\variable{x}}}{\Bp{\variable{x}}}$}{\Rule{Assume}}
\gaalineNC{2}{$\Ap{\constant{d}}$}{\Rule{Assume}}
\end{gproof}
Next we use \Rule{$\forall$-Elim} on line 1. 
Note that there are no restrictions on the use of \Rule{$\forall$-Elim}.
\begin{gproof}[\label{GQDExampleB}]
\galineNC{1}{$\universal{\variable{x}}\parhorseshoe{\Ap{\variable{x}}}{\Bp{\variable{x}}}$}{\Rule{Assume}}
\gaalineNC{2}{$\Ap{\constant{d}}$}{\Rule{Assume}}
\gaalineNC{3}{$\horseshoe{\Ap{\constant{d}}}{\Bp{\constant{d}}}$}{\Rule{$\forall$-Elim}, 1}
\end{gproof}
Although we could have substituted any constant for $\variable{x}$ in line 3, we chose $\constant{d}$ so that we can next apply \Rule{$\HORSESHOE$-Elim}. 
\begin{gproof}[\label{GQDExampleC}]
\galineNC{1}{$\universal{\variable{x}}\parhorseshoe{\Ap{\variable{x}}}{\Bp{\variable{x}}}$}{\Rule{Assume}}
\gaalineNC{2}{$\Ap{\constant{d}}$}{\Rule{Assume}}
\gaalineNC{3}{$\horseshoe{\Ap{\constant{d}}}{\Bp{\constant{d}}}$}{\Rule{$\forall$-Elim}, 1}
\gaalineNC{4}{$\Bp{\constant{d}}$}{\Rule{$\HORSESHOE$-Elim}, 2,3}
\end{gproof}
And this completes the derivation. 

The next example demonstrates \Rule{$\exists$-Intro}. Here we show that $\universal{\variable{y}}\bparhorseshoe{\existential{\variable{x}}\Dpp{\variable{x}}{\variable{y}}}{\Bp{\variable{y}}},\Dpp{\constant{a}}{\constant{b}}\sststile{}{}\Bp{\constant{b}}$.
As before, we start with the assumptions.
\begin{gproof}[\label{GQDExampleD}]
\galineNC{1}{$\universal{\variable{y}}\bparhorseshoe{\existential{\variable{x}}\Dpp{\variable{x}}{\variable{y}}}{\Bp{\variable{y}}}$}{\Rule{Assume}}
\gaalineNC{2}{$\Dpp{\constant{a}}{\constant{b}}$}{\Rule{Assume}}
\end{gproof}
Again as before, we need to use \Rule{$\forall$-Elim} so we can eventually use \Rule{$\HORSESHOE$-Elim} to finish. 
\begin{gproof}[\label{GQDExampleE}]
\galineNC{1}{$\universal{\variable{y}}\bparhorseshoe{\existential{\variable{x}}\Dpp{\variable{x}}{\variable{y}}}{\Bp{\variable{y}}}$}{\Rule{Assume}}
\gaalineNC{2}{$\Dpp{\constant{a}}{\constant{b}}$}{\Rule{Assume}}
\gaalineNC{3}{$\horseshoe{\existential{\variable{x}}\Dpp{\variable{x}}{\constant{b}}}{\Bp{\variable{\constant{b}}}}$}{\Rule{$\forall$-Elim}, 1}
\end{gproof}
Again we strategically chose the constant we did, $\constant{b}$, so that using \Rule{$\HORSESHOE$-Elim} will get us the right result. 
But we can't apply \Rule{$\HORSESHOE$-Elim} yet, since the \CAPS{lhs} of the horseshoe in line 3, $\existential{\variable{x}}\Dpp{\variable{x}}{\constant{b}}$, is an existential sentence, while what we have on line 2, $\Dpp{\constant{a}}{\constant{b}}$, is not. 
But, by using \Rule{$\exists$-Intro} we can easily get what we need:
\begin{gproof}[\label{GQDExampleF}]
\galineNC{1}{$\universal{\variable{y}}\bparhorseshoe{\existential{\variable{x}}\Dpp{\variable{x}}{\variable{y}}}{\Bp{\variable{y}}}$}{\Rule{Assume}}
\gaalineNC{2}{$\Dpp{\constant{a}}{\constant{b}}$}{\Rule{Assume}}
\gaalineNC{3}{$\horseshoe{\existential{\variable{x}}\Dpp{\variable{x}}{\constant{b}}}{\Bp{\variable{\constant{b}}}}$}{\Rule{$\forall$-Elim}, 1}
\gaalineNC{4}{$\existential{\variable{x}}\Dpp{\variable{x}}{\constant{b}}$}{\Rule{$\exists$-Intro}, 2}
\end{gproof}
Note that just as with \Rule{$\forall$-Elim} there are no restrictions on \Rule{$\HORSESHOE$-Elim}.
Also note that we could have used \Rule{$\HORSESHOE$-Elim} to generalize the constant $\constant{b}$, getting $\existential{\variable{x}}\Dpp{\constant{a}}{\variable{x}}$, but that wouldn't have helped us.
Also, we could have generalized using a different variable, getting, say $\existential{\variable{z}}\Dpp{\variable{z}}{\constant{b}}$.
But again that wouldn't have helped us. 
Any constant is legal to instantiate with \Rule{$\forall$-Elim}, but often only one is a wise choice.

We now have the right setup for \Rule{$\HORSESHOE$-Elim}:
\begin{gproof}[\label{GQDExampleG}]
\galineNC{1}{$\universal{\variable{y}}\bparhorseshoe{\existential{\variable{x}}\Dpp{\variable{x}}{\variable{y}}}{\Bp{\variable{y}}}$}{\Rule{Assume}}
\gaalineNC{2}{$\Dpp{\constant{a}}{\constant{b}}$}{\Rule{Assume}}
\gaalineNC{3}{$\horseshoe{\existential{\variable{x}}\Dpp{\variable{x}}{\constant{b}}}{\Bp{\variable{\constant{b}}}}$}{\Rule{$\forall$-Elim}, 1}
\gaalineNC{4}{$\existential{\variable{x}}\Dpp{\variable{x}}{\constant{b}}$}{\Rule{$\exists$-Intro}, 2}
\gaalineNC{5}{$\Bp{\constant{b}}$}{\Rule{$\HORSESHOE$-Elim}, 3,4}
\end{gproof}
And this completes the derivation. 

The next example demonstrates \Rule{$\exists$-Elim}. 
This is our first rule with restrictions.
We show that $\universal{\variable{x}}\negation{\Qpp{\variable{x}}{\constant{b}}},\existential{\variable{z}}\pardisjunction{\Qpp{\variable{z}}{\constant{b}}}{\Gp{\variable{z}}}\sststile{}{}\existential{\variable{x}}\Gp{\variable{x}}$.
As before, we start by setting out the assumptions. 
\begin{gproof}[\label{GQDExampleH}]
\galineNC{1}{$\universal{\variable{x}}\negation{\Qpp{\variable{x}}{\constant{b}}}$}{\Rule{Assume}}
\gaalineNC{2}{$\existential{\variable{z}}\pardisjunction{\Qpp{\variable{z}}{\constant{b}}}{\Gp{\variable{z}}}$}{\Rule{Assume}}
\end{gproof}

When using both \Rule{$\exists$-Elim} and \Rule{$\forall$-Elim}, it’s generally best to take the instance of the existential for \Rule{$\exists$-Elim} first and use \Rule{$\forall$-Elim} after. In some cases
you want the \Rule{$\forall$-Elim} instance to match the \Rule{$\exists$-Elim} constant, but in other cases you want it to be different.
Now, according to \Rule{$\exists$-Elim}, we can get a sentence $\CAPTHETA$ in this case by showing that $\horseshoe{\pardisjunction{\Qpp{\variable{t}}{\constant{b}}}{\Gp{\variable{t}}}}{\CAPTHETA}$, for some constant $\variable{t}$ that fits the restriction on \Rule{$\exists$-Elim}.
(Just what constants will work and what sentence $\CAPTHETA$ we want will take some thought.)
So, we need to use \Rule{$\HORSESHOE$-Intro}. 
\begin{gproof}[\label{GQDExampleI}]
\galineNC{1}{$\universal{\variable{x}}\negation{\Qpp{\variable{x}}{\constant{b}}}$}{\Rule{Assume}}
\gaalineNC{2}{$\existential{\variable{z}}\pardisjunction{\Qpp{\variable{z}}{\constant{b}}}{\Gp{\variable{z}}}$}{\Rule{Assume}}
\gaaaproof{
\gaaalineSS{3}{$\disjunction{\Qpp{\constant{a}}{\constant{b}}}{\Gp{\constant{a}}}$}{\Rule{Assume}}
\gaaalinendSS{}{}{}
\gaaalinendSS{}{$\qquad\vdots$}{}
\gaaalinendSS{}{}{}
\gaaalineSS{$\integer{n}$}{$\CAPTHETA$}{}
}
\gaalineNC{$\integer{n}+1$}{$\horseshoe{\pardisjunction{\Qpp{\constant{a}}{\constant{b}}}{\Gp{\constant{a}}}}{\CAPTHETA}$}{\Rule{$\HORSESHOE$-Intro}, 3--$\integer{n}$}
\gaalineNC{$\integer{n}+2$}{$\CAPTHETA$}{\Rule{$\exists$-Elim}, 2,$\integer{n}+1$}
\end{gproof}
Here we have chosen $\constant{a}$ to substitute for $\variable{z}$ in line 3 because (so long as we pick $\CAPTHETA$ correctly) it will meet our restriction. 
According to the restriction on \Rule{$\exists$-Elim}, the constant we pick can't appear in any open assumptions (the assumption used to derive the needed conditional, in this case line 3, is not open by the time we apply the rule). 
It also can't appear in the scope of the existential quantifier, which in this case is $\pardisjunction{\Qpp{\variable{z}}{\constant{b}}}{\Gp{\variable{z}}}$. 
Since $\constant{a}$ does not appear in any open assumptions and does not appear in $\pardisjunction{\Qpp{\variable{z}}{\constant{b}}}{\Gp{\variable{z}}}$, it will meet our restriction.
(Clearly other constants would have also met the restriction.)
Now we have to decide what sentence $\CAPTHETA$ enables us to finish the derivation. 
Note that if we derive $\Gp{\variable{t}}$ for any constant $\variable{t}\neq\constant{a}$, then we can use \Rule{$\exists$-Intro} to get the needed sentence $\existential{\variable{x}}\Gp{\variable{x}}$. 
(We will set this up, but we see in a moment that this first guess won't work.)
We can't pick $\variable{t}=\constant{a}$ because then we couldn't apply \Rule{$\exists$-Elim} on line $\integer{n}+2$.
\begin{gproof}[\label{GQDExampleJ}]
\galineNC{1}{$\universal{\variable{x}}\negation{\Qpp{\variable{x}}{\constant{b}}}$}{\Rule{Assume}}
\gaalineNC{2}{$\existential{\variable{z}}\pardisjunction{\Qpp{\variable{z}}{\constant{b}}}{\Gp{\variable{z}}}$}{\Rule{Assume}}
\gaaaproof{
\gaaalineSS{3}{$\disjunction{\Qpp{\constant{a}}{\constant{b}}}{\Gp{\constant{a}}}$}{\Rule{Assume}}
\gaaalinendSS{}{}{}
\gaaalinendSS{}{$\qquad\vdots$}{}
\gaaalinendSS{}{}{}
\gaaalineSS{$\integer{n}$}{$\Gp{\constant{c}}$}{}
}
\gaalineNC{$\integer{n}+1$}{$\horseshoe{\pardisjunction{\Qpp{\constant{a}}{\constant{b}}}{\Gp{\constant{a}}}}{\Gp{\constant{c}}}$}{\Rule{$\HORSESHOE$-Intro}, 3--$\integer{n}$}
\gaalineNC{$\integer{n}+2$}{$\Gp{\constant{c}}$}{\Rule{$\exists$-Elim}, 2,$\integer{n}+1$}
\gaalineNC{$\integer{n}+3$}{$\existential{\variable{x}}\Gp{\variable{x}}$}{\Rule{$\exists$-Intro}, $\integer{n}+2$}
\end{gproof}
Now we only need to complete the derivation by deriving $\Gp{\constant{c}}$ from $\pardisjunction{\Qpp{\constant{a}}{\constant{b}}}{\Gp{\constant{a}}}$. 
We can \emph{try} do this by using \Rule{$\forall$-Elim} and \Rule{D.S.}, but it becomes clear at once that this won't work.
\begin{gproof}[\label{GQDExampleK}]
\galineNC{1}{$\universal{\variable{x}}\negation{\Qpp{\variable{x}}{\constant{b}}}$}{\Rule{Assume}}
\gaalineNC{2}{$\existential{\variable{z}}\pardisjunction{\Qpp{\variable{z}}{\constant{b}}}{\Gp{\variable{z}}}$}{\Rule{Assume}}
\gaaaproof{
\gaaalineSS{3}{$\disjunction{\Qpp{\constant{a}}{\constant{b}}}{\Gp{\constant{a}}}$}{\Rule{Assume}}
\gaaalineSS{4}{$\negation{\Qpp{\constant{a}}{\constant{b}}}$}{\Rule{$\forall$-Elim}, 1}
\gaaalinendSS{5}{$\Gp{\constant{a}}$}{\Rule{D.S.}, 3,4}
\gaaalinendSS{}{}{}
\gaaalinendSS{}{$\qquad\vdots$}{}
\gaaalinendSS{}{}{}
\gaaalineSS{$\integer{n}$}{$\Gp{\constant{c}}$}{}
}
\gaalineNC{$\integer{n}+1$}{$\horseshoe{\pardisjunction{\Qpp{\constant{a}}{\constant{b}}}{\Gp{\constant{a}}}}{\Gp{\constant{c}}}$}{\Rule{$\HORSESHOE$-Intro}, 3--$\integer{n}$}
\gaalineNC{$\integer{n}+2$}{$\Gp{\constant{c}}$}{\Rule{$\exists$-Elim}, 2,$\integer{n}+1$}
\gaalineNC{$\integer{n}+3$}{$\existential{\variable{x}}\Gp{\variable{x}}$}{\Rule{$\exists$-Intro}, $\integer{n}+2$}
\end{gproof}
It should be clear that this \Rule{$\forall$-Elim}/\Rule{D.S.} strategy won't work, since for it to work we'd need the token of $\GG$ that appears in line 3 to be followed by the \emph{same} constant that follows the token of $\GG$ that appears in line $\integer{n}$. 
But that constant is the constant we substitute in for \Rule{$\exists$-Elim}, and we would then violate the restriction for the rule. 

So we need to go back to $\CAPTHETA$ and consider another strategy. 
This time we try letting $\CAPTHETA=\existential{\variable{x}}\Gp{\variable{x}}$.
\begin{gproof}[\label{GQDExampleL}]
\galineNC{1}{$\universal{\variable{x}}\negation{\Qpp{\variable{x}}{\constant{b}}}$}{\Rule{Assume}}
\gaalineNC{2}{$\existential{\variable{z}}\pardisjunction{\Qpp{\variable{z}}{\constant{b}}}{\Gp{\variable{z}}}$}{\Rule{Assume}}
\gaaaproof{
\gaaalineSS{3}{$\disjunction{\Qpp{\constant{a}}{\constant{b}}}{\Gp{\constant{a}}}$}{\Rule{Assume}}
\gaaalinendSS{}{}{}
\gaaalinendSS{}{$\qquad\vdots$}{}
\gaaalinendSS{}{}{}
\gaaalineSS{$\integer{n}$}{$\existential{\variable{x}}\Gp{\variable{x}}$}{}
}
\gaalineNC{$\integer{n}+1$}{$\horseshoe{\pardisjunction{\Qpp{\constant{a}}{\constant{b}}}{\Gp{\constant{a}}}}{\existential{\variable{x}}\Gp{\variable{x}}}$}{\Rule{$\HORSESHOE$-Intro}, 3--$\integer{n}$}
\gaalineNC{$\integer{n}+2$}{$\existential{\variable{x}}\Gp{\variable{x}}$}{\Rule{$\exists$-Elim}, 2,$\integer{n}+1$}
\end{gproof}
Note that we still keep our choice of $\constant{a}$ on line 3 within the restrictions of \Rule{$\exists$-Elim}. 
Now we can try again at finishing the proof. 
This time we can:
\begin{gproof}[\label{GQDExampleM}]
\galineNC{1}{$\universal{\variable{x}}\negation{\Qpp{\variable{x}}{\constant{b}}}$}{\Rule{Assume}}
\gaalineNC{2}{$\existential{\variable{z}}\pardisjunction{\Qpp{\variable{z}}{\constant{b}}}{\Gp{\variable{z}}}$}{\Rule{Assume}}
\gaaaproof{
\gaaalineSS{3}{$\disjunction{\Qpp{\constant{a}}{\constant{b}}}{\Gp{\constant{a}}}$}{\Rule{Assume}}
\gaaalineSS{4}{$\negation{\Qpp{\constant{a}}{\constant{b}}}$}{\Rule{$\forall$-Elim}, 1}
\gaaalineSS{5}{$\Gp{\constant{a}}$}{\Rule{D.S.}, 3,4}
\gaaalineSS{6}{$\existential{\variable{x}}\Gp{\variable{x}}$}{\Rule{$\exists$-Intro}, 5}
}
\gaalineNC{7}{$\horseshoe{\pardisjunction{\Qpp{\constant{a}}{\constant{b}}}{\Gp{\constant{a}}}}{\existential{\variable{x}}\Gp{\variable{x}}}$}{\Rule{$\HORSESHOE$-Intro}, 3--6}
\gaalineNC{8}{$\existential{\variable{x}}\Gp{\variable{x}}$}{\Rule{$\exists$-Elim}, 2,7}
\end{gproof}
And this completes the derivation. 

The final example demonstrates \Rule{$\forall$-Intro}. 
This rule also has restrictions. 
We show that $\universal{\variable{x}}\parhorseshoe{\Ap{\variable{x}}}{\Bp{\variable{x}}},\universal{\variable{x}}\parhorseshoe{\Bp{\variable{x}}}{\Hp{\variable{x}}}\sststile{}{}\universal{\variable{x}}\parhorseshoe{\Ap{\variable{x}}}{\Hp{\variable{x}}}$.
It's worth mentioning that the fact that all the examples so far have two sentences on the \CAPS{lhs} of the turnstile is just a coincidence.
As before, we start by setting out the assumptions.
\begin{gproof}[\label{GQDExampleN}]
\galineNC{1}{$\universal{\variable{x}}\parhorseshoe{\Ap{\variable{x}}}{\Bp{\variable{x}}}$}{\Rule{Assume}}
\gaalineNC{2}{$\universal{\variable{x}}\parhorseshoe{\Bp{\variable{x}}}{\Hp{\variable{x}}}$}{\Rule{Assume}}
\end{gproof}
Before moving forward, we need to think about how \Rule{$\forall$-Intro} works. 
According to the rule, if we want to get $\universal{\variable{x}}\parhorseshoe{\Ap{\variable{x}}}{\Hp{\variable{x}}}$ by using \Rule{$\forall$-Intro}, we need to first get $\horseshoe{\Ap{\variable{t}}}{\Hp{\variable{t}}}$ on some line, where $\variable{t}$ is some constant that does not appear in any unboxed assumptions, or in $\horseshoe{\Ap{\variable{x}}}{\Hp{\variable{x}}}$.
Since no constants appear in $\horseshoe{\Ap{\variable{x}}}{\Hp{\variable{x}}}$, it provides no constraints. 
Further, there are no constants in either unboxed assumption in derivation \ref{GQDExampleN}, so we are free to chose any constant we like. 
Pick $\constant{a}$. 
So we want to derive $\horseshoe{\Ap{\constant{a}}}{\Hp{\constant{a}}}$.
\begin{gproof}[\label{GQDExampleO}]
\galineNC{1}{$\universal{\variable{x}}\parhorseshoe{\Ap{\variable{x}}}{\Bp{\variable{x}}}$}{\Rule{Assume}}
\gaalineNC{2}{$\universal{\variable{x}}\parhorseshoe{\Bp{\variable{x}}}{\Hp{\variable{x}}}$}{\Rule{Assume}}
\gaaaproof{
\gaaalineSS{3}{$\Ap{\constant{a}}$}{\Rule{Assume}}
\gaaalinendSS{}{}{}
\gaaalinendSS{}{$\qquad\vdots$}{}
\gaaalinendSS{}{}{}
\gaaalineSS{$\integer{n}$}{$\Hp{\constant{a}}$}{}
}
\gaalineNC{$\integer{n}+1$}{$\horseshoe{\Ap{\constant{a}}}{\Hp{\constant{a}}}$}{\Rule{$\HORSESHOE$-Intro}, 3--$\integer{n}$}
\gaalineNC{$\integer{n}+2$}{$\universal{\variable{x}}\parhorseshoe{\Ap{\variable{x}}}{\Hp{\variable{x}}}$}{\Rule{$\forall$-Intro}, $\integer{n}+1$}
\end{gproof}
Now we just have to finish the derivation of $\Hp{\constant{a}}$ from $\Ap{\constant{a}}$ without introducing any new unboxed assumptions with constant $\constant{a}$. 
(Of course, if we did the bottom half of this proof wouldn't work, since we coudn't close the assumption on line 3 with \Rule{$\HORSESHOE$-Intro} if unboxed assumptions appeared after line 3.)
A natural next step is to use \Rule{$\forall$-Elim} on lines 1 and 2.
\begin{gproof}[\label{GQDExampleP}]
\galineNC{1}{$\universal{\variable{x}}\parhorseshoe{\Ap{\variable{x}}}{\Bp{\variable{x}}}$}{\Rule{Assume}}
\gaalineNC{2}{$\universal{\variable{x}}\parhorseshoe{\Bp{\variable{x}}}{\Hp{\variable{x}}}$}{\Rule{Assume}}
\gaaaproof{
\gaaalineSS{3}{$\Ap{\constant{a}}$}{\Rule{Assume}}
\gaaalineSS{4}{$\horseshoe{\Ap{\constant{a}}}{\Bp{\constant{a}}}$}{\Rule{$\forall$-Elim}, 1}
\gaaalineSS{5}{$\horseshoe{\Bp{\constant{a}}}{\Hp{\constant{a}}}$}{\Rule{$\forall$-Elim}, 2}
\gaaalinendSS{}{}{}
\gaaalinendSS{}{$\qquad\vdots$}{}
\gaaalinendSS{}{}{}
\gaaalineSS{$\integer{n}$}{$\Hp{\constant{a}}$}{}
}
\gaalineNC{$\integer{n}+1$}{$\horseshoe{\Ap{\constant{a}}}{\Hp{\constant{a}}}$}{\Rule{$\HORSESHOE$-Intro}, 3--$\integer{n}$}
\gaalineNC{$\integer{n}+2$}{$\universal{\variable{x}}\parhorseshoe{\Ap{\variable{x}}}{\Hp{\variable{x}}}$}{\Rule{$\forall$-Intro}, $\integer{n}+1$}
\end{gproof}
Now getting to $\Hp{\constant{a}}$ is just a matter of using \Rule{$\HORSESHOE$-Elim}:
\begin{gproof}[\label{GQDExampleQ}]
\galineNC{1}{$\universal{\variable{x}}\parhorseshoe{\Ap{\variable{x}}}{\Bp{\variable{x}}}$}{\Rule{Assume}}
\gaalineNC{2}{$\universal{\variable{x}}\parhorseshoe{\Bp{\variable{x}}}{\Hp{\variable{x}}}$}{\Rule{Assume}}
\gaaaproof{
\gaaalineSS{3}{$\Ap{\constant{a}}$}{\Rule{Assume}}
\gaaalineSS{4}{$\horseshoe{\Ap{\constant{a}}}{\Bp{\constant{a}}}$}{\Rule{$\forall$-Elim}, 1}
\gaaalineSS{5}{$\horseshoe{\Bp{\constant{a}}}{\Hp{\constant{a}}}$}{\Rule{$\forall$-Elim}, 2}
\gaaalineSS{6}{$\Bp{\constant{a}}$}{\Rule{$\HORSESHOE$-Elim}, 3,4}
\gaaalineSS{7}{$\Hp{\constant{a}}$}{\Rule{$\HORSESHOE$-Elim}, 5,6}
}
\gaalineNC{8}{$\horseshoe{\Ap{\constant{a}}}{\Hp{\constant{a}}}$}{\Rule{$\HORSESHOE$-Intro}, 3--7}
\gaalineNC{9}{$\universal{\variable{x}}\parhorseshoe{\Ap{\variable{x}}}{\Hp{\variable{x}}}$}{\Rule{$\forall$-Intro}, 8}
\end{gproof}
And this completes the derivation. Notice that since there are infinitely many constants in \GQL{}, and any of them would have worked in the derivation, there are infinitely many derivations of this
sentence.

\subsection{Decidability}\label{Section:Intro to Decidability}
There are multiple algorithms to follow for applying the rules which, if there does exist a derivation, will halt when the last line written down is the sentence to be derived. 
For \GSD{} the algorithms are intuitive and straightforward (see Sec. \pmvref{Section:Completeness for GSD}), while for \GQD{} (the basic derivation system for \GQL{} we define in Sec. \pmvref{Section GQD}) they are much more complicated.

But although these algorithms are guaranteed to end in a derivation if the sentence can be derived, there are at least two reasons why you don't want to do most of your derivations using them. 
First, the derivations produced by them tend to be much longer and more complicated than is necessary. 
You will almost always be able to come up with a much shorter and more direct proof on your own. 
Second, at least for \GQD{} (but not \GSD{}), although \emph{if} there is a derivation the algorithms will ``find it'', if there is \emph{not} a derivation then the algorithms may never ``find out''. 
That is, if there is not a derivation then the algorithms (any one you pick) do essentially one of two things: either they halt in a way that indicates there is no derivation, or they never halt. 
If you happen to be working on a problem in the latter case and you're only following the algorithm, then you'll never find out whether there is a derivation. 
If a derivation system, like \GQD{}, has this feature, then it's said to be \idf{undecidable}. 
If there is an algorithm that always halts either in a derivation or with an indication that there's no derivation (as in the case of \GSD{}), then the system is said to be \idf{decidable} and the algorithm is said to be a \niidf{decision procedure}\index{decision procedure}.
\GQD{} is undecidable, while \GSD{} and \GQD{}1 are decidable.\footnote{Enderton \citeyearpar{Enderton2010} provides a general, contemporary introduction to computability and decision procedures. Kleene \citeyearpar[ch.~5]{Kleene1967} provides a lucid and concise discussion within roughly the framework devolved here.}
We will prove \GQD{}1 decidability in section \pmvref{Decidability and Churchs Theorem}.

\subsection{Shortcut Rules for \GQD{}}

All of the shortcut rules for \GSD{}, both the standard and exchange rules, can be carried over as shortcut rules for \GQD{}. 
In addition, there are four new shortcut rules for \GQD{}, the quantifier negation rules. 
These are found in table \ref{GQDplus}.
%\begin{table}[!ht]
%\renewcommand{\arraystretch}{1.5}
%\begin{center}
%\begin{tabular}{ p{1in} l l } %p{2.2in} p{2in}
%\toprule
%\textbf{Name} & \textbf{Given} & \textbf{May Add} \\ 
%\midrule
\renewcommand{\arraystretch}{1.5}
\begin{longtable}[c]{ p{1in} l l } %p{2.2in} p{2in}
\toprule
\textbf{Name} & \textbf{Given} & \textbf{May Add} \\ 
\midrule
\endfirsthead
\multicolumn{3}{c}{\emph{Continued from Previous Page}}\\
\toprule
\textbf{Name} & \textbf{Given} & \textbf{May Add} \\ 
\midrule
\endhead
\bottomrule
\caption{Exchange Short-Cut Rules for \GQD{}}\\[-.15in]
\multicolumn{3}{c}{\emph{Continued next Page}}\\
\endfoot
\bottomrule
\caption{Exchange Short-Cut Rules for \GQD{}}\\
\endlastfoot
\label{GQDplus}\Rule{QN} & $\negation{\universal{\BETA}{\CAPPHI}}$ & $\existential{\BETA}\negation{{\CAPPHI}}$ \\
 & $\existential{\BETA}\negation{{\CAPPHI}}$ & $\negation{\universal{\BETA}{\CAPPHI}}$  \\
 & $\negation{\existential{\BETA}{\CAPPHI}}$ & $\universal{\BETA}\negation{{\CAPPHI}}$ \\
 &  $\universal{\BETA}\negation{{\CAPPHI}}$ & $\negation{\existential{\BETA}{\CAPPHI}}$ \\
\end{longtable}
%\bottomrule
%\end{tabular}
%\end{center}
%\caption{Exchange Short-Cut Rules for \GQD{} (\GQDP{})}
%\label{GQDplus}
%\end{table}

Just as we had to slightly modify the definition of sanctioning used in \GSD{} for the basic (intro and elimination) rules and standard shortcut rules for \GQD{}, we also have to slightly modify the definition of sanctioning used in \GSDP{} (Def. \pmvref{ExchangeRuleSanctioning}) for the exchange shortcut rules for \GQD{} that make up \GQDP{}.
\begin{majorILnc}{\LnpDC{GQDExchangeRuleSanctioning}}
An exchange shortcut rule \Rule{R} of \GQDP{} (a rule from table \pmvref{GSDplus2}, or table \pmvref{GQDplus}), applied to a line with a \GQL{} formula $\CAPPSI$, \emph{sanctions} writing down sentence $\CAPPSI^*$ \Iff
\begin{cenumerate}
\item there is some substitution of \GQL{} formulas that, for the given schema of \Rule{R}, results in a formula $\CAPPHI$ and, for the may-add schema, results in a formula $\CAPPHI^*$,
\item $\CAPPHI$ is a subformula of $\CAPPSI$, and
\item $\CAPPSI^*$ is the \GSL{} sentence you get when you replace one instance (token) of $\CAPPHI$ with an instance (token) of $\CAPPHI^*$ in $\CAPPSI$. 
\end{cenumerate}
\end{majorILnc}

\subsection{Shortcut Rule Elimination Theorem for \GQD{}}\label{Shortcut Rule Elimination Theorem Section}

% we want to show that theorem \ref{GSD Shortcut Theorem3} still holds; or rather, that: if you can derive it in GQD+, then you can derive it in GSD. I think the proof of \ref{GSD Shortcut Theorem3} will carry over, so long as we make sure \ref{GSD Shortcut Theorem2} and \ref{GSD Shortcut Theorem} still hold. The proof for \ref{GSD Shortcut Theorem3} does need to be addressed again, since the derivation schemas for the shortcut rules for GQD+ need to be careful about variable substitutions. For \ref{GSD Shortcut Theorem2}, we need to show that the shortcut rules for GQD+ are provably equivalent (make hw?) and we need to show that theorem \ref{ExchangeRuleTheorem} holds for all GQL sentences too. To do this we just need to extend the recursive proof with two new clauses for univ and ext quantifiers. I think that's all we have to do to extend \ref{GSD Shortcut Theorem3}.

In this section we want to extend the Shortcut Rule Elimination Theorem (Thm. \pmvref{GSD Shortcut Theorem3}) to \GQD{}.
\begin{THEOREM}{\LnpTC{GQD Shortcut Theorem3} Shortcut Rule Elimination Theorem for \GQDP{}:}
For all \GQL{} sentences $\CAPPHI_1,\ldots,\CAPPHI_{\integer{m}}$ and $\CAPPSI$, if $\CAPPSI$ can be derived from $\CAPPHI_1,\ldots,\CAPPHI_{\integer{m}}$ in \GQDP{}, then $\CAPPSI$ can be derived from $\CAPPHI_1,\ldots,\CAPPHI_{\integer{m}}$ in \GQD{}.
\end{THEOREM}
\noindent{}The same proof we used for Shortcut Rule Elimination Theorem for \GSD{} (Thm. \ref{GSD Shortcut Theorem3}) will work for this version, so long as appropriate versions of theorems \mvref{GSD Shortcut Theorem} and \mvref{GSD Shortcut Theorem2} hold for the shortcut rules of \GQD{}.
Specifically:
\begin{THEOREM}{\LnpTC{GQD Shortcut Theorem}}
For all \GQL{} sentences $\CAPTHETA_1,\ldots,\CAPTHETA_{\integer{n}},\DELTA$ and rules \Rule{R$_1$}$,\ldots,$\Rule{R$_\integer{p}$}, if
\begin{cenumerate}
\item $\DELTA$ can be derived from $\CAPTHETA_1,\ldots,\CAPTHETA_{\integer{n}}$ using rules \Rule{R$_1$}$,\ldots,$\Rule{R$_\integer{p}$} and the basic rules of \GSD{}, and
\item every application of a rule \Rule{R$_1$} is derivable using the rules \Rule{R$_2$}, $\ldots$, \Rule{R$_\integer{p}$} and the basic rules of \GQD{} (recall Def. \pmvref{RuleInstanceDerivability}),
\end{cenumerate}
then $\DELTA$ can be derived from $\CAPTHETA_1,\ldots,\CAPTHETA_{\integer{n}}$ using only rules \Rule{R$_2$}$,\ldots,$\Rule{R$_\integer{p}$} and the basic rules of \GQD{}.
\end{THEOREM}
\begin{THEOREM}{\LnpTC{GQD Shortcut Theorem2}}
For all standard and exchange shortcut rules \Rule{R} (see tables \ref{GSDplus1}, \ref{GSDplus2}, and \ref{GQDplus}), every application of \Rule{R} is derivable using the basic rules of \GQD{} (see tables \ref{GSD} and \ref{GQD}).
\end{THEOREM}
\noindent{}We leave it to the reader to prove the Shortcut Rule Elimination Theorem (Thm. \ref{GQD Shortcut Theorem3}) using theorems \ref{GQD Shortcut Theorem} and \ref{GQD Shortcut Theorem2}.

Turning to the proofs for theorems \ref{GQD Shortcut Theorem} and \ref{GQD Shortcut Theorem2}, note that nothing in the proof of \ref{GSD Shortcut Theorem} depended on any special features of \GSD{}. 
Thus, the proof of theorem \ref{GSD Shortcut Theorem} can be adapted to \ref{GQD Shortcut Theorem} just by changing all the references to \GSD{} to references to \GQD{}. 
But unfortunately theorem \ref{GQD Shortcut Theorem2} is not nearly as straightforward.  

It is helpful to break theorem \ref{GQD Shortcut Theorem2} into two parts: (i) the claim that, for all standard shortcut rules (table \ref{GSDplus1}), every application is derivable using the basic rules of \GQD{}, and (ii) the claim that, for all the exchange shortcut rules (tables \ref{GSDplus2} and \ref{GQDplus}), every application is derivable using the basic rules of \GQD{}. 
Proving part (i) of theorem \ref{GQD Shortcut Theorem2} is no different from proving it for theorem \ref{GSD Shortcut Theorem2}; the same arguments using the derivation schemas written for theorem \ref{GSD Shortcut Theorem2} will work.
But nothing done so far will help with part (ii).
This is because applications of exchange shortcut rules in \GQDP{}, even those shared with \GSDP{} (see table \ref{GSDplus2}), use a different definition of sanctioning than was used in \GSDP{} (compare Def. \ref{ExchangeRuleSanctioning} and \ref{GQDExchangeRuleSanctioning}). 
According to definition \mvref{GQDExchangeRuleSanctioning}, in \GQDP{} exchange rules can be applied not only to subsentences, but also to subformulas. 
We have to show that allowing exchange rules to be applied not only to subsentences, but also to subformulas, doesn't prevent us from deriving their applications using only the basic rules.

To do this we use (1) an extended (and generalized) version of the Restricted Replacement Theorem for \GSD{} (Thm. \pmvref{ExchangeRuleTheorem}) and (2) the fact that any two formulas got by substituting \GQL{} formulas into the may-add and given schemas of the exchange shortcut rules for \GQD{} are provably equivalent. 
(We ask the reader to prove this second fact in exercise \pmvref{exer:GQDSCprovablyequiv}.)
\begin{THEOREM}{\LnpTC{GQD Replacement Theorem} The Replacement Theorem for \GQD{}:}
If $\CAPPHI$ and $\CAPPHI^*$ are provably equivalent formulas of \GQL{}, and $\CAPTHETA$ and $\CAPTHETA^*$ differ only in that $\CAPTHETA$ contains the subformula $\CAPPHI$ in one place where $\CAPTHETA^*$ contains the subformula $\CAPPHI^*$, then $\CAPTHETA$ and $\CAPTHETA^*$ are provably equivalent.
\end{THEOREM}
\noindent{}Before proving this theorem, we need to define when two \emph{formulas} of \GQL{} are provably equivalent. 
The definition given for \GSL{} sentences (def. \pmvref{GSDprovablyequivalent}) will not carry over to \GQL{} formulas, since formulas just aren't the sort of thing which can be derived. 
For example, we would like to be able to say that $\parhorseshoe{\Qp{\variable{x}}}{\Gp{\variable{y}}}$ and $\pardisjunction{\negation{\Qp{\variable{x}}}}{\Gp{\variable{y}}}$ are provably equivalent even though we cannot derive the formula $\triplebar{\parhorseshoe{\Qp{\variable{x}}}{\Gp{\variable{y}}}}{\pardisjunction{\negation{\Qp{\variable{x}}}}{\Gp{\variable{y}}}}$ (because it's not a sentence). 

The way we extend the notion of provable equivalence is through the universal closure of a formula. 
\begin{majorILnc}{\LnpDC{Universal Closure}}
The \df{universal closure} of a formula $\CAPTHETA$, written $\forall\CAPTHETA$, is the sentence that results by prefixing universal quantifiers in alphabetical order for all free variables of $\CAPTHETA$. 
\end{majorILnc}
\noindent{}E.g., $\forall\bpartriplebar{\parhorseshoe{\Qp{\variable{x}}}{\Gp{\variable{y}}}}{\pardisjunction{\negation{\Qp{\variable{x}}}}{\Gp{\variable{y}}}}$, the universal closure of $\bpartriplebar{\parhorseshoe{\Qp{\variable{x}}}{\Gp{\variable{y}}}}{\pardisjunction{\negation{\Qp{\variable{x}}}}{\Gp{\variable{y}}}}$, is $\universal{\variable{x}}\universal{\variable{y}}\bpartriplebar{\parhorseshoe{\Qp{\variable{x}}}{\Gp{\variable{y}}}}{\pardisjunction{\negation{\Qp{\variable{x}}}}{\Gp{\variable{y}}}}$.

We can now define provable equivalence for \GQL{} formulas using the universal closure:
\begin{majorILnc}{\LnpDC{GQL Provably Equivalent}}
Two \GQL{} formulas $\CAPTHETA$ and $\CAPPHI$ are \nidf{provably equivalent}\index{provably equivalent!formulas of \GQL{}|textbf} \Iff the universal closure of the formula that has a biconditional as main connective and $\CAPTHETA$ and $\CAPPHI$ as immediate constituents is derivable in \GQD{}; in other words, \Iff $\sststile{}{}\forall\bpartriplebar{\CAPTHETA}{\CAPPHI}$ in \GQD{}. 
\end{majorILnc}
\noindent{}It's important to note that this definition really is a generalization of definition \mvref{GSDprovablyequivalent}.
If $\CAPTHETA$ and $\CAPPHI$ are sentences of \GSL{} (recall that every sentence of \GSL{} is also a formula of \GQL{}), then they are provably equivalent on definition \ref{GSDprovablyequivalent} \Iff they are provably equivalent on definition \ref{GQL Provably Equivalent}.

Before moving to the proof of the Replacement Theorem for \GQD{} (Thm. \pncmvref{GQD Replacement Theorem}), it is convenient to prove the following one-step Replacement Lemmas:
\begin{THEOREM}{\LnpTC{OneStepReplacementLemmas} One-step Replacement Lemmas:}
If $\sststile{}{}\forall\partriplebar{\CAPPHI}{\CAPPHI^*}$, then:
\begin{cenumerate}
\item $\sststile{}{}\forall\partriplebar{\negation{\CAPPHI}}{\negation{\CAPPHI^*}}$
\item\label{exampleonesteplemma}
$\sststile{}{}\forall\partriplebar{\parconjunction{\CAPPHI}{\conjunction{\CAPPHI_1}{\conjunction{\ldots}{\CAPPHI_{\integer{p}}}}}}{\parconjunction{\CAPPHI^*}{\conjunction{\CAPPHI_1}{\conjunction{\ldots}{\CAPPHI_{\integer{p}}}}}}$
\item[] \hspace{1in} $\vdots$
\item $\sststile{}{}\forall\partriplebar{\parconjunction{\CAPPHI_1}{\conjunction{\ldots}{\conjunction{\CAPPHI_{\integer{p}}}{\CAPPHI}}}}{\parconjunction{\CAPPHI_1}{\conjunction{\ldots}{\conjunction{\CAPPHI_{\integer{p}}}{\CAPPHI^*}}}}$
\item 
$\sststile{}{}\forall\partriplebar{\pardisjunction{\CAPPHI}{\disjunction{\CAPPHI_1}{\disjunction{\ldots}{\CAPPHI_{\integer{p}}}}}}{\pardisjunction{\CAPPHI^*}{\disjunction{\CAPPHI_1}{\disjunction{\ldots}{\CAPPHI_{\integer{p}}}}}}$
\item[] \hspace{1in} $\vdots$
\item $\sststile{}{}\forall\partriplebar{\pardisjunction{\CAPPHI_1}{\disjunction{\ldots}{\disjunction{\CAPPHI_{\integer{p}}}{\CAPPHI}}}}{\pardisjunction{\CAPPHI_1}{\disjunction{\ldots}{\disjunction{\CAPPHI_{\integer{p}}}{\CAPPHI^*}}}}$
\item $\sststile{}{}\forall\bpartriplebar{\parhorseshoe{\CAPPHI}{\CAPPSI}}{\parhorseshoe{\CAPPHI^*}{\CAPPSI}}$
\item $\sststile{}{}\forall\bpartriplebar{\parhorseshoe{\CAPPSI}{\CAPPHI}}{\parhorseshoe{\CAPPSI}{\CAPPHI^*}}$
\item $\sststile{}{}\forall\bpartriplebar{\partriplebar{\CAPPHI}{\CAPPSI}}{\partriplebar{\CAPPHI^*}{\CAPPSI}}$
\item $\sststile{}{}\forall\bpartriplebar{\partriplebar{\CAPPSI}{\CAPPHI}}{\partriplebar{\CAPPSI}{\CAPPHI^*}}$
\end{cenumerate}
And, if $\sststile{}{}\forall\universal{\BETA}\bpartriplebar{\CAPPHI}{\CAPPHI^*}$, then:
\begin{enumerate}[label=(\arabic*), leftmargin=1.85\parindent,
labelindent=.35\parindent, labelsep=*, itemsep=0pt, start=10]%
\item $\sststile{}{}\forall\bpartriplebar{\universal{\BETA}\CAPPHI}{\universal{\BETA}\CAPPHI^*}$
\item $\sststile{}{}\forall\bpartriplebar{\existential{\BETA}\CAPPHI}{\existential{\BETA}\CAPPHI^*}$
\end{enumerate}
\end{THEOREM}
\noindent{}We prove \ref{exampleonesteplemma} for the case of a 2-place conjunction and leave the rest to the reader to prove in a similar way.
We use the following notation. 
If $\CAPPHI$ is a \GQL{} formula, then let $\variable{x}_1,\ldots,\variable{x}_{\integer{m}}$ be the complete list of free variables in $\CAPPHI$. 
Further, let $\CAPPHI\constant{c_{\integer{1}}}\ldots\constant{c_{\integer{\integer{m}}}}/\variable{x}_1\ldots\variable{x}_{\integer{m}}$ be the formula you get by substituting $\constant{c_1}$ for $\variable{x}_1$, $\ldots$, and $\constant{c_{\integer{m}}}$ for $\variable{x}_{\integer{m}}$.
\begin{PROOFOF}{Thm. \ref{OneStepReplacementLemmas}, \ref{exampleonesteplemma}, for 2-place Conjunctions}
Assume that $\sststile{}{}\forall\partriplebar{\CAPPHI}{\CAPPHI^*}$. 
Then consider some derivation $\Derivation{D}$ in \GQD{} of $\forall\partriplebar{\CAPPHI}{\CAPPHI^*}$.
The basic idea is to extend this derivation to a derivation of $\forall\partriplebar{\parconjunction{\CAPPHI}{\CAPPSI}}{\parconjunction{\CAPPHI^*}{\CAPPSI}}$ by first stripping away the initial quantifiers, then manipulating the truth functional connectives, and, finally, restoring the quantifiers. In detail, the new extended derivation should go (to save space when numbering lines, let $\integer{q}=\integer{n}+\integer{m}$):
\begin{gproofnn}
\glinend{ }{$\qquad\vdots$}{ }
\gline{$\integer{n}$}{$\forall\partriplebar{\CAPPHI}{\CAPPHI^*}$}{last line of $\Derivation{D}$}
\gline{$\integer{n}+1$}{$\forall[\partriplebar{\CAPPHI}{\CAPPHI^*}\constant{c}_1/\variable{x}_1]$}{\Rule{$\forall$-Elim}, $\integer{n}$}
\glinend{ }{$\qquad\vdots$}{ }
\gline{$\integer{n}+\integer{m}$}{$\partriplebar{\CAPPHI}{\CAPPHI^*}\constant{c_{\integer{1}}}\ldots\constant{c_{\integer{\integer{m}}}}/\variable{x}_1\ldots\variable{x}_{\integer{m}}$}{\Rule{$\forall$-Elim}, $\integer{n}+\integer{m}-1$}
\gaproof{
\galine{$\integer{q}+1$}{$\parconjunction{\CAPPHI}{\CAPPSI}\constant{c_{\integer{1}}}\ldots\constant{c_{\integer{\integer{m}}}}/\variable{x}_1\ldots\variable{x}_{\integer{m}}$}{\Rule{Assume}}
\galine{$\integer{q}+2$}{$\CAPPHI\constant{c_{\integer{1}}}\ldots\constant{c_{\integer{\integer{m}}}}/\variable{x}_1\ldots\variable{x}_{\integer{m}}$}{\Rule{$\WEDGE$-Elim}, $\integer{q}+1$}
\galine{$\integer{q}+3$}{$\CAPPHI^*\constant{c_{\integer{1}}}\ldots\constant{c_{\integer{\integer{m}}}}/\variable{x}_1\ldots\variable{x}_{\integer{m}}$}{\Rule{$\TRIPLEBAR$-Elim}, $\integer{q}$, $\integer{q}+2$}
\galine{$\integer{q}+4$}{$\CAPPSI\constant{c_{\integer{1}}}\ldots\constant{c_{\integer{\integer{m}}}}/\variable{x}_1\ldots\variable{x}_{\integer{m}}$}{\Rule{$\WEDGE$-Elim}, $\integer{q}+1$}
\galine{$\integer{q}+5$}{$\parconjunction{\CAPPHI^*}{\CAPPSI}\constant{c_{\integer{1}}}\ldots\constant{c_{\integer{\integer{m}}}}/\variable{x}_1\ldots\variable{x}_{\integer{m}}$}{\Rule{$\WEDGE$-Intro}, $\integer{q}+3$, $\integer{q}+4$}
}
\gline{$\integer{q}+6$}{${\parconjunction{\CAPPHI}{\CAPPSI}\constant{c_{\integer{1}}}\ldots\constant{c_{\integer{\integer{m}}}}/\variable{x}_1\ldots\variable{x}_{\integer{m}}}\HORSESHOE$}{ }
\glinend{}{$\qquad{\parconjunction{\CAPPHI^*}{\CAPPSI}\constant{c_{\integer{1}}}\ldots\constant{c_{\integer{\integer{m}}}}/\variable{x}_1\ldots\variable{x}_{\integer{m}}}$}{\Rule{$\HORSESHOE$-Intro}, $\integer{q}+1$--$\integer{q}+5$}

\gaproof{
\galine{$\integer{q}+7$}{$\parconjunction{\CAPPHI^*}{\CAPPSI}\constant{c_{\integer{1}}}\ldots\constant{c_{\integer{\integer{m}}}}/\variable{x}_1\ldots\variable{x}_{\integer{m}}$}{\Rule{Assume}}
\galine{$\integer{q}+8$}{$\CAPPHI^*\constant{c_{\integer{1}}}\ldots\constant{c_{\integer{\integer{m}}}}/\variable{x}_1\ldots\variable{x}_{\integer{m}}$}{\Rule{$\WEDGE$-Elim}, $\integer{q}+7$}
\galine{$\integer{q}+9$}{$\CAPPSI\constant{c_{\integer{1}}}\ldots\constant{c_{\integer{\integer{m}}}}/\variable{x}_1\ldots\variable{x}_{\integer{m}}$}{\Rule{$\WEDGE$-Elim}, $\integer{q}+7$}
\galine{$\integer{q}+10$}{$\CAPPHI\constant{c_{\integer{1}}}\ldots\constant{c_{\integer{\integer{m}}}}/\variable{x}_1\ldots\variable{x}_{\integer{m}}$}{\Rule{$\TRIPLEBAR$-Elim}, $\integer{q}$, $\integer{q}+8$}
\galine{$\integer{q}+11$}{$\parconjunction{\CAPPHI}{\CAPPSI}\constant{c_{\integer{1}}}\ldots\constant{c_{\integer{\integer{m}}}}/\variable{x}_1\ldots\variable{x}_{\integer{m}}$}{\Rule{$\WEDGE$-Intro}, $\integer{q}+9$, $\integer{q}+10$}
}

\gline{$\integer{q}+12$}{${\parconjunction{\CAPPHI^*}{\CAPPSI}\constant{c_{\integer{1}}}\ldots\constant{c_{\integer{\integer{m}}}}/\variable{x}_1\ldots\variable{x}_{\integer{m}}}\HORSESHOE$}{ }
\glinend{}{$\qquad{\parconjunction{\CAPPHI}{\CAPPSI}\constant{c_{\integer{1}}}\ldots\constant{c_{\integer{\integer{m}}}}/\variable{x}_1\ldots\variable{x}_{\integer{m}}}$}{\Rule{$\HORSESHOE$-Intro}, $\integer{q}+7$--$\integer{q}+11$}

\gline{$\integer{q}+13$}{$[{\parconjunction{\CAPPHI}{\CAPPSI}}\TRIPLEBAR$}{ }
\glinend{}{$\qquad{\parconjunction{\CAPPHI^*}{\CAPPSI}]\constant{c_{\integer{1}}}\ldots\constant{c_{\integer{\integer{m}}}}/\variable{x}_1\ldots\variable{x}_{\integer{m}}}$}{\Rule{$\TRIPLEBAR$-Intro}, $\integer{q}+6$, $\integer{q}+12$}

\gline{$\integer{q}+14$}{$\forall[{\parconjunction{\CAPPHI}{\CAPPSI}}\TRIPLEBAR$}{ }
\glinend{}{$\qquad{\parconjunction{\CAPPHI^*}{\CAPPSI}]\constant{c_{\integer{1}}}\ldots\constant{c_{\integer{\integer{m}-1}}}/\variable{x}_1\ldots\variable{x}_{\integer{m}-1}}$}{\Rule{$\forall$-Intro}, $\integer{q}+13$}

\glinend{ }{$\qquad\vdots$}{ }

\gline{$\integer{n}+2\integer{m}$}{$\forall\bpartriplebar{\parconjunction{\CAPPHI}{\CAPPSI}}{\parconjunction{\CAPPHI^*}{\CAPPSI}}$}{\Rule{$\forall$-Intro}, $\integer{n}+2\integer{m}-13$}

\end{gproofnn}
\noindent{}It is important to note that all the constants introduced on lines $\integer{n}+1$ through $\integer{n}+\integer{m}$ need to be new constants that do not appear in any previous lines. 
If not, then there's no guarantee that we be able to do \Rule{$\forall$-Intro} on the end lines. 
\end{PROOFOF}

\begin{PROOFOF}{Thm. \ref{GQD Replacement Theorem}}
Just as with the proof of the Restricted Replacement Theorem for \GSD{} (Thm. \pmvref{ExchangeRuleTheorem}), the proof for the Replacement Theorem for \GQD{} is a recursive proof. But, since the definition of provably equivalent is different (we've extended it to formulas of \GQL{}) we can't simply extend the proof of theorem \ref{ExchangeRuleTheorem} by adding new cases to the inheritance step for the quantifiers. 

Assume that $\CAPPHI$ and $\CAPPHI^*$ are provably equivalent formulas of \GQL{} (assume that $\sststile{}{}\forall\partriplebar{\CAPPHI}{\CAPPHI^*}$), that $\CAPPHI$ is a subformula of $\CAPTHETA$, and that $\CAPTHETA^*$ is the result of replacing $\CAPPHI$ with $\CAPPHI^*$ in $\CAPTHETA$.
\begin{description}
\item[Base Step:]
Similar to the base step in the proof of theorem \ref{ExchangeRuleTheorem}, in the base case $\CAPTHETA$ is atomic and so has no subformula other than itself.
So, if $\CAPPHI$ is a subformula of $\CAPTHETA$, then $\CAPPHI=\CAPTHETA$. Hence $\CAPTHETA^*=\CAPPHI^*$. Since $\CAPPHI$ and $\CAPPHI^*$ are provably equivalent, it follows immediately that $\CAPTHETA$ and $\CAPTHETA^*$ are provably equivalent. 

\item[Inheritance Step:] \hfill 

\begin{description}
\item[Recursive Assumption:] 
Assume that the theorem holds for formulas $\CAPPSI$, $\CAPPSI_1$, $\ldots$, $\CAPPSI_{\integer{k}}$; that is, assume that if $\CAPPSI^*$ is the result of replacing $\CAPPHI$ with $\CAPPHI^*$, then $\sststile{}{}\forall\partriplebar{\CAPPSI}{\CAPPSI^*}$, and similarly for the others.

\item[Negation:]
Assume that $\CAPTHETA=\;\negation{\CAPPSI}$.
Either $\CAPPHI=\CAPTHETA$, in which case it trivially follows that $\sststile{}{}\forall\partriplebar{\CAPTHETA}{\CAPTHETA^*}$, or $\CAPPHI$ is a subformula of $\CAPPSI$ (and hence $\CAPTHETA^*=\;\negation{\CAPPSI^*}$).
By the recursive assumption, $\sststile{}{}\forall\partriplebar{\CAPPSI}{\CAPPSI^*}$.
It follows by the One-step Replacement Lemma (Thm. \ref{OneStepReplacementLemmas}) that $\sststile{}{}\forall\partriplebar{\negation{\CAPPSI}}{\negation{\CAPPSI^*}}$. 

\item[Conjunction:]
Assume that $\CAPTHETA=\conjunction{\CAPPSI_1}{\conjunction{\ldots}{\CAPPSI_{\integer{k}}}}$.
Either $\CAPPHI=\CAPTHETA$, in which case it trivially follows that $\sststile{}{}\forall\partriplebar{\CAPTHETA}{\CAPTHETA^*}$, or $\CAPPHI$ is a subformula of one of the conjuncts $\CAPPSI_{\integer{i}}$ (and hence $\CAPTHETA^*=\conjunction{\CAPPSI_1}{\conjunction{\ldots}{\conjunction{\CAPPSI_{\integer{i}}^*}{\conjunction{\ldots}{\CAPPSI_{\integer{k}}}}}}$).
By the recursive assumption, $\sststile{}{}\forall\partriplebar{\CAPPSI_{\integer{i}}}{\CAPPSI_{\integer{i}}^*}$.
It follows by the One-step Replacement Lemma (Thm. \ref{OneStepReplacementLemmas}) that $\sststile{}{}\partriplebar{\parconjunction{\CAPPSI_1}{\conjunction{\ldots}{\conjunction{\CAPPSI_{\integer{i}}}{\conjunction{\ldots}{\CAPPSI_{\integer{k}}}}}}}{\parconjunction{\CAPPSI_1}{\conjunction{\ldots}{\conjunction{\CAPPSI_{\integer{i}}^*}{\conjunction{\ldots}{\CAPPSI_{\integer{k}}}}}}}$.

\item[Disjunction:]
This case is left to the reader.

\item[Conditional]
This case is also left to the reader.

\item[Biconditional:]
This case is also left to the reader.

\item[Universal:]
Assume that $\CAPTHETA=\universal{\BETA}\CAPPSI$. 
Either $\CAPPHI=\CAPTHETA$, in which case it trivially follows that $\sststile{}{}\forall\partriplebar{\CAPTHETA}{\CAPTHETA^*}$, or $\CAPPHI$ is a subformula of $\CAPPSI$ (and hence $\CAPTHETA^*=\universal{\BETA}\CAPPSI^*$).
By the recursive assumption, $\sststile{}{}\forall\partriplebar{\CAPPSI}{\CAPPSI^*}$.
To derive $\forall\partriplebar{\universal{\BETA}\CAPPSI}{\universal{\BETA}\CAPPSI^*}$, start with the derivation of $\forall\partriplebar{\CAPPSI}{\CAPPSI^*}$. 
Extend it by adding as many steps of \Rule{$\forall$-Elim} as needed to get to the sentence $\partriplebar{\CAPPSI}{\CAPPSI^*}\constant{c_{\integer{1}}}\ldots\constant{c_{\integer{\integer{m}}}}/\variable{x}_1\ldots\variable{x}_{\integer{m}}$.
Then using \Rule{$\forall$-Intro} first on $\BETA$, then on the others we can get $\forall\universal{\BETA}\partriplebar{\CAPPSI}{\CAPPSI^*}$ on the line. 
Hence $\sststile{}{}\forall\universal{\BETA}\partriplebar{\CAPPSI}{\CAPPSI^*}$,
so by the One-step Replacement Lemma (Thm. \ref{OneStepReplacementLemmas}) we get that $\sststile{}{}\forall\bpartriplebar{\universal{\BETA}\CAPPSI}{\universal{\BETA}\CAPPSI^*}$.

\item[Existential:] This case is exactly the same, except that a different result from the One-step Replacement Lemma is used.

\end{description}

\item[Closure Step:]
Since the inheritance step covers all the ways to generate \GQL{} formulas, we've shown that the theorem holds for all \GQL{} formulas $\CAPTHETA$.
\end{description}
\end{PROOFOF}
\begin{PROOFOF}{Thm. \ref{GQD Shortcut Theorem2}, Part (ii)}
Since any two formulas $\CAPPHI$ and $\CAPPHI^*$ got by substituting \GQL{} formulas into the may-add and given schemas of the exchange shortcut rules from \GQDP{} (tables \ref{GSDplus2} and \ref{GQDplus}) are provably equivalent, it follows from the Replacement Theorem for \GQD{} (Thm. \pncmvref{GQD Replacement Theorem}) that if $\CAPTHETA^*$ is a sentence sanctioned by an exchange rule applied to some sentence $\CAPTHETA$, then $\CAPTHETA$ and $\CAPTHETA^*$ are provably equivalent. That is, $\sststile{}{}\forall\partriplebar{\CAPTHETA}{\CAPTHETA^*}$. Since $\CAPTHETA$ and $\CAPTHETA^*$ are sentences, the universal closure of their biconditional $\triplebar{\CAPTHETA}{\CAPTHETA^*}$ is just the biconditional itself, so  $\sststile{}{}\partriplebar{\CAPTHETA}{\CAPTHETA^*}$.
But since  $\sststile{}{}\partriplebar{\CAPTHETA}{\CAPTHETA^*}$, it should be clear that $\CAPTHETA\sststile{}{}\CAPTHETA^*$. 
Thus, any application of an exchange rule from \GQDP{} is derivable using the basic rules of \GQD{} alone. 
\end{PROOFOF}


%%%%%%%%%%%%%%%%%%%%%%%%%%%%%%%%%%%%%%%%%%%%%%%%%%
\section{Exercises}
%%%%%%%%%%%%%%%%%%%%%%%%%%%%%%%%%%%%%%%%%%%%%%%%%%

\notocsubsection{\GQD{} Shortcut Rules}{exer:GQDSCprovablyequiv}
Prove that any two formulas got by substituting \GQL{} formulas into the may-add and given schemas of the exchange shortcut rules for \GQD{} are provably equivalent (Def. \pmvref{GQL Provably Equivalent}).
That is, show that the following hold for all \GQL{} formulas $\CAPPHI,\CAPPHI_1,\CAPPHI_2,\CAPTHETA,\CAPPSI$ by writing the appropriate derivation schemas.
Note that all but (7) and (8) deal with exchange shortcut rules from \GSDP{}.
For these virtually all the work has been done in exercise \ref{exercisesGSDshortcutrules}; all you need to do is show how to put the derivation schemas done there together and how to remove and put back on the quantifiers needed to make the universal closure.
\begin{multicols}{2}
\begin{enumerate}
\item $\sststile{}{}\forall[\negation{\parconjunction{\CAPPHI_1}{\CAPPHI_2}}\TRIPLEBAR\pardisjunction{\negation{\CAPPHI_1}}{\negation{\CAPPHI_2}}]$

% %%\item $\disjunction{\negation{\CAPPHI_1}}{\disjunction{\ldots}{\negation{\CAPPHI_{\integer{n}}}}}$, $\negation{\parconjunction{\CAPPHI_1}{\conjunction{\ldots}{\CAPPHI_{\integer{n}}}}}$
 
\item $\sststile{}{}\forall[\negation{\pardisjunction{\CAPPHI_1}{\CAPPHI_2}}\TRIPLEBAR\parconjunction{\negation{\CAPPHI_1}}{\negation{\CAPPHI_2}}]$ 
 
% %%\item $\conjunction{\negation{\CAPPHI_1}}{\conjunction{\ldots}{\negation{\CAPPHI_{\integer{n}}}}}$, $\negation{\pardisjunction{\CAPPHI_1}{\disjunction{\ldots}{\CAPPHI_{\integer{n}}}}}$ 
 
\item $\sststile{}{}\forall[\negation{\negation{\CAPPHI}}\TRIPLEBAR\CAPPHI]$

% %%\item $\CAPPHI$, $\negation{\negation{\CAPPHI}}$ 

\item $\sststile{}{}\forall[\parhorseshoe{\CAPPHI}{\CAPTHETA}\TRIPLEBAR\pardisjunction{\negation{\CAPPHI}}{\CAPTHETA}]$ 

% %%\item $\disjunction{\negation{\CAPPHI}}{\CAPTHETA}$, $\horseshoe{\CAPPHI}{\CAPTHETA}$
 
\item $\sststile{}{}\forall[\parhorseshoe{\CAPPHI}{\CAPTHETA}\TRIPLEBAR\parhorseshoe{\negation{\CAPTHETA}}{\negation{\CAPPHI}}]$ 

% %%\item $\horseshoe{\negation{\CAPTHETA}}{\negation{\CAPPHI}}$, $\horseshoe{\CAPPHI}{\CAPTHETA}$ 
 
\item $\sststile{}{}\forall[\negation{\parhorseshoe{\CAPPHI}{\CAPTHETA}}\TRIPLEBAR\parconjunction{\CAPPHI}{\negation{\CAPTHETA}}]$

% %%\item $\conjunction{\CAPPHI}{\negation{\CAPTHETA}}$, $\negation{\parhorseshoe{\CAPPHI}{\CAPTHETA}}$

\item $\sststile{}{}\forall[\negation{\universal{\BETA}{\CAPPHI}}\TRIPLEBAR\existential{\BETA}\negation{{\CAPPHI}}]$

\item $\sststile{}{}\forall[\negation{\existential{\BETA}{\CAPPHI}}\TRIPLEBAR\universal{\BETA}\negation{{\CAPPHI}}]$

\end{enumerate}
\end{multicols}
\begin{enumerate}[start=9]
\item $\sststile{}{}\forall[\parconjunction{\CAPTHETA}{\pardisjunction{\CAPPHI_1}{\CAPPHI_2}}\TRIPLEBAR\pardisjunction{\parconjunction{\CAPTHETA}{\CAPPHI_1}}{\parconjunction{\CAPTHETA}{\CAPPHI_2}}]$

% %%\item $\disjunction{\parconjunction{\CAPTHETA}{\CAPPHI_1}}{\disjunction{\ldots}{\parconjunction{\CAPTHETA}{\CAPPHI_{\integer{n}}}}}$, $\conjunction{\CAPTHETA}{\pardisjunction{\CAPPHI_1}{\disjunction{\ldots}{\CAPPHI_{\integer{n}}}}}$

\item $\sststile{}{}\forall[\parconjunction{\pardisjunction{\CAPPHI_1}{\CAPPHI_2}}{\CAPTHETA}\TRIPLEBAR\pardisjunction{\parconjunction{\CAPPHI_1}{\CAPTHETA}}{\parconjunction{\CAPPHI_2}{\CAPTHETA}}]$
 
% %%\item $\disjunction{\parconjunction{\CAPPHI_1}{\CAPTHETA}}{\disjunction{\ldots}{\parconjunction{\CAPPHI_{\integer{n}}}{\CAPTHETA}}}$, $\conjunction{\pardisjunction{\CAPPHI_1}{\disjunction{\ldots}{\CAPPHI_{\integer{n}}}}}{\CAPTHETA}$
 
\item $\sststile{}{}\forall[\pardisjunction{\CAPTHETA}{\parconjunction{\CAPPHI_1}{\CAPPHI_2}}\TRIPLEBAR\parconjunction{\pardisjunction{\CAPTHETA}{\CAPPHI_1}}{\pardisjunction{\CAPTHETA}{\CAPPHI_2}}]$
 
% %%\item $\conjunction{\pardisjunction{\CAPTHETA}{\CAPPHI_1}}{\conjunction{\ldots}{\pardisjunction{\CAPTHETA}{\CAPPHI_{\integer{n}}}}}$, $\disjunction{\CAPTHETA}{\parconjunction{\CAPPHI_1}{\conjunction{\ldots}{\CAPPHI_{\integer{n}}}}}$

\item $\sststile{}{}\forall[\pardisjunction{\parconjunction{\CAPPHI_1}{\CAPPHI_2}}{\CAPTHETA}\TRIPLEBAR\parconjunction{\pardisjunction{\CAPPHI_1}{\CAPTHETA}}{\pardisjunction{\CAPPHI_2}{\CAPTHETA}}]$

% %%\item $\conjunction{\pardisjunction{\CAPPHI_1}{\CAPTHETA}}{\conjunction{\ldots}{\pardisjunction{\CAPPHI_{\integer{n}}}{\CAPTHETA}}}$, $\disjunction{\parconjunction{\CAPPHI_1}{\conjunction{\ldots}{\CAPPHI_{\integer{n}}}}}{\CAPTHETA}$

\item $\sststile{}{}\forall[\partriplebar{\CAPTHETA}{\CAPPSI}\TRIPLEBAR\pardisjunction{\parconjunction{\CAPTHETA}{\CAPPSI}}{\parconjunction{\negation{\CAPTHETA}}{\negation{\CAPPSI}}}]$
\end{enumerate}

\notocsubsection{\GQD{} Practice Problems}{GQD Practice Problems} 
Write derivations for each of the following using only the rules specified by the instructor. 
It is probably a good idea to do the problems in order, as the earlier ones tend to be easier than the later ones. 
\begin{multicols}{2}
\begin{enumerate}
\item $\sststile{}{}\horseshoe{\universal{\variable{x}}\universal{\variable{y}}\Qpp{\variable{x}}{\variable{y}}}{\universal{\variable{z}}\Qpp{\variable{z}}{\variable{z}}}$
\item $\sststile{}{}\horseshoe{\universal{\variable{x}}\universal{\variable{y}}\Qpp{\variable{x}}{\variable{y}}}{\universal{\variable{x}}\universal{\variable{y}}\Qpp{\variable{y}}{\variable{x}}}$
\item $\sststile{}{}\horseshoe{\universal{\variable{x}}\parconjunction{\Qp{\variable{x}}}{\Gp{\variable{x}}}}{\bparconjunction{\universal{\variable{x}}\Qp{\variable{x}}}{\universal{\variable{x}}\Gp{\variable{x}}}}$
\item $\sststile{}{}\horseshoe{\bparconjunction{\universal{\variable{x}}\Qp{\variable{x}}}{\universal{\variable{x}}\Gp{\variable{x}}}}{\universal{\variable{x}}\parconjunction{\Qp{\variable{x}}}{\Gp{\variable{x}}}}$
\item $\sststile{}{}\horseshoe{\bpardisjunction{\universal{\variable{x}}\Qp{\variable{x}}}{\universal{\variable{x}}\Gp{\variable{x}}}}{\universal{\variable{x}}\pardisjunction{\Qp{\variable{x}}}{\Gp{\variable{x}}}}$
\item $\sststile{}{}\horseshoe{\universal{\variable{x}}\parhorseshoe{\Qp{\variable{x}}}{\Gp{\variable{x}}}}{\bparhorseshoe{\universal{\variable{x}}\Qp{\variable{x}}}{\universal{\variable{x}}\Gp{\variable{x}}}}$
\item $\sststile{}{}\horseshoe{\universal{\variable{x}}\parconjunction{\Pl}{\Qp{\variable{x}}}}{\parconjunction{\Pl}{\universal{\variable{x}}\Qp{\variable{x}}}}$
\item $\sststile{}{}\horseshoe{\parconjunction{\Pl}{\universal{\variable{x}}\Qp{\variable{x}}}}{\universal{\variable{x}}\parconjunction{\Pl}{\Qp{\variable{x}}}}$

\item $\sststile{}{}\horseshoe{\universal{\variable{x}}\pardisjunction{\Pl}{\Qp{\variable{x}}}}{\pardisjunction{\Pl}{\universal{\variable{x}}\Qp{\variable{x}}}}$
\item $\sststile{}{}\horseshoe{\pardisjunction{\Pl}{\universal{\variable{x}}\Qp{\variable{x}}}}{\universal{\variable{x}}\pardisjunction{\Pl}{\Qp{\variable{x}}}}$

\item $\sststile{}{}\horseshoe{\universal{\variable{x}}\parhorseshoe{\Pl}{\Qp{\variable{x}}}}{\parhorseshoe{\Pl}{\universal{\variable{x}}\Qp{\variable{x}}}}$
\item $\sststile{}{}\horseshoe{\parhorseshoe{\Pl}{\universal{\variable{x}}\Qp{\variable{x}}}}{\universal{\variable{x}}\parhorseshoe{\Pl}{\Qp{\variable{x}}}}$

\item $\sststile{}{}\horseshoe{\existential{\variable{x}}\universal{\variable{y}}\Qpp{\variable{x}}{\variable{y}}}{\universal{\variable{y}}\existential{\variable{x}}\Qpp{\variable{x}}{\variable{y}}}$

\item $\sststile{}{}\horseshoe{\universal{\variable{x}}\parhorseshoe{\Qp{\variable{x}}}{\Gp{\variable{x}}}}{\bparhorseshoe{\existential{\variable{x}}\Qp{\variable{x}}}{\existential{\variable{x}}\Gp{\variable{x}}}}$

\item $\sststile{}{}\horseshoe{\existential{\variable{x}}\parconjunction{\Qp{\variable{x}}}{\Gp{\variable{x}}}}{\bparconjunction{\existential{\variable{x}}\Qp{\variable{x}}}{\existential{\variable{x}}\Gp{\variable{x}}}}$

\item $\sststile{}{}\horseshoe{\bpardisjunction{\existential{\variable{x}}\Qp{\variable{x}}}{\existential{\variable{x}}\Gp{\variable{x}}}}{\existential{\variable{x}}\pardisjunction{\Qp{\variable{x}}}{\Gp{\variable{x}}}}$

\item $\sststile{}{}\horseshoe{\existential{\variable{x}}\parconjunction{\Pl}{\Qp{\variable{x}}}}{\parconjunction{\Pl}{\existential{\variable{x}}\Qp{\variable{x}}}}$
\item $\sststile{}{}\horseshoe{\parconjunction{\Pl}{\existential{\variable{x}}\Qp{\variable{x}}}}{\existential{\variable{x}}\parconjunction{\Pl}{\Qp{\variable{x}}}}$

\item $\sststile{}{}\horseshoe{\existential{\variable{x}}\pardisjunction{\Pl}{\Qp{\variable{x}}}}{\pardisjunction{\Pl}{\existential{\variable{x}}\Qp{\variable{x}}}}$
\item $\sststile{}{}\horseshoe{\pardisjunction{\Pl}{\existential{\variable{x}}\Qp{\variable{x}}}}{\existential{\variable{x}}\pardisjunction{\Pl}{\Qp{\variable{x}}}}$

\item $\sststile{}{}\horseshoe{\existential{\variable{x}}\parhorseshoe{\Pl}{\Qp{\variable{x}}}}{\parhorseshoe{\Pl}{\existential{\variable{x}}\Qp{\variable{x}}}}$
\item $\sststile{}{}\horseshoe{\parhorseshoe{\Pl}{\existential{\variable{x}}\Qp{\variable{x}}}}{\existential{\variable{x}}\parhorseshoe{\Pl}{\Qp{\variable{x}}}}$

\item $\sststile{}{}\horseshoe{\universal{\variable{x}}\parhorseshoe{\Qp{\variable{x}}}{\Pl}}{\bparhorseshoe{\existential{\variable{x}}\Qp{\variable{x}}}{\Pl}}$
\item $\sststile{}{}\horseshoe{\bparhorseshoe{\existential{\variable{x}}\Qp{\variable{x}}}{\Pl}}{\universal{\variable{x}}\parhorseshoe{\Qp{\variable{x}}}{\Pl}}$
\end{enumerate}
\end{multicols}
\begin{enumerate}[start=25]
\item $\sststile{}{}\horseshoe{\universal{\variable{x}}\existential{\variable{y}}\parconjunction{\Qp{\variable{x}}}{\Gp{\variable{y}}}}{\bparconjunction{\universal{\variable{x}}\Qp{\variable{x}}}{\existential{\variable{y}}\Gp{\variable{y}}}}$

\item $\sststile{}{}\horseshoe{\bparconjunction{\universal{\variable{x}}\Qp{\variable{x}}}{\existential{\variable{y}}\Gp{\variable{y}}}}{\universal{\variable{x}}\existential{\variable{y}}\parconjunction{\Qp{\variable{x}}}{\Gp{\variable{y}}}}$

\item $\sststile{}{}\horseshoe{\universal{\variable{x}}\existential{\variable{y}}\parconjunction{\Qp{\variable{x}}}{\Gp{\variable{y}}}}{\existential{\variable{y}}\universal{\variable{x}}\parconjunction{\Qp{\variable{x}}}{\Gp{\variable{y}}}}$

\item $\sststile{}{}\horseshoe{\universal{\variable{x}}\existential{\variable{y}}\pardisjunction{\Qp{\variable{x}}}{\Gp{\variable{y}}}}{\bpardisjunction{\universal{\variable{x}}\Qp{\variable{x}}}{\existential{\variable{y}}\Gp{\variable{y}}}}$

\item $\sststile{}{}\horseshoe{\bpardisjunction{\existential{\variable{y}}\Gp{\variable{y}}}{\universal{\variable{x}}\Qp{\variable{x}}}}{\universal{\variable{x}}\existential{\variable{y}}\pardisjunction{\Qp{\variable{x}}}{\Gp{\variable{y}}}}$

\item $\sststile{}{}\horseshoe{\existential{\variable{y}}\universal{\variable{x}}\pardisjunction{\Qp{\variable{x}}}{\Gp{\variable{y}}}}{\universal{\variable{x}}\existential{\variable{y}}\pardisjunction{\Qp{\variable{x}}}{\Gp{\variable{y}}}}$

\item $\sststile{}{}\horseshoe{\existential{\variable{y}}\universal{\variable{x}}\parhorseshoe{\Qp{\variable{x}}}{\Gp{\variable{y}}}}{\universal{\variable{x}}\existential{\variable{y}}\parhorseshoe{\Qp{\variable{x}}}{\Gp{\variable{y}}}}$

\item $\sststile{}{}\horseshoe{\universal{\variable{x}}\existential{\variable{y}}\pardisjunction{\Qp{\variable{x}}}{\Gp{\variable{y}}}}{\existential{\variable{y}}\universal{\variable{x}}\pardisjunction{\Qp{\variable{x}}}{\Gp{\variable{y}}}}$

\item $\sststile{}{}\horseshoe{\bparhorseshoe{\existential{\variable{y}}\Qp{\variable{y}}}{\existential{\variable{x}}\Gp{\variable{x}}}}{\existential{\variable{y}}\universal{\variable{x}}\parhorseshoe{\Qp{\variable{x}}}{\Gp{\variable{y}}}}$

\item $\sststile{}{}\horseshoe{\existential{\variable{y}}\universal{\variable{x}}\parhorseshoe{\Qp{\variable{x}}}{\Gp{\variable{y}}}}{\bparhorseshoe{\existential{\variable{x}}\Qp{\variable{x}}}{\existential{\variable{x}}\Gp{\variable{x}}}}$

\item 
$\sststile{}{}\horseshoe{\universal{\variable{x}}\existential{\variable{y}}\parhorseshoe{\Qp{\variable{x}}}{\Gp{\variable{y}}}}{\existential{\variable{y}}\universal{\variable{x}}\parhorseshoe{\Qp{\variable{x}}}{\Gp{\variable{y}}}}$

\item $\sststile{}{}\horseshoe{\existential{\variable{x}}\existential{\variable{y}}\parconjunction{\Qp{\variable{x}}}{\negation{\Qp{\variable{y}}}}}{\bparconjunction{\existential{\variable{x}}\Qp{\variable{x}}}{\existential{\variable{x}}\negation{\Qp{\variable{x}}}}}$

\item $\sststile{}{}\horseshoe{\existential{\variable{x}}\universal{\variable{y}}\parhorseshoe{\Qp{\variable{x}}}{\Gp{\variable{y}}}}{\bparhorseshoe{\universal{\variable{x}}\Qp{\variable{x}}}{\universal{\variable{x}}\Gp{\variable{x}}}}$

\item $\sststile{}{}\horseshoe{\bparconjunction{\existential{\variable{x}}\Qp{\variable{x}}}{\existential{\variable{x}}\negation{\Qp{\variable{x}}}}}{\existential{\variable{x}}\existential{\variable{y}}\parconjunction{\Qp{\variable{x}}}{\negation{\Qp{\variable{y}}}}}$

\item $\sststile{}{}\horseshoe{\bparhorseshoe{\universal{\variable{x}}\Qp{\variable{x}}}{\universal{\variable{x}}\Gp{\variable{x}}}}{\existential{\variable{x}}\universal{\variable{y}}\parhorseshoe{\Qp{\variable{x}}}{\Gp{\variable{y}}}}$

\item $\sststile{}{}\horseshoe{\bpardisjunction{\negation{\existential{\variable{x}}\Qp{\variable{x}}}}{\universal{\variable{x}}\Qp{\variable{x}}}}{\universal{\variable{x}}\universal{\variable{y}}\parhorseshoe{\Qp{\variable{x}}}{\Qp{\variable{y}}}}$
\end{enumerate}

%\theendnotes
