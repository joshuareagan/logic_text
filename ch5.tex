%%%%%%%%%%%%%%%%%%%%%%%%%%%%%%%%%%%%%%%%%%%%%%%%%%
\chapter{Translations}\label{Translations}
%%%%%%%%%%%%%%%%%%%%%%%%%%%%%%%%%%%%%%%%%%%%%%%%%%
% \AddToShipoutPicture*{\BackgroundPicC}

%%%%%%%%%%%%%%%%%%%%%%%%%%%%%%%%%%%%%%%%%%%%%%%%%%
\section{SL Applications}\label{SLApplications}
%%%%%%%%%%%%%%%%%%%%%%%%%%%%%%%%%%%%%%%%%%%%%%%%%%


Sentences in English have meanings.
Declarative sentences say something about the world, and in virtue of their various meanings they are either true or false.
By contrast, the sentences of \GSL{} and \GQL{} are, in themselves, meaningless strings of symbols.
Any meaning ascribed to \GSL{} and \GQL{} sentences comes from an assigned model.

The models in Chapter \ref{sententiallogic} do not distinguish between different sentences of the same truth value, because for the purposes of assessing logical truth in \GSL{} truth value is all that matters.
But if we want to translate English into \GSL{} we must specify the models less directly: by associating English sentences with sentence letters.

\subsection{Using \GSL{} Models for Translation}\label{GSLTranslationModels}

Let's say we have some declarative sentences of English that we would like to translate and model in \GSL{}.
To translate, we first look for the smallest declarative clauses, i.e. the \mention{atomic} parts, and we associate each with a sentence letter.
We decide whether each of these atoms is true or false and then define a model $\IntA$ that makes the appropriate assignment to the corresponding sentence letter.
If truth values are assigned according to what we know about reality, then $\IntA$ is a \mention{model} of reality, or at least that part of reality described by the English sentences.
Alternatively, we can assign truth values so that they do not match reality, in which case $\IntA$ is a model of a non-actual state of affairs.
Finally, we can assign values to sentences about which the facts are unknown, in which case $\IntA$ is a model of a hypothetical state of affairs.

English sentences are not all equally suitable for translation to \GQL{}.
Ideal English sentences are unambiguous, not vague, and do not have a truth value that changes over time.
More practically, we can also model sentences that are unambiguous in the appropriate context, specific enough to have determinate truth values, and whose truth values do not change during some period of interest.
One source of well-behaved sentences is mathematics.
For other sentences we can usually supply enough context to meet our criteria.

To illustrate, we might say that \mention{Texas is a US state,} is a true sentence of English.
We mean that Texas is a US state at the time the sentence is uttered.
By convention, English sentences in the present tense are assessed relative to the time of utterance.
But without such context matters aren't so clear.
The sentence is false of Texas in 1844 and true of Texas now, so its truth value has varied over time.
The sentence \mention{Texas was a US state at noon CDT on May 1, 1983,} is less ambiguous, and does not require any context-sensitive judgment about time to assess.

In practice, however, we make background assumptions about what is relevant, who is being discussed, the time of assessment, and so on.
E.g., \mention{Washington was elected President,} has an obvious and natural interpretation, one that's distinct from \mention{Joan Washington was elected President of the Montrose Dog Walkers Association.}

Note that we speak of translating \emph{sentences} and not, e.g., propositions.
There is a controversy in the philosophy of logic over which objects are the most fundamental bearers of truth value: sentences, propositions, judgments, statements, etc.
Without weighing in on that debate, we prefer sentences over the alternatives for a logic textbook.
We argued in Chapter \ref{introduction} that there is no uncontroversial definition of a sentence of English.
Even so, there is less dispute about what a sentence is than what propositions, judgments, or statements are.

There is difference between sentence tokens and sentence types.
We prefer not to deal with the difficulties associated with that distinction in this text.
We assume that all sentence tokens of a type have the same truth value, but that isn't true in general.\footnote{
	See \citealt{Grandy1993} for details.
}

\subsection{\GSL{} Translation Keys}\label{GSLTranslationKeys}

We use a \idf{translation key} to associate atomic English sentences with \GSL{} sentence letters.
A translation key is different from a model, but it determines which sentence letter assignments a \GSL{} model has to make.
A complete model also needs the truth value of each sentence, which isn't supplied by the translation key.
Once we've mapped atomic sentences of English to sentence letters we can use the key to translate complex English sentences into \GSL{}.
Complex sentences are composed of (i) atomic sentences and (ii) logical connectives of English.

When is a sentence of English simple enough to be considered \mention{atomic}?
In putting together a key we should capture as much of the logical structure of the English as is practical and useful.  
All else being equal, translation keys that hide logical structure in a sentence letter are deficient.
For example, the following is not ordinarily a good candidate for sentence letter assignment:

\begin{smenumerate}
	\item Mares eat oats and does eat oats and little lambs eat ivy.
\end{smenumerate}

\noindent{}The word \mention{and} plays a truth-functional role in this sentence, analogous to the role that \mention{$\WEDGE$} plays in \GSL{}.
A better translation key would assign each of \mention{Mares eat oats}, \mention{Does eat oats}, and \mention{Little lambs eat ivy} its own sentence letter.
The original sentence of English can then be expressed as a conjunction in \GSL{}.
In general, the goal of translation to \GSL{} is to replace the truth-functional connectives of English sentences with appropriate logical connectives from \GSL{}.


\begin{description}[itemsep=0em]
	\item[Translation Key:] \hfill{} 
	\begin{description}[itemsep=0em]
		\item $\Ml$: Mares eat oats.
		\item $\Dl$: Does eat oats.
		\item $\Ll$: Little lambs eat ivy.
	\end{description} 
\end{description}

\subsection{Connectives}\label{GSLConnectives and Trans}

Before considering other examples we need to say a little more about connectives.\index{logical connectives} 
In Chapter \ref{sententiallogic} (in definition \pmvref{Basic Symbols of GSL}) we listed the \sq{logical connectives} of \GSL{}.
We've seen the role they play in constructing sentences of \GSL{} and in fixing their truth values in a model. 
But we haven't said anything about what connectives are in general. 

The essential idea of a sentential \idf{connective}---whether a \sq{logical} one in some formal language or an expression in a natural language like English---is that it's a part of language that can be used to combine one or more sentences of the language into a new sentence.\footnote{Other connectives play a subtler role, as with the quantifiers of QL.
In QL quantifiers can be used to combine predicates to make a more complex predicate.} 
We say that one connective is within the \idf{scope} of a second connective iff the first connective is within a sentence directly connected with the second connective.
The \idf{main connective} is the one connective of the sentence that's not within the scope of any other connectives. 

A connective is \niidf{truth-functional}\index{connective!truth-functional} iff the truth value of every new sentence formed by that connective depends solely on the truth value of the constituent sentences. 
(If we're talking about a connective in a formal language like \GSL{}, we mean the truth value of a sentence \emph{in a model}.)
Not all connectives of English are truth-functional.
In particular, there are modal connectives---such as \mention{necessarily} and \mention{possibly}---that defy truth-functional characterization.
For example, the sentence

\begin{menumerate}
	\item Necessarily, bachelors are unmarried.
\end{menumerate}

\noindent{}seems plausibly true.  If we replace \mention{bachelors are unmarried} with another true claim, however, the result is not so plausible:

\begin{menumerate}
	\item Necessarily, there is life on earth.
\end{menumerate}

\noindent{}Both \mention{bachelors are unmarried} and \mention{There is life on earth} are true.
But it is easier to imagine all life on earth perishing than for bachelors to be married.
That there is life on earth is contingent.
The truth of a sentence with a modal main connective depends on more than the truth value of the part of the sentence governed by the modal connective.  

By contrast, the five logical connectives of \GSL{} are all truth-functional.
This disparity means that \GSL{} is not capable of capturing all of the logical structure of English.
The best it can do is to approximate certain features of English.
Some formal languages have the resources to capture the logical structure of modal claims.
We introduce one such language in Chapter \ref{furtherdirections}.
We hasten to add, however, that \GSL{} has sufficient resources to render many English sentences without significantly distorting their basic meaning.

\subsection{Conjunctions and Negations}\label{Translating GSLConjunctionsAndNegations}

A logical connective of \GSL{} is a suitable translation of a (truth-functional) connective of English iff they express the same truth function (see section \pmvref{Truth Functions Truth Tables and Boolean Operators}).
That is, iff the new sentences formed by those connectives share the same truth value whenever the constituent sentences that make them up share the same truth values.
We've already mentioned one such pair of corresponding connectives: \mention{and} and \mention{$\WEDGE$}.
The sentence

\begin{menumerate}
	\item\label{GSLTransConjunction1} The sun is shining \emph{and} the birds are chirping.
\end{menumerate}

\noindent{}is true \Iff the following two sentences are also true:

\begin{menumerate}
	\item The sun is shining.
	\item The birds are chirping.
\end{menumerate}

\noindent{}The same holds of other conjunctions in English.  The \GSL{} connective \mention{$\WEDGE$} is a good translation of \mention{and} most of the time.
Sometimes the structure of English conjunctions is more hidden, as in the following example:

\begin{menumerate}
	\item\label{GSLTransConjunction2} Nathanael Green \emph{and} `Light Horse Harry' Lee were officers in the Continental Army.
\end{menumerate}

\noindent{}A translation key should not break the sentence at the word \mention{and} and give the \CAPS{lhs}---\mention{Nathanael Green}---its own sentence letter.
The \CAPS{lhs} is not a declarative sentence.
It's clear that \ref{GSLTransConjunction2} expresses the conjunction of the following two sentences:

\begin{menumerate}
	\item Nathanael Green was an officer in the Continental Army.
	\item `Light Horse Harry' Lee was an officer in the Continental Army.
\end{menumerate}

\noindent{}It is appropriate to map each of these latter sentences to a sentence letter---say, $\Nl$ and $\Ll$, respectively---and express \mention{Nathanael Green and `Light Horse Harry' Lee were officers in the Continental Army} as a conjunction of these sentence letters---e.g., $\parconjunction{\Nl}{\Ll}$.

Other words of English also play the role of conjunction. Consider:

\begin{menumerate}
	\item\label{GSLTransConjunction3} Mary wants to drive her mother's car, \emph{but} she is \emph{not} old enough.
\end{menumerate}

\noindent{}The word \mention{but} connects two independent claims.
The sentence as a whole is true \Iff the \CAPS{lhs} and the \CAPS{rhs} are each true.
So \mention{but} plays the same truth-functional role as \mention{$\WEDGE$} in \GSL{}.

The word \mention{not} is another connective; it corresponds to the \mention{$\NEGATION$} of \GSL{}. 
The words ``not'' and ``no'' are typically translated with $\NEGATION$. 

Negations in English are not always so obvious.
``Innumerable'' is not numerable, ``unmistakable'' is not mistakable, ``never'' is not ever, ``immoral'' is not moral, ``clueless'' is to have no clue, and so on. 
Yet while the prefix ``in-'' is often negation, as in ``inevitable'', it is also often a direction, as in ``influx''.
The latter case is not one of negation.

Notice that in the \CAPS{rhs} of \ref{GSLTransConjunction3}, context indicates that \mention{she} is Mary, and that mention of her being not \mention{old enough} is about the fact that she is not old enough \emph{to drive her mother's car}.
English is, in this respect, more efficient than \GSL{}.
To translate this sentence into \GSL{} we need something like:

\begin{description}[itemsep=0em]
	\item[Translation Key:] \hfill{} 
	\begin{description}[itemsep=0em]
		\item $\Al$: Mary wants to drive her mother's car.
		\item $\Ol$: Mary is old enough to drive her mother's car.
	\end{description} 
\end{description}

\noindent{}We take out the \mention{not} because we can represent that with \mention{$\NEGATION$}.
We render the original sentence in \GSL{} as $\parconjunction{\Al}{\negation{\Ol}}$.
This is indistinguishable from the result of translating

\begin{menumerate}
	\item Mary wants to drive her mother's car \emph{and} she is not old enough to drive her mother's car.
\end{menumerate}

\noindent{}In English we typically say \mention{$\PHI$ but $\THETA$} instead of \mention{$\PHI$ and $\THETA$} to signal to listeners that they are about to hear something surprising.
They may expect that $\PHI$ and $\THETA$ are an unlikely pairing.
This difference from the word \mention{and} cannot be captured in \GSL{}---it is lost in translation.
\GSL{} translations are approximations of English, and as such they can't always preserve the original meaning.

The word \mention{and} can do more than serve as a conjunction.
Sometimes the order of the conjuncts affects meaning.
Compare:

\begin{menumerate}
	\item\label{GSLTransConjunction4} John took off his clothes and went to bed.
\end{menumerate}
\begin{menumerate}
	\item\label{GSLTransConjunction5} John went to bed and took off his clothes.
\end{menumerate}

The order of the conjuncts suggests that John performed these actions in a different order.
Yet if we translate \ref{GSLTransConjunction4} and \ref{GSLTransConjunction5} into \GSL{} the results are truth-functionally equivalent.
We think this equivalence is defensible, because \ref{GSLTransConjunction4} and \ref{GSLTransConjunction5} don't explicitly describe the order of events.
We tend to read order into them because it's usual in conversation to recount events in chronological order.
This is not a requirement, however.
There is nothing contradictory about the following:

\begin{menumerate}
	\item John went to bed and took off his clothes, but I don't know in which order.
\end{menumerate}

\begin{table}
	\renewcommand{\arraystretch}{1.5}%
	\begin{center}
		\begin{tabular}{ l l } %p{2.2in} p{2in}
			\toprule
			\textbf{English} & \textbf{\GSL{}} \\ 
			\midrule
			$\PHI$ and $\THETA$ & $\conjunction{\CAPPHI}{\CAPTHETA}$ \\
			both $\PHI$ and $\THETA$ & $\conjunction{\CAPPHI}{\CAPTHETA}$ \\
			$\PHI$, but $\THETA$ & $\conjunction{\CAPPHI}{\CAPTHETA}$ \\
			$\PHI$, but not $\THETA$ & $\conjunction{\CAPPHI}{\negation{\CAPTHETA}}$ \\
			not $\PHI$, but $\THETA$ & $\conjunction{\negation{\CAPPHI}}{\CAPTHETA}$ \\
			\bottomrule
		\end{tabular} 
		\caption{Translations for Common English Conjunctions}
		\label{TransTableD} 
	\end{center}
\end{table}

\subsection{Disjunctions}\label{Translating GSLDisjunctions}

Another connective of English is the word \mention{or}:

\begin{menumerate}
	\item \emph{Either} the tigers will get us \emph{or} the lions will.
\end{menumerate}

\noindent{}The sentence as a whole is true \Iff a clause on either side of the \mention{or} is true.
This matches the truth-function of the \mention{$\VEE$} connective in \GSL{}.

\begin{description}[itemsep=0em]
	\item[Translation Key:] \hfill{} 
	\begin{description}[itemsep=0em]
		\item $\Il$: The tigers will get us.
		\item $\Ll$: The lions will get us.
	\end{description} 
\end{description}

\noindent{}The resulting translation: $\pardisjunction{\Il}{\Ll}$.
We again use context to fill out the sentence \mention{the lions will [get us]}.  

It might not seem quite right to translate \mention{or} as \mention{$\VEE$}. 
In \GSL{} $\pardisjunction{\CAPPHI}{\CAPTHETA}$ is true when both $\CAPPHI$ and $\CAPTHETA$ are true on some model. 
But sometimes in English when we say \mention{$\PHI$ or $\THETA$} we mean that one or the other of $\PHI$ and $\THETA$ is true---but not both. 
A disjunction in which both disjuncts can be true is \niidf{inclusive}\index{inclusive disjunction}, while a disjunction in which only one disjunct can be true is \niidf{exclusive}.\index{exclusive disjunction}
Should \mention{$\PHI$ or $\THETA$} be considered exclusive?
Is it incorrect to translate it as $\pardisjunction{\CAPPHI}{\CAPTHETA}$?
 
These are complicated questions that we cannot fully answer here. 
We believe, however, that \mention{or} is typically inclusive in English and so best translated as \mention{$\VEE$}.
Only in certain conversational contexts is the exclusive disjunction clearly intended.
We motivate our stance with a few examples.\footnote{This discussion borrows from Smith \citeyear[117--22]{Smith2012}.}
 	
One case in which \mention{or} sounds exclusive involves disjuncts that cannot possibly both be true. 
I might say \q{Lydia either took a boat or a plane}. 
It's not physically possible for Lydia to travel by distinct modes of travel at the same time.
However, we must distinguish the meaning of the sentence from what we otherwise know to be possible.
We know that it's impossible to travel by boat and by plane at the same time.
It's tempting to assimilate that fact into the sentence's meaning, but that leap is unjustified. 
Someone could understand \q{Lydia either took a boat or a plane} and agree that it's true, but also think that both disjuncts are true.
Perhaps Lydia's journey was disjointed and involved both modes of travel.
To conclude that this disjunction is exclusive requires more information than the sentence itself conveys.

There are two cases in which it's more plausible that \mention{or} is exclusive.  
The first case involves commands or rules. 
If you are at a fancy restaurant and a waiter says, \q{You may have the soup or a salad}, it's typically understood that you may have one or the other, but not both.

\begin{table}
	\renewcommand{\arraystretch}{1.5}%
	\begin{center}
		\begin{tabular}{ l l } %p{2.2in} p{2in}
			\toprule
			\textbf{English} & \textbf{\GSL{}} \\ 
			\midrule
			$\PHI$ or $\THETA$ & $\disjunction{\CAPPHI}{\CAPTHETA}$ \\
			either $\PHI$ or $\THETA$ & $\disjunction{\CAPPHI}{\CAPTHETA}$ \\
			$\PHI$ or $\THETA$, but not both & $\conjunction{\pardisjunction{\CAPPHI}{\CAPTHETA}}{\negation{\parconjunction{\CAPPHI}{\CAPTHETA}}}$ \\
			\bottomrule
		\end{tabular} 
		\caption{Translations for Common English Disjunctions}
		\label{TransTableC} 
	\end{center}
\end{table}

The second case involves elliptical clauses. 
The phrase \mention{$\PHI$ or $\THETA$} can be short for \mention{$\PHI$ or $\THETA$, but not both}. 
If so, then the intended message is exclusive. 
But this isn't a case in which the \mention{or} itself expresses exclusive disjunction. 
The \mention{or} plus the tacit phrase \mention{but not both} together express exclusive disjunction.

Special cases like these, which are rare but salient, might lead you to think that disjunctions in English are sometimes exclusive.
However, in these cases there are context clues indicating their exclusive character, separate from the meaning of the relevant sentences.
We know as a matter of cultural familiarity that restaurants don't offer \emph{both} soup \emph{and} salad unless we pay extra.  
When intergalactic visitors discover Earth and visit our restaurants, we recommend that waiters make the exclusivity of the disjunction explicit.
Otherwise there could be an interstellar diplomatic incident!
They can even use \GSL{} to do so:
 	
 \begin{menumerate}
 	\item The intergalactic visitor may have the soup \emph{or} a salad, \emph{but} \emph{not} both.
 \end{menumerate}
 	
\noindent{}This sentence has several truth-functional connectives, so we must be careful to understand which connectives govern which.
The \mention{but} governs everything else, so the whole sentence is a conjunction.
The visitor may have the soup or a salad, \emph{and} he may not have both the soup and a salad.
The translation key:

\begin{description}[itemsep=0em]
	\item[Translation Key:] \hfill{} 
	\begin{description}[itemsep=0em]
		\item $\Pl$: The visiting intergalactic visitor may have the soup.
		\item $\Dl$: The visiting intergalactic visitor may have a salad.
	\end{description} 
\end{description}

\noindent{}First we translate the \CAPS{lhs} of the \mention{but}: $\pardisjunction{\Pl}{\Dl}$.
If we translate the \CAPS{rhs} without the \mention{not}---i.e., \mention{the intergalactic visitor may have both soup and a salad}---we get: $\parconjunction{\Pl}{\Dl}$.
We negate the result to account for the \mention{not}: $\negation{\parconjunction{\Pl}{\Dl}}$.
Putting it all together, the resulting translation is: $\parconjunction{\pardisjunction{\Pl}{\Dl}}{\negation{\parconjunction{\Pl}{\Dl}}}$.

\subsection{Conditionals and Biconditionals}\label{Translating GSLConditionals}

Another connective of English is the \emph{conditional}.
Conditionals are expressible in many ways, but perhaps the most distinctive way is \mention{if $\ldots$ then $\ldots$.}

\begin{menumerate}
	\item If George Washington crosses the Delaware River then the Hessians will be defeated.
\end{menumerate}

\noindent{}Is the conditional of English a truth-functional connective?
Philosophers have been arguing over how to answer this question since the ancient debate between Diodorus Cronus and Philo of Megara over 2000 years ago.
SL has only truth-functional connectives, so if the conditional is not truth-functional then \mention{$\HORSESHOE$} cannot fully capture its meaning.
We can show, however, that \mention{$\HORSESHOE$} is the best truth-functional representation of the conditional.
As we discuss translations the reader will get a sense of how well the model fits English.

Let's consider a few examples.
An \GSL{} sentence of the form $\parhorseshoe{\CAPPHI}{\CAPTHETA}$ is false on a model $\IntA$ \Iff $\IntA$ makes $\CAPPHI$ true and $\CAPTHETA$ false.
That gives us the correct assessment of the English sentence:

\begin{menumerate}
	\item If $2+2=4$ then $4+4=6$.
\end{menumerate}

\noindent{}On all other combinations of truth values for $\CAPPHI$ and $\CAPTHETA$, \GSL{} sentences of the form $\parhorseshoe{\CAPPHI}{\CAPTHETA}$ are true.
Consider a case in which both the \CAPS{lhs} and the \CAPS{rhs} of a conditional are false.
Some such conditionals are true:

\begin{menumerate}
	\item\label{GSLTransGW1} If George Washington landed on the moon then George Washington landed on the moon.
\end{menumerate}

\noindent{}This sentence is not merely true. It's a logical truth.
Other instances are more dubious: 

\begin{menumerate}
	\item\label{GSLTransGW2} If Soviet cosmonauts landed on the moon in 1968, then George Washington was never elected President.
\end{menumerate}

\noindent{}No truth-functional connective allows us to distinguish the last two examples.
In both cases we can only refer to the truth values of each side: the \CAPS{lhs} and the \CAPS{rhs} are false.
In order to admit \ref{GSLTransGW1} as true, we must also admit \ref{GSLTransGW2}.
This is one price we must pay for truth-functional conditionals.

Here is a sentence in which both the \CAPS{lhs} and the \CAPS{rhs} are true:

\begin{menumerate}
	\item\label{GSLTransGW3} If George Washington crossed the Delaware River then George Washington crossed the Delaware River.
\end{menumerate}

\noindent{}As with \ref{GSLTransGW1} this is not merely true, it's a logical truth.
Consider an \GSL{} model that makes $\CAPPHI$ false and $\CAPTHETA$ true, and hence makes $\parhorseshoe{\CAPPHI}{\CAPTHETA}$ true.
We can replace the \CAPS{lhs} of \ref{GSLTransGW3} with a false conjunction and the result is still true:

\begin{menumerate}
	\item If George Washington crossed the Delaware River and Thomas Jefferson invented bifocals, then George Washington crossed the Delaware River.
\end{menumerate}

\noindent{}The true inventor of bifocals was probably Benjamin Franklin.
Even though the conjunction on the \CAPS{lhs} is false, the sentence is still a logical truth.
The corresponding \GSL{} translation is \CAPS{tft}: $\parhorseshoe{\parconjunction{\Gl}{\Bl}}{\Gl}$.
So, while the \mention{$\HORSESHOE$} is not a perfect match for the English language conditional, it is the best truth-functional translation.

\begin{table}
	\renewcommand{\arraystretch}{1.5}%
	\begin{center}
		\begin{tabular}{ l l } %p{2.2in} p{2in}
			\toprule
			\textbf{English} & \textbf{\GSL{}} \\ 
			\midrule
			if $\PHI$, then $\THETA$ & $\horseshoe{\CAPPHI}{\CAPTHETA}$ \\
			$\PHI$ only if $\THETA$ & $\horseshoe{\CAPPHI}{\CAPTHETA}$ \\
			$\PHI$ if $\THETA$ & $\horseshoe{\CAPTHETA}{\CAPPHI}$ \\
			$\PHI$ provided that $\THETA$ & $\horseshoe{\CAPTHETA}{\CAPPHI}$ \\
			provided $\PHI$, $\THETA$ & $\horseshoe{\CAPPHI}{\CAPTHETA}$ \\
			$\PHI$ assuming that $\THETA$ & $\horseshoe{\CAPTHETA}{\CAPPHI}$ \\
			assuming $\PHI$, $\THETA$ & $\horseshoe{\CAPPHI}{\CAPTHETA}$ \\
			for $\PHI$, it's necessary that $\THETA$ & $\horseshoe{\CAPPHI}{\CAPTHETA}$ \\
			for $\PHI$, it's sufficient that $\THETA$ & $\horseshoe{\CAPTHETA}{\CAPPHI}$ \\
			$\PHI$ if and only if $\THETA$ & $\triplebar{\CAPPHI}{\CAPTHETA}$ \\
			for $\PHI$ it's necessary and sufficient that $\THETA$ & $\triplebar{\CAPPHI}{\CAPTHETA}$ \\
			$\PHI$ just when $\THETA$ & $\triplebar{\CAPPHI}{\CAPTHETA}$ \\
			$\PHI$ just in case $\THETA$ & $\triplebar{\CAPPHI}{\CAPTHETA}$ \\
			\bottomrule
		\end{tabular}% 
		\caption{Translations for Common English Conditionals and Biconditionals}
		\label{TransTableA}
	\end{center}
\end{table}

Biconditionals in English are closely related to conditionals, so they're subject to some of the same worries.
Nevertheless, we treat them as truth-functional connectives for the purpose of translating them into \GSL{}:

\begin{menumerate}
	\item Ruth may play outside if and only if she cleans her room.
\end{menumerate}

	\begin{description}[itemsep=0em]
		\item[Translation Key:] \hfill{} 
		\begin{description}[itemsep=0em]
			\item $\Hl$: Ruth may play outside.
			\item $\Cl$: Ruth cleans her room.
		\end{description} 
	\end{description}

\noindent{}With this key we translate the sentence as: $\partriplebar{\Hl}{\Cl}$.
We use \mention{$\TRIPLEBAR$} to translate English biconditionals.

Sometimes the word \mention{if} is used to express a biconditional:

\begin{menumerate}
	\item\label{RuthIf} Ruth may play outside if she cleans her room.
\end{menumerate}

\noindent{}But what if she doesn't clean her room?
Presumably she will not be allowed to play outside.
Yet this is not the literal meaning of the word \mention{if}.
We use context to supply the implicit \mention{but not otherwise} to the sentence.
It is well-known that parents use punishments and rewards to shape child behavior, so we rightly interpret \ref{RuthIf} as shorthand.

In general it is our policy is to translate what each sentence says literally.
If it is important for an argument to articulate what is being added by context, that should be specified explicitly.

\subsection{Further Examples}\label{GSLTransExamples}

Let's look at a few more translations.

\begin{majorILnc}{\LnpEC{GSLTranslationExampleB}} %implicature
	\begin{menumerate}
		\item\label{GSLTransSentenceE} Either Lydia flew or both Jackson and Helen took off early.
	\end{menumerate} 
	The first step is to identify the main connective of the sentence: \mention{either $\ldots$ or}.
	We translate \mention{either $\ldots$ or} as $\VEE$: 
	\begin{menumerate}
		\item\label{GSLTransSentenceF} (Lydia flew) $\VEE$ (both Jackson and Helen took off early).
	\end{menumerate}
	Notice that the word ``both'' clarifies the logical structure of the sentence.
	Without it, the sentence is, ``Either Lydia flew or Jackson and Helen took off early.''
	This is ambiguous, and could be read as saying that either Lydia or Jackson flew. 
	
	Next we identify and translate the main connectives in the connected clauses. 
	There are no connectives in \mention{Lydia flew}, so there is nothing to do with it. 
	There is one connective in \mention{both Jackson and Helen took off early}, \mention{both $\ldots$ and}. 
	We translate this as \mention{$\WEDGE$}:  
	\begin{menumerate}
		\item\label{GSLTransSentenceG} (Lydia flew) $\VEE$ ((Jackson took off early) $\WEDGE$ (Helen took off early)).
	\end{menumerate}
	\begin{description}[itemsep=0em]
		\item[Translation Key:] \hfill{} 
		\begin{description}[itemsep=0em]
			\item[] $\Nl$: Lydia flew.
			\item[] $\Jl$: Jackson took off early. 
			\item[] $\Hl$: Helen took off early. 
		\end{description}
	\end{description}
	\begin{menumerate}
		\item\label{GSLTransSentenceH} $\disjunction{\Nl}{\parconjunction{\Jl}{\Hl}}$
	\end{menumerate}
\end{majorILnc}

\begin{majorILnc}{\LnpEC{GSLTranslationExampleC}}
	\begin{menumerate}
		\item\label{GSLTransSentenceI} If Lydia got the job and Jackson didn't, then Lydia will take off tomorrow and Helen will have to come in.
	\end{menumerate}
	We identify the main connective, \mention{if $\ldots$ then}. 
	This is translated as \mention{$\rightarrow$}: 
	\begin{menumerate}
		\item\label{GSLTransSentenceJ} (Lydia got the job and Jackson didn't) $\HORSESHOE$ (Lydia will take off tomorrow and Helen will have to come in).
	\end{menumerate}
	Next we look at the \CAPS{LHS} of the conditional, \mention{Lydia got the job and Jackson didn't}. 
	The main (and only) connective of this sentence is \mention{and}, which we translate as \mention{$\WEDGE$}: 
	\begin{menumerate}
		\item\label{GSLTransSentenceK} ((Lydia got the job) $\WEDGE$ (Jackson didn't get the job)) $\HORSESHOE$ (Lydia will take off tomorrow and Helen will have to come in).
	\end{menumerate}

	\noindent{}The right-hand conjunct \mention{Jackson didn't} is an elliptical clause. 
	Lydia got the job and Jackson didn't \emph{get the job}.
	We make this tacit phrase explicit by putting it in brackets. 
	The clause \mention{Jackson didn't} has a negation in it. It combines \mention{not} with \mention{Jackson got the job}. 
	We finish the translation by translating the conjunction on the \CAPS{RHS} of the sentence: 
	\begin{menumerate}
		\item\label{GSLTransSentenceL} ((Lydia got the job) $\WEDGE$ ( $\NEGATION$ (Jackson got the job))) $\HORSESHOE$ ((Lydia will take off tomorrow) $\WEDGE$ (Helen will have to come in)).
	\end{menumerate}
	There are no more connectives, so we are ready to construct a translation key. 
	\begin{description}[itemsep=0em]
		\item[Translation Key:] \hfill{} 
		\begin{description}[itemsep=0em]
			\item[] $\Nl$: Lydia got the job.
			\item[] $\Jl$: Jackson got the job. 
			\item[] $\Il$: Lydia will take off tomorrow.
			\item[] $\Hl$: Helen will have to come in late.  
		\end{description}
	\end{description}
	\begin{menumerate}
		\item\label{GSLTransSentenceM} $\horseshoe{\parconjunction{\Nl}{\negation{\Jl}}}{\parconjunction{\Il}{\Hl}}$
	\end{menumerate}
\end{majorILnc}


\begin{majorILnc}{\LnpEC{GSLTranslationExampleE}}
	Some connectives in English do not correspond to any single \GSL{} connective.
	However, we can express any truth-function using some combination of the \GSL{} connectives that we \emph{do} have.
	For example: 
	\begin{menumerate}
		\item\label{GSLTransSentenceU} Neither Lydia nor Helen were late.
	\end{menumerate} 
	The connective is \mention{neither $\ldots$ nor}, and it joins \mention{Lydia [was late]} and \mention{Helen [was] late}. 
	There is no single \GSL{} connective that has the same truth-function as \mention{neither $\ldots$ nor}.\footnote{%
		But recall the logical connective \CAPS{NOR}, discussed at the end of section \mvref{Disjunctive Normal Form}. 
		\CAPS{NOR} correctly translates \mention{neither $\ldots$ nor}, but isn't part of \GSL{}. 
	} 
	Recall theorem \mvref{Truth-functional Expressive Completeness of GSL}, which says that any truth-functional connective can be expressed in \GSL{}. 
	It follows that there is some combination of \GSL{} connectives we can use to translate \mention{neither $\PHI$ nor $\THETA$}.
	There are two equivalent translations: $\negation{\pardisjunction{\CAPPHI}{\CAPTHETA}}$ and $\conjunction{\negation{\CAPPHI}}{\negation{\CAPTHETA}}$.
	\begin{table}
		\renewcommand{\arraystretch}{1.5}%
		\begin{center}
			\begin{tabular}{ l l } %p{2.2in} p{2in}
				\toprule
				\textbf{English} & \textbf{\GSL{}} \\ 
				\midrule
				not $\PHI$ & $\negation{\CAPPHI}$ \\
				it's not the case that $\PHI$ & $\negation{\CAPPHI}$ \\
				$\PHI$ unless $\THETA$ & $\horseshoe{\negation{\CAPTHETA}}{\CAPPHI}$ \\
				$\PHI$ unless $\THETA$ & $\disjunction{\CAPTHETA}{\CAPPHI}$ \\
				unless $\PHI$, $\THETA$ & $\horseshoe{\negation{\CAPPHI}}{\CAPTHETA}$ \\
				$\PHI$ if not $\THETA$ & $\horseshoe{\negation{\CAPTHETA}}{\CAPPHI}$ \\
				neither $\PHI$ nor $\THETA$ & $\negation{\pardisjunction{\CAPPHI}{\CAPTHETA}}$ \\
				& $\conjunction{\negation{\CAPPHI}}{\negation{\CAPTHETA}}$ \\
				not both $\PHI$ and $\THETA$ & $\negation{\parconjunction{\CAPPHI}{\CAPTHETA}}$ \\
				& $\disjunction{\negation{\CAPPHI}}{\negation{\CAPTHETA}}$ \\
				\bottomrule
			\end{tabular}
			\caption{Translations for Common English Negations and Complex Connectives}
			\label{TransTableF} 
		\end{center}
	\end{table}  
	So we have: 
	\begin{menumerate}
		\item\label{GSLTransSentenceV} $\NEGATION$ ((Lydia was late) $\VEE$ (Helen was late)).
	\end{menumerate} 
	With the key: 
	\begin{description}[itemsep=0em]
		\item[Translation Scheme:] \hfill{} 
		\begin{description}[itemsep=0em]
			\item $\Nl$: Lydia was late.
			\item $\Hl$: Helen was late. 
		\end{description} 
	\end{description}
	The final translation of sentence \ref{GSLTransSentenceU} is: 
	\begin{menumerate}
		\item\label{GSLTransSentenceW} $\negation{\pardisjunction{\Nl}{\Hl}}$
	\end{menumerate} 
\end{majorILnc}

\noindent{}The tables in this chapter should be thought of as rough-and-ready guides. 
Although many particular uses of \mention{and} express conjunction, not all do. 
Sometimes \mention{and} doesn't function as a connective at all, e.g. as in \mention{it will be years and years before the trees bear fruit} \citep[107]{Smith2012}.
Other times \mention{and} functions as a connective, but expresses a conditional instead of a conjunction, e.g. \mention{study hard, and you will pass the exam} \citep[107]{Smith2012}.

To translate a connective from English appropriately you must first determine what is being expressed, drawing on context as necessary to resolve ambiguities.
For example, ``Maria and Paul are married,'' can convey the conjunction ``Maria is married and Paul is married,'' or it can express ``Maria and Paul are married to each other,'' in which case the `and' is not a conjunction, but is serving a different logical purpose to be discussed in the next section.

%%%%%%%%%%%%%%%%%%%%%%%%%%%%%%%%%%%%%%%%%%%%%%%%%%
\section{QL Applications}
%%%%%%%%%%%%%%%%%%%%%%%%%%%%%%%%%%%%%%%%%%%%%%%%%%

\subsection{Constants and Predicates}
Let's say we want to translate the following sentence into \emph{\GSL{}}:

\begin{smenumerate}
	\item\label{GQLTrans1} Mary is happy, smart, adorable, and a child.
\end{smenumerate}

\noindent{}The only connective we can translate is the \mention{and}.
The translation key:

\begin{description}[itemsep=0em]
	\item[Translation Key:] \hfill{} 
	\begin{description}[itemsep=0em]
		\item[] $\Hl$: Mary is happy.
		\item[] $\Rl$: Mary is smart. 
		\item[] $\Al$: Mary is adorable.
		\item[] $\Cl$: Mary is a child.  
	\end{description}
\end{description}

\noindent{}The \GSL{} result is: $\parconjunction{\conjunction{\Hl}{\Rl}}{\conjunction{\Al}{\Cl}}$.
This translation might work for certain purposes, but the conjuncts have no logical connection with each other.
Nothing in \GQL{} indicates that each conjunct is about the same person.

In \GQL{} we can express that shared logical connection with constants and predicates.
Translation keys of \GQL{} connect nouns to constants and English predicates to \GQL{} predicates.
Let's use the following \GQL{} key to translate \ref{GQLTrans1}:

\begin{description}[itemsep=0em]
	\item[Translation Key:] \hfill{} 
	\begin{description}[itemsep=0em]
		\item[] $\constant{m}$: Mary
		\item[] $\Hl\variable{t}$: $\variable{t}$ a child. 
		\item[] $\Rl\variable{t}$: $\variable{t}$ is smart. 
		\item[] $\Al\variable{t}$: $\variable{t}$ is adorable.
		\item[] $\Cl\variable{t}$: $\variable{t}$ is a child.  
	\end{description}
\end{description}

\noindent{}The result is: $\parconjunction{\conjunction{\Hp{\constant{m}}}{\Rp{\constant{m}}}}{\conjunction{\Ap{\constant{m}}}{\Cp{\constant{m}}}}$.
This translation allows us to see that we are predicating several things about the same person.
On to another example:

\begin{menumerate}
	\item\label{GQLTrans2} Ronnie and Demaryius are athletic, but Peyton isn't.
\end{menumerate}

\noindent{}We \emph{could} translate this as a conjunction with three conjuncts in \GSL{}.
However, this would not make clear that the same predicate either applies, or doesn't, to each of the three.
Instead, let's use the following \GQL{} key:

\begin{description}[itemsep=0em]
	\item[Translation Key:] \hfill{} 
	\begin{description}[itemsep=0em]
		\item[] $\constant{r}$: Ronnie
		\item[] $\constant{d}$: Demaryius
		\item[] $\constant{p}$: Peyton
		\item[] $\Al\variable{t}$: $\variable{t}$ is athletic.
	\end{description}
\end{description}

\noindent{}The result: $\parconjunction{\conjunction{\Ap{\constant{r}}}{\Ap{\constant{d}}}}{\negation{\Ap{\constant{p}}}}$.
We see the common predicate in each conjunct.

The translation keys here resemble the \GQL{} models we provided earlier.\footnote{
	See table \ref{table:Example Interpretations} in Chapter \ref{quantifierlogic}.
}
One difference is that we don't have a domain assigned in either of the \GQL{} keys above.
We could add a domain to each of the above, and interpret the constant and predicate lines of the key as making assignments from the domain.
Hence, we can usually treat informal model assignments as translation keys.

\subsection{Quantifiers}

The quantifier \mention{$\forall$} corresponds to the English phrases \mention{all} or \mention{every}.
Let's say we want to claim that every member of some set is also a member of some other set:

\begin{menumerate}
	\item\label{GQLTrans3} All dogs are furry.
\end{menumerate}

\noindent{}There are two sets: the set of all dogs and the set of all furry things.
Sentence \ref{GQLTrans3} effectively claims that the set of dogs is a subset of the set of furry things.
How do we translate this into \GQL{}?
The word \mention{all} typically calls for a universal quantifier.

It isn't obvious from the sentence itself, but whenever we want to make a claim of the form \mention{All $\PHI$ are $\THETA$}, we nearly always want to translate it as: $\universal{\ALPHA}\parhorseshoe{\CAPPHI}{\CAPTHETA}$.
That is, with a \mention{$\forall$} governing an \mention{$\HORSESHOE$}.
This makes more sense if we paraphrase \ref{GQLTrans3} in MathEnglish as: \mention{For all $\variable{x}$, if $\variable{x}$ is a dog then $\variable{x}$ is furry.}

For the rest of the translations in this section we use the following model description as our translation key:

\begin{description}[itemsep=0em]
	\item[Animals model:] \hfill{} 
	\begin{description}[itemsep=0em]
		\item[] $\emph{Animals}(\variable{U})$: All animals.
		\item[] $\emph{Animals}(\Ap{'})$: is a mammal.
		\item[] $\emph{Animals}(\Cp{'})$: is a cat.
		\item[] $\emph{Animals}(\Dp{'})$: is a dog.
		\item[] $\emph{Animals}(\Ep{'})$: is energetic.
		\item[] $\emph{Animals}(\Hp{'})$: is a happy.
		\item[] $\emph{Animals}(\Rp{'})$: is furry.
		\item[] $\emph{Animals}(\Ap{''})$: is smarter than.
	\end{description}
\end{description}

\noindent{}We translate \ref{GQLTrans3} as: $\universal{\variable{x}}\parhorseshoe{\Dp{\variable{x}}}{\Rp{\variable{x}}}$.

Other English words can be translated with the universal quantifier.
The following uses the word \mention{no} to make a universal claim:

\begin{menumerate}
	\item\label{GQLTrans4} No dogs are furry.
\end{menumerate}

\noindent{}Again, we can think of this as a claim about two sets: the set of dogs and the set of furry things.
This sentence is tantamount to a claim that these sets are disjoint, i.e., that nothing is a member of both.
We usually translate such claims into the following form in \GQL{}: $\universal{\ALPHA}\parhorseshoe{\CAPPHI}{\negation{\CAPTHETA}}$.
To understand why, consider the following MathEnglish paraphrase: \mention{For all $\variable{x}$, if $\variable{x}$ is a dog then $\variable{x}$ is not furry.}
The resulting translation is: $\universal{\variable{x}}\parhorseshoe{\Dp{\variable{x}}}{\negation{\Rp{\variable{x}}}}$.

The word \mention{only} can also be used for universal claims:

\begin{menumerate}
	\item\label{GQLTrans5} Only dogs are furry.
\end{menumerate}

\noindent{}This is translated in the same way as sentence \ref{GQLTrans3}, except that we reverse the order of the LHS and the RHS of the conditional governed by the \mention{$\forall$}.
It's equivalent to the claim that \mention{All furry things are dogs.}
So: $\universal{\variable{x}}\parhorseshoe{\Rp{\variable{x}}}{\Dp{\variable{x}}}$.

The \GQL{} symbol \mention{$\exists$} corresponds to the English phrases \mention{there exists}, \mention{there is}, or \mention{some}.
Consider the following existential sentences.

\begin{menumerate}
	\item\label{GQLTrans6} Some dogs are furry.
	\item\label{GQLTrans7} Some dogs are not furry.
\end{menumerate}

\noindent{}If we again think of the set of dogs and the set of furry things, \ref{GQLTrans6} is a claim that there is at least one element that is a member of each set.
Notice that we are interpreting \ref{GQLTrans6} as a claim about at least one object.
In English we typically think that \ref{GQLTrans6} is a claim about at least two dogs.
For now we ignore certain plural/singular distinctions in the way we interpret existential claims.
For our purposes, \mention{some} means \mention{at least one}.
We handle \mention{some} in a more satisfactory way when we add to \GQL{} in Chapter \ref{furtherdirections}.

For now, we translate sentences like \ref{GQLTrans6} into the form: $\existential{\ALPHA}\parconjunction{\CAPPHI}{\CAPTHETA}$.
So, for \ref{GQLTrans6} itself: $\existential{\variable{x}}\parconjunction{\Dp{\variable{x}}}{\Rp{\variable{x}}}$.
And we account for the \mention{not} in sentence \ref{GQLTrans7} as follows: $\existential{\variable{x}}\parconjunction{\Dp{\variable{x}}}{\negation{\Rp{\variable{x}}}}$.

We can also translate sentence \ref{GQLTrans4}, \mention{No dogs are furry}, as an existential governed by a negation.
We could translate \mention{There is a furry dog} as: $\existential{\variable{x}}\parconjunction{\Rp{\variable{x}}}{\Dp{\variable{x}}}$.
We can negate the result to capture the meaning of \ref{GQLTrans4}: $\negation{\existential{\variable{x}}\parconjunction{\Rp{\variable{x}}}{\Dp{\variable{x}}}}$.
In fact, this latter translation is logically equivalent to the one we gave earlier: $\universal{\variable{x}}\parhorseshoe{\Dp{\variable{x}}}{\negation{\Rp{\variable{x}}}}$.
To see this, join these two sentences together with a biconditional, \mention{$\TRIPLEBAR$}, and prove that the result is \CAPS{qt}.

Let's translate more complicated sentences into \GQL{}:

\begin{menumerate}
	\item\label{GQLTrans8} All happy dogs are furry and energetic.
\end{menumerate}

\noindent{}We want to translate this as a universal quantifier governing a conditional, but we must be careful to translate each side of the conditional correctly.
We paraphrase \ref{GQLTrans8} in MathEnglish as: For all $\variable{x}$, if ($\variable{x}$ is happy and $\variable{x}$ is a dog) then ($\variable{x}$ is furry and $\variable{x}$ is energetic).
So we can consider each side of the conditional as a conjunction: $\universal{\variable{x}}\parhorseshoe{\parconjunction{\Hp{\variable{x}}}{\Dp{\variable{x}}}}{\parconjunction{\Rp{\variable{x}}}{\Ep{\variable{x}}}}$.

\begin{menumerate}
	\item\label{GQLTrans9} All cats and dogs are mammals.
\end{menumerate}

\noindent{}This sentence is also going to be translated as a universal quantifier governing a conditional, but the word \mention{and} can be tricky.  Here `and' seems to conjoin predicates, not sentences.  We may be tempted to translate \ref{GQLTrans9} as: $\universal{\variable{x}}\parhorseshoe{\parconjunction{\Cp{\variable{x}}}{\Dp{\variable{x}}}}{\Ap{\variable{x}}}$.  But this \GQL{} sentence can be translated into MathEnglish as: \mention{For every $\variable{x}$, if ($\variable{x}$ is a cat and $\variable{x}$ is a dog) then $\variable{x}$ is a mammal.}  But that's silly.  Unless mad scientists are involved, nothing is both a cat and a dog.  Instead, we should translate the \mention{and} in \ref{GQLTrans9} as a \mention{$\VEE$}:  $\universal{\variable{x}}\parhorseshoe{\pardisjunction{\Cp{\variable{x}}}{\Dp{\variable{x}}}}{\Ap{\variable{x}}}$.  Let's translate this \GQL{} sentence into MathEnglish: \mention{For all $\variable{x}$, if ($\variable{x}$ is a cat or $\variable{x}$ is a dog), then $\variable{x}$ is a mammal.}  Take a moment to see how this better expresses the meaning of \ref{GQLTrans9}.

Consider a sentence with multiple quantifiers:

\begin{menumerate}
	\item\label{GQLTrans10} All dogs are smarter than all cats.
\end{menumerate}

\noindent{}There are two instances of the word \mention{all}, so we use two universal quantifiers in the translation.
Consider a paraphrase into MathEnglish: \mention{For every $\variable{x}$, if $\variable{x}$ is a dog then (for all $\variable{y}$, if $\variable{y}$ is a cat then $\variable{x}$ is smarter than $\variable{y}$).}
So we translate \ref{GQLTrans10} as: $\universal{\variable{x}}\parhorseshoe{\Dp{\variable{x}}}{\universal{\variable{y}}\parhorseshoe{\Cp{\variable{y}}}{\App{\variable{x}}{\variable{y}}}}$.

We have not said much about the role of domains in translations.
Because the point of translations, other than the sheer joy of doing it, is to evaluate arguments, it is appropriate to choose a domain suitable for the arguments in question.
Often a suitable choice of domain simplifies the translations.
If we are translating arguments that mention both dogs and natural numbers, we need to include both in the domain and to have a predicate for each.
If the arguments deal only with dogs then we can take the domain to be dogs.
You may be able to do without a predicate for `dog', depending on the nature of the argument.

As we observed before many English sentences are ambiguous.
One systematic ambiguity is in sentences of the form \mention{All As are not Bs}, which can either mean that it is not true that \mention{All As are Bs}, or that all \mention{As are not-Bs}.
Context or content usually indicate what is meant:  \mention{All sheep are not good pets} likely has the first meaning: $\negation{\universal{\variable{x}}\parhorseshoe{\Sl\variable{x}}{\parconjunction{\Gp{\variable{x}}}{\Pp{\variable{x}}}}}$.
But \mention{All sharks are not good pets} has the second: $\universal{\variable{x}}\parhorseshoe{\Sl\variable{x}}{\negation{\parconjunction{\Gp{\variable{x}}}{\Pp{\variable{x}}}}}$.
One of the values of formalization is that we can clearly and unambiguously express the structure of both.

It is important to distinguish ambiguous sentences---English sentences that have more than one meaning---from sentences for which there is more than one good translation, but which are equivalent.
We see this with \mention{No sharks are good pets}: $\universal{\variable{x}}\parhorseshoe{\Sl\variable{x}}{\negation{\parconjunction{\Gp{\variable{x}}}{\Pp{\variable{x}}}}}$ and $\negation{\existential{\variable{x}}\parconjunction{\Sl\variable{x}}{\parconjunction{\Gp{\variable{x}}}{\Pp{\variable{x}}}}}$ are equally good (and equivalent) translations.

%%%%%%%%%%%%%%%%%%%%%%%%%%%%%%%%%%%%%%%%%%%%%%%%%%
\section{Exercises}
%%%%%%%%%%%%%%%%%%%%%%%%%%%%%%%%%%%%%%%%%%%%%%%%%%

\notocsubsection{\GSL{} to English Translations}{ex:GSL to English Translations}
Given the following translation key, translate the following \GSL{} sentences into English. 
\begin{description}[itemsep=0em]
	\item[Translation Key:] \hfill{} 
	\begin{description}[itemsep=0em]
		\item $\Cl$: Cindy the Capybara is a picky eater.
		\item $\Ol$: Oscar the Ocelot sleeps all day.
		\item $\Rl$: Ralph the Rhinoceros goes for a swim.
		\item $\Al$: France is east of Spain. 
	\end{description} 
\end{description}
\begin{multicols}{2}
	\begin{enumerate}
		\item $\horseshoe{\Cl}{\Ol}$
		\item $\horseshoe{\Ol}{\Cl}$
		\item $\negation{\parhorseshoe{\Rl}{\Al}}$
		\item $\horseshoe{\negation{\Rl}}{\Al}$
		\item $\disjunction{\parconjunction{\Cl}{\Ol}}{\negation{\partriplebar{\Cl}{\Al}}}$
		\item $\conjunction{\Cl}{\pardisjunction{\Ol}{\negation{\partriplebar{\Cl}{\Al}}}}$
		\item $\horseshoe{\Al}{\negation{\parconjunction{\Cl}{\Rl}}}$
		\item $\horseshoe{\parconjunction{\Cl}{\Rl}}{\negation{\Al}}$
	\end{enumerate}
\end{multicols}

%\begin{table}[!ht]
%\renewcommand{\arraystretch}{1.5}
%\begin{center}
\begin{figure}
\begin{longtable}[c]{ l l l } %p{2.2in} p{2in}
	\toprule
	&\textbf{Symbol} & \textbf{Model Assignment} \\
	\midrule 
	\endfirsthead
	\multicolumn{3}{c}{\emph{Continued from Previous Page}}\\
	\toprule
	&\textbf{Symbol} & \textbf{Model Assignment} \\
	\midrule 
	\endhead
	\bottomrule
	\caption{Model for Translations in Section \ref{Translation Problems}}\\ %[-.15in]
	\multicolumn{3}{c}{\emph{Continued next Page}}\\
	\endfoot
	\bottomrule
	\caption{Model for Translations in Section \ref{Translation Problems}}\\%
	\endlastfoot%
	\label{Trans Int Table}% 
	Universe:& & The set of states \\ \addlinespace[.25cm]
	Constants:& $\constant{c}$& CA\\
	& $\constant{m}$& MT\\
	& $\constant{h}$& RI\\
	& $\constant{e}$& TX\\ \addlinespace[.25cm]
	1 place predicates: &$\Pp{'}$& Pacific states\\
	&$\Ap{'}$& Atlantic states\\
	&$\Gp{'}$& Gulf states\\
	&$\Mp{'}$& Mountainous states\\
	&$\Cp{'}$& Coastal states\\ \addlinespace[.25cm]
	2 place predicates:&$\Lp{''}$& is larger than (area)\\
	&$\Bp{''}$& borders\\
\end{longtable}
\caption{Model for Translations in Section \ref{Translation Problems}}
\end{figure}

\notocsubsection{English to \GSL{} Translations \#1}{ex:English to GSL Translations 1}
Using some sensible translation key translate the following English sentences into \GSL{}. 
\begin{enumerate}
	\item If the sprockets come in on time, then we can fill the order.
	\item Only if the sprockets come in on time can we fill the order. 
	\item Either the order gets filled, or the cogs come in late and the sprockets never show up. 
	\item It's not the case that the sprockets need to come in for the order to be filled. 
	\item While filling the order is important, getting the sprockets in is more so. 
	\item The sprockets and cogs are late, but it's still not the case that we can't fill the order on time. 
	\item Assuming the order gets out on time, the sprockets will fail to arrive only if the cogs are either late or defective. 
	\item The spork is the least appreciated utensil. 
	\item They dined on mince, and slices of quince, [which] they ate with a runcible spoon. (1871, Edward Lear, “Owl \& Pussy-Cat” in \emph{Nonsense Songs})
	\item You eat with a spork if and only if you eat with a foon. 
	\item Although Jan will be amused, if you eat with a spork Jill will leave or at least not laugh.
	\item If you have a runcible spoon, then you don't need a fork, knife, or spoon. 
\end{enumerate}

\notocsubsection{English to \GSL{} Translations \#2}{ex:English to GSL Translations 2}
Using some sensible translation key translate the following English sentences into \GSL{}. 
\begin{enumerate}
	\item If 14-year-olds had the vote, I'd be president. (Evel Knievel)
	\item If Miami beats Cornell today and Penn State defeats Michigan State Miami will win the tournament.
	\item Should senator Ervin run again, he would be a formidable opponent. 
	\item Provided, but only provided, that the French Fleet is sailed forthwith for British harbors, His Majesty's Government give their full consent to an armistice for France. (Churchill, June 1940)
	\item For the tenability of the thesis that mathematics is logic it is not only sufficient but also necessary that all mathematical expressions be capable of definition on the basis solely of logical ones. (W.V.O. Quine)
\end{enumerate}


\notocsubsection{Translations}{Translation Problems} Translate each of the following English sentences into \GQL{} sentences about the model $\IntA$ given in figure \mvref{Trans Int Table}.
\begin{multicols}{2}
	\begin{enumerate}
		\item {All Pacific states that border a mountainous state are coastal.}
		\item {Some Atlantic state and some mountainous state both share a border with a state that is neither.}
		\item {All states are coastal and mountainous if and only if they are Pacific.}
		\item {All Atlantic states smaller than Montana share a border with Rhode Island.}
		\item {Only mountainous Pacific states are coastal.}
		\item {A Pacific state is mountainous.}
		\item {No state is larger than itself.}
		\item {Every non-mountainous state borders a state that is larger.}
		\item {Some Pacific states are mountainous.}
		\item {All Pacific states are mountainous.}
		\item {All Pacific states are larger than all Atlantic states.}
		\item {No Gulf state is mountainous.}
		\item {All Atlantic states are not mountainous.}
		\item {Some Gulf state is larger than all states that border it.}
		\item {Some Gulf state is an Atlantic state.}
		\item {All states that border a Pacific state are mountainous.}
		\item {Any state that is mountainous is larger than Rhode Island.}
		\item {Any state that is mountainous is larger than all Atlantic
			states.}
		\item {If any state is mountainous, California is.}
		\item {If any state is mountainous, it is larger than Rhode Island.}
		\item {Any state that has no bordering states is mountainous.}
		\item {All states that are bigger than all mountainous states are
			coastal.}
		\item {No state is bigger than Montana unless it is coastal.}
	\end{enumerate}
\end{multicols}

\notocsubsection{More Translations}{ex:More Translations}
Translate each of the following English sentences into \GQL{} sentences.
\begin{multicols}{2}
	\begin{enumerate}
		\item {All beavers avoid some kangaroo.}
		\item {All beavers avoid all kangaroos.}
		\item {Some beaver avoids all kangaroos.}
		\item {Every kangaroo is avoided by some beaver.}
		\item {All beavers avoid any kangaroo that frightens them.}
		\item {Some beavers avoid any kangaroo that frightens them.}
		\item {No kangaroo frightens any beaver.}
		\item {No beaver is frightened by any kangaroo.}
		\item {No beaver avoids a kangaroo unless the kangaroo frightens it.}
		\item {Some kangaroo frightens itself.}
		\item {No beaver avoids a kangaroo unless the beaver frightens the kangaroo.}
		\item {Any kangaroo that is frightened of itself is frightened by any beaver.}
		\item {Beavers avoid kangaroos only if they frighten them.}
		\item {Kangaroos that frighten beavers frighten themselves.}
		\item {All kangaroos avoid any kangaroo that avoids them.}
		\item {When a kangaroo frightens a beaver, the beaver avoids it.}
		\item {Beavers only avoid kangaroos.}
		\item {Beavers are frightened of all kangaroos unless they avoid them.}
		\item {Some beavers avoid only kangaroos that frighten them.}
		\item {No beaver that avoids all kangaroos frightens itself.}
	\end{enumerate}
\end{multicols}



%\theendnotes