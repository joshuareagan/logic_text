% *******                   Logic Textbook                         *******
% *******  by Richard Grandy & Michael Barkasi & Joshua Reagan       *******
%Note: \raggedbottom is set right before ch 1
\documentclass[11pt,fleqn,twoside,openright]{report}%leqno   
%\includeonly{ch1,ch4}  
\usepackage{etex}
%\usepackage{leading}
%\leading{15pt} % with caslon, use 14.4 (or 14.7)pt, use 15pt for Palatino and cardo
%\usepackage[firstpage]{draftwatermark}
% % % % % XeTeX and other font stuff:
\usepackage{amsmath,amssymb}
\usepackage{fontspec}
\usepackage{cancel}
\setmainfont[
Ligatures = {Required, Common, Contextual},
Numbers = {Monospaced}, %Proportional, OldStyle
SlantedFont = {* Slanted},
Scale = 1.00,
Mapping=tex-text
]{Latin Modern Roman}
\setsansfont[
Scale=MatchLowercase, %=MatchLowercase, =1.00
BoldFont = Source Sans Pro Semibold,
Mapping=tex-text
]{Source Sans Pro} 
\usepackage{unicode-math}
\setmathfont[math-style=TeX]{Latin Modern Math}
%unicode-math
\usepackage[factor=600]{microtype}
\newcommand{\uppercasespacing}[1]{{\addfontfeature{LetterSpace=2.0}#1}} 
\renewcommand{\textsc}[1]{\uppercasespacing{\MakeUppercase{#1}}}
\newcommand{\textSC}[1]{{\addfontfeatures{Letters={UppercaseSmallCaps,SmallCaps}}#1}}
\newcommand{\osf}[1]{{\addfontfeatures{Numbers={OldStyle}}{#1}}}
\newcommand{\proportional}[1]{{\addfontfeatures{Numbers={Proportional}}{#1}}}
\newcommand{\tabluarnumbers}[1]{{\addfontfeatures{Numbers=Monospaced}#1}}
\newcommand{\set}[1]{\left\{#1\right\}}
\usepackage{realscripts}
   % gives commands: 
      % \textsuperscript
	  % \textsubscript
	  % \textsubsuperscript
	  % \setlength\supsupersep{2pt}
   % But if the font has the right features but not the full set of glyphs, need:
      % \textsuperscript*
	  % \textsubscript*
   % which use the ``fake'' versions
\newcommand{\smallfaction}[1]{{\addfontfeatures{Fractions=On}#1}} 
   % Note: just type in fraction with slash, e.g.
      % \smallfaction{1/2}
\newcommand{\textfaction}[2]{%
    {\addfontfeatures{VerticalPosition={Numerator}}#1}{/}%
	{\addfontfeatures{VerticalPosition={Denominator}}#2}}
% %\newcommand*\rfrac[2]{{}^{#1}\!/_{#2}}
%% \rfrac is a command to allow slanted fractions, here used only sparingly in the many-valued logic section.
\usepackage{xfrac}
\newcommand{\rfrac}[2]{\sfrac{#1}{#2}}
\newcommand{\entails}{\sdtstile{}{}}
% % % % 
\usepackage{turnstile}
%for shorter turnstiles:
\renewcommand{\makever}[4]
{\ifthenelse{\equal{#1}{s}}{\rule[-0.4#3]{#2}{.8#3}}{}
\ifthenelse{\equal{#1}{d}}{\rule[-0.5#3]{#2}{#3}
\hspace{#4}
\rule[-0.5#3]{#2}{#3}}{}
\ifthenelse{\equal{#1}{t}}{\rule[-0.5#3]{#2}{#3}
\hspace{#4}
\rule[-0.5#3]{#2}{#3}
\hspace{#4}
\rule[-0.5#3]{#2}{#3}}{}}%   
\usepackage{multicol}
\usepackage{relsize}
\usepackage{makeidx}
\makeindex   
\usepackage[usenames,dvipsnames]{xcolor}
\definecolor{DarkBlue}{HTML}{000090} % Some of these defined colors may not actually be used below
\definecolor{DDarkBlue}{HTML}{003399}
\definecolor{RRed}{HTML}{812811}
\definecolor{LightGray}{HTML}{444444}               
\usepackage{tcolorbox}
\newcommand{\commentaryspace}{

    \vspace{5mm}
}
\usepackage[font=sf]{caption}
\usepackage{booktabs}
\usepackage{longtable}
\usepackage{varioref}
\usepackage{marginnote}
\usepackage[noload]{qtree}
\usepackage{tikz}
\usepackage{tikz-qtree-compat}
\usepackage[splitrule,bottom]{footmisc} % "hang" is another option to use 
   %\setlength{\footnotemargin}{0.65em}       %Manually set footnote margin, use with "hang"
%These next new commands are to define the arrows used in a few examples in chapter two that point from variables to the quantifiers that bind them.
\newcommand{\NextLineRef}[1][]{%
    \tikz [overlay,remember picture] \draw [->, out=-10, in=20, distance=0.25cm, thick, #1] 
        (-1ex,-0.25ex) to (-5.75ex,-0.75ex);
}
\newcommand{\NextLineRefB}[1][]{%
    \tikz [overlay,remember picture] \draw [->, out=-10, in=20, distance=0.25cm, thick, #1] 
        (-1ex,-0.25ex) to (-13.5ex,-1.5ex);
}
\newcommand{\NextLineRefC}[1][]{%
    \tikz [overlay,remember picture] \draw [->, out=-10, in=20, distance=0.25cm, thick, #1] 
        (-0.5ex,1.75ex) to (-9.1ex,2ex); %(-11.5ex,2ex);
}
\newcommand{\NextLineRefD}[1][]{%
    \tikz [overlay,remember picture] \draw [->, out=-10, in=20, distance=0.25cm, thick, #1] 
        (-1ex,-0.75ex) to (-7.25ex,-0.75ex);
}
\newcommand{\NextLineRefE}[1][]{%
    \tikz [overlay,remember picture] \draw [->, out=-10, in=20, distance=0.25cm, thick, #1] 
        (-1ex,-0.25ex) to (-17ex,-0.5ex); %(-18.5ex,-0.5ex);
}
\newcommand{\NextLineRefF}[1][]{%
    \tikz [overlay,remember picture] \draw [->, out=-10, in=20, distance=0.25cm, thick, #1] 
        (-1ex,1.75ex) to (-14.8ex,2ex); %(-17ex,2ex);
}
\newcommand{\NextLineRefG}[1][]{%
    \tikz [overlay,remember picture] \draw [->, out=-10, in=20, distance=0.25cm, thick, #1] 
        (-1ex,-0.25ex) to (-21ex,-0.5ex);
}
\newcommand{\NextLineRefH}[1][]{%
	\tikz [overlay,remember picture] \draw [->, out=-10, in=20, distance=0.25cm, thick, #1] 
	(-1ex,-0.25ex) to (-15.3ex,-0.5ex); %(-18.5ex,-0.5ex);
}
\newcommand{\NextLineRefI}[1][]{%
	\tikz [overlay,remember picture] \draw [->, out=-10, in=20, distance=0.25cm, thick, #1] 
	(-1ex,-0.25ex) to (-19ex,-0.5ex);
}
\newcommand{\NextLineRefJ}[1][]{%
	\tikz [overlay,remember picture] \draw [->, out=-10, in=20, distance=0.25cm, thick, #1] 
	(-1ex,-0.25ex) to (-6.75ex,-0.75ex);
}
\newcommand{\NextLineRefK}[1][]{%
	\tikz [overlay,remember picture] \draw [->, out=-10, in=20, distance=0.25cm, thick, #1] 
	(-1ex,-0.25ex) to (-4.75ex,-0.75ex);
}

\usepackage{eso-pic}
%The package above and the new commands below define what's needed for the upper right-hand pictures near the chapter headings.
\newcommand\BackgroundPicA{%
\put(0,0){%
\parbox[b][\paperheight]{\paperwidth}{%
\vspace{1.5in}
\hfill
\includegraphics[width=.38\paperwidth,height=\paperheight,%
keepaspectratio]{CLF3.png}\hspace*{1.43in}%
\vfill
}}}
\newcommand\BackgroundPicB{%
\put(0,0){%
\parbox[b][\paperheight]{\paperwidth}{%
\vspace{1.5in}
\hfill
\includegraphics[width=.38\paperwidth,height=\paperheight,%
keepaspectratio]{fj2.png}\hspace*{1.43in}%
\vfill
}}}
\newcommand\BackgroundPicC{%
\put(0,0){%
\parbox[b][\paperheight]{\paperwidth}{%
\vspace{1.5in}
\hfill
\includegraphics[width=.38\paperwidth,height=\paperheight,%
keepaspectratio]{Tur.png}\hspace*{1.43in}%
\vfill
}}}
\usepackage{wallpaper}
\usepackage[inline]{enumitem}  % for control over lists
\labelformat{enumi}{#1} %{(#1)}
%These commands define the custom running lists/counters used for many of the examples.
\newenvironment{smenumerate}{%
\begin{enumerate}[series=m, itemsep=0em]% 
}{%
\end{enumerate}%
}
\newenvironment{menumerate}{%
\begin{enumerate}[resume*=m]%,start=1
}{%
\end{enumerate}%
}
\newenvironment{RESTARTmenumerate}{%
\begin{enumerate}[resume*=m,start=1]%
}{%
\end{enumerate}%
}
% For self-contained lists, like clauses in a definition
\newenvironment{cenumerate}{%
\begin{enumerate}[label=(\arabic*), leftmargin=1.85\parindent,
labelindent=.35\parindent, labelsep=*, itemsep=0pt]%,start=1, parsep=0pt
}{%
\end{enumerate}%
}
%These define some things for the in-text citations
\renewcommand{\reftextfaceafter}{on the \reftextvario{next}{next} page}%
\renewcommand{\reftextfacebefore}{on the \reftextvario{previous}{previous} page}%
\renewcommand{\reftextbefore}{on the \reftextvario{previous}{previous} page}%
%\renewcommand{\reftextfaraway}[1]{p. \pageref{#1}}
%Some custom commands for intra-text hyperreferences
\newcommand{\mvref}[1]{\ref{#1} (\vpageref*{#1})}
\newcommand{\pmvref}[1]{\ref{#1}, \vpageref{#1}}
\newcommand{\pncmvref}[1]{\ref{#1} \vpageref{#1}}
%These control the header
\usepackage{fancyhdr}
\setlength{\headheight}{15.2pt}
\renewcommand{\headrulewidth}{0pt} 
\renewcommand{\footrulewidth}{0pt}
\newcommand{\CAPS}[1]{\uppercasespacing{\MakeUppercase{#1}}}% % %
% Any changes to the command \CAPS must manually be made to the title of the section "DNF and the TFE Replacement Theorem"
% was: {\textls[20]{#1}} % % % This controls the (extra) spacing between strings of capital letters, like TFT and QT
%\expandafter\index\expandafter{\CAPS}
%some custom commands (used to get these from the local "philosophy" package): 
\newcommand{\df}[1]{\textit{#1}\index{#1|textbf}}
\newcommand{\underdf}[2]{\emph{#1}\index{#2!#1|textbf}}
\newcommand{\nidf}[1]{\textit{#1}}
\newcommand{\mention}[1]{`#1'}
\newcommand{\Iff}{iff }
\newcommand{\IFF}{if, and only if, }
%Makes it easier to change some of the fonts on the definition/theorem/example enivornments: 
\newcommand{\descriptionfont}{\bfseries\sffamily} 
\newcommand{\theoremfont}{\bfseries\sffamily}
\newcommand{\definitionfont}{\bfseries\sffamily}
\newcommand{\examplefont}{\bfseries\sffamily}
\newcommand{\prooffont}{\bfseries\sffamily}
\setlist[description]{font=\descriptionfont{}}  % this command is part of enumitem; control look of description label.
\newcommand{\idf}[1]{\emph{#1}\index{#1|textbf}} % informal, or in-line, definition
\newcommand{\seealsoidf}[2]{\emph{#1}\index{#1|seealso{#2}}}
\newcommand{\underidf}[2]{\emph{#1}\index{#2!#1|textbf}}
\newcommand{\niidf}[1]{\textit{#1}}
\newcommand{\Rule}[1]{\textsl{#1}}  % for names of inference rules, was using \textsc{#1}
\newcommand{\sq}[1]{{``}#1{''}}
\newcommand{\distinction}[2]{#1/#2}
\newcommand{\q}[1]{{``}#1{''}}
%The counter for the definition/theorem/example environments
\newcounter{DefThm}[chapter]
\renewcommand\theDefThm{\thechapter.\arabic{DefThm}} % \arabic{chapter}.\arabic{DefThm}
\newcommand{\DC}[1]{\refstepcounter{DefThm}(Definition \arabic{DefThm}\label{#1})}
\newcommand{\TC}[1]{\refstepcounter{DefThm}(Theorem \arabic{DefThm}\label{#1})}
\newcommand{\EC}[1]{\refstepcounter{DefThm}(Example \arabic{DefThm}\label{#1})}
\newcommand{\npDC}[1]{\refstepcounter{DefThm}\arabic{DefThm}\label{#1}}
\newcommand{\npTC}[1]{\refstepcounter{DefThm}\arabic{DefThm}\label{#1}}
\newcommand{\npEC}[1]{\refstepcounter{DefThm}\arabic{DefThm}\label{#1}}
\newcommand{\LnpDC}[1]{\refstepcounter{DefThm}{\definitionfont{}Definition \theDefThm.\label{#1}}}
\newcommand{\LnpTC}[1]{\refstepcounter{DefThm}{\theoremfont{}Theorem \theDefThm.\label{#1}}}
\newcommand{\LnpEC}[1]{\refstepcounter{DefThm}{\examplefont{}Example \theDefThm.\label{#1}}}
% This package for setting page geometry and layout
\usepackage[paper=letterpaper,heightrounded=true,inner=1.43in,outer=1.43in,top=1.5in,bottom=1.5in]{geometry} %useful options not loaded: showcrop, showframe, marginparwidth=3.5cm, marginparsep=-3.5cm,
% Some other new environments used below:
\newenvironment{major}[1]
{

\medskip
\noindent{{#1}:}}%
{

\medskip}
\newenvironment{majorIL}[1]
{

\medskip
\noindent{{#1}:}}%
{ }
\newenvironment{majorILnc}[1]
{

\medskip
\noindent{\sffamily\bfseries{}#1}}%
{

\medskip}
\newenvironment{PROOF}
{

\medskip
\noindent{\prooffont{}Proof:}}%
{\hfill \ensuremath{\blacksquare}

\medskip}
\newenvironment{PROOFOF}[1]
{

\medskip
\noindent{\prooffont{}Proof of #1:}}%
{\hfill \ensuremath{\blacksquare}

\medskip}
\newenvironment{SUBPROOF}
{

\medskip
\noindent{\prooffont{}Subproof:}}%
{\hfill \ensuremath{\blacksquare} %\;\blacksquare}

\medskip}
\newenvironment{THEOREM}[1]
{

\medskip
\noindent{#1}}% \itshape{}
{ %\hfill \ensuremath{\blacktriangleleft}

\medskip}
\newenvironment{commentary}{\begin{tcolorbox}}{\end{tcolorbox}}
\newenvironment{minor}[1]
{\emph{{#1}:}}%
{\par}
% Section Heading Controls:
\usepackage[toctitles,nobottomtitles*]{titlesec}
\renewcommand{\bottomtitlespace}{.05\textheight}
\titleformat{\chapter}[display]{\huge\sffamily\bfseries}{\huge{}\sffamily\bfseries\color{DDarkBlue}Chapter \thechapter}{.1em}{\vspace{.075em}}[]
\titlespacing{\chapter}{0pt}{-.27in}{1.27in}
\titleformat{\section}[hang]{\Large\bf\sffamily\color{DDarkBlue}}{\thesection}{1em}{}[{\color{Black}\titlerule}]
\titlespacing*{\section}{0pt}{20pt}{0.2cm}
\titleformat{\subsection}[hang]{\large\bf\sffamily}{\thesubsection}{1em}{}
\titlespacing*{\subsection}{0pt}{10pt}{0.2cm}
\titleformat{\subsubsection}[runin]{\bf}{\thesubsubsection}{1em}{}
\titlespacing{\subsubsection}{0pt}{1em}{1em}
%\renewcommand\thesection{\arabic{section}}%   %this command redefines the representation of the section counter so it doesn't display the chapter number before the section number.
  \newcommand{\notocsubsection}[2]{%
    \refstepcounter{subsection}%
    \subsection*{\thesubsection \quad #1}\label{#2}}% 
% Table of Content Controls:
\usepackage{tocloft}%
\setcounter{tocdepth}{1}%                                         %Table of Contents depth
\renewcommand{\cfttoctitlefont}{\huge\sffamily\bfseries}% %make sure this matches chapter heading space
\renewcommand{\cftaftertoctitle}{}
\renewcommand{\cftchapfont}{\large\bfseries\sffamily\selectfont{}}
\renewcommand{\cftchappagefont}{\large\bfseries\sffamily\selectfont{}}
\renewcommand{\cftsecfont}{\normalsize}
\renewcommand{\cftsecpagefont}{\normalsize}
\renewcommand{\cftsubsecfont}{\normalsize}
\renewcommand{\cftsubsecpagefont}{\normalsize}
\setlength{\cftaftertoctitleskip}{1in}
\setlength{\cftchapindent}{0in}
\setlength{\cftsecindent}{0in} 
\setlength{\cftsubsecindent}{.15in} 
\setlength{\cftbeforesecskip}{0in} 
%this command is for doublepage clears with a blank page:
\let\origdoublepage\cleardoublepage
\newcommand{\clearemptydoublepage}{%
  \clearpage
  {\pagestyle{empty}\origdoublepage}%
}
%this might cause problems, but should be okay:
\let\cleardoublepage\clearemptydoublepage
% This is the start of the big "logic2" style file:
\usepackage[
%propositional constants
A=A,
B=B,
C=C,
D=D,
E=E,
F=F,
G=G,
H=H,
I=I,
J=J,
K=K,
L=L,
M=M,
N=N,
O=O,
P=P,
Q=Q,
R=R,
S=S,
T=T,
U=U,
V=V,
W=W,
X=X,
Y=Y,
Z=Z,
%perdicate letters%%%%%%%%%%%%%%
predparL=,
predparR=,
%PredArgComma=2, %1=comma, 2=no comma
Ap=A, %\text{A},
Bp=B, %\text{B},
Cp=C, %\text{C},
Dp=D, %\text{D},
Ep=E, %\text{E},
Fp=F, %\text{F},
Gp=G, %\text{G},
Hp=H, %\text{H},
Ip=I, %\text{I},
Jp=J, %\text{J},
Kp=K, %\text{K},
Lp=L, %\text{L},
Mp=M, %\text{M},
Np=N, %\text{N},
Op=O, %\text{O},
Pp=P, %\text{P},
Qp=Q, %\text{Q},
Rp=R, %\text{R},
Sp=S, %\text{S},
Tp=T, %\text{T},
Up=U, %\text{U},
Vp=V, %\text{V},
Wp=W, %\text{W},
Xp=X, %\text{X},
Yp=Y, %\text{Y},
Zp=Z, %\text{Z},
%object language variables%%%%%%%%%%%
w=w,
x=x,
y=y,
z=z,
%metalanguage constants for formula
alpha=\alpha,   %use macro \ALPHA, \BETA, etc
beta=\beta,
chi=\chi,
delta=\delta,
epsilon=\epsilon,
phi=\Phi,
gamma=\gamma,
eta=\eta,
iota=\iota,
kappa=\kappa,
lambda=\lambda,
mu=\mu,
nu=\nu,
Greeko=o,
pi=\pi,
theta=\Theta,
rho=\rho,
sigma=\sigma,
tau=\tau,
upsilon=\upsilon,
omega=\omega,
xi=\xi,
psi=\Psi,
zeta=\zeta,
capalpha=A,  %use macro \CAPALPHA, \CAPBETA, etc
capbeta=B,
capchi=X,
capdelta=\Delta,
capepsilon=E,
capphi=\phi,
capgamma=\Gamma,
capeta=H,
capiota=I,
capkappa=K,
caplambda=\Lambda,
capmu=M,
capnu=N,
capGreeko=O,
cappi=\Pi,
captheta=\theta,
caprho=P,
capsigma=\Sigma,
captau=T,
capupsilon=\Upsilon,
capomega=\Omega,
capxi=\Xi,
cappsi=\psi,
capzeta=Z,
%connectives  use macros \disjunction,\conjunction,\negation,\horseshoe, and \triplebar (plus
quantifierparL={},%  the proper number of arguments; affix as prefixs par, bpar, or cpar (i.e.
quantifierparR={},%  \pardisjunction to wrap in parentheses.
manybrackets=1, %1=yes, 2=no
IntA={I},
True=\text{true},
False=\text{false}
]{logic2}
%\usepackage{eucal} %changes \mathcal, so interpretations look better
%\usepackage{eufrak} %changes \mathfrak, so stuff looks better (TeX says redundant if amsfonts used?)
\newcommand{\constant}[1]{\mathrm{#1}} % for object-lang constants
\newcommand{\variable}[1]{#1} % for object-lang variables
\newcommand{\integer}[1]{#1}
%\DeclareMathAlphabet{\mathpzc}{OT1}{pzc}{m}{it}
\newcommand{\world}[1]{\mathit{#1}}
\newcommand{\TrueB}{\mathsf{T}}
\newcommand{\FalseB}{\mathsf{F}}
\renewcommand{\IntA}{\mathfrak{m}}
\newcommand{\m}{\mathfrak{m}}
\newcommand{\tfe}{\CAPS{tfe} }
\newcommand{\lhs}{\CAPS{lhs} }
\newcommand{\rhs}{\CAPS{rhs} }
\newcommand{\LP}{\text{LP}}
\newcommand{\HHH}{\text{AR}}
\newcommand{\ORD}[1]{\text{ORD}#1}
\newcommand{\Derivation}[1]{\mathit{#1}}
\newcommand{\e}{\text{e}}
\newcommand{\As}[2]{\mathfrak{m^{\mathrm{#1}}_{\mathrm{#2}}}}
\newcommand{\model}[2]{\mathfrak{m^{\mathrm{#1}}_{\mathrm{#2}}}}
%\newcommand{\As}{\mathfrak{a}}
\newcommand{\PosIntEM}[2]{Pos Int^{#1+}_{\mathit{#2}}}
\newcommand{\BWEDGE}{\mathlarger{\mathlarger{\WEDGE}}}
%\renewcommand{\baselinestretch}{1.5}
\newcommand{\BOX}{\ensuremath \raisebox{-.95pt}{$\Box\,$}} %.75pt
\newcommand{\DIAMOND}{\ensuremath \raisebox{-.6pt}{$\Diamond\,$}} %.75pt
\newcommand{\GSL}{\textsc{sl}} %{GSL} %
\newcommand{\GQL}{\textsc{ql}} %{GQL} %
\newcommand{\SL}{\textsc{sl}} %{SL} %
\newcommand{\QL}{\textsc{ql}} %{QL} %
\newcommand{\PL}{\textsc{pl}} %{QL} %
\newcommand{\MGSL}{\textsc{msl}} %
\newcommand{\GQLI}{\textsc{qli}} %
\newcommand{\GSD}{\textsc{sd}} %{GSD} %
\newcommand{\GSDP}{\textsc{sd}$^+$} %{GSD} %
\newcommand{\GQD}{\textsc{qd}} %{GQD}%
\newcommand{\GQDP}{\textsc{qd}$^+$} %{GQD} %
\newcommand{\GQDPP}{\textsc{qd}$^+_{m}$} %
\newcommand{\GQDI}{\textsc{qdi}} %{GQDI} %
\newcommand{\GQDIP}{\textsc{qdi}$^+$} %{GQDI} %
\newcommand{\SF}{$S5$}
\newcommand{\SFP}{$S5^+$}
\newcommand{\SO}{$S1$}
\newcommand{\Language}[1]{#1}
\newcommand{\DerivationSystem}[1]{#1}
%\renewcommand{\underline}[1]{{\color{Green}#1}}
%\usepackage{endnotes}
%\let\footnote=\endnote
\renewcommand{\v}{\variable{v}}
\renewcommand{\u}{\variable{u}}
\renewcommand{\a}{\constant{a}}
\renewcommand{\b}{\constant{b}}
\renewcommand{\c}{\constant{c}}
\renewcommand{\d}{\constant{d}}
% More Bib stuff:
\newcommand{\ICP}{;} %should match "\setcitestyle{citesep={;}}"
\usepackage[round]{natbib}
\setcitestyle{notesep={: }}% %punctuation seperating year from pages
\setcitestyle{aysep={}}% %punctuation seperating author and year
\setcitestyle{yysep={,}}% %punctuation seperating years for multiple
                               %%% works from same author
\setcitestyle{citesep={\ICP{}}}% %punctuation between citations
\def\bibfont{\normalsize }
\renewcommand{\bibname}{Works Cited}
%Examples:
%\citet{refname}   author, (year)
%\citep{refname}  (author, year)
%\citep[p.~5]{refname}   (author, year, p. 5)
%\citeyearpar{refname}  (year)
%\citetext{\citealp[24--26]{Burge1986}; \citeyear[68,~98--99]{Burge2010}}  
                        %(Burge, 1986: 24--25; 2010: 68, 98--99)
\usepackage[%
unicode,pdfencoding=auto,
    pagebackref=false,%
    linktoc=page,%
    pdfnewwindow=true,%               % links in new window
    colorlinks=true,%                 % false: boxed links; true: colored links
    linkcolor=Blue,%                   % color of internal links
    citecolor=Blue,%                  % color of links to bibliography
    filecolor=Black,%magenta,               % color of file links
    urlcolor=Black%cyan                    % color of external links
%    unicode,
%    pdfencoding=auto
]{hyperref}%
\begin{document}
\pagenumbering{roman} % Roman numerals
\setcounter{page}{1}%
\pagestyle{empty}
%\restoregeometry

\vspace*{.8in}
%\noindent{\makebox[4.75in][s]{\Huge\sffamily{}M a t h e m a t i c a l L o g i c}}
\noindent{\sffamily\fontsize{35}{35}\selectfont{}\color{DDarkBlue}{\color{Gray}Mathematical}Logic} %\textls[30]

%\medskip
%\noindent{\LARGE{}\fontencoding{T1}\sffamily\fontseries{b}\fontshape{it}\selectfont{}\textls[30]{***A subtitle would go here, if we had one***}}

\vspace{1.25in}
\noindent{}{\LARGE\sffamily\selectfont{}Richard Grandy}

\vspace*{12pt}
\noindent{}{\LARGE\sffamily\selectfont{}Michael Barkasi}

\vspace*{12pt}
\noindent{}{\LARGE\sffamily\selectfont{}Joshua Reagan}

\vfill
\noindent{\LARGE{}\sffamily\selectfont{}{Edition 1.0.0}} % could use \textls[30] for effect

\newpage
\pagestyle{empty}

\vspace*{.25in}
%\begin{center} % We can center this material, but my preference is to keep it ragged-right (mb)
{\small 
\noindent{}Copyright \textcopyright{} 2025 Richard Grandy, Michael Barkasi, and Joshua Reagan

\bigskip
\noindent{}This work is licensed under the Creative Commons Attribution-NonCommercial-NoDerivs 3.0 Unported License. 

\noindent{}To view a copy of the license, visit http://creativecommons.org/licenses/by-nc-nd/3.0/.

\bigskip
\noindent{}You are free {to Share} -- to copy, distribute and transmit the work, 

\noindent{}under the following conditions: 

\noindent{}{Attribution} -- You must attribute the work to the authors (but not in any way that suggests that they endorse you or your use of the work); 

\noindent{}{Noncommercial} -- You may not use this work for commercial purposes;

\noindent{}{No Derivative Works} -- You may not alter, transform, or build upon this work.

\vfill
\noindent{}This book is set in Latin Modern (for text and math fonts, released under the GUST Font License) and Adobe Source Sans Pro (for headings, released by Adobe under the OFL). 

\bigskip
\noindent{}Edition 1.0.0, Dec 23, 2025
}
%\end{center}

\newpage%
\clearemptydoublepage
\pagestyle{plain}
\tableofcontents

\newpage%
\clearemptydoublepage
\pagestyle{plain}

\chapter*{Preface}
\addcontentsline{toc}{chapter}{Preface}

\section*{What is mathematical logic?}

Logic is the study of logical consequence and logical truth.
Logical consequence is a relation between a set of sentences $\Delta$ (sometimes called \mention{premises}) and a sentence $\PHI$ (sometimes called a \mention{conclusion}) such that $\PHI$ \mention{follows from} $\Delta$.
For example, \mention{Fran is a Wallaby} is a logical consequence of \mention{If Fran was born last year, then Fran is a Wallaby} and \mention{Fran was born last year}.
A sentence $\PHI$ is a logical truth if its truth can be known solely in virtue of logic.
For example, \mention{If Fran was born last year, then Fran was born last year} seems true solely as a matter of logic.

These definitions are insufficiently precise for the needs of the scrupulous philosopher or mathematician.
In \emph{mathematical} logic\index{mathematical logic} we use mathematical methods to study logical consequence and logical truth. 
In this textbook we define abstract, formal languages which have as sentences strings of basic symbols.
For sentences of these formal languages, we rigorously define (i) the relation of entailment and (ii) formal derivation systems. 
Both entailment and formal derivations provide a mathematical approximation of logical consequence.
We also show how logical truth can be defined in terms of logical consequence.
In rough terms, a logical truth can be understood as a sentence that follows from the empty set of sentences.
The goal of this textbook is to give the reader an introduction to these mathematical definitions and methods.  


\section*{What material is covered here?}
The main subject of this text is classical first-order logic (or as we call it, quantificational logic), with sentential logic (sometimes called \mention{propositional} logic) introduced first. 
We focus mostly on a quantificational logic that doesn't include identity and function symbols.  
Our conjunction and disjunction operators are of arbitrary (but finite) arity because that more closely mirrors typical English use.  
Many-valued logic, modal logic, and the identity operator are briefly introduced in the last chapter.  The main purpose there is to show how sentential and quantificational logic can be extended to treat other logical operators and to cover a wider range of applications. 

The main goal of the text is to prove that for quantificational logic, whether one uses formal derivations or proofs of logical entailment, one always gets the same results. 
There are various ways to define the semantics for quantificational logic.\footnote{For example, game-theoretic semantics and substitutional quantifiers; see \citealp{Dunn1968}, \citealp{Stevenson1973}, \citealp{Hintikka1996}, \citealp{Leblanc2001}, \citealp{Westerstahl2001}, \citealp{Jacquette2002}.}  We use Benson Mates' modification of the standard Tarskian model-theoretic semantics.\footnote{\citealp{Mates1972}} 
The derivation system in this text is a natural deduction style system.\footnote{This style of system was first developed by Gentzen in \citeyearpar{Gentzen1934}, \citealt[26]{Hodges2001}. If you are curious, derivation systems roughly divide into four types: natural deduction, Hilbert-style axiomatic systems, Gentzen-style sequent calculi, and proof tableaux; see \citealt[24]{Hodges2001b} for an overview.} 
In the system used here only conditional elimination can discharge assumptions; this tends to make the other rules simpler. 
We use Fitch-style derivations,\footnote{\citealt{Fitch1952}} but following Donald Kalish and Richard Montague\footnote{\citealp{Kalish1964}} we discharge assumptions by drawing a box around each completed subproof.
Rather than using the more commonly found Henkin-style proofs\footnote{\citealt{Henkin1949}. For one of many modern treatments, see \citealp[ch.~11.4]{Bergmann2003}.} to prove strong completeness, we prove it constructively. We believe this makes the connection between syntactic and semantic methods more intuitive.  In Chapter \ref{completenesschapter} we provide an algorithm (``The Method'') which, for any sentence $\PHI$ entailed by some set of sentences $\Delta$, produces a derivation of $\PHI$ from $\Delta$. 
The proof used here owes much to Willard Quine's completeness proof.\footnote{\citealp{Quine1982}}

\section*{How can I effectively use this book?}

Reading mathematical texts is difficult, especially for those not accustomed to the style of prose typical of the genre.
Following a mathematical argument requires significant cognitive effort and concentration, so prepare yourself.
The most fruitful way to approach this text is with pencil and paper. 
As you read, work out the examples, write the proofs, and find counterexamples on your own.
There's no reason to stick to the text---look for your own proofs and work out new examples. 
Effective use of the text requires active engagement. 

We have tried our best to prepare a helpful and clear text.
A large stock of examples is included and most intratext references include a page number.
Many of the proofs in the text, especially the recursive proofs, leave steps for the reader to complete.
These are always steps (or whole proofs) for which the reader will have seen something similar before.  
Many of the examples have been worked through in detail.
Our hope is that the examples show clearly all the ``moving parts''---that they show how the answers to problems follow directly from the definitions and theorems. 
You probably won't learn much logic by reading the text once or twice, so work through the examples and theorems. 

\section*{Issues in the Philosophy of Logic}\label{issuesinphillogic}
 
In mathematical logic, one begins with rigorous mathematical definitions of ``sentence'', truth, logical truth, entailment, and derivation and then proves theorems about them.
A primary concern in logic is how to make sense of reasoning in natural languages and the notion of logical consequence. 
How the two---the rigorously defined mathematical concepts like entailment and ``pretheoretic'' concepts like logical consequence---relate is a natural question, and one that's received a great deal of attention by philosophers.\footnote{See Blanchette's \citeyearpar{Blanchette2001} and Shapiro's \citeyearpar{Shapiro2005c} for an introduction.}

Because the aim of this text is to introduce mathematical logic and not philosophical logic, we largely set these questions aside. 
In some places, usually for the sake of exposition or motivation, we make claims that raise issues squarely in the domain of philosophy of logic.\footnote{For example, we say in section \ref{Formal Languages} that formal languages can be thought of as models of natural languages; and much of the discussion on identity in section \ref{Sec:Quantifier Logic with Identity} raises issues as well.} 
Given the aims and scope of this text, however, we aren't able to point out all the relevant philosophical concerns, let alone do much to motivate our potentially contentious claims.

\section*{History of Logic}

Logic has a long history going back to Aristotle. 
Aristotle claims that he was the first to work out any systematic treatment of logical consequence (\emph{Sophistical Refutations}, 34, 183b34--36; citation from \citealp[27]{Smith1995}), and as Robin Smith notes, ``we have no reason to dispute this'' \citeyearpar[27]{Smith1995}.
Aristotle's work gave rise to a tradition of work in logic that extends through the Stoics, Byzantine commentators, early Islamic philosophy, and European Scholasticism. 
Modern mathematical logic has roots in this Aristotelian tradition as well (though it has roots elsewhere too, e.g. foundational work in mathematics). 
However, the Aristotelian tradition is beyond the expertise the authors, and so we have not attempted to cite relevant work from it. 

\newpage%
\clearemptydoublepage
%\thispagestyle{empty}

\chapter*{Acknowledgments}
\addcontentsline{toc}{chapter}{Acknowledgments}

\noindent{}This book began as a series of a lecture handouts written for the yearly mathematical logic class in the philosophy department at Rice University. 
The handouts were primarily written by Richard Grandy from 2007 to 2011, with both Stan Husi and Jacob Mills (TAs for the class during that time) contributing. 
Michael Barkasi began the process of typesetting these handouts in \LaTeX{} for the fall of 2012 (at which time he was the TA).
By next summer he finished this process, and over the next year he compiled and reorganized the handouts into a draft textbook, adding in material as needed. 
Since then Joshua Reagan (TA from 2014-17) and Richard Grandy have revised the text and added new material.

As with most other textbooks, no theorems are original and their presentation is, for the most part, what you'll find elsewhere. 
The proofs for all the theorems---along with the actual writing and presentation---have been done from scratch by one or more of the authors; 
but the general methods and approaches used are just those the authors learned from others. 
Thus this textbook owes just as much to previous work in logic as most other modern textbooks. It passes on to students a body of results and a general conceptual framework developed by many logicians over the years. 
It does so in roughly the style and presentation in which those results were passed to the authors themselves, with some changes they think are improvements. 
We note these as they come up. 

A few more specific acknowledgments are called for. 
At the time the first author was preparing the original handouts he was using Merrie Bergmann, James Moor, and Jack Nelson's textbook \emph{The Logic Book} \citeyearpar{Bergmann2003}; hence some of the terminology and organization in this book reflects that. 
Benson Mates' classic \emph{Elementary Logic} \citeyearpar{Mates1972}, the previous text, and Church's \emph{Introduction to Elementary Logic} \citeyearpar{Church1956} also influenced the first author's approach. 
The second author has relied heavily on Wilfred Hodges's rich survey article ``Elementary Predicate Logic'' \citeyearpar{Hodges2001} for historical citations. 
The name ``Dragnet'' for theorem \ref{The Dragnet Theorem} comes from Michael Smith, another former TA for the first author. 
The problems in exercise \mvref{ex:English to GSL Translations 2} comes from Howard Pospesel \& David Marans, \emph{Arguments: Deductive Logic Exercises} \citeyearpar{Pospesel1978}, which Pospesel and Marans have generously released to the public.   


\newpage
\clearemptydoublepage
\pagestyle{headings}
\pagestyle{myheadings}
\markright{ }

% % % % % % % % Raggedbottom!!! HERE
\raggedbottom
% % % % % % % %
%Controls for fancy headers, if used.
\pagestyle{fancy}
%\renewcommand{\chaptermark}[1]{ \markboth{#1}{} }                       % can only redefine 2 of the 3 "marks"
\renewcommand{\sectionmark}[1]{ \markboth{}{\thesection. #1} }
\renewcommand{\subsectionmark}[1]{ \markboth{}{\thesubsection. #1} }
\fancyhf{}
\chead{\sffamily\rightmark}
\fancyhead[LE]{\sffamily\thepage}
\fancyhead[RO]{\sffamily\thepage}
%\rhead{\thepage}

\pagenumbering{arabic} % Roman numerals
\setcounter{page}{1}


\addtocontents{toc}{\protect\thispagestyle{empty}}
%%%%%%%%%%%%%%%%%%%%%%%%%%%%%%%%%%%%%%%%%%%%%%%%%%
\chapter{Introduction}\label{introduction}
%%%%%%%%%%%%%%%%%%%%%%%%%%%%%%%%%%%%%%%%%%%%%%%%%%
%\AddToShipoutPicture*{\BackgroundPicA}

%%%%%%%%%%%%%%%%%%%%%%%%%%%%%%%%%%%%%%%%%%%%%%%%%%
\section{What is Logic?}\label{What Is Logic}
%%%%%%%%%%%%%%%%%%%%%%%%%%%%%%%%%%%%%%%%%%%%%%%%%%
\subsection{Rational Creatures}\label{Rational Creatures}

Humans are rational creatures. 
As such, we reason about our circumstances and the world in general so we may understand them better. 
An improved understanding of the world helps us make informed decisions, which (ideally) tend to bring about preferable outcomes. 
If you find yourself lost in a labyrinth and hunted by a hungry minotaur, clever reasoning could help you escape. 
Poor reasoning could get you eaten! 

At least some of our reasoning is \emph{discursive}. 
Discursive reasoning is an iterative process: start with a set of initial claims and then infer a result from them, perhaps repeating the process until a certain conclusion is reached. 
Such a chain of inferences can be written down or otherwise recorded as an argument.

Knowledge of logic helps us exploit a particularly reliable kind of discursive reasoning. 
It enables us to judge an argument from any source according to independent, principled criteria.
Not only can we assess the strength of arguments presented to us; we can construct rigorous arguments of our own to share with others. 
Rationality thus has an important social aspect. 
By sharing arguments with each other we can, to some extent, coordinate beliefs and behaviors, and thereby complete tasks beyond the ability of any one person. 

The social role of argumentation provides a clue about the subject matter of logic. 
Arguments are shared by way of language. 
One can give an argument by writing it on paper, typing it in an e-mail, stating it in a speech, etc. 
Once inscribed or encoded in one linguistic medium or another, it can then be assessed by others. 
Accordingly, logicians must attend to certain public features of language. 

%%%%%%%%%%%%%%%%%%%%%%%%%%%%%%%%%%%%%%%%%%%%%%%%%%
\subsection{Logical Consequence}\label{Logical Consequence}

Logic is the study of \emph{logical consequence} and \emph{logical truth}. 
When we say that one sentence `logically follows' from another, we mean that the first sentence is a logical consequence of the other. 
But how can we identify when some claim is a logical consequence of another? 
As suggested previously, we must attend to certain features of the language in which the argument is given. 

One sentence $\CAPPSI$ is a logical consequence of another sentence $\CAPPHI$ if and only if the truth of $\CAPPHI$ guarantees, in virtue of its logical structure, the truth of $\CAPPSI$.

Three points are worth making about this provisional definition. First, don't be afraid of the Greek letters: they're just variables for sentences. Second, the phrase `if and only if' is one we use throughout the text, sometimes abbreviated as `iff'. `$\CAPPHI$ if and only if $\CAPPSI$' is equivalent to `If $\CAPPHI$ then $\CAPPSI$ and if $\CAPPSI$ then $\CAPPHI$'. Third, a solid understanding of this definition depends on having a clear notion of \emph{logical structure}. One of the central goals of this textbook is to provide such a notion.

Consider the following two sentences:
\begin{smenumerate}
	\item George Washington was in the Continental Army and Nathanael Green was in the Continental Army.
	\item Nathanael Green was in the Continental Army.
\end{smenumerate}
The second sentence is a logical consequence of the first. If the first is true the second must be too. The next two sentences share the same structure as the last two:
\begin{menumerate}
	\item Benjamin Franklin was a delegate to the Continental Congress and Thomas Jefferson was a delegate to the Continental Congress.
	\item Thomas Jefferson was a delegate to the Continental Congress.
\end{menumerate}
As above, the second sentence is a logical consequence of the first. We can show the common logical structure of the preceding pairs of sentences using a \emph{schema}:
\begin{menumerate}
	\item $\CAPPHI$ and $\CAPPSI$.
	\item $\CAPPSI$.
\end{menumerate}
We have replaced the non-logical content of each sentence with Greek letters and kept the logical word `and'. The Greek letters serve as variables which stand for clauses of the English language. Any substitution of appropriate English clauses for the letters $\CAPPHI$ and $\CAPPSI$ in the above schema generates a pair of sentences in which the latter follows from the former.\footnote{We say more about what counts as an appropriate substitution later in the chapter.}

A sentence can also be the logical consequence of a set of sentences. For example, given the sentences,
\begin{menumerate}
	\item The sun is shining.
	\item\label{Birds} The birds are chirping.
\end{menumerate}
the following is a logical consequence:
\begin{menumerate}
	\item The sun is shining and the birds are chirping.
\end{menumerate}
We can represent the logical structure of these sentences in schematic form:
\begin{menumerate}
	\item $\CAPPHI$.
	\item $\CAPPSI$.
	\item $\CAPPHI$ and $\CAPPSI$.
\end{menumerate}
Any substitution of appropriate English clauses for the letters $\CAPPHI$ and $\CAPPSI$ in this schema generates three sentences in which the last follows logically from the first two.

%%%%%%%%%%%%%%%%%%%%%%%%%%%%%%%%%%%%%%%%%%%%%%%%%%
\subsection{Logical Truth}\label{Logical Truth}

A sentence is a \idf{logical truth} if and only if the logical structure of the sentence guarantees its truth.  For example,
\begin{menumerate}
	\item\label{Sun} If the sun is shining then the sun is shining.
\end{menumerate}
This sentence must be true, regardless of whether the sun is shining. Contrast that with sentence \ref{Birds}; if no birds exist, then \ref{Birds} can't be true.  There is nothing special about the non-logical content of \ref{Sun}.  We could replace `the sun is shining' with `it's raining outside' and get another logical truth:
\begin{menumerate}
	\item If it's raining outside then it's raining outside.
\end{menumerate}
The schema of these two logical truths is:
\begin{menumerate}
	\item If $\CAPPHI$ then $\CAPPHI$.
\end{menumerate}
The non-logical content is replaced with $\CAPPHI$ and the words `if\dots then' are retained. Any substitution of an appropriate clause of English for $\CAPPHI$ generates a logical truth.
More generally, logically true sentences have a structure such that, whatever appropriate clause(s) we substitute for the non-logical parts, the result is also a logical truth. 

To construct general schemas from particular examples, as we did above, requires distinguishing between the logical structure and the non-logical content of a sentence. 
We take for granted that certain ``logical'' words play a special role in constituting the logical structure of an English sentence---e.g., words such as `and', `or', `if\ldots  then', `not', `all', `some', among many others.

Logical consequence and logical truth are closely related concepts. In fact, each can be defined in terms of the other, though we won't specify the connection until we give a mathematically precise definition of each. Unfortunately, natural languages such as English have features that make these concepts difficult (or even impossible) to define adequately. 

%%%%%%%%%%%%%%%%%%%%%%%%%%%%%%%%%%%%%%%%%%%%%%%%%%
\section{Natural and Formal Languages}\label{Preliminaries}
%%%%%%%%%%%%%%%%%%%%%%%%%%%%%%%%%%%%%%%%%%%%%%%%%%
\subsection{Natural Languages}\label{Natural Languages}

Calling English a \niidf{natural}\index{language!natural} language is only meant to indicate that it developed gradually and informally over time, and wasn't the product of, say, scholars officially defining the grammar and vocabulary from scratch. 
English, Greek, German, Russian, Spanish, Mandarin, and Hindi are all natural in this sense. 

Let's examine some of the obstacles to achieving mathematical precision in natural language. 
First, it is indeterminate which collections of words and letters are genuine sentences. 
The following strings are, perhaps, neither clearly sentences nor clearly not sentences of English:\footnote{Sentences \ref{Chom1} and \ref{Chom2} are from \citealp{Chomsky1957}.}
\begin{menumerate}
\item\label{Chom1} Colorless green ideas sleep furiously. 
\item\label{Chom2} Furiously sleep ideas green colorless. 
\item Seventeen is purple.
\item Green is a prime number.
\item Someone someone admires admires someone.
\end{menumerate}
There are no principled, universally accepted rules that determine whether each is officially a sentence of English. 
Without a clear distinction between sentences and non-sentences, we cannot be certain which are appropriate for substitution in logical schemas like those given in the previous section.

A second problem with natural languages is that they contain paradoxical sentences. 
A \underidf{paradoxical}{sentence} sentence is one that's true if and only if it's false. 
The most famous examples are those that engender the liar paradox, often called \underidf{liar}{sentence} sentences.\footnote{See the following: \citealt{Tarski1983,Tarski1944}, \citealp{Kripke1975}, \citealp{Barwise1987}, and \citealp{Gupta2001}.}
The following are two examples.
\begin{menumerate}
\item\label{liar1} Sentence \ref{liar1} is false.
\item\label{liar2} The second sentence on this page with exactly twelve words is false.
\end{menumerate}
Paradoxical sentences seem to be simultaneously both true and false. Try assuming that \ref{liar1} is false, and it seems to follow that it's also true. Try assuming that \ref{liar1} is true, and it comes out false.  Both outcomes are impossible because they're each contradictory. 
Paradoxical sentences frustrate systematic and coherent logical analysis, so logicians prefer to exclude them altogether.

A third problem is that natural languages contain \underidf{ungrounded}{sentence} sentences. 
The following pair is a good example:
\begin{menumerate} 
\item\label{si} Sentence \ref{sii} is true.
\item\label{sii} Sentence \ref{si} is true.
\end{menumerate}
These sentences seem to be unhinged from reality. 
We can assume consistently that both are true, and we can assume consistently that both are false, regardless of any substantive fact about the world. 
Most logicians prefer to exclude ungrounded sentences along with the paradoxical ones.

An important note is in order. 
Paradoxical and ungrounded sentences are different from \underidf{contradictory}{sentence} (or ``self-contradictory'') ones. 
The following is a contradiction, not a paradox:
\begin{menumerate} 
	\item The door is open and it isn't open.
\end{menumerate}
Unlike a paradoxical sentence, which seems to take both truth values, a contradictory sentence must be false. 
There is nothing wrong with contradictory sentences from a logician's point of view. 
They are perfectly legitimate relatives of logical truths. 
Just as a logical truth must be true, a contradiction (i.e., a logical falsehood) must be false. 
Contradictory sentences are self-refuting, but they take a single truth value and so are acceptable for our purposes. 

A fourth problem with natural language is that many, if not most, sentences of natural languages have ambiguous or context-dependent meanings. 
Consider the following. 
\begin{menumerate}
\item\label{syn1} Wealthy Americans flee south in record numbers.
\item\label{syn2} We gave the bananas to the monkeys because they
were hungry.
\item\label{syn3} We gave the bananas to the monkeys because they
were tasty.
\item\label{syn4} We gave the bananas to the monkeys because they
were there.
\item\label{he} Paul insulted George because he thought he was
someone else. 
\item\label{car} Sally's car is red. 
\item\label{tall} He is tall. 
\item\label{here} He is the tallest one here. 
\item\label{indexical} You are tall. 
\item\label{bank} Sally went to the bank. 
\end{menumerate} 
We can often resolve ambiguous sentence meaning by considering the particular circumstances in which a sentence is spoken or written. 
For instance, the ambiguity may come from an indexical term like \mention{you} (e.g., \ref{indexical}), in which case the intended referent is clear once we figure out who is speaking or writing to whom. 
Sometimes the meaning of a term is context-dependent, as with \mention{tall} (e.g., \ref{tall}). 
Five feet would be tall for a seven-year old, but not for an adult. 
Other cases are more complicated. 
Does \mention{Sally's car} in \ref{car} mean the car Sally owns, leases, or something else? 
Is \mention{red} the color of the car's exterior or interior? 
Is it light red, dark red, or something in between? 

Pronoun reference can be difficult to resolve, even in context (e.g., `he' in \ref{he}, \ref{tall}, \ref{here}). 
Finally, sometimes a word in the sentence has multiple distinct meanings (e.g., \mention{bank} in \ref{bank}) or the overall syntax of the sentence is ambiguous (e.g., \ref{syn1}--\ref{syn4}). As a result the intended meaning of the sentence may not be recoverable from the context without additional information.

Ambiguous sentences can make careful logical analysis difficult or impossible. The following looks like a logical truth:
\begin{menumerate}
	\item If she is tall then she is tall.
\end{menumerate}
But what if the first `she' refers to one person and the second refers to another? 

%\footnote{%
%A natural response to ambiguity is that typically it's a feature of sentence \emph{types} (see \pmvref{Some Distinctions} of this chapter). 
%Often the ambiguity is solved at the level of tokens: the context of utterance or writing typically makes it so that ambiguities in a given token sentence's type are resolved. 
%But this isn't always the case. 
%\label{pragmaticsfootnote}
%}

\subsection{Formal Languages}\label{Formal Languages}
The logician avoids these difficulties by using carefully constructed formal languages.\footnote{Poorly (or deviously) constructed formal languages can have the same problems as natural ones.} 
A \underidf{formal}{sentence} language has the following features: 
\begin{cenumerate}
	\item\label{formal1} There is an explicit list of permitted symbols.
	\item\label{formal2} There is a definition of which strings of symbols count as sentences.   
\end{cenumerate} 
A \idf{string} of symbols is just a row or sequence of symbols. 
For example, \mention{fq3H7} is a string consisting of \mention{f} followed by \mention{q}, \mention{3}, \mention{H}, and \mention{7}. The order of the characters is important; \mention{f7q} is not the same string as \mention{7fq}. 

Formal languages have a precisely defined syntax, but no fixed semantics. 
Definitions of sentences in formal languages make no reference to meanings of the sentences or their parts. 
These definitions make reference only to the \emph{form}, or shape of the symbols and their arrangements in strings. 
We should think of the symbols and sentences of a formal language as lacking inherent meaning.

In a well-constructed formal language we can check whether any given string is a genuine sentence, solving the indeterminacy problem faced by natural languages.
We also exclude the features of natural languages that lead to paradoxical and ungrounded sentences. 
Sentences of formal languages have no inherent meaning, but there are ways to assign meaning to them while avoiding the difficulties of natural language. 
In chapter \ref{Translations} we provide methods of translating English sentences into constructed formal languages.

By turning to formal languages we are using a familiar (and often successful) strategy of replacing a hard messy problem with a more tractable mathematical one. 
For example, geometry doesn't deal with points or lines in the physical world; the points and lines of geometry have zero area and zero width, respectively. 
Nevertheless, geometry provides useful models of the physical world when physical points and lines approximate geometrical ones sufficiently.
When the logical features of our formal language sufficiently reflect the logical features of our natural language, mathematical idealization is a productive strategy.\footnote{%  
	Historically the move to formal languages in the study of logic (and mathematics) has been fruitful. 
	One example, important for the development of modern logic, comes from set theory. 
	By using formal methods logicians found that the na\"{i}ve axioms of set theory are inconsistent \citetext{see \citealp{Demopoulos2005} for a quick overview of Russell's paradox, or \citealp[ch~1]{Smullyan2010} for a more careful discussion}.
} 

How do our formal languages relate to natural languages like English?
We think of the formal languages as abstract models of important parts of their natural counterparts. 
We cover several formal languages in this text: \GSL{} (chapter \ref{sententiallogic}), \GQL{}1 (chapter \ref{quantifierlogic1}), \GQL{} (chapter \ref{quantifierlogic}), and later \MGSL{} and \GQLI{} (chapter \ref{furtherdirections}), will serve as progressively more detailed models of English. 
After learning how to define these formal languages and how to translate between them and English, the logic student should have a better understanding of the logical structure of English and of logical consequence more generally.
The student should keep in mind that these formal languages cannot perfectly capture all the desirable logical features of English. 
There is a limit to both the range of English sentences they model and the accuracy with which they model them. 

\section{Some Distinctions}\label{Some Distinctions} 
\subsection{Use and Mention}\label{usemention}
There are three distinctions that are important in the modern study of logic. 
The first is the \distinction{use}{mention} distinction\index{\distinction{use}{mention} distinction}. 
For any word or expression (string of words), we can either \emph{use} the word or expression, or instead \emph{mention} it. 
\emph{Using} words and expressions is what we normally do, while \emph{mentioning} a word or expression is one way to talk about that word or expression. 
For example, you could say (i) that a cow is a four-legged bovine, or you could instead say (ii) that \mention{cow} has three letters. 
In case (i) we are talking about a particular kind of animal, while in case (ii) we are talking about the \emph{word} for that animal. 
In this textbook words that are mentioned are put in single quotes.\index{single quotes} Here are some examples:
\begin{menumerate}
\item Houston has over 2 million inhabitants. [True]
\item Houston has over 10 million inhabitants. [False]
\item \mention{Houston} has more than 2 million inhabitants. [False]
\item \mention{Houston} has more than 4 letters. [True]
\item Austin has more than 4 letters. [False]
\item\label{twentyeight} \mention{Austin} has exactly 6 letters. [True]
\end{menumerate}

\subsection{Type and Token}\label{typetoken}
The next distinction is the \distinction{type}{token} distinction.\index{\distinction{type}{token} distinction} 
How many letters does \mention{Houston} have?  The answer depends on whether the asker means letter tokens or letter types.
A \idf{token} is a physical object or event, while a \idf{type} is an abstract \emph{kind} of physical object or event. 
Tokens are located in space and time, while types presumably are not. 
The word \mention{Houston} has 7 letter tokens and 6 letter types, because the letter \mention{o} occurs twice.
In the sentence \mention{Ron went on and on} there are five word tokens and four word types.\footnote{In the sentence \mention{Ron went on and on}, there are three occurrences of the string \mention{on} with one of them in the name \mention{Ron}.  However, this instance doesn't count as a word token in English, because it's only a piece of a word, not a complete word.  Other cases in English are more ambiguous.  In the compound word \mention{footnote} does \mention{note} count as a word token?} 
In \mention{radar} there are five letter tokens and three letter types. 

\subsection{Object Language and Metalanguage}\label{objectandmetalanguage}
The third distinction is that between an object language and a metalanguage. 
Returning to the \distinction{use}{mention} distinction, if one mentions a word, then the \idf{object language}\index{language!object} is the language of the word being mentioned and the \idf{metalanguage}\index{language!meta-} is the language in which that word is mentioned. 
For example, if I say \q{The German word for dog is \mention{Hund},} I've used English (the metalanguage) to talk about German (the object language). 
In this text the metalanguage will always be English augmented with some carefully selected mathematical notation. 
The object language will be a formal language. 

Usually there's a little more to being a metalanguage than the mere fact that it's the language used to talk about another. 
Often there's some specific purpose. 
For example, the metalanguage might be used to give definitions of words in the object language. Or, as in our case, the metalanguage might be used to define when a sentence of the object language is true, or to describe when the logical consequence relationship holds between sentences of the object language. 

%%%%%%%%%%%%%%%%%%%%%%%%%%%%%%%%%%%%%%%%%%%%%%%%%%
\section{MathEnglish}\label{MathEnglish}
%%%%%%%%%%%%%%%%%%%%%%%%%%%%%%%%%%%%%%%%%%%%%%%%%%

The various formal languages in this text are to be object languages.  However, before we can define our first formal language (\GSL{}) we need to familiarize the reader with certain mathematical concepts.  The metalanguage contains a mix of English, mathematical symbols (e.g., set theory operators), and other technical jargon.  We call our metalanguage \idf{MathEnglish}.   

\subsection{Metavariables}\label{metavariables}
One important tool for MathEnglish is the \emph{metavariable}, which we define as a variable for strings of the object language.\index{variables!MathEnglish}  
We use lowercase Greek letters as our metavariables, as we did in the schemas given at the beginning of the chapter.
The three we use most commonly are \mention{$\CAPPHI$} (phi), pronounced \q{fie}, \mention{$\CAPTHETA$} (theta), pronounced \q{theta}, and \mention{$\CAPPSI$} (psi), pronounced \q{sigh}.
These Greek letters are not in the object language, but instead are used in the metalanguage for talking about the object language. 

An example will make their role more clear.  Let's say that MathEnglish is our metalanguage and (regular) English is our object language.\footnote{As discussed earlier, in later chapters we use a formal language and not English as our object language, but for the purpose of illustration we temporarily ignore the problems of natural language.}  Recall the logically true sentence from the beginning of the chapter:

\begin{quote}
\noindent{}If the sun is shining then the sun is shining.
\end{quote}

\noindent{}We can replace `the sun is shining' with any other assertion and get another logically true sentence.  With this in mind, consider the following claim of MathEnglish: 

\noindent{}\begin{quote}All English sentences of the form \mention{If $\CAPPHI$ then $\CAPPHI$} are logical truths.\end{quote}

\noindent{}The use of `$\CAPPHI$' helps us talk about \emph{all} English sentences of a certain form as logically true, or at least all English sentences of a certain kind. 
As logicians we are concerned with truth and truth-preservation, so we are only going to address strings with plausible truth values, i.e. declarative sentences. 
A declarative sentence is one that makes an assertion. 
By restricting the substitutions to declaratives, we rule out substitutions such as `Ergle bergle barfle narfle,' 'nef34fwjh9ufh32,' `Kentucky fried chicken,' and `Are you cooking?'. 
We cannot legitimately affirm truth or falsity of these non-assertions. 
In the rest of the text, when we refer to \mention{all} sentences of a natural language we only mean the declarative ones.

In any case, metavariables are indispensable tools for making general claims about object languages.


\subsection{Sets}\label{sets}

MathEnglish makes use of set theory.  A \emph{set} is just a collection of items called \emph{elements}. Elements can be anything, like animals, people, planets, cartoon characters, or even abstract objects such as numbers.  A set can be specified using curly brackets, as in the following set of integers from 1 to 4:

\begin{center}
\noindent{}$\{1, 2, 3, 4\}$
\end{center}

\noindent{}The order of elements in set notation doesn't matter, so the following set is identical to the last:

\begin{center}
\noindent{}$\{2, 3, 1, 4\}$
\end{center}

\noindent{}Furthermore, repetition of an element in set notation is to be disregarded; an element can only be in the set once.  Accordingly, the following sets are the same set as the last two:

\begin{center}
\noindent{}$\{1, 2, 2, 3, 3, 3, 4, 4, 4, 4\}$\\
\noindent{}$\{2, 2, 3, 1, 4, 2, 3, 1, 4, 4, 3\}$
\end{center}

\noindent{}Sets don't have to be finite. Consider the set of positive integers:

\begin{center}
\noindent{}$\{1, 2, 3, 4, ..., 1002, 1003, 1004, ... \}$
\end{center}

There is one set that doesn't have any elements, and it's called either the \emph{null set} or the \emph{empty set}.  Symbolically the empty set is denoted by either $\{ \}$ or $\emptyset$.

Set membership is expressed with the `$\in$' symbol.  Let the Greek letters $\Delta$ and $\Gamma$ stand for arbitrary sets. (They are pronounced \mention{delta} and \mention{gamma}, respectively.)  We assert that $7$ is an element of set $\Delta$ and $3$ is an element of $\Gamma$ as follows: $7 \in \Delta$, $3 \in \Gamma$. On this notation,

\begin{center}
	\noindent{}$2\in\{1, 2, 3\}$
\end{center}

\noindent{}is true and,

\begin{center}
	\noindent{}$4\in\{1, 2, 3\}$ 
\end{center}

\noindent{}is false.

We use lowercase Greek letters $\alpha$ and $\beta$ (pronounced \mention{alpha} and \mention{beta}, respectively) as metavariables for elements. For example, if $\alpha$ and $\beta$ are odd numbers and $\Delta$ is the set of even numbers, then $\alpha + \beta \in \Delta$.

\begin{majorILnc}{\LnpDC{Subset}} For any two sets $\Delta$ and $\Gamma$, $\Delta$ is a \df{subset} of $\Gamma$ \Iff for each $\alpha$ such that $\alpha \in \Delta$, $\alpha \in \Gamma$.
\end{majorILnc} 

\noindent{}For example, $\{1, 2, 3\}$ is a subset of $\{1, 2, 3, 4\}$. The set $\{1, 2\}$ is a subset of each of the two previous sets.  The symbol `$\subseteq$' denotes the subset relation.  So if $\Delta$ is a subset of $\Gamma$, $\Delta \subseteq \Gamma$. Note that by the above definition every set is a subset of itself.

\begin{majorILnc}{\LnpDC{Proper Subset}} For any two sets $\Delta$ and $\Gamma$, $\Delta$ is a \df{proper subset} of $\Gamma$ \Iff 
	\begin{cenumerate}
		\item $\Delta \subseteq \Gamma$ and
		\item there is some $\alpha$ such that $\alpha \notin \Delta$ and $\alpha \in \Gamma$.
	\end{cenumerate}
\end{majorILnc}

\noindent{}As you can probably guess, the \mention{$\notin$} symbol means \mention{is not an element of}. The symbolic notation for \mention{$\Delta$ is a proper subset of $\Gamma$} is $\Delta \subset \Gamma$. No set is a proper subset of itself.

Take care to avoid confusing the membership, subset, and proper subset relations.

\begin{menumerate}
	\item $2\in\{1, 2, 3\}$ [True]
	\item $2\subseteq\{1, 2, 3\}$ [False]
	\item $\{2, 3\}\subseteq\{1, 2, 3\}$ [True]
	\item $\{2, 3\}\in\{1, 2, 3\}$ [False]
	\item $\{2, 3\}\subset\{1, 2, 3\}$ [True]
	\item $\{1, 2, 3\}\subset\{1, 2, 3\}$ [False]
\end{menumerate}

\subsection{More on Sets: Union and Intersection}\label{moreonsets}

Sometimes a set is defined by reference to other sets. Say that there is a set $\Delta$ that contains all the members of two other sets, $\Gamma_1$ and $\Gamma_2$, and nothing else. Then $\Delta$ is said to be the \emph{union} of $\Gamma_1$ and $\Gamma_2$. Let's give a strict definition:

\begin{majorILnc}{\LnpDC{Union}} $\Delta=\Gamma_1\cup\Gamma_2$ \Iff 
	\begin{cenumerate}
		\item for each $\alpha \in \Gamma_1$, $\alpha \in \Delta$ and
		\item for each $\alpha \in \Gamma_2$, $\alpha \in \Delta$ and
		\item for each $\alpha \in \Delta$, $\alpha \in \Gamma_1$ or $\alpha \in \Gamma_2$ (or both).
	\end{cenumerate}	
\end{majorILnc} 

\noindent{}The symbol \mention{$\cup$} means \mention{union}. Some examples:

\begin{center}
	\noindent{}$\{$Mercury, Jupiter, Mars, Saturn$\}=\{$Mercury, Jupiter$\}\cup\{$Mars, Saturn$\}$.
\end{center}

\begin{center}
	\noindent{}$\{$Newton, Bacon, Boyle$\}=\{$Newton, Bacon$\}\cup\{$Bacon, Boyle$\}$.
\end{center}

Another useful concept is \emph{intersection}.  The set $\Delta$ is the intersection of two sets $\Gamma_1$ and $\Gamma_2$ \Iff $\Delta$ contains all the elements shared by both $\Gamma_1$ and $\Gamma_2$ but nothing else. More strictly:

\begin{majorILnc}{\LnpDC{Intersection}} $\Delta=\Gamma_1\cap\Gamma_2$ \Iff 
	\begin{cenumerate}
		\item for each $\alpha$ such that $\alpha \in \Gamma_1$ and $\alpha \in \Gamma_2$, $\alpha \in \Delta$ and
		\item for each $\alpha \in \Delta$, $\alpha \in \Gamma_1$ and $\alpha \in \Gamma_2$.
	\end{cenumerate}	
\end{majorILnc} 

The symbol \mention{$\cap$} means \mention{intersection}.
The intersection of $\{$Mercury, Jupiter$\}$ and $\{$Mars, Saturn$\}$ is the empty set, $\{ \}$, because the former two sets have no members in common.

\begin{center}
	\noindent{}$\{$Mercury, Jupiter$\}\cap\{$Mars, Saturn$\}=\{ \}$.
\end{center}

\noindent{}By contrast, the intersection of $\{$1, 5, 6, 9$\}$ and $\{$2, 3, 5, 6$\}$ is $\{$5, 6$\}$.

\begin{center}
	\noindent{}$\{$1, 5, 6, 9$\}\cap\{$2, 3, 5, 6$\}=\{$5, 6$\}$.
\end{center}

\noindent{}Another example:

\begin{center}
	\noindent{}$\{$Carnap, Quine$\}=\{$Anscombe, Quine, Carnap$\}\cap\{$Carnap, Lewis, Quine$\}$.
\end{center}

\subsection{Ordered Pairs and Ordered n-tuples}\label{orderedpairs}

Sometimes we would like to describe a collection in which, unlike sets, the order does matter.  For that purpose we have \emph{ordered pairs} and \emph{ordered $n$-tuples}.  An ordered pair is a two-place collection in which the order matters.  To differentiate ordered pairs from sets, we indicate them using angled brackets rather than curly brackets.  So, while the sets $\{$1, 2$\}$ and $\{$2, 1$\}$ are identical, the ordered pairs $\langle$1, 2$\rangle$ and $\langle$2, 1$\rangle$ are not.

An ordered collection having more than two places is an $n$-tuple, where $n$ is the number of places needed.  So, $\langle$Mercury, Venus, Earth$\rangle$ is a 3-tuple, $\langle$Jupiter, Saturn, Uranus, Neptune$\rangle$ is a 4-tuple, and so on.

Also unlike sets, for ordered pairs and $n$-tuples, repetition makes a difference. For example, $\langle 1, 2, 3\rangle$ is different from $\langle 1, 2, 3, 3, 3\rangle$.

\subsection{Mathematical Proofs}\label{Mathematical Proofs}

We use proofs to establish that something has certain mathematical properties. 
A proof works from one or more definitions to some interesting result, called a \emph{theorem}. 
Much of this textbook is filled with proofs of theorems about the various formal languages defined in later chapters.
If we want to prove an intermediate result that isn't by itself particularly interesting but which is handy for use in one or more other proofs, we sometimes call it a \emph{lemma}.

To illustrate, let's look at a simple proof.

\begin{THEOREM}{\LnpTC{Transitivity of Subset}}
	Let $\Delta_1$, $\Delta_2$, and $\Delta_3$ be sets.
	If $\Delta_1\subseteq\Delta_2$ and $\Delta_2\subseteq\Delta_3$ then $\Delta_1\subseteq\Delta_3$.
\end{THEOREM}
\begin{PROOF}
	Assume that $\Delta_1\subseteq\Delta_2$ and $\Delta_2\subseteq\Delta_3$.
	Since $\Delta_1\subseteq\Delta_2$, then by the definition of $\subseteq$, any $\alpha$ in $\Delta_1$ is also in $\Delta_2$.
	And since $\Delta_2\subseteq\Delta_3$, any $\alpha$ in $\Delta_2$ is also in $\Delta_3$.
	It then follows that if $\alpha\in\Delta_1$ then $\alpha\in\Delta_3$. 
	Thus, by the definition of $\subseteq$, $\Delta_1\subseteq\Delta_3$.
\end{PROOF}

\begin{commentary}
	Some students are not used to the structure and rigor of mathematical proofs.
	As an aid to understanding we sometimes include commentary to explain how a particular proof works.
	We use a grayed box to signify that this commentary is not itself part of the proof.

	\commentaryspace
	Here we are trying to prove a \emph{conditional} theorem, i.e., a statement of the form ``If $\Al$ then $\Bl$.''
	The typical way to demonstrate such theorems is to start by assuming the antecedent, $\Al$.
	Then we work from that assumption to show that the consequent, $\Bl$, must follow.
	Each step is justified by reference to a given definition or by mathematical reasoning, often in the language of set theory.
\end{commentary}

\noindent{}We have proved that \mention{$\subseteq$} is transitive. 
Having achieved this result, we permit ourselves to use it throughout the rest of the text without proving it again. 
The end of a completed proof is indicated with a black box: $\blacksquare$.

\subsection{Recursive Definitions}\label{Recursive Definitions}

In later chapters we use MathEnglish to define concepts such as \emph{sentence}, \emph{truth}, and \emph{entailment} for a given object language. 
Many of these definitions are recursive.
Recursive definitions are an invaluable tool for logic because they enable us to give a rigorous finite definition of an infinite set.
They can be understood as defining which elements are in a given set.

\begin{majorILnc}{\LnpDC{Definition of Recursive Definition}}
A \df{recursive definition} of $\Delta$ has three clauses:
\begin{cenumerate}
\item the \df{base clause(s)}, which specifies one or more objects as elements of $\Delta$,
\item the \df{generating clause(s)}, which specifies one or more ways of generating (or finding) other objects that are elements of $\Delta$, and
\item the \df{closure clause}, which specifies that something is in $\Delta$ only if it can be shown to be so by applications of the first two clauses.
\end{cenumerate}
\end{majorILnc}

\noindent{}Think of the base clause as the place to specify specific elements for inclusion in the set.
The generating clause is more powerful.
It defines elements of the set by reference to more basic elements already included.
The closure clause is usually a simple statement, but it is essential to specify what isn't in the set.

One concept that we can define recursively is \emph{natural number}.
\begin{majorILnc}{\LnpDC{Natural Number}} The set $\mathbb{N}$ of natural numbers:
	\begin{cenumerate}
		\item Base Clause: Let $0$ be a natural number. I.e. $0 \in \mathbb{N}$.
		\item Generating Clause: If $\integer{n} \in \mathbb{N}$ then $\integer{n}+1 \in \mathbb{N}$.
		\item Closure Clause: Nothing else is in $\mathbb{N}$.  
	\end{cenumerate}
\end{majorILnc}
This definition unambiguously identifies infinitely many numbers as natural numbers. 
Obviously $0$ is a natural number. 
Is $7$ a natural number? 
It is. We know $0$ is a natural number because the base clause says so. 
Thus, according to the generating clause, $0+1$ (i.e., $1$) is also a natural number. Apply the generating clause again to show that $1+1$ (i.e., $2$) is natural number. One can continue applying the generating clause until $7$ is reached. 
Is $-3$ a natural number? No. $-3$ is less than $0$ and the generating clause only defines larger numbers as natural. The closure clause rules out the inclusion of anything else.

Let's use a recursive definition to identify which people are ancestors of French mathematician Blaise Pascal.

\begin{cenumerate}
	\item Base Clause: Blaise Pascal's mother and father are ancestors of Blaise Pascal.
	\item Generating Clause: If $x$ is an ancestor of Blaise Pascal and $y$ is a parent of $x$, then $y$ is also an ancestor of Blaise Pascal.
	\item Closure Clause: No one else is an ancestor of Blaise Pascal.
\end{cenumerate}
\noindent{}This definition picks out the mother and father of Blaise Pascal, their respective mothers and fathers, the mothers and fathers of the latter, and so on.  It excludes everyone else. Recursive definitions are powerful, because in a few lines we can fix an unambiguous collection that is arbitrarily large. The closure clause is usually relatively trivial (e.g., ``Nothing else is an $x$.'').

\subsection{Conclusion}\label{Conclusion}

Now we have most of the mathematical tools we need to define our first formal language, \emph{Sentential Logic}. Anything else we need will be given along the way.

%%%%%%%%%%%%%%%%%%%%%%%%%%%%%%%%%%%%%%%%%%%%%%%%%%
\section{Exercises}
%%%%%%%%%%%%%%%%%%%%%%%%%%%%%%%%%%%%%%%%%%%%%%%%%%

\notocsubsection{Sets}{ex:Sets}

\begin{enumerate}
	\item True or false: $\{2, 3\}\subseteq\{1, 2, 3\}$
    \item True or false: $\{1, 2, 3\}\subset\{1, 2, 3\}$
	\item True or false: $\{2, 3\}\in\{1, 2, 3\}$
    \item True or false: $\{2, 3\}\in\{1, \{2, 3\}\}$
    \item True or false: $2\in\{1, \{2, 3\}\}$
    \item True or false: $\{1, 2, 3\}=\{1, \{2, 3\}\}$
    \item True or false: $\{1, 2, 3\}=\{3, 3, 1, 2, 3, 2, 2, 1\}$
\end{enumerate}

\notocsubsection{Unions and Intersections}{ex:Unions and Intersections}

\begin{enumerate}
	\item True or false: $\{2,4,6\}\cup\{1,2,3\}\subseteq\{1,2,3,4,5,6\}$
	\item True or false: $\{2,4,6\}\cap\{1,2,3\}=\{1,2,3,4,5,6\}$
    \item True or false: $\{1,2\}\cap\{3,4\}\subseteq\{1,2\}\cup\{3,4\}$
	\item True or false: $\{1\}\cup\{2, 3\}=\{1, \{2, 3\}\}$
\end{enumerate}

\notocsubsection{Proofs}{ex:Simple Proofs}
Prove each of the following.
\begin{enumerate}
	\item $\Delta_1\cap\Delta_2\subseteq\Delta_1\cup\Delta_2$.
	\item If $\Delta_1\subseteq\Delta_2$ then $\Delta_1\cup\Delta_3\subseteq\Delta_2\cup\Delta_3$.
    \item If $\Delta_1\subseteq\Delta_2$ then $\Delta_1\cap\Delta_3\subseteq\Delta_2\cap\Delta_3$.
\end{enumerate}

\notocsubsection{Recursive Definitions}{ex:Recursive Definitions}
Define the following recursively.
\begin{enumerate}
	\item The ancestors of George Washington.
    \item All positive even numbers.
	\item All integers.
\end{enumerate}

%\theendnotes



%\addtocontents{toc}{\protect\thispagestyle{empty}}
%%%%%%%%%%%%%%%%%%%%%%%%%%%%%%%%%%%%%%%%%%%%%%%%%%
\chapter{Sentential Logic}\label{sententiallogic}
%%%%%%%%%%%%%%%%%%%%%%%%%%%%%%%%%%%%%%%%%%%%%%%%%%
%\AddToShipoutPicture*{\BackgroundPicA}


%%%%%%%%%%%%%%%%%%%%%%%%%%%%%%%%%%%%%%%%%%%%%%%%%%
\section{The Language \GSL{}}\label{The Language GSL}
%%%%%%%%%%%%%%%%%%%%%%%%%%%%%%%%%%%%%%%%%%%%%%%%%%

\subsection{Sentences of \GSL{}}\label{Sentences of GSL}
Our first task is to define the syntax of the basic formal language \GSL{}.\footnote{Work on this sentential logic originated with George Boole \citeyearpar{Boole1854} and Augustus De Morgan \citeyearpar{DeMorgan1847,DeMorgan1860}. 
	As Christine Ladd-Franklin notes \citeyearpar[17]{LaddFranklin1883} before giving her own, variations quickly followed from William S. Jevons, Ernst Schr\"oder, Hugh McColl, and Charles S. Peirce.} 
The language we're starting with is variously called \idf{sentential logic} or \niidf{propositional logic}\index{propositional logic|see{sentential logic}}. 
We use the former name, because it's easier to get a grasp on what a sentence is. What a proposition is is a matter of intense philosophical debate.\footnote{For historical and contemporary discussions of propositions, see \citealt{Frege1892}, \citealt[13,47]{Russell1903}, \citealt[26]{Church1956}, \citealt[ch.~1]{Quine1986}, \citealt[ch.~3]{Schiffer1987}, \citealt{Grandy1993}, \citealt{Bealer1998b}, \citealp{King2007}, \citealt{Soames2010}.}

As stated in the previous chapter, each formal language\index{language} requires (1) a list of basic symbols, and (2) a definition of which sequences of those symbols count as sentences\index{sentence}. Think of these as determining the proper grammar of the language. 

\begin{majorILnc}{\LnpDC{Basic Symbols of GSL}} The \df{basic symbols} of \GSL{} are of three kinds:\footnote{The commas and ellipses are \emph{not} symbols of \GSL{}.}
\begin{cenumerate}
\item Logical Connectives: $\NEGATION$, $\WEDGE$, $\VEE$, $\HORSESHOE$, $\TRIPLEBAR$
\item Punctuation Symbols: (, )
\item Sentence Letters: $\Al,\Bl, \ldots, \Tl, \Al_1,\Bl_1, \ldots, \Tl_1, \Al_2, \Bl_2, \ldots$  
\end{cenumerate}
\end{majorILnc} 
\noindent{}Logical connectives are sometimes called logical symbols, logical operators, logical terms, or logical functors.%
\footnote{%
In other textbooks there are sometimes different symbols used for these connectives. 
Along with $\NEGATION$, $\neg$ and $-$ are used for negation, $\&$ and $\cdot$ for conjunction, $\supset$ and $\Rightarrow$ for conditionals, and $\equiv$ for biconditionals. $\VEE$ is almost universally the symbol for disjunction.
} 
One could draw important distinctions between symbols, terms, and operators, but the names are just as often used interchangeably.
The \emph{sentence letters} are italicized capital letters of the Roman alphabet from \mention{$\Al$} to \mention{$\Tl$}.  To give ourselves an infinite supply, any of these letters concatenated with a subscripted positive integer is also a sentence letter; e.g. $\Cl$ and a subscripted $7$ can be combined to get the new sentence letter $\Cl_7$. Keep in mind that $\Cl_7$ and $\Cl$ are different sentences.

\begin{majorILnc}{\LnpDC{Recursive definition of Sentences of GSL}} The \nidf{sentences} \underdf{of \GSL{}}{sentence} are given by the following recursive definition:
\begin{description}
\item[Base Clause:] Every sentence letter is a sentence.
\item[Generating Clauses:] \hfill
\begin{cenumerate}
\item If $\CAPPHI$ is a sentence, then so is $\negation{\CAPPHI}$.\footnote{Remember from Chapter 1 that $\CAPPHI$ and $\CAPTHETA$ are used as metavariables. In this definition they stand for sentences of \GSL{}.}
\item If $\CAPPHI$ and $\CAPTHETA$ are sentences, then so are both $\parhorseshoe{\CAPPHI}{\CAPTHETA}$ and $\partriplebar{\CAPPHI}{\CAPTHETA}$.
\item If all of $\CAPPHI_1,\CAPPHI_2,\ldots,\CAPPHI_{\integer{n}}$ are sentences (where $\integer{n}$ is an integer $\geq2$), then so are $\parconjunction{\CAPPHI_1}{\conjunction{\CAPPHI_2}{\conjunction{\ldots}{\CAPPHI_{\integer{n}}}}}$ and $\pardisjunction{\CAPPHI_1}{\disjunction{\CAPPHI_2}{\disjunction{\ldots}{\CAPPHI_{\integer{n}}}}}$.
\end{cenumerate}
\item[Closure Clause:] No string is an \GSL{} sentence unless it counts as such according to the above base and generating clauses.
\end{description}
\end{majorILnc}

The sentences specified by the base clause, i.e. lone sentence letters, are called \underidf{atomic}{sentence} sentences.
In generating clause (2), sentences of the first form are called \idf{conditionals}, and those of the second are \idf{biconditionals}. 
The left-hand side of the conditional is traditionally called the \idf{antecedent}\index{LHS} and the right-hand side the \idf{consequent}\index{RHS}. 
We often use the alternative terminology \CAPS{lhs} (left-hand side) and \CAPS{rhs} (right-hand side).

Generating clause (3) makes our version of \GSL{} slightly different from what you'll find in other logic texts.\footnote{Some linguists have used the definition we use, including \citealt{Gleitman1965}.} 
Sentences of the first form of clause (3) are called \idf{conjunctions} and their component sentences \idf{conjuncts}. Sentences of the second form are called \idf{disjunctions} and their component sentences \idf{disjuncts}. 
Many logic books treat conjunction and disjunction as binary (2-place), e.g., \mention{$\pardisjunction{\Al}{\Bl}$}. According to our clause (3) \mention{$\pardisjunction{\disjunction{\Al}{\Bl}}{\Cl}$} is also a perfectly good sentence. Textbooks that treat disjunctions as binary require an extra set of parentheses to generate an equivalent sentence: e.g., \mention{$\pardisjunction{\pardisjunction{\Al}{\Bl}}{\Cl}$}.\footnote{We'll explain which sentences count as \mention{equivalent} later in the chapter.} This is also an acceptable sentence of \GSL{}, but the extra parentheses don't add interesting information. We prefer the definitions of conjunction and disjunction given above because they're closer to natural English and they allow us to avoid unnecessary parentheses.

The closure clause excludes strings that the other clauses don't define as sentences. 
This is needed to prevent nonsense such as \mention{$(\Bl(\HORSESHOE{}\Al$} from counting as an \GSL{} sentence. 
There is no way to generate the string from repeated applications of the base and generating clauses, so it is ruled out. 

Additionally, \mention{$\negation{\CAPPHI}$}, \mention{$\parhorseshoe{\CAPPHI}{\CAPTHETA}$}, etc. in the generating clauses are \idf{sentence schemas}. 
They aren't \GSL{} sentences because the Greek letters aren't given as symbols of the language. 
The substitution of an \GSL{} sentence for each Greek letter in a schema results in a \GSL{} sentence, however. 
For example, the substitution $\CAPPHI=\;$\mention{$\Al$}, $\CAPTHETA=\;$\mention{$\partriplebar{\Cl}{\Dl}$} into the schema \mention{$\parhorseshoe{\CAPPHI}{\CAPTHETA}$} results in the sentence \mention{$\parhorseshoe{\Al}{\partriplebar{\Cl}{\Dl}}$}.\footnote{Often when Greek letters \mention{$\CAPPHI$}, \mention{$\CAPPSI$}, \mention{$\CAPTHETA$}, etc. are used it will tacitly be assumed that they are ranging over \GSL{} sentences and not mere strings of \GSL{} symbols. 
Of course, we can still use them when needed to range over all strings of \GSL{} symbols as we did in the definition just given of \GSL{} sentences.}

\begin{majorILnc}{\LnpEC{Example of Recursive definition of GSL sentences}}
$\parconjunction{\cpardisjunction{\Al}{\parhorseshoe{\Dl}{\Bl}}}{\negation{\Gl}}$ is a sentence of \GSL{}. 
Proof: The base clause of definition \ref{Recursive definition of Sentences of GSL} shows that $\Al$, $\Dl$, $\Bl$, and $\Gl$ are all sentences; they are all sentence letters. 
From $\Dl$ and $\Bl$, and by generating clause (2), we know $\parhorseshoe{\Dl}{\Bl}$ is a sentence. 
From $\Gl$ and generating clause (1), $\negation{\Gl}$ is a sentence. 
From $\Al$ and $\parhorseshoe{\Dl}{\Bl}$, and generating clause (3), $\cpardisjunction{\Al}{\parhorseshoe{\Dl}{\Bl}}$ is a sentence. 
And, finally, from $\cpardisjunction{\Al}{\parhorseshoe{\Dl}{\Bl}}$ and $\negation{\Gl}$, and generating clause (3), we see that $\parconjunction{\cpardisjunction{\Al}{\parhorseshoe{\Dl}{\Bl}}}{\negation{\Gl}}$ is a sentence. 
\end{majorILnc}

Unlike English, for any sequence of \GSL{} symbols it is possible to \emph{prove} whether it is a sentence.

\subsection{Official and Unofficial Sentences of \GSL{}}\label{Unofficial Sentences of GSL}

Definition \ref{Recursive definition of Sentences of GSL} is the definition of an \emph{official} sentence of \GSL{}.
For convenience' sake we often work with \emph{un}official sentences. 
\begin{majorILnc}{\LnpDC{Unofficial Sentence of GSL}}
A string of symbols is an \nidf{unofficial} sentence\index{sentence!unofficial (of \GSL{})|textbf} \Iff we can obtain it from an official sentence by
\begin{cenumerate}
\item deleting outer parentheses, or
\item replacing one or more pairs of official round parentheses ( ) with square brackets [ ] or curly brackets \{ \}.
\end{cenumerate}
\end{majorILnc}
\noindent{}Thus \mention{$\conjunction{\Al}{\conjunction{\Bl}{\Cl}}$} is an unofficial sentence, as are
\begin{multicols}{2}
\begin{smenumerate}
\item\label{usex1} $\negation{\pardisjunction{[\Al\wedge\Bl]}{[\Cl\wedge\Dl]}}$
\item $\conjunction{\parconjunction{\Al}{\Bl}}{\Cl}$
\item $\horseshoe{\parconjunction{\Al}{\Bl}}{\Cl}$
\item $\parconjunction{\Al}{[\Bl\rightarrow\Cl]}$
\item $\disjunction{\negation{\Al}}{\Cl}$
\item $[\parconjunction{\Al}{\Bl}\wedge\Cl]$
\item $\conjunction{\{\Al\wedge\Bl\}}{\Cl}$
\item\label{usexL} $\conjunction{\Al}{[\Bl\rightarrow\Cl]}$
\end{smenumerate}
\end{multicols}
\noindent{}From an unofficial sentence we can unambiguously reconstruct the related official sentence. 
Throughout the rest of this text we usually drop outer parentheses, but we will consistently use the standard parentheses ( ). 
The reader should feel free to use brackets [ ] or curly parentheses \{ \} as is helpful. 

\subsection{A Comment on Use and Mention}\label{use mention comment}

Most of the time when you see Greek letters in the text, as in definition \mvref{Recursive definition of Sentences of GSL}, we are \emph{using} them, not mentioning them. 
Thus it's appropriate that in definition \ref{Recursive definition of Sentences of GSL} we did not put them in single quotes.\index{single quotes}
By contrast, we generally \emph{mention} official and unofficial \GSL{} sentences rather than use them.
Because we mention \GSL{} sentences so often it would be tedious to put them in quotes. 
Therefore we refrain from doing so unless there is some special reason to do so.  
We are also less strict when mentioning the basic symbols of \GSL{}.\footnote{This is the usual convention, e.g. \citealt[7]{Hodges2001}.}

\subsection{Other Properties of Sentences}\label{Other Properties of GSL Sentences}
Next we define four important properties and related features of sentences: subsentence, order, main connective, and construction tree. 
\begin{majorILnc}{\LnpDC{Subsentences}}
The following clauses define when one sentence is a \df{subsentence} of another:
\begin{cenumerate}
\item\label{ss1} Every sentence is a subsentence of itself.
\item $\CAPPHI$ is a subsentence of $\negation{\CAPPHI}$.
\item $\CAPPHI$ and $\CAPTHETA$ are subsentences of $\parhorseshoe{\CAPPHI}{\CAPTHETA}$ and $\partriplebar{\CAPPHI}{\CAPTHETA}$.
\item\label{ss4} Each of $\CAPPHI_1,\CAPPHI_2,\ldots,\CAPPHI_{\integer{n}}$ is a subsentence of $\parconjunction{\CAPPHI_1}{\conjunction{\CAPPHI_2}{\conjunction{\ldots}{\CAPPHI_{\integer{n}}}}}$\\ and $\pardisjunction{\CAPPHI_1}{\disjunction{\CAPPHI_2}{\disjunction{\ldots}{\CAPPHI_{\integer{n}}}}}$.
\item\label{ss5} (Transitivity) If $\CAPPHI$ is a subsentence of $\CAPTHETA$ and $\CAPTHETA$ is a subsentence of $\CAPPSI$, then $\CAPPHI$ is a subsentence of $\CAPPSI$.
\item\label{ss6} That's all. 
\end{cenumerate}
\end{majorILnc}
\begin{majorILnc}{\LnpEC{SubSentenceExampleA}}
	The sentence $\parhorseshoe{\Bl}{\Cl}$ has 3 subsentences:
	\begin{cenumerate}
		\item $\parhorseshoe{\Bl}{\Cl}$
		\item $\Bl$
		\item $\Cl$
	\end{cenumerate}
\end{majorILnc}
\noindent{}Subsentences are counted by token, not type. Hence the similar $\parhorseshoe{\Bl}{\Bl}$ also has three subsentences; the two tokens of $\Bl$ are counted separately.
\begin{majorILnc}{\LnpEC{SubSentenceExampleB}}
$\conjunction{\cpardisjunction{\Al}{\parhorseshoe{\Dl}{\Bl}}}{\negation{\Gl}}$ has 8 subsentences:
\begin{multicols}{2}
\begin{cenumerate}
\item $\conjunction{\cpardisjunction{\Al}{\parhorseshoe{\Dl}{\Bl}}}{\negation{\Gl}}$
\item $\disjunction{\Al}{\parhorseshoe{\Dl}{\Bl}}$
\item $\horseshoe{\Dl}{\Bl}$
\item $\Al$
\item $\Dl$
\item $\Bl$
\item $\negation{\Gl}$
\item $\Gl$
\end{cenumerate}
\end{multicols}
\end{majorILnc}
\begin{majorILnc}{\LnpDC{Proper Subsentences}}
	A sentence $\CAPPHI$ is a \df{proper subsentence} of $\CAPPSI$ \Iff $\CAPPHI$ is a subsentence of, but isn't identical to $\CAPPSI$.
\end{majorILnc}
\noindent{}Thus, while each sentence is a subsentence of itself, no sentence is a proper subsentence of itself.
\begin{majorILnc}{\LnpDC{Order}}
The following clauses define the \df{order} of every \GSL{} sentence.\footnote{\citetext{\citealt{Post1921}, \citealt[11]{Hodges2001}}} Let $\ORD{\CAPPHI}$ be the order of $\CAPPHI$. Then: 
\begin{cenumerate}
\item If $\CAPPHI$ is an atomic sentence (a sentence letter), then $\ORD{\CAPPHI}=1$.
\item For any sentence $\CAPPHI$, $\ORD{\negation{\CAPPHI}}=\ORD{\CAPPHI}+1$.
\item For any sentences $\CAPPHI$ and $\CAPTHETA$, $\ORD{\parhorseshoe{\CAPPHI}{\CAPTHETA}}$ is one greater than the max of $\ORD{\CAPPHI}$ and $\ORD{\CAPTHETA}$. Likewise, $\ORD{\partriplebar{\CAPPHI}{\CAPTHETA}}$ is one greater than the max of $\ORD{\CAPPHI}$ and $\ORD{\CAPTHETA}$.
\item For any sentences $\CAPPHI_1,\ldots,\CAPPHI_\integer{n}$, $\ORD{\parconjunction{\CAPPHI_1}{\conjunction{\ldots}{\CAPPHI_\integer{n}}}}$ is one greater than the max of $\ORD{\CAPPHI_1}$, $\ldots$, $\ORD{\CAPPHI_\integer{n}}$.
\item For any sentences $\CAPPHI_1,\ldots,\CAPPHI_\integer{n}$, $\ORD{\pardisjunction{\CAPPHI_1}{\disjunction{\ldots}{\CAPPHI_\integer{n}}}}$ is one greater than the max of $\ORD{\CAPPHI_1}$, $\ldots$, $\ORD{\CAPPHI_\integer{n}}$. 
\item That's all.
\end{cenumerate}
\end{majorILnc}
\begin{majorILnc}{\LnpEC{OrderExampleA}}
We'll give the order for all the subsentences of $\conjunction{\cpardisjunction{\Al}{\parhorseshoe{\Dl}{\Bl}}}{\negation{\Gl}}$.
The order of an atomic sentence is 1, so $\Al$, $\Dl$, $\Bl$, $\Gl$ each have order 1.
The order of $\horseshoe{\CAPPHI}{\CAPPSI}$ is $1$ plus the maximum of the orders of $\CAPPHI$ and $\CAPPSI$; thus the order of $\horseshoe{\Dl}{\Bl}$ is 2.  
Because $\ORD{\negation{\CAPPHI}}=\ORD{\CAPPHI}+1$, $\ORD{(\negation{\Gl})}=\ORD{(\Gl)}+1=2$. 
Because the order of a disjunction $\disjunction{\CAPPHI_\integer{1}}{\disjunction{\ldots}{\CAPPHI_{\integer{n}}}}$ is $1$ plus the maximum order of the disjuncts, the order of $\disjunction{\Al}{\parhorseshoe{\Dl}{\Bl}}$ is 3.
Similarly for conjunctions, the order of $\conjunction{\cpardisjunction{\Al}{\parhorseshoe{\Dl}{\Bl}}}{\negation{\Gl}}$ is 4.
\end{majorILnc}
\begin{majorILnc}{\LnpDC{GSL Main connective}}
The \nidf{main connective}\index{main connective!of GSL|textbf} is the connective token (or tokens) that occur(s) in the sentence but in no proper subsentence.  
\end{majorILnc}
\begin{majorILnc}{\LnpEC{GSLMainConnectiveExampleA}}
The main connective in each of the following sentences has been underlined.
\begin{multicols}{2}
\begin{cenumerate}
\item $(\Al\VEE(\Dl\HORSESHOE\Bl))\, \underline{\WEDGE}\, \negation{\Gl}$
\item $\Ll\, \underline{\VEE}\, \Kl\, \underline{\VEE}\, \Hl $
\item $\Ll\, \underline{\VEE}\, \parhorseshoe{\Al}{\Bl}\, \underline{\VEE}\, \Hl $
\item $\underline{\NEGATION}(\Ll\VEE\Kl\VEE\Hl)$
\item $(((\Dl\!\HORSESHOE\!\El)\VEE\Al)\underline{\WEDGE}(\NEGATION\Bl\WEDGE\NEGATION(\Cl\!\TRIPLEBAR\!\Hl)))$
\item $(\NEGATION\Bl\, \underline{\WEDGE}\, \NEGATION(\Cl\TRIPLEBAR\Hl))$
\end{cenumerate}
\end{multicols}
\end{majorILnc}

\begin{majorILnc}{\LnpDC{Construction Tree}}
The \df{construction tree} for a sentence is a diagram of how the sentence is generated through the recursive clauses of the definition of \GSL{} sentences. We put atomic sentences as leaves at the top, and the generating clauses specify how we can join nodes of the tree together (starting with the leaves at the top) into new nodes. The complete sentence is the node at the base of the tree. 
\end{majorILnc}
\begin{majorILnc}{\LnpEC{ConstructionTreeExampleA}}
Give the construction tree for $\conjunction{\cpardisjunction{\Al}{\parhorseshoe{\Dl}{\Bl}}}{\negation{\Gl}}$.
\begin{center}
\begin{tikzpicture}[grow=up]
\tikzset{level distance=40pt}
\tikzset{level 1/.style={level distance=60pt}}
\tikzset{sibling distance=32pt}
\tikzset{every tree node/.style={align=center,anchor=north}}
	\Tree%http://angasm.org/papers/qtree/    http://www.ling.upenn.edu/advice/latex/qtree/qtreenotes.pdf
[.{$\conjunction{\cpardisjunction{\Al}{\parhorseshoe{\Dl}{\Bl}}}{\negation{\Gl}}$}
  [.{$\Gl$\\ $\negation{\Gl}$} %!{\qsetw{3in}}
%   [.{$\Gl$}
%   ]
  ]
  [.{$\disjunction{\Al}{\parhorseshoe{\Dl}{\Bl}}$} 
    [.{$\horseshoe{\Dl}{\Bl}$} 
      [.{$\Bl$} 
      ]
      [.{$\Dl$} %!{\qsetw{2in}}
      ] 
    ]
    [.{$\Al$}
    ]    
  ]
]%
	%\caption{Example formula tree}
	%\label{fig:ExampleFormulaTree}
\end{tikzpicture}
\end{center}
\end{majorILnc}
\begin{majorILnc}{\LnpEC{ConstructionTreeExampleB}}
Give the construction tree for $((\Cl \WEDGE \Dl) \HORSESHOE \Al)\TRIPLEBAR (\Dl \VEE \Hl)$.
\begin{center}
\begin{tikzpicture}[grow=up]
\tikzset{level distance=42pt}
\tikzset{sibling distance=32pt}
\tikzset{every tree node/.style={align=center,anchor=north}}
	\Tree%http://angasm.org/papers/qtree/    http://www.ling.upenn.edu/advice/latex/qtree/qtreenotes.pdf
[.{$((\Cl \WEDGE \Dl) \HORSESHOE \Al)\TRIPLEBAR (\Dl \VEE \Hl)$}
  [.{$\Dl\VEE \Hl$} %!{\qsetw{3in}}
    [.{$\Hl$} %!{\qsetw{2in}}
    ]   
    [.{$\Dl$}
    ]  
  ]
  [.{$(\Cl\WEDGE \Dl)\HORSESHOE \Al$} 
    [.{$\Al$} %!{\qsetw{2in}}
    ]  
    [.{$\Cl\WEDGE \Dl$} 
      [.{$\Dl$} %!{\qsetw{2in}}
      ]     
      [.{$\Cl$} 
      ]
    ]
  ]
]%
	%\caption{Example formula tree}
	%\label{fig:ExampleFormulaTree}
\end{tikzpicture}
\end{center}
\end{majorILnc}
\noindent{}There are some helpful relationships between the construction tree of a sentence,  its order, its subsentences, and its main connective. 
The subsentences of a sentence are the nodes in the sentence's construction tree.
The order of a sentence is the number of nodes of its longest branch from root to leaf, i.e., the height of the tree. 
The main connective of a sentence is the connective added last (at the bottom) of the construction tree. 
\begin{majorILnc}{\LnpEC{ConstructionTreeExampleC}}
Consider again the construction tree for $\conjunction{\cpardisjunction{\Al}{\parhorseshoe{\Dl}{\Bl}}}{\negation{\Gl}}$. 
Find the order of the sentence by counting the height of the branches of the tree. 
(We count up as we work our way down the branches.)
Note that the answer we get agrees with that computed in example \mvref{OrderExampleA}.
\begin{center}
\begin{tikzpicture}[grow=up]
\tikzset{level distance=40pt}
\tikzset{level 1/.style={level distance=60pt}}
\tikzset{sibling distance=40pt}
\tikzset{every tree node/.style={align=center,anchor=north}}
	\Tree%http://angasm.org/papers/qtree/    http://www.ling.upenn.edu/advice/latex/qtree/qtreenotes.pdf
[.{$\conjunction{\cpardisjunction{\Al}{\parhorseshoe{\Dl}{\Bl}}}{\negation{\Gl}}$ /4}
  [.{$\Gl$ /1\\ $\negation{\Gl}$ /2} %!{\qsetw{3in}}
%   [.{$\Gl$ /1}
%   ]
  ]
  [.{$\disjunction{\Al}{\parhorseshoe{\Dl}{\Bl}}$ /3} 
    [.{$\horseshoe{\Dl}{\Bl}$ /2} 
      [.{$\Bl$ /1} 
      ]
      [.{$\Dl$ /1} %!{\qsetw{2in}}
      ] 
    ]
    [.{$\Al$ /1}
    ]    
  ]
]%
	%\caption{Example formula tree}
	%\label{fig:ExampleFormulaTree}
\end{tikzpicture}
\end{center}
\end{majorILnc}

\subsection{How Many \GSL{} Sentences are There?}
There are at least as many sentence letters as there are natural numbers (countably infinite), and the sentence letters are a proper subset of the set of sentences. 
Are there more sentences than natural numbers?  
Below we prove there are not by matching up sentences with natural numbers.
\begin{THEOREM}{\LnpTC{Number of sentences}}
The number of \GSL{} sentences is equal to the number of natural numbers.
\end{THEOREM}
\begin{PROOF}
First, assign each sentence letter a natural number that only contains the digit \mention{$1$}, for example:
\begin{center}
\begin{tabular}{ c c c c c }
$\Al$ & $\Bl$ & $\Cl$ & $\Dl$ & $\ldots$ \\
1 & 11 & 111 & 1111 & $\ldots$ \\
\end{tabular}
\end{center}
Next, assign numbers to the other symbols of \GSL{}, for example:
\begin{center}
\begin{tabular}{ c c c c c c c }
$\NEGATION$ & $\WEDGE$ & $\VEE$ & $\HORSESHOE$ & $\TRIPLEBAR$ & ( & ) \\
2 & 3 & 4 & 5 & 6 & 7 & 8 \\
\end{tabular}
\end{center}
Given any sentence, replace its symbols with the associated numbers.
For example, \mbox{$\negation{\parconjunction{\Al}{\conjunction{\Bl}{\negation{\Dl}}}}$} gets mapped to 2713113211118.
For any sentence of \GSL{}, there is a unique natural number defined by this process. 
For any natural number we can determine if it represents an \GSL{} sentence, and if so which one.
\end{PROOF}

\noindent{}This is not the most efficient way of representing sentences with numbers, but it is a simple one that avoids the use of special properties (e.g., being a prime number).


%%%%%%%%%%%%%%%%%%%%%%%%%%%%%%%%%%%%%%%%%%%%%%%%%%
\section{Models}\label{Interpretations}
%%%%%%%%%%%%%%%%%%%%%%%%%%%%%%%%%%%%%%%%%%%%%%%%%%

\GSL{} is a formal language and as was mentioned in section \ref{Formal Languages} sentences of formal languages are mere strings of symbols.
Because they lack inherent meanings---they don't ``say anything'' about the world---sentences of \GSL{} lack a truth value. That is, they are neither true nor false. 
Even so nothing stops us from \emph{assigning} truth values to \GSL{} sentences. 
In this section we explain how to do that consistently. 
You might ask why we don't first assign \GSL{} sentences meanings, then determine whether they are true or false based on those meanings. 
There are difficulties associated with assigning meanings that would complicate our project unnecessarily.\footnote{Later we will \emph{interpret} \GSL{} sentences as having meanings (chapter \ref{Translations}).} The point of giving formal definitions of an \GSL{} sentence and model is to provide a rigorous basis for studying logical truth and logical consequence. 
One of the most important discoveries in logic was that for \GSL{} only truth values matter; all other details of meaning are irrelevant.

\subsection{Truth in a Model}\label{Truth in an Interpretation} 

The truth value of a sentence depends on a model.

\begin{majorILnc}{\LnpDC{Definition of GSL interpretation}}
A \df{model of a \GSL{} sentence $\CAPPHI$} is an assignment of a truth value, either $\True$ or $\False$, to each sentence letter in $\CAPPHI$.
\end{majorILnc}

\noindent{}In mathematical terms, a model of $\CAPPHI$ is a function from the set of sentence letters of $\CAPPHI$ to the set of truth values: $\{\TrueB, \FalseB\}$. 
We use the letter \mention{$\IntA$} (a fraktur-style \mention{m}) as the symbol for a model. 
If a model $\IntA$ assigns a sentence letter $\CAPPSI$ the value $\True$, we write $\IntA(\CAPPSI)=\TrueB$.
For the value $\False$, we instead write $\IntA(\CAPPSI)=\FalseB$.

To illustrate, a model for $\pardisjunction{\disjunction{\Al}{\Bl}}{\Cl}$ must make a truth value assignment to each of the sentence letters $\Al$, $\Bl$, and $\Cl$.  Any model for $\pardisjunction{\disjunction{\Al}{\Bl}}{\Cl}$ is therefore also a model for $\parconjunction{\conjunction{\Al}{\Bl}}{\Cl}$ and $\parconjunction{\parhorseshoe{\Al}{\Bl}}{\negation{\Cl}}$; they all have the same sentence letters.

We often speak informally of \emph{models} without making reference to any particular \GSL{} sentence.  It's useful to talk this way because any given model of $\CAPPHI$ is a model of any other \GSL{} sentence with the same sentence letters, or a subset of them.  If $\IntA$ makes assignments to $\Al$, $\Bl$, and $\Cl$, then $\IntA$ is a model of all the sentences that only contain sentence letters from that list.  Accordingly, we define a model for a \emph{set} of \GSL{} sentences.

\begin{majorILnc}{\LnpDC{Definition of Model for Set}}
	$\IntA$ is a \df{model of a set of sentences $\Delta$} \Iff $\IntA$ is a model for each sentence in $\Delta$.
\end{majorILnc}

Consider a model $\IntA$ that makes a truth value assignment to every sentence letter of \GSL{}.  No sentence letter lacks an assignment, so it follows that $\IntA$ is a model for the set of all \GSL{} sentences.  The following definition characterizes such models as \emph{models of \GSL{}}.

\begin{majorILnc}{\LnpDC{Definition of Model for SL}}
	$\IntA$ is a \df{model of \GSL{}} \Iff $\IntA$ is a model of every sentence of \GSL{}.
\end{majorILnc}

Models can be uniform; for example, there is a model that assigns $\True$ to every sentence letter.  Or they can be systematic and mildly complicated, such as the model that alternately assigns $\True$ or $\False$ to a list of sentence letters.
A model can assign truth values in any other pattern you can think of, or even random assignments. 
\emph{Every} possible function from sentence letters to truth values is a model.

Our models assign one of two truth values to each sentence letter, but that's not the only way to define them. Other definitions of \mention{model} assign more than two.  Our assumption that there are only two truth values simplifies analysis and is widely shared, but whether two is enough is a matter of intense philosophical debate.  Nevertheless, even if our definition of a model is a simplification, it is a historically fruitful one and helps us better understand logical consequence.  In chapter \ref{furtherdirections} we discuss formal languages with additional truth values. 

A model of $\CAPPHI$ only assigns truth values to the sentence letters of $\CAPPHI$. It does not directly assign a truth value to any of the non-atomic sentences of \GSL{}. Thus, we must specify how each of the logical connectives operates on lower-order truth values in order to fix a truth value, in some given model, for non-atomic \GSL{} sentences.  While the truth assignments to the sentence letters vary by model, the truth functions of the connectives do not.\footnote{We discuss \emph{truth functions} further in section \ref{Truth Functions Truth Tables and Boolean Operators}.}

\begin{majorILnc}{\LnpDC{True on a GSL interpretation}} The following clauses define whether an \GSL{} sentence $\CAPTHETA$ is \nidf{$\True$} or \nidf{$\False$} on a model $\IntA$ for $\CAPTHETA$. The relevant clause is determined by which main connective $\CAPTHETA$ has, if any:
\begin{cenumerate}
\item $\CAPTHETA$ is a sentence letter. $\CAPTHETA$ is $\True$ on $\IntA$ \Iff $\IntA$ assigns $\True$ to it, i.e. $\IntA(\CAPTHETA)=\TrueB$.
\item $\CAPTHETA$ is of the form $\negation{\CAPPHI}$ (i.e. is a negation). $\CAPTHETA$ is $\True$ on $\IntA$ \Iff $\CAPPHI$ is $\False$ on $\IntA$.
\item\label{GSL true conjunction} $\CAPTHETA$ is of the form $\parconjunction{\CAPPHI_1}{\conjunction{\CAPPHI_2}{\conjunction{\ldots}{\CAPPHI_{\integer{n}}}}}$ (i.e. is a conjunction).   $\CAPTHETA$ is $\True$ on $\IntA$ \Iff each of the conjuncts $\CAPPHI_1, \CAPPHI_2, \ldots, \CAPPHI_n$ is $\True$ on $\IntA$.
\item $\CAPTHETA$ is of the form $\pardisjunction{\CAPPHI_1}{\disjunction{\CAPPHI_2}{\disjunction{\ldots}{\CAPPHI_{\integer{n}}}}}$ (i.e. is a disjunction). $\CAPTHETA$ is $\True$ on $\IntA$ \Iff at least one of the disjuncts $\CAPPHI_1, \CAPPHI_2, \ldots, \CAPPHI_n$ is $\True$ on $\IntA$.
\item\label{GSL true horseshoe} $\CAPTHETA$ is of the form $\horseshoe{\CAPPHI}{\CAPPSI}$ (i.e. is a conditional). $\CAPTHETA$ is $\True$ on $\IntA$ \Iff the \CAPS{lhs} $\CAPPHI$ is $\False$ or the \CAPS{rhs} $\CAPPSI$ is $\True$ on $\IntA$ (or both).
\item $\CAPTHETA$ is of the form $\triplebar{\CAPPHI}{\CAPPSI}$ (i.e., is a biconditional). $\CAPTHETA$ is $\True$ on $\IntA$ \Iff $\CAPPHI$ and $\CAPPSI$ have the same truth value on $\IntA$.
\item A sentence is $\False$ on $\IntA$ \Iff it's not $\True$ on $\IntA$.
\end{cenumerate}
\end{majorILnc}

For every model $\IntA$ for sentence $\CAPPHI$, the above definition (i.e., the definition of truth) fixes a unique truth value for $\CAPPHI$.\footnote{If a model $\IntA$ is \emph{not} a model for some \GSL{} sentence $\CAPPHI$, then $\IntA$ does \emph{not} fix a truth value for $\CAPPHI$.}

Although there are an infinite number of \GSL{} sentences letters to which a model can assign truth values, only what the model assigns to a finite number of those sentence letters matters for determining the value of any given sentence.  
Unsurprisingly, only sentence letters that appear in the sentence matter. (We prove this later in the chapter.) 
For example, when assessing the value of $\horseshoe{\Al}{\Bl}$ on $\IntA$ only the assignments to $\Al$ and $\Bl$ are relevant; other assignments may be disregarded. It follows that if two models $\IntA_1$ and $\IntA_2$ assign the same truth values to $\Al$ and $\Bl$, respectively, then they fix the same truth value for $\horseshoe{\Al}{\Bl}$. 

If $\IntA$ makes assignments to $\Al$ and $\Bl$ but no other sentence letters, we say that $\IntA$ is a \emph{minimal model} for $\horseshoe{\Al}{\Bl}$. More generally:

\begin{majorILnc}{\LnpDC{Definition of Minimal SL Model}}
	$\IntA$ is a \df{minimal model of $\CAPPHI$} \Iff $\IntA$ makes assignments to every sentence letter in $\CAPPHI$ but to no other sentence letters.
\end{majorILnc}

\noindent{}There are only four minimal models for the sentence $\horseshoe{\Al}{\Bl}$, because there are only $4$ combinations of truth values that can be assigned to $\Al$ and $\Bl$. The number of distinct minimal models for a sentence $\CAPPHI$ is $2^n$, where $n$ is the number of sentence letter (types) in $\CAPPHI$.

There are several different ways to compute the truth value of an \GSL{} sentence in a model.
We demonstrate some informal ones in the following examples, and then develop a systematic method in section \mvref{Proceduresfortesting}. 

\begin{majorILnc}{\LnpEC{GSLTVExampleA}}
	Give the truth value of $\disjunction{\Al}{\negation{\Bl}}$ on a model $\IntA$ such that $\IntA(\Al)=\FalseB$ and $\IntA(\Bl)=\FalseB$.
	
	One way to compute the truth value of this sentence in $\IntA$ is to read off the values of all the subsentences, using definition \ref{True on a GSL interpretation} (the definition of truth in \GSL{}), until we finally get to the sentence itself.
	
	\begin{PROOF}	
		From the negation clause of the definition of truth and the fact that $\IntA(\Bl)=\FalseB$, it follows that $\negation{\Bl}$ is true on $\IntA$. According to the disjunction clause of the definition of truth $\disjunction{\Al}{\negation{\Bl}}$ is true on $\IntA$ \Iff at least one of the disjuncts is true, i.e., $\Al$ and $\negation{\Bl}$. So because $\negation{\Bl}$ is true on $\IntA$, $\disjunction{\Al}{\negation{\Bl}}$ is too.
	\end{PROOF}

\end{majorILnc}

\begin{majorILnc}{\LnpEC{GSLTVExampleB}}
	Give the truth value of $\conjunction{\cpardisjunction{\Al}{\parhorseshoe{\Dl}{\Bl}}}{\negation{\Gl}}$ on a model $\IntA$ such that $\IntA(\Al)=\FalseB$, $\IntA(\Dl)=\TrueB$, $\IntA(\Bl)=\TrueB$, and $\IntA(\Gl)=\TrueB$.
	
	When computing the truth value for a complicated sentence it can help to be strategic about which subsentences to look at first. Often you don't need to determine the value of every subsentence. The main connective can be an important clue about where to start. The sentence $\conjunction{\cpardisjunction{\Al}{\parhorseshoe{\Dl}{\Bl}}}{\negation{\Gl}}$ is a conjunction, which is false on a model if any one of its conjuncts is.
	
	\begin{PROOF}
		From the negation clause of the definition of truth and the fact that $\IntA(\Gl)=\TrueB$, it follows that $\negation{\Gl}$ is $\False$ on $\IntA$. From the conjunction clause of the definition of truth, $\conjunction{\cpardisjunction{\Al}{\parhorseshoe{\Dl}{\Bl}}}{\negation{\Gl}}$ is true on $\IntA$ \Iff each conjunct is. But $\negation{\Gl}$ is false on $\IntA$, so $\conjunction{\cpardisjunction{\Al}{\parhorseshoe{\Dl}{\Bl}}}{\negation{\Gl}}$ is too.
	\end{PROOF}
	
	One way to make sure we compute the truth values of the subsentences in an appropriate order is by following the branches in the construction tree of the sentence. 
	The idea is to start at the top of the construction tree (the truth values of which are given by the model) and work our way down the branches. 
	We can even use the tree itself as an aid to computing truth values, writing the truth value of each subsentence next to it on the tree as we go. 
	Let's illustrate with $\conjunction{\cpardisjunction{\Al}{\parhorseshoe{\Dl}{\Bl}}}{\negation{\Gl}}$ on $\IntA$:
	\begin{center}
		\begin{tikzpicture}[grow=up]
		\tikzset{level distance=40pt}
		\tikzset{level 1/.style={level distance=60pt}}
		\tikzset{sibling distance=32pt}
		\tikzset{every tree node/.style={align=center,anchor=north}}
		\Tree%http://angasm.org/papers/qtree/    http://www.ling.upenn.edu/advice/latex/qtree/qtreenotes.pdf
		[.{$\conjunction{\cpardisjunction{\Al}{\parhorseshoe{\Dl}{\Bl}}}{\negation{\Gl}}$ /$\FalseB$}
		[.{$\negation{\Gl}$ /$\FalseB$} %!{\qsetw{3in}}
		[.{$\Gl$ /$\TrueB$}
		]
		]
		%[.{$\Gl$ /$\TrueB$\\ $\negation{\Gl}$ /$\FalseB$} %!{\qsetw{3in}}
		%%   [.{$\Gl$ /$\TrueB$}
		%%   ]
		%]
		[.{$\disjunction{\Al}{\parhorseshoe{\Dl}{\Bl}}$ /$\TrueB$} 
		[.{$\horseshoe{\Dl}{\Bl}$ /$\TrueB$} 
		[.{$\Bl$ /$\FalseB$} 
		]
		[.{$\Dl$ /$\FalseB$} %!{\qsetw{2in}}
		] 
		]
		[.{$\Al$ /$\TrueB$}
		]    
		]
		]%
		%\caption{Example formula tree}
		%\label{fig:ExampleFormulaTree}
		\end{tikzpicture}
	\end{center}
\end{majorILnc}
\begin{majorILnc}{\LnpEC{GSLTVExampleC}}
	Compute the truth value of the sentence $((\Cl \WEDGE \Dl) \HORSESHOE \Al)\TRIPLEBAR (\Dl \VEE \Hl)$ on a model $\IntA$ such that $\IntA(\Cl)=\TrueB$, $\IntA(\Dl)=\TrueB$, $\IntA(\Al)=\FalseB$, and $\IntA(\Hl)=\TrueB$.
	
	We again use a construction tree to illustrate how to work toward the answer.
	\begin{center}
		\begin{tikzpicture}[grow=up]
		\tikzset{level distance=42pt}
		\tikzset{sibling distance=32pt}
		\Tree%http://angasm.org/papers/qtree/    http://www.ling.upenn.edu/advice/latex/qtree/qtreenotes.pdf
		[.{$((\Cl \WEDGE \Dl) \HORSESHOE \Al)\TRIPLEBAR (\Dl \VEE \Hl)$ /$\FalseB$}
		[.{$\Dl\VEE \Hl$ /$\TrueB$} %!{\qsetw{3in}}
		[.{$\Hl$ /$\TrueB$} %!{\qsetw{2in}}
		]  
		[.{$\Dl$ /$\TrueB$}
		]
		]
		[.{$(\Cl\WEDGE \Dl)\HORSESHOE \Al$ /$\FalseB$} 
		[.{$\Al$ /$\FalseB$} %!{\qsetw{2in}}
		]   
		[.{$\Cl\WEDGE \Dl$ /$\TrueB$}
		[.{$\Dl$ /$\TrueB$} %!{\qsetw{2in}}
		]     
		[.{$\Cl$ /$\TrueB$} 
		]
		]
		]
		]%
		%\caption{Example formula tree}
		%\label{fig:ExampleFormulaTree}
		\end{tikzpicture}
	\end{center}
\end{majorILnc}
%\begin{majorILnc}{\LnpEC{GSLTVExampleC}}
%
%\end{majorILnc}
 
\subsection{Truth Functions and Truth Tables}\label{Truth Functions Truth Tables and Boolean Operators}
A truth function is any function $f:\{\TrueB,\FalseB\}\times\ldots\times\{\TrueB,\FalseB\}\Rightarrow\{\TrueB,\FalseB\}$, i.e., from sequences of truth values to truth values.  The middle clauses of the definition of truth in a model---\mvref{True on a GSL interpretation}---associate each logical connective of \GSL{} with a truth function. 
Often these truth functions are given by a truth table, though they can be written out explicitly in other ways.  
For example, the truth function associated with conditionals in the definition of truth can be represented as:
\begin{center} 
	$f(\TrueB,\TrueB)=\TrueB$ \\
	$f(\TrueB,\FalseB)=\FalseB$ \\
	$f(\FalseB,\TrueB)=\TrueB$ \\
	$f(\FalseB,\FalseB)=\TrueB$ \\
\end{center}
Or, for short:
\begin{menumerate} 
	\item\hspace{1in}$f(v_1,v_2)=
	\begin{cases}
	\FalseB{} & \text{ if } v_1=\TrueB\text{ and }v_2=\FalseB \\
	\TrueB{} & \text{ otherwise}
	\end{cases}$
\end{menumerate}
The \emph{truth table} for the conditional is:
\begin{menumerate}
	\item\hspace{2in}\begin{tabular}[t]{c | c c}
		$\HORSESHOE$ & $\TrueB$ & $\FalseB$ \\
		\hline
		& & \\[-.25cm]
		$\TrueB$ & $\TrueB$ & $\FalseB$ \\
		$\FalseB$ & $\TrueB$ & $\TrueB$  
	\end{tabular}
\end{menumerate}
or can be written alternatively as:
\begin{menumerate}
	\item\hspace{1.9in}\begin{tabular}[t]{c c c}
		$\CAPPHI$ & $\CAPPSI$ & $\horseshoe{\CAPPHI}{\CAPPSI}$ \\
		\hline 
		& & \\[-.25cm]
		$\TrueB$ & $\TrueB$ & $\TrueB$ \\
		$\TrueB$ & $\FalseB$ & $\FalseB$ \\
		$\FalseB$ & $\TrueB$ & $\TrueB$ \\
		$\FalseB$ & $\FalseB$ & $\TrueB$ \\
	\end{tabular}
\end{menumerate}

The truth functions for $\NEGATION$ and $\WEDGE$ are good translations of \mention{not} and \mention{and} in English, respectively.  The symbol $\VEE$ is a good translation of the inclusive sense of \mention{or} in which the overall sentence is true \Iff at least one component is true.  When we come to translations in chapter \ref{Translations}, we will discuss the exclusive sense of \mention{or} for which the overall sentence is true just in case exactly one component is true. Our truth function for $\HORSESHOE$ is the best truth-functional translation of \mention{if...then...} in English, but its adequacy is a matter of controversy. The debate began more than 2,000 years ago, when the ancient philosophers Diodorus Cronus and Philo of Megara argued about whether conditionals could adequately be captured by truth-functional semantics. Perhaps the next 2,000 years will bring the controversy to a close.

\subsection{Logical Truth: TFT, TFF, \& TFC}\label{TFT TFF TFI}

A randomly chosen sentence will probably be true on some models and false on others. 
However, some special sentences are true on all models. 
One example of such is the sentence $\disjunction{\Al}{\negation{\Al}}$.  Others are false on all models, e.g. $\conjunction{\Al}{\negation{\Al}}$.
\begin{majorILnc}{\LnpDC{GSL TFT}}
A sentence $\CAPPHI$ of \GSL{} is \nidf{truth functionally true}\index{truth!truth functional|textbf} (\CAPS{tft})\index{TFT|see{truth, truth functional}} \Iff it is $\True$ on all models for $\CAPPHI$.
\end{majorILnc}
\begin{majorILnc}{\LnpDC{GSL TFF}}
A sentence $\CAPPHI$ of \GSL{} is \nidf{truth functionally false}\index{falsehood!truth functional|textbf} (\CAPS{tff})\index{TFF|see{falsehood, truth functional}} \Iff it is $\False$ on all models for $\CAPPHI$.
\end{majorILnc}
\begin{majorILnc}{\LnpDC{GSL TFI}}
A sentence $\CAPPHI$ of \GSL{} is \nidf{truth functionally contingent}\index{indeterminate!truth functional|textbf} (\CAPS{tfc})\index{TFI|see{indeterminate, truth functional}} \Iff it is $\True$ on one model for $\CAPPHI$ and $\False$ on another. 
\end{majorILnc}
\begin{majorILnc}{\LnpEC{TFTExampleA}}
	Prove that $\conjunction{\Al}{\negation{\Al}}$ is \CAPS{tff}. 
	\begin{PROOF}
		Any model $\IntA$ for $\conjunction{\Al}{\negation{\Al}}$ has to assign either $\TrueB$ or $\FalseB$ to $\Al$. 
		If it assigns $\TrueB$ to $\Al$, then $\negation{\Al}$ will be false on $\IntA$.
		So $\conjunction{\Al}{\negation{\Al}}$ is false on $\IntA$.
		But if $\IntA$ assigns $\FalseB$ to $\Al$, then again $\conjunction{\Al}{\negation{\Al}}$ is false on $\IntA$. 
		Either way, the sentence is false on $\IntA$.
		This holds in all models $\IntA$, so $\conjunction{\Al}{\negation{\Al}}$ is \CAPS{tff}.
	\end{PROOF}
\end{majorILnc}
\begin{majorILnc}{\LnpEC{TFTExampleB}}
Similar reasoning shows that any sentence of the form $\conjunction{\CAPPHI}{\negation{\CAPPHI}}$ is \CAPS{tff}.
By definition \mvref{True on a GSL interpretation}, $\CAPPHI$ is either true or false on any model $\IntA$. 
If $\CAPPHI$ is false on $\IntA$, then $\conjunction{\CAPPHI}{\negation{\CAPPHI}}$ is false on $\IntA$.
If $\CAPPHI$ is true on $\IntA$, then $\negation{\CAPPHI}$ is false on $\IntA$ and the conjunction $\conjunction{\CAPPHI}{\negation{\CAPPHI}}$ is false on $\IntA$.
Either way, $\conjunction{\CAPPHI}{\negation{\CAPPHI}}$ is false on $\IntA$, and this holds for all models. 
\end{majorILnc}
\begin{majorILnc}{\LnpEC{TFTExampleC}}
Similar reasoning will show that any sentence of the form $\disjunction{\CAPPHI}{\negation{\CAPPHI}}$ is \CAPS{tft}.
We leave it to the reader to adapt the argument. 
\end{majorILnc}
\begin{majorILnc}{\LnpEC{TFTExampleE}}
In examples \mvref{GSLTVExampleA} and \mvref{GSLTVExampleC} we saw that the sentence $\conjunction{\cpardisjunction{\Al}{\parhorseshoe{\Dl}{\Bl}}}{\negation{\Gl}}$ is false in some models.
To see that it's true in some models, consider a model that's just like the one used in example \mvref{GSLTVExampleA}, but assigns $\FalseB$ to $\Gl$ instead of $\TrueB$. 
In that model the sentence is true. 
Hence the sentence is \CAPS{tfc}. 
\end{majorILnc}
\begin{majorILnc}{\LnpEC{TFTExampleD}}
The sentence $\horseshoe{\Bl}{\parhorseshoe{\Cl}{\Bl}}$ is \CAPS{tft}.
Any model $\IntA$ will assign either $\TrueB$ or $\FalseB$ to $\Bl$. 
If it assigns $\FalseB$ to $\Bl$, then $\horseshoe{\Bl}{\parhorseshoe{\Cl}{\Bl}}$ is true on $\IntA$, because, according to the def. of truth for $\HORSESHOE$, a conditional is true if the \CAPS{lhs} is false.
If $\IntA$ assigns $\TrueB$ to $\Bl$, then $\horseshoe{\Cl}{\Bl}$ is true on $\IntA$, again because of the def. of truth for $\HORSESHOE$. 
But if $\horseshoe{\Cl}{\Bl}$ is true on $\IntA$, it follows that $\horseshoe{\Bl}{\parhorseshoe{\Cl}{\Bl}}$ is true on $\IntA$. 
\end{majorILnc}%
\begin{majorILnc}{\LnpEC{TFTExampleF}}
The sentence $\conjunction{\negation{\Cl}}{\bparconjunction{\parhorseshoe{\Bl}{\Cl}}{\Bl}}$ is \CAPS{tff}.
There are a few ways we can show this, but here we'll use a different approach than we used in the previous examples.
Assume that the sentence \emph{isn't} \CAPS{tff}.
Then there is some model $\IntA$ that makes it true.
By def. of truth for $\WEDGE$, it follows that both conjuncts $\negation{\Cl}$ and $\conjunction{\parhorseshoe{\Bl}{\Cl}}{\Bl}$ are true on this $\IntA$.
If $\negation{\Cl}$ is true on $\IntA$, then $\Cl$ is false on $\IntA$; so $\IntA$ assigns $\FalseB$ to $\Cl$.
If $\conjunction{\parhorseshoe{\Bl}{\Cl}}{\Bl}$ is true on $\IntA$, then both conjuncts are true; 
so both $\horseshoe{\Bl}{\Cl}$ and $\Bl$ are true on $\IntA$.
Thus $\Cl$ is true on $\IntA$ too; 
so $\IntA$ assigns $\TrueB$ to $\Cl$.
But $\IntA$ can't assign both $\FalseB$ and $\TrueB$ to $\Cl$, so there can't be any model $\IntA$ that makes $\conjunction{\negation{\Cl}}{\bparconjunction{\parhorseshoe{\Bl}{\Cl}}{\Bl}}$ true. 
\end{majorILnc}

The method of demonstration used in example \ref{TFTExampleF} is called \emph{Indirect Proof} or \emph{reductio ad absurdum} (or sometimes just \mention{\emph{reductio}} or \mention{RAA}).  On this method, we assume the opposite of our desired conclusion and then work our way to a contradiction.  Because we know that a contradiction can't be true, we then infer that the original assumption must be false.  For many problems and theorems, RAA is the easiest method to use.  It is called \mention{\emph{reductio ad absurdum}} because it `reduces' the initial assumption to a contradiction, an absurdity.

Although the definitions of \CAPS{tft}, \CAPS{tff}, and \CAPS{tfc} are specific to \GSL{} (after all, the models to which each definition refers are models of \GSL{}), we can use essentially the same definitions for \emph{any} formal language for which we have some notion of a model. 
Broadly speaking, we can think of sentences which fit these definitions as being (respectively) \niidf{logically true}\index{logical!truth}\index{truth!logical}, \niidf{logically false}\index{logical!falsehood}\index{falsehood!logical}, and \niidf{logically contingent}\index{logical!indeterminate}\index{indeterminate!logical}. Hereafter we'll sometimes use the more general term \mention{logical truth} instead of \mention{truth functional truth} when it's clear we're talking about \GSL{}.
A sentence that's a logical truth is sometimes said to be \niidf{valid}, or said to be a \niidf{tautology}.\index{sentence!valid|see{truth, logical}}\index{tautology|see{truth, logical}} 
We will avoid these terms in this context.  

\subsection{Procedures for Testing TFT, TFF, \& TFC}\label{Proceduresfortesting}

In the above examples (\ref{TFTExampleA}--\ref{TFTExampleF}) we used the definitions of \CAPS{tft}, \CAPS{tff}, \& \CAPS{tfc} (definitions \ref{GSL TFT}--\ref{GSL TFI}) and the definition for truth (definition \mvref{True on a GSL interpretation}) to show whether a given \GSL{} sentence was \CAPS{tft}, \CAPS{tff}, or \CAPS{tfc}. 
But there are methods or procedures for systematically testing whether a given \GSL{} sentence is \CAPS{tft}, \CAPS{tff}, or \CAPS{tfc}. 

Perhaps the most well known method is using truth tables.%
\footnote{%
On a historical note, Hodges \citeyearpar[5]{Hodges2001} says that Peirce \citeyearpar{Peirce1902} was the first to use truth tables, but his student Ladd-Franklin \citeyearpar[62]{LaddFranklin1883} had something similar.
} 
Later in this chapter we'll prove theorem \mvref{thm:localityoftruth}, which says that when looking at a given \GSL{} sentence $\CAPPHI$ we only need to think about the \emph{minimal} models that assign truth values to the sentence letters appearing in $\CAPPHI$. 
One way to test whether $\CAPPHI$ is \CAPS{tft}, \CAPS{tff}, or \CAPS{tfc} is to write down all the possible assignments of truth values to the sentence letters of $\CAPPHI$, then for each compute the truth value of $\CAPPHI$ on that assignment. 
The number of possible assignments (minimal models) are finite. 
In fact if there are $\integer{n}$ sentence letters in $\CAPPHI$ there will be $2^{\integer{n}}$ possible assignments. 
If $\CAPPHI$ turns out true in all these possible assignments, then $\CAPPHI$ is \CAPS{tft}. 
If it turns out false in all of them, then $\CAPPHI$ is \CAPS{tff}. 
And if it is true in some assignments and false in others, then $\CAPPHI$ is \CAPS{tfc}. 

We can diagrammatically represent this procedure by arranging all the possible assignments of truth values to sentences letters appearing in the given sentence $\CAPPHI$ and the truth value of $\CAPPHI$ on those assignments in rows. 
We will use $\conjunction{\cpardisjunction{\Al}{\parhorseshoe{\Dl}{\Bl}}}{\negation{\Gl}}$ as an example. 
If $\CAPPHI$ has $\integer{n}$ sentence letters, there will be $2^{\integer{n}}$ rows. 
Because this sentence has 4 sentence letters, its table will have 16 rows. 
We write the 4 sentence letters on a top row and then fill out the assignments by starting at the far right sentence letter ($\Gl$), putting $\TrueB$ and $\FalseB$ alternating below it for the 16 rows. (See table \ref{truthtableexample}.) 
\begin{table}[!ht]
\begin{center}
\begin{tabular}{ c c c c c}
$\Al$ & $\Bl$ & $\Dl$ & $\Gl$ & $\parconjunction{\cpardisjunction{\Al}{\parhorseshoe{\Dl}{\Bl}}}{\negation{\Gl}}$ \\
\hline
$ $ & $ $ & & & \\[-.25cm]
$\TrueB$ & $\TrueB$ & $\TrueB$ & $\TrueB$ & $\FalseB$ \\
$\TrueB$ & $\TrueB$ & $\TrueB$ & $\FalseB$& $\TrueB$ \\
$\TrueB$ & $\TrueB$ & $\FalseB$ & $\TrueB$ & $\FalseB$ \\
$\TrueB$ & $\TrueB$ & $\FalseB$ & $\FalseB$  & $\TrueB$ \\
$\TrueB$ &  $\FalseB$& $\TrueB$ & $\TrueB$	&$\FalseB$ \\
$\TrueB$ & $\FalseB$ & $\TrueB$ & $\FalseB$	& $\TrueB$  \\
$\TrueB$ &$\FalseB$  & $\FalseB$& $\TrueB$	&$\FalseB$ \\
$\TrueB$ & $\FalseB$ &$\FalseB$	& $\FalseB$	& $\TrueB$ \\
$\FalseB$	& $\TrueB$ & $\TrueB$ & $\TrueB$	& $\FalseB$ \\
$\FalseB$	& $\TrueB$ & $\TrueB$ & $\FalseB$	& $\TrueB$ \\
$\FalseB$	& $\TrueB$ & $\FalseB$&	$\TrueB$ &$\FalseB$ \\
$\FalseB$	& $\TrueB$ & $\FalseB$& $\FalseB$	& $\TrueB$ \\
$\FalseB$	& $\FalseB$	& $\TrueB$ & $\TrueB$	&$\FalseB$ \\
$\FalseB$	& $\FalseB$	& $\TrueB$ & $\FalseB$	& $\FalseB$ \\
$\FalseB$	& $\FalseB$	& $\FalseB$& $\TrueB$	& $\FalseB$ \\
$\FalseB$	& $\FalseB$& $\FalseB$& $\FalseB$	& $\TrueB$ \\
\end{tabular}
\end{center}
\caption{Sample Truth Table}
\label{truthtableexample}
\end{table}
Next, move to the second-to-the-right sentence letter ($\Dl$) and write 2 $\TrueB$'s and 2 $\FalseB$'s alternating below it for 16 rows. 
Then move to the sentence letter to the left of the last one ($\Bl$) and write 4 $\TrueB$'s and 4 $\FalseB$'s alternating below it until all the rows are filled. 
Finally, move to the last sentence letter ($\Al$) and write 8 $\TrueB$'s and 8 $\FalseB$'s alternative below it until all the rows are filled. 
This will fill out the 16 rows in such a way that each row gives a distinct possible assignment of truth values to sentence letters and no possible assignment is missed. 
In general the pattern is to start at the far right column alternating $\TrueB$ and $\FalseB$, then move to the left doubling the number of $\TrueB$'s and $\FalseB$'s that appeared in the previous column. 

With all the possible assignments to the sentence letters filled out, we write the sentence itself to the right of the sentence letters and under it put, in the respective rows, its truth value on that assignment. 
The truth value of the sentence on the given assignment can be computed in any number of ways, e.g., using trees as was done in example \mvref{GSLTVExampleB}. 
Once the truth value of the sentence has been computed for all the rows it's a trivial matter to read off the table whether the sentence is \CAPS{tft}, \CAPS{tff}, or \CAPS{tfc}. 
If $\TrueB$ is under the sentence in every row, then it's \CAPS{tft}. 
If $\FalseB$ is in every row, then it's \CAPS{tff}.
And if each $\TrueB$ and $\FalseB$ appear in at least one row, then it's \CAPS{tfc}. 
By looking at table \ref{truthtableexample} we can see that $\parconjunction{\cpardisjunction{\Al}{\parhorseshoe{\Dl}{\Bl}}}{\negation{\Gl}}$ is \CAPS{tfc}.

The process of filling out a truth table is entirely \mention{mechanical.}  That is, one can carry out the process by following purely formal rules.  Once we have a truth table for some \GSL{} sentence, we can again use purely formal rules to determine whether it's \CAPS{tft}, \CAPS{tff}, or \CAPS{tfc}.  Therefore, no creativity is necessary to determine whether any given sentence is, e.g., \CAPS{tft}.

\begin{majorILnc}{\LnpEC{TFTExampleA2}}
We already saw in example \mvref{TFTExampleA} that $\conjunction{\Al}{\negation{\Al}}$ is \CAPS{tff}. 
The following truth table confirms this. 
\begin{center}
\begin{tabular}{ c c }
$\Al$ & $\conjunction{\Al}{\negation{\Al}}$ \\
\hline
$ $ & $ $ \\[-.25cm]
$\TrueB$ & $\FalseB$ \\
$\FalseB$ & $\FalseB$ \\
\end{tabular}
\end{center}
\end{majorILnc}
\begin{majorILnc}{\LnpEC{TFTExampleD2}}
In example \mvref{TFTExampleD} we saw that $\horseshoe{\Bl}{\parhorseshoe{\Cl}{\Bl}}$ is \CAPS{tft}.
This is confirmed by the following truth table. 
\begin{center}
\begin{tabular}{ c c c }
$\Bl$ & $\Cl$ & $\horseshoe{\Bl}{\parhorseshoe{\Cl}{\Bl}}$ \\
\hline
$ $ & $ $ & $ $ $ $ \\[-.25cm]
$\TrueB$ & $\TrueB$ & $\TrueB$ \\
$\TrueB$ & $\FalseB$& $\TrueB$ \\
$\FalseB$ & $\TrueB$ & $\TrueB$ \\
$\FalseB$ & $\FalseB$  & $\TrueB$ \\
\end{tabular}
\end{center}
\end{majorILnc}
\begin{majorILnc}{\LnpEC{TFTExampleF2}}
We saw in example \mvref{TFTExampleF} that the sentence $\conjunction{\negation{\Cl}}{\bparconjunction{\parhorseshoe{\Bl}{\Cl}}{\Bl}}$ is \CAPS{tff}.
The following truth table confirms this.  
\begin{center}
\begin{tabular}{ c c c }
$\Bl$ & $\Cl$ & $\conjunction{\negation{\Cl}}{\bparconjunction{\parhorseshoe{\Bl}{\Cl}}{\Bl}}$ \\
\hline
$ $ & $ $ & \\[-.25cm]
$\TrueB$ & $\TrueB$ & $\FalseB$ \\
$\TrueB$ & $\FalseB$& $\FalseB$ \\
$\FalseB$ & $\TrueB$ & $\FalseB$ \\
$\FalseB$ & $\FalseB$  & $\FalseB$ \\
\end{tabular}
\end{center}
\end{majorILnc} 

The various formal derivation systems used in logic provide another class of procedures for testing \GSL{} sentences for \CAPS{tft} and \CAPS{tff} (not all derivation systems provide a procedure for testing for \CAPS{tfc}, but semantic tableaux do). 
In chapter \ref{Derivations} we develop one such derivation system, which is a specific variant of what's called a natural deduction system.\index{natural deduction} 
Some derivation systems, in particular variants of semantic tableaux,\index{semantic tableaux} provide more or less direct ways of testing for \CAPS{tft} (logical truths), \CAPS{tff} (logical falsehoods), and \CAPS{tfc} (logical contingencies).%
\footnote{%
A comprehensive introductory treatment of semantic tableaux (called truth trees by the author) is given in Nicholas J.J. Smith's \citeyearpar{Smith2012} textbook \emph{Logic: The Laws of Truth}. 
Smith also gives a more detailed and thorough introduction to truth tables. 
}
Other derivation systems, such as variants of natural deduction systems, Hilbert-style axiomatic systems, and Gentzen-style sequent calculi, don't on their own provide direct means of testing for them. 
But with these specific algorithms or ways of using the derivation systems can be developed (as we do in chapter \ref{completenesschapter}) that provide a testing procedure. 

While truth tables are a convenient method for answering many questions about \GSL{} sentences, two warnings are in order.  First, for complex sentences the size of the truth table can be very large.  Remember, the number of rows needed for a truth table is $2^n$ where $n$ is the number of different sentence letters.  Second, while truth tables are useful in \GSL{}, there is nothing comparable for our later formal languages.  The sooner you learn to analyze sentences directly, rather than relying on a truth table, the better.  For example, consider the sentence $\disjunction{\parhorseshoe{\Al}{\parconjunction{\Bl}{\negation{\pardisjunction{\Cl}{\Dl}}}}}{\parhorseshoe{\El}{\Al}}$.  You \emph{could} evaluate this sentence with a 32 line truth table.  Or, you can instead note that if $\Al$ is true in a model, then $\parhorseshoe{\El}{\Al}$ is also true (def. of truth for $\HORSESHOE$), and so the whole sentence is true; and if $\Al$ is false on the model, then $\parhorseshoe{\Al}{\parconjunction{\Bl}{\negation{\pardisjunction{\Cl}{\Dl}}}}$ is also true (def. of truth for $\HORSESHOE$), and the whole sentence is true.  Every model must make $\Al$ either true or false, so the whole sentence is true either way.  Therefore it is \CAPS{TFT}.

%%%%%%%%%%%%%%%%%%%%%%%%%%%%%%%%%%%%%%%%%%%%%%%%%%
\section{Entailment and other Relations}
%%%%%%%%%%%%%%%%%%%%%%%%%%%%%%%%%%%%%%%%%%%%%%%%%%

\subsection{Entailment}\label{Entailment}
We've discussed what it is for a sentence of \GSL{} to be true in a model and the classification of \CAPS{tft}, \CAPS{tff}, and \CAPS{tfc} sentences. 
Now we turn to the notion of entailment. 
Entailment is a 2-place relation that holds between sets of sentences and individual sentences.  The symbol \mention{$\:\sdtstile{}{}\:$}, called the double turnstile,\index{double turnstile}\index{$\sdtstile{}{}$} is used to represent entailment. 
We write that $\Delta$ entails $\CAPTHETA$ as follows: \mention{\:$\Delta\sdtstile{}{}\CAPTHETA\:$}. 
(The \mention{\:$\sdtstile{}{}\:$} is \emph{not} a symbol of \GSL{}. 
Like the Greek letters it is a symbol of MathEnglish.)

\begin{majorILnc}{\LnpDC{GSL Generalized Further Entailment}}
	If $\Delta$ is a set of \GSL{} sentences and $\CAPTHETA$ is an \GSL{} sentence, then the following are equivalent ways to define when $\Delta$ entails $\CAPTHETA$:
	\begin{cenumerate}
		\item $\Delta\sdtstile{}{}\CAPTHETA$ \Iff every model for $\Delta$ and $\CAPTHETA$ that makes all sentences in $\Delta$ $\True$ also makes $\CAPTHETA$ $\True$.
		\item $\Delta\sdtstile{}{}\CAPTHETA$ \Iff every model for $\Delta$ and $\CAPTHETA$ either makes at least one sentence in $\Delta$ $\False$ or makes $\CAPTHETA$ $\True$.
	\end{cenumerate}
\end{majorILnc}

\noindent{}$\Delta$ can be the empty set or an infinite set. 
If $\Delta$ is the empty set then we simply write \mention{$\:\sdtstile{}{}\CAPTHETA$}. The following are a consequence of the definition of entailment:

\begin{cenumerate}
		\item $\CAPPHI_1,\CAPPHI_2,\CAPPHI_3,\ldots,\CAPPHI_{\integer{n}}\sdtstile{}{}\CAPTHETA$ \Iff every model for $\CAPPHI_1$, $\CAPPHI_2$, $\CAPPHI_3$, $\ldots$, $\CAPPHI_{\integer{n}}$, and $\CAPTHETA$ that makes all of $\CAPPHI_1$, $\CAPPHI_2$, $\CAPPHI_3$, $\ldots$ $\CAPPHI_{\integer{n}}$ $\True$ also makes $\CAPTHETA$ $\True$.
		\item $\CAPPHI_1,\CAPPHI_2,\CAPPHI_3,\ldots,\CAPPHI_{\integer{n}}\sdtstile{}{}\CAPTHETA$ \Iff every model for $\CAPPHI_1$, $\CAPPHI_2$, $\CAPPHI_3$, $\ldots$, $\CAPPHI_{\integer{n}}$, and $\CAPTHETA$ either makes at least one of $\CAPPHI_1$, $\CAPPHI_2$, $\CAPPHI_3$, $\ldots$ $\CAPPHI_{\integer{n}}$ $\False$ or makes $\CAPTHETA$ $\True$.
\end{cenumerate}


\noindent{}And for an even more narrow consequence:

\begin{cenumerate}
\item A sentence $\CAPPHI$ {entails} another sentence $\CAPTHETA$ \Iff every model for $\CAPPHI$ and $\CAPTHETA$ that makes $\CAPPHI$ $\True$ also makes $\CAPTHETA$ $\True$.
\item A sentence $\CAPPHI$ {entails} another sentence $\CAPTHETA$ \Iff there is no model for $\CAPPHI$ and $\CAPTHETA$ that makes $\CAPPHI$ $\True$ and $\CAPTHETA$ $\False$.
\end{cenumerate}


\noindent{}Above we mentioned that \:$\Delta\sdtstile{}{}\CAPTHETA\:$ may hold even in cases where $\Delta$ is the empty set: $\:\sdtstile{}{}\CAPTHETA$.  If $\:\sdtstile{}{}\CAPTHETA$, then every model for $\CAPTHETA$ must make a sentence to the left of the turnstile $\False$, or make $\CAPTHETA$ true. 
But there are no sentences to the left of the turnstile for a model to make false. 
So, every model for $\CAPTHETA$ must make $\CAPTHETA$ true. 
Therefore:
\begin{THEOREM}{\LnpTC{entailmentTFT theorem}}
For all \GSL{} sentences $\CAPTHETA$, $\:\sdtstile{}{}\CAPTHETA$ \Iff $\CAPTHETA$ is \CAPS{tft}.
\end{THEOREM} 
\noindent{}Before moving on to examples, it's worth noting that the definition of entailment given here only makes sense for sentences of \GSL{}.
It can't be applied directly to meaningful English sentences. 
But as we discuss in section \mvref{SLApplications}, through translations between \GSL{} and English we can begin to talk about entailment between English sentences. 
And this will be at least a start to elucidating the notion of logical consequence in English.\index{logical consequence}

\begin{majorILnc}{\LnpEC{GSLEntailmentExA}}
$\parconjunction{\Al}{\Bl}\sdtstile{}{}\Bl$, that is, any model that makes $\parconjunction{\Al}{\Bl}$ true makes $\Bl$ true as well. 
\end{majorILnc}
\begin{PROOF}
We must show that for all models $\IntA$, if $\IntA$ makes $\parconjunction{\Al}{\Bl}$ true, then it also makes $\Bl$ true. 
Assume a model $\IntA$ makes $\parconjunction{\Al}{\Bl}$ true. 
By the definition of truth for $\WEDGE$---clause 3, \mvref{True on a GSL interpretation}---both $\Al$ and $\Bl$ are true in $\IntA$ as well. 
\end{PROOF}
\begin{majorILnc}{\LnpEC{GSLEntailmentExB}}
$\Bl\sdtstile{}{}\pardisjunction{\Al}{\Bl}$. We leave the proof to the reader.
\end{majorILnc}
\begin{majorILnc}{\LnpEC{GSLEntailmentExC}}
For all $\CAPPHI$, for some $\CAPTHETA$, $\CAPPHI\sdtstile{}{}\CAPTHETA$. That is, every \GSL{} sentence entails at least one \GSL{} sentence.
\end{majorILnc}
\begin{PROOF}
We need to show that no matter what \GSL{} sentence $\CAPPHI$ you pick, there's always some \GSL{} sentence $\CAPTHETA$ which is entailed by it, i.e. always some $\CAPTHETA$ such that $\CAPPHI\sdtstile{}{}\CAPTHETA$. 
But this is easy: for every \GSL{} sentence $\CAPPHI$ you pick, just let $\CAPTHETA=\CAPPHI$. 
Every \GSL{} sentence entails itself.  Thus, no matter what \GSL{} sentence you pick, there's always some sentence entailed by it. 
\end{PROOF}
\begin{majorILnc}{\LnpEC{GSLEntailmentExD}}
For some $\CAPTHETA$, for all $\CAPPHI$, $\CAPPHI\sdtstile{}{}\CAPTHETA$. That is, there's some \GSL{} sentence that's entailed by all \GSL{} sentences.
\end{majorILnc}
\begin{PROOF}
We need to find some $\CAPTHETA$ such that all \GSL{} sentences entail it. 
Let $\CAPTHETA$ be $\pardisjunction{\Al}{\negation{\Al}}$. 
This sentence is \CAPS{tft}, so no model makes it false. 
No matter what sentence $\CAPPHI$ we pick, there is no model that makes $\CAPPHI$ true and $\CAPTHETA$ false. 
So, by the definition of $\:\sdtstile{}{}$ (\ref{GSL Generalized Further Entailment}), $\CAPPHI$ entails $\CAPTHETA$.
\end{PROOF}
\begin{majorILnc}{\LnpEC{GSLEntailmentExE}}
For some $\CAPPHI$, for all $\CAPTHETA$, $\CAPPHI\sdtstile{}{}\CAPTHETA$. We leave the proof to the reader.
\end{majorILnc}

\subsection{Procedures for Testing Entailment}\label{TestingEntailment}

In the previous examples (\ref{GSLEntailmentExA}--\ref{GSLEntailmentExE}) we used the definition of entailment (definition \ref{GSL Generalized Further Entailment}) to reason about whether a given entailment holds. 
But there are various procedures for mechanically checking whether a given entailment holds. 
Truth tables provide the simplest way. 

To test whether a set of sentences $\Delta$ entails a sentence $\CAPTHETA$, we first write down all the sentence letters that appear in $\CAPTHETA$ and that appear in the sentences in $\Delta$. 
Again we put these on the left side of the truth table and under them (following the procedure outlined in section \pmvref{Proceduresfortesting}) we write all the possible assignments of truth values. 
Then to the right of the sentence letters we write each of the sentences from $\Delta$ (in separate columns) and to the right of those we write $\CAPTHETA$. 
Under $\CAPTHETA$ and all the sentences from $\Delta$ we write the truth value of that sentence based on the assignment of truth values listed in that row. 
Then $\Delta$ entails $\CAPTHETA$ iff there is no row in the truth table in which all the sentences in $\Delta$ are true and $\CAPTHETA$ is false. Any rows of the truth table that do not make all of the \CAPS{lhs} sentences true are irrelevant.  After marking one of the \CAPS{lhs} sentences as false we can omit the rest of the row.
\begin{majorILnc}{\LnpEC{GSLEntailmentExA2}}
In example \ref{GSLEntailmentExA} we said that $\parconjunction{\Al}{\Bl}\sdtstile{}{}\Bl$. 
We can show this with a truth table.  
\begin{center}
\begin{tabular}{ c c c c }
$\Al$ & $\Bl$ & $\parconjunction{\Al}{\Bl}$ & $\Bl$ \\
\hline
$ $ & $ $ & & \\[-.25cm]
$\TrueB$ & $\TrueB$ & $\TrueB$ & $\TrueB$ \\
$\TrueB$ & $\FalseB$& $\FalseB$ & $\FalseB$ \\
$\FalseB$ & $\TrueB$ & $\FalseB$ & $\TrueB$ \\
$\FalseB$ & $\FalseB$  & $\FalseB$ & $\FalseB$ \\
\end{tabular}
\end{center}
\end{majorILnc}
\begin{majorILnc}{\LnpEC{GSLEntailmentExB2}}
In example \ref{GSLEntailmentExB} we said that $\Bl\sdtstile{}{}\pardisjunction{\Al}{\Bl}$. 
The following truth table shows this. 
\begin{center}
\begin{tabular}{ c c c c }
$\Al$ & $\Bl$ & $\Bl$ & $\pardisjunction{\Al}{\Bl}$ \\
\hline
$ $ & $ $ & & \\[-.25cm]
$\TrueB$ & $\TrueB$ & $\TrueB$ & $\TrueB$ \\
$\TrueB$ & $\FalseB$& $\FalseB$ & $\TrueB$ \\
$\FalseB$ & $\TrueB$ & $\TrueB$ & $\TrueB$ \\
$\FalseB$ & $\FalseB$  & $\FalseB$ & $\FalseB$ \\
\end{tabular}
\end{center}
\end{majorILnc}
\begin{majorILnc}{\LnpEC{GSLEntailmentExAA}}
We conclude with a more complicated example. 
Here we want to show that $\disjunction{\Al}{\Bl},\horseshoe{\negation{\Cl}}{\negation{\Al}},\horseshoe{\Bl}{\Cl}\sdtstile{}{}\Cl$.
\begin{center}
\begin{tabular}{ c c c c c c c }
$\Al$ & $\Bl$ & $\Cl$ & $\disjunction{\Al}{\Bl}$ & $\horseshoe{\negation{\Cl}}{\negation{\Al}}$ & $\horseshoe{\Bl}{\Cl}$ & $\Cl$ \\
\hline
$ $ & $ $ & & & & & \\[-.25cm]
$\TrueB$ & $\TrueB$ & $\TrueB$ & $\TrueB$ & $\TrueB$ & $\TrueB$ & $\TrueB$\\
$\TrueB$ & $\TrueB$ & $\FalseB$& $\TrueB$ & $\FalseB$ & $\FalseB$ &  $\FalseB$\\
$\TrueB$ & $\FalseB$ & $\TrueB$ & $\TrueB$ & $\TrueB$ & $\TrueB$ &  $\TrueB$\\
$\TrueB$ & $\FalseB$ & $\FalseB$  & $\TrueB$ & $\FalseB$ & $\TrueB$ & $\FalseB$\\
$\FalseB$ & $\TrueB$ & $\TrueB$ & $\TrueB$ & $\TrueB$ & $\TrueB$ & $\TrueB$\\
$\FalseB$ & $\TrueB$ & $\FalseB$& $\TrueB$ & $\TrueB$ & $\FalseB$ & $\FalseB$\\
$\FalseB$ & $\FalseB$ & $\TrueB$ & $\FalseB$ & $\TrueB$ & $\TrueB$ & $\TrueB$\\
$\FalseB$ & $\FalseB$ & $\FalseB$  & $\FalseB$ & $\TrueB$ & $\TrueB$ & $\FalseB$\\
\end{tabular}
\end{center}
\end{majorILnc}

\subsection{Basic Results on Entailment}\label{Basic Results on Entailment} 
A simple, but important theorem involves moving sentences from one side of the turnstile to the other.
\begin{THEOREM}{\LnpTC{Exponentiation of Entailment} \GSL{} Exportation Theorem:} For all \GSL{} sentences $\CAPPHI$ and $\CAPTHETA$, $\CAPPHI\sdtstile{}{}\CAPTHETA$ \Iff $\:\sdtstile{}{}\horseshoe{\CAPPHI}{\CAPTHETA}$.
\end{THEOREM}
\noindent{}In order to prove an iff-statement like this one (i.e., a biconditional), we need to prove two things. 
First, we have to prove that if the left-hand side---in this example, $\CAPPHI\sdtstile{}{}\CAPTHETA$---is true, then the right-hand side---in this example, $\:\sdtstile{}{}\horseshoe{\CAPPHI}{\CAPTHETA}$---must also be true. 
Second, we have to prove that if the right-hand side is true, then the left-hand side must also be true. 
Proofs of a biconditional generally involve these two parts.\footnote{There are other ways of proving a biconditional that don't involve explicitly carrying out these steps, but the method we use here is often the most straightforward.} 
The convention is that the first part of the proof (the one that shows that if the left-hand side is true, then the right-hand side is true) will be marked with a left-to-right arrow \mention{$\Rightarrow$},\index{$\Leftarrow$, $\Rightarrow$} and the other part of the proof will be marked with a right-to-left arrow \mention{$\Leftarrow$}. 
\begin{PROOF}
$(\Rightarrow)$ To begin, we assume that the left-hand side, $\CAPPHI\sdtstile{}{}\CAPTHETA$, is true.  We want to show, without using any additional information, that the right-hand side, $\:\sdtstile{}{}\horseshoe{\CAPPHI}{\CAPTHETA}$, is also true.  Given the definition of $\:\sdtstile{}{}$ and the truth of $\CAPPHI\sdtstile{}{}\CAPTHETA$, it follows that on every model on which $\CAPPHI$ is true, $\CAPTHETA$ is also true.

There are two possibilities for the truth value of $\CAPPHI$: either (a) $\CAPPHI$ is true or (b) $\CAPPHI$ is false.  (a) Assume a model $\IntA$ such that $\CAPPHI$ is true.  Because $\CAPPHI\sdtstile{}{}\CAPTHETA$, that means that $\IntA$ makes $\CAPTHETA$ true.  By the definition of truth for $\HORSESHOE$, $\IntA$ makes $\parhorseshoe{\CAPPHI}{\CAPTHETA}$ true.  (If a model makes the \CAPS{rhs} of a conditional true, it makes the whole conditional true.)  Remember this result!  This is close to our goal.

Now for the second possibility.  (b) Assume a model $\IntA$ that makes $\CAPPHI$ false.  According to the definition of truth for $\HORSESHOE$, $\IntA$ makes $\parhorseshoe{\CAPPHI}{\CAPTHETA}$ true. (If a model makes the \CAPS{lhs} of a conditional false, it makes the whole conditional true.)

No matter what model we pick, $\CAPPHI$ must be true or false.  Either way, we see---in (a) and (b)---that the model must make $\parhorseshoe{\CAPPHI}{\CAPTHETA}$ true.  In other words, all models make $\parhorseshoe{\CAPPHI}{\CAPTHETA}$ true.  Which in turn means that $\parhorseshoe{\CAPPHI}{\CAPTHETA}$ is \CAPS{tft} (def. of \CAPS{tft}, \pmvref{GSL TFT}).  And according to theorem \mvref{entailmentTFT theorem}, it follows that the entailment $\:\sdtstile{}{}\parhorseshoe{\CAPPHI}{\CAPTHETA}$ holds.  

$(\Leftarrow)$ Now assume that $\:\sdtstile{}{}\parhorseshoe{\CAPPHI}{\CAPTHETA}$ is true, with the goal of showing that $\CAPPHI\sdtstile{}{}\CAPTHETA$ is also true.  By the definition of $\:\sdtstile{}{}$, the truth of $\:\sdtstile{}{}\parhorseshoe{\CAPPHI}{\CAPTHETA}$ implies that $\parhorseshoe{\CAPPHI}{\CAPTHETA}$ is \CAPS{tft}.  By the definition of \CAPS{tft}, this means that $\parhorseshoe{\CAPPHI}{\CAPTHETA}$ is true on every model $\IntA$.

By the definition of truth of $\HORSESHOE$, this means that for every model $\IntA$, either the \CAPS{lhs}, $\CAPPHI$, is false or the \CAPS{rhs}, $\CAPTHETA$, is true.  Hence there is no model that makes $\CAPPHI$ true and $\CAPTHETA$ false.  So, by the definition of $\:\sdtstile{}{}$, that means that $\CAPPHI\sdtstile{}{}\CAPTHETA$.
\end{PROOF}

There are a number of other theorems that are similar to, and expand upon, theorem \ref{Exponentiation of Entailment}. Here we state them and leave the proofs to the reader.
\begin{THEOREM}
{\LnpTC{expo generalizations}}
\begin{cenumerate}
\item If $\CAPPHI_1$, $\CAPPHI_2$, $\ldots$, $\CAPPHI_{\integer{n}}$ and $\CAPPSI$ are \GSL{} sentences, then
\begin{itemize}
\item[] $\CAPPHI_1,\CAPPHI_2,\ldots,\CAPPHI_{\integer{n}}\sdtstile{}{}\CAPPSI$ \Iff $\CAPPHI_2,\ldots,\CAPPHI_{\integer{n}}\sdtstile{}{}\parhorseshoe{\CAPPHI_1}{\CAPPSI}$
\item[] $\CAPPHI_1,\CAPPHI_2,\ldots,\CAPPHI_{\integer{n}}\sdtstile{}{}\CAPPSI$ \Iff $\CAPPHI_1,\CAPPHI_3,\ldots,\CAPPHI_{\integer{n}}\sdtstile{}{}\parhorseshoe{\CAPPHI_2}{\CAPPSI}$
\item[] \hspace{1in} $\vdots$
\item[] $\CAPPHI_1,\CAPPHI_2,\ldots,\CAPPHI_{n}\sdtstile{}{}\CAPPSI$ \Iff $\CAPPHI_1,\ldots,\CAPPHI_{n-1}\sdtstile{}{}\parhorseshoe{\CAPPHI_{\integer{n}}}{\CAPPSI}$
\end{itemize} 
\item If $\CAPPHI_1$, $\CAPPHI_2$, $\ldots$, $\CAPPHI_{\integer{n}}$ and $\CAPPSI$ are \GSL{} sentences, then
\begin{itemize}
\item[] $\CAPPHI_1,\CAPPHI_2,\ldots,\CAPPHI_{\integer{n}}\sdtstile{}{}\CAPPSI$ \Iff $\CAPPHI_1,\ldots,\CAPPHI_{n-1}\sdtstile{}{}\parhorseshoe{\CAPPHI_{\integer{n}}}{\CAPPSI}$
\item[] $\CAPPHI_1,\CAPPHI_2,\ldots,\CAPPHI_{\integer{n}}\sdtstile{}{}\CAPPSI$ \Iff $\CAPPHI_1,\ldots,\CAPPHI_{n-2}\sdtstile{}{}\parhorseshoe{\CAPPHI_{n-1}}{\parhorseshoe{\CAPPHI_{\integer{n}}}{\CAPPSI}}$
\item[] \hspace{1in} $\vdots$
\item[] $\CAPPHI_1,\CAPPHI_2,\ldots,\CAPPHI_{n}\sdtstile{}{}\CAPPSI$ \Iff $\sdtstile{}{}\parhorseshoe{\CAPPHI_1}{\parhorseshoe{\ldots}{\parhorseshoe{\CAPPHI_{n-1}}{\parhorseshoe{\CAPPHI_{\integer{n}}}{\CAPPSI}}}}$
\end{itemize}
\item If $\CAPPHI$ and $\CAPPSI$ are \GSL{} sentences and $\Delta$ is a set of \GSL{} sentences containing $\CAPPHI$ (i.e, $\CAPPHI\in\Delta$) and $\Delta^*$ is $\Delta$ with $\CAPPHI$ removed, then $\Delta\sdtstile{}{}\CAPPSI$ \Iff $\Delta^*\sdtstile{}{}\parhorseshoe{\CAPPHI}{\CAPPSI}$.
\item If $\CAPPHI$ and $\CAPPSI$ are \GSL{} sentences, then $\CAPPHI\sdtstile{}{}\CAPPSI$ \Iff $\CAPPHI,\negation{\CAPPSI}\sdtstile{}{}\parconjunction{\Al}{\negation{\Al}}$.
\item If $\CAPPHI_1,\ldots,\CAPPHI_{\integer{n}}$ and $\CAPPSI_1,\ldots,\CAPPSI_{\integer{n}}$ are all \GSL{} sentences, then
\begin{itemize}
\item[] $\CAPPHI_1,\ldots,\CAPPHI_{\integer{n}}\sdtstile{}{}\pardisjunction{\CAPPSI_1}{\disjunction{\ldots}{\CAPPSI_{\integer{n}}}}$ \Iff $\CAPPHI_1,\ldots,\CAPPHI_{\integer{n}},\negation{\CAPPSI_1}\sdtstile{}{}\pardisjunction{\CAPPSI_2}{\disjunction{\ldots}{\CAPPSI_{\integer{n}}}}$
\item[] $\CAPPHI_1,\ldots,\CAPPHI_{\integer{n}}\sdtstile{}{}\pardisjunction{\CAPPSI_1}{\disjunction{\ldots}{\CAPPSI_{\integer{n}}}}$ \Iff $\CAPPHI_1,\ldots,\CAPPHI_{\integer{n}},\negation{\CAPPSI_2}\sdtstile{}{}\pardisjunction{\CAPPSI_1}{\disjunction{\CAPPSI_3}{\disjunction{\ldots}{\CAPPSI_{\integer{n}}}}}$
\item[] \hspace{1in} $\vdots$
\item[] $\CAPPHI_1,\ldots,\CAPPHI_{\integer{n}}\sdtstile{}{}\pardisjunction{\CAPPSI_1}{\disjunction{\ldots}{\CAPPSI_{\integer{n}}}}$ \Iff $\CAPPHI_1,\ldots,\CAPPHI_{\integer{n}},\negation{\CAPPSI_{\integer{n}}}\sdtstile{}{}\pardisjunction{\CAPPSI_1}{\disjunction{\ldots}{\CAPPSI_{n-1}}}$
\end{itemize} 
\end{cenumerate}
\end{THEOREM}

\subsection{Other Relations}\label{Other Relations}

We can define a number of other important relations between \GSL{} sentences in terms of entailment:

\begin{majorILnc}{\LnpDC{GSL TFE}}
Two sentences $\CAPTHETA$ and $\CAPPHI$ are \index{equivalent sentences!truth functional|textbf} \nidf{truth functionally equivalent} (\CAPS{tfe}) \Iff all models for $\CAPTHETA$ and $\CAPPHI$ assign them the same truth value, which is the same as saying they entail each other: both $\CAPTHETA\sdtstile{}{}\CAPPHI$ and $\CAPPHI\sdtstile{}{}\CAPTHETA$.
\end{majorILnc}
\begin{majorILnc}{\LnpDC{GSL Contradictory}}
Two sentences $\CAPTHETA$ and $\CAPPHI$ are \nidf{truth functionally contradictory}\index{contradictory!truth functional|textbf} \Iff all models for $\CAPTHETA$ and $\CAPPHI$ assign them opposite truth values, which is the same as saying that each sentence is \CAPS{tfe} to the negation of the other.
\end{majorILnc}
\begin{majorILnc}{\LnpDC{GSL Contrary}}
Two sentences $\CAPTHETA$ and $\CAPPHI$ are \nidf{truth functionally contrary}\index{contraries!truth functional|textbf} \Iff they cannot both be $\True$ in the same model $\IntA$. (This is the same as saying that each entails the negation of the other, or $\conjunction{\CAPTHETA}{\CAPPHI}\sdtstile{}{}\conjunction{\Al}{\negation{\Al}}$.)
\end{majorILnc}
\begin{majorILnc}{\LnpDC{GSL subcontrary}}
Two sentences $\CAPTHETA$ and $\CAPPHI$ are \nidf{truth functionally subcontrary}\index{subcontraries!truth functional|textbf} \Iff they cannot both be $\False$ in the same model $\IntA$. (This is the same as saying that the negation of each entails the other, or $\sdtstile{}{}\disjunction{\CAPTHETA}{\CAPPHI}$.)
\end{majorILnc}
\begin{majorILnc}{\LnpDC{GSL Independence}}
Two sentences $\CAPTHETA$ and $\CAPPHI$ are \nidf{truth functionally independent}\index{independent sentences!truth functional|textbf} \Iff none of the above hold (including entailments), i.e. \Iff there are four models:
\begin{cenumerate}
\item A model in which both $\CAPTHETA$ and $\CAPPHI$ are $\True$; 
\item A model in which both $\CAPTHETA$ and $\CAPPHI$ are $\False$;
\item A model in which $\CAPTHETA$ is $\True$ and $\CAPPHI$ is $\False$; and
\item A model in which $\CAPTHETA$ is $\False$ and $\CAPPHI$ is $\True$.
\end{cenumerate}
\end{majorILnc}
\begin{majorILnc}{\LnpEC{TFE Ex 4}}
Each pair of contradictory sentences is also contrary (this follows trivially from the definitions). 
But sentences can be contrary without being contradictory.
\end{majorILnc}
\begin{PROOF}
$\conjunction{\Cl}{\Dl}$ and $\conjunction{\Cl}{\negation{\Dl}}$ are contrary but not contradictory.

(Contrary:) If $\conjunction{\Cl}{\Dl}$ is true in a model $\IntA$, then $\Cl$ and $\Dl$ are each true on $\IntA$. 
So $\negation{\Dl}$ is false in $\IntA$, and $\conjunction{\Cl}{\negation{\Dl}}$ is false in $\IntA$.
If $\conjunction{\Cl}{\negation{\Dl}}$ is true in $\IntA$, then both $\Cl$ and $\negation{\Dl}$ are true on $\IntA$.
$\Dl$ is false in $\IntA$, and hence $\conjunction{\Cl}{\Dl}$ is false in $\IntA$.
Thus, by def. \ref{GSL Contrary}, the pair is contrary.

(Not Contradictory:) Any model $\IntA$ that assigns $\FalseB$ to $\Cl$ makes both $\conjunction{\Cl}{\Dl}$ and $\conjunction{\Cl}{\negation{\Dl}}$ false.
By def. \ref{GSL Contradictory}, the pair is not contradictory.
\end{PROOF}
\begin{majorILnc}{\LnpEC{TFE Ex 5}}
Contradictory sentences are also subcontrary, but sentences can be subcontrary without being contradictory. 
\end{majorILnc}
\begin{PROOF}
$\Dl$ and $\disjunction{\Cl}{\negation{\Dl}}$ are subcontrary but not contradictory.

(Subcontrary:) Assume that $\Dl$ is false on a model $\IntA$.
It follows that $\negation{\Dl}$ is true on $\IntA$ and so is $\disjunction{\Cl}{\negation{\Dl}}$. 
Alternatively, assume that $\disjunction{\Cl}{\negation{\Dl}}$ is false on $\IntA$.  Then both $\Cl$ and $\negation{\Dl}$ are false on $\IntA$.
So $\Dl$ is true on $\IntA$.
By def. \ref{GSL subcontrary}, the pair is subcontrary.

(Not Contradictory:) Any model that assigns $\TrueB$ to both $\Dl$ and $\Cl$ makes both $\Dl$ and $\disjunction{\Cl}{\negation{\Dl}}$ true. 
So by def. \ref{GSL Contradictory}, the pair isn't contradictory.
\end{PROOF}
\begin{majorILnc}{\LnpEC{TFE Ex 6}}
If two sentences are both contrary and subcontrary, they are contradictory. 
\end{majorILnc}
\begin{PROOF}
If two sentences are contrary, then by definition \ref{GSL Contrary} any model $\IntA$ that makes one true makes the other false. 
If two sentences are subcontrary, then by definition \ref{GSL subcontrary} any model $\IntA$ that makes one false makes the other true. 
Because every model either makes a sentence true or makes it false, no model assigns two sentences that are contrary and subcontrary the same truth value. 
So by definition \ref{GSL Contradictory}, two sentences that are contrary and subcontrary are also contradictory. 
\end{PROOF}
\begin{majorILnc}{\LnpEC{TFE Ex 1}}
$\disjunction{\Al}{\parconjunction{\Bl}{\Dl}}$ and $\conjunction{\pardisjunction{\Al}{\Bl}}{\pardisjunction{\Al}{\Dl}}$ are \CAPS{tfe}.
\end{majorILnc}
\begin{PROOF}
Assume that $\disjunction{\Al}{\parconjunction{\Bl}{\Dl}}$ is true on some $\IntA$. 
By the def. of truth of $\VEE$, either $\Al$ is true on $\IntA$ or $\conjunction{\Bl}{\Dl}$ is true on $\IntA$.
If $\Al$ is true on $\IntA$, then both $\disjunction{\Al}{\Bl}$ and $\disjunction{\Al}{\Dl}$ are true on $\IntA$.
So, $\conjunction{\pardisjunction{\Al}{\Bl}}{\pardisjunction{\Al}{\Dl}}$ is true on $\IntA$. 
If, alternatively, $\conjunction{\Bl}{\Dl}$ is true on $\IntA$, then then both $\Bl$ and $\Dl$ are true on $\IntA$, and so both $\disjunction{\Al}{\Bl}$ and $\disjunction{\Al}{\Dl}$ are true on $\IntA$.
Hence, $\conjunction{\pardisjunction{\Al}{\Bl}}{\pardisjunction{\Al}{\Dl}}$ is true on $\IntA$. 
In either case, $\conjunction{\pardisjunction{\Al}{\Bl}}{\pardisjunction{\Al}{\Dl}}$ is true on $\IntA$.
Thus,  $\disjunction{\Al}{\parconjunction{\Bl}{\Dl}}\sdtstile{}{}\conjunction{\pardisjunction{\Al}{\Bl}}{\pardisjunction{\Al}{\Dl}}$.

We leave it to the reader to show that $\conjunction{\pardisjunction{\Al}{\Bl}}{\pardisjunction{\Al}{\Dl}}\sdtstile{}{}\disjunction{\Al}{\parconjunction{\Bl}{\Dl}}$. 
It then follows by definition \ref{GSL TFE} that the two sentences are \CAPS{tfe}.
\end{PROOF}
\begin{majorILnc}{\LnpEC{TFE Ex 2}}
For any \GSL{} sentences $\CAPPHI$ and $\CAPTHETA$, $\horseshoe{\CAPPHI}{\CAPTHETA}$ and $\disjunction{\negation{\CAPPHI}}{\CAPTHETA}$ are \CAPS{tfe}.
\end{majorILnc}
\begin{PROOF}
Assume that $\horseshoe{\CAPPHI}{\CAPTHETA}$ is true on some model $\IntA$.
Then on $\IntA$ either $\CAPPHI$ is false or $\CAPTHETA$ is true. 
Hence either $\negation{\CAPPHI}$ or $\CAPTHETA$ is true on $\IntA$. 
It follows that $\disjunction{\negation{\CAPPHI}}{\CAPTHETA}$ is true on $\IntA$. 
Hence by the definition of $\:\sdtstile{}{}$, $\horseshoe{\CAPPHI}{\CAPTHETA}\sdtstile{}{}\disjunction{\negation{\CAPPHI}}{\CAPTHETA}$.

We leave it to the reader to show that $\disjunction{\negation{\CAPPHI}}{\CAPTHETA}\sdtstile{}{}\horseshoe{\CAPPHI}{\CAPTHETA}$.
It then follows by definition \ref{GSL TFE} that the two sentences are \CAPS{tfe}.
\end{PROOF}
\begin{majorILnc}{\LnpEC{TFE Ex 3}}
For any \GSL{} sentence $\CAPPHI$, $\negation{\negation{\CAPPHI}}$ and $\CAPPHI$ are \CAPS{tfe}.
\end{majorILnc}
\begin{PROOF}
The Proof is left to the reader.
\end{PROOF}

\subsection{Procedures for Testing Other Relations} 

As before we can use truth tables to test for the following relationships: truth-functional equivalence, truth-functional contradictory, truth-functional contrary, truth-functional subcontrary, and truth-functional independence. 
By definition, two sentences $\CAPPHI$ and $\CAPTHETA$ are truth functionally equivalent iff they have the same truth value on every assignment. 
So we can test for truth-functional equivalence by putting both $\CAPPHI$ and $\CAPTHETA$ in a truth table and seeing if they have the same truth value in every row. 
\begin{majorILnc}{\LnpEC{TFE Ex 1 2}}
In example \mvref{TFE Ex 1} we said that $\disjunction{\Al}{\parconjunction{\Bl}{\Dl}}$ and $\conjunction{\pardisjunction{\Al}{\Bl}}{\pardisjunction{\Al}{\Dl}}$ are \CAPS{tfe}. 
We can show this using truth tables. 
\begin{center}
\begin{tabular}{ c c c c c }
$\Al$ & $\Bl$ & $\Dl$ & $\disjunction{\Al}{\parconjunction{\Bl}{\Dl}}$ & $\conjunction{\pardisjunction{\Al}{\Bl}}{\pardisjunction{\Al}{\Dl}}$ \\
\hline
$ $ & $ $ & & & \\[-.25cm]
$\TrueB$ & $\TrueB$ & $\TrueB$ & $\TrueB$ & $\TrueB$ \\
$\TrueB$ & $\TrueB$ & $\FalseB$& $\TrueB$ & $\TrueB$ \\
$\TrueB$ & $\FalseB$ & $\TrueB$ & $\TrueB$ & $\TrueB$\\
$\TrueB$ & $\FalseB$ & $\FalseB$  & $\TrueB$ & $\TrueB$\\
$\FalseB$ & $\TrueB$ & $\TrueB$ & $\TrueB$ & $\TrueB$\\
$\FalseB$ & $\TrueB$ & $\FalseB$& $\FalseB$ & $\FalseB$\\
$\FalseB$ & $\FalseB$ & $\TrueB$ & $\FalseB$ & $\FalseB$\\
$\FalseB$ & $\FalseB$ & $\FalseB$  & $\FalseB$ & $\FalseB$\\
\end{tabular}
\end{center}
\end{majorILnc}
\begin{majorILnc}{\LnpEC{TFE Ex 2 2}}
In example \mvref{TFE Ex 2} we said that for any \GSL{} sentences $\CAPPHI$ and $\CAPTHETA$, $\horseshoe{\CAPPHI}{\CAPTHETA}$ and $\disjunction{\negation{\CAPPHI}}{\CAPTHETA}$ are \CAPS{tfe}. 
This can be shown using truth tables. 
\begin{center}
\begin{tabular}{ c c c c c }
$\CAPPHI$ & $\CAPTHETA$ & $\horseshoe{\CAPPHI}{\CAPTHETA}$ & $\disjunction{\negation{\CAPPHI}}{\CAPTHETA}$ & \\
\hline
$ $ & $ $ & & & \\[-.25cm]
$\TrueB$ & $\TrueB$ & $\TrueB$ & $\TrueB$ \\
$\TrueB$ & $\FalseB$& $\FalseB$ & $\FalseB$ \\
$\FalseB$ & $\TrueB$ & $\TrueB$ & $\TrueB$\\
$\FalseB$ & $\FalseB$  & $\TrueB$ & $\TrueB$\\
\end{tabular}
\end{center}
In this example we're not looking at actual \GSL{} sentences; instead we have sentence schemas. 
But for reasons to be discussed below this procedure still works: this truth table shows that for any two sentences $\CAPPHI$ and $\CAPTHETA$, $\horseshoe{\CAPPHI}{\CAPTHETA}$ and $\disjunction{\negation{\CAPPHI}}{\CAPTHETA}$ are \CAPS{tfe}.
\end{majorILnc}
While truth tables provide guaranteed answers to these questions, students should not become reliant on this method. 
For one thing, this method doesn't work for more complex languages, and it is important to practice with the simpler \GSL{} structures the reasoning skills and methods needed in later chapters. 
Secondly, truth tables can quickly become very large and tedious, and often simple reasoning suffices. 
More generally, truth tables give an answer but often don’t give insight.

For example, the truth table for the sentence $\horseshoe{\parconjunction{\parhorseshoe{\Al}{\Bl}}{\parhorseshoe{\Bl}{\Cl}}}{\parhorseshoe{\Al}{\Cl}}$ requires 8 lines. 
But we can reason that the only way for this sentence to be false is if $\parconjunction{\parhorseshoe{\Al}{\Bl}}{\parhorseshoe{\Bl}{\Cl}}$ is true in $\IntA$ and $\parhorseshoe{\Al}{\Cl}$ is false. 
$\parhorseshoe{\Al}{\Cl}$ can only be false in $\IntA$ if $\Al$ is $\True$ and $\Cl$ is $\FalseB$ in $\IntA$. 
Now we look at $\Bl$ and ask what $\IntA$ should assign it to make the sentence $\False$; 
we see that if $\Bl$ is $\TrueB$, $\parhorseshoe{\Bl}{\Cl}$ is false and the LHS is false so the whole sentence is true; 
if $\Bl$ is $\FalseB$ in $\IntA$, then $\parhorseshoe{\Al}{\Bl}$ is false, the LHS is false and the whole sentence is true. 
Therefore \emph{there is no model that makes the sentence $\FalseB$} and it is \CAPS{tft}. 
We have discovered a pattern. 
And if we consider a similar but more complex sentence in which $\Al$, $\Bl$ and $\Cl$ are systematically replaced by complex sentences, the size of the truth-table blows up, but the reasoning above remains simple.


%%%%%%%%%%%%%%%%%%%%%%%%%%%%%%%%%%%%%%%%%%%%%%%%%%
\section{Recursive Proofs}\label{Recursive Proofs}
%%%%%%%%%%%%%%%%%%%%%%%%%%%%%%%%%%%%%%%%%%%%%%%%%%

\subsection{The Method of Recursive Proof}
Many logic concepts are characterized by recursive definitions.  
To prove a theorem about recursively defined concept, we usually want to employ a method called \df{recursive proof}. 
The structure of a recursive proof mirrors the structure of a recursive definition.

\begin{majorILnc}{\LnpDC{Definition of Recursive Proof}}
	Let $\Delta$ be some set whose members are defined recursively. To prove that all members of $\Delta$ have some property $\CAPPHI$ we use a \df{recursive proof}, which works as follows:
	\begin{description}
		\item[Base Step:] Show that everything identified by the base clause of the recursive definition has $\CAPPHI$.  
		\item[Inheritance Step:] Show that $\CAPPHI$ is inherited; i.e., show that if the previous objects from which new objects are generated (or found) by the generating clause have $\CAPPHI$ then the new ones have $\CAPPHI$ too.
		\item[Closure Step:] Finally, show that completing the base and inheritance steps is sufficient to show that all members of $\Delta$ have $\CAPPHI$. 
		
	\end{description}
\end{majorILnc}
The inheritance step usually has two parts. The first part is a \emph{recursive assumption}.  We will \emph{assume} that some previous objects---the objects from which new ones are generated by the generating clause---already have the property in question.  Generally we make this assumption by selecting metavariables to represent the previous objects.  We are entitled to this assumption because we know with certainty that there are some such previous objects.  At the very least, the objects previously identified in the base clause have the property.  Second, we prove that the new, generated objects must also have the property in question.

How much we actually write down in the closure step will vary from proof to proof. 
Sometimes we will simply note that the closure condition (``that's all'') ensures nothing has been left out by our proof. 
Other times, if the proof is more complicated, we might also reiterate what we have just shown. 

Throughout the book we signify the end of a completed proof with a black box: $\blacksquare$.

\subsection{Recursive Proof and Mathematical Induction}
Many important mathematical concepts are defined recursively. 
We already saw the recursive definition of natural numbers. 
If we want to \emph{prove} something about all natural numbers, we can use mathematical induction\index{principle of mathematical induction}.  \emph{Mathematical induction} is a method of proof in which one shows (i) that some property holds of the first natural number, and (ii) that for any natural number $n$ having that property, the successor of $n$ also has that property.  Mathematical induction is one sort of recursive proof:
\begin{description}
	\item[Base Step:] $0$ has property $\CAPPHI$. 
	\item[Inheritance Step:] Whenever $\integer{n}$ has property $\CAPPHI$, its successor, $n+1$, also has $\CAPPHI$.
\end{description}
Therefore, we can conclude that
\begin{description}
	\item[Closure Step:] all natural numbers have property $\CAPPHI$.
\end{description}

Let's go through an example of mathematical induction. This first proof isn't very exciting but it illustrates how recursive proofs work.

\begin{majorILnc}{\LnpEC{English Recursive Proof 1}} 
	Let's prove that there is no largest natural number $n$, i.e., that there are an infinity of natural numbers. 
	
	\begin{PROOF}	\begin{cenumerate}
			\item Base Step: The number $0$ isn't the largest; $1$ is larger.
			\item Inheritance Step: First we'll make our recursive assumption.  Assume that $n$ isn't the largest natural number.
			
			We want to show that the \emph{successor} of $n$---i.e., $n+1$---isn't the largest natural number either.  This is easy.  $n+1$ can't be the largest because $n+1<n+2$.  
			
			So, we've shown that the natural number $n+1$ can't be the largest.
			\item Closure Step: So, no natural number $n$ is the largest.
	\end{cenumerate}	\end{PROOF}
\end{majorILnc}

\noindent{}Let's look at another, slightly more difficult recursive proof.

\begin{majorILnc}{\LnpEC{English Recursive Proof 2}} 
	For every natural number $n$, there is a set that has exactly $n$ elements. 
	\begin{PROOF}\begin{cenumerate}
			\item Base Step: The empty set, $\emptyset$, has $0$ elements.
			\item Inheritance Step: Recursive Assumption: Assume that for natural number $n$ there is a corresponding set $\Delta$ containing only the natural numbers $i$ such that $0<i \leq n$.  This means that $\Delta$ has exactly $n$ elements. (To illustrate, if $n=1$, then $\Delta=\{1\}$; if $n=2$, then $\Delta=\{1,2\}$; etc.)
			
			We want to show that for the successor of $n$, $n+1$, there is another set $\Delta'$ with $n+1$ elements.  We can construct such a set by defining $\Delta'$ as $\Delta\cup\{n+1\}$.  The set $\{n+1\}$ has exactly one element, and this element isn't in $\Delta$.  Because $\Delta$ has $n$ elements and $\Delta'=\Delta\cup\{n+1\}$, it follows that $\Delta'$ has one more element than $\Delta$, i.e., that it has $n+1$ elements.
			
			We've shown that for the natural number $n+1$ there is a corresponding set $\Delta'$ with $n+1$ elements.
			\item Closure Step: So, for every natural number $n$ there is a corresponding set with $n$ elements.
	\end{cenumerate}\end{PROOF}
\end{majorILnc}


\noindent{}Two things can give people trouble with the structure of recursive proofs (aside from intrinsically hard problems).  
One is that the base case is often easy or trivial. 
Make sure you have done it correctly but don't worry if it seems too easy. 

Second, some students initially think that the recursive assumption in the inheritance step assumes what we are trying to prove, but that isn't so.  Think of the recursive assumption this way: \emph{if} some group of objects has property X, then from that we can prove that another group of objects also has X.  We already know that some objects have the property in question: the object(s) identified in the base step.  The recursive assumption lets us give a label to these previously identified objects bearing the property in question. In the last example, we used the variable $n$ to stand for numbers that we already know have a corresponding set.  That allowed us to show that other objects---in this example, designated by $n+1$---also have the same property.



%%%%%%%%%%%%%%%%%%%%%%%%%%%%%%%%%%%%%%%%%%%%%%%%%%
\section{Recursive Proofs in SL}\label{recursive proofs in SL}
%%%%%%%%%%%%%%%%%%%%%%%%%%%%%%%%%%%%%%%%%%%%%%%%%%

\subsection{Recursive Proof Examples in SL}

Let's illustrate recursive proofs further by establishing two simple results about \GSL{} sentences.  

We'll use the method to prove more surprising and substantial results throughout the rest of this book.
\begin{majorILnc}{\LnpEC{Recursive Proof Ex 1}} 
In this example we prove that every (official) sentence has exactly as many left parentheses as right.
\begin{PROOF}
\begin{description}
\item[Base Step:] All atomic sentences (i.e., sentence letters) have zero left parentheses and zero right parentheses. 
And, of course, $0=0$.
\item[Inheritance Step:] Suppose $\CAPPHI$ and $\CAPTHETA,\CAPTHETA_1,\CAPTHETA_2,\ldots,\CAPTHETA_{\integer{n}}$ each have exactly as many left parentheses as right, and are of order $k$ or less. 
(This is our recursive assumption.)  We want to show that sentences of order $k+1$ have the same property.


New sentences can be generated as $\parconjunction{\CAPTHETA_1}{\conjunction{\CAPTHETA_2}{\conjunction{\ldots}{\CAPTHETA_{\integer{n}}}}}$, $\pardisjunction{\CAPTHETA_1}{\disjunction{\CAPTHETA_2}{\disjunction{\ldots}{\CAPTHETA_{\integer{n}}}}}$, $\parhorseshoe{\CAPPHI}{\CAPTHETA}$, and $\partriplebar{\CAPPHI}{\CAPTHETA}$. 
In all these cases the resulting parentheses are all those of $\CAPPHI$ and $\CAPTHETA$ plus the new matching outer ones; so, the resulting sentence still has exactly as many left parentheses as right.
The only other way new sentences can be generated is as $\negation{\CAPPHI}$, in which case the resulting parentheses are just those of $\CAPPHI$ and that, by the recursive assumption, has exactly as many left parentheses as right.
\item[Closure Step:] Because the inheritance step covers all the ways of generating an \GSL{} sentence, for all \GSL{} sentences the number of left parentheses is the same as the number of right parentheses.\end{description}
\end{PROOF}
\end{majorILnc}
\begin{majorILnc}{\LnpEC{Recursive Proof Ex 2}} 
In this example we prove that in every official \GSL{} sentence, the number of left parentheses is greater than or equal to the number of arrows.
\begin{PROOF}
Let $\LP\CAPPHI$ be the number of left parentheses in $\CAPPHI$ and $\HHH\CAPPHI$ be the number of arrows in $\CAPPHI$.
\begin{description}
\item[Base Step:] In the base case $\CAPPHI$ is atomic, so $\LP\CAPPHI=0$ and $\HHH\CAPPHI=0$.  Thus, $\LP\CAPPHI=\HHH\CAPPHI$ and so $\LP\CAPPHI\geq\HHH\CAPPHI$.
 This holds for \emph{all} atomic $\CAPPHI$.
\item[Inheritance Step:] \hfill{}
\begin{description}
\item[Recursive Assumption:] Assume the above property holds for some $\CAPTHETA$, $\CAPTHETA_1$, $\CAPTHETA_2$, $\ldots$ $\CAPTHETA_{\integer{n}}$, which are all of order $k$ or less. 
That is, assume that $\LP\CAPTHETA\geq\HHH\CAPTHETA$, $\LP\CAPTHETA_1\geq\HHH\CAPTHETA_1$, $\LP\CAPTHETA_2\geq\HHH\CAPTHETA_2$, $\ldots$ $\LP\CAPTHETA_{\integer{n}}\geq\HHH\CAPTHETA_{\integer{n}}$.  Let's show that the sentences of order $k+1$ have the same property. 
\item[Negation:] $\LP\negation{\CAPTHETA}=\LP\CAPTHETA$ and $\HHH\negation{\CAPTHETA}=\HHH\CAPTHETA$. By assumption $\LP\CAPTHETA\geq\HHH\CAPTHETA$, so $\LP\negation{\CAPTHETA}\geq\HHH\negation{\CAPTHETA}$ too.
\item[Conditional:] $\LP\parhorseshoe{\CAPTHETA_1}{\CAPTHETA_2}=\LP\CAPTHETA_1+\LP\CAPTHETA_2+1$ and $\HHH\parhorseshoe{\CAPTHETA_1}{\CAPTHETA_2}=\HHH\CAPTHETA_1+\HHH\CAPTHETA_2+1$. Because $\LP\CAPTHETA_1\geq\HHH\CAPTHETA_1$ and $\LP\CAPTHETA_2\geq\HHH\CAPTHETA_2$, clearly $\LP\CAPTHETA_1+\LP\CAPTHETA_2+1\geq\HHH\CAPTHETA_1+\HHH\CAPTHETA_2+1$ too. So $\LP\parhorseshoe{\CAPTHETA_1}{\CAPTHETA_2}\geq\HHH\parhorseshoe{\CAPTHETA_1}{\CAPTHETA_2}$.
\item[Biconditional:] $\LP\partriplebar{\CAPTHETA_1}{\CAPTHETA_2}=\LP\CAPTHETA_1+\LP\CAPTHETA_2+1$ and $\HHH\partriplebar{\CAPTHETA_1}{\CAPTHETA_2}=\HHH\CAPTHETA_1+\HHH\CAPTHETA_2$. 
Because $\LP\CAPTHETA_1\geq\HHH\CAPTHETA_1$ and $\LP\CAPTHETA_2\geq\HHH\CAPTHETA_2$, clearly $\LP\CAPTHETA_1+\LP\CAPTHETA_2+1\geq\HHH\CAPTHETA_1+\HHH\CAPTHETA_2$ too. 
So $\LP\partriplebar{\CAPTHETA_1}{\CAPTHETA_2}\geq\HHH\partriplebar{\CAPTHETA_1}{\CAPTHETA_2}$.
\item[Disjunction:] We have that $\LP\pardisjunction{\CAPTHETA_1}{\disjunction{\CAPTHETA_2}{\disjunction{\ldots}{\CAPTHETA_{\integer{n}}}}}=\LP\CAPTHETA_1+\LP\CAPTHETA_2+\ldots+\LP\CAPTHETA_{\integer{n}}+1$ and $\HHH\pardisjunction{\CAPTHETA_1}{\disjunction{\CAPTHETA_2}{\disjunction{\ldots}{\CAPTHETA_{\integer{n}}}}}=\HHH\CAPTHETA_1+\HHH\CAPTHETA_2+\ldots+\HHH\CAPTHETA_{\integer{n}}$. 
Because $\LP\CAPTHETA_1\geq\HHH\CAPTHETA_1$, $\LP\CAPTHETA_2\geq\HHH\CAPTHETA_2$, $\ldots$ $\LP\CAPTHETA_{\integer{n}}\geq\HHH\CAPTHETA_{\integer{n}}$, we have that $\LP\CAPTHETA_1+\LP\CAPTHETA_2+\ldots+\LP\CAPTHETA_{\integer{n}}+1\geq\HHH\CAPTHETA_1+\HHH\CAPTHETA_2+\ldots+\HHH\CAPTHETA_{\integer{n}}$. 
So, $\LP\pardisjunction{\CAPTHETA_1}{\disjunction{\CAPTHETA_2}{\disjunction{\ldots}{\CAPTHETA_{\integer{n}}}}}\geq\HHH\pardisjunction{\CAPTHETA_1}{\disjunction{\CAPTHETA_2}{\disjunction{\ldots}{\CAPTHETA_{\integer{n}}}}}$ too.
\item[Conjunction:] The argument is literally the same as that for disjunction, just erase every token of \mention{$\VEE$} and replace it with a token of \mention{$\!\WEDGE\!$}.
\end{description}
\item[Closure Step:] These are all the ways of constructing \GSL{} sentences, and in every case, if the number of left parentheses is greater than or equal to the number of arrows in the component sentences, then that still holds in the new sentences generated from those components.\end{description}
\end{PROOF}
\end{majorILnc}%

\noindent{}In both examples we are trying to prove something about all \GSL{} sentences.
In both cases our recursive assumption is something like: assume that the property in question already holds, in particular, for some \GSL{} sentences $\CAPPHI_1,\ldots,\CAPPHI_{\integer{m}}$ of order $k$ or less.
We do this because in the inheritance step we're trying to show that if you build new \GSL{} sentences (using the logical connectives) out of old ones for which the property holds, then the property also holds for the new ones. 
One way to do this---the one just used here---is to assume that the property holds for some particular sentences $\CAPPHI_1,\ldots,\CAPPHI_{\integer{m}}$ and then show that it holds for those built from those sentences using the logical connectives. 


\subsection{Minimal Model Theorem}\label{minimal model theorem} 

Earlier we claimed that the only sentence letter assignments that matter for the truth value of some \GSL{} sentence $\CAPPHI$ are those actually in $\CAPPHI$.  We're now ready to prove it.

\begin{THEOREM}{\LnpTC{thm:localityoftruth}}
	If two models for $\CAPPHI$, $\As{}{1}$ and $\As{}{2}$, make the same assignments to all the sentence letters contained in $\CAPPHI$, then $\CAPPHI$ is true on $\As{}{1}$ \Iff $\CAPPHI$ is true on $\As{}{2}$.
\end{THEOREM}
\begin{PROOF}
	\begin{description}
		\item[Base Step:]  Let $\CAPTHETA$ be a \GSL{} sentence of order 1.  It follows that $\CAPTHETA$ must be a lone sentence letter.  If $\As{}{1}$ and $\As{}{2}$ make the same assignments for all the sentence letters, then $\CAPTHETA$, which is just one sentence letter, is true on $\As{}{1}$ \Iff $\CAPTHETA$ is true on $\As{}{2}$.
		
		\item[Inheritance Step:] 
		\begin{description}
			\item[Recursive Assumption:] Assume that for each sentence $\CAPTHETA$ of order $n$ or less, $\CAPTHETA$ is true on $\As{}{1}$ \Iff $\CAPTHETA$ is true on $\As{}{2}$.
			\item[Negation:] Say that $\CAPPSI$ is of the form $\negation{\CAPTHETA}$.  $\negation{\CAPTHETA}$ is true on $\As{}{1}$ \Iff $\CAPTHETA$ is false on $\As{}{1}$ (definition of truth, $\NEGATION$).  And $\negation{\CAPTHETA}$ is true on $\As{}{2}$ \Iff $\CAPTHETA$ is false on $\As{}{2}$.  By our recursive assumption (RA), $\CAPTHETA$ is true on $\As{}{1}$ \Iff it's true on $\As{}{2}$.  It follows that $\negation{\CAPTHETA}$ is true on $\As{}{1}$ \Iff $\negation{\CAPTHETA}$ is true on $\As{}{2}$. I.e., $\CAPPSI$ is true on $\As{}{1}$ \Iff it's true on $\As{}{2}$.
			\item[Conditional:] Say that $\CAPPSI$ is of the form $\horseshoe{\CAPTHETA_1}{\CAPTHETA_2}$.  $\horseshoe{\CAPTHETA_1}{\CAPTHETA_2}$ is true on $\As{}{1}$ \Iff either $\CAPTHETA_1$ is false or $\CAPTHETA_2$ is true on $\As{}{1}$ (definition of truth, $\HORSESHOE$).  The same holds on model $\As{}{2}$.  By RA, $\CAPTHETA_1$ is true on $\As{}{1}$ \Iff it's true on $\As{}{2}$; and $\CAPTHETA_2$ is true on $\As{}{1}$ \Iff it's true on $\As{}{2}$.  Therefore, there are four possibilities:  (i) $\CAPTHETA_1$ is true on both $\As{}{1}$ and $\As{}{2}$, and $\CAPTHETA_2$ is true on both too; (ii) $\CAPTHETA_1$ is false on both $\As{}{1}$ and $\As{}{2}$, and $\CAPTHETA_2$ is false on both too; (iii) $\CAPTHETA_1$ is false on both $\As{}{1}$ and $\As{}{2}$, and $\CAPTHETA_2$ is true on both models; and (iv) $\CAPTHETA_1$ is true on both $\As{}{1}$ and $\As{}{2}$, and $\CAPTHETA_2$ is false on both models.  In cases (i) through (iii), $\horseshoe{\CAPTHETA_1}{\CAPTHETA_2}$ is true on both $\As{}{1}$ and $\As{}{2}$.  In case (iv), $\horseshoe{\CAPTHETA_1}{\CAPTHETA_2}$ is false on both $\As{}{1}$ and $\As{}{2}$.  Thus, $\CAPPSI$ is true on $\As{}{1}$ \Iff it's true on $\As{}{2}$.
			\item[Biconditional:] Say that $\CAPPSI$ is of the form $\triplebar{\CAPTHETA_1}{\CAPTHETA_2}$.  $\triplebar{\CAPTHETA_1}{\CAPTHETA_2}$ is true on $\As{}{1}$ \Iff $\CAPTHETA_1$ and $\CAPTHETA_2$ have the same truth value on $\As{}{1}$ (definition of truth, $\TRIPLEBAR$).  The same holds on model $\As{}{2}$.  By RA, $\CAPTHETA_1$ is true on $\As{}{1}$ \Iff it's true on $\As{}{2}$; and $\CAPTHETA_2$ is true on $\As{}{1}$ \Iff it's true on $\As{}{2}$.  There are two possibilities: (i) $\CAPTHETA_1$ and $\CAPTHETA_2$ share the same truth value on both $\As{}{1}$ and $\As{}{2}$; and (ii) $\CAPTHETA_1$ and $\CAPTHETA_2$ have differing truth values on both $\As{}{1}$ and $\As{}{2}$.  In case (i), $\triplebar{\CAPTHETA_1}{\CAPTHETA_2}$ is true on both $\As{}{1}$ and $\As{}{2}$.  In case (ii), $\triplebar{\CAPTHETA_1}{\CAPTHETA_2}$ is false on both $\As{}{1}$ and $\As{}{2}$.  Thus, $\CAPPSI$ is true on $\As{}{1}$ \Iff it's true on $\As{}{2}$.
			\item[Disjunction:] Say that $\CAPPSI$ is of the form $\pardisjunction{\CAPTHETA_1}{\disjunction{\CAPTHETA_2}{\disjunction{\ldots}{\CAPTHETA_{\integer{n}}}}}$.  $\pardisjunction{\CAPTHETA_1}{\disjunction{\CAPTHETA_2}{\disjunction{\ldots}{\CAPTHETA_{\integer{n}}}}}$ is true on $\As{}{1}$ \Iff at least one disjunct is true on $\As{}{1}$ (definition of truth, $\VEE$).  As before, the same holds on $\As{}{2}$.  By RA, for every disjunct $\CAPTHETA_{k}$, $\CAPTHETA_{k}$ is true on $\As{}{1}$ \Iff it's true on $\As{}{2}$.  So, either one disjunct (at least) is true on both $\As{}{1}$ and $\As{}{2}$, or none of the disjuncts is true on either of $\As{}{1}$ and $\As{}{2}$.  Either way, $\CAPPSI$ is true on $\As{}{1}$ \Iff it's true on $\As{}{2}$.
			\item[Conjunction:] Say that $\CAPPSI$ is of the form $\parconjunction{\CAPTHETA_1}{\conjunction{\CAPTHETA_2}{\conjunction{\ldots}{\CAPTHETA_{\integer{n}}}}}$.  $\parconjunction{\CAPTHETA_1}{\conjunction{\CAPTHETA_2}{\conjunction{\ldots}{\CAPTHETA_{\integer{n}}}}}$ is true on $\As{}{1}$ \Iff every conjunct is true on $\As{}{1}$ (definition of truth, $\WEDGE$).  Again, the same holds on $\As{}{2}$.  By RA, for every conjunct $\CAPTHETA_{k}$, $\CAPTHETA_{k}$ is true on $\As{}{1}$ \Iff it's true on $\As{}{2}$.  Thus, either every conjunct is true on both $\As{}{1}$ and $\As{}{2}$, or at least one conjunct is false on both $\As{}{1}$ and $\As{}{2}$.  Either way, $\CAPPSI$ is true on $\As{}{1}$ \Iff it's true on $\As{}{2}$.
		\end{description}
		\item[Closure Step:] There is no other way to form a sentence of \GSL{} $\CAPPHI$, so the above clauses are sufficient to prove that if $\As{}{1}$ and $\As{}{2}$ make the same assignments for all the sentence letters, then $\CAPPHI$ is true on $\As{}{1}$ \Iff $\CAPPHI$ is true on $\As{}{2}$.
	\end{description}
\end{PROOF}


\subsection{Main Connective Theorem}\label{additional recur examples} 

Earlier (\ref{GSL Main connective}) we defined the main connective of a sentence $\CAPPHI$ as the connective that isn't in any proper subsentence of $\CAPPHI$.  Atomic sentences don't have any connectives, and so don't have main connectives.  Most non-atomic sentences have just one main connective, but there is an exception: conjunctions (or disjunctions) with $n$ tokens of \mention{$\WEDGE$} (or \mention{$\VEE$}) that are all the main connectives, where $n>1$.  In such sentences, $n+1$ is the number of conjuncts (disjuncts).  For example, all three of the \mention{$\WEDGE$} tokens in the following sentence are the main connectives: $\parconjunction{\conjunction{\Bl}{\Al}}{\conjunction{\Hl}{\Gl}}$.  Let's call sentences that have multiple tokens of \mention{$\WEDGE$} as main connectives \mention{extended conjunctions}, and those that have multiple tokens of \mention{$\VEE$} as main connectives \mention{extended disjunctions}.  With this new terminology, let's turn to another recursive proof.

\begin{THEOREM}{\LnpTC{Recur Main Connective} Main Connective Theorem:} 
For\index{main connective!theorem} every \GSL{} sentence $\CAPPHI$, one of the following holds: (i) $\CAPPHI$ has no main connective; (ii) $\CAPPHI$ has exactly one main connective token; or (iii) $\CAPPHI$ is an extended conjunction (or disjunction) with $n-1$ main connective tokens, where $n$ is the number of conjuncts (disjuncts).
\end{THEOREM}
\begin{PROOFOF}{Thm. \ref{Recur Main Connective}, Main Connective Theorem}
\begin{description}
\item[Base Step:] 
Every \GSL{} sentence of order 1 is atomic (i.e., a lone sentence letter) and has no connectives at all.  Therefore it has no main connective.
 
\item[Inheritance Step:] 
Assume that the theorem holds for every sentence with order $\integer{i}$ or less, and that $\CAPPHI$ is of order $\integer{i}+1$, where $i\geq 1$.  (This is the recursive assumption.)

By the definition of an \GSL{} sentence (\ref{Recursive definition of Sentences of GSL}), $\CAPPHI$ will have to have one of the following forms:
$\negation{\CAPPSI}$, $\parhorseshoe{\CAPTHETA_1}{\CAPTHETA_2}$, $\partriplebar{\CAPTHETA_1}{\CAPTHETA_2}$, $\parconjunction{\CAPTHETA_1}{\conjunction{\ldots}{\CAPTHETA_{\integer{n}}}}$, or $\pardisjunction{\CAPTHETA_1}{\disjunction{\ldots}{\CAPTHETA_{\integer{n}}}}$.  For each of these possible sentence schemas for $\CAPPHI$, there is at least one connective that isn't in any proper subsentence of $\CAPPHI$.  It follows that $\CAPPHI$ must have \emph{at least} one main connective token.

\begin{description}

\item[Negation:] 
Assume (for \emph{reductio}) that $\CAPPHI$ has more than one main connective token, and that the leftmost of these tokens is a negation.  Given that the leftmost is a \mention{$\NEGATION$} token, $\CAPPHI$ must be of the form $\negation{\CAPPSI}$ (by def. of \GSL{} sentence, \ref{Recursive definition of Sentences of GSL}).  If this is so, then the other main connective tokens of $\CAPPHI$ must be in $\CAPPSI$.  But $\CAPPSI$ is a proper subsentence of $\CAPPHI$, so any connective in $\CAPPSI$ can't be a main connective of $\CAPPHI$ (def. of main connective).  Therefore, contrary to our assumption for \emph{reductio}, there must be exactly one main connective token in $\CAPPHI$: the \mention{$\NEGATION$}.

\item[Conditional:]
Assume (again, for \emph{reductio}) that $\CAPPHI$ has more than one main connective token, and that the leftmost of these is a \mention{$\HORSESHOE$}.  By the definition of \GSL{} sentence (\ref{Recursive definition of Sentences of GSL}), it follows that $\CAPPHI$ must be of the form $\horseshoe{\CAPTHETA_1}{\CAPTHETA_2}$.  Accordingly, any other main connective tokens of $\CAPPHI$ must be in $\CAPTHETA_2$.  However, $\CAPTHETA_2$ is a proper subsentence of $\CAPPHI$, and thus can't contain a main connective of $\CAPPHI$.  So, contrary to our assumption for \emph{reductio}, there must be exactly one main connective token for $\CAPPHI$: the \mention{$\HORSESHOE$}.

\item[Biconditional:]
This clause is the same as the \mention{Conditional} clause above, except with \mention{$\TRIPLEBAR$} in place of \mention{$\HORSESHOE$}.

\item[Conjunction:]
Assume that the leftmost main connective token in $\CAPPHI$ is a \mention{$\WEDGE$}.  Assume (for \emph{reductio}) that there is some main connective of $\CAPPHI$ other than \mention{$\WEDGE$}.  According to the definition of \GSL{} sentence (\ref{Recursive definition of Sentences of GSL}), $\CAPPHI$ must be of the form $\parconjunction{\CAPTHETA_1}{\conjunction{\CAPTHETA_2}{\conjunction{\CAPTHETA_3}{\conjunction{\ldots}{\CAPTHETA_{\integer{n}}}}}}$, where $\integer{n}$ is the number of conjuncts.  It follows that the non-\mention{$\WEDGE$} main connective token must be in one of the conjuncts of $\CAPPHI$.  But the conjuncts of $\CAPPHI$ are proper subsentences of $\CAPPHI$, and so can't contain any main connective tokens.  So, contrary to our assumption for \emph{reductio}, all of the main connective tokens of $\CAPPHI$ are \mention{$\WEDGE$}.

Given that $\CAPPHI$ is of the form $\parconjunction{\CAPTHETA_1}{\conjunction{\CAPTHETA_2}{\conjunction{\CAPTHETA_3}{\conjunction{\ldots}{\CAPTHETA_{\integer{n}}}}}}$, $n$ is the number of conjuncts.  We want to show that the number of main connective tokens, $k$, is $n-1$.  Assume that $k>n-1$\footnote{I.e., that the number of main connective tokens is greater than (the number of conjuncts, minus one )....}.  But if $k>n-1$ then there must be two \mention{$\WEDGE$} tokens adjacent to each other; $\CAPPHI$ is an \GSL{} sentence, so this is impossible.  Assume instead that $k<n-1$\footnote{I.e., that the number of main connective tokens is fewer than (the number of conjuncts, minus one)....}.  But then there must be two conjuncts adjacent to each other, not separated by a connective.  Again, $\CAPPHI$ is an \GSL{} sentence, so this is impossible.  Therefore $k=n-1$.  So, when $n$ is the number of conjuncts in $\CAPPHI$, $n-1$ is the number of main connective tokens.

There must be at least two conjuncts in $\CAPPHI$ (definition of \GSL{} sentence). Assuming only two conjuncts in $\CAPPHI$, there is exactly one main connective token.  Alternatively, when the number of conjuncts, $n$, is greater than two, there are $n-1$ main connective tokens and $\CAPPHI$ is an extended conjunction.

\item[Disjunction:]
This clause is the same as the \mention{Conjunction} clause above, except with \mention{$\VEE$} in place of \mention{$\WEDGE$}.


\end{description}

\item[Closure Step:] 
We've shown that sentences with order 1 have no main connective, and that every sentence of order of 2 or greater either has exactly one main connective token, or is an extended conjunction (or disjunction) with $n$ main connective tokens, where $n+1$ is the number of conjuncts (disjuncts).  There is no other way to construct an \GSL{} sentence than the ways described in the previous two clauses.

\end{description}
\end{PROOFOF}

%%%%%%%%%%%%%%%%%%%%%%%%%%%%%%%%%%%%%%%%%%%%%%%%%%
\section[Disjunctive Normal Form]{Disjunctive Normal Form}\label{DNF and the TFE Replacement Theorem}
%%%%%%%%%%%%%%%%%%%%%%%%%%%%%%%%%%%%%%%%%%%%%%%%%%

Which sentences below are easy to evaluate? 
Which are difficult?
\begin{multicols}{2}
\begin{menumerate}
\item\label{dnf1} $\horseshoe{\negation{\bparhorseshoe{\Al}{\negation{\parconjunction{\Cl}{\Bl}}}}}{\parhorseshoe{\Al}{\Cl}}$
\item\label{dnf2} $\conjunction{\Al}{\conjunction{\Rl}{\conjunction{\Al}{\negation{\Rl}}}}$
\item\label{dnf3} $\negation{\cparhorseshoe{\negation{\bparhorseshoe{\Al}{\negation{\Rl}}}}{\parhorseshoe{\Al}{\Rl}}}$
\item\label{dnf4} $\disjunction{\bpardisjunction{\negation{\Al}}{\disjunction{\negation{\Cl}}{\negation{\Bl}}}}{\bpardisjunction{\negation{\Al}}{\Cl}}$
\end{menumerate}
\end{multicols}
\noindent{}If we have the truth values of the sentence letters, then clearly \ref{dnf2} and \ref{dnf4} are much simpler to figure out than \ref{dnf1} and \ref{dnf3}. 
In general, we find that the truth values for certain sentences are easier to calculate than others. (Some sentences, we might say, are more \sq{transparent}). 
Say that we want to figure out the truth value of a difficult sentence and we know that it's \CAPS{tfe} to an easy sentence. Then we could figure out the truth value of the difficult one by figuring out the truth value of the easy one.  After all, they're equivalent!  In the above list of sentences, \ref{dnf1} is TFE to \ref{dnf4} and \ref{dnf2} is TFE to \ref{dnf3}.

If a sentence is fairly complicated we might not know whether it's \CAPS{tfe} to a simpler, more transparent sentence. 
To get around this we can simplify the complicated sentence by a systematic series of steps, each of which replaces some subsentence with a simpler sentence that we know is \CAPS{tfe} to the subsentence being replaced.  

\subsection{The \CAPS{tfe} Replacement Theorem}\label{The TFE Replacement Theorem}
First we have to verify that equivalence transformation is legitimate, so we will prove:
\begin{THEOREM}{\LnpTC{TFE Replacement} Truth Functional Equivalence Replacement:}
Let $\CAPPHI$ be a subsentence of $\CAPTHETA$.  If $\CAPPHI$ and $\CAPPHI^*$ are truth functionally equivalent, and $\CAPTHETA^*$ is the result of replacing one occurrence of $\CAPPHI$ by $\CAPPHI^*$ in $\CAPTHETA$, then $\CAPTHETA$ and $\CAPTHETA^*$ are truth functionally equivalent.
\end{THEOREM}
\noindent{}Before turning to the proof, recall from section \ref{Other Relations}, definition \mvref{GSL TFE} that two sentences $\CAPPHI$ and $\CAPPHI^*$ are truth functionally equivalent \Iff all models assign them the same truth value.  This is the same as saying they entail each other, i.e., that $\CAPPHI^*\sdtstile{}{}\CAPPHI$ and $\CAPPHI\sdtstile{}{}\CAPPHI^*$. 

\begin{PROOF}
\begin{description}
\item[Base Step:] Suppose that $\CAPTHETA$ is an atomic sentence. 
In this case $\CAPTHETA$ and $\CAPPHI$ are the same. 
So, $\CAPTHETA^*$ and $\CAPPHI^*$ are the same. 
Thus, by hypothesis (the hypothesis being that $\CAPPHI$ and $\CAPPHI^*$ are truth functionally equivalent), $\CAPTHETA^*$ and $\CAPTHETA$ are truth functionally equivalent.

\item[Inheritance Step:] For this proof we must consider each connective separately. 
\begin{description}

\item[Negation:] Suppose $\CAPTHETA$ is a negation $\negation{\CAPPSI}$, for some sentence $\CAPPSI$ which either is identical to $\CAPPHI$, or of which $\CAPPHI$ is a subsentence. 

Assume that $\CAPPSI$ and $\CAPPSI^*$ ($\CAPPSI^*$ the result of replacing at least one occurrence of $\CAPPHI$ with $\CAPPHI^*$ in $\CAPPSI$) are truth functionally equivalent, i.e. have the same truth value on every model, and are of order $k$ or less. 
(This is our recursive assumption.)
Now, by supposition $\CAPTHETA^*$ is the same sentence as $\parnegation{\CAPPSI}^*$, which with a little thought one can see is the same sentence as $\NEGATION(\CAPPSI^*)$. 

Next, note that by the definition of $\True$ in a model, $\NEGATION(\CAPPSI^*)$ is $\True$ in a model $\IntA$ \Iff $\CAPPSI^*$ is $\False$ in $\IntA$ \Iff $\CAPPSI$ is $\False$ in $\IntA$ \Iff $\negation{\CAPPSI}$ is $\True$ in $\IntA$. So, $\CAPTHETA^*$ is $\True$ in $\IntA$ \Iff $\negation{\CAPPSI}$ is $\True$ in $\IntA$.

So, $\CAPTHETA^*$ is $\True$ in $\IntA$ \Iff $\CAPTHETA$ is $\True$ in $\IntA$, so by definition $\CAPTHETA$ and $\CAPTHETA^*$ are truth functionally equivalent.

\item[Conditional:] Suppose $\CAPTHETA$ is a conditional $\parhorseshoe{\CAPPSI_1}{\CAPPSI_2}$, were at least one of $\CAPPSI_1$ and $\CAPPSI_2$ has $\CAPPHI$ as a subsentence, or is identical to $\CAPPHI$. 
So $\CAPTHETA^*$ is $\parhorseshoe{\CAPPSI_1}{\CAPPSI_2}^*$, which with some thought one can see is either (case 1) $\parhorseshoe{{\CAPPSI_1}^*}{\CAPPSI_2}$ or (case 2) $\parhorseshoe{\CAPPSI_1}{{\CAPPSI_2}^*}$.

Suppose it is the first case, and assume that $\CAPPSI_1$ is truth functionally equivalent to ${\CAPPSI_1}^*$, and that both are of order $k$ or less.
(This is our recursive assumption.)

We know that  $\parhorseshoe{{\CAPPSI_1}^*}{\CAPPSI_2}$ is $\True$ in a model $\IntA$ \Iff ${\CAPPSI_1}^*$ is $\False$ in $\IntA$ or $\CAPPSI_2$ is $\True$ in $\IntA$. 
But that holds \Iff $\CAPPSI_1$ is $\False$ in $\IntA$ or $\CAPPSI_2$ is $\True$ in $\IntA$, and that holds \Iff $\parhorseshoe{\CAPPSI_1}{\CAPPSI_2}$ is $\True$ in $\IntA$. 

In the first case, $\CAPTHETA^*$ is $\True$ in $\IntA$ \Iff $\CAPTHETA$ is $\True$ in $\IntA$, so by definition $\CAPTHETA$ and $\CAPTHETA^*$ are truth functionally equivalent.

Showing that $\CAPTHETA$ and $\CAPTHETA^*$ are truth functionally equivalent in case 2 is left to the reader.

\item[Biconditional:] Showing that $\CAPTHETA$ and $\CAPTHETA^*$ are truth functionally equivalent when $\CAPTHETA$ is a biconditional of the form $\partriplebar{\CAPPSI_1}{\CAPPSI_2}$ is left to the reader.

\item[Conjunction:] Suppose that $\CAPTHETA$ is a conjunction $\parconjunction{\CAPPSI_1}{\conjunction{\CAPPSI_2}{\conjunction{\CAPPSI_3}{\conjunction{\ldots}{\CAPPSI_{\integer{n}}}}}}$, where at least one of $\CAPPSI_1$, $\CAPPSI_2$, $\CAPPSI_3$, $\ldots$ $\CAPPSI_{\integer{n}}$ has $\CAPPHI$ as a subsentence, or is identical to $\CAPPHI$. 

So $\CAPTHETA^*$ is $\parconjunction{\CAPPSI_1}{\conjunction{\CAPPSI_2}{\conjunction{\CAPPSI_3}{\conjunction{\ldots}{\CAPPSI_{\integer{n}}}}}}^*$. Since $\CAPTHETA^*$ is the result of replacing \emph{one} occurrence of $\CAPPHI$ in $\CAPTHETA$ with $\CAPPHI^*$, it's not hard to see that $\parconjunction{\CAPPSI_1}{\conjunction{\CAPPSI_2}{\conjunction{\CAPPSI_3}{\conjunction{\ldots}{\CAPPSI_{\integer{n}}}}}}^*$ can \emph{only} one of either $\parconjunction{\CAPPSI_1^*}{\conjunction{\CAPPSI_2}{\conjunction{\ldots}{\CAPPSI_{\integer{n}}}}}$ or $\parconjunction{\CAPPSI_1}{\conjunction{\CAPPSI_2^*}{\conjunction{\ldots}{\CAPPSI_{\integer{n}}}}}$ or $\ldots$ or $\parconjunction{\CAPPSI_1}{\conjunction{\CAPPSI_2}{\conjunction{\ldots}{\CAPPSI_{\integer{n}}^*}}}$. 
But, clearly, there's no difference which it is. 
So, without loss of generality say that $\CAPTHETA^*$ is $\parconjunction{\CAPPSI_1}{\conjunction{\CAPPSI_2^*}{\conjunction{\CAPPSI_3}{\conjunction{\ldots}{\CAPPSI_{\integer{n}}}}}}$. 

As before, assume that $\CAPPSI_2$ and ${\CAPPSI_2}^*$ are truth functionally equivalent, and are of order $k$ or less. 
(This is our recursive assumption.)
So, for any model $\IntA$, $\CAPTHETA$ is $\True$ in $\IntA$ \Iff $\parconjunction{\CAPPSI_1}{\conjunction{\CAPPSI_2}{\conjunction{\CAPPSI_3}{\conjunction{\ldots}{\CAPPSI_{\integer{n}}}}}}$ is $\True$ in $\IntA$ \Iff $\parconjunction{\CAPPSI_1}{\conjunction{\CAPPSI_2^*}{\conjunction{\CAPPSI_3}{\conjunction{\ldots}{\CAPPSI_{\integer{n}}}}}}$ is is $\True$ in $\IntA$ \Iff $\parconjunction{\CAPPSI_1}{\conjunction{\CAPPSI_2}{\conjunction{\CAPPSI_3}{\conjunction{\ldots}{\CAPPSI_{\integer{n}}}}}}^*$ is $\True$ in $\IntA$ \Iff $\CAPTHETA^*$ is $\True$ in $\IntA$.

Therefore $\CAPTHETA$ and $\CAPTHETA^*$ are truth functionally equivalent.

\item[Disjunction:] The disjunction case is similar and it is left to the reader. 
\end{description}
\item[Closure Step:] Those are the only ways \GSL{} sentences can be formed; hence the theorem is proved.
\end{description}
\end{PROOF}
\begin{majorILnc}{\LnpEC{TFE Replacement Example}}
We can use this theorem to show that \ref{dnf1} and \ref{dnf4} above are equivalent, as are \ref{dnf2} and \ref{dnf3}. 
Consider \ref{dnf1} and \ref{dnf4}. 
(We leave \ref{dnf2} and \ref{dnf3} to the reader.)
For any formulas $\CAPPHI$ and $\CAPTHETA$, 
\begin{menumerate}
\item\label{dnf5} $\parhorseshoe{\CAPPHI}{\CAPTHETA}$ and $\pardisjunction{\negation{\CAPPHI}}{\CAPTHETA}$ are \CAPS{tfe} (See ex. \pmvref{TFE Ex 2})
\item\label{dnf6} $\negation{\negation{\CAPPHI}}$ and $\CAPPHI$ are \CAPS{tfe} (See ex. \pmvref{TFE Ex 3})
\item\label{dnf7} $\negation{\parconjunction{\CAPTHETA}{\CAPPHI}}$ and $\pardisjunction{\negation{\CAPTHETA}}{\negation{\CAPPHI}}$ are \CAPS{tfe} (See \ref{HW Entailment 4} and \ref{HW Entailment 12}, section \ref{Entailment Problems for GSL})
\item\label{dnf8} $\pardisjunction{\CAPPHI}{\disjunction{\CAPTHETA}{\CAPPSI}}$ and $\pardisjunction{\CAPPHI}{\pardisjunction{\CAPTHETA}{\CAPPSI}}$ are \CAPS{tfe} (Obvious)
\end{menumerate}
By making substitutions starting with \ref{dnf1} that the theorem says result in successive truth functionally equivalent sentences, we can get from \ref{dnf1} to \ref{dnf4} and thereby have shown that \ref{dnf4} is truth functionally equivalent to \ref{dnf1}.
\begin{menumerate}
\item $\horseshoe{\negation{\bparhorseshoe{\Al}{\negation{\parconjunction{\Cl}{\Bl}}}}}{\parhorseshoe{\Al}{\Cl}}$ [\ref{dnf1}]
\item $\disjunction{\negation{\negation{\bparhorseshoe{\Al}{\negation{\parconjunction{\Cl}{\Bl}}}}}}{\parhorseshoe{\Al}{\Cl}}$ [\ref{dnf5}]
\item $\disjunction{\bparhorseshoe{\Al}{\negation{\parconjunction{\Cl}{\Bl}}}}{\parhorseshoe{\Al}{\Cl}}$ [\ref{dnf6}]
\item $\disjunction{\bpardisjunction{\negation{\Al}}{\negation{\parconjunction{\Cl}{\Bl}}}}{\pardisjunction{\negation{\Al}}{\Cl}}$ [\ref{dnf5}]
\item $\disjunction{\bpardisjunction{\negation{\Al}}{\pardisjunction{\negation{\Cl}}{\negation{\Bl}}}}{\pardisjunction{\negation{\Al}}{\Cl}}$ [\ref{dnf7}]
\item $\disjunction{\bpardisjunction{\negation{\Al}}{\disjunction{\negation{\Cl}}{\negation{\Bl}}}}{\pardisjunction{\negation{\Al}}{\Cl}}$ [\ref{dnf8}]
\end{menumerate}
\end{majorILnc}

\subsection{Disjunctive Normal Form}\label{Disjunctive Normal Form}

Sentences in disjunctive normal form are especially easy to evaluate.  
\begin{majorILnc}{\LnpDC{DNF Definition}}
A \GSL{} sentence is in \df{disjunctive normal form} (\CAPS{dnf})\index{DNF|see{disjunctive normal form}} \Iff
\begin{cenumerate}
\item it contains no conditional ($\HORSESHOE$) or biconditional ($\TRIPLEBAR$),
\item negations ($\NEGATION$) only govern sentence letters, and
\item no conjunction ($\WEDGE$) contains a disjunction ($\VEE$) as a subsentence.
\end{cenumerate}
\end{majorILnc}
\noindent{}A typical example of a sentence in \CAPS{dnf} is $\disjunction{\parconjunction{\Ql}{\negation{\Rl}}}{\parconjunction{\negation{\Pl}}{\Rl}}$.  The truth conditions for this sentence are easy to see.  The sentence $\disjunction{\parconjunction{\Ql}{\negation{\Rl}}}{\parconjunction{\negation{\Pl}}{\Rl}}$ is true on a model $\IntA$ when either $\IntA(\Ql)=\TrueB$ and $\IntA(\Rl)=\FalseB$, or $\IntA(\Pl)=\FalseB$ and $\IntA(\Rl)=\TrueB$.  Some less typical examples are:
\begin{menumerate}
\item $\Ql$
\item $\negation{\Rl}$
\item $\conjunction{\Ql}{\negation{\Rl}}$
\item $\disjunction{\negation{\Ql}}{\Rl}$
\end{menumerate}
\CAPS{dnf} is important because we can prove the following theorem.
\begin{THEOREM}{\LnpTC{Disjunctive Normal Form Theorem} The Disjunctive Normal Form Theorem:}
Every sentence of \GSL{} is truth functionally equivalent to an \GSL{} sentence which is in \CAPS{dnf}.
\end{THEOREM}
\begin{PROOF}
The proof relies on three lemmas, each of which can be established rigorously by recursive proof.  We leave the details of these lemmas to the reader.

Here we provide a process showing how to turn any given sentence into one that's in \CAPS{dnf}. We proceed in three stages, corresponding to the three lemmas that are necessary for a proof.
\begin{description}
\item[Step A:] If a subsentence of $\CAPPHI$ has a conditional or biconditional as its main connective, i.e., is of the form $\parhorseshoe{\CAPPSI}{\CAPTHETA}$ or $\partriplebar{\CAPTHETA}{\CAPPSI}$, replace the subsentence by $\pardisjunction{\negation{\CAPPSI}}{\CAPTHETA}$ or $\disjunction{\parconjunction{\CAPPSI}{\CAPTHETA}}{\parconjunction{\negation{\CAPPSI}}{\negation{\CAPTHETA}}}$ respectively. 
Repeat as necessary to obtain a sentence $\CAPPHI'$ without conditionals or biconditionals.
\item[Step B:] \hfill{}
\begin{cenumerate}
\item Replace any subsentence of the form $\negation{\negation{\CAPPSI}}$ in $\CAPPHI'$ with $\CAPPSI$.
\item Replace any subsentence of the form $\negation{\parconjunction{\CAPPSI}{\CAPTHETA}}$ in $\CAPPHI'$ with $\pardisjunction{\negation{\CAPPSI}}{\negation{\CAPTHETA}}$. 
\item Replace $\negation{\pardisjunction{\CAPPSI}{\CAPTHETA}}$ in $\CAPPHI'$ with $\parconjunction{\negation{\CAPPSI}}{\negation{\CAPTHETA}}$. 
%(These last two are known as DeMorgan's laws after the logician who first explicitly formulated them.)
\end{cenumerate} 
Repeat as necessary to obtain $\CAPPHI''$ in which negations govern nothing but sentence letters.
\item[Step C:] The only thing that could prevent $\CAPPHI''$ from being in \CAPS{dnf} is that some conjunctions govern some disjunctions, i.e., there is a subsentence  $\conjunction{\CAPTHETA}{\pardisjunction{\CAPPSI_1}{\disjunction{\CAPPSI_2}{\disjunction{\ldots}{\CAPPSI_{\integer{n}}}}}}$, or the reverse $\conjunction{\pardisjunction{\CAPPSI_1}{\disjunction{\CAPPSI_2}{\disjunction{\ldots}{\CAPPSI_{\integer{n}}}}}}{\CAPTHETA}$. 
Those subsentences can be replaced by the equivalent $\disjunction{\parconjunction{\CAPPSI_1}{\CAPTHETA}}{\disjunction{\parconjunction{\CAPPSI_2}{\CAPTHETA}}{\disjunction{\ldots}{\parconjunction{\CAPPSI_{\integer{n}}}{\CAPTHETA}}}}$. 
Repeat as necessary.
\end{description}
\end{PROOF}

\noindent{}A recursive proof would be more rigorous---it would have a clause for each \GSL{} connective, and would explain in each clause how to construct from each subsentence another \CAPS{tfe} subsentence that is in \CAPS{dnf}. 

\CAPS{dnf} sentences allow us see some of the advantages of formal languages.  For instance, we can construct a simple, mechanical process that will tell us when a \CAPS{dnf} sentence is \CAPS{tff}.

Let $\CAPPHI$ be some \CAPS{dnf} sentence, and let $\CAPTHETA_1$, $\CAPTHETA_2$, $\CAPTHETA_3$, \ldots, and $\CAPTHETA_n$ each be the disjuncts of $\CAPPHI$.  We can see that $\CAPPHI$ is \CAPS{tff} \Iff every disjunct $\CAPTHETA_i$ is \CAPS{tff}.  Because each $\CAPTHETA_i$ is a conjunction with negated and unnegated sentence letters as the conjuncts, there is only one way that it can be \CAPS{tff}.  A $\CAPTHETA_i$ is \CAPS{tff} \Iff it has some sentence letter $\CAPPSI$ as one conjunct and $\negation{\CAPPSI}$ as another.

It follows that we can use the following process to determine whether a \CAPS{dnf} sentence $\CAPPHI$ is \CAPS{tff}. Check every disjunct of $\CAPPHI$ to see if it has some $\CAPPSI$ and $\negation{\CAPPSI}$ as conjuncts.  If so, then $\CAPPHI$ is \CAPS{tff}; otherwise it isn't.  For example, consider the following \CAPS{dnf} sentence:

\begin{center}
\noindent{}$\disjunction{\parconjunction{\Ql}{\conjunction{\Rl}{\negation{\Ql}}}}{\disjunction{\parconjunction{\negation{\Ql}}{\conjunction{\Rl}{\Rl}}}{\parconjunction{\Ol}{\conjunction{\Rl}{\negation{\Ol}}}}}$
\end{center}

\noindent{}The first disjunct, $\parconjunction{\Ql}{\conjunction{\Rl}{\negation{\Ql}}}$, is \CAPS{tff} because it has $\Ql$ and $\negation{\Ql}$ as conjuncts; the third disjunct, $\parconjunction{\Ol}{\conjunction{\Rl}{\negation{\Ol}}}$, is also \CAPS{tff}.  But the second disjunct, $\parconjunction{\negation{\Ql}}{\conjunction{\Rl}{\Rl}}$, isn't \CAPS{tff}. So, the whole sentence isn't \CAPS{tff}.

If we were to replace the second disjunct with $\parconjunction{\negation{\Ql}}{\conjunction{\Rl}{\negation{\Rl}}}$, so that the new whole sentence is:

\begin{center}
	\noindent{}$\disjunction{\parconjunction{\Ql}{\conjunction{\Rl}{\negation{\Ql}}}}{\disjunction{\parconjunction{\negation{\Ql}}{\conjunction{\Rl}{\negation{\Rl}}}}{\parconjunction{\Ol}{\conjunction{\Rl}{\negation{\Ol}}}}}$
\end{center}

\noindent{}\ldots then the result \emph{is} \CAPS{tff}, because each disjunct has a sentence letter and its negation as conjuncts.

We can now construct a mechanical method that can determine whether \emph{any} \GSL{} sentence $\CAPPHI$ is \CAPS{tff}.  First, we use the process in \ref{Disjunctive Normal Form Theorem} to construct an equivalent \CAPS{dnf} sentence, $\CAPPHI^*$.  Then we use the method given above to determine whether $\CAPPHI^*$ is \CAPS{tff}.  Because they are equivalent, $\CAPPHI^*$ is \CAPS{tff} \Iff $\CAPPHI$ is \CAPS{tff}.  No creativity is needed to apply this method---each individual step requires nothing more than following simple instructions.

We can even extend this method to determine whether any \GSL{} sentence is \CAPS{tft}.  We will take advantage of the fact that if you put a negation in front of a \CAPS{tft} sentence $\CAPPHI$, the resulting sentence, $\negation{\CAPPHI}$, is \CAPS{tff}.  So, all that is necessary to see whether $\CAPPHI$ is \CAPS{tft} is to negate $\CAPPHI$, put $\negation{\CAPPHI}$ into \CAPS{dnf}, and then to see whether the final result is \CAPS{tff}.  If so, then $\CAPPHI$ is \CAPS{tft}!  If not, then $\CAPPHI$ isn't.  As before, this process is entirely mechanical or \emph{formal}.  Note that, along with the truth table method in section \ref{Proceduresfortesting}, we now have two rather different formal methods for determining logical truth in \GSL{}.  In later chapters our methods will be closer to the \CAPS{dnf} approach.

Any given sentence of \GSL{} is truth functionally equivalent to more than one \CAPS{dnf} sentence. 
A sentence has \CAPS{dnf}s that differ slightly for at least two reasons.
First, for any sentence $\CAPPHI$ and sentence letter $\CAPPSI$, if $\CAPPHI$ is in \CAPS{dnf}, then $\disjunction{\CAPPHI}{\parconjunction{\CAPPSI}{\negation{\CAPPSI}}}$ is \CAPS{tfe} to $\CAPPHI$ and also in \CAPS{dnf}.
Second, sometimes a sentence in \CAPS{dnf} can be simplified. Thus 
\begin{menumerate}
\item $\disjunction{\parconjunction{\Ql}{\conjunction{\Rl}{\Ol}}}{\disjunction{\parconjunction{\Ql}{\conjunction{\Rl}{\Nl}}}{\parconjunction{\Ql}{\conjunction{\Rl}{\negation{\Ol}}}}}$
\end{menumerate} can be simplified to the \CAPS{dnf}
\begin{samepage}
\begin{menumerate}
\item $\disjunction{\parconjunction{\Ql}{\Rl}}{\parconjunction{\Ql}{\conjunction{\Rl}{\Nl}}}$
\end{menumerate} and further to 
\begin{menumerate}
\item $\parconjunction{\Ql}{\Rl}$.
\end{menumerate}
\end{samepage}

%\bigskip
%%%%%%%%%%%%%%%%%%%%%%%%%%%%%%%%%%%%%%%%%%%%%%%%%%
\section[Truth Functional Expressiveness]{Truth Functional Expressiveness}\label{Truth Functional Expressiveness} 
%%%%%%%%%%%%%%%%%%%%%%%%%%%%%%%%%%%%%%%%%%%%%%%%%%


The definition of truth in a model (def. \pmvref{True on a GSL interpretation}) associates each of the logical connectives of \GSL{} with a truth function.\footnote{See section \ref{Truth Functions Truth Tables and Boolean Operators} for more details.}
We can think of the logical connectives of \GSL{} as truth functions---i.e., as having a meaning that's exhausted by the definition of truth. 
One might ask whether the five logical connectives of \GSL{} cover all the possible logical connectives.

In one sense it's obvious that they do not. 
For example, we could introduce a new connective, $\%$, choose some number of places for it, and give some clause that describes how the truth value of a sentence with $\%$ as the main connective depends on the truth value of the component parts. 
As long as this clause differs from any of those in the definition of truth, $\%$ is distinct from the five in \GSL{}.

But although the five logical connectives of \GSL{} obviously do not exhaust all the possible connectives, there's still a sense in which they might indirectly cover them all. 
Even if $\%$ is distinct from all the connectives of \GSL{}, maybe there's still some sentence schema that is truth functionally equivalent to $\%$, and that uses only (but not necessarily all of) the five connectives of \GSL{}. 
For example, say $\%$ is a 3-place connective, such that a sentence $\CAPTHETA_1$ $\%$ $\CAPTHETA_2$ $\%$ $\CAPTHETA_3$ is true on a model $\IntA$ \Iff at least two of the $\CAPTHETA$ are true on $\IntA$.  So, the sentence is true if $\CAPTHETA_1$ and $\CAPTHETA_2$ are true, or if $\CAPTHETA_1$ and $\CAPTHETA_3$ are true, or if $\CAPTHETA_2$ and $\CAPTHETA_3$ are true.  We can express this without \mention{$\%$}, using $\WEDGE$ for \mention{and} and $\VEE$ for \mention{or}:  $\disjunction{\parconjunction{\CAPTHETA_1}{\CAPTHETA_2}}{\disjunction{\parconjunction{\CAPTHETA_1}{\CAPTHETA_3}}{\parconjunction{\CAPTHETA_2}{\CAPTHETA_3}}}$.  Even though $\%$ is a connective that isn't in our language, we can use the connectives of \GSL{} to construct a truth functionally equivalent sentence. 


A logical connective $\%$ is \niidf{definable}\index{definability} in terms of some set of other logical connectives $\Delta$ \Iff there's some sentence schema using only connectives from $\Delta$ that's truth functionally equivalent to $\%$. 
With this in mind, we can ask whether every logical connective is definable in terms of the five connectives of \GSL{}.
We can also ask if any of the connectives of \GSL{} are definable in terms of the others.
We answer the first of these questions with theorem \pmvref{Truth-functional Expressive Completeness of GSL}.
The second we consider now in the following examples.\footnote{See \citetext{\citealt{Post1921}, \citealt[17]{Hodges2001}}.}

\begin{majorILnc}{\LnpEC{GSL Connective ID 1}}
	We can define conjunction using negation and disjunction.
\end{majorILnc}
\begin{PROOF}
	Any sentence $\conjunction{\CAPPHI_1}{\conjunction{\ldots}{\CAPPHI_{\integer{n}}}}$ is \CAPS{tfe} to the sentence $\negation{\pardisjunction{\negation{\CAPPHI_1}}{\disjunction{\ldots}{\negation{\CAPPHI_{\integer{n}}}}}}$.
\end{PROOF}

\begin{majorILnc}{\LnpEC{GSL Connective ID 2}}
	We can define disjunction using negation and conjunction.
\end{majorILnc}
\begin{PROOF}
	Any sentence $\disjunction{\CAPPHI_1}{\disjunction{\ldots}{\CAPPHI_{\integer{n}}}}$ is \CAPS{tfe} to the sentence $\negation{\parconjunction{\negation{\CAPPHI_1}}{\conjunction{\ldots}{\negation{\CAPPHI_{\integer{n}}}}}}$.
\end{PROOF}

\begin{majorILnc}{\LnpEC{GSL Connective ID two and half}}
	We can define conditional using negation and disjunction.
\end{majorILnc}
\begin{PROOF}
	Any sentence $\horseshoe{\CAPPHI_1}{\CAPPHI_2}$ is \CAPS{tfe} to the sentence $\disjunction{\negation{\CAPPHI_1}}{\CAPPHI_2}$.
\end{PROOF}

\begin{majorILnc}{\LnpEC{GSL Connective ID 3}}
	The pairs $\NEGATION$ and $\WEDGE$, and $\NEGATION$ and $\VEE$ are each adequate to define the remaining connectives in \GSL{}.
\end{majorILnc}
\begin{PROOF}
	By example \ref{GSL Connective ID 2}, with $\NEGATION$ and $\WEDGE$ we can define $\VEE$. 
	By example \ref{TFE Ex 2}, $\horseshoe{\CAPPHI}{\CAPTHETA}$ and $\disjunction{\negation{\CAPPHI}}{\CAPTHETA}$ are \CAPS{tfe}.
	So we can define $\HORSESHOE$ with $\NEGATION$ and $\WEDGE$. 
	As the reader can check, $\triplebar{\CAPPHI}{\CAPPSI}$ is \CAPS{tfe} to $\conjunction{\parhorseshoe{\CAPPHI}{\CAPPSI}}{\parhorseshoe{\CAPPSI}{\CAPPHI}}$.
	So we can define $\TRIPLEBAR$ with $\NEGATION$ and $\WEDGE$.
	
	We leave it to the reader to show that $\NEGATION$ and $\VEE$ are adequate to define the remaining connectives in \GSL{}.
\end{PROOF}

\noindent{}We don't actually need all five connectives, but for the sake of convenience we keep them all.  We also asked the following: 
Are the five operations we have enough? 
That is, are there other logical operations we can't express and should add notation for? 
It turns out that our connectives are adequate. 
There are no other logical operations we can't express.
Although it does not strictly depend on the \CAPS{dnf} theorem, the idea behind that theorem lets us prove this.
\begin{THEOREM}{\LnpTC{Truth-functional Expressive Completeness of GSL} The Truth-functional Expressive Completeness Theorem:}
Any truth-functional connective of any fixed number of arguments (ternary, quadernary, etc.) is already expressible in \GSL{}.
\end{THEOREM}
\begin{PROOF}
Any truth functional connective of a fixed number of arguments assigns $\TrueB$ or $\FalseB$ depending only on the values of the components, so it can be exactly described by a truth table. 
For example, consider the 4-place operation \% given by truth \mbox{table \ref{DNFtruthtable}} below.
\begin{table}[!ht]
\begin{center}
\begin{tabular}{ c c c c c}
$\CAPPHI_1$ & $\CAPPHI_2$ & $\CAPPHI_3$ & $\CAPPHI_4$ & $\text{\%}(\CAPPHI_1,\CAPPHI_2,\CAPPHI_3,\CAPPHI_4)$ \\
\hline
$ $ $ $ \\[-.25cm]
$\TrueB$ & $\TrueB$ & $\TrueB$ & $\TrueB$ & $\TrueB$ \\
$\TrueB$ & $\TrueB$ & $\TrueB$ & $\FalseB$&$\FalseB$ \\
$\TrueB$ & $\TrueB$ & $\FalseB$ & $\TrueB$ & $\TrueB$ \\
$\TrueB$ & $\TrueB$ & $\FalseB$ & $\FalseB$  &$\FalseB$ \\
$\TrueB$ &  $\FalseB$& $\TrueB$ & $\TrueB$	&$\FalseB$ \\
$\TrueB$ & $\FalseB$ & $\TrueB$ & $\FalseB$	& $\TrueB$ \\
$\TrueB$ &$\FalseB$  & $\FalseB$& $\TrueB$	&$\FalseB$ \\
$\TrueB$ & $\FalseB$ &$\FalseB$	& $\FalseB$	&$\FalseB$ \\
$\FalseB$	& $\TrueB$ & $\TrueB$ & $\TrueB$	& $\TrueB$ \\
$\FalseB$	& $\TrueB$ & $\TrueB$ & $\FalseB$	&$\FalseB$ \\
$\FalseB$	& $\TrueB$ & $\FalseB$&	$\TrueB$ &$\FalseB$ \\
$\FalseB$	& $\TrueB$ & $\FalseB$& $\FalseB$	&$\FalseB$ \\
$\FalseB$	& $\FalseB$	& $\TrueB$ & $\TrueB$	&$\FalseB$ \\
$\FalseB$	& $\FalseB$	& $\TrueB$ & $\FalseB$	&$\FalseB$ \\
$\FalseB$	& $\FalseB$	& $\FalseB$& $\TrueB$	& $\TrueB$ \\
$\FalseB$	& $\FalseB$& $\FalseB$& $\FalseB$	&$\FalseB$ \\
\end{tabular}
\end{center}
\caption{Truth Table for \%}
\label{DNFtruthtable}
\end{table}
We could attempt to find a complicated sentence in terms of various
connectives that would express this, but it will be better for our purposes to construct
a \CAPS{dnf} equivalent systematically. We know from the first line that the expression $\text{\%}(\CAPPHI_1,\CAPPHI_2,\CAPPHI_3,\CAPPHI_4)$ is
true when all components are, that is, if $\parconjunction{\CAPPHI_1}{\conjunction{\CAPPHI_2}{\conjunction{\CAPPHI_3}{\CAPPHI_4}}}$ is true; we know from the third line it is true when the first two, $\CAPPHI_1$ and $\CAPPHI_2$, and fourth, $\CAPPHI_4$, are true and the third, $\CAPPHI_3$, false, i.e., $\parconjunction{\CAPPHI_1}{\conjunction{\CAPPHI_2}{\conjunction{\negation{\CAPPHI_3}}{\CAPPHI_4}}}$. We also know it is true when only the first, $\CAPPHI_1$, and third, $\CAPPHI_3$, are true, i.e., $\parconjunction{\CAPPHI_1}{\conjunction{\negation{\CAPPHI_2}}{\conjunction{\CAPPHI_3}{\negation{\CAPPHI_4}}}}$, when the first, $\CAPPHI_1$, is false and the other three, $\CAPPHI_2$, $\CAPPHI_3$, and $\CAPPHI_4$, are true i.e., $\parconjunction{\negation{\CAPPHI_1}}{\conjunction{\CAPPHI_2}{\conjunction{\CAPPHI_3}{\CAPPHI_4}}}$ and when all but the fourth, $\CAPPHI_4$, are false, i.e., $\parconjunction{\negation{\CAPPHI_1}}{\conjunction{\negation{\CAPPHI_2}}{\conjunction{\negation{\CAPPHI_3}}{\CAPPHI_4}}}$. Because each of these conjunctions is true exactly when the corresponding line is true the whole sentence will be true when any one of them is true, i.e., it is equivalent to the formula: 
\begin{menumerate} 
\item $\disjunction{\parconjunction{\CAPPHI_1}{\conjunction{\CAPPHI_2}{\conjunction{\CAPPHI_3}{\CAPPHI_4}}}}{\disjunction{\parconjunction{\CAPPHI_1}{\conjunction{\CAPPHI_2}{\conjunction{\negation{\CAPPHI_3}}{\CAPPHI_4}}}}{\parconjunction{\CAPPHI_1}{\conjunction{\negation{\CAPPHI_2}}{\conjunction{\CAPPHI_3}{\negation{\CAPPHI_4}}}}}}\:\VEE$\\ $\disjunction{\parconjunction{\negation{\CAPPHI_1}}{\conjunction{\CAPPHI_2}{\conjunction{\CAPPHI_3}{\CAPPHI_4}}}}{\parconjunction{\negation{\CAPPHI_1}}{\conjunction{\negation{\CAPPHI_2}}{\conjunction{\negation{\CAPPHI_3}}{\CAPPHI_4}}}}$
\end{menumerate}
We could simplify this sentence further, but that is not important for our purposes. 
We now can see how to read off from any truth table for any operation on any fixed number of sentences a \CAPS{dnf} representation that is equivalent. 
Our five connectives are enough. 
\end{PROOF}

%We know that we can define conjunction using negation and disjunction (ex. \pmvref{GSL Connective ID 1}), and we can define disjunction using conjunction and negation (ex. \pmvref{GSL Connective ID 2}).
Since either of the pairs \mention{$\NEGATION$} and \mention{$\!\WEDGE\!$}, or \mention{$\NEGATION$} and \mention{$\VEE$} are adequate to define the remaining connectives in \GSL{} (see ex. \pmvref{GSL Connective ID 1}) we can see that either of those pairs is adequate to define all truth-functional connectives.
We can even improve on that though, because there are two connectives either of which would be adequate all by itself. 
One is the Sheffer stroke\index{Sheffer stroke|see{NAND}}, named after the logician who first demonstrated its properties and the symbol he used, \mention{|}, but it is sometimes called NAND.\index{NAND} 
Its definition is that $(\CAPPHI_1|\CAPPHI_2|\ldots|\CAPPHI_{\integer{n}})$ is true \Iff at least one component is false. So, $(\CAPPHI_1|\CAPPHI_1)$ is equivalent to $\negation{\CAPPHI}$. 
And $(\negation{\CAPPHI_1}|\negation{\CAPPHI_2}|\ldots|\negation{\CAPPHI_{\integer{n}}})$ is true just in case at least one component is false. That means at least one $\CAPPHI_i$ is true, which is to say that the sentence is equivalent to a disjunction. 
The other connective that is adequate by itself is NOR,\index{NOR} which is defined to be true \Iff all the components are false. It is left to the reader as an optional exercise to show how to define the other connectives using NOR. 

Thus, to get a language just as expressive as \GSL{}, we only need one logical connective (either NAND or NOR), not five. 
If we had a taste for cutting down basic symbols, we could go further and generate an infinite set of sentence letters by using just one symbol, \mention{$\Al$}, and generating new sentence letters by concatenating prime marks \mention{$'$} to it. 
Then all we need are parentheses (though there are ways to do without these too). 
Such a language is sparse, but it is just as expressive as \GSL{}.  Most people would find such a language difficult to work with, but computers love them.



%%%%%%%%%%%%%%%%%%%%%%%%%%%%%%%%%%%%%%%%%%%%%%%%%%
\section{Exercises}
%%%%%%%%%%%%%%%%%%%%%%%%%%%%%%%%%%%%%%%%%%%%%%%%%%

\notocsubsection{Recursive Definition Problems}{ex:Recursive Definitions Problems}

\begin{enumerate}
\item Although it's not framed as one, definition \mvref{Order} of order is a recursive definition. Rewrite it so that the base, generating, and closure clauses are explicit. 
\item Although we don't give a recursive definition, a recursive definition can be given for the unofficial \GSL{} sentences. (Definition \pmvref{Unofficial Sentence of GSL} is the definition we give.) Write down a recursive definition for unofficial \GSL{} sentences.
\end{enumerate}

\notocsubsection{Construction Trees}{ex:Construction Trees}
Write the construction tree for each of the following \GSL{} sentences.
\begin{multicols}{2}
\begin{enumerate}
\item $\negation{\negation{\negation{\Bl}}}$
\item $\negation{\pardisjunction{\Bl}{\parhorseshoe{\Al}{\Al}}}$
\item $\pardisjunction{\negation{\Bl}}{\parhorseshoe{\Al}{\Al}}$
\item $\parhorseshoe{\partriplebar{\parconjunction{\Al}{\Bl}}{\Al}}{\negation{\parhorseshoe{\Bl}{\Cl}}}$
\item $\parhorseshoe{\parconjunction{\partriplebar{\Al}{\Bl}}{\Al}}{\negation{\parhorseshoe{\Bl}{\Cl}}}$
\item $\parconjunction{\Pl}{\parconjunction{\Ql}{\Rl}}$
\item $\parconjunction{\parconjunction{\Pl}{\Ql}}{\Rl}$
\item $\parhorseshoe{\parconjunction{\Pl}{\negation{\Rl}}}{\negation{\Ql}}$
\item $\parconjunction{\Pl}{\parhorseshoe{\negation{\Rl}}{\negation{\Ql}}}$
\item $\negation{\pardisjunction{\parhorseshoe{\Pl}{\Ql}}{\parhorseshoe{\Pl}{\Ql}}}$
\end{enumerate}
\end{multicols}

\notocsubsection{Official and Unofficial Sentences}{ex:Official and Unofficial Sentences} 
Which of these are official sentences? Which are unofficial? Which are neither official nor unofficial sentences (i.e., not a sentence in any sense)? If
neither, how could you make it either an official or unofficial sentence? Note: there
might be multiple different ways to make it an official or unofficial sentence. Finally, if it's a sentence (official or unofficial), then give its order and the number of subsentences in it. 
\begin{multicols}{2}
\begin{enumerate}
\item {$\parhorseshoe{\Al}{\conjunction{\Bl}{\Cl}}$}
\item {$\parhorseshoe{\Al}{\bparhorseshoe{\Bl}{\Cl}}$}
\item {$\horseshoe{\Al}{\parconjunction{\Bl}{\conjunction{\Cl}{\Bl}}}$}
\item {$\parhorseshoe{\Al}{\parconjunction{\CAPTHETA}{\Cl}}$}
\item {$\parhorseshoe{\Al}{\parconjunction{\Bl}{\disjunction{\Cl}{\Dl}}}$}
\item {$\parhorseshoe{\Al}{\parconjunction{\Zl}{\Cl}}$}
\item {$\parhorseshoe{\negation{\Al}}{\parconjunction{\Bl_{374}}{\Cl}}$}
\item {$\parhorseshoe{\Al}{\parconjunction{\Bl}{\Cl}}$}
\item {$\parnegation{\parconjunction{\Bl}{\Cl}}$}
\item {$\bparconjunction{\Bl}{\conjunction{\negation{\negation{\Ml}}}{\Dl}}$}
\end{enumerate}
\end{multicols}

\notocsubsection{Truth in a Model}{ex:GSLTruth in an Interpretation}
Consider the model $\IntA_1$ such that $\IntA_1(\Al)=\TrueB$, $\IntA_1(\Bl)=\FalseB$, $\IntA_1(\Cl)=\TrueB$, $\IntA_1(\Dl)=\FalseB$, and $\IntA_1(\El)=\TrueB$; and the model $\IntA_2$ such that $\IntA_2(\Al)=\TrueB$, $\IntA_2(\Bl)=\TrueB$, $\IntA_2(\Cl)=\FalseB$, $\IntA_2(\Dl)=\FalseB$, and $\IntA_2(\El)=\FalseB$.
Give the truth values of each of the following \GSL{} sentences on each of these two models.
\begin{multicols}{2}
\begin{enumerate}
\item $\horseshoe{\pardisjunction{\Al}{\Bl}}{\parconjunction{\Cl}{\Dl}}$
\item $\horseshoe{\pardisjunction{\Al}{\Bl}}{\parconjunction{\Cl}{\negation{\Dl}}}$
\item $\horseshoe{\parconjunction{\Cl}{\Dl}}{\pardisjunction{\Al}{\Bl}}$
\item $\conjunction{\parhorseshoe{\Al}{\Cl}}{\El}$
\item $\negation{\conjunction{\parhorseshoe{\Bl}{\El}}{\Dl}}$
\item $\disjunction{\negation{\pardisjunction{\Al}{\Bl}}}{\parconjunction{\Al}{\conjunction{\Bl}{\parhorseshoe{\El}{\El}}}}$
\item $\disjunction{\Al}{\parhorseshoe{\Bl}{\parhorseshoe{\El}{\El}}}$
\item $\disjunction{\parconjunction{\Al}{\El}}{\disjunction{\parconjunction{\negation{\Dl}}{\Cl}}{\parconjunction{\Bl}{\negation{\Al}}}}$
\end{enumerate}
\end{multicols}

\notocsubsection{\CAPS{tft}, \CAPS{tff}, and \CAPS{tfc}}{ex:TFT, TFF, and TFI}
For each of the following say whether the sentence is \CAPS{tfc}, \CAPS{tff} or \CAPS{tft}. 
If it is \CAPS{tfc}, give a model which makes the sentence $\True$ and another model which makes it $\False$. 
If it is \CAPS{tff}, justify your answer without truth tables (i.e., explain why there is no model which makes the sentence $\True$). 
If it is \CAPS{tft}, again justify your answer without truth tables (i.e., explain why every model makes the sentence $\True$).

%\begin{multicols}{2}
\begin{enumerate}
\item {$\horseshoe{\parhorseshoe{\Al}{\Bl}}{\pardisjunction{\negation{\Bl}}{\negation{\Al}}}$}
\item {$\horseshoe{\parconjunction{\Al}{\Bl}}{\partriplebar{\Al}{\Bl}}$}
\item {$\horseshoe{\pardisjunction{\negation{\Al}}{\negation{\Bl}}}{\negation{\parconjunction{\Al}{\Bl}}}$}
\item {$\disjunction{\Al}{\parhorseshoe{\Al}{\Bl}}$}
\item {$\horseshoe{\negation{\pardisjunction{\Al}{\Bl}}}{\parconjunction{\negation{\Al}}{\negation{\Bl}}}$}
\item {$\horseshoe{\negation{\partriplebar{\Al}{\Bl}}}{\partriplebar{\negation{\Al}}{\Bl}}$}
\item {$\horseshoe{\parconjunction{\Al}{\pardisjunction{\Bl}{\Cl}}}{\pardisjunction{\parconjunction{\Al}{\Bl}}{\Cl}}$}
\item {$\horseshoe{\negation{\Al}}{\parhorseshoe{\Al}{\Bl}}$}
\item {$\horseshoe{\parconjunction{\negation{\Al}}{\negation{\Bl}}}{\negation{\pardisjunction{\Al}{\Bl}}}$}
\item {$\horseshoe{\Al}{\parhorseshoe{\Al}{\Bl}}$}
\item {$\horseshoe{\partriplebar{\Al}{\Bl}}{\parconjunction{\Al}{\Bl}}$}
\item {$\horseshoe{\negation{\parhorseshoe{\Al}{\Bl}}}{\Al}$}
\item {$\horseshoe{\negation{\parconjunction{\Al}{\Bl}}}{\pardisjunction{\negation{\Al}}{\negation{\Bl}}}$}
\item {$\horseshoe{\Al}{\parhorseshoe{\Bl}{\Al}}$}
\item {$\horseshoe{\parhorseshoe{\Al}{\Bl}}{\parconjunction{\Al}{\negation{\Bl}}}$}
\item {$\negation{\pardisjunction{\Al}{\parhorseshoe{\Al}{\Bl}}}$}
\end{enumerate}
%\end{multicols}


\notocsubsection{Entailment Problems for \GSL{}}{Entailment Problems for GSL}
For each of the following, without using truth tables show whether or not the entailment holds. 
There are a number of different methods for thinking through these problems.  
Remember that an entailment means that on all $\IntA$ if the \CAPS{lhs} is $\True$ then the \CAPS{rhs} is also $\True$.  One approach is to show that making the \CAPS{lhs} $\True$ forces the \CAPS{rhs} to be $\True$.  Another is to show that making the \CAPS{rhs} $\False$ forces the \CAPS{lhs} to be $\False$.  Both are examples of arguing that it is not possible for the \CAPS{lhs} to be $\True$ and the \CAPS{rhs} $\False$.  Another method is showing that all $\IntA$ either make the \CAPS{lhs} $\False$ or the \CAPS{rhs} $\True$.  If the entailment does not hold, you can show this by providing a counterexample.
\begin{multicols}{2}
\begin{enumerate}
\item {$\negation{\Al}\sdtstile{}{}\parhorseshoe{\Al}{\Bl}$}
\item {$\parhorseshoe{\Al}{\Bl}\sdtstile{}{}\pardisjunction{\negation{\Bl}}{\negation{\Al}}$}
\item {$\parconjunction{\negation{\Al}}{\negation{\Bl}}\sdtstile{}{}\:\negation{\pardisjunction{\Al}{\Bl}}$}
\item\label{HW Entailment 4} {$\pardisjunction{\negation{\Al}}{\negation{\Bl}}\sdtstile{}{}\:\negation{\parconjunction{\Al}{\Bl}}$}
\item {$\Al\sdtstile{}{}\parhorseshoe{\Al}{\Bl}$}
\item {$\negation{\partriplebar{\Al}{\Bl}}\sdtstile{}{}\partriplebar{\negation{\Al}}{\Bl}$} 
\vfill
\item {$\parconjunction{\Al}{\pardisjunction{\Bl}{\Cl}}\sdtstile{}{}\pardisjunction{\parconjunction{\Al}{\Bl}}{\Cl}$}
\item {$\negation{\pardisjunction{\Al}{\Bl}}\sdtstile{}{}\parconjunction{\negation{\Al}}{\negation{\Bl}}$}
\item {$\Al\sdtstile{}{}\parhorseshoe{\Bl}{\Al}$}
\item {$\parconjunction{\Al}{\Bl}\sdtstile{}{}\partriplebar{\Al}{\Bl}$}
\item {$\negation{\parhorseshoe{\Al}{\Bl}}\sdtstile{}{}\parhorseshoe{\Bl}{\Al}$}
\item\label{HW Entailment 12} {$\negation{\parconjunction{\Al}{\Bl}}\sdtstile{}{}\pardisjunction{\negation{\Al}}{\negation{\Bl}}$}
\item {$\negation{\parhorseshoe{\Al}{\Bl}}\sdtstile{}{}\Al$}
\item {$\partriplebar{\Al}{\Bl}\sdtstile{}{}\parconjunction{\Al}{\Bl}$}
\end{enumerate}
\end{multicols}

%\notocsubsection{Testing for Entailment}{ex:Testing for Entailment}
%For each problem in exercise \ref{Entailment Problems for GSL}, use truth tables to test whether the entailment holds. 

\notocsubsection{More Entailment Problems for \GSL{}}{ex:More Entailment Problems for GSL} 
Show whether, for any \GSL{} sentence $\CAPPHI$, $\CAPTHETA$, and $\CAPPSI$, each of the following statements is true.
%Show whether each of the following entailments hold, for any sentences got by substituting into the schemas.
\begin{enumerate}
\item {$\sdtstile{}{}\disjunction{\parhorseshoe{\CAPPHI}{\CAPTHETA}}{\parhorseshoe{\CAPTHETA}{\CAPPHI}}$}
\item {Either $\sdtstile{}{}\parhorseshoe{\CAPPHI}{\CAPTHETA}$, or $\sdtstile{}{}\parhorseshoe{\CAPTHETA}{\CAPPHI}$}
\item {If $\sdtstile{}{}\parhorseshoe{\CAPPHI}{\CAPTHETA}$ and $\sdtstile{}{}\CAPPHI$, then $\sdtstile{}{}\CAPTHETA$}
\item {If $\sdtstile{}{}\parhorseshoe{\CAPPHI}{\CAPTHETA}$ and $\sdtstile{}{}\:\negation{\CAPPHI}$, then $\sdtstile{}{}\:\negation{\CAPTHETA}$}
\item {$\sdtstile{}{}\disjunction{\parhorseshoe{\CAPPHI}{\CAPTHETA}}{\parhorseshoe{\CAPTHETA}{\CAPPSI}}$}
\item {If $\parconjunction{\CAPPHI}{\CAPTHETA}\sdtstile{}{}\CAPPSI$, then both $\CAPPHI\sdtstile{}{}\CAPPSI$ and $\CAPTHETA\sdtstile{}{}\CAPPSI$}
\item {If $\CAPPSI\sdtstile{}{}\parconjunction{\CAPPHI}{\CAPTHETA}$, then both $\CAPPSI\sdtstile{}{}\CAPTHETA$ and $\CAPPSI\sdtstile{}{}\CAPPHI$}
\end{enumerate}

\notocsubsection{Even More Entailment Problems: Truth-preservation Lemma}{exercises:truth-preservation lemma} 
Show that, for any \GSL{} sentence $\CAPPHI$, $\CAPTHETA$, and $\CAPPSI$, each of the following entailments holds.
Showing that these entailments hold will be helpful later, since they are needed in the proof of theorems \mvref{Soundess of Basic GSD Rules} and \mvref{Soundness of Std Shortcut Applications}.
\begin{multicols}{2}
\begin{enumerate}
\item $\CAPPHI\sdtstile{}{}\CAPPHI$.
\item $\horseshoe{\CAPTHETA}{\CAPPSI},\CAPTHETA\sdtstile{}{}\CAPPSI$
\item $\conjunction{\CAPTHETA}{\CAPPSI}\sdtstile{}{}\CAPPSI$
\item $\conjunction{\CAPTHETA}{\CAPPSI}\sdtstile{}{}\CAPTHETA$
\item $\CAPTHETA,\CAPPSI\sdtstile{}{}\conjunction{\CAPTHETA}{\CAPPSI}$
\item $\disjunction{\CAPTHETA}{\CAPPSI},\horseshoe{\CAPTHETA}{\CAPPHI},\horseshoe{\CAPPSI}{\CAPPHI}\sdtstile{}{}\CAPPHI$
\item $\CAPTHETA\sdtstile{}{}\disjunction{\CAPTHETA}{\CAPPSI}$
\item $\horseshoe{\CAPTHETA}{\parconjunction{\CAPPSI}{\negation{\CAPPSI}}}\sdtstile{}{}\negation{\CAPTHETA}$
\item $\horseshoe{\negation{\CAPTHETA}}{\parconjunction{\CAPPSI}{\negation{\CAPPSI}}}\sdtstile{}{}\CAPTHETA$
\item $\horseshoe{\CAPTHETA}{\CAPPSI},\horseshoe{\CAPPSI}{\CAPTHETA}\sdtstile{}{}\triplebar{\CAPTHETA}{\CAPPSI}$
\item $\triplebar{\CAPTHETA}{\CAPPSI},\CAPPSI\sdtstile{}{}\CAPTHETA$
\item $\triplebar{\CAPTHETA}{\CAPPSI},\CAPTHETA\sdtstile{}{}\CAPPSI$
\item $\horseshoe{\CAPPSI}{\CAPTHETA},\negation{\CAPTHETA}\sdtstile{}{}\negation{\CAPPSI}$
\item $\disjunction{\CAPPSI}{\CAPTHETA},\negation{\CAPTHETA}\sdtstile{}{}\CAPPSI$
\item $\disjunction{\CAPTHETA}{\CAPPSI},\negation{\CAPPSI}\sdtstile{}{}\CAPTHETA$
\item $\CAPTHETA,\negation{\CAPTHETA}\sdtstile{}{}\CAPPSI$
\item $\triplebar{\CAPPSI}{\CAPTHETA}\sdtstile{}{}\triplebar{\negation{\CAPPSI}}{\negation{\CAPTHETA}}$
\end{enumerate}
\end{multicols}

\notocsubsection{Truth Functional Equivalence}{exercises:GSDTFETheorem} 
Without using truth tables, show that each of the following pairs of sentences is \CAPS{tfe}, for any sentences got by substituting into the schemas. 
Showing that these pairs are \CAPS{tfe} will be helpful later, since it's both needed for the proof of theorem \mvref{Soundness of Std Shortcut Applications} and it amounts to proving theorem \mvref{ExchangeRuleGSDSoundnessLemma} (including for \Rule{$\TRIPLEBAR$-Exchange}), which is needed to prove theorem \mvref{ExchangeRuleGSDSoundness}.
\begin{enumerate}

\item $\negation{\parconjunction{\CAPPHI_1}{\conjunction{\ldots}{\CAPPHI_{\integer{n}}}}}$, $\disjunction{\negation{\CAPPHI_1}}{\disjunction{\ldots}{\negation{\CAPPHI_{\integer{n}}}}}$

%\item $\disjunction{\negation{\CAPPHI_1}}{\disjunction{\ldots}{\negation{\CAPPHI_{\integer{n}}}}}$, $\negation{\parconjunction{\CAPPHI_1}{\conjunction{\ldots}{\CAPPHI_{\integer{n}}}}}$
 
\item $\negation{\pardisjunction{\CAPPHI_1}{\disjunction{\ldots}{\CAPPHI_{\integer{n}}}}}$, $\conjunction{\negation{\CAPPHI_1}}{\conjunction{\ldots}{\negation{\CAPPHI_{\integer{n}}}}}$ 
 
%\item $\conjunction{\negation{\CAPPHI_1}}{\conjunction{\ldots}{\negation{\CAPPHI_{\integer{n}}}}}$, $\negation{\pardisjunction{\CAPPHI_1}{\disjunction{\ldots}{\CAPPHI_{\integer{n}}}}}$ 
 
\item $\negation{\negation{\CAPPHI}}$, $\CAPPHI$

%\item $\CAPPHI$, $\negation{\negation{\CAPPHI}}$ 

\item $\horseshoe{\CAPPHI}{\CAPTHETA}$, $\disjunction{\negation{\CAPPHI}}{\CAPTHETA}$ 

%\item $\disjunction{\negation{\CAPPHI}}{\CAPTHETA}$, $\horseshoe{\CAPPHI}{\CAPTHETA}$
 
\item $\horseshoe{\CAPPHI}{\CAPTHETA}$, $\horseshoe{\negation{\CAPTHETA}}{\negation{\CAPPHI}}$ 

%\item $\horseshoe{\negation{\CAPTHETA}}{\negation{\CAPPHI}}$, $\horseshoe{\CAPPHI}{\CAPTHETA}$ 
 
\item $\negation{\parhorseshoe{\CAPPHI}{\CAPTHETA}}$, $\conjunction{\CAPPHI}{\negation{\CAPTHETA}}$

%\item $\conjunction{\CAPPHI}{\negation{\CAPTHETA}}$, $\negation{\parhorseshoe{\CAPPHI}{\CAPTHETA}}$
 
\item $\conjunction{\CAPTHETA}{\pardisjunction{\CAPPHI_1}{\disjunction{\ldots}{\CAPPHI_{\integer{n}}}}}$, $\disjunction{\parconjunction{\CAPTHETA}{\CAPPHI_1}}{\disjunction{\ldots}{\parconjunction{\CAPTHETA}{\CAPPHI_{\integer{n}}}}}$

%\item $\disjunction{\parconjunction{\CAPTHETA}{\CAPPHI_1}}{\disjunction{\ldots}{\parconjunction{\CAPTHETA}{\CAPPHI_{\integer{n}}}}}$, $\conjunction{\CAPTHETA}{\pardisjunction{\CAPPHI_1}{\disjunction{\ldots}{\CAPPHI_{\integer{n}}}}}$
 

\item $\conjunction{\pardisjunction{\CAPPHI_1}{\disjunction{\ldots}{\CAPPHI_{\integer{n}}}}}{\CAPTHETA}$, $\disjunction{\parconjunction{\CAPPHI_1}{\CAPTHETA}}{\disjunction{\ldots}{\parconjunction{\CAPPHI_{\integer{n}}}{\CAPTHETA}}}$
 
%\item $\disjunction{\parconjunction{\CAPPHI_1}{\CAPTHETA}}{\disjunction{\ldots}{\parconjunction{\CAPPHI_{\integer{n}}}{\CAPTHETA}}}$, $\conjunction{\pardisjunction{\CAPPHI_1}{\disjunction{\ldots}{\CAPPHI_{\integer{n}}}}}{\CAPTHETA}$
 
 
\item $\disjunction{\CAPTHETA}{\parconjunction{\CAPPHI_1}{\conjunction{\ldots}{\CAPPHI_{\integer{n}}}}}$, $\conjunction{\pardisjunction{\CAPTHETA}{\CAPPHI_1}}{\conjunction{\ldots}{\pardisjunction{\CAPTHETA}{\CAPPHI_{\integer{n}}}}}$
 
%\item $\conjunction{\pardisjunction{\CAPTHETA}{\CAPPHI_1}}{\conjunction{\ldots}{\pardisjunction{\CAPTHETA}{\CAPPHI_{\integer{n}}}}}$, $\disjunction{\CAPTHETA}{\parconjunction{\CAPPHI_1}{\conjunction{\ldots}{\CAPPHI_{\integer{n}}}}}$

\item $\disjunction{\parconjunction{\CAPPHI_1}{\conjunction{\ldots}{\CAPPHI_{\integer{n}}}}}{\CAPTHETA}$, $\conjunction{\pardisjunction{\CAPPHI_1}{\CAPTHETA}}{\conjunction{\ldots}{\pardisjunction{\CAPPHI_{\integer{n}}}{\CAPTHETA}}}$

%\item $\conjunction{\pardisjunction{\CAPPHI_1}{\CAPTHETA}}{\conjunction{\ldots}{\pardisjunction{\CAPPHI_{\integer{n}}}{\CAPTHETA}}}$, $\disjunction{\parconjunction{\CAPPHI_1}{\conjunction{\ldots}{\CAPPHI_{\integer{n}}}}}{\CAPTHETA}$

\item $\triplebar{\CAPTHETA}{\CAPPSI}$, $\disjunction{\parconjunction{\CAPTHETA}{\CAPPSI}}{\parconjunction{\negation{\CAPTHETA}}{\negation{\CAPPSI}}}$

\end{enumerate}

%\notocsubsection{Testing for Truth Functional Equivalence}{exercises:TestingGSDTFETheorem} 
%Use truth tables to show that the sentences in each of the pairs given in exercise \ref{exercises:GSDTFETheorem} are \CAPS{tfe}. 

\notocsubsection{Relations Between \GSL{} Sentences}{ex:Relations Between GSL Sentences}
What relations hold among these sentences? Specifically, say whether they are contradictory, contrary, subcontrary, independent, or truth functionally equivalent. (So, you need to supply 30 answers for each problem: for each sentence $\CAPPHI$, you must determine for each of the 5 relations whether it holds between $\CAPPHI$ and each of the 6 other sentences.) 
\begin{multicols}{2}
\begin{enumerate}
\item {$\Al$}
\item {$\conjunction{\Al}{\Bl}$}
\item {$\conjunction{\negation{\Al}}{\Bl}$}
\item {$\horseshoe{\Al}{\Cl}$}
\item {$\horseshoe{\Al}{\negation{\Cl}}$}
\item {$\disjunction{\parconjunction{\Al}{\Bl}}{\Cl}$}
\end{enumerate}
\end{multicols}
\begin{enumerate}[start=7]
\item {$\conjunction{\Dl}{\negation{\Dl}}$}
\end{enumerate}

%\notocsubsection{Testing for Relations Between \GSL{} Sentences}{ex:Testing Relations Between GSL Sentences}
%Write a single joint truth table for the 7 sentences listed in exercise \ref{ex:Relations Between GSL Sentences}. 
%Read off from that table, for each pair of sentences, whether they are contradictory, contrary, subcontrary, independent, or truth functionally equivalent. 

\notocsubsection{Recursive Definitions and Proofs}{ex:Recursive Definitions and Proofs}

\begin{enumerate}
	\item Give a recursive definition of the even positive integers. 
	\item Give a recursive definition of the odd positive integers that are greater than 100. 
	\item Prove that all positive multiples of 10 are also multiples of 5.
	\item For each even positive integer $n$, prove that dividing $n$ by 2 results in another positive integer.
\end{enumerate}


\notocsubsection{SL Recursive Proofs}{ex:SL Recursive Proofs} 
Prove each of the following claims using a recursive proof. 
\begin{enumerate}
\item In every \GSL{} sentence which is \CAPS{tff}, there is a subsentence which is \CAPS{tfc}. \emph{Hint:} The proof and the basic idea behind it is very easy; don't over-think it. 
\item\label{localityoftruth} If two models $\IntA$ and $\IntA'$ agree on all of the sentence letters of $\CAPPHI$, then $\CAPPHI$ is $\True$ in $\IntA$ \Iff $\CAPPHI$ is $\True$ in $\IntA'$. (This is theorem \pmvref{thm:localityoftruth}.) To be explicit, we say that $\IntA$ and $\IntA'$ agree on the sentence letters of $\CAPPHI$ \Iff they assign the same truth value to each of those sentence letters, i.e. both assign $\True$ or both assign $\False$. We say that $\IntA$ and $\IntA'$ agree on a sentence \Iff they assign the same truth value to the sentence.
\item Show that every \CAPS{tff} sentence of \GSL{} contains at least one \mention{$\NEGATION$}. \emph{Hint:}
To prove this, it is useful to prove a more specific statement, namely that if $\CAPPHI$ is an \GSL{} sentence that does not contain any negations then $\CAPPHI$ is $\True$ on the model that assigns $\True$ to all sentence letters.
\item For every sentence $\CAPPHI$ of \GSL{}, the number of left parentheses occurring in $\CAPPHI$ is less than the number of subsentences. In other words, if LP$\CAPPHI$ is the number of left parentheses in $\CAPPHI$ and SS$\CAPPHI$ is the number of subsentences, then LP$\CAPPHI$ $<$ SS$\CAPPHI$.
\item The number of subsentences in any official \GSL{} sentence $\CAPPHI$ is equal to: the number of tokens of sentence letters in $\CAPPHI$ plus the number of tokens of negation in $\CAPPHI$ plus the number of tokens of left parentheses in $\CAPPHI$.
\item For every sentence $\CAPPHI$ of \GSL{}, there is a \CAPS{tfe} sentence $\CAPPHI'$ without conditionals or biconditionals.
\end{enumerate}

\notocsubsection{DNF}{ex:DNF} 
Put the following into disjunctive normal form.
\begin{multicols}{2}
\begin{enumerate}
\item {$\disjunction{\negation{\parhorseshoe{\Ql}{\Rl}}}{\parhorseshoe{\Ql}{\negation{\Rl}}}$}
\item {$\conjunction{\Ol}{\parhorseshoe{\Ol}{\Ql}}$}
\item {$\horseshoe{\bparhorseshoe{\parhorseshoe{\Ql}{\Rl}}{\Ql}}{\Ql}$}
\item {$\conjunction{\negation{\parhorseshoe{\Ql}{\Rl}}}{\pardisjunction{\Ol}{\Pl}}$}
\end{enumerate}
\end{multicols}


%\theendnotes



%%%%%%%%%%%%%%%%%%%%%%%%%%%%%%%%%%%%%%%%%%%%%%%%%%
\chapter{Quantifier Language I}\label{quantifierlogic1}
%%%%%%%%%%%%%%%%%%%%%%%%%%%%%%%%%%%%%%%%%%%%%%%%%%
% \AddToShipoutPicture*{\BackgroundPicB}

%%%%%%%%%%%%%%%%%%%%%%%%%%%%%%%%%%%%%%%%%%%%%%%%%%
\section{The Language \GQL{}1}
%%%%%%%%%%%%%%%%%%%%%%%%%%%%%%%%%%%%%%%%%%%%%%%%%%

%\setcounter{DefThm}{0}

\subsection{Sentences of \GQL{}1}\label{Sec:GQLSymbols1}
\GSL{} allows us to investigate certain aspects of logical consequence and English-language inference well, but leaves out a great deal of interest.  The \mention{atoms} of \GSL{} are sentence letters, which are each assigned either $\TrueB$ or $\FalseB$ by a model.  Sentence letters can only represent declarative sentences of the English language, which means that \GSL{} is too coarse-grained to mimic the \emph{parts} of a sentence.  While \GSL{} can capture some forms of logical consequence in English, it neglects others.  For example:

\begin{RESTARTmenumerate}
\item All women are mortal.
\item Ophelia is a woman.

Therefore,

\item Ophelia is mortal.
\end{RESTARTmenumerate}

\noindent{}The third sentence is a logical consequence of the first two, but \GSL{} isn't capable of representing this as an entailment.  Let sentence letter $\Al$ stand for \mention{All women are mortal,} let $\Bl$ stand for \mention{Ophelia is a woman,} and let $\Cl$ stand for \mention{Ophelia is mortal}.  
The entailment claim $\Al, \Bl \:\sdtstile{}{}\: \Cl$ does not hold, because there is a model $\IntA$ that assigns $\TrueB$ to $\Al$ and $\Bl$, but assigns $\FalseB$ to $\Cl$.  We need a formal language more expressive than \GSL{}.

Our new language needs symbols that represent named objects, including people like Ophelia.  It also needs symbols that mimic the predicates of English.  Predicates correspond roughly to what you get if you take an English sentence and remove a name, leaving a blank, e.g.: 

\begin{menumerate}
	\item \mention{John is tall} $\Rightarrow$ \mention{\_\_\_\_\_\_ is tall}
	\item \mention{Ophelia is a woman} $\Rightarrow$ \mention{\_\_\_\_\_\_ is a woman}
	\item \mention{Ophelia is mortal} $\Rightarrow$ \mention{\_\_\_\_\_\_ is mortal}	
\end{menumerate}

\noindent{}If we want to apply a predicate without invoking a name, we'll need to use a variable, which functions somewhat like a pronoun in English.  And to account for the word \mention{all} in our new language, we will need to use a \emph{quantifier}.  (We'll discuss quantifiers a bit later.)  In this chapter we outline a formal language with these features, and thus which can capture the kind of logical consequence exhibited above.

Many logic texts call the following language \PL{}, for \idf{predicate language}. 
We use \mention{\QL{}} for \idf{quantifier language}, because the quantifiers are more important than the predicates. However, before turning to the full language of \GQL{} (in the next chapter), we first consider a simpler sublanguage which we'll call \mention{\GQL{}1}.  We call it \mention{\GQL{}1} because the predicates will all be 1-place.\footnote{In \GQL{} there will be many-place predicates.  The basics of a less formal version of \GQL{}1 were developed by Aristotle over 2,000 years before Gottlob Frege and others developed the full language we're calling \GQL{}.  \GQL{} was a big step for mankind.}
 
\GQL{}1 has all the basic symbols of \GSL{}, plus a few more. 
\begin{majorILnc}{\LnpDC{Symbols of GQL1}}
The \df{basic symbols} of \GQL{}1 are:
\begin{cenumerate}
\item Logical Connectives: those of \GSL{}, plus $\forall$ and $\exists$
\item Punctuation Symbols: those of \GSL{}
\item Sentence Letters: those of \GSL{}
\item Individual Constants: $\constant{a}$, $\constant{b}$, $\constant{c}$, $\constant{d}$, $\ldots$, $\constant{p}$, $\constant{a}_1$, $\constant{b}_1$, $\constant{c}_1$, $\ldots$, $\constant{p}_1$, $\constant{a}_2$, $\ldots$
\item Individual Variables:\index{variables!individual (GQL1)|textbf} $\variable{u}$, $\variable{v}$, $\variable{w}$, $\variable{x}$, $\variable{y}$, $\variable{z}$, $\variable{u}_1$, $\variable{v}_1$, $\ldots$, $\variable{z}_1$, $\variable{u}_2$, $\ldots$
\item 1-Place Predicates: $\Ap{'}$, $\Bp{'}$, $\ldots$, $\Tp{'}$, $\Ap{'}_1$, $\Bp{'}_1$, $\ldots$, $\Tp{'}_1$, $\Ap{'}_2$, $\Bp{'}_2$, $\ldots$
\end{cenumerate}
\end{majorILnc}
\noindent{}The most prominent additions are the new logical connectives, the two quantifiers.  The first, \mention{$\forall$}, is called the \idf{universal quantifier}. 
It corresponds roughly to the words \mention{all} or \mention{every} in English. 
The second, \mention{$\exists$}, is called the \idf{existential quantifier}. 
It corresponds roughly to \mention{there exists}, \mention{there is}, or \mention{some}, as in \mention{Some elephants live a long time.}\footnote{Although Frege, Peirce, and Mitchell first introduced quantifiers, the notation used here comes from Russell, who Church \citeyearpar[288]{Church1956} says modified Peano's notation.}

The individual constants of \GQL{} correspond roughly to names in English. 
They are lowercase Roman letters that start at \mention{$\constant{a}$} and stop at \mention{$\constant{p}$}, and then they start over with subscripted integers.  So we have, for example, $\constant{a_1}$, $\constant{a_2}$, $\constant{a_3}$, and so on. 

Next, the individual variables correspond roughly to pronouns in English. 
They are Roman lowercase letters that go from \mention{$\variable{u}$} to \mention{$\variable{z}$} and then start over with subscripted positive integers. 


We also have 1-place predicates in \GQL{}1. The one-place predicates are capital Roman letters going from \mention{$\Ap{'}$} to \mention{$\Tp{'}$} and then starting over with subscripted integers.  There are an infinite number of each of the individual constants, variables, and 1-place predicates. 
This is so that however complex a sentence or argument we want to analyze, we never run out of \GQL{}1 symbols.

\subsection{Formulas of \GQL{}1}\label{Formulas of GQL1}
Our ultimate interest is in \emph{sentences} of \GQL{}1, but to get to the sentences we have to work with a larger set of strings called \mention{formulas.}\index{formulas} 
Formulas are defined recursively, starting with the atomic formulas (that is, those given by the base clause of the recursive definition).  
But before giving the definition we need to expand MathEnglish and add another kind of metavariable for \GQL{}1 individual variables.\index{variables!MathEnglish}\footnote{MathEnglish variables are symbols of the metalanguage while individual variables of \GQL{}1 are symbols of the object language.} 
We use lowercase Greek letters to stand for variables of \GQL{}1, usually but not necessarily always from the beginning of the Greek alphabet (e.g., \mention{$\ALPHA$} and \mention{$\BETA$}). Although we also used lowercase Greek letters as variables for \GSL{} sentences and will continue to do so for \GQL{}1 sentences and formulas, confusion shouldn't arise as we typically use \mention{$\CAPPHI$}, \mention{$\CAPPSI$}, and \mention{$\CAPTHETA$} for \GSL{} and \GQL{}1 sentences and use \mention{$\ALPHA$} and \mention{$\BETA$} for \GQL{}1 variables. 
\begin{majorILnc}{\LnpDC{Definition of Formula of GQL1}} The \nidf{formulas} \underdf{of \GQL{}1}{formulas} are given by the following recursive definition:
\begin{description}
\item[Base Clauses:] \hfill{}
\begin{cenumerate}
\item A sentence letter (atomic sentence of \GSL{}) is a formula.
\item\label{atomic pred} A 1-place predicate followed by one individual constant or variable is a formula.
\end{cenumerate}
\item[Generating Clauses:] \hfill{}
\begin{cenumerate}
\item If $\CAPPHI$ is a formula, then so is $\negation{\CAPPHI}$.
\item If $\CAPPHI$ and $\CAPTHETA$ are formulas, then so are $\parhorseshoe{\CAPPHI}{\CAPTHETA}$ and $\partriplebar{\CAPPHI}{\CAPTHETA}$.
\item\label{GQL conj disj} If all of $\CAPPHI_1,\CAPPHI_2,\CAPPHI_3,\CAPPHI_4,\ldots,\CAPPHI_{\integer{n}}$ are formulas (the list must include at least two formulas and be finite), then so are $\parconjunction{\CAPPHI_1}{\conjunction{\CAPPHI_2}{\conjunction{\CAPPHI_3}{\conjunction{\CAPPHI_4}{\conjunction{\ldots}{\CAPPHI_{\integer{n}}}}}}}$ and $\pardisjunction{\CAPPHI_1}{\disjunction{\CAPPHI_2}{\disjunction{\CAPPHI_3}{\disjunction{\CAPPHI_4}{\disjunction{\ldots}{\CAPPHI_{\integer{n}}}}}}}$.\footnote{Remember 
that this generating clause is non-standard; most logic books treat conjunction and disjunction as binary (two-place) operations. Our definition is closer to English and avoids unnecessary parentheses.}
\item\label{GQL quant} If $\ALPHA$ is a \GQL{}1 variable and $\CAPPHI$ is a formula that does not contain an expression of the form $\universal{\ALPHA}$ or $\existential{\ALPHA}$, then $\universal{\ALPHA}\CAPPHI$ and $\existential{\ALPHA}\CAPPHI$ are formulas.
\end{cenumerate}
\item[Closure Clause:] A string of symbols is a formula only if it can be generated by the clauses above.
\end{description}
\end{majorILnc}
\noindent{}For example, $\App{'}{\constant{b}}$ is a formula, and so is $\Dpp{'}{\variable{x}_4}$. Each of these formulas is atomic.  Formulas of the form $\universal{\ALPHA}\CAPPHI$ are called \underidf{universal}{formulas} formulas and formulas of the form $\existential{\ALPHA}\CAPPHI$ are called \underidf{existential}{formulas}. 
Here are some additional examples of \GQL{}1 formulas. 
\begin{multicols}{2}
\begin{enumerate}
\item $\universal{\variable{x}}\Jpp{'}{\variable{x}}$ 
\item $\negation{\existential{\variable{y}}\Kpp{'}{\variable{x}}}$ 
%\item $\universal{\variable{y}}\Gppp{''}{\variable{x}}{\variable{y}}$ 
%\item $\existential{\variable{y}}\Gppp{''}{\variable{x}}{\variable{y}}$ 
%\item $\universal{\variable{z}}\Gppp{''}{\variable{x}}{\variable{y}}$ 
\item $\existential{\variable{z}}\Lpp{'_{12}}{\constant{b}}$
\item $\existential{\variable{y}}\negation{\universal{\variable{x}}\Gpp{'}{\variable{y}}}$ 
%\item $\existential{\variable{x}}\universal{\variable{y}}\Gppp{''}{\variable{x}}{\variable{y}}$ 
%\item $\universal{\variable{x}}\universal{\variable{y}}\Gppp{''}{\variable{x}}{\variable{y}}$ 
\item $\universal{\variable{x}}\existential{\variable{y}}\Hpp{'}{\variable{z}}$ 
\item $\parhorseshoe{\universal{\variable{x}}\universal{\variable{z}}\Ppp{'}{\variable{z}_{176}}}{\universal{\variable{y}}\Gpp{'}{\variable{y}}}$ 
%\item $\universal{\variable{x}}\existential{\variable{z}}\Gppp{''}{\variable{x}}{\variable{y}}$
\end{enumerate}
\end{multicols}
\noindent{}Contrarily, $\universal{\variable{x}}\universal{\x}\Gpp{'}{\variable{x}}$ is \emph{not} a formula.  That's because it's of the form $\universal{\variable{x}}\CAPPHI$ where $\CAPPHI$ is a formula that contains the expression $\universal{\variable{x}}$.\footnote{Continuing the practice started in section \ref{use mention comment}, we will not always put symbols, expressions, and sentences of \GQL{}1 that are mentioned (instead of used) in single quotes. 
For example, the tokens of the universal and existential quantifiers in definition \mvref{Bound Variable} should, strictly speaking, be in quotes because they \emph{mention} the symbols. 
Being stringent in the use of single quotes---and overly sensitive to the \distinction{use}{mention} distinction in general---tends to cloud what are relatively clear and straightforward concepts.  Accordingly, we will tend to favor conceptual clarity over notational rigor. 
Wherever it may be helpful, we will provide footnotes with more rigorous and detailed explanation.} Neither is $\universal{\constant{a}}\Gp{'}\constant{a}$, because $\forall$ \emph{must} be paired with a variable, and \mention{$\constant{a}$} is a constant.

Finally, we have unofficial formulas, just as we had unofficial sentences in \GSL{} (compare with def. \pmvref{Unofficial Sentence of GSL}).
\begin{majorILnc}{\LnpDC{Unofficial Formula of GQL}}
A string of symbols is an \nidf{unofficial} formula\index{formulas!unofficial|textbf} \Iff we can obtain it from an official formula by
\begin{cenumerate}
\item deleting outer parentheses,
\item replacing official parentheses ( ) with square brackets [ ] or curly brackets \{ \}, or
\item omitting primes $'$ on a predicate letter.
\end{cenumerate}
\end{majorILnc}
\noindent{}As in \GSL{}, from an unofficial sentence we can unambiguously reconstruct the corresponding official
sentence.
\subsection{Other Properties of Formulas}\label{Other Properties of Formulas1} 
As in section \ref{Other Properties of GSL Sentences} for sentences of \GSL{}, we can define the concepts of subformula, order, main connective, and construction tree for formulas of \GQL{}1. 
\begin{majorILnc}{\LnpDC{GQL Subformulas}}
The \df{subformulas} of a formula are defined parallel to that of an \GSL{} sentence (see definition \pmvref{Subsentences}) with the extra clauses that adding a quantifier adds one new subformula, the whole formula. 
\end{majorILnc}
\noindent{}Importantly, the quantifier phrase is \emph{not} a subformula. 
Thus, $\universal{\variable{x}}\universal{\variable{z}}\Gpp{'}{\variable{x}}$ has three subformulas: $\universal{\variable{x}}\universal{\variable{z}}\Gpp{'}{\variable{x}}$, $\universal{\variable{z}}\Gpp{'}{\variable{x}}$, and $\Gpp{'}{\variable{x}}$. 
$\universal{\variable{x}}$ is not a subformula, and neither is $\universal{\variable{x}}\Gpp{'}{\variable{x}}$.
\begin{majorILnc}{\LnpDC{GQL Order}}
The \df{order} of a formula is defined parallel to that of an \GSL{}
sentence (see def. \pmvref{Order}) with the extra clauses that adding a quantifier adds one to the order of the new formula.
\end{majorILnc}
\begin{majorILnc}{\LnpDC{GQL Main Connective}}
The \nidf{main connective}\index{main connective!of GQL|textbf} for a formula is defined as before (def. \pmvref{GSL Main connective}). 
It is the connective token (or tokens) that occur(s) in the formula but in no proper subformula.
\end{majorILnc}
\begin{majorILnc}{\LnpDC{GQL Construction Tree}}
The \df{construction tree} for a formula is defined as before (def. \pmvref{Construction Tree}) with the obvious extensions. 
As before, the order of a formula is the height of the construction tree's longest branch, measured by counting nodes.  Each node in the tree (including the bottom node) is a subsentence, and the main connective is the connective added at the very bottom of the tree. 
\end{majorILnc}
\begin{majorILnc}{\LnpEC{GQL1SubformulaPropertiesExampleA}}
Consider the (unofficial) formula $\horseshoe{\universal{\variable{x}}\Hp{'\variable{x}}}{\universal{\variable{x}}\Gp{'\variable{x}}}$. 
It's a conditional; i.e., its main connective is the arrow.  The construction tree of the formula is:
\begin{center}
	\begin{tikzpicture}[grow=up]
	\tikzset{level distance=50pt}
	\tikzset{sibling distance=40pt}
	\tikzset{every tree node/.style={align=center,anchor=north}}
	\Tree%http://angasm.org/papers/qtree/    http://www.ling.upenn.edu/advice/latex/qtree/qtreenotes.pdf
	[.{$\horseshoe{\universal{\variable{x}}\Hp{'\variable{x}}}{\universal{\variable{x}}\Gp{'\variable{x}}}$}
	[.{$\Gp{'\variable{x}}$\\ $\universal{\variable{x}}\Gp{'\variable{x}}$} %!{\qsetw{3in}} 
	]
	[.{$\Hp{'\variable{x}}$\\ $\universal{\variable{x}}\Hp{'\variable{x}}$}     
	]
	]%
	%\caption{Example formula tree}
	%\label{fig:ExampleFormulaTree}
	\end{tikzpicture}
\end{center}
We can read off from this construction tree that the order of the formula is 3.  It has five subformulas:
\begin{enumerate}[label=(\arabic*), leftmargin=1.85\parindent,
labelindent=.35\parindent, labelsep=*, itemsep=0pt]%,start=1
\item $\horseshoe{\universal{\variable{x}}\Hp{'\variable{x}}}{\universal{\variable{x}}\Gp{'\variable{x}}}$
\end{enumerate}
\vspace*{-.5cm}
\begin{multicols}{2}
\begin{enumerate}[label=(\arabic*), leftmargin=1.85\parindent,
labelindent=.35\parindent, labelsep=*, itemsep=0pt, start=2]
\item $\universal{\variable{x}}\Hp{'\variable{x}}$
\item $\universal{\variable{x}}\Gp{'\variable{x}}$
\item $\Hp{'\variable{x}}$
\item $\Gp{'\variable{x}}$
\end{enumerate}
\end{multicols}
\end{majorILnc}
\begin{majorILnc}{\LnpEC{GQL1SubformulaPropertiesExampleB}}
Next consider the formula $\universal{\variable{x}}\parhorseshoe{\Hp{'\variable{x}}}{\Gp{'\variable{x}}}$, which many people confuse with the conditional from example \ref{GQL1SubformulaPropertiesExampleA}. 
Consider carefully the differences between the two.
This is a universal formula; i.e., its main connective is the universal quantifier.  The construction tree of the formula is:
\begin{center}
	\begin{tikzpicture}[grow=up]
	\tikzset{level distance=50pt}
	\tikzset{sibling distance=40pt}
	\tikzset{every tree node/.style={align=center,anchor=north}}
	\Tree%http://angasm.org/papers/qtree/    http://www.ling.upenn.edu/advice/latex/qtree/qtreenotes.pdf
	[.{$\horseshoe{\Hp{'\variable{x}}}{\Gp{'\variable{x}}}$\\ $\universal{\variable{x}}\parhorseshoe{\Hp{'\variable{x}}}{\Gp{'\variable{x}}}$} %!{\qsetw{3in}}
	[.{$\Gp{'\variable{x}}$}
	] 
	[.{$\Hp{'\variable{x}}$}
	] 
	]%
	%\caption{Example formula tree}
	%\label{fig:ExampleFormulaTree}
	\end{tikzpicture}
\end{center}
And we can read off from this construction tree that the order of the formula is 3.  It has four subformulas:
\begin{multicols}{2}
\begin{cenumerate}
\item $\universal{\variable{x}}\parhorseshoe{\Hp{'\variable{x}}}{\Gp{'\variable{x}}}$
\item $\parhorseshoe{\Hp{'\variable{x}}}{\Gp{'\variable{x}}}$
\item $\Hp{'\variable{x}}$
\item $\Gp{'\variable{x}}$
\end{cenumerate}
\end{multicols}
\end{majorILnc}
\begin{majorILnc}{\LnpEC{GQLSubformulaPropertiesExampleC}}
For a more complicated example consider $\disjunction{\existential{\variable{x}}\parconjunction{\universal{y}\Epp{'}{\variable{y}}}{\App{'}{\variable{x}}}}{\universal{\variable{z}}\parhorseshoe{\existential{\variable{y}}\Hp{'\constant{a}}}{\Gp{'\variable{x}}}}$.
This is a disjunction; i.e., its main connective is vee, $\VEE$.  The construction tree of the formula is:
\begin{center}
	\begin{tikzpicture}[grow=up]
	\tikzset{level distance=50pt}
	\tikzset{level 1/.style={level distance=65pt}}
	\tikzset{sibling distance=40pt}
	\tikzset{every tree node/.style={align=center,anchor=north}}
	\Tree%http://angasm.org/papers/qtree/    http://www.ling.upenn.edu/advice/latex/qtree/qtreenotes.pdf
	[.{$\disjunction{\existential{\variable{x}}\parconjunction{\universal{y}\Epp{'}{\variable{y}}}{\Ap{'\variable{x}}}}{\universal{\variable{z}}\parhorseshoe{\existential{\variable{y}}\Hp{'\constant{a}}}{\Gp{'\variable{x}}}}$}
	[.{$\horseshoe{\existential{\variable{y}}\Hp{'\constant{a}}}{\Gp{'\variable{x}}}$\\ $\universal{\variable{z}}\parhorseshoe{\existential{\variable{y}}\Hp{'\constant{a}}}{\Gp{'\variable{x}}}$}
	[.{$\text{ }$\\ $\Gp{'\variable{x}}$}
	]    
	[.{$\Hp{'\constant{a}}$\\ $\existential{\variable{y}}\Hp{'\constant{a}}$}
	]
	]
	[.{$\conjunction{\universal{y}\Epp{'}{\variable{y}}}{\Ap{'\variable{x}}}$\\ $\existential{\variable{x}}\parconjunction{\universal{y}\Epp{'}{\variable{y}}}{\Ap{'\variable{x}}}$} %!{\qsetw{3in}}
	[.{$\text{ }$\\ $\Ap{'\variable{x}}$}
	]    
	[.{$\Epp{'}{\variable{y}}$\\ $\universal{y}\Epp{'}{\variable{y}}$}
	] 
	]
	]%
	%\caption{Example formula tree}
	%\label{fig:ExampleFormulaTree}
	\end{tikzpicture}
\end{center}
We can read off from this construction tree that the order of the formula is 5. It has eleven subformulas:
\begin{enumerate}[label=(\arabic*), leftmargin=1.85\parindent,
labelindent=.35\parindent, labelsep=*, itemsep=0pt]%,start=1
\item $\disjunction{\existential{\variable{x}}\parconjunction{\universal{y}\Epp{'}{\variable{y}}}{\Ap{'\variable{x}}}}{\universal{\variable{z}}\parhorseshoe{\existential{\variable{y}}\Hp{'\constant{a}}}{\Gp{'\variable{x}}}}$
\end{enumerate}
\vspace*{-.5cm}
\begin{multicols}{2}
\begin{enumerate}[label=(\arabic*), leftmargin=1.85\parindent,
labelindent=.35\parindent, labelsep=*, itemsep=0pt, start=2]%,start=1
\item $\existential{\variable{x}}\parconjunction{\universal{y}\Epp{'}{\variable{y}}}{\Ap{'\variable{x}}}$
\item $\conjunction{\universal{y}\Epp{'}{\variable{y}}}{\Ap{'\variable{x}}}$
\item $\universal{y}\Epp{'}{\variable{y}}$
\item $\Ap{'\variable{x}}$
\item $\Epp{'}{\variable{y}}$
\item $\universal{\variable{z}}\parhorseshoe{\existential{\variable{y}}\Hp{'\constant{a}}}{\Gp{'\variable{x}}}$
\item $\horseshoe{\existential{\variable{y}}\Hp{'\constant{a}}}{\Gp{'\variable{x}}}$
\item $\existential{\variable{y}}\Hp{'\constant{a}}$
\item $\Gp{'\variable{x}}$
\item $\Hp{'\constant{a}}$
\end{enumerate}
\end{multicols}
\end{majorILnc}

\subsection{Sentences of \GQL{}1}\label{Sentences of GQL1} 
Now that we have defined which strings of basic symbols are formulas of \GQL{}1 we can define which are sentences. 
The definition of a sentence depends on a few supporting definitions. We'll give the supporting definitions afterwards.
\begin{majorILnc}{\LnpDC{GQL1 Sentence}}
A string of \GQL{} symbols is a \nidf{sentence}\index{sentence!of \GQL{}1|textbf} \Iff it is a formula that contains no free variables.
\end{majorILnc}
\begin{majorILnc}{\LnpDC{Free Variable}}
A variable (token) in a formula $\CAPPHI$ is \underdf{free}{variables} \Iff it is not bound.
\end{majorILnc}
\begin{majorILnc}{\LnpDC{Scope Definition}}
	In a formula $\existential{\alpha}\CAPPHI$ or $\universal{\alpha}\CAPPHI$, we say that $\CAPPHI$ is the \df{scope} of the quantifier $\existential{\alpha}$ or $\universal{\alpha}$. 
\end{majorILnc}
\noindent{}This definition applies whether $\existential{\alpha}\CAPPHI$ or $\universal{\alpha}\CAPPHI$ is a stand-alone formula or is instead a proper subformula of some other formula.
\begin{majorILnc}{\LnpDC{Bound Variable}}
A variable token $\alpha$ in a formula $\CAPPHI$ is \underdf{bound}{variables} \Iff either (i) it appears as part of a quantifier expression with that variable, $\existential{\alpha}$ or $\universal{\alpha}$; or (ii) it occurs within the scope of a quantifier expression with that variable, $\existential{\alpha}$ or $\universal{\alpha}$. 
\end{majorILnc}
\noindent{}E.g., in the sentence $\existential{\variable{x}}\parconjunction{\Gp{'\variable{x}}}{\Hp{'\variable{x}}}$, the first token of $\variable{x}$ is bound because it's part of the quantifier expression \mention{$\existential{\variable{x}}$}.  The second and third tokens of $\variable{x}$ are bound because they are within the scope of a quantifier expression with that variable.

Alternatively, we can understand when a variable is bound by thinking of a tree of the sentence. A variable token is bound \Iff a quantifier with the same variable appears below or at the same level as (but still on the same branch as) that token. 
The quantifier that binds a variable is the \emph{first} quantifier that appears below or at the same level as (but still on the same branch as) the variable. 
\begin{majorILnc}{\LnpDC{Atomic Sentence of GQL}}
An \underdf{atomic}{sentence} \nidf{sentence} of \GQL{}1 is an atomic formula of \GQL{}1 which is also a sentence; that is, it is an atomic formula of \GQL{}1 that has no free variable.
\end{majorILnc}
\noindent{}We have unofficial sentences too, just as we have unofficial formulas.
\begin{majorILnc}{\LnpDC{Unofficial Sentence of GQL}}
A string of symbols is an \nidf{unofficial} sentence\index{sentence!unofficial (of \GQL{})|textbf} \Iff it's an unofficial formula that contains no free variables. In other words, we can get an unofficial sentence from an official one by
\begin{cenumerate}
\item deleting outer parentheses,
\item replacing official parentheses ( ) with square brackets [ ] or curly brackets \{ \}, or
\item omitting primes $'$ on the predicate letters.
\end{cenumerate}
\end{majorILnc}
\begin{majorILnc}{\LnpEC{GQLSentenceFreeVariableExampleA}}
Both formulas $\horseshoe{\universal{\variable{x}}\Hp{'\variable{x}}}{\universal{\variable{x}}\Gp{'\variable{x}}}$ and $\universal{\variable{x}}\parhorseshoe{\Hp{'\variable{x}}}{\Gp{'\variable{x}}}$ from examples \ref{GQL1SubformulaPropertiesExampleA} and \ref{GQL1SubformulaPropertiesExampleB} are sentences, because in both formulas all variables are bound.
The following arrows point from each variable to the quantifier that binds it.  (We'll ignore the variables in quantifier expressions; it's obvious which quantifiers they belong to.)
\begin{cenumerate}
\item $\horseshoe{\universal{\variable{x}}\Hp{'\variable{x}}\NextLineRef[black, distance=6, out=-70, in=-50]}{\universal{\variable{x}}\Gp{'\variable{x}}}\NextLineRef[black, distance=6, out=-70, in=-50]$
\item $\universal{\variable{x}}\parhorseshoe{\Hp{'\variable{x}}\NextLineRefJ[black, out=-50, in=-50]}{\Gp{'\variable{x}}\NextLineRefB[black, distance=9, out=205, in=315]}$
\end{cenumerate}

\medskip
\noindent{}To see that these variables are bound by those quantifiers, look at the construction trees of the formulas from examples \ref{GQL1SubformulaPropertiesExampleA} and \ref{GQL1SubformulaPropertiesExampleB}. You will see that in each case the indicated quantifier is the first that appears below, but still on the same branch as, the token variable.
\end{majorILnc}
\begin{majorILnc}{\LnpEC{GQLSentenceFreeVariableExampleB}}
The formula $\disjunction{\existential{\variable{x}}\parconjunction{\universal{y}\Epp{'}{\variable{x}}}{\Ap{'\variable{x}}}}{\universal{\variable{z}}\parhorseshoe{\existential{\variable{y}}\Hp{'\constant{a}}}{\Gp{'\variable{x}}}}$ is not a sentence.
That's because it has a free variable. 
The free variable in the formula is underlined, while arrows point from each bound variable to the quantifier that binds it (again, ignoring variables in quantifier expressions). 

\smallskip
\begin{cenumerate}
\item $\disjunction{\existential{\variable{x}}\parconjunction{\universal{y}\Epp{'}{\variable{x}\NextLineRefC[black, distance=15, out=135, in=45]}}{\Ap{'\variable{x}}\NextLineRefH[black, distance=13, out=295, in=305]}}{\universal{\variable{z}}\parhorseshoe{\existential{\variable{y}}\Hp{'\constant{a}}}{\Gp{'\underline{\variable{x}}}}}$
\end{cenumerate}

\medskip
\noindent{}The two quantifiers on the \CAPS{rhs} of the disjunction are not binding any token variables, besides the ones that appear in the quantifier expressions themselves.
\end{majorILnc}
\begin{majorILnc}{\LnpEC{GQLSentenceFreeVariableExampleC}}
In each of the following three formulas the free variable tokens are underlined, and arrows go from variable tokens to the quantifiers that bind them.

\smallskip
\begin{enumerate}[label=(\arabic*), leftmargin=1.85\parindent,
labelindent=.35\parindent, labelsep=*, itemsep=8pt]
\item $\universal{\variable{x}}\horseshoe{\parconjunction{\Bl}{\existential{\variable{z}}\Kpp{'}{\variable{x}\NextLineRefF[black, distance=17, out=135, in=145]}}}{\existential{\variable{x}}\Np{\variable{x}}\NextLineRefK[black, out=-50, in=-50]}$
\item $\conjunction{\Dl}{\pardisjunction{\existential{\variable{u}}\universal{\variable{w}}\Ppp{'}{\variable{u}\NextLineRefC[black, distance=12, out=135, in=45]}}{\Cpp{'}{\underline{\variable{y}}}}}$
\item $\existential{\variable{z}}\partriplebar{\parconjunction{\Hpp{'}{\underline{\variable{x}}}}{\Gl}}{\Dp{\variable{z}}\NextLineRefI[black, distance=25, out=205, in=315]}$
\end{enumerate}

\medskip
\noindent{}Because (1) has no free variables while (2) and (3) do, (1) is a sentence but (2) and (3) are not. 
\end{majorILnc}

%%%%%%%%%%%%%%%%%%%%%%%%%%%%%%%%%%%%%%%%%%%%%%%%%%
\section{Models}\label{GQL1 Interpretations}
%%%%%%%%%%%%%%%%%%%%%%%%%%%%%%%%%%%%%%%%%%%%%%%%%%

As with \GSL{}, sentences of \GQL{}1 have no inherent semantics.  
But, also as with \GSL{}, we can give models for sentences of \GQL{}1, and these models allow us to investigate entailment for \GQL{}1.  There are many ways to carry out the details of \GQL{}1 semantics (i.e., give models), but each leads to essentially the same results.  They all end up with the same set of logical truths and the same entailment relation.\footnote{See \citealt[49ff]{Hodges2001}, \citeyear{Hodges1997}.}

\emph{Sentences} of \GQL{}1 correspond roughly to the sentences of English, and we need a definition for \mention{model} such that each model gives each \GQL{}1 sentence a determinate truth value.   Formulas that aren't sentences correspond to grammatical sentences of English that don't have a determinate truth value, because they contain pronouns. For example, consider the English sentence \mention{He is the author of Waverley.}  The sentence may be either true or false, depending on who \mention{he} is.  Often we implicitly invoke context in order to determine the referent of a pronoun. Someone reading a biography of the English author Sir Walter Scott may read the above sentence and evaluate it as true.  However, in a conversation in which \mention{he} refers to Aristotle, we'll evaluate the sentence as false.  By itself, however, it can't be evaluated without further specification.  As with pronouns of English, the unbound variables of \GQL{}1 formulas aren't going to have any context-independent \mention{referent} (i.e., assignment).  We don't want formulas of \GQL{}1 that aren't sentences to get determinate truth values from the models of \GQL{}1.  

\subsection{Models in \GQL{}1}\label{Interpretations in GQL1}
All that is required for an \GSL{} model for $\CAPPHI$ is the assignment of a truth value to each sentence letter in $\CAPPHI$. 
\GQL{}1 has predicate letters and constants, which will require a different kinds of assignments. 
Furthermore, \GQL{}1 has quantifiers.  As we noted before, the quantifiers roughly correspond to English words such as \emph{all} or \emph{some}, so each model must specify a set of objects that the quantifiers are about.  In more technical language, a \GQL{}1 model must fix a domain of objects over which we can quantify. 
\begin{majorILnc}{\LnpDC{GQL1 Interpretation}} 
A \df{model} for $\CAPPHI$, $\IntA$, consists of:
\begin{cenumerate}
\item an assignment of a truth value $\TrueB$ or $\FalseB$ to each sentence letter in $\CAPPHI$; 
\item a single, non-empty set $\integer{U}$, called the \df{universe} or \df{domain};
\item an assignment of a subset of $\integer{U}$ to each 1-place predicate in $\CAPPHI$;
\item an assignment of an object from $\integer{U}$ to each individual constant in $\CAPPHI$.\footnote{Recall the discussions of \mention{set}, \mention{subset}, and \mention{element} in section \ref{sets} in chapter \ref{introduction}.}
\end{cenumerate}
\end{majorILnc}
\noindent{}As with \GSL{} models, we use the following notational conventions: 
Given some sentence letter, like $\PP$, $\IntA(\PP)$ is the truth value $\IntA$ assigns to $\PP$.
Given an individual constant, like $\constant{a}$, $\IntA(\constant{a})$ is the object from $\integer{U}$ assigned by $\IntA$ to $\constant{a}$. 
Given a 1-place predicate, like $\Gp{'}$, $\IntA(\Gp{'})$ is the set of objects from $\integer{U}$ that are assigned to $\Gp{'}$ by $\IntA$.%
\footnote{%
We pause here to make two points for those keeping careful score:
\begin{enumerate*}[label=(\arabic*)]
\item For those comfortable with the abstract notion of a function, we can think of models as functions from the set of basic symbols of \GQL{}1 (less the logical operators, variables, and parentheses) to the kinds of objects mentioned in definition \mvref{GQL1 Interpretations} (objects or subsets of $\integer{U}$). 
Thought of in this way, this notation is just the normal notation for functions.
\item\label{pointtwo} Those trying to keep careful track of the \distinction{use}{mention} distinction\index{\distinction{use}{mention} distinction}\index{single quotes} should note that here we've been especially loose. 
Because it's the symbols \mention{$\PP$}, \mention{$\constant{a}$}, \mention{$\Gp{'}$} (etc) themselves that the model maps to some object (a truth value, subset of $\integer{U}$, etc), we really should put the argument of the model (construed as a function) in single quotes. E.g., we should write \mention{$\IntA(\text{`}\PP\text{'})$}, not \mention{$\IntA(\PP)$}, when denoting the object from $\integer{U}$ assigned to the \emph{symbol} \mention{$\PP$} by $\IntA$. 
\end{enumerate*}
\label{Int Footnote}
} 


Earlier we said that the individual constants are roughly similar to proper names in English.  One difference is that, in \GQL{}1, each individual constant in $\CAPPHI$ corresponds to exactly one object in the domain.  In English, on the other hand, some proper names---e.g., \mention{John Smith}---correspond to more than one person, and some---e.g., \mention{Mordecai Alonzo Frazzle III}---do not correspond to any person.  While we require that each constant in $\CAPPHI$ is assigned an object from the domain, we do not require that different constants be assigned different objects. 

We distinguish different models by affixing integers as subscripts to the symbol \mention{$\As{}{}$}.  So, for example, $\As{}{1}$, $\As{}{2}$, $\As{}{3}$, \ldots, $\As{}{316}$, etc., are each different models.

As with \GSL{}, we have \GQL{}1 models for sets of sentences:

\begin{majorILnc}{\LnpDC{Definition of Model for QL1 Set}}
	Given that $\Delta$ is a set of \GQL{}1 sentences, $\IntA$ is a \df{model for $\Delta$} \Iff $\IntA$ is a model for each sentence in $\Delta$.
\end{majorILnc}

There are also models that make assignments to all the sentence letters, constants, and 1-place predicates of \GQL{}1.  Any such model is a model for every \GQL{}1 sentence.  Let's call these \emph{models for \GQL{}1}:

\begin{majorILnc}{\LnpDC{Definition of Model for QL1}}
	$\IntA$ is a \df{model for \GQL{}1} \Iff $\IntA$ is a model for every sentence of \GQL{}1.
\end{majorILnc}

As we define truth in \GQL{}1, we need to make sure every model for $\CAPPHI$ fixes a unique truth value for $\CAPPHI$.\footnote{We define truth so that it applies only to \emph{sentences}, not \emph{formulas}.  Almost all other textbooks define truth (or a related concept) for formulas, and then in turn use that to define truth of a sentence.  We find this an unnecessarily circuitous way of defining truth in \GQL{}1.  Following \citealp{Mates1972}, we instead define truth of a sentence directly, bypassing truth for formulas altogether.} 
But there is a price we must pay for \GQL{}1's superior models.  \GQL{}1 is more complicated than \GSL{}, and will require some additional metalinguistic tools.  Before defining truth for all \GQL{}1 sentences, we must define \mention{terms} and \mention{model variants}.

\begin{majorILnc}{\LnpDC{Terms}}
The \idf{individual terms} of \GQL{}1 are the constants and variables of \GQL{}1.  We will use \mention{$\variable{q}$},\mention{$\variable{r}$}, \mention{$\variable{s}$}, and \mention{$\variable{t}$}  (along with subscripts) as MathEnglish variables for them.
\end{majorILnc}
\noindent{}This means that italic roman \mention{$\variable{q}$}, \mention{$\variable{r}$}, \mention{$\variable{s}$}, \mention{$\variable{t}$}, and the Greek \mention{$\ALPHA$}, \mention{$\BETA$}, etc., can be MathEnglish variables for \GQL{} variables.
However, while \mention{$\ALPHA$}, \mention{$\BETA$}, etc. stand \emph{only} for variables, the italic roman letters can also range over \GQL{}1 constants.  The use of metavariables for individual terms in the object language simplifies our notation considerably.

At this point we could define truth for sentences without quantifiers, though for now we offer only a short sketch.  The sentence $\Gp{\constant{a}}$ is true on $\IntA$ \Iff the object $\IntA$ assigns to \mention{$\constant{a}$} is an element of the set $\IntA$ assigns to \mention{$\GG$}; i.e., \Iff $\IntA(\constant{a})\in\IntA(\GG)$.  If the object $\IntA$ assigns to \mention{$\constant{a}$} isn't a member of the set assigned to \mention{$\GG$}, $\Gp{\constant{a}}$ is false on $\IntA$.

Quantifiers require more complexity. A sentence like $\universal{\variable{x}}\Ep{\variable{x}}$ is true \Iff every object in the domain, $\integer{U}$, is an element of $\IntA(\EE)$. So if $\IntA$ assigns $\integer{U}$ the set of even integers and $\IntA(\EE)=\integer{U}$, $\universal{\variable{x}}\Ep{\variable{x}}$ is true. But if instead $\IntA(\EE)=\{2, 4, 6\}$, then there are objects in the domain (e.g. $8$) not in $\IntA(\EE)$, and so $\universal{\variable{x}}\Ep{\variable{x}}$ is false. For simple quantified sentences this quick definition is good enough; but it won't work for more complex sentences, such as $\universal{\variable{x}}\existential{\variable{y}}\parhorseshoe{\Hp{\variable{y}}}{\Gp{\variable{x}}}$.

To define truth for quantified sentences more precisely, we first need a convenient way to refer to models that make identical assignments everywhere except at one constant.

\begin{majorILnc}{\LnpDC{Variant}}
A $\variable{t}$-\nidf{variant}\index{model!$\variable{t}$-variant} of $\As{}{}$ is any model $\As{*}{}$ that makes any assignment to $\variable{t}$ but otherwise makes exactly the same assignments as $\As{}{}$.\footnote{One result of this definition is that, for every term $\variable{t}$, any model $\As{}{}$ that assigns something to $\variable{t}$ is a $\variable{t}$-variant of itself.}
\end{majorILnc}
\noindent{}Put another way, a $\variable{t}$-variant of $\As{}{}$ makes all the same assignments as $\As{}{}$ except possibly at constant $\variable{t}$. But what if $\As{}{}$ doesn't assign anything to $\variable{t}$? In that case, the $\variable{t}$-variants are all the models that are identical with $\As{}{}$ but that also assign something to $\variable{t}$.

We will generally denote $\variable{t}$-variants of a model $\As{}{}$ by affixing $\variable{t}$ as a superscript to \mention{$\As{}{}$}. For example, $\As{\constant{c}}{}$ is a $\constant{c}$-variant of $\As{}{}$.  On this example $\As{\constant{c}}{}$ and $\As{}{}$ make all the same assignments, except possibly to the constant $\constant{c}$.
We extend this notation when the symbol denoting the original model is itself complex.
So, given an $\constant{c}$-variant of model $\As{}{}$, i.e., $\As{\constant{c}}{}$, $\As{\constant{cd}}{}$ is a $\constant{d}$-variant of $\As{\constant{c}}{}$.  For another example, if $\As{\constant{e}}{4}$ is an $\constant{e}$-variant of $\As{}{4}$, then $\As{\constant{e}}{4}$ and $\As{}{4}$ make identical assignments to everything except maybe $\constant{e}$. 

We need one more piece of notation before we can get to the definition of truth.
\begin{majorILnc}{\LnpDC{MathEnglishVariableSub1}}
	If $\CAPPHI$ is a \GQL{}1 formula and $\variable{t}$ and $\variable{s}$ are terms, then $\CAPPHI\variable{s}/\variable{t}$ is the formula you get by replacing each unbound token of $\variable{t}$ in $\CAPPHI$ with a token of $\variable{s}$.
\end{majorILnc}

\begin{majorILnc}{\LnpEC{MathEnglishVariableSubEx1}}
	\begin{cenumerate}
		\item If $\CAPPHI$ is $\Al$, then $\CAPPHI\variable{y}/\variable{x}$ is $\Al$.
		\item If $\CAPPHI$ is $\Bp{\variable{x}}$, then $\CAPPHI\variable{y}/\variable{x}$ is $\Bp{\variable{y}}$.
		\item If $\CAPPHI$ is $\Bp{\variable{y}}$, then $\CAPPHI\variable{y}/\variable{x}$ is $\Bp{\variable{y}}$.
		\item If $\CAPPHI$ is $\Bp{\variable{x}}$, then $\CAPPHI\variable{y}/\variable{w}$ is $\Bp{\variable{x}}$.
		\item If $\CAPPHI$ is $\universal{x}\Bp{\variable{x}}$, then $\CAPPHI\variable{y}/\variable{x}$ is $\universal{x}\Bp{\variable{x}}$.
		\item If $\CAPPHI$ is $\conjunction{\Cp{\variable{x}}}{\universal{x}\Bp{\variable{x}}}$, then $\CAPPHI\variable{y}/\variable{x}$ is $\conjunction{\Cp{\variable{y}}}{\universal{x}\Bp{\variable{x}}}$.
		\item If $\CAPPHI$ is $\existential{\variable{y}}\parconjunction{\Cp{\variable{x}}}{\universal{x}\Bp{\variable{x}}}$, then $\CAPPHI\variable{y}/\variable{x}$ is $\existential{\variable{y}}\parconjunction{\Cp{\variable{y}}}{\universal{x}\Bp{\variable{x}}}$.
		\item If $\CAPPHI$ is $\existential{\variable{y}}\parconjunction{\Cp{\variable{x}}}{\universal{x}\Bp{\variable{x}}}$, then $\CAPPHI\constant{a}/\variable{x}$ is $\existential{\variable{y}}\parconjunction{\Cp{\constant{a}}}{\universal{x}\Bp{\variable{x}}}$.
		\item If $\CAPPHI$ is $\existential{\variable{y}}\parconjunction{\Cp{\variable{x}}}{\Bp{\variable{x}}}$, then $\CAPPHI\constant{a}/\variable{x}$ is $\existential{\variable{y}}\parconjunction{\Cp{\constant{a}}}{\Bp{\constant{a}}}$.
	\end{cenumerate}
\end{majorILnc}

%\subsection{Truth in a Model: Preliminary Ideas}\label{GQL Truth in an Interpretation Prelims}

\subsection{Truth in a Model}\label{GQL1 Truth in an Interpretation}
We are finally ready to define truth in a model for \GQL{}1.

\begin{majorILnc}{\LnpDC{Truth for GQL1 Sentence}}
The following clauses fix when a \GQL{}1 sentence $\CAPTHETA$ is \nidf{$\True$} (or \nidf{$\False$}) on a model for $\CAPTHETA$, $\IntA$:
\begin{cenumerate}
\item A sentence letter $\CAPPHI$ is $\True$ on $\IntA$ \Iff $\As{}{}$ assigns $\True$ to it, i.e. \Iff $\As{}{}(\CAPPHI)=\TrueB$.
\item An atomic sentence $\Pp{\variable{t}}$ with a 1-place predicate $\PP$ and an individual term $\variable{t}$ is $\True$ on $\IntA$ \Iff what $\IntA$ assigns to the individual term $\variable{t}$ is in the set $\IntA$ assigns to the predicate, i.e. \Iff $\IntA(\variable{t})\in\IntA(\PP)$.
\item A negation $\negation{\CAPPHI}$ is $\True$ on $\IntA$ \Iff the unnegated formula $\CAPPHI$ is $\False$ on $\IntA$.
\item A conjunction $\parconjunction{\CAPPHI_1}{\conjunction{\ldots}{\CAPPHI_{\integer{n}}}}$ is $\True$ on $\IntA$ \Iff all conjuncts $\CAPPHI_1,\ldots,\CAPPHI_{\integer{n}}$ are $\True$ on $\IntA$.
\item A disjunction $\pardisjunction{\CAPPHI_1}{\disjunction{\ldots}{\CAPPHI_{\integer{n}}}}$ is $\True$ on $\IntA$ \Iff at least one disjunct $\CAPPHI_1,\ldots,\CAPPHI_{\integer{n}}$ is $\True$ on $\IntA$.
\item A conditional $\parhorseshoe{\CAPPSI}{\CAPPHI}$ is $\True$ on $\IntA$ \Iff the \CAPS{lhs} $\CAPPSI$ is $\False$ or the \CAPS{rhs} $\CAPPHI$ is $\True$ on $\IntA$, or both.
\item A biconditional $\partriplebar{\CAPPSI}{\CAPPHI}$ is $\True$ on $\IntA$ \Iff both sides, $\CAPPSI$ and $\CAPPHI$, have the same truth value on $\IntA$.
\item\label{GQL1TruthUnvQuant} A universal quantification $\universal{\ALPHA}\CAPPHI$ is $\True$ on $\IntA$ \Iff $\CAPPHI\variable{t}/\ALPHA$ is $\True$ on \emph{all} $\variable{t}$-variants of $\IntA$ (where $\variable{t}$ is the first constant not in $\CAPPHI$).
\item An existential quantification $\existential{\ALPHA}\CAPPHI$ is $\True$ on $\IntA$ \Iff $\CAPPHI\variable{t}/\ALPHA$ is $\True$ on \emph{some} $\variable{t}$-variant of $\IntA$ (where $\variable{t}$ is the first constant not in $\CAPPHI$).
\item A sentence $\CAPPHI$ is $\False$ on $\IntA$ \Iff $\CAPPHI$ is not $\True$ on $\IntA$.
\end{cenumerate}
\end{majorILnc}

The following table has two example models, \emph{Pos Int} and \emph{States}:

\begin{longtable}[c]{ l l l l } %p{2.2in} p{2in}
	\toprule
	&\textbf{Symbol} & \multicolumn{2}{c}{\textbf{Model}} \\ \cmidrule(l){3-4}
	& & \textbf{Pos Int} & \textbf{States} \\
	\midrule 
	\endfirsthead
	\multicolumn{4}{c}{\emph{Continued from Previous Page}}\\
	\toprule
	&\textbf{Symbol} & \multicolumn{2}{c}{\textbf{Model}} \\ \cmidrule(l){3-4}
	& & \textbf{Pos Int} & \textbf{States} \\
	\midrule 
	\endhead
	\bottomrule
	\caption{Example Models}\\[-.15in]
	\multicolumn{4}{c}{\emph{Continued next Page}}\\
	\endfoot
	\bottomrule
	\caption{Example Models}\\%
	\endlastfoot%
	\label{table:Partial Models}%
	%\begin{tabular}{ l l l l } %p{2in} p{2in} %\begin{tabular}{ p{1in} l l } %p{2.2in} p{2in}
	%\toprule
	%&\textbf{Symbol} & \multicolumn{2}{c}{\textbf{Interpretation}} \\ \cmidrule(l){3-4}
	%& & \textbf{Pos Int} & \textbf{States} \\
	%\midrule 
	{Universe:} & & The set of positive integers & The set of current US states \\ \addlinespace[.25cm]
	{Sent. Let.:}& A&$\True$&$\False$\\
	& B&$\True$&$\False$\\
	& C&$\False$&$\True$\\
	& D&$\True$&$\False$\\
	& E&$\True$&$\False$\\
	& G&$\False$&$\True$\\ \addlinespace[.25cm]
	{Constants:}&$\constant{a}$&1&Louisiana\\
	&$\constant{b}$&9&Maine\\
	&$\constant{c}$&72&Georgia\\
	&$\constant{d}$&3&Nebraska\\
	&$\constant{e}$&1&New Mexico\\
	&$\constant{f}$&2&Texas\\ \addlinespace[.25cm]
	{1-place:}&$\Ap{'}$&all pos int&Midwestern\\
	&$\Bp{'}$&empty set&name with $>5$ letters\\
	&$\Cp{'}$&even&Coastal\\
	&$\Dp{'}$&odd&on the Pacific coast\\
	&$\Ep{'}$&prime&\{Ohio\}\\
	&$\Gp{'}$&multiple of 7&\{Ohio,Alabama\}\\ \addlinespace[.25cm]
	%\bottomrule
\end{longtable}

\begin{majorILnc}{\LnpEC{GQL1TruthExamplePA}}
	The sentence $\negation{\Gp{\constant{b}}}$ is true on the model \emph{Pos Int} given in table \mvref{table:Partial Models}. 
\end{majorILnc}
\begin{PROOF}
	The model \emph{Pos Int} has as its domain (or universe) the set of positive integers.  
	Pos Int assigns $9$ to $\constant{b}$, and the set of multiples of $7$ (i.e., $\{7, 14, 21, 28, ...\}$) to $\GG$. 
	
	The number $9$ is not a multiple of $7$, so $\emph{Pos Int}(\constant{b})\notin\emph{Pos Int}(\GG)$.\footnote{As a reminder, $\emph{Pos Int}(\constant{b})\notin\emph{Pos Int}(\GG)$ asserts that what \emph{Pos Int} assigns to $\constant{b}$ is not an element of the set that \emph{Pos Int} assigns to $\GG$.}  To see this, we need only to look at the set assigned to $\GG$ and see if $7$ is a member; it's not.  So, $\Gp{\constant{b}}$ is false on \emph{Pos Int}.  It follows that \emph{Pos Int} makes $\negation{\Gp{\constant{b}}}$ true. 
\end{PROOF}

\begin{majorILnc}{\LnpEC{GQL1TruthExampleA}}
The sentence ${\parhorseshoe{\Gp{\constant{d}}}{\Dp{\constant{e}}}}$ is true on the model \emph{Pos Int} given in table \mvref{table:Partial Models}. 
\end{majorILnc}
\begin{PROOF}
$\parhorseshoe{\Gp{\constant{d}}}{\Dp{\constant{e}}}$ is true on \emph{Pos Int} \Iff the \CAPS{LHS} is false or the \CAPS{RHS} is true.  The \CAPS{LHS}, $\Gp{\constant{d}}$, is true on \emph{Pos Int} \Iff what the model assigns to $\constant{d}$ is a member of the set assigned to $\GG$.  \emph{Pos Int} assigns the number $3$ to $\constant{d}$, but $3$ isn't a multiple of $7$.  So, $\emph{Pos Int}(\constant{d})\notin\emph{Pos Int}(\GG)$, and $\Gp{\constant{d}}$ is false on \emph{Pos Int}.  It follows that $\parhorseshoe{\Gp{\constant{d}}}{\Dp{\constant{e}}}$ is true on \emph{Pos Int}.
\end{PROOF}

\begin{majorILnc}{\LnpEC{GQL1TruthExampleB}}
The sentence $\disjunction{\Ap{\constant{a}}}{\Gp{\constant{c}}}$ is false on the model \emph{States} given in table \mvref{table:Partial Models}.
\end{majorILnc}
\begin{PROOF}
The model \emph{States} assigns Louisiana to $\constant{a}$, Georgia to $\constant{c}$, the set of Midwestern states to $\AA$, and the set $\{$Ohio, Alabama$\}$ to $\GG$.

$\Ap{\constant{a}}$ is true on \emph{States} \Iff what the model assigns to $\constant{a}$ is a member of the set assigned to $\AA$, i.e., $\emph{States}(\constant{a})\in\emph{States}(\AA)$.  But Louisiana isn't in the set of Midwestern states; while it might be unclear exactly which states are genuinely Midwestern, Louisiana definitely doesn't belong.  So $\Ap{\constant{a}}$ is false on \emph{States}.

$\Gp{\constant{c}}$ is true on \emph{States} \Iff what the model assigns to $\constant{c}$ is in the set assigned to $\GG$; i.e., \Iff $\emph{States}(\constant{c})\in\emph{States}(\GG)$.  But \emph{States} assigns $\{$Ohio, Alabama$\}$ to $\GG$.  \emph{States} assigns Georgia to $\constant{c}$, and Georgia isn't in that set.  Thus, \emph{States} makes $\Gp{\constant{c}}$ false.

Both disjuncts are false on \emph{States}, so \emph{States} makes $\disjunction{\Ap{\constant{a}}}{\Gp{\constant{c}}}$ false.
\end{PROOF}
\noindent{}Often, whether a sentence is true depends on the model selected.  Because the above models involve things such as numbers and states, whether the given sentence is true depends on facts about numbers and states. 
When we're trying to compute the truth value of a sentence in a given model, the truth value depends not only on our definitions, but also on facts about the objects in the universe of the model. 

\begin{majorILnc}{\LnpEC{GQL1TruthExampleC}}
(i) The sentence $\horseshoe{\universal{\variable{x}}\Dp{\variable{x}}}{\universal{\variable{x}}\Gp{\variable{x}}}$ is true on the model \emph{Pos Int}, given in table \mvref{table:Partial Models}, while (ii) the sentence $\universal{\variable{x}}\parhorseshoe{\Dp{\variable{x}}}{\Gp{\variable{x}}}$ is false on that same model.
\end{majorILnc}
\begin{PROOF}
These two sentences look deceptively similar, but they have different truth values on the model \emph{Pos Int}.  

(i) $\horseshoe{\universal{\variable{x}}\Dp{\variable{x}}}{\universal{\variable{x}}\Gp{\variable{x}}}$ is true on \emph{Pos Int} \Iff \emph{Pos Int} either makes the \CAPS{lhs} is false or the \CAPS{RHS} true.

$\universal{\variable{x}}\Dp{\variable{x}}$ is true on \emph{Pos Int} \Iff for the first constant not in the sentence, $\constant{t}$, every $\constant{t}$-variant of \emph{Pos Int} makes $\Dp{\constant{t}}$ true.\footnote{Remember that $\variable{t}$ isn't itself a constant---it's a metavariable that stands for a constant in the object language, \GQL{}1.}  We \emph{must} use \mention{$\constant{a}$} as our constant; there are no constants in $\Dp{\variable{x}}$, so we pick the first one available. Thus, $\universal{\variable{x}}\Dp{\variable{x}}$ is true on \emph{Pos Int} \Iff every object in the domain of \emph{Pos Int} that could be assigned to $\constant{a}$ makes $\Dp{\constant{a}}$ come out true.

\emph{Pos Int} assigns the set of odd numbers to $\DD$.  But what happens when we consider the $\constant{a}$-variant of \emph{Pos Int} such that $\emph{Pos Int}^{\constant{a}}(\constant{a})=2$?  The number $2$ isn't in the set of odd numbers; i.e., $\emph{Pos Int}^{\constant{a}}(\constant{a})\notin\emph{Pos Int}^{\constant{a}}(\DD)$.  So, there is a $\emph{Pos Int}^{\constant{a}}$ such that $\Dp{\constant{a}}$ is false.  That means, in turn, that \emph{Pos Int} makes $\universal{\variable{x}}\Dp{\variable{x}}$ false.  Thus, $\horseshoe{\universal{\variable{x}}\Dp{\variable{x}}}{\universal{\variable{x}}\Gp{\variable{x}}}$ is true on \emph{Pos Int}.

(ii) $\universal{\variable{x}}\parhorseshoe{\Dp{\variable{x}}}{\Gp{\variable{x}}}$ is true on \emph{Pos Int} \Iff every $\constant{a}$-variant of \emph{Pos Int} makes $\horseshoe{\Dp{\constant{a}}}{\Gp{\constant{a}}}$ true.  

Is there any $\constant{a}$-variant of \emph{Pos Int} that makes $\Dp{\constant{a}}$ true and $\Gp{\constant{a}}$ false?  Yes.  \emph{Pos Int} assigns the set of odd numbers to $\DD$ and the set of multiples of $7$ to $\GG$.  Let $\emph{Pos Int}^{\constant{a}}(\constant{a})=3$.  The number $3$ is odd, but it isn't a multiple of $7$.  So, $\emph{Pos Int}^{\constant{a}}(\constant{a})\in\emph{Pos Int}^{\constant{a}}(\DD)$ and $\emph{Pos Int}^{\constant{a}}(\constant{a})\notin\emph{Pos Int}^{\constant{a}}(\GG)$.  Thus, $\emph{Pos Int}^{\constant{a}}$ makes $\Dp{\constant{a}}$ true and $\Gp{\constant{a}}$ false, which in turn makes $\horseshoe{\Dp{\constant{a}}}{\Gp{\constant{a}}}$ false.  Because there is a $\constant{a}$-variant of \emph{Pos Int} that makes $\horseshoe{\Dp{\constant{a}}}{\Gp{\constant{a}}}$ false, $\universal{\variable{x}}\parhorseshoe{\Dp{\variable{x}}}{\Gp{\variable{x}}}$ is false on \emph{Pos Int}.
\end{PROOF}


\begin{majorILnc}{\LnpEC{GQLTruthExampleD}}
	On the model \emph{Pos Int}, given in table \mvref{table:Partial Models}, (i) the sentence $\existential{\variable{x}}\parconjunction{\Cp{\variable{x}}}{\Dp{\variable{x}}}$ is false but (ii) $\parconjunction{\existential{\variable{x}}\Cp{\variable{x}}}{\existential{\variable{x}}\Dp{\variable{x}}}$ is true.
\end{majorILnc}
\begin{PROOF}
	As with the last example, these two sentences are deceptively similar.  They have different truth on \emph{Pos Int}, however.
	
(i) First let's consider the sentence $\existential{\variable{x}}\parconjunction{\Cp{\variable{x}}}{\Dp{\variable{x}}}$.  $\existential{\variable{x}}\parconjunction{\Cp{\variable{x}}}{\Dp{\variable{x}}}$ is true on \emph{Pos Int} \Iff there is an $\constant{a}$-variant of \emph{Pos Int} that makes $\conjunction{\Cp{\constant{a}}}{\Dp{\constant{a}}}$ true.  

Is there an $\constant{a}$-variant of $\emph{Pos Int}$ such that $\conjunction{\Cp{\constant{a}}}{\Dp{\constant{a}}}$ is true?  No.  The only way an $\constant{a}$-variant can make $\conjunction{\Cp{\constant{a}}}{\Dp{\constant{a}}}$ true is if it makes both conjuncts true.  But \emph{Pos Int} assigns the even numbers to $\CC$ and the odd numbers to $\DD$.  There is no number that is both even and odd!  It follows that every variant must make either $\Cp{\constant{a}}$ or $\Dp{\constant{a}}$ false.

$\existential{\variable{x}}\parconjunction{\Cp{\variable{x}}}{\Dp{\variable{x}}}$ is false on \emph{Pos Int}.

(ii) $\parconjunction{\existential{\variable{x}}\Cp{\variable{x}}}{\existential{\variable{x}}\Dp{\variable{x}}}$ is true on \emph{Pos Int} \Iff both conjuncts are true on \emph{Pos Int}.

$\existential{\variable{x}}\Cp{\variable{x}}$ is true on \emph{Pos Int} \Iff there is an $\constant{a}$-variant of \emph{Pos Int} that makes $\Cp{\constant{a}}$ true.  Is there any such $\constant{a}$-variant?  Yes; consider the variant $\emph{Pos Int}^{\constant{a}}$ that assigns $4$ to $\constant{a}$.  $\emph{Pos Int}^{\constant{a}}$ assigns the even numbers to $\CC$, just like $\emph{Pos Int}$.  The number $4$ is even, so $\emph{Pos Int}^{\constant{a}}(\constant{a})\in\emph{Pos Int}^{\constant{a}}(\CC)$.  Thus, $\emph{Pos Int}^{\constant{a}}$ makes $\Cp{\constant{a}}$ true, and so \emph{Pos Int} makes $\existential{\variable{x}}\Cp{\variable{x}}$ true.

Similar reasoning holds for the other conjunct.  $\existential{\variable{x}}\Dp{\variable{x}}$ is true on \emph{Pos Int} \Iff there is an $\constant{a}$-variant of \emph{Pos Int} that makes $\Dp{\constant{a}}$ true.  $\emph{Pos Int}^{\constant{a}}$ assigns the odd numbers to $\DD$.  The number $3$ is odd, so when $\emph{Pos Int}^{\constant{a}}(\constant{a})=3$, $\emph{Pos Int}^{\constant{a}}(\constant{a})\in\emph{Pos Int}^{\constant{a}}(\DD)$.  Thus, $\emph{Pos Int}^{\constant{a}}$ makes $\Dp{\constant{a}}$ true. It follows that \emph{Pos Int} makes $\existential{\variable{x}}\Dp{\variable{x}}$ true.
	
\emph{Pos Int} makes $\parconjunction{\existential{\variable{x}}\Cp{\variable{x}}}{\existential{\variable{x}}\Dp{\variable{x}}}$ true.
\end{PROOF}	

\subsection{Minimal Models in \GQL{}1}\label{Minimal Models in GQL1}

We may calculate the truth value of a \GQL{}1 sentence $\CAPPHI$ on some $\IntA$ \Iff $\IntA$ is a model for $\CAPPHI$.  If $\IntA$ \emph{isn't} a model for $\CAPPHI$, then $\CAPPHI$ has no truth value on it.  Be careful!  Even if $\As{}{1}$ is a model for some sentence $\CAPPHI$ and $\As{}{2}$ is a model for some sentence $\CAPPSI$, it doesn't follow that either $\As{}{1}$ or $\As{}{2}$ is a model for, say, $\horseshoe{\CAPPHI}{\CAPPSI}$.

We were able to calculate the truth values of the previous example problems using the two models in table \mvref{table:Partial Models} because none of the sentences contained sentence letters, constants, or predicates that weren't given an assignment in the table.  Those assignments not mentioned, in effect, don't matter for truth evaluation.  The following theorem proves that the unmentioned assignments won't make a difference.

As with \GSL{}, we have \emph{minimal models} in \GQL{}1:

\begin{majorILnc}{\LnpDC{Definition of Minimal QL1 Model}}
	Model $\IntA$ is a \df{minimal model for $\CAPPHI$} \Iff $\IntA$ makes the minimum assignments necessary for $\IntA$ to be a model for $\CAPPHI$; i.e., makes assignments to the universe $\integer{U}$, to each sentence letter, constant, and 1-place predicate in $\CAPPHI$, but to nothing else.
\end{majorILnc}

For an example minimal model, consider the sentence $\parconjunction{\existential{\variable{x}}\Cp{\variable{x}}}{\existential{\variable{x}}\Dp{\variable{x}}}$.  This sentence has two 1-place predicates, $\CC$ and $\DD$, and no sentence letters or constants.  A minimal model $\IntA$ will not make assignments to any sentence letter or constant.  $\IntA$ will assign subsets of $\integer{U}$ to $\CC$ and $\DD$, but won't make an assignment to any other 1-place predicate.

When calculating the value of a sentence $\CAPPHI$ on a model, we only need to worry about the assignments to the symbols in $\CAPPHI$ and the universe.  We can ignore the other assignments.  The following theorem demonstrates this.  (This is the \GQL{}1 version of theorem \ref{thm:localityoftruth} in chapter \ref{sententiallogic}.)

\begin{THEOREM}{\LnpTC{Two Models}} 
	Let $\CAPPHI$ be any \GQL{}1 sentence.  If there are two models for $\CAPPHI$, $\As{}{1}$ and $\As{}{2}$, that have the same domain, $\integer{U}$, and make the same assignments for all the sentence letters, individual constants, and 1-place predicates contained in $\CAPPHI$, then $\CAPPHI$ is true on $\As{}{1}$ \Iff $\CAPPHI$ is true on $\As{}{2}$.
\end{THEOREM}	
\begin{PROOF}
	\begin{description}
		\item[Base Step:]  Let $\CAPTHETA$ be a sentence of order 1.  $\CAPTHETA$ must be either (i) a sentence letter, or (ii), a 1-place predicate followed by a constant.
		
		(i) If $\CAPTHETA$ is a sentence letter, and, as we assumed, $\As{}{1}$ and $\As{}{2}$ make the same assignments for all the sentence letters, then $\CAPTHETA$ is true on $\As{}{1}$ \Iff $\CAPTHETA$ is true on $\As{}{2}$.
		
		(ii) If $\CAPTHETA$ is a 1-place predicate, $\PP$, followed by a constant, $\variable{t}$, and, as we assumed, $\As{}{1}$ and $\As{}{2}$ make the same assignments for all the constants and 1-place predicates, then $\As{}{1}(\PP)=\As{}{2}(\PP)$ and $\As{}{1}(\variable{t})=\As{}{2}(\variable{t})$.  It follows that $\As{}{1}(\variable{t})\in\As{}{1}(\PP)$ \Iff $\As{}{2}(\variable{t})\in\As{}{2}(\PP)$.  Thus, $\CAPTHETA$ is true on $\As{}{1}$ \Iff $\CAPTHETA$ is true on $\As{}{2}$.
		
		Either way, $\CAPTHETA$ is true on $\As{}{1}$ \Iff $\CAPTHETA$ is true on $\As{}{2}$.
		
		\item[Inheritance Step:] 
		\begin{description}
			\item[Recursive Assumption:] Assume that for each \GQL{}1 sentence $\CAPTHETA$ of order $n$, $\CAPTHETA$ is true on $\As{}{1}$ \Iff $\CAPTHETA$ is true on $\As{}{2}$.  Let $\CAPPSI$ be of order $n+1$.
			
			\item[Negation:]  The reasoning here is exactly the same reasoning in the corresponding clause of theorem \ref{thm:localityoftruth} in chapter \ref{sententiallogic}.
			
			\item[Conditional, Biconditional, Disjunction, Conjunction:] Same as above.
			
			\item[Universal Quantification:]  Say that the sentence $\CAPPSI$ is of the form $\universal{\ALPHA}\CAPTHETA$.  Because $\universal{\ALPHA}\CAPTHETA$ is a sentence, it has no unbound variables.  So, dropping the quantifier, $\CAPTHETA$ has a lone free variable: $\ALPHA$.
			
			$\universal{\ALPHA}\CAPTHETA$ is true on $\As{}{1}$ \Iff $\CAPTHETA\variable{t}/\ALPHA$ is true on every $\variable{t}$-variant of $\As{}{1}$, where $\variable{t}$ is the first constant not in $\CAPTHETA$ (definition of truth, $\forall$).  The same holds for $\As{}{2}$.  
			
			Recall that $\As{}{1}$ and $\As{}{2}$ have the same domain, $\integer{U}$.  It follows that for each $\variable{t}$-variant of $\As{}{1}$, there is a corresponding $\variable{t}$-variant of $\As{}{2}$ that assigns the same object from $\integer{U}$ to $\variable{t}$; i.e., for each $\As{\constant{t}}{1}$ there is a $\As{\constant{t}}{2}$ such that $\As{\constant{t}}{1}(\variable{t})=\As{\constant{t}}{2}(\variable{t})$.  By Recursive Assumption (RA), $\universal{\ALPHA}\CAPTHETA$ is of order $n+1$, so $\CAPTHETA\variable{t}/\ALPHA$ is of order $n$.  Also by RA, for each $\variable{t}$-variant of $\As{}{1}$ and its corresponding $\variable{t}$-variant of $\As{}{2}$, $\CAPTHETA\variable{t}/\ALPHA$ is true on $\As{}{1}$ \Iff it's true on $\As{}{2}$.
			
			Either (i) $\CAPTHETA\variable{t}/\ALPHA$ is true on every $\variable{t}$-variant of $\As{}{1}$ or (ii) it isn't.  
			
			(i)  Assume $\CAPTHETA\variable{t}/\ALPHA$ \emph{is} true on every $\variable{t}$-variant of $\As{}{1}$.  It follows that for each $\variable{t}$-variant of $\As{}{1}$ that makes $\CAPTHETA\variable{t}/\ALPHA$ true, the corresponding $\variable{t}$-variant of $\As{}{2}$ also makes $\CAPTHETA\variable{t}/\ALPHA$ true.  Thus, $\CAPTHETA\variable{t}/\ALPHA$ is true on every $\variable{t}$-variant of $\As{}{2}$.  Therefore $\universal{\ALPHA}\CAPTHETA$ is true on both $\As{}{1}$ and $\As{}{2}$.
			
			(ii)  Assume $\CAPTHETA\variable{t}/\ALPHA$ \emph{isn't} true on every $\variable{t}$-variant of $\As{}{1}$.  There must be a $\variable{t}$-variant of $\As{}{1}$ that makes $\CAPTHETA\variable{t}/\ALPHA$ false.  Call it $\As{}{1}\'$.  So, there is some element in $\integer{U}$ such that when it is assigned to $\variable{t}$ by $\As{}{1}\'$, $\CAPTHETA\variable{t}/\ALPHA$ comes out false.  It follows that there is a corresponding $\variable{t}$-variant of $\As{}{2}$ that makes the same assignment to $\variable{t}$, call it $\As{}{2}\'$.  Therefore, $\CAPTHETA\variable{t}/\ALPHA$ is false on $\As{}{2}\'$, and so $\universal{\ALPHA}\CAPTHETA$ is false on both $\As{}{1}$ and $\As{}{2}$.
			
			From (i) and (ii), we may conclude that $\universal{\ALPHA}\CAPTHETA$ is true on $\As{}{1}$ \Iff $\universal{\ALPHA}\CAPTHETA$ is true on $\As{}{2}$.			
			
			\item[Existential Quantification:] Say that $\CAPPSI$ is of the form $\existential{\ALPHA}\CAPTHETA$.  As in the last clause, $\existential{\ALPHA}\CAPTHETA$ is a sentence, so it has no unbound variables.  $\CAPTHETA$ has one free variable, $\ALPHA$.
			
			$\existential{\ALPHA}\CAPTHETA$ is true on $\As{}{1}$ \Iff $\CAPTHETA\variable{t}/\ALPHA$ is true on at least one $\variable{t}$-variant of $\As{}{1}$, where $\variable{t}$ is the first constant not in $\CAPTHETA$ (definition of truth, $\exists$).  The same holds for $\As{}{2}$.
			
			As before, $\As{}{1}$ and $\As{}{2}$ have the same domain.  For each $\variable{t}$-variant of $\As{}{1}$, there is a corresponding $\variable{t}$-variant of $\As{}{2}$ that assigns the same object from $\integer{U}$ to $\variable{t}$. That is, for each $\As{\constant{t}}{1}$ there is a $\As{\constant{t}}{2}$ such that $\As{\constant{t}}{1}(\variable{t})=\As{\constant{t}}{2}(\variable{t})$.  By Recursive Assumption (RA), $\existential{\ALPHA}\CAPTHETA$ is of order $n+1$, so $\CAPTHETA\variable{t}/\ALPHA$ is of order $n$.  Also by RA, for each $\variable{t}$-variant of $\As{}{1}$ and its corresponding $\variable{t}$-variant of $\As{}{2}$, $\CAPTHETA\variable{t}/\ALPHA$ is true on $\As{}{1}$ \Iff it's true on $\As{}{2}$.
			
			Either (i) $\CAPTHETA\variable{t}/\ALPHA$ is true on some $\variable{t}$-variant of $\As{}{1}$ or (ii) it isn't.  
			
			(i)  Assume $\CAPTHETA\variable{t}/\ALPHA$ is true on some $\variable{t}$-variant of $\As{}{1}$.  Call this $\variable{t}$-variant $\As{}{1}\'$.  There is some element in $\integer{U}$ such that when it is assigned to $\variable{t}$ by $\As{}{1}\'$, $\CAPTHETA\variable{t}/\ALPHA$ comes out true.  There is a corresponding $\variable{t}$-variant of $\As{}{2}$ that makes the same assignment to $\variable{t}$; let's call it $\As{}{2}\'$.  So, $\CAPTHETA\variable{t}/\ALPHA$ is true on $\As{}{2}\'$.  Therefore $\existential{\ALPHA}\CAPTHETA$ is true on both $\As{}{1}$ and $\As{}{2}$.
			
			(ii)  Assume $\CAPTHETA\variable{t}/\ALPHA$ is false on \emph{every} $\variable{t}$-variant of $\As{}{1}$.  For each $\variable{t}$-variant of $\As{}{1}$, there is a corresponding $\variable{t}$-variant of $\As{}{2}$ that makes the same assignment from the domain to $\variable{t}$.  Thus, for each $\variable{t}$-variant of $\As{}{1}$ that makes $\CAPTHETA\variable{t}/\ALPHA$ false, the corresponding $\variable{t}$-variant of $\As{}{2}$ also makes $\CAPTHETA\variable{t}/\ALPHA$ false.  Therefore, $\CAPTHETA\variable{t}/\ALPHA$ is false on every $\variable{t}$-variant of $\As{}{2}$, and $\existential{\ALPHA}\CAPTHETA$ is false on both $\As{}{1}$ and $\As{}{2}$.
			
			By (i) and (ii), $\existential{\ALPHA}\CAPTHETA$ is true on $\As{}{1}$ \Iff $\existential{\ALPHA}\CAPTHETA$ is true on $\As{}{2}$.
			
		\end{description}
		\item[Closure Step:] There is no other way to form a \GQL{}1 sentence $\CAPPHI$, so the above clauses are sufficient to show that if $\As{}{1}$ and $\As{}{2}$ have the same domain and make the same assignments, then $\CAPPHI$ is true on $\As{}{1}$ \Iff $\CAPPHI$ is true on $\As{}{2}$.
	\end{description}
\end{PROOF}





\subsection{Logical Truth: QT, QF, \& QC}\label{QT QT QI}
Just as we have a rigorous notion of \emph{logical truth}, \emph{falsity}, and \emph{contingency} for \GSL{} (see section \ref{TFT TFF TFI}), we have analogous notions for \GQL{}1.
\begin{majorILnc}{\LnpDC{QT}}
A sentence $\CAPPHI$ of \GQL{} is \nidf{quantificationally true}\index{truth!quantificational|textbf} (\CAPS{qt})\index{QT|see{truth, quantificational}} iff it is true on every model for $\CAPPHI$.
\end{majorILnc} 
\begin{majorILnc}{\LnpDC{QF}}
A sentence $\CAPPHI$ of \GQL{} is \nidf{quantificationally false}\index{falsehood!quantificational|textbf} (\CAPS{qf})\index{QF|see{falsehood, quantificational}} iff it is false on every model for $\CAPPHI$.
\end{majorILnc} 
\begin{majorILnc}{\LnpDC{QI}}
A sentence $\CAPPHI$ of \GQL{} is \nidf{quantificationally contingent}\index{indeterminate!quantificational|textbf} (\CAPS{qc})\index{QI|see{indeterminate, quantificational}} iff there's at least one model $\As{}{1}$ on which it's true and at least one model $\As{}{2}$ on which it's false.
\end{majorILnc} 
\begin{majorILnc}{\LnpEC{GQL1LogicallTruthExampleA}}
(i) The sentence $\universal{\variable{x}}\pardisjunction{\Bp{\variable{x}}}{\negation{\Bp{\variable{x}}}}$ is \CAPS{qt}, but (ii) $\universal{\variable{x}}\parconjunction{\Bp{\variable{x}}}{\negation{\Bp{\variable{x}}}}$ is \CAPS{qf}.
\end{majorILnc}
\begin{PROOF}
(i) $\universal{\variable{x}}\pardisjunction{\Bp{\variable{x}}}{\negation{\Bp{\variable{x}}}}$ is true on $\IntA$ \Iff every $\constant{a}$-variant of $\IntA$, $\As{\constant{a}}{}$, makes $\disjunction{\Bp{\constant{a}}}{\negation{\Bp{\constant{a}}}}$ true.

One of two things must be true for each $\As{\constant{a}}{}$.  Either what $\As{\constant{a}}{}$ assigns to $\constant{a}$ is in the set it assigns to $\BB$ or it isn't.  I.e., either $\As{\constant{a}}{}(\constant{a})\in\As{\constant{a}}{}(\BB)$ or  $\As{\constant{a}}{}(\constant{a})\notin\As{\constant{a}}{}(\BB)$.  If it is, then $\Bp{\constant{a}}$ is true on $\As{\constant{a}}{}$, and so is $\disjunction{\Bp{\constant{a}}}{\negation{\Bp{\constant{a}}}}$.  If it isn't, then $\negation{\Bp{\constant{a}}}$ is true on $\As{\constant{a}}{}$, and so is $\disjunction{\Bp{\constant{a}}}{\negation{\Bp{\constant{a}}}}$.  So, $\disjunction{\Bp{\constant{a}}}{\negation{\Bp{\constant{a}}}}$ is true on every $\As{\constant{a}}{}$.

Therefore, $\universal{\variable{x}}\pardisjunction{\Bp{\variable{x}}}{\negation{\Bp{\variable{x}}}}$ is true on $\IntA$.  Because we assumed nothing particular about $\IntA$, it makes no difference for our argument what assignments it makes.  Thus what we concluded of $\IntA$---that $\universal{\variable{x}}\pardisjunction{\Bp{\variable{x}}}{\negation{\Bp{\variable{x}}}}$ is true---holds for all models.  Thus, $\universal{\variable{x}}\pardisjunction{\Bp{\variable{x}}}{\negation{\Bp{\variable{x}}}}$ is \CAPS{qt}.

(ii) $\universal{\variable{x}}\parconjunction{\Bp{\variable{x}}}{\negation{\Bp{\variable{x}}}}$ is true on $\IntA$ \Iff every $\constant{a}$-variant of $\IntA$, $\As{\constant{a}}{}$, makes $\conjunction{\Bp{\constant{a}}}{\negation{\Bp{\constant{a}}}}$ true.

As before, either $\As{\constant{a}}{}(\constant{a})\in\As{\constant{a}}{}(\BB)$ or  $\As{\constant{a}}{}(\constant{a})\notin\As{\constant{a}}{}(\BB)$.  If the former, then $\negation{\Bp{\constant{a}}}$ is false on $\As{\constant{a}}{}$, and so is $\conjunction{\Bp{\constant{a}}}{\negation{\Bp{\constant{a}}}}$.  If the latter, then $\Bp{\constant{a}}$ is false on $\As{\constant{a}}{}$, and so is $\conjunction{\Bp{\constant{a}}}{\negation{\Bp{\constant{a}}}}$.  So, $\conjunction{\Bp{\constant{a}}}{\negation{\Bp{\constant{a}}}}$ is false on every $\As{\constant{a}}{}$.

Thus, $\universal{\variable{x}}\parconjunction{\Bp{\variable{x}}}{\negation{\Bp{\variable{x}}}}$ is false on $\IntA$.  We assumed nothing particular about $\IntA$, so its assignments don't matter for our argument.  What we concluded of $\IntA$---that $\universal{\variable{x}}\parconjunction{\Bp{\variable{x}}}{\negation{\Bp{\variable{x}}}}$ is false---holds for all models.  Therefore, $\universal{\variable{x}}\parconjunction{\Bp{\variable{x}}}{\negation{\Bp{\variable{x}}}}$ is \CAPS{qf}.
\end{PROOF}
In the last example, we gave sustained arguments to prove that sentences were either \CAPS{qt} or \CAPS{qf}.  To show that a \GQL{}1 sentence is \CAPS{qc}, we use a different strategy.  A sentence is \CAPS{qc} \Iff it's true on one model and false on another (definition of \CAPS{qc}).  That means all we have to do is provide two appropriate models to demonstrate that a sentence is \CAPS{qc}.

For example, we determined earlier that $\horseshoe{\universal{\variable{x}}\Dp{\variable{x}}}{\universal{\variable{x}}\Gp{\variable{x}}}$ is true on the model \emph{Pos Int}.  If we can find a model in which it is false, we will have shown it is \CAPS{qc}.  We can modify \emph{Pos Int} to make a new falsifying model, \emph{Pos Int$^*$}.  Let \emph{Pos Int$^*$} assign the entire domain to $\DD$ so that it makes $\universal{\variable{x}}\Dp{\variable{x}}$ true.  We can keep the assignment \emph{Pos Int} makes to $\GG$, the multiples of 7.  So, on \emph{Pos Int$^*$}, $\universal{\variable{x}}\Gp{\variable{x}}$ is false.  Thus, $\horseshoe{\universal{\variable{x}}\Dp{\variable{x}}}{\universal{\variable{x}}\Gp{\variable{x}}}$ is false on \emph{Pos Int$^*$}.

The definition of model only requires that predicates be assigned a subset of the domain; the assigned subset can be the entire domain, as in this example, or the empty set.  You will find that using the empty set or the entire domain as assignments is often helpful in  constructing models for a desired outcome.

\begin{majorILnc}{\LnpEC{GQL1LogicallTruthExampleB}}
	The sentence $\disjunction{\universal{\variable{x}}\Bp{\variable{x}}}{\universal{\variable{x}}\negation{\Bp{\variable{x}}}}$ is \CAPS{qc}.
\end{majorILnc}
\begin{PROOF}
Let $\IntA_1$ be a model with a single element universe $\integer{U}=\{1\}$ such that $\IntA_1(\BB)=\{1\}$. 
This makes $\universal{\variable{x}}\Bp{\variable{x}}$ true, which in turn makes $\disjunction{\universal{\variable{x}}\Bp{\variable{x}}}{\universal{\variable{x}}\negation{\Bp{\variable{x}}}}$ true in $\IntA_1$.  Everything in the domain--keeping in mind that \mention{everything} is just the object \mention{1}---is also in the set assigned to $\BB$.  So, when we look at the part of the sentence $\universal{\variable{x}}\Bp{\variable{x}}$ governed by the quantifier and replace the variable with a constant (i.e., $\Bp{\constant{a}}$), then no matter what the model assigns to the constant, the result is true.

Let $\IntA_2$ be a model with a two element universe $\integer{U}=\{1,2\}$ such that $\IntA_2(\BB)=\{1\}$.
Then $\disjunction{\universal{\variable{x}}\Bp{\variable{x}}}{\universal{\variable{x}}\negation{\Bp{\variable{x}}}}$ is false in $\IntA_2$, because both disjuncts are false.  $\universal{\variable{x}}\Bp{\variable{x}}$ is false on $\IntA_2$ because not everything in the domain is in the set assigned to $\BB$; and $\universal{\variable{x}}\negation{\Bp{\variable{x}}}$ is false on $\IntA_2$ because the model doesn't make $\BB$ the empty set.

Thus, the sentence $\disjunction{\universal{\variable{x}}\Bp{\variable{x}}}{\universal{\variable{x}}\negation{\Bp{\variable{x}}}}$ is \CAPS{qc}.
\end{PROOF}
\begin{majorILnc}{\LnpEC{GQL1LogicallTruthExampleC}}
The sentence $\horseshoe{\universal{\variable{x}}\Dp{\variable{x}}}{\negation{\existential{\variable{x}}\negation{\Dp{\variable{x}}}}}$ is \CAPS{qt}.
\end{majorILnc}
\begin{PROOF}
Assume that some model $\IntA$ makes $\horseshoe{\universal{\variable{x}}\Dp{\variable{x}}}{\negation{\existential{\variable{x}}\negation{\Dp{\variable{x}}}}}$ false. $\horseshoe{\universal{\variable{x}}\Dp{\variable{x}}}{\negation{\existential{\variable{x}}\negation{\Dp{\variable{x}}}}}$ is false on $\IntA$ \Iff the \CAPS{lhs} is true and the \CAPS{rhs} is false. Thus, $\universal{\variable{x}}\Dp{\variable{x}}$ is true on $\IntA$ and $\negation{\existential{\variable{x}}\negation{\Dp{\variable{x}}}}$ is false on $\IntA$. 

From the truth of $\universal{\variable{x}}\Dp{\variable{x}}$, it follows that every $\constant{a}$-variant of $\IntA$ makes $\Dp{\constant{a}}$ true.

Because $\IntA$ makes $\negation{\existential{\variable{x}}\negation{\Dp{\variable{x}}}}$ false, it follows that $\existential{\variable{x}}\negation{\Dp{\variable{x}}}$ is true.  Given that $\IntA$ makes $\existential{\variable{x}}\negation{\Dp{\variable{x}}}$ true, there must be some $\constant{a}$-variant, $\As{\constant{a}}{}$, that makes $\negation{\Dp{\constant{a}}}$ true.  On $\As{\constant{a}}{}$, $\Dp{\constant{a}}$ must be false.

But we already concluded that every $\constant{a}$-variant of $\IntA$ makes $\Dp{\constant{a}}$ true!  It's contradictory for $\As{\constant{a}}{}$ to make $\Dp{\constant{a}}$ both true and false.  Our original assumption, that some model makes $\horseshoe{\universal{\variable{x}}\Dp{\variable{x}}}{\negation{\existential{\variable{x}}\negation{\Dp{\variable{x}}}}}$ false, must be wrong.  Therefore, $\horseshoe{\universal{\variable{x}}\Dp{\variable{x}}}{\negation{\existential{\variable{x}}\negation{\Dp{\variable{x}}}}}$ is true on all models, and is thus \CAPS{qt}.
\end{PROOF}


%%%%%%%%%%%%%%%%%%%%%%%%%%%%%%%%%%%%%%%%%%%%%%%%%%
\section{Entailment and other Relations}\label{GQL1 Entailment and other Relations}
%%%%%%%%%%%%%%%%%%%%%%%%%%%%%%%%%%%%%%%%%%%%%%%%%%

We defined the following notions in \GSL{}: entailment, equivalence, contradictory, contrary, subcontrary, and logical independence. 
Now we extend these notions to \GQL{}1 sentences. 
While the definitions for \GSL{} refer to models of \GSL{} sentences, the corresponding definitions for \GQL{}1 refer to models of \GQL{}1 sentences. 
To remind ourselves of this difference, we will talk of \emph{truth functional} entailment, \emph{truth functional} equivalence, etc., for \GSL{} sentences, but talk of \emph{quantificational} entailment, \emph{quantificational} equivalence, etc., for \GQL{}1. 
Nevertheless, we don't want to overstate the differences: the concepts underlying the definitions are exactly the same whether we're working in \GSL{} or \GQL{}1.

\begin{majorILnc}{\LnpDC{GQL1 Definition of Entailment}}
 A set $\Delta$ of \GQL{}1 sentences, possibly empty or infinite, quantificationally {entails} another \GQL{}1 sentence $\CAPTHETA$ \Iff every model for $\Delta$ and $\CAPTHETA$ either makes at least one sentence in $\Delta$ $\False$ or makes $\CAPTHETA$ $\True$; i.e. \Iff every model for $\Delta$ and $\CAPTHETA$ which makes all sentences in $\Delta$ $\True$ also makes $\CAPTHETA$ $\True$.
\end{majorILnc}

\noindent{}As we did with \GSL{} (section \ref{Entailment}), we'll give two narrower consequences of this definition. 

\begin{cenumerate}
\item A finite set of \GQL{}1 sentences $\CAPPHI_1,\ldots,\CAPPHI_{\integer{n}}$ quantificationally {entails} another \GQL{}1 sentence $\CAPTHETA$ \Iff every model for $\CAPPHI_1,\ldots,\CAPPHI_{\integer{n}}$, and $\CAPTHETA$ either makes at least one of $\CAPPHI_1,\ldots,\CAPPHI_{\integer{n}}$ $\False$ or makes $\CAPTHETA$ $\True$; i.e. \Iff every model for $\CAPPHI_1,\ldots,\CAPPHI_{\integer{n}}$, and $\CAPTHETA$ that makes all of $\CAPPHI_1,\ldots,\CAPPHI_{\integer{n}}$ $\True$ also makes $\CAPTHETA$ $\True$.
\item A sentence $\CAPPHI$ of \GQL{} \df{quantificationally entails} another sentence $\CAPTHETA$ of \GQL{}1 \Iff every model for $\CAPPHI$ and $\CAPTHETA$ either makes $\CAPPHI$ $\False$ or makes $\CAPTHETA$ $\True$; i.e. \Iff every model for $\CAPPHI$ and $\CAPTHETA$ that makes $\CAPPHI$ $\True$ also makes $\CAPTHETA$ $\True$.
\end{cenumerate}

\noindent{}Also as before we'll use the double turnstile to represent the entailment relation. 
Thus if $\CAPPHI$ quantificationally entails $\CAPTHETA$, we'll write \mention{$\CAPPHI\sdtstile{}{}\CAPTHETA$}. 
If the finite set of sentences  $\CAPPHI_1,\ldots,\CAPPHI_{\integer{n}}$ quantificationally {entails} $\CAPTHETA$, we'll write \mention{$\CAPPHI_1,\ldots,\CAPPHI_{\integer{n}}\sdtstile{}{}\CAPTHETA$}. 
And, if a set $\Delta$ of \GQL{} sentences quantificationally {entails} $\CAPTHETA$ we'll write \mention{$\Delta\sdtstile{}{}\CAPTHETA$}.

\begin{majorILnc}{\LnpEC{GQL Entailment Example}}
	Show whether the following holds: $\universal{\variable{x}}\Gp{\variable{x}}\sdtstile{}{}\Gp{\constant{a}}$.
\end{majorILnc}
\begin{PROOF}
	Assume a model for $\universal{\variable{x}}\Gp{\variable{x}}$ and $\Gp{\constant{a}}$, $\IntA$, such that $\universal{\variable{x}}\Gp{\variable{x}}$ is true.
	By the definition of truth for $\forall$, it follows that $\Gp{\constant{a}}$ is true on all $\constant{a}$-variants of $\IntA$.  The model $\IntA$ is an $\constant{a}$-variant of itself,\footnote{For any term $\constant{t}$ and for any model $\IntA$, $\IntA$ is a $\constant{t}$-variant of itself!} so $\Gp{\constant{a}}$ is true on $\IntA$.
	
	Our only assumption is that the model $\IntA$ makes the \CAPS{lhs} of the double turnstile true.  It follows that $\Gp{\constant{a}}$ is true, so the entailment holds.
\end{PROOF}
\begin{majorILnc}{\LnpEC{GQL Entailment Example 2}}
	$\universal{\variable{x}}\Gp{\variable{x}}\sdtstile{}{}\Gp{\constant{b}}$
\end{majorILnc}
\begin{PROOF}
	This entailment is slightly harder to prove than the last.  It's a quirk of our definition of truth that makes the last so easy to establish.  By changing the constant in the sentence on the \CAPS{rhs} from \mention{$\constant{a}$} to \mention{$\constant{b}$}, we add a few steps to our proof.
	
	Assume a model $\IntA$ such that $\universal{\variable{x}}\Gp{\variable{x}}$ is true.
	So $\Gp{\constant{a}}$ is true on all $\constant{a}$-variants of $\IntA$.  
	
	There is some $\constant{a}$-variant, $\As{\constant{a}}{}$, that assigns to $\constant{a}$ the exact same object from $\integer{U}$ that $\IntA$ assigns to $\constant{b}$.  (All $\constant{a}$-variants of $\IntA$ have the same domain as $\IntA$ itself.)  So, we know that $\As{\constant{a}}{}(\constant{a})\in\As{\constant{a}}{}(\GG)$.  $\As{\constant{a}}{}$ and $\IntA$ assign the same set to $\GG$, so $\As{\constant{a}}{}(\constant{a})\in\As{}{}(\GG)$.  And because $\As{\constant{a}}{}(\constant{a})$ is the same object as $\IntA(\constant{b})$, it follows that $\As{}{}(\constant{b})\in\As{}{}(\GG)$.  And so $\Gp{\constant{b}}$ is true on $\IntA$.  The entailment holds.
\end{PROOF}

\begin{majorILnc}{\LnpEC{GQL1Entailment}}
	$\universal{\variable{x}}\parhorseshoe{\Cp{\variable{x}}}{\Dp{\variable{x}}}, \Cp{\constant{o}}\sdtstile{}{}\Dp{\constant{o}}$
\end{majorILnc}
\begin{PROOF}
	The entailment holds.  Assume some model $\IntA$ such that $\universal{\variable{x}}\parhorseshoe{\Cp{\variable{x}}}{\Dp{\variable{x}}}$ and $\Cp{\constant{o}}$ are true.  By the definition of truth for $\forall$, it follows that $\parhorseshoe{\Cp{\constant{a}}}{\Dp{\constant{a}}}$ is true on all $\constant{a}$-variants of $\IntA$.  Let's take the object that $\IntA$ assigns to \mention{$\constant{o}$} and name it \mention{Ophelia}.  Now take the $\constant{a}$-variant that assigns Ophelia to \mention{$\constant{a}$} and call it $\As{\constant{a}}{}$.  (We know there is such an assignment because $\IntA$ and $\As{\constant{a}}{}$ have the same universe.)  Thus, $\IntA(\constant{o})=\As{\constant{a}}{}(\constant{a})$.  And $\As{\constant{a}}{}$ is an $\constant{a}$-variant of $\IntA$, so they make all the same assignments to the predicate letters.
	
	The model $\IntA$ makes $\Cp{\constant{o}}$ true, therefore $\IntA(\constant{o})\in\IntA(\Cp{})$.  Because $\As{\constant{a}}{}(\constant{a})=\IntA(\constant{o})$ and $\As{\constant{a}}{}(\Cp{})=\IntA(\Cp{})$, it follows by substitution that $\As{\constant{a}}{}(\constant{a})\in\As{\constant{a}}{}(\Cp{})$.  Hence, $\As{\constant{a}}{}$ makes $\Cp{\constant{a}}$ true.  And because $\parhorseshoe{\Cp{\constant{a}}}{\Dp{\constant{a}}}$ is also true on $\As{\constant{a}}{}$, it follows that $\As{\constant{a}}{}$ makes $\Dp{\constant{a}}$ true.
	
	From this it follows that $\As{\constant{a}}{}(\constant{a})\in\As{\constant{a}}{}(\Dp{})$.  We know that $\As{\constant{a}}{}(\constant{a})=\IntA(\constant{o})$ and $\As{\constant{a}}{}(\Dp{})=\IntA(\Dp{})$, so by substitution we get: $\IntA(\constant{o})\in\IntA(\Dp{})$.  Thus, $\IntA$ makes $\Dp{\constant{o}}$ true.
	
	Any model that makes the LHS true also makes the RHS true.  Therefore, the entailment holds.\footnote{This entailment resembles the argument discussed at the beginning of the chapter: (1) All women are mortal, (2) Ophelia is a woman, therefore (3) Ophelia is mortal.  To see this, interpret $\CC$ as the set of women and $\DD$ as the set of mortals.}
\end{PROOF}	


Assume that we have an entailment that holds when the set $\Delta$ is empty; i.e., $\sdtstile{}{}\CAPPHI$.  When an entailment holds, every model $\IntA$ must either make a sentence on the left false or the sentence on the right true (definition of $\sdtstile{}{}$).  Because, in this case, there are no sentences on the \CAPS{lhs}, every model must make the \CAPS{rhs}, $\CAPPHI$, true.  Therefore, as was the case with \GSL{} in chapter \ref{sententiallogic}, $\sdtstile{}{}\CAPPHI$ \Iff $\CAPPHI$ is \CAPS{qt}.

\begin{majorILnc}{\LnpDC{GQL1 TFE}}
Two \GQL{}1 sentences $\CAPTHETA$ and $\CAPPHI$ are \nidf{quantificationally equivalent}\index{equivalent sentences!quantificational|textbf} \Iff all models for $\CAPTHETA$ and $\CAPPHI$ assign them the same truth value, which is the same as saying they entail each other, i.e. $\CAPTHETA\sdtstile{}{}\CAPPHI$ and $\CAPPHI\sdtstile{}{}\CAPTHETA$.
\end{majorILnc}
\begin{majorILnc}{\LnpDC{GQL1 contradictory}}
Two \GQL{}1 sentences $\CAPTHETA$ and $\CAPPHI$ are \nidf{quantificationally contradictory}\index{contradictory!quantificational|textbf} \Iff all models for $\CAPTHETA$ and $\CAPPHI$ assign them opposite truth values, which is the same as saying that each sentence is equivalent to the negation of the other.
\end{majorILnc}
\begin{majorILnc}{\LnpDC{GQL1 contrary}}
Two \GQL{}1 sentences $\CAPTHETA$ and $\CAPPHI$ are \nidf{quantificationally contrary}\index{contraries!quantificational|textbf} \Iff they cannot both be $\True$ in the same model $\IntA$.
\end{majorILnc}
\begin{majorILnc}{\LnpDC{GQL1 subcontrary}}
Two \GQL{}1 sentences $\CAPTHETA$ and $\CAPPHI$ are \nidf{quantificationally subcontrary}\index{subcontraries!quantificational|textbf} \Iff they cannot both be $\False$ in the same model $\IntA$.
\end{majorILnc}
\begin{majorILnc}{\LnpDC{GQL1 Independent}}
Two \GQL{}1 sentences $\CAPTHETA$ and $\CAPPHI$ are \nidf{quantificationally independent}\index{independent sentences!quantificational|textbf} \Iff none of the above hold (including entailment), i.e. \Iff there are four models:
\begin{cenumerate}
	\item A model in which both $\CAPTHETA$ and $\CAPPHI$ are $\True$; 
	\item A model in which both $\CAPTHETA$ and $\CAPPHI$ are $\False$;
	\item A model in which $\CAPTHETA$ is $\True$ and $\CAPPHI$ is $\False$; and
	\item A model in which $\CAPTHETA$ is $\False$ and $\CAPPHI$ is $\True$.
\end{cenumerate}
\end{majorILnc}

First, a minor note.
In \GQL{}1 we have both formulas and sentences. 
(Remember that all sentences are also formulas, but not all formulas are sentences.) 
Because we do not assess formulas that \emph{aren't} sentences for truth value, none of the definitions above apply to them. 
These definitions only make sense for \GQL{}1 sentences. 

Except for the fact that we're considering sentences of \GQL{}1 instead of \GSL{}, and models for \GQL{}1 sentences instead of models for \GSL{} sentences, these definitions are exactly the same as the corresponding ones for \GSL{}. 
We might say that these definitions have the same \sq{structure}. 
The \emph{ideas} of equivalence, being contradictory, etc., haven't changed, even though the details of the definitions are a little different.

The following four facts from \GSL{} also hold for \GQL{}1 (compare with the examples in section \ref{Other Relations}).
(1) Contradictory sentences are also contrary, but sentences can be contrary without being contradictory: e.g. $\conjunction{\Cl}{\Dl}$ and $\conjunction{\Cl}{\negation{\Dl}}$.
(2) Contradictory sentences are also subcontrary, but sentences can be subcontrary without being contradictory: e.g. $\Dl$ and $\disjunction{\Cl}{\negation{\Dl}}$.
(3) If two sentences are both contrary and subcontrary, they are contradictory.
(4) Any two atomic sentences are independent of each other.
Because every sentence of \GSL{} is also a sentence of \GQL{}1, the examples given here still work. 

Finally, recall from section \ref{Basic Results on Entailment} that in \GSL{} we have the important simple theorem (Thm. \pmvref{Exponentiation of Entailment}) that for all \GSL{} sentences $\CAPPHI$ and $\CAPTHETA$, $\CAPPHI\sdtstile{}{}\CAPTHETA$ \Iff $\sdtstile{}{}\parhorseshoe{\CAPPHI}{\CAPTHETA}$. 
The same theorem holds for \GQL{}1 sentences, and the proof is more or less the same. 
\begin{THEOREM}{\LnpTC{Exponentiation of Entailment GQL} \GQL{}1 Exportation Theorem:} For all \GQL{}1 sentences $\CAPPHI$ and $\CAPTHETA$, $\CAPPHI\sdtstile{}{}\CAPTHETA$ \Iff $\:\sdtstile{}{}\parhorseshoe{\CAPPHI}{\CAPTHETA}$.
\end{THEOREM}
\noindent{}In addition, all the generalizations of this theorem given in the next (Thm. \ref{expo generalizations}) also hold for \GQL{}1 sentences, and again the proofs are more or less the same. We will not explicitly restate this theorem for \GQL{}1.


%%%%%%%%%%%%%%%%%%%%%%%%%%%%%%%%%%%%%%%%%%%%%%%%%%
\section{Exercises}
%%%%%%%%%%%%%%%%%%%%%%%%%%%%%%%%%%%%%%%%%%%%%%%%%%

\notocsubsection{Formulas, Order, and Subformulas}{ex:Formulas, Order, and Subformulas1} Which of the following are \GQL{}1 \emph{formulas}? 
For those that are formulas, what is their order? 
How many subformulas does each have?
\begin{multicols}{2}
\begin{enumerate}
\item {$\universal{\variable{x}}\parhorseshoe{\Hpp{'}{\variable{x}}}{\Gpp{'}{\variable{x}}}$}
\item {$\universal{\variable{x}}\parhorseshoe{\Hpp{'}{\variable{x}}}{\Gpp{''}{\variable{x}}}$}
\item {$\universal{\variable{x}}\parhorseshoe{\Hpp{'}{\variable{x}}}{\Gpp{'_7}{\variable{x}}}$}
\item {$\universal{\variable{x}}\universal{\variable{z}}\parhorseshoe{\Hpp{'}{\variable{x}}}{\Gppp{''}{\variable{x}}{\variable{y}}}$}
\item {$\existential{\variable{y}}\universal{\variable{x}}\parhorseshoe{\Hpp{'}{\variable{x}}}{\Gpp{'}{\variable{x}}}$}
\item {$\universal{\variable{t}}\parhorseshoe{\Hpp{'}{\variable{x}}}{\Gpp{'}{\variable{x}}}$}
\item {$\disjunction{\Hpp{'}{\variable{x}}}{\Gpp{'}{\variable{x}}}$}
\item {$\universal{\variable{x}}\parconjunction{\Hpp{'}{\variable{y}}}{\Gpp{'}{\variable{z}}}$}
\end{enumerate}
\end{multicols}


\begin{longtable}[c]{ l l l l } %p{2.2in} p{2in}
	\toprule
	&\textbf{Symbol} & \multicolumn{2}{c}{\textbf{Model}} \\ \cmidrule(l){3-4}
	& & \textbf{Pos Int} & \textbf{States} \\
	\midrule 
	\endfirsthead
	\multicolumn{4}{c}{\emph{Continued from Previous Page}}\\
	\toprule
	&\textbf{Symbol} & \multicolumn{2}{c}{\textbf{Model}} \\ \cmidrule(l){3-4}
	& & \textbf{Pos Int} & \textbf{States} \\
	\midrule 
	\endhead
	\bottomrule
	\caption{Example Models}\\[-.15in]
	\multicolumn{4}{c}{\emph{Continued next Page}}\\
	\endfoot
	\bottomrule
	\caption{Example Models}\\%
	\endlastfoot%
	\label{table:Partial Models Again}%
	%\begin{tabular}{ l l l l } %p{2in} p{2in} %\begin{tabular}{ p{1in} l l } %p{2.2in} p{2in}
	%\toprule
	%&\textbf{Symbol} & \multicolumn{2}{c}{\textbf{Interpretation}} \\ \cmidrule(l){3-4}
	%& & \textbf{Pos Int} & \textbf{States} \\
	%\midrule 
	{Universe:} & & The set of positive integers & The set of states \\ \addlinespace[.25cm]
	{Sent. Let.:}& A&$\True$&$\False$\\
	& B&$\True$&$\False$\\
	& C&$\False$&$\True$\\
	& D&$\True$&$\False$\\
	& E&$\True$&$\False$\\
	& G&$\False$&$\True$\\ \addlinespace[.25cm]
	{Constants:}&$\constant{a}$&1&Louisiana\\
	&$\constant{b}$&9&Maine\\
	&$\constant{c}$&72&Georgia\\
	&$\constant{d}$&3&Nebraska\\
	&$\constant{e}$&1&New Mexico\\
	&$\constant{f}$&2&Texas\\ \addlinespace[.25cm]
	{1-place:}&$\Ap{'}$&all pos int&Midwestern\\
	&$\Bp{'}$&empty set&name with $>5$ letters\\
	&$\Cp{'}$&even&Coastal\\
	&$\Dp{'}$&odd&on the Pacific coast\\
	&$\Ep{'}$&prime&\{Ohio\}\\
	&$\Gp{'}$&multiple of 7&\{Ohio,Alabama\}\\ \addlinespace[.25cm]
	%\bottomrule
\end{longtable}

\notocsubsection{Truth in a Model}{ex:Truth in an Interpretation1} Give the truth value of each of the following sentences on both of the models found in table \mvref{table:Partial Models Again}. 
\begin{multicols}{2}
\begin{enumerate}
\item $\existential{\variable{x}}\Gp{\variable{x}}$
\item $\negation{\existential{\variable{x}}\Gp{\variable{x}}}$
\item $\existential{\variable{x}}\negation{\Gp{\variable{x}}}$
\item $\universal{\variable{x}}\Gp{\variable{x}}$
\item $\negation{\universal{\variable{x}}\Gp{\variable{x}}}$
\item $\universal{\variable{x}}\negation{\Gp{\variable{x}}}$
\item $\conjunction{\existential{\variable{x}}\Cp{\variable{x}}}{\existential{\variable{x}}\Dp{\variable{x}}}$
\item $\existential{\variable{x}}\parconjunction{\Cp{\variable{x}}}{\Dp{\variable{x}}}$
\item $\negation{\existential{\variable{x}}\parconjunction{\Cp{\variable{x}}}{\Dp{\variable{x}}}}$
\item $\universal{\variable{x}}\parconjunction{\Cp{\variable{x}}}{\Dp{\variable{x}}}$
\item $\universal{\variable{x}}\parhorseshoe{\Cp{\variable{x}}}{\Dp{\variable{x}}}$
\item $\horseshoe{\universal{\variable{x}}\Cp{\variable{x}}}{\universal{\variable{x}}\Dp{\variable{x}}}$
\item $\negation{\universal{\variable{x}}\parhorseshoe{\Cp{\variable{x}}}{\Dp{\variable{x}}}}$
\item $\existential{\variable{x}}\parhorseshoe{\Cp{\variable{x}}}{\Dp{\variable{x}}}$
\end{enumerate}
\end{multicols}

\notocsubsection{Quantificational Truth Problems}{ex:Quantificational Truth Problems} 
For each sentence below, say whether or not it's a quantificational truth. 
If so, prove it. 
If not, give a model $\IntA$ that makes it false.
\begin{multicols}{2}
\begin{enumerate}
\item {$\disjunction{\universal{\variable{y}}\bparhorseshoe{\Ap{\variable{y}}}{\Bp{\variable{y}}}}{\universal{\variable{y}}\bparhorseshoe{\Bp{\variable{y}}}{\Ap{\variable{y}}}}$}
\item {$\disjunction{\existential{\variable{y}}\bparhorseshoe{\Ap{\variable{y}}}{\Bp{\variable{y}}}}{\existential{\variable{y}}\bparhorseshoe{\Bp{\variable{y}}}{\Ap{\variable{y}}}}$}
\item {$\horseshoe{\universal{\variable{y}}\bparhorseshoe{\Ap{\variable{y}}}{\Bp{\variable{y}}}}{\bparhorseshoe{\existential{\variable{y}}\Ap{\variable{y}}}{\existential{\variable{y}}\Bp{\variable{y}}}}$}
\item {$\horseshoe{\existential{\variable{y}}\bparhorseshoe{\Ap{\variable{y}}}{\Bp{\variable{y}}}}{\bparhorseshoe{\existential{\variable{y}}\Ap{\variable{y}}}{\existential{\variable{y}}\Bp{\variable{y}}}}$}
\item {$\horseshoe{\existential{\variable{y}}\bparhorseshoe{\Ap{\variable{y}}}{\Bp{\variable{y}}}}{\bparhorseshoe{\universal{\variable{y}}\Ap{\variable{y}}}{\universal{\variable{y}}\Bp{\variable{y}}}}$}
\item {$\horseshoe{\universal{\variable{y}}\negation{\Ap{\variable{y}}}}{\negation{\existential{\variable{y}}\Ap{\variable{y}}}}$}
\item {$\horseshoe{\negation{\existential{\variable{y}}\Ap{\variable{y}}}}{\universal{\variable{y}}\negation{\Ap{\variable{y}}}}$}
\item {$\horseshoe{\negation{\universal{\variable{y}}\Ap{\variable{y}}}}{\existential{\variable{y}}\negation{\Ap{\variable{y}}}}$}
\end{enumerate}
\end{multicols}
\begin{enumerate}[start=9]
\item {$\horseshoe{\universal{\variable{y}}\bparhorseshoe{\Ap{\variable{y}}}{\Bp{\variable{y}}}}{\bparhorseshoe{\universal{\variable{y}}\Ap{\variable{y}}}{\universal{\variable{y}}\Bp{\variable{y}}}}$}
\item {$\horseshoe{\universal{\variable{z}}\bparhorseshoe{\Ap{\variable{z}}}{\pardisjunction{\Bp{\variable{z}}}{\Cp{\variable{z}}}}}{\cpardisjunction{\universal{\variable{z}}\bparhorseshoe{\Ap{\variable{z}}}{\Bp{\variable{z}}}}{\universal{\variable{z}}\bparhorseshoe{\Ap{\variable{z}}}{\Cp{\variable{z}}}}}$}
\item {$\horseshoe{\universal{\variable{y}}\bparhorseshoe{\Ap{\variable{y}}}{\Bp{\variable{y}}}}{\cparhorseshoe{\universal{\variable{y}}\bparhorseshoe{\Bp{\variable{y}}}{\Cp{\variable{y}}}}{\universal{\variable{y}}\bparhorseshoe{\Ap{\variable{y}}}{\Cp{\variable{y}}}}}$}
\item {$\horseshoe{\universal{\variable{y}}\bparhorseshoe{\Ap{\variable{y}}}{\Bp{\variable{y}}}}{\cparhorseshoe{\universal{\variable{y}}\bparhorseshoe{\Cp{\variable{y}}}{\Bp{\variable{y}}}}{\universal{\variable{y}}\bparhorseshoe{\Ap{\variable{y}}}{\Cp{\variable{y}}}}}$}
\item {$\horseshoe{\universal{\variable{y}}\bparhorseshoe{\Ap{\variable{y}}}{\Bp{\variable{y}}}}{\cparhorseshoe{\existential{\variable{y}}\bparhorseshoe{\Bp{\variable{y}}}{\Cp{\variable{y}}}}{\universal{\variable{y}}\bparhorseshoe{\Ap{\variable{y}}}{\Cp{\variable{y}}}}}$}
\item {$\horseshoe{\universal{\variable{y}}\bparhorseshoe{\Ap{\variable{y}}}{\Bp{\variable{y}}}}{\cparhorseshoe{\existential{\variable{y}}\bparhorseshoe{\Bp{\variable{y}}}{\Cp{\variable{y}}}}{\existential{\variable{y}}\bparhorseshoe{\Ap{\variable{y}}}{\Cp{\variable{y}}}}}$}
\end{enumerate}


\notocsubsection{Entailment Problems for \GQL{}1}{Entailment Problems for GQL1} For each entailment below, either prove that it holds or show that it doesn't hold by giving a model that make the sentences on the \CAPS{lhs} of the turnstile true and the sentence on the \CAPS{rhs} false.
\begin{multicols}{2}
\begin{enumerate}
\item {$\universal{\variable{y}}\parhorseshoe{\Ap{\variable{y}}}{\Bp{\variable{y}}}\text{, }\universal{\variable{y}}\Ap{\variable{y}}\sdtstile{}{}\universal{\variable{y}}\Bp{\variable{y}}$}
\item {$\universal{\variable{y}}\parhorseshoe{\Ap{\variable{y}}}{\Bp{\variable{y}}}\text{, }\existential{\variable{y}}\Ap{\variable{y}}\sdtstile{}{}\existential{\variable{y}}\Bp{\variable{y}}$}
\item {$\existential{\variable{y}}\parhorseshoe{\Ap{\variable{y}}}{\Bp{\variable{y}}}\text{, }\existential{\variable{y}}\Ap{\variable{y}}\sdtstile{}{}\existential{\variable{y}}\Bp{\variable{y}}$}
\item {$\horseshoe{\universal{\variable{y}}\Ap{\variable{y}}}{\universal{\variable{y}}\Bp{\variable{y}}}\sdtstile{}{}\universal{\variable{y}}\parhorseshoe{\Ap{\variable{y}}}{\Bp{\variable{y}}}$}
\item {$\existential{\variable{y}}\pardisjunction{\Ap{\variable{y}}}{\Bp{\variable{y}}}\sdtstile{}{}\disjunction{\existential{\variable{y}}\Ap{\variable{y}}}{\existential{\variable{y}}\Bp{\variable{y}}}$}
\item {$\existential{\variable{y}}\parhorseshoe{\Ap{\variable{y}}}{\Bp{\variable{y}}}\text{, }\universal{\variable{y}}\Ap{\variable{y}}\sdtstile{}{}\universal{\variable{y}}\Bp{\variable{y}}$}
\end{enumerate}
\end{multicols}
\begin{enumerate}[start=7]
\item {$\universal{\variable{z}}\bparhorseshoe{\Ap{\variable{z}}}{\pardisjunction{\Bp{\variable{z}}}{\Cp{\variable{z}}}}\sdtstile{}{}\cpardisjunction{\universal{\variable{z}}\bparhorseshoe{\Ap{\variable{z}}}{\Bp{\variable{z}}}}{\universal{\variable{z}}\bparhorseshoe{\Ap{\variable{z}}}{\Cp{\variable{z}}}}$}
\item {$\universal{\variable{y}}\parhorseshoe{\Ap{\variable{y}}}{\Bp{\variable{y}}}\text{, }\existential{\variable{y}}\parhorseshoe{\Bp{\variable{y}}}{\Cp{\variable{y}}}\sdtstile{}{}\existential{\variable{y}}\parhorseshoe{\Ap{\variable{y}}}{\Cp{\variable{y}}}$}
\end{enumerate}

\notocsubsection{Relations Between \GQL{}1 Sentences}{ex:Relations Between GQL1 Sentences} For each sentence below, say whether it entails, it's entailed by, is equivalent to, contradicts, is contrary to, is subcontrary to, or is independent from each of the other sentences. 
\begin{multicols}{2}
\begin{enumerate}
\item {$\universal{\variable{z}}\parhorseshoe{\Gp{\variable{z}}}{\Dp{\variable{z}}}$}
\item {$\horseshoe{\universal{\variable{z}}\Gp{\variable{z}}}{\universal{\variable{z}}\Dp{\variable{z}}}$}
\item {$\existential{\variable{z}}\parconjunction{\Gp{\variable{z}}}{\negation{\Dp{\variable{z}}}}$}
\item {$\existential{\variable{z}}\parconjunction{\Gp{\variable{z}}}{\Dp{\variable{z}}}$}
\item {$\universal{\variable{z}}\parconjunction{\Gp{\variable{z}}}{\Dp{\variable{z}}}$}
\item {$\existential{\variable{z}}\parhorseshoe{\Gp{\variable{z}}}{\Dp{\variable{z}}}$}
\end{enumerate}
\end{multicols}
\begin{enumerate}[start=7] 
\item {$\universal{\variable{z}}\parhorseshoe{\Gp{\variable{z}}}{\negation{\Dp{\variable{z}}}}$}
\end{enumerate}



%\theendnotes



%%%%%%%%%%%%%%%%%%%%%%%%%%%%%%%%%%%%%%%%%%%%%%%%%%
\chapter{Quantifier Language II}\label{quantifierlogic}
%%%%%%%%%%%%%%%%%%%%%%%%%%%%%%%%%%%%%%%%%%%%%%%%%%
% \AddToShipoutPicture*{\BackgroundPicB}

%%%%%%%%%%%%%%%%%%%%%%%%%%%%%%%%%%%%%%%%%%%%%%%%%%
\section{The Language \GQL{}}
%%%%%%%%%%%%%%%%%%%%%%%%%%%%%%%%%%%%%%%%%%%%%%%%%%

%\setcounter{DefThm}{0}

\subsection{Symbols}\label{Sec:GQLSymbols}
In this chapter we add many-place predicates to \GQL{}1.
The resulting language is \GQL{}, and its development was a significant event in the history of logic.\footnote{%
	The development of \GQL{} goes back to Gottlob Frege \citeyearpar{Frege1879,Frege1891,Frege1893}, O. H. Mitchell \citeyearpar{Mitchell1883} and Charles S. Peirce \citeyearpar{Peirce1883}, with Frege's work being independent of and unknown to the latter two. See \citealp[288]{Church1956} and \citealp[34]{Hodges2001}.  %
 It's probably safe to say that Frege and Peirce/Mitchell developed quantificational logic independently, but the extent to which Peirce and his students (like Mitchell) knew of Frege's work is a matter of debate.
It's clear they at least knew of Frege.
E.g., Ladd-Franklin \citeyearpar{LaddFranklin1883} cites Frege's \citeyearpar{Frege1879} through a review of it by Ernst Schr\"oder. 
See \citep{Dipert1984} for a brief discussion of this history.}
The 2-place predicates correspond roughly to what you get if you take an English sentence and remove two names, leaving blanks, e.g.:

\begin{RESTARTmenumerate}
	\item \mention{Goliath is taller than David} $\Rightarrow$ \mention{\_\_\_\_\_\_ is taller than \_\_\_\_\_\_}
	\item \mention{Juliet Capulet loves Romeo Montague} $\Rightarrow$ \mention{\_\_\_\_\_\_ loves \_\_\_\_\_\_}
\end{RESTARTmenumerate}

\noindent{}We may think of 2-place predicates as representing a 2-place \emph{relation}.
The \mention{taller than} relation holds between two objects when one is taller than the other.
The \mention{loves} relation holds when one person---or object of whatever kind---loves another.
We may understand 3-place predicates in a similar way.
For example:

\begin{menumerate}
	\item \mention{Three is between two and four} $\Rightarrow$ \mention{\_\_\_\_\_\_ is between \_\_\_\_\_\_ and \_\_\_\_\_\_}
\end{menumerate}

For any $n\geq2$, an $n$-place predicate represents an $n$-place relation. 
The introduction of many-place predicates significantly increases the power of our formal language. 
Consider the following argument:

\begin{menumerate}
\item All horses are animals.

Therefore,

\item All horses' tails are animals' tails.
\end{menumerate}

\noindent{}The argument is a good one, but it cannot be expressed as an entailment in either \GSL{} or \GQL{}1. 
\GQL{} can handle such arguments because the \mention{is the tail of} relation can be represented with a 2-place predicate. 

\GQL{} has all the basic symbols of \GQL{}1, plus predicate letters for $n$-placed predicates, for every integer $n$ such that $n\geq2$. 
\begin{majorILnc}{\LnpDC{Symbols of GQL}}
The \df{basic symbols} of \GQL{} are:
\begin{cenumerate}
\item Logical Connectives, Punctuation Symbols, Sentence Letters, Individual Constants, Individual Variables: same as \GQL{}1
\item 1-Place Predicates: $\Ap{'}$, $\Bp{'}$, $\ldots$, $\Tp{'}$, $\Ap{'}_1$, $\Bp{'}_1$, $\ldots$, $\Tp{'}_1$, $\Ap{'}_2$, $\Bp{'}_2$, $\ldots$
\item 2-Place Predicates: $\Ap{''}$, $\Bp{''}$, $\ldots$, $\Tp{''}$, $\Ap{''}_1$, $\Bp{''}_1$, $\ldots$, $\Tp{''}_1$, $\Ap{''}_2$, $\Bp{''}_2$, $\ldots$
\item 3-Place Predicates: $\Ap{'''}$, $\Bp{'''}$, $\ldots$
\item[] \hspace{.5in} . . . and so on for all positive integers.
\end{cenumerate}
\end{majorILnc}

\noindent{}The superscripted prime marks express the arity, or number of places, of the predicate.
The subscripted integers allow for an endless supply of predicates.
For every integer $\integer{n}$, \GQL{} contains an infinite number of $\integer{n}$-place predicates. 

\subsection{Formulas of \GQL{}}\label{Formulas of GQL}
As with \GQL{}1 we must define \GQL{} formulas\index{formulas} before getting to \GQL{} sentences.
The second base clause is the only one that differs from those of the \GQL{}1 definition of formula.
\begin{majorILnc}{\LnpDC{Definition of Formula of GQL}} The \nidf{formulas} \underdf{of \GQL{}}{formulas} are given by the following recursive definition:
\begin{description}
\item[Base Clauses:] \hfill{}
\begin{cenumerate}
\item A sentence letter is a formula.
\item An $\integer{n}$-place predicate followed by $\integer{n}$ tokens of individual constants or variables is a formula.
\end{cenumerate}
\item[Generating Clauses:] \hfill{}
\begin{cenumerate}
\item If $\CAPPHI$ is a formula then so is $\negation{\CAPPHI}$.
\item If $\CAPPHI$ and $\CAPTHETA$ are formulas then so are $\parhorseshoe{\CAPPHI}{\CAPTHETA}$ and $\partriplebar{\CAPPHI}{\CAPTHETA}$.
\item If all of $\CAPPHI_1,\CAPPHI_2,\CAPPHI_3,\CAPPHI_4,\ldots,\CAPPHI_{\integer{n}}$ are formulas (the list must include at least two formulas and be finite) then so are $\parconjunction{\CAPPHI_1}{\conjunction{\CAPPHI_2}{\conjunction{\CAPPHI_3}{\conjunction{\CAPPHI_4}{\conjunction{\ldots}{\CAPPHI_{\integer{n}}}}}}}$ and $\pardisjunction{\CAPPHI_1}{\disjunction{\CAPPHI_2}{\disjunction{\CAPPHI_3}{\disjunction{\CAPPHI_4}{\disjunction{\ldots}{\CAPPHI_{\integer{n}}}}}}}$.
\item If $\CAPPHI$ is a formula and it does not contain an expression of the form $\universal{\ALPHA}$ or $\existential{\ALPHA}$ for some \GQL{} variable $\ALPHA$, then $\universal{\ALPHA}\CAPPHI$ and $\existential{\ALPHA}\CAPPHI$ are formulas.
\end{cenumerate}
\item[Closure Clause:] A string of symbols is a formula \Iff it can be generated by the clauses above.
\end{description}
\end{majorILnc}
\noindent{}Formulas that can be constructed from the base clauses are \emph{atomic}.
$\App{'}{\constant{b}}$ is an atomic formula, as is $\Appp{''}{\variable{x}}{\constant{a}}$. 
$\Bpp{''}{\variable{x}}$ is not a formula because it has a 2-place predicate followed only by one individual variable. 
To determine whether some string is a formula, we must count \emph{tokens} of variables and constants. 
For example, $\Cppp{'''}{\variable{x}}{\variable{x}\variable{x}}$ is a formula because it has a 3-place predicate followed by three variable tokens.

$\Gppp{''}{\variable{x}}{\variable{y}}$ is a formula, so by clause 4 it follows that the following are also formulas.
This list is not exhaustive.
\begin{multicols}{2}
\begin{menumerate}
\item $\universal{\variable{x}}\Gppp{''}{\variable{x}}{\variable{y}}$ 
\item $\existential{\variable{x}}\Gppp{''}{\variable{x}}{\variable{y}}$ 
\item $\existential{\variable{z}}\Gppp{''}{\variable{x}}{\variable{y}}$
\item $\existential{\variable{y}}\universal{\variable{x}}\Gppp{''}{\variable{x}}{\variable{y}}$ 
\item $\universal{\variable{x}}\existential{\variable{y}}\Gppp{''}{\variable{x}}{\variable{y}}$ 
\item $\universal{\variable{x}}\universal{\variable{z}}\Gppp{''}{\variable{x}}{\variable{y}}$ 
\end{menumerate}
\end{multicols}
\noindent{}$\universal{\variable{x}}\universal{\x}\Gppp{''}{\variable{x}}{\variable{y}}$ is \emph{not} a formula, because it's of the form $\universal{\variable{x}}\CAPPHI$ where $\CAPPHI$ is a formula that contains the expression $\universal{\variable{x}}$.

As in \GSL{} and \GQL{}1 there are unofficial formulas.
\begin{majorILnc}{\LnpDC{Unofficial Formula of GQL}}
A string of symbols is an \nidf{unofficial} formula\index{formulas!unofficial|textbf} \Iff we can obtain it from an official formula by
\begin{cenumerate}
\item deleting outer parentheses,
\item replacing official parentheses ( ) with square brackets [ ] or curly brackets \{ \}, or
\item omitting primes $'$ on a predicate letter.
\end{cenumerate}
\end{majorILnc}

\noindent{}A unique official formula can always be reconstructed from an unofficial formula.

\subsection{Other Properties of Formulas}\label{Other Properties of Formulas} 
The concepts of subformula, order, main connective, and construction tree for formulas of \GQL{} are the same as in \GQL{}1.\footnote{See section \ref{Other Properties of Formulas1} of the last chapter.} 
\begin{majorILnc}{\LnpEC{GQLSubformulaPropertiesExampleC}}
Consider the formula $\disjunction{\existential{\variable{x}}\parconjunction{\universal{y}\Eppp{''}{\variable{x}}{\variable{y}}}{\App{'}{\variable{x}}}}{\universal{\variable{z}}\parhorseshoe{\existential{\variable{y}}\Hp{'\constant{a}}}{\Gp{'\variable{x}}}}$.
This is a disjunction; its main connective is vee, $\VEE$.
It has eleven subformulas:
\begin{enumerate}[label=(\arabic*), leftmargin=1.85\parindent,
labelindent=.35\parindent, labelsep=*, itemsep=0pt]%,start=1
\item $\disjunction{\existential{\variable{x}}\parconjunction{\universal{y}\Eppp{''}{\variable{x}}{\variable{y}}}{\Ap{'\variable{x}}}}{\universal{\variable{z}}\parhorseshoe{\existential{\variable{y}}\Hp{'\constant{a}}}{\Gp{'\variable{x}}}}$
\end{enumerate}
\vspace*{-.5cm}
\begin{multicols}{2}
\begin{enumerate}[label=(\arabic*), leftmargin=1.85\parindent,
labelindent=.35\parindent, labelsep=*, itemsep=0pt, start=2]%,start=1
\item $\existential{\variable{x}}\parconjunction{\universal{y}\Eppp{''}{\variable{x}}{\variable{y}}}{\Ap{'\variable{x}}}$
\item $\conjunction{\universal{y}\Eppp{''}{\variable{x}}{\variable{y}}}{\Ap{'\variable{x}}}$
\item $\universal{y}\Eppp{''}{\variable{x}}{\variable{y}}$
\item $\Ap{'\variable{x}}$
\item $\Eppp{''}{\variable{x}}{\variable{y}}$
\item $\universal{\variable{z}}\parhorseshoe{\existential{\variable{y}}\Hp{'\constant{a}}}{\Gp{'\variable{x}}}$
\item $\horseshoe{\existential{\variable{y}}\Hp{'\constant{a}}}{\Gp{'\variable{x}}}$
\item $\existential{\variable{y}}\Hp{'\constant{a}}$
\item $\Gp{'\variable{x}}$
\item $\Hp{'\constant{a}}$
\end{enumerate}
\end{multicols}
The construction tree of the formula is:
\begin{center}
\begin{tikzpicture}[grow=up]
\tikzset{level distance=50pt}
\tikzset{level 1/.style={level distance=65pt}}
\tikzset{sibling distance=40pt}
\tikzset{every tree node/.style={align=center,anchor=north}}
	\Tree%http://angasm.org/papers/qtree/    http://www.ling.upenn.edu/advice/latex/qtree/qtreenotes.pdf
[.{$\disjunction{\existential{\variable{x}}\parconjunction{\universal{y}\Eppp{''}{\variable{x}}{\variable{y}}}{\Ap{'\variable{x}}}}{\universal{\variable{z}}\parhorseshoe{\existential{\variable{y}}\Hp{'\constant{a}}}{\Gp{'\variable{x}}}}$}
  [.{$\universal{\variable{z}}\parhorseshoe{\existential{\variable{y}}\Hp{'\constant{a}}}{\Gp{'\variable{x}}}$}
    [.{$\horseshoe{\existential{\variable{y}}\Hp{'\constant{a}}}{\Gp{'\variable{x}}}$}
       [.{$\text{ }$\\ $\Gp{'\variable{x}}$}
       ]    
       [.{$\existential{\variable{y}}\Hp{'\constant{a}}$}
         [.{$\Hp{'\constant{a}}$}
		 ]
       ]
	]
  ]
  [.{$\existential{\variable{x}}\parconjunction{\universal{y}\Eppp{''}{\variable{x}}{\variable{y}}}{\Ap{'\variable{x}}}$} %!{\qsetw{3in}}
    [.{$\conjunction{\universal{y}\Eppp{''}{\variable{x}}{\variable{y}}}{\Ap{'\variable{x}}}$}
       [.{$\text{ }$\\ $\Ap{'\variable{x}}$}
       ]    
       [.{$\universal{y}\Eppp{''}{\variable{x}}{\variable{y}}$}
	     [.{$\Eppp{''}{\variable{x}}{\variable{y}}$}
		 ]
       ] 
	]
  ]
]%
	%\caption{Example formula tree}
	%\label{fig:ExampleFormulaTree}
\end{tikzpicture}
\end{center}
As you can see from the construction tree, the order of the formula is 5. 
\end{majorILnc}

\subsection{Sentences of \GQL{}}\label{Sentences of GQL} 
Sentences, atomic sentences, and unofficial sentences of \GQL{} are defined exactly as in \GQL{}1.\footnote{See section \ref{Sentences of GQL1} of the last chapter.} 


%%%%%%%%%%%%%%%%%%%%%%%%%%%%%%%%%%%%%%%%%%%%%%%%%%
\section{Models}\label{GQL Interpretations}
%%%%%%%%%%%%%%%%%%%%%%%%%%%%%%%%%%%%%%%%%%%%%%%%%%

\subsection{Models in \GQL{}}\label{Interpretations in GQL}
\GQL{} models are essentially the same as models of \GQL{}1 except that they accommodate many-place predicates.

\begin{majorILnc}{\LnpDC{GQL Interpretation}} 
	A \df{model} for $\CAPPHI$, $\IntA$, consists of:
	\begin{cenumerate}
		\item an assignment of a truth value $\TrueB$ or $\FalseB$ to each sentence letter in $\CAPPHI$; 
		\item a non-empty set $\integer{U}$, called the \df{universe} or \df{domain};
		\item an assignment of an element from $\integer{U}$ to each individual constant in $\CAPPHI$;
		\item an assignment of a subset of $\integer{U}$ to each 1-place predicate in $\CAPPHI$;
		\item an assignment of a set of ordered $\integer{n}$-tuples to each $\integer{n}$-place predicate in $\CAPPHI$.
		The elements in each $\integer{n}$-tuple are members of $\integer{U}$.
	\end{cenumerate}
\end{majorILnc}

\noindent{} Clause (5) is the only new part of the definition.

To illustrate \GQL{} models consider the $\integer{3}$-place predicate $\Bp{'''}$.
Let there be a model $\IntA$ on which $\IntA(U)=\set{1,2,3,4}$ and $\Bp{'''}$ stands for the \mention{between} relation for positive integers.
Then $\IntA(\Bp{'''})$ is a set of $3$-tuples $\langle x, y, z\rangle$ such that $y<x<z$:

\bigskip
\noindent{}$\IntA(\Bp{'''})=\{\langle 2, 1, 3\rangle, \langle 2, 1, 4\rangle, \langle 3, 1, 4\rangle, \langle 3, 2, 4\rangle\}$ 
\bigskip

For a second illustration, consider a model $\IntA_2$ whose domain contains exactly three people: Jack, Jill, and Bill.
Let Jack be taller than Jill and Jill be taller than Bill.
Let $\IntA_2(\Ap{''})$ is the \mention{taller than} relation.
The assignment to $\Ap{''}$ is the following set of ordered pairs:

\bigskip
\noindent{}$\IntA_2(\Ap{''})=\{\langle$Jack, Jill$\rangle, \langle$Jill, Bill$\rangle, \langle$Jack, Bill$\rangle\}$ 
\bigskip

\subsection{Truth in a Model}\label{GQL Truth in an Interpretation}

The definition of truth in a model for \GQL{} is exactly the same as in \GQL{}1 except with an additional clause for many-place predicates.

\begin{majorILnc}{\LnpDC{Truth for GQL Sentence}}
The following clauses fix when a \GQL{} sentence $\CAPTHETA$ is \nidf{$\True$} (or \nidf{$\False$}) on a model for $\CAPTHETA$, $\IntA$:
\begin{cenumerate}
	\item A sentence letter $\CAPPHI$ is $\True$ on $\IntA$ \Iff $\As{}{}(\CAPPHI)=\TrueB$.
	\item An atomic sentence $\Pp{\variable{t}}$ with a 1-place predicate $\PP$ and an individual term $\variable{t}$ is $\True$ on $\IntA$ \Iff $\IntA(\variable{t})\in\IntA(\PP)$.
	\item\label{formtruthatomicn} An atomic sentence $\Pp{\variable{t}_1\ldots\variable{t}_{\integer{n}}}$ with an $\integer{n}$-place predicate $\PP$ is $\True$ on $\IntA$ \Iff $\langle \As{}{}(\variable{t}_1),\As{}{}(\variable{t}_2),\ldots,\As{}{}(\variable{t}_{\integer{n}}) \rangle \in \As{}{}(\PP)$. 
	\item A negation $\negation{\CAPPHI}$ is $\True$ on $\IntA$ \Iff $\CAPPHI$ is $\False$ on $\IntA$.
	\item A conjunction $\parconjunction{\CAPPHI_1}{\conjunction{\ldots}{\CAPPHI_{\integer{n}}}}$ is $\True$ on $\IntA$ \Iff all of $\CAPPHI_1,\ldots,\CAPPHI_{\integer{n}}$ are $\True$ on $\IntA$.
	\item A disjunction $\pardisjunction{\CAPPHI_1}{\disjunction{\ldots}{\CAPPHI_{\integer{n}}}}$ is $\True$ on $\IntA$ \Iff at least one of $\CAPPHI_1,\ldots,\CAPPHI_{\integer{n}}$ is $\True$ on $\IntA$.
	\item A conditional $\parhorseshoe{\CAPPSI}{\CAPPHI}$ is $\True$ on $\IntA$ \Iff the \CAPS{lhs} $\CAPPSI$ is $\False$ or the \CAPS{rhs} $\CAPPHI$ is $\True$ on $\IntA$.
	\item A biconditional $\partriplebar{\CAPPSI}{\CAPPHI}$ is $\True$ on $\IntA$ \Iff $\CAPPSI$ and $\CAPPHI$ have the same truth value on $\IntA$.
	\item\label{GQLTruthUnvQuant} A universal quantification $\universal{\ALPHA}\CAPPHI$ is $\True$ on $\IntA$ \Iff $\CAPPHI\variable{t}/\ALPHA$ is $\True$ on \emph{all} $\variable{t}$-variants of $\IntA$, where $\variable{t}$ is the first constant not in $\CAPPHI$.
	\item An existential quantification $\existential{\ALPHA}\CAPPHI$ is $\True$ on $\IntA$ \Iff $\CAPPHI\variable{t}/\ALPHA$ is $\True$ on \emph{some} $\variable{t}$-variant of $\IntA$, where $\variable{t}$ is the first constant not in $\CAPPHI$.
	\item A sentence $\CAPPHI$ is $\False$ on $\IntA$ \Iff $\CAPPHI$ is not $\True$ on $\IntA$.
\end{cenumerate}
\end{majorILnc}

\noindent{}Clause (3) is the only new one. To see how \GQL{} truth works for a two-place predicate, consider a model $\IntA$ with the following assignments:\\

\noindent{}$\IntA(\constant{j})=$ Jack\\
\noindent{}$\IntA(\constant{i})=$ Jill\\
\noindent{}$\IntA(\constant{b})=$ Bill\\
\noindent{}$\IntA(\Ap{''})=\{\langle$Jack, Jill$\rangle, \langle$Jill, Bill$\rangle, \langle$Jack, Bill$\rangle\}$\\

\begin{majorILnc}{\LnpEC{GQLTruthEasyExampleA}}
Show that the sentence $\Ap{''\constant{j}\constant{b}}$ is true on $\IntA$.
\end{majorILnc}
\begin{commentary}
	To determine the truth value of $\Ap{''\constant{j}\constant{b}}$ we must check whether $\langle\IntA(\constant{j}),\IntA(\constant{b})\rangle$ is a member of the set $\IntA(\Ap{''})$. 
\end{commentary}
\begin{PROOF}
	According to $\IntA$, $\langle\IntA(\constant{j}),\IntA(\constant{b})\rangle=\langle$Jack, Bill$\rangle$. 
	We find that $\langle$Jack, Bill$\rangle\in\IntA(\Ap{''})$, so $\Ap{''\constant{j}\constant{b}}$ is true on $\IntA$.
\end{PROOF}

\begin{majorILnc}{\LnpEC{GQLTruthEasyExampleB}}
Show that the sentence $\Ap{''\constant{b}\constant{i}}$ is false on $\IntA$.
\end{majorILnc}
\begin{PROOF}
	Since $\langle\IntA(\constant{b}),\IntA(\constant{i})\rangle=\langle$Bill, Jill$\rangle$ and $\langle$Bill, Jill$\rangle\notin\IntA(\Ap{''})$, it follows that $\Ap{''\constant{b}\constant{i}}$ is false on $\IntA$.
\end{PROOF}
\begin{commentary}
	Remember that $\langle$Bill, Jill$\rangle$ is not the same as $\langle$Jill, Bill$\rangle$.  Order matters!
\end{commentary}

\noindent{}Let's try more complicated examples using the models provided in figure \mvref{table:Example Interpretations}.

\begin{figure}
\begin{longtable}[c]{ l l l l } %p{2.2in} p{2in}
	\toprule
	&\textbf{Symbol} & \multicolumn{2}{c}{\textbf{Model}} \\ \cmidrule(l){3-4}
	& & \textbf{Pos Int} & \textbf{States} \\
	\midrule 
	\endfirsthead
	\multicolumn{4}{c}{\emph{Continued from Previous Page}}\\
	\toprule
	&\textbf{Symbol} & \multicolumn{2}{c}{\textbf{Model}} \\ \cmidrule(l){3-4}
	& & \textbf{Pos Int} & \textbf{States} \\
	\midrule 
	\endhead
	\bottomrule
	\caption{Example Models}\\[-.15in]
	\multicolumn{4}{c}{\emph{Continued next Page}}\\
	\endfoot
	\bottomrule
	\caption{Example Models}\\%
	\endlastfoot%
	\label{table:Example Interpretations}%
	%\begin{tabular}{ l l l l } %p{2in} p{2in} %\begin{tabular}{ p{1in} l l } %p{2.2in} p{2in}
	%\toprule
	%&\textbf{Symbol} & \multicolumn{2}{c}{\textbf{Interpretation}} \\ \cmidrule(l){3-4}
	%& & \textbf{Pos Int} & \textbf{States} \\
	%\midrule 
	{Universe:} & & The set of positive integers & The set of US states (2024) \\ \addlinespace[.25cm]
	{Sent. Let.:}& A&$\True$&$\False$\\
	& B&$\True$&$\False$\\
	& C&$\False$&$\True$\\
	& D&$\True$&$\False$\\
	& E&$\True$&$\False$\\
	& G&$\False$&$\True$\\ \addlinespace[.25cm]
	{Constants:}&$\constant{a}$&1&Louisiana\\
	&$\constant{b}$&9&Maine\\
	&$\constant{c}$&72&Georgia\\
	&$\constant{d}$&3&Nebraska\\
	&$\constant{e}$&1&New Mexico\\
	&$\constant{f}$&2&Texas\\ \addlinespace[.25cm]
	{1-place:}&$\Ap{'}$&all pos int&Midwestern\\
	&$\Bp{'}$&empty set&name with $>5$ letters\\
	&$\Cp{'}$&even&Coastal\\
	&$\Dp{'}$&odd&on the Pacific Coast\\
	&$\Ep{'}$&prime&\{Ohio\}\\
	&$\Gp{'}$&multiple of 7&\{Ohio, Alabama\}\\ \addlinespace[.25cm]
	{2-place:}&$\Ap{''}$&first $>$ second&share a border\\
	&$\Bp{''}$&are equal&first is north of second\\
	&$\Cp{''}$&first = 2 times second&first $>$ second (area)\\
	&$\Dp{''}$&sum of them equals 7&first $>$ second (population)\\
	&$\Ep{''}$&first $<$ second&first is west of second\\
	&$\Gp{''}$&are relatively prime&both coastal, or neither\\ \addlinespace[.25cm]
	{3-place:}&$\Ap{'''}$&all equal&all same population\\
	&$\Bp{'''}$&first $<$ second $<$ third&first is north of others\\
	&$\Cp{'''}$&all odd or all even&first $>$ second $>$ third (area)\\
	&$\Dp{'''}$&first + second = third&first + second $>$ third (area)\\
	&$\Ep{'''}$&first $\times$ second = third&first is west of the others\\
	&$\Gp{'''}$&are all relatively prime& at least two coastal \\
	%\bottomrule
\end{longtable}
\caption{Two \GQL{} models}
\end{figure}

If you look over the many-place predicates in figure \ref{table:Example Interpretations} you'll notice that we don't list explicit sets of $n$-tuples. 
Instead we provide a brief description of some relation which corresponds to some such set. 
This is our usual practice. 
Writing out all $n$-tuples explicitly would be tedious for the model \emph{States} and impossible for the model \emph{Pos Int}.

\begin{majorILnc}{\LnpEC{GQLTruthExampleA1}}
Determine the truth value of $\universal\variable{y}\existential\variable{x}\App{''\variable{x}}{\variable{y}}$ on model \emph{Pos Int} (figure \ref{table:Example Interpretations}).
\end{majorILnc}
\begin{commentary}
	Intuitively, this sentence can be read as \mention{For each $\variable{y}$, there is some $\variable{x}$ such that $\variable{x}>\variable{y}$.}
	That is, for each positive integer there is another that is greater.
	Thus, we may expect that $\universal\variable{y}\existential\variable{x}\App{''\variable{x}}{\variable{y}}$ is true on \emph{Pos Int}.
	This insight cannot serve as a proof that the sentence is true, but it can guide our efforts as we construct a proof according to the proper definitions.
\end{commentary}
\begin{PROOF}
	$\universal\variable{y}\existential\variable{x}\App{''\variable{x}}{\variable{y}}$ is true on \emph{Pos Int} \Iff $\existential\variable{x}\App{''\variable{x}}{\constant{a}}$ is true on all $\constant{a}$-variants of \emph{Pos Int} (definition of truth, $\forall$). 
	Let there be an $\constant{a}$-variant of \emph{Pos Int}, $\emph{Pos Int}^{\constant{a}}$, such that $\emph{Pos Int}^{\constant{a}}(a)=n$, where $n$ is an arbitrary positive integer.
	%So, to establish the truth of the sentence we can show that all assignments to $\constant{a}$ make $\existential\variable{x}\App{''\variable{x}}{\constant{a}}$ true.
	The sentence $\existential\variable{x}\App{''\variable{x}}{\constant{a}}$ is true on $\emph{Pos Int}^{\constant{a}}$ \Iff $\App{''\constant{b}}{\constant{a}}$ is true on some $\constant{b}$-variant of $\emph{Pos Int}^{\constant{a}}$ (definition of truth, $\exists$).
	%So, to show that an $\constant{a}$-assignment makes $\existential\variable{x}\App{''\variable{x}}{\constant{a}}$ true, we can provide any $\constant{b}$-assignment that makes $\App{''\constant{b}}{\constant{a}}$ true.
	$\emph{Pos Int}(\App{''}{})$ is the set of ordered pairs $\langle x, y \rangle$ such that $x>y$.
	Let $\emph{Pos Int}^{\constant{a}\constant{b}}$ be a $\constant{b}$-variant of $\emph{Pos Int}^{\constant{a}}$ such that $\emph{Pos Int}^{\constant{a}\constant{b}}(b)=n+1$.
	Clearly $\langle n+1, n \rangle\in\emph{Pos Int}(\App{''}{})$.
	Hence, regardless of what $n$ is, $\existential\variable{x}\App{''\variable{x}}{\constant{a}}$ is true on $\emph{Pos Int}^{\constant{a}}$.
	Because nothing about this argument depends upon a specific assignment to $\variable{a}$, it holds for all $\constant{a}$-variants of \emph{Pos Int}. 
	Therefore $\universal\variable{y}\existential\variable{x}\App{''\variable{x}}{\variable{y}}$ is true on \emph{Pos Int}.
\end{PROOF}

\begin{majorILnc}{\LnpEC{GQLTruthExampleA2}}
	Determine the truth value of $\existential\variable{x}\universal\variable{y}\App{''\variable{x}}{\variable{y}}$ on model \emph{Pos Int} (figure \ref{table:Example Interpretations}).
\end{majorILnc}
\begin{commentary}
	Intuitively, this sentence can be read as \mention{There is some $\variable{x}$ such that for all $\variable{y}$, $\variable{x}>\variable{y}$.}
	That is, there is some positive integer that is greater than all others.
	This is false, so we expect that $\existential\variable{x}\universal\variable{y}\App{''\variable{x}}{\variable{y}}$ is false on \emph{Pos Int}.
	But, as before, intuition is no substitute for proof.
\end{commentary}
\begin{PROOF}
	$\existential\variable{x}\universal\variable{y}\App{''\variable{x}}{\variable{y}}$ is true on \emph{Pos Int} \Iff $\universal\variable{y}\App{''\constant{a}}{\variable{y}}$ is true on some $\constant{a}$-variant of \emph{Pos Int}. 
	Let there be an $\constant{a}$-variant of \emph{Pos Int}, $\emph{Pos Int}^{\constant{a}}$, such that $\emph{Pos Int}^{\constant{a}}(a)=n$, where $n$ is an arbitrary positive integer.
	%So, to show that it is false we must find an assignment to $\constant{a}$ such that $\universal\variable{y}\App{''\constant{a}}{\variable{y}}$ is false.
	The sentence $\universal\variable{y}\App{''\constant{a}}{\variable{y}}$ is true on $\emph{Pos Int}^{\constant{a}}$ \Iff $\App{''\constant{a}}{\constant{b}}$ is true on all $\constant{b}$-variants of $\emph{Pos Int}^{\constant{a}}$.
	%So, to show that $\universal\variable{y}\App{''\constant{a}}{\variable{y}}$ is false on some $\constant{a}$-assignment, we need some additional $\constant{b}$-assignment that makes $\App{''\constant{a}}{\constant{b}}$ false.
	$\emph{Pos Int}(\App{''}{})$ is the set of ordered pairs $\langle x, y \rangle$ such that $x>y$.
	Let $\emph{Pos Int}^{\constant{a}\constant{b}}$ be a $\constant{b}$-variant of $\emph{Pos Int}^{\constant{a}}$ such that $\emph{Pos Int}^{\constant{a}\constant{b}}(b)=n+1$.
	Clearly $\langle n, n+1 \rangle\notin\emph{Pos Int}(\App{''}{})$.
	Hence, regardless of what $n$ is, $\App{''\constant{a}}{\constant{b}}$ is false on $\emph{Pos Int}^{\constant{a}\constant{b}}$.
	Thus, $\universal\variable{y}\App{''\constant{a}}{\variable{y}}$ is false on all $\constant{a}$-variants of \emph{Pos Int}. 
	Therefore, $\existential\variable{x}\universal\variable{y}\App{''\variable{x}}{\variable{y}}$ is false on \emph{Pos Int}.
\end{PROOF}

\noindent{}The only difference between $\universal\variable{y}\existential\variable{x}\App{''\variable{x}}{\variable{y}}$ and $\existential\variable{x}\universal\variable{y}\App{''\variable{x}}{\variable{y}}$ is the order of their quantifiers. 
That change is enough to transform the meaning of the sentence completely. 
Quantifier order matters!

\begin{majorILnc}{\LnpEC{GQLTruthExampleB}}
Show that $\universal{\variable{x}}\universal{\variable{y}}\universal{\variable{z}}\parhorseshoe{\parconjunction{\Cpp{\variable{x}}{\variable{y}}}{\Dppp{\variable{x}}{\variable{y}}{\variable{z}}}}{\Bppp{\variable{y}}{\variable{x}}{\variable{z}}}$ is true on the model \emph{Pos Int}. 
\end{majorILnc}
\begin{PROOF}
\raggedright$\universal{\variable{x}}\universal{\variable{y}}\universal{\variable{z}}\parhorseshoe{\parconjunction{\Cpp{\variable{x}}{\variable{y}}}{\Dppp{\variable{x}}{\variable{y}}{\variable{z}}}}{\Bppp{\variable{y}}{\variable{x}}{\variable{z}}}$ is true on \emph{Pos Int} \Iff $\universal{\variable{y}}\universal{\variable{z}}\parhorseshoe{\parconjunction{\Cpp{\constant{a}}{\variable{y}}}{\Dppp{\constant{a}}{\variable{y}}{\variable{z}}}}{\Bppp{\variable{y}}{\constant{a}}{\variable{z}}}$ is true on all $\constant{a}$-variants of \emph{Pos Int}, $\emph{Pos Int}^{\constant{a}}$.
$\universal{\variable{y}}\universal{\variable{z}}\parhorseshoe{\parconjunction{\Cpp{\constant{a}}{\variable{y}}}{\Dppp{\constant{a}}{\variable{y}}{\variable{z}}}}{\Bppp{\variable{y}}{\constant{a}}{\variable{z}}}$ is true on some $\emph{Pos Int}^{\constant{a}}$ \Iff $\universal{\variable{z}}\parhorseshoe{\parconjunction{\Cpp{\constant{a}}{\constant{b}}}{\Dppp{\constant{a}}{\constant{b}}{\variable{z}}}}{\Bppp{\constant{b}}{\constant{a}}{\variable{z}}}$ is true on all $\constant{b}$-variants of $\emph{Pos Int}^{\constant{a}}$, $\emph{Pos Int}^{\constant{a}\constant{b}}$.
$\universal{\variable{z}}\parhorseshoe{\parconjunction{\Cpp{\constant{a}}{\constant{b}}}{\Dppp{\constant{a}}{\constant{b}}{\variable{z}}}}{\Bppp{\constant{b}}{\constant{a}}{\variable{z}}}$ is true on some $\emph{Pos Int}^{\constant{a}\constant{b}}$ \Iff $\parhorseshoe{\parconjunction{\Cpp{\constant{a}}{\constant{b}}}{\Dppp{\constant{a}}{\constant{b}}{\constant{c}}}}{\Bppp{\constant{b}}{\constant{a}}{\constant{c}}}$ is true on all $\constant{c}$-variants of $\emph{Pos Int}^{\constant{a}\constant{b}}$, $\emph{Pos Int}^{\constant{a}\constant{b}\constant{c}}$.
So, if there is some $\emph{Pos Int}^{\constant{a}\constant{b}\constant{c}}$ such that $\parhorseshoe{\parconjunction{\Cpp{\constant{a}}{\constant{b}}}{\Dppp{\constant{a}}{\constant{b}}{\constant{c}}}}{\Bppp{\constant{b}}{\constant{a}}{\constant{c}}}$ is false, then $\universal{\variable{x}}\universal{\variable{y}}\universal{\variable{z}}\parhorseshoe{\parconjunction{\Cpp{\variable{x}}{\variable{y}}}{\Dppp{\variable{x}}{\variable{y}}{\variable{z}}}}{\Bppp{\variable{y}}{\variable{x}}{\variable{z}}}$ is false on \emph{Pos Int}; otherwise, it's true.
Assume for indirect proof that there are assignments to $\constant{a}$, $\constant{b}$, and $\constant{c}$ such that $\Cpp{\constant{a}}{\constant{b}}$ and $\Dppp{\constant{a}}{\constant{b}}{\constant{c}}$ are true and $\Bppp{\constant{b}}{\constant{a}}{\constant{c}}$ is false.
Let $\emph{Pos Int}^{\constant{a}\constant{b}\constant{c}}(b)=n$ for some arbitrary positive integer $n$.
Since $\emph{Pos Int}(\CC)$ is the set of positive integer pairs $\langle \variable{u},\variable{v}\rangle$ such that $\variable{u}=2\variable{v}$, and $\Cpp{\constant{a}}{\constant{b}}$ is true on $\emph{Pos Int}^{\constant{a}\constant{b}\constant{c}}$, it follows that $\emph{Pos Int}^{\constant{a}\constant{b}\constant{c}}(a)=2n$.
And since $\emph{Pos Int}(\DD)$ is the set of positive integer triples $\langle \variable{u},\variable{v},\variable{w}\rangle$ such that $\variable{u}+\variable{v}=\variable{w}$, and $\Dppp{\constant{a}}{\constant{b}}{\constant{c}}$ is true on $\emph{Pos Int}^{\constant{a}\constant{b}\constant{c}}$, it follows that $\emph{Pos Int}^{\constant{a}\constant{b}\constant{c}}(c)=n+2n=3n$.
Since $\emph{Pos Int}(\BB)$ is the set of positive integer triples $\langle \variable{v},\variable{u},\variable{w}\rangle$ such that $\variable{v}<\variable{u}<\variable{w}$, then $\langle \constant{b},\constant{a},\constant{c}\rangle=\langle n, 2n, 3n \rangle$.
And clearly $\langle n, 2n, 3n \rangle\in\emph{Pos Int}(\BB)$.
It follows that $\Bppp{\constant{b}}{\constant{a}}{\constant{c}}$ is true on $\emph{Pos Int}^{\constant{a}\constant{b}\constant{c}}$.
But we had assumed that $\Bppp{\constant{b}}{\constant{a}}{\constant{c}}$ is false on $\emph{Pos Int}^{\constant{a}\constant{b}\constant{c}}$. $\bot$

So there is no $\emph{Pos Int}^{\constant{a}\constant{b}\constant{c}}$ such that $\Cpp{\constant{a}}{\constant{b}}$ and $\Dppp{\constant{a}}{\constant{b}}{\constant{c}}$ are true and $\Bppp{\constant{b}}{\constant{a}}{\constant{c}}$ is false.
Therefore $\universal{\variable{x}}\universal{\variable{y}}\universal{\variable{z}}\parhorseshoe{\parconjunction{\Cpp{\variable{x}}{\variable{y}}}{\Dppp{\variable{x}}{\variable{y}}{\variable{z}}}}{\Bppp{\variable{y}}{\variable{x}}{\variable{z}}}$ is true on \emph{Pos Int}.
\end{PROOF}

\begin{majorILnc}{\LnpEC{GQLTruthExampleB2}}
	Show that $\universal{\variable{x}}\universal{\variable{y}}\universal{\variable{z}}\parhorseshoe{\parconjunction{\Cpp{\variable{x}}{\variable{y}}}{\Dppp{\variable{x}}{\variable{y}}{\variable{z}}}}{\Bppp{\variable{y}}{\variable{x}}{\variable{z}}}$ is false on the model \emph{States}. 
\end{majorILnc}
\begin{PROOF}
	As in the last example, assume for indirect proof that there are assignments to $\constant{a}$, $\constant{b}$, and $\constant{c}$ such that $\Cpp{\constant{a}}{\constant{b}}$ and $\Dppp{\constant{a}}{\constant{b}}{\constant{c}}$ are true and $\Bppp{\constant{b}}{\constant{a}}{\constant{c}}$ is false.
	Let $\emph{States}^{\constant{a}\constant{b}\constant{c}}(a)=\text{Alaska}$, $\emph{States}^{\constant{a}\constant{b}\constant{c}}(b)=\text{Delaware}$, and $\emph{States}^{\constant{a}\constant{b}\constant{c}}(c)=\text{Rhode Island}$.
	Alaska has an area of approximately $1.7\times{}10^6\text{ km}^2$, Delaware an area of approximately $2.5\times{}10^3\text{ km}^2$, and Rhode Island an area of approximately $1.5\times{}10^3\text{ km}^2$.
	Hence $\emph{States}^{\constant{a}\constant{b}\constant{c}}(\constant{a})>\emph{States}^{\constant{a}\constant{b}\constant{c}}(\constant{b})$ (area), $\emph{States}^{\constant{a}\constant{b}\constant{c}}(\constant{a})+\emph{States}^{\constant{a}\constant{b}\constant{c}}(\constant{b})>\emph{States}^{\constant{a}\constant{b}\constant{c}}(\constant{c})$ (area), but $\emph{States}^{\constant{a}\constant{b}\constant{c}}(\constant{b})$ is not north of both $\emph{States}^{\constant{a}\constant{b}\constant{c}}(\constant{a})$ and $\emph{States}^{\constant{a}\constant{b}\constant{c}}(\constant{c})$; that is, Delaware is not north of Alaska. 
	So, by clause (3) of the definition of truth, \ref{Truth for GQL Sentence}, $\Cpp{\constant{a}}{\constant{b}}$ and $\Dppp{\constant{a}}{\constant{b}}{\constant{c}}$ are true on $\emph{States}^{\constant{a}\constant{b}\constant{c}}$, while $\Bppp{\constant{b}}{\constant{a}}{\constant{c}}$ is false on $\emph{States}^{\constant{a}\constant{b}\constant{c}}$. 
	So by the definition of truth ($\WEDGE$ and $\HORSESHOE$), $\parhorseshoe{\parconjunction{\Cpp{\constant{a}}{\constant{b}}}{\Dppp{\constant{a}}{\constant{b}}{\constant{c}}}}{\Bppp{\constant{b}}{\constant{a}}{\constant{c}}}$ is false on $\emph{States}^{\constant{a}\constant{b}\constant{c}}$. 
	
	There is a $\constant{c}$-variant of $\emph{States}^{\constant{a}\constant{b}}$ on which $\parhorseshoe{\parconjunction{\Cpp{\constant{a}}{\constant{b}}}{\Dppp{\constant{a}}{\constant{b}}{\constant{c}}}}{\Bppp{\constant{b}}{\constant{a}}{\constant{c}}}$ is false.  It follows (by the definition of truth for $\forall$) that $\universal{\variable{z}}\parhorseshoe{\parconjunction{\Cpp{\constant{a}}{\constant{b}}}{\Dppp{\constant{a}}{\constant{b}}{\variable{z}}}}{\Bppp{\constant{b}}{\constant{a}}{\variable{z}}}$ is false on $\emph{States}^{\constant{a}\constant{b}}$.  So, there is, in turn, a $\constant{b}$-variant of $\emph{States}^{\constant{a}}$ on which $\universal{\variable{z}}\parhorseshoe{\parconjunction{\Cpp{\constant{a}}{\constant{b}}}{\Dppp{\constant{a}}{\constant{b}}{\variable{z}}}}{\Bppp{\constant{b}}{\constant{a}}{\variable{z}}}$ is false.  So, again, by the definition of truth for $\forall$, that $\universal{\variable{y}}\universal{\variable{z}}\parhorseshoe{\parconjunction{\Cpp{\constant{a}}{\variable{y}}}{\Dppp{\constant{a}}{\variable{y}}{\variable{z}}}}{\Bppp{\variable{y}}{\constant{a}}{\variable{z}}}$ is false on $\emph{States}^{\constant{a}}$.
	
	Finally, because there is an $\constant{a}$-variant of \emph{States} on which $\universal{\variable{y}}\universal{\variable{z}}\parhorseshoe{\parconjunction{\Cpp{\constant{a}}{\variable{y}}}{\Dppp{\constant{a}}{\variable{y}}{\variable{z}}}}{\Bppp{\variable{y}}{\constant{a}}{\variable{z}}}$ is false, $\universal{\variable{x}}\universal{\variable{y}}\universal{\variable{z}}\parhorseshoe{\parconjunction{\Cpp{\variable{x}}{\variable{y}}}{\Dppp{\variable{x}}{\variable{y}}{\variable{z}}}}{\Bppp{\variable{y}}{\variable{x}}{\variable{z}}}$ is false on \emph{States} (definition of truth, $\forall$).
\end{PROOF}

\subsection{Logical Truth: QT, QF, \& QC}\label{QT QF QI GQL}
The concepts of quantificational truth (\CAPS{qt}), quantificational falsehood (\CAPS{qf}), and quantificational contingency (\CAPS{qc}) are defined for \GQL{} exactly as in \GQL{}1.\footnote{See section \vref{QT QT QI}.} 

%%%%%%%%%%%%%%%%%%%%%%%%%%%%%%%%%%%%%%%%%%%%%%%%%%
\subsection{Entailment and other Relations}\label{GQL Entailment and other Relations}
%%%%%%%%%%%%%%%%%%%%%%%%%%%%%%%%%%%%%%%%%%%%%%%%%%

The concepts for entailment and the other logical relations are also defined for \GQL{} exactly as in \GQL{}1.\footnote{See section \vref{GQL1 Entailment and other Relations}.} 


%%%%%%%%%%%%%%%%%%%%%%%%%%%%%%%%%%%%%%%%%%%%%%%%%%
\section{The Dragnet Theorem}\label{Dragnet Theorem}
%%%%%%%%%%%%%%%%%%%%%%%%%%%%%%%%%%%%%%%%%%%%%%%%%%

In example \ref{GQL Entailment Example 2} we established that $\universal{\variable{x}}\Gp{\variable{x}}\sdtstile{}{}\Gp{\constant{b}}$ holds by reasoning as follows:

\begin{enumerate}[label=(\roman*)]
	\item Any model $\IntA$ that makes $\universal{\variable{x}}\Gp{\variable{x}}$ true makes $\Gp{\constant{a}}$ true on all $\constant{a}$-variants of $\IntA$.
	\item All $\constant{a}$-variants of $\IntA$ share the same domain as $\m$, so there is some $\constant{a}$-variant of $\IntA$, $\As{\constant{a}}{}$, such that $\As{\constant{a}}{}(\constant{a})=\IntA(\constant{b})$.
	\item Because $\As{\constant{a}}{}(\constant{a})\in\As{\constant{a}}{}(\GG)$ and $\As{\constant{a}}{}(\GG)=\IntA(\GG)$, $\As{\constant{a}}{}(\constant{a})\in\As{}{}(\GG)$.
	\item Because $\As{\constant{a}}{}(\constant{a})\in\As{}{}(\GG)$ and $\As{\constant{a}}{}(\constant{a})=\IntA(\constant{b})$, $\As{}{}(\constant{b})\in\As{}{}(\GG)$.
	Thus, $\Gp{\constant{b}}$ is true on $\IntA$.
\end{enumerate}

\noindent{}By analogous reasoning we can show that $\universal{\variable{x}}\Gp{\variable{x}}$ entails $\Gp{\constant{c}}$, $\Gp{\constant{d}}$, $\Gp{\constant{e}}$, and so on.
But it seems as if we should be able to save ourselves some work by proving a more general result.
Can we show that the entailment holds regardless of which constant follows $\GG$?
Yes, but demonstrating this is easier if we first develop \GQL{} metatheory a bit more.
For such proofs we use a metavariable, in this case $\variable{t}$: $\universal{\variable{x}}\Gp{\variable{x}}\sdtstile{}{}\Gp{\variable{x}}(\variable{t}/\variable{x})$, where $\variable{t}$ is any constant.
Remember from section \pmvref{MathEnglishVariableSubEx1} that $\CAPPHI\variable{t}/\variable{x}$ is the result of replacing every unbound token of $\variable{x}$ with a token of $\variable{t}$ in $\CAPPHI$.

We also want to prove stronger entailments, such as: $\universal{\variable{x}}\CAPPHI\sdtstile{}{}\CAPPHI\constant{b}/\variable{x}$, where $\CAPPHI$ is some \GQL{} formula with only $\variable{x}$ free.

To simplify the task of proving such general theorems we use the \mention{Dragnet Theorem}.
Informally, the theorem says: ``If you have a true sentence and replace the constants in it, but keep the same elements assigned to the new constants, you get another true sentence.''
Dragnet is extraordinarily helpful for proving many significant results about \GQL{} (e.g., in theorem \pmvref{Soundness of Quantifier Logic}, and \pmvref{MethodLemmaC}).
Although the claim behind Dragnet is fairly straightforward, its proof is long and complicated.
\\
\begin{commentary}
	The name ``Dragnet'' comes from the Dragnet radio and TV show.
    The theorem bears a close resemblance to the show's opening narration:
	``The story you are about to see is true. Only the names have been changed to protect the innocent.'' -- Sergeant Joe Friday, LAPD.
\end{commentary}

\begin{THEOREM}{\LnpTC{The Dragnet Theorem} The Dragnet Theorem:} If
\begin{cenumerate}
\item there exists: \\(a) a list of distinct constants $\variable{t}_1$, $\variable{t}_2$, $\ldots$, $\variable{t}_{\variable{i}}$, $\variable{s}_1$, $\variable{s}_2$, $\ldots$, $\variable{s}_{\variable{i}}$, \\(b) a \GQL{} sentence $\CAPPHI$ that contains $\variable{t}_1$, $\variable{t}_2$, $\ldots$, $\variable{t}_{\variable{i}}$ but not $\variable{s}_1$, $\variable{s}_2$, $\ldots$, $\variable{s}_{\variable{i}}$, \\(c) a second sentence $\CAPPHI^*=\CAPPHI\variable{s}_1/\variable{t}_1,\variable{s}_2/\variable{t}_2,\ldots,\variable{s}_{\integer{i}}/\variable{t}_{\integer{i}}$, and
\item (a) $\As{}{1}$ and $\As{}{2}$ are \GQL{} models that have the same domain $\integer{U}$ and make the same assignments to everything in $\CAPPHI$ except $\variable{t}_1$, $\variable{t}_2$, $\ldots$, $\variable{t}_{\variable{i}}$, and \\(b) $\As{}{1}(\variable{t}_1)=\As{}{2}(\variable{s}_1)$, $\As{}{1}(\variable{t}_2)=\As{}{2}(\variable{s}_2)$, $\ldots$, $\As{}{1}(\variable{t}_{\integer{i}})=\As{}{2}(\variable{s}_{\integer{i}})$,
\end{cenumerate} 
Then: $\CAPPHI$ is true on $\As{}{1}$ iff $\CAPPHI^*$ is true on $\As{}{2}$.\footnote{
	See \citealt[66]{Mates1972} and \citealt[577]{Bergmann2003} for different versions of essentially the same theorem.
}

\end{THEOREM}

\begin{PROOF}

\begin{commentary}
In this proof we treat $^*$ as a function that takes a \GQL{} formula $\CAPPHI$ with constants $\variable{t}_1$, $\variable{t}_2$, $\ldots$, $\variable{t}_{\variable{i}}$ and returns an otherwise identical formula in which those constants are replaced with ones not in $\CAPPHI$, $\variable{s}_1$, $\variable{s}_2$, $\ldots$, $\variable{s}_{\variable{i}}$, respectively: $\CAPPHI^*=\CAPPHI\variable{s}_1/\variable{t}_1,\variable{s}_2/\variable{t}_2,\ldots,\variable{s}_{\integer{i}}/\variable{t}_{\integer{i}}$.
This stipulation allows us to take for granted that any pair of sentences $\CAPPSI$ and $\CAPPSI^*$ satisfies Dragnet condition (1) above. 
Additionally, throughout the proof we assume that $\As{}{1}$ and $\As{}{2}$ are two arbitrary models that satisfy Dragnet condition (2) above.
\end{commentary}

\begin{description}
\item[Base Step:] $\CAPPHI$ is atomic. There are three cases:
\begin{cenumerate}
\item $\CAPPHI$ is a sentence letter. Then there are no constants and $\CAPPHI$ and $\CAPPHI^*$ are be identical.
Clearly $\CAPPHI$ is true on $\As{}{1}$ iff $\CAPPHI^*$ is true on $\As{}{2}$.

\item $\CAPPHI$ is a predicate letter $\PP$ followed by $n$ constants, $\Pp{\variable{q}_{\integer{1}}\variable{q}_{\integer{2}}\ldots\variable{q}_{\integer{n}}}$.
Some or all of these constants are to be replaced in $\CAPPHI^*$.
We label the constants to be replaced $\variable{t}_1,\variable{t}_2,\ldots,\variable{t}_\integer{i}$, and their replacements $\variable{s}_1,\variable{s}_2,\ldots,\variable{s}_\integer{i}$.
Then $\CAPPHI^*=\CAPPHI\variable{s}_1/\variable{t}_1,\variable{s}_2/\variable{t}_2,\ldots,\variable{s}_{\integer{i}}/\variable{t}_{\integer{i}}$.
Let's assume, without loss of generality, that $\CAPPHI=\Pp{\variable{q}_{\integer{1}}\ldots\variable{t}_1\ldots\variable{t}_2\ldots\variable{t}_{\integer{i}}\ldots\variable{q}_{\integer{n}}}$.
So $\CAPPHI^*$ is $\Pp{\variable{q}_{\integer{1}}\ldots\variable{s}_1\ldots\variable{s}_2\ldots\variable{s}_{\integer{i}}\ldots\variable{q}_{\integer{n}}}$.
By the definition of truth, 

\begin{center}
$\CAPPHI$ is true on $\As{}{1}$ iff $\langle\As{}{1}(\variable{q}_{\integer{1}}),\ldots,\As{}{1}(\variable{t}_1),\ldots,\As{}{1}(\variable{t}_{\integer{i}}),\ldots,\As{}{1}(\variable{q}_{\integer{n}})\rangle\in\As{}{1}(\PP)$.
\end{center}

By Dragnet condition (2), $\As{}{1}(\PP)=\As{}{2}(\PP)$, so we substitute:
\begin{center}
	$\CAPPHI$ is true on $\As{}{1}$ iff $\langle\As{}{1}(\variable{q}_{\integer{1}}),\ldots,\As{}{1}(\variable{t}_1),\ldots,\As{}{1}(\variable{t}_{\integer{i}}),\ldots,\As{}{1}(\variable{q}_{\integer{n}})\rangle\in\As{}{2}(\PP)$.
\end{center}

By Dragnet condition (2), $\As{}{1}(\variable{t}_1)=\As{}{2}(\variable{s}_1)$, $\ldots$, $\As{}{1}(\variable{t}_{\integer{i}})=\As{}{2}(\variable{s}_{\integer{i}})$, so we again substitute:
\begin{center}
$\CAPPHI$ is true on $\As{}{1}$ iff $\langle\As{}{1}(\variable{q}_{\integer{1}}),\ldots,\As{}{2}(\variable{s}_1),\ldots,\As{}{2}(\variable{s}_{\integer{i}}),\ldots,\As{}{1}(\variable{q}_{\integer{n}})\rangle\in\As{}{2}(\PP)$.
\end{center}

$\As{}{1}$ and $\As{}{2}$ otherwise make all the same assignments.
So for each constant $\variable{q_k}$ in $\CAPPHI$ that \emph{isn't} replaced in $\CAPPHI^*$, $\As{}{1}(\variable{q}_{\integer{k}})=\As{}{2}(\variable{q}_{\integer{k}})$: 
\begin{center}
$\CAPPHI$ is true on $\As{}{1}$ iff $\langle\As{}{2}(\variable{q}_{\integer{1}}),\ldots,\As{}{2}(\variable{s}_1),\ldots,\As{}{2}(\variable{s}_{\integer{i}}),\ldots,\As{}{2}(\variable{q}_{\integer{n}})\rangle\in\As{}{2}(\PP)$.
\end{center}

Thus, by the definition of truth:
\begin{center}
    $\CAPPHI$ is true on $\As{}{1}$ iff $\CAPPHI^*$ is true on $\As{}{2}$.
\end{center}
\end{cenumerate}
\item[Inheritance Step:] \hfill 
\begin{description}
\item[Recursive Assumption] Assume that the Dragnet property holds of all \GQL{} sentences of order $k$ or less and that $\CAPPHI$ is an \GQL{} sentence of order $k+1$.
Consider all the ways to construct a sentence of order $k+1$:

\item[Negation:] $\CAPPHI$ is of the form $\negation{\CAPPSI}$.
Since the symbol \mention{$\NEGATION$} is not a constant it is not replaced in $\CAPPHI^*$.
Then $\CAPPHI^*=\negation{\CAPPSI^*}$.
Since $\CAPPSI$ is of order $k$, by recursive assumption (RA):

\begin{center}
$\CAPPSI$ is true on $\As{}{1}$ iff $\CAPPSI^*$ is true on $\As{}{2}$,
\end{center}

It follows that:

\begin{center}
$\CAPPSI$ is false on $\As{}{1}$ iff $\CAPPSI^*$ is false on $\As{}{2}$.
\end{center}

$\CAPPSI$ is false on $\As{}{1}$ \Iff $\negation{\CAPPSI}$ is true on $\As{}{1}$.
The same holds of $\CAPPSI^*$.
Thus:

\begin{center}
$\negation{\CAPPSI}$ is true on $\As{}{1}$ iff $\negation{\CAPPSI^*}$ is true on $\As{}{2}$.
\end{center}

\item[Conjunction:] $\CAPPHI$ is of the form $\parconjunction{\CAPPSI_1}{\conjunction{\CAPPSI_2}{\conjunction{\ldots}{\CAPPSI_{\integer{n}}}}}$. 
Since the symbol \mention{$\WEDGE$} is not a constant these tokens are not replaced in $\CAPPHI^*$.
Then $\CAPPHI^*=\parconjunction{\CAPPSI_1^*}{\conjunction{\CAPPSI_2^*}{\conjunction{\ldots}{\CAPPSI_{\integer{n}}^*}}}$. 
Let $\CAPPSI_{\integer{j}}$ be the $\integer{j}^{th}$ conjunct of $\CAPPHI$, where $1\leq j\leq n$.
Each $\CAPPSI_{\integer{j}}$ is of order $k$ or lower.
So, by RA:

\begin{center}
For all $\CAPPSI_{\integer{j}}$, $\CAPPSI_j$ is true on $\As{}{1}$ iff $\CAPPSI_j^*$ is true on $\As{}{2}$.
\end{center}

From these \mention{iff} clauses it follows that:

\begin{center}
All of $\CAPPSI_1$, $\CAPPSI_2$, $\ldots$, $\CAPPSI_{\integer{n}}$ are true on $\As{}{1}$ iff all of $\CAPPSI_1^*$, $\CAPPSI_2^*$, $\ldots$, $\CAPPSI_{\integer{n}}^*$ are true on $\As{}{2}$.
\end{center}

Thus, by the definition of truth for $\WEDGE$:

\begin{center}
$\parconjunction{\CAPPSI_1}{\conjunction{\CAPPSI_2}{\conjunction{\ldots}{\CAPPSI_{\integer{n}}}}}$ is true on $\As{}{1}$ iff  $\parconjunction{\CAPPSI_1^*}{\conjunction{\CAPPSI_2^*}{\conjunction{\ldots}{\CAPPSI_{\integer{n}}^*}}}$ is true on $\As{}{2}$.
\end{center}

\item[Disjunction:] We leave this case for the reader as an exercise.

\item[Conditional:] $\CAPPHI$ is of the form $\parhorseshoe{\CAPPSI}{\CAPTHETA}$.
Since the symbol \mention{$\HORSESHOE$} is not a constant it is not replaced in $\CAPPHI^*$.
Then $\parhorseshoe{\CAPPSI}{\CAPTHETA}^*=\parhorseshoe{\CAPPSI^*}{\CAPTHETA^*}$.
By the definition of truth for $\HORSESHOE$:

\begin{center}
$\parhorseshoe{\CAPPSI}{\CAPTHETA}$ is true on $\As{}{1}$
  iff  either
   (i) $\CAPPSI$ is false on $\As{}{1}$, or (ii)
   $\CAPTHETA$ is true on $\As{}{1}$.
\end{center}

The sentence $\CAPPSI$ is of order $k$ or lower.
Then, by RA, $\CAPPSI$ is true on $\As{}{1}$ iff $\CAPPSI^*$ is true on $\As{}{2}$.
It follows that $\CAPPSI$ is false on $\As{}{1}$ iff $\CAPPSI^*$ is false on $\As{}{2}$.
So we substitute into part (i):

\begin{center}
$\parhorseshoe{\CAPPSI}{\CAPTHETA}$ is true on $\As{}{1}$
  iff  either  
	  (i) $\CAPPSI^*$ is false on $\As{}{2}$,
        or (ii) $\CAPTHETA$ is true on $\As{}{1}$.
\end{center}

Similarly, $\CAPTHETA$ is of order $k$ or lower.
Then by RA, $\CAPTHETA$ is true on $\As{}{1}$ iff $\CAPTHETA^*$ is true on $\As{}{2}$.
So we substitute into part (ii):

\begin{center}
$\parhorseshoe{\CAPPSI}{\CAPTHETA}$ is true on $\As{}{1}$
 iff either 
  (i) $\CAPPSI^*$ is false on $\As{}{2}$,
 or (ii) $\CAPTHETA^*$ is true on $\As{}{2}$.
\end{center}

By the definition of truth for $\HORSESHOE$:

\begin{center}
$\parhorseshoe{\CAPPSI}{\CAPTHETA}$ is true on $\As{}{1}$
 iff $\parhorseshoe{\CAPPSI^*}{\CAPTHETA^*}$ is true on $\As{}{2}$.
\end{center}

And since $\parhorseshoe{\CAPPSI}{\CAPTHETA}^*=\parhorseshoe{\CAPPSI^*}{\CAPTHETA^*}$:

\begin{center}
$\parhorseshoe{\CAPPSI}{\CAPTHETA}$ is true on $\As{}{1}$ iff $\parhorseshoe{\CAPPSI}{\CAPTHETA}^*$ is true on $\As{}{2}$.
\end{center}

\item[Biconditional:] We leave this case for the reader as an exercise.

\item[Quantifier Preface:] $\CAPPHI$ is of the form $\universal{\BETA}\CAPPSI$ or $\existential{\BETA}\CAPPSI$.
Then formula $\CAPPSI$ is of order $k$. To reduce the work needed for the quantifier clauses, we first prove an intermediate result.
Let $\variable{q}$ be the first constant not in $\CAPPSI$ and let $\variable{r}$ be the first constant not in $\CAPPSI^*$.

\begin{commentary}
	We cannot assume that $\variable{q}=\variable{r}$.
	Consider the case such that $\CAPPSI=\parhorseshoe{\Dp{\variable{x}}}{\Gpp{\variable{x}}{\constant{b}}}$ and $\CAPPSI^*=\CAPPSI\constant{a}/\constant{b}=\parhorseshoe{\Dp{\variable{x}}}{\Gpp{\variable{x}}{\constant{a}}}$.
	The first constant not in $\parhorseshoe{\Dp{\variable{x}}}{\Gpp{\variable{x}}{\constant{b}}}$ is $\constant{a}$, and the first constant not in $\parhorseshoe{\Dp{\variable{x}}}{\Gpp{\variable{x}}{\constant{a}}}$ is $\constant{b}$.
\end{commentary}

Since $\m_1$ and $\m_2$ satisfy Dragnet condition (2), $\m_1(U)=\m_2(U)$.
Let $\Delta$ be the domain shared by $\m_1$ and $\m_2$.
Then:

\begin{center}
	For each $\delta\in\Delta$ there is a $\variable{q}$-variant of $\As{}{1}$, $\model{\variable{q}}{1}(\variable{q})=\delta$, and an $\variable{r}$-variant of $\As{}{2}$, $\model{\variable{r}}{2}(\variable{r})=\delta$, such that $\CAPPSI\variable{q}/\BETA$ is true on $\model{\variable{q}}{1}$ \Iff $\CAPPSI^*\variable{r}/\BETA$ is true on $\model{\variable{r}}{2}$.
\end{center}

\begin{commentary}
	The purpose of this claim may not be obvious.
	If you are willing to accept on faith that it's a useful result, feel free to continue.
	If not, then skip down to the \mention{Universal Quantification} clause and read through that until you come to the step that cites the Quantifier Preface.
	Once you understand why this Quantifier Preface is helpful, then come back and pick up where you left off.
\end{commentary}

\begin{SUBPROOF}

	\begin{commentary}
		It may be tempting to try to derive this result by applying the recursive assumption directly to $\CAPPSI\variable{q}/\BETA$ and $\CAPPSI^*\variable{r}/\BETA$.
		However, it's possible that $\variable{q}$ is in $\CAPPSI\variable{r}/\BETA$ or $\variable{r}$ is in $\CAPPSI\variable{q}/\BETA$.
		Either one violates Dragnet condition (1).
		\commentaryspace
		Our strategy is to use a third sentence, $\CAPPSI^*\variable{u}/\BETA$, where $\variable{u}$ is a variable that is not in either $\CAPPSI\variable{q}/\BETA$ or $\CAPPSI^*\variable{r}/\BETA$.
		We first prove the desired result with $\CAPPSI^*\variable{r}/\BETA$ and $\CAPPSI^*\variable{u}/\BETA$ (i.e. that there are variants of $\m_1$/$\m_2$ such that one sentence is true \Iff the other is).
		Then we do the same with $\CAPPSI\variable{q}/\BETA$ and $\CAPPSI^*\variable{u}/\BETA$.
		Finally, we can \mention{factor out} $\CAPPSI^*\variable{u}/\BETA$ as a middle term, and prove the desired result for $\CAPPSI\variable{q}/\BETA$ and $\CAPPSI^*\variable{r}/\BETA$.
	\end{commentary}

	% \begin{commentary}
	% 	We proceed in two parts.
	% 	First, we show that $\CAPPSI\variable{q}/\BETA$ and $\CAPPSI^*\variable{r}/\BETA$ satisfy Dragnet condition (1).
	% 	Second, we show that each corresponding $\model{\variable{q}}{1}$ and $\model{\variable{r}}{2}$ pair satisfy Dragnet condition (2).
	% 	Since $\CAPPSI$ is of order $k$ we are then justified in using the recursive assumption to complete the subproof.		
	% \end{commentary}

	(Part 1:) Let $\variable{u}$ be a constant not in either $\CAPPSI^*\variable{r}/\BETA$ or $\CAPPSI^*\variable{u}/\BETA$.
	Then $\CAPPSI^*\variable{r}/\BETA$ and $\CAPPSI^*\variable{u}/\BETA$ are the same except that the latter sentence has $\variable{u}$ in place of $\variable{r}$.
	These sentences satisfy Dragnet condition (1).

	Let $\delta$ be an arbitrary element of $\Delta$.
	Since $\Delta=\m_2(U)$, $\delta\in\m_2(U)$, and so there is an $\variable{r}$-variant of $\m_2$, $\model{\variable{r}}{2}$, such that $\model{\variable{r}}{2}(r)=\delta$.
	And there is a $\variable{u}$-variant of $\m_2$, $\model{\variable{u}}{2}$, such that $\model{\variable{u}}{2}(u)=\delta$.
	So, $\model{\variable{r}}{2}$ and $\model{\variable{u}}{2}$ make the same assignments to everything in $\CAPPSI^*\variable{r}/\BETA$ besides $\variable{r}$, and $\model{\variable{u}}{2}=\model{\variable{r}}{2}$.
	These models satisfy Dragnet condition (2).
	
	$\CAPPSI^*\variable{r}/\BETA$ and $\CAPPSI^*\variable{u}/\BETA$ are of order $k$.
    Thus, by RA:

	\begin{center}
		For each $\delta\in\Delta$ there is a $\variable{r}$-variant of $\As{}{2}$, $\model{\variable{r}}{2}(\variable{q})=\delta$, and a $\variable{u}$-variant of $\As{}{2}$, $\model{\variable{u}}{2}(\variable{u})=\delta$, such that $\CAPPSI^*\variable{r}/\BETA$ is true on $\model{\variable{r}}{2}$ \Iff $\CAPPSI^*\variable{u}/\BETA$ is true on $\model{\variable{u}}{2}$.
	\end{center}
	
	(Part 2:) The formula $\CAPPSI$ contains the terms $\variable{t}_1,\variable{t}_2,\ldots,\variable{t}_{\variable{i}},\BETA$.
	Then $\CAPPSI\variable{q}/\BETA$ is a sentence that contains the constants $\variable{t}_1,\variable{t}_2,\ldots,\variable{t}_{\variable{i}},\variable{q}$.
	And $\CAPPSI^*\variable{u}/\BETA$ is a sentence that contains the constants $\variable{s}_1,\variable{s}_2,\ldots,\variable{s}_{\variable{i}},\variable{u}$.
	It follows that $\CAPPSI\variable{q}/\BETA$ and $\CAPPSI^*\variable{u}/\BETA$ are exactly alike, except that $\variable{t}_1,\variable{t}_2,\ldots,\variable{t}_{\variable{i}},\variable{q}$ in $\CAPPSI\variable{q}/\BETA$ are replaced with $\variable{s}_1,\variable{s}_2,\ldots,\variable{s}_{\variable{i}},\variable{u}$ in $\CAPPSI^*\variable{u}/\BETA$.
	So the sentences $\CAPPSI\variable{q}/\BETA$ and $\CAPPSI^*\variable{u}/\BETA$ satisfy Dragnet condition (1).
	
	Let $\delta$ be an arbitrary element of $\Delta$.
	Since $\Delta=\m_1(U)$, $\delta\in\m_1(U)$, and so there is a $\variable{q}$-variant of $\m_1$, $\model{\variable{q}}{1}$, such that $\model{\variable{q}}{1}(q)=\delta$.
	And since $\Delta=\m_2(U)$, $\delta\in\m_2(U)$, and so there is a $\variable{u}$-variant of $\m_2$, $\model{\variable{u}}{2}$, such that $\model{\variable{u}}{2}(u)=\delta$.
	Hence $\As{\variable{q}}{1}(\variable{q})=\As{\variable{u}}{2}(\variable{u})$.

	It was assumed that $\As{}{1}(\variable{t}_1)=\As{}{2}(\variable{s}_1)$, $\As{}{1}(\variable{t}_2)=\As{}{2}(\variable{s}_2)$, $\ldots$, $\As{}{1}(\variable{t}_{\integer{i}})=\As{}{2}(\variable{s}_{\integer{i}})$.
	The model $\As{}{1}$ differs from $\As{\variable{q}}{1}$ only on the assignment to $\variable{q}$.
	But $\variable{q}$ is the first constant not in $\CAPPSI$, so $\As{}{1}$ and $\As{\variable{q}}{1}$ make the same assignments to each of $\variable{t}_1$, $\ldots$, $\variable{t}_\integer{i}$.
	Analogous reasoning shows that $\As{}{2}$ and $\As{\variable{u}}{2}$ make the same assignments to each of $\variable{s}_1$, $\ldots$, $\variable{s}_\integer{i}$.
	Therefore, $\As{q}{1}(\variable{t}_1)=\As{u}{2}(\variable{s}_1)$, $\As{q}{1}(\variable{t}_2)=\As{u}{2}(\variable{s}_2)$, $\ldots$, $\As{q}{1}(\variable{t}_{\integer{i}})=\As{u}{2}(\variable{s}_{\integer{i}})$, and $\As{\variable{q}}{1}(\variable{q})=\As{\variable{u}}{2}(\variable{u})$.
	
	And since $\As{\variable{q}}{1}$ and $\As{\variable{u}}{2}$ are variants of $\m_1$ and $\m_2$, respectively, they make the same assignments to everything in $\CAPPSI\variable{q}/\BETA$ apart from $\variable{t}_1,\variable{t}_2,\ldots,\variable{t}_{\variable{i}},\variable{q}$.
	Thus, $\As{\variable{q}}{1}$ and $\As{\variable{u}}{2}$ satisfy Dragnet condition (2).
	The sentences $\CAPPSI\variable{q}/\BETA$ and $\CAPPSI^*\variable{u}/\BETA$ are each of order $k$, so, by RA,
	
	\begin{center}
		For each $\delta\in\Delta$ there is a $\variable{q}$-variant of $\As{}{1}$, $\model{\variable{q}}{1}=\delta$, and a $\variable{u}$-variant of $\As{}{2}$,\\$\model{\variable{u}}{2}=\delta$, such that $\CAPPSI\variable{q}/\BETA$ is true on $\model{\variable{q}}{1}$ \Iff $\CAPPSI^*\variable{u}/\BETA$ is true on $\model{\variable{u}}{2}$.
	\end{center}

	From the conclusions of parts 1 and 2, it therefore follows that:

	\begin{center}
		For each $\delta\in\Delta$ there is a $\variable{q}$-variant of $\As{}{1}$, $\model{\variable{q}}{1}=\delta$, and an $\variable{r}$-variant of $\As{}{2}$,\\$\model{\variable{r}}{2}=\delta$, such that $\CAPPSI\variable{q}/\BETA$ is true on $\model{\variable{q}}{1}$ \Iff $\CAPPSI^*\variable{r}/\BETA$ is true on $\model{\variable{r}}{2}$.
	\end{center}

\end{SUBPROOF}

\begin{description}

	\item[Universal Quantification:] $\CAPPHI$ is of the form $\universal{\BETA}\CAPPSI$, where $\CAPPSI$ is a formula that has exactly one free variable, $\BETA$.
	Since neither \mention{$\forall$} nor \mention{$\BETA$} is a constant, $\CAPPHI^*=\universal{\BETA}\CAPPSI^*$.
	Since $\m_1$ and $\m_2$ satisfy Dragnet condition (2), $\m_1(U)=\m_2(U)$.
	Call this shared domain $\Delta$.
	By the Quantifier Preface, for each $\delta\in\Delta$ there is a $\variable{q}$-variant of $\As{}{1}$, $\model{\variable{q}}{1}$, and an $\variable{r}$-variant of $\As{}{2}$, $\model{\variable{r}}{2}$, such that $\CAPPSI\variable{q}/\BETA$ is true on $\model{\variable{q}}{1}$ \Iff $\CAPPSI^*\variable{r}/\BETA$ is true on $\model{\variable{r}}{2}$.
	From this it follows that:

	\begin{center}
		$\CAPPSI\variable{q}/\BETA$ is true on every $\variable{q}$-variant of $\As{}{1}$ iff\\
		$\CAPPSI^*\variable{r}/\BETA$ is true on every $\variable{r}$-variant of $\As{}{2}$.
	\end{center}

	By the definition of truth of $\forall$:

	\begin{center}
		$\universal{\BETA}\CAPPSI$ is true on $\As{}{1}$ iff
		$\CAPPSI\variable{q}/\BETA$ is true on every $\variable{q}$-variant of $\As{}{1}$.
	\end{center}

	And:

	\begin{center}
		$\universal{\BETA}\CAPPSI^*$ is true on $\As{}{2}$ iff
		$\CAPPSI^*\variable{r}/\BETA$ is true on every $\variable{r}$-variant of $\As{}{2}$.
	\end{center}

	Substitute these two results into the first to get:

	\begin{center}
		$\universal{\BETA}\CAPPSI$ is true on $\As{}{1}$
		iff $\universal{\BETA}\CAPPSI^*$ is true on $\As{}{2}$.
	\end{center}

	\item[Existential Quantification:] $\CAPPHI$ is of the form $\existential{\BETA}\CAPPSI$, where $\CAPPSI$ is a formula that has exactly one free variable, $\BETA$.
	Since neither \mention{$\exists$} nor \mention{$\BETA$} is a constant, $\CAPPHI^*=\existential{\BETA}\CAPPSI^*$.
	Since $\m_1$ and $\m_2$ satisfy Dragnet condition (2), $\m_1(U)=\m_2(U)$.
	Call this shared domain $\Delta$.
	By the Quantifier Preface, for each $\delta\in\Delta$ there is a $\variable{q}$-variant of $\As{}{1}$, $\model{\variable{q}}{1}$, and an $\variable{r}$-variant of $\As{}{2}$, $\model{\variable{r}}{2}$, such that $\CAPPSI\variable{q}/\BETA$ is true on $\model{\variable{q}}{1}$ \Iff $\CAPPSI^*\variable{r}/\BETA$ is true on $\model{\variable{r}}{2}$.
	From this it follows that:

	\begin{center}
		$\CAPPSI\variable{q}/\BETA$ is true on some $\variable{q}$-variant of $\As{}{1}$ iff\\
		$\CAPPSI^*\variable{r}/\BETA$ is true on some $\variable{r}$-variant of $\As{}{2}$.
	\end{center}

	By the definition of truth of $\exists$:

	\begin{center}
		$\existential{\BETA}\CAPPSI$ is true on $\As{}{1}$ iff
		$\CAPPSI\variable{q}/\BETA$ is true on some $\variable{q}$-variant of $\As{}{1}$.
	\end{center}

	And:

	\begin{center}
		$\existential{\BETA}\CAPPSI^*$ is true on $\As{}{2}$ iff
		$\CAPPSI^*\variable{r}/\BETA$ is true on some $\variable{r}$-variant of $\As{}{2}$.
	\end{center}

	Substitute these two results into the first to get:

	\begin{center}
		$\existential{\BETA}\CAPPSI$ is true on $\As{}{1}$
		iff $\existential{\BETA}\CAPPSI^*$ is true on $\As{}{2}$.
	\end{center}

\end{description}

\end{description}

\item[Closure Step:] There is no other way to construct a \GQL{} sentence, so we have shown that the Dragnet property holds of all \GQL{} sentences.
\end{description}
\end{PROOF}

Dragnet is particularly useful when making arguments about sentences with unknown structures.

\begin{majorILnc}{\LnpEC{DragnetExampleTwo}}
Let $\CAPPHI$ be a formula whose only free variable is $\variable{x}$.
Prove that $\universal{\variable{x}}\CAPPHI\sdtstile{}{}\CAPPHI\constant{b}/\variable{x}$.  
\end{majorILnc} 
 
\begin{PROOF}
Suppose $\universal{\variable{x}}\CAPPHI$ is true on $\IntA$. 
Then every $\variable{t}$-variant of $\IntA$ makes $\CAPPHI\variable{t}/\variable{x}$ true.
There is a $\variable{t}$-variant, $\As{\variable{t}}{}$, that makes the same assignment to $\variable{t}$ that $\IntA$ assigns to $\constant{b}$.
So $\CAPPHI\variable{t}/\variable{x}$ is true on $\As{\variable{t}}{}$.
The sentences $\CAPPHI\variable{t}/\variable{x}$ and $\CAPPHI\constant{b}/\variable{x}$ are the same, except that $\variable{t}$ is replaced by $\constant{b}$ in the latter.
So they satisfy Dragnet condition (1).
The models $\IntA$ and $\As{\variable{t}}{}$ make the same assignments to everything in $\CAPPHI\variable{t}/\variable{x}$ besides $\variable{t}$, and $\As{\variable{t}}{}(\variable{t})=\IntA(\constant{b})$.
So they satisfy Dragnet condition (2).
Thus, by Dragnet, $\IntA$ makes $\CAPPHI\constant{b}/\variable{x}$ true.
Therefore $\universal{\variable{x}}\CAPPHI\sdtstile{}{}\CAPPHI\constant{b}/\variable{x}$.
\end{PROOF} 

When might the Dragnet theorem be useful? 
Pay attention to the two Dragnet conditions.
When you can show that those hold there is a decent chance Dragnet can help.
In some cases you will need to be strategic in your selection of a model variant.

While the Dragnet theorem is helpful it can be a bit unwieldy.
So we prove another, the \mention{Free Choice} theorem, which builds on Dragnet and saves us some hassle.

\begin{THEOREM}{\LnpTC{The Free Choice Theorem} The Free Choice Theorem:}
(i) A \GQL{} sentence of the form $\universal\ALPHA\CAPPHI$ is true on some model $\IntA$ \Iff all $\variable{s}$-variants of $\IntA$ make $\CAPPHI\variable{s}/\ALPHA$ true, where $\variable{s}$ is any constant not in $\CAPPHI$; and (ii) a \GQL{} sentence of the form $\existential\ALPHA\CAPPHI$ is true on some model $\IntA$ \Iff some $\variable{s}$-variant of $\IntA$ makes $\CAPPHI\variable{s}/\ALPHA$ true, where $\variable{s}$ is any constant not in $\CAPPHI$.
\end{THEOREM}

\begin{PROOF}
(i) Let $\CAPPHI$ be a formula with only $\ALPHA$ free.
Let $\variable{t}$ be the first constant not in $\CAPPHI$ and let $\variable{s}$ be an arbitrary constant not in $\CAPPHI$.
There are two cases.
(Case: 1) $\variable{s}=\variable{t}$. By the definition of truth for $\forall$, $\universal\ALPHA\CAPPHI$ is true on $\IntA$ \Iff all $\variable{s}$-variants of $\IntA$ make $\CAPPHI\variable{s}/\ALPHA$ true.

(Case: 2) $\variable{s}\not=\variable{t}$.
The sentences $\CAPPHI\variable{t}/\ALPHA$ and $\CAPPHI\variable{s}/\ALPHA$ are the same, except $\CAPPHI\variable{s}/\ALPHA$ has $\variable{s}$ where $\CAPPHI\variable{t}/\ALPHA$ has $\variable{t}$.
So they satisfy condition (1) of Dragnet.

Let $\m$ be some model for $\universal\ALPHA\CAPPHI$.
All variants of $\m$ share a domain.
So for each element of $\m(U)$, there is a $\variable{t}$-variant of $\IntA$ and a corresponding $\variable{s}$-variant of $\IntA$ that assign this element to $\variable{t}$ and $\variable{s}$, respectively.
Let $\As{\variable{t}}{}$ and $\As{\variable{s}}{}$ be variants such that $\As{\variable{t}}{}(\variable{t})=\As{\variable{s}}{}(\variable{s})$.
They're both variants of $\m$, so they make the same assignments to everything in $\CAPPHI\variable{t}/\ALPHA$ besides $\variable{t}$.
Thus $\As{\variable{t}}{}$ and $\As{\variable{s}}{}$ satisfy condition (2) of Dragnet.

By Dragnet,	$\CAPPHI\variable{t}/\ALPHA$ is true on $\IntA$ \Iff $\CAPPHI\variable{s}/\ALPHA$ is true on $\As{\variable{s}}{}$.
We assumed nothing special about $\variable{s}$ except that it's a constant not in $\CAPPHI$.
So, all $\variable{t}$-variants of $\IntA$ make $\CAPPHI\variable{t}/\ALPHA$ true \Iff all $\variable{s}$-variants of $\IntA$ make $\CAPPHI\variable{s}/\ALPHA$ true.
Therefore, by the definition of truth of $\forall$, $\universal\ALPHA\CAPPHI$ is true on $\m$ \Iff all $\variable{s}$-variants of $\IntA$ make $\CAPPHI\variable{s}/\ALPHA$ true.

(ii) We leave this part as an exercise for the reader.
\end{PROOF}

\begin{majorILnc}{\LnpEC{FreeChoiceExampleOne}}
	Let $\CAPPHI$ and $\CAPPSI$ be formulas whose only free variable is $\variable{x}$.  Prove that $\universal{\variable{x}}\parhorseshoe{\CAPPHI}{\CAPPSI}\sdtstile{}{}\horseshoe{\universal{\variable{x}}\CAPPHI}{\universal{\variable{x}}\CAPPSI}$.  
\end{majorILnc} 
 
\begin{PROOF}
	Let $\IntA$ be some model that makes $\universal{\variable{x}}\parhorseshoe{\CAPPHI}{\CAPPSI}$ true.
	Then, by the definition of truth for $\forall$, $\parhorseshoe{\CAPPHI}{\CAPPSI}\variable{t}/\variable{x}$ is true on all $\variable{t}$-variants of $\IntA$.
	Since $\CAPPHI$ and $\CAPPSI$ are parts of $\horseshoe{\CAPPHI}{\CAPPSI}$, and the first constant not in $\horseshoe{\CAPPHI}{\CAPPSI}$ is $\variable{t}$, $\variable{t}$ is not in $\CAPPHI$ or $\CAPPSI$.
	There are two cases.

	(Case 1:) $\CAPPHI\variable{t}/\variable{x}$ is false on at least one $\variable{t}$-variant of $\IntA$.
	Then, by the Free Choice theorem, $\universal{\variable{x}}\CAPPHI$ is false on $\IntA$, and thus, by the definition of truth for $\HORSESHOE$, $\horseshoe{\universal{\variable{x}}\CAPPHI}{\universal{\variable{x}}\CAPPSI}$ is true on $\IntA$.

	\begin{commentary}
		We cannot directly use the definition of truth to conclude that $\universal{\variable{x}}\CAPPHI$ is false on $\IntA$ because we cannot prove that $\variable{t}$ is the first constant not in $\CAPPHI$.
		For all we know, it isn't.
		Working around this with Dragnet would be a bit of work, but Free Choice makes it easy.
	\end{commentary}

	(Case 2:) $\CAPPHI\variable{t}/\variable{x}$ is not false on at any $\variable{t}$-variant of $\IntA$.
	So it's true on all $\variable{t}$-variants.
	Since $\parhorseshoe{\CAPPHI}{\CAPPSI}\variable{t}/\variable{x}$ is also true on all $\variable{t}$-variants, then by the definition of truth for $\HORSESHOE$, $\CAPPSI\variable{t}/\variable{x}$ is true on all $\variable{t}$-variants.
	By the Free Choice theorem, $\universal{\variable{x}}\CAPPSI$ is true on $\IntA$, and thus, by the definition of truth for $\HORSESHOE$, $\horseshoe{\universal{\variable{x}}\CAPPHI}{\universal{\variable{x}}\CAPPSI}$ is true on $\IntA$.

	Nothing particular was assumed about $\IntA$.
	Any model that makes $\universal{\variable{x}}\parhorseshoe{\CAPPHI}{\CAPPSI}$ true also makes $\horseshoe{\universal{\variable{x}}\CAPPHI}{\universal{\variable{x}}\CAPPSI}$ true.
	Therefore, $\universal{\variable{x}}\parhorseshoe{\CAPPHI}{\CAPPSI}\sdtstile{}{}\horseshoe{\universal{\variable{x}}\CAPPHI}{\universal{\variable{x}}\CAPPSI}$.	
\end{PROOF}

%%%%%%%%%%%%%%%%%%%%%%%%%%%%%%%%%%%%%%%%%%%%%%%%%%
\section{Exercises}
%%%%%%%%%%%%%%%%%%%%%%%%%%%%%%%%%%%%%%%%%%%%%%%%%%

\notocsubsection{Formulas, Order, and Subformulas}{ex:Formulas, Order, and Subformulas} Which of the following are \emph{formulas}? 
For those that are formulas, what is their order? 
How many subformulas does each have?
\begin{multicols}{2}
\begin{enumerate}
\item {$\universal{\variable{x}}\parhorseshoe{\Hpp{'}{\variable{x}}}{\Gpp{'}{\variable{x}}}$}
\item {$\universal{\variable{x}}\parhorseshoe{\Hpp{'}{\variable{x}}}{\Gpp{''}{\variable{x}}}$}
\item {$\universal{\variable{x}}\parhorseshoe{\Hpp{'}{\variable{x}}}{\Gpp{'_7}{\variable{x}}}$}
\item {$\universal{\variable{x}}\universal{\variable{z}}\parhorseshoe{\Hpp{'}{\variable{x}}}{\Gppp{''}{\variable{x}}{\variable{y}}}$}
\item {$\universal{\variable{x}}\universal{\variable{y}}\parhorseshoe{\Hpp{'}{\variable{x}}}{\Gppp{''}{\variable{x}}{\variable{y}}}$}
\item {$\universal{\variable{x}}\universal{\variable{z}}\parhorseshoe{\Hpp{'}{\constant{a}}}{\Gppp{''}{\variable{x}}{\variable{y}}}$}
\item {$\universal{\variable{x}}\universal{\variable{z}}\parhorseshoe{\Hp{\variable{x}}}{\Gppp{''}{\variable{x}}{\variable{y}}}$}
\item {$\universal{\variable{x}}\universal{\variable{z}}\parhorseshoe{\Wpp{'}{\variable{x}}}{\Gppp{''}{\variable{x}}{\variable{y}}}$}
\item {$\universal{\variable{x}}\universal{\variable{z}}\bparhorseshoe{\Hpp{'}{\variable{x}}}{\Gppp{''}{\variable{x}}{\variable{y}}}$}
\item {$\universal{\variable{x}_9}\universal{\variable{z}}\parhorseshoe{\Hpp{'}{\variable{x}_9}}{\Gppp{''}{\variable{x}}{\variable{y}}}$}
\item {$\negation{\universal{\variable{x}}\universal{\variable{y}}\parhorseshoe{\Hpp{'}{\variable{x}}}{\Gppp{''}{\variable{x}}{\variable{y}}}}$}
\item {$\universal{\constant{b}}\universal{\variable{y}}\parhorseshoe{\Hpp{'}{\constant{b}}}{\Gppp{''}{\variable{x}}{\variable{y}}}$}
\item {$\universal{\variable{x}}\parhorseshoe{\universal{\variable{x}}\Hpp{'}{x}}{\Gppp{''}{\variable{x}}{\variable{y}}}$}
\item {$\universal{\variable{x}}\parhorseshoe{\universal{\variable{z}}\Hpp{'}{x}}{\Gppp{''}{\variable{x}}{\variable{y}}}$}
\item {$\universal{\variable{y}}\parhorseshoe{\universal{\variable{x}}\Hpp{'}{x}}{\universal{\variable{x}}\Gppp{''}{\variable{x}}{\variable{y}}}$}
\end{enumerate}
\end{multicols}

\notocsubsection{Sentences and Order}{ex:Sentences and Order} For each of the following say whether it is an official sentence, an unofficial sentence, an official formula but not a sentence, an unofficial formula but not a sentence or none of the above. 
If it is a formula or sentence (official or unofficial) say what its order is and how many sub\emph{formulas} it has.
\begin{multicols}{2}
\begin{enumerate}
\item {$\universal{\variable{x}}\universal{\variable{y}}\parhorseshoe{\Hpp{'}{\variable{x}}}{\Gppp{''}{\variable{x}}{\variable{y}}}$}
\item {$\universal{\variable{x}}\existential{\variable{z}}\parhorseshoe{\Hpp{'}{\variable{x}}}{\Gppp{''}{\variable{x}}{\variable{y}}}$}
\item {$\universal{\variable{x}}\universal{\variable{y}}\parconjunction{\Hp{\variable{x}}}{\conjunction{\Gpp{\variable{x}}{\variable{y}}}{\Kp{\variable{y}}}}$}
\item {$\universal{\variable{x}}\universal{\variable{y}}\parhorseshoe{\Hp{\variable{x}}}{\horseshoe{\Gpp{\variable{x}}{\variable{y}}}{\Kp{\variable{y}}}}$}
\item {$\universal{\variable{x}}\universal{\variable{y}}\parconjunction{\negation{\Hp{\variable{x}}}}{\conjunction{\Gpp{\variable{z}}{\variable{y}}}{\Kp{\variable{y}}}}$}
\item {$\universal{\variable{x}}\universal{\variable{y}}\parconjunction{\Hp{\variable{x}}}{\parconjunction{\Gpp{\variable{z}}{\variable{y}}}{\Kp{\variable{y}}}}$}
\item {$\universal{\variable{x}}\universal{\variable{y}}\parconjunction{\Hpp{'}{\variable{x}}}{\parconjunction{\Gppp{''}{\variable{z}}{\variable{y}}}{\Kpp{''}{\variable{y}}}}$}
\item {$\universal{\variable{x}}\universal{\variable{y}}\conjunction{\Hp{\variable{x}}}{\parconjunction{\Gpp{\variable{z}}{\variable{y}}}{\Kp{\variable{y}}}}$}
\item {$\universal{\variable{x}}\existential{\variable{z}}\universal{\variable{y}}\parconjunction{\Hp{\variable{x}}}{\conjunction{\Gpp{\variable{x}}{\variable{y}}}{\Kp{\variable{y}}}}$}
\item {$\universal{\variable{x}}\existential{\variable{y}}\universal{\variable{z}}\parconjunction{\Hp{\variable{x}}}{\conjunction{\universal{\variable{y}}\Gpp{\variable{x}}{\variable{y}}}{\Kp{\variable{y}}}}$}
\end{enumerate}
\end{multicols}

\notocsubsection{Truth in a Model}{ex:Truth in an Interpretation} Give the truth value of each of the following sentences on both of the models found in figure \mvref{table:Example Interpretations Exercise}. 
\begin{multicols}{2}
\begin{enumerate}
\item {$\universal{\variable{x}}\universal{\variable{y}}\cparhorseshoe{\parconjunction{\Ap{\variable{x}}}{\Bp{\variable{y}}}}{\App{\variable{x}}{\variable{y}}}$}
\item {$\universal{\variable{x}}\universal{\variable{y}}\cparhorseshoe{\parconjunction{\Cp{\variable{x}}}{\Dp{\variable{y}}}}{\Dpp{\variable{x}}{\variable{y}}}$}
\item {$\universal{\variable{x}}\cparhorseshoe{\parconjunction{\Cp{\variable{x}}}{\Ep{\variable{x}}}}{\Cpp{\variable{x}}{\constant{a}}}$}
\item {$\universal{\variable{x}}\universal{\variable{y}}\cparhorseshoe{\Cpp{\variable{x}}{\variable{y}}}{\App{\variable{x}}{\variable{y}}}$}
\item {$\horseshoe{\universal{\variable{x}}\Cp{\variable{x}}}{\universal{\variable{y}}\Dp{\variable{y}}}$}
\item {$\universal{\variable{z}}\universal{\variable{w}}\cparhorseshoe{\parconjunction{\Gp{\variable{z}}}{\Gp{\variable{w}}}}{\negation{\Gpp{\variable{z}}{\variable{w}}}}$}
\item {$\universal{\variable{z}}\universal{\variable{w}}\universal{\variable{x}}\cparhorseshoe{\Cppp{\variable{x}}{\variable{z}}{\variable{w}}}{\bpartriplebar{\Cp{\variable{x}}}{\Cp{\variable{w}}}}$}
\item {$\universal{\variable{x}}\cparhorseshoe{\Ap{\variable{x}}}{\existential{\variable{y}}\parconjunction{\Cp{\variable{y}}}{\Bpp{\variable{y}}{\variable{x}}}}$}
\item {$\existential{\variable{y}}\existential{\variable{x}}\parconjunction{\Cpp{\variable{x}}{\variable{y}}}{\Dpp{\variable{x}}{\variable{y}}}$}
\item {$\existential{\variable{y}}\existential{\variable{x}}\parconjunction{\Epp{\variable{x}}{\variable{y}}}{\Gpp{\variable{x}}{\variable{y}}}$}
\item {$\universal{\variable{x}}\cparhorseshoe{\Gp{\variable{x}}}{\universal{\variable{w}}\bparhorseshoe{\App{\variable{x}}{\variable{w}}}{\pardisjunction{\Ap{\variable{w}}}{\Cp{\variable{w}}}}}$}
\item {$\existential{\variable{x}}\cparhorseshoe{\Cp{\variable{x}}}{\universal{\variable{y}}\Cp{\variable{y}}}$}
\item {$\universal{\variable{z}}\universal{\variable{w}}\universal{\variable{x}}\cparhorseshoe{\Dppp{\variable{x}}{\variable{z}}{\variable{w}}}{\App{\variable{x}}{\variable{w}}}$}
\item {$\universal{\variable{y}}\bparhorseshoe{\Ap{\variable{y}}}{\existential{\variable{x}}\parconjunction{\Ep{\variable{x}}}{\App{\variable{x}}{\variable{y}}}}$}
\item {$\universal{\variable{x}}\universal{\variable{y}}\parhorseshoe{\Gpp{\variable{x}}{\variable{y}}}{\Gpp{\variable{y}}{\variable{x}}}$}
\end{enumerate}
\end{multicols}

\begin{figure}
\begin{longtable}[c]{ l l l l } %p{2.2in} p{2in}
	\toprule
	&\textbf{Symbol} & \multicolumn{2}{c}{\textbf{Model}} \\ \cmidrule(l){3-4}
	& & \textbf{Pos Int} & \textbf{States} \\
	\midrule 
	\endfirsthead
	\multicolumn{4}{c}{\emph{Continued from Previous Page}}\\
	\toprule
	&\textbf{Symbol} & \multicolumn{2}{c}{\textbf{Model}} \\ \cmidrule(l){3-4}
	& & \textbf{Pos Int} & \textbf{States} \\
	\midrule 
	\endhead
	\bottomrule
	\caption{Example Models}\\[-.15in]
	\multicolumn{4}{c}{\emph{Continued next Page}}\\
	\endfoot
	\bottomrule
	\caption{Example Models}\\%
	\endlastfoot%
	\label{table:Example Interpretations Exercise}%
	%\begin{tabular}{ l l l l } %p{2in} p{2in} %\begin{tabular}{ p{1in} l l } %p{2.2in} p{2in}
	%\toprule
	%&\textbf{Symbol} & \multicolumn{2}{c}{\textbf{Interpretation}} \\ \cmidrule(l){3-4}
	%& & \textbf{Pos Int} & \textbf{States} \\
	%\midrule 
	{Universe:} & & The set of positive integers & The set of US states (2024) \\ \addlinespace[.25cm]
	{Sent. Let.:}& A&$\True$&$\False$\\
	& B&$\True$&$\False$\\
	& C&$\False$&$\True$\\
	& D&$\True$&$\False$\\
	& E&$\True$&$\False$\\
	& G&$\False$&$\True$\\ \addlinespace[.25cm]
	{Constants:}&$\constant{a}$&1&Louisiana\\
	&$\constant{b}$&9&Maine\\
	&$\constant{c}$&72&Georgia\\
	&$\constant{d}$&3&Nebraska\\
	&$\constant{e}$&1&New Mexico\\
	&$\constant{f}$&2&Texas\\ \addlinespace[.25cm]
	{1-place:}&$\Ap{'}$&all pos int&Midwestern\\
	&$\Bp{'}$&empty set&name with $>5$ letters\\
	&$\Cp{'}$&even&Coastal\\
	&$\Dp{'}$&odd&one of original 13\\
	&$\Ep{'}$&prime&\{Ohio\}\\
	&$\Gp{'}$&multiple of 7&\{Ohio, Alabama\}\\ \addlinespace[.25cm]
	{2-place:}&$\Ap{''}$&first $>$ second&share a border\\
	&$\Bp{''}$&are equal&first is north of second\\
	&$\Cp{''}$&first = 2 times second&first $>$ second (area)\\
	&$\Dp{''}$&sum of them equals 7&first $>$ second (population)\\
	&$\Ep{''}$&first $<$ second&first is west of second\\
	&$\Gp{''}$&are relatively prime&both coastal, or neither\\ \addlinespace[.25cm]
	{3-place:}&$\Ap{'''}$&all equal&all same population\\
	&$\Bp{'''}$&first $<$ second $<$ third&first is north of others\\
	&$\Cp{'''}$&all odd or all even&first $>$ second $>$ third (area)\\
	&$\Dp{'''}$&first + second = third&first + second $>$ third (area)\\
	&$\Ep{'''}$&first $\times$ second = third&first is west of the others\\
	&$\Gp{'''}$&are all relatively prime& at least two coastal \\
	%\bottomrule
\end{longtable}
\caption{Two \GQL{} models}
\end{figure}

\notocsubsection{Quantificational Truth Problems}{ex:More Quantificational Truth Problems} For each sentence below say whether it's a quantificational truth. 
If so, prove it. 
If not, show it by giving a model $\IntA$ that makes it false.
\begin{multicols}{2} 
\begin{enumerate}
\item {$\horseshoe{\universal{\variable{x}}\universal{\variable{y}}\Hpp{\variable{x}}{\variable{y}}}{\universal{\variable{y}}\universal{\variable{x}}\Hpp{\variable{x}}{\variable{y}}}$}
\item {$\horseshoe{\existential{\variable{x}}\existential{\variable{y}}\Hpp{\variable{x}}{\variable{y}}}{\existential{\variable{y}}\existential{\variable{x}}\Hpp{\variable{x}}{\variable{y}}}$}
\item {$\horseshoe{\universal{\variable{x}}\existential{\variable{y}}\Hpp{\variable{x}}{\variable{y}}}{\existential{\variable{y}}\universal{\variable{x}}\Hpp{\variable{x}}{\variable{y}}}$}
\item {$\horseshoe{\existential{\variable{y}}\universal{\variable{x}}\Hpp{\variable{x}}{\variable{y}}}{\universal{\variable{x}}\existential{\variable{y}}\Hpp{\variable{x}}{\variable{y}}}$}
\item {$\universal{\variable{x}}\bparhorseshoe{\Ap{\variable{x}}}{\existential{\variable{y}}\parconjunction{\Hpp{\variable{x}}{\variable{y}}}{\Bp{\variable{y}}}}$}
\item {$\existential{\variable{y}}\bparconjunction{\Ap{\variable{y}}}{\universal{\variable{z}}\parhorseshoe{\Bp{\variable{z}}}{\Hpp{\variable{y}}{\variable{z}}}}$}
\end{enumerate}
\end{multicols}
\begin{enumerate}[start=7]
\item {$\universal{\variable{x}}\bparconjunction{\existential{\variable{y}}\Hpp{\variable{x}}{\variable{y}}}{\conjunction{\negation{\Hpp{\variable{x}}{\variable{x}}}}{\universal{\variable{y}}\universal{z}\cparhorseshoe{\parconjunction{\Hpp{\variable{x}}{\variable{y}}}{\Hpp{\variable{y}}{\variable{z}}}}{\Hpp{\variable{x}}{\variable{z}}}}}$}
\end{enumerate}

\notocsubsection{Preliminary Dragnet Practice Problems}{ex:Preliminary Dragnet Practice Problems} 
\begin{enumerate}
\item If $\CAPPHI$ is $\horseshoe{\universal{\variable{z_{\integer{3}}}}\parconjunction{\Gpp{\variable{z_{\integer{1}}}}{\variable{z_{\integer{3}}}}}{\existential{\variable{x}}\Bppp{\variable{x}}{\variable{z_{\integer{3}}}}{\variable{y_{\integer{2}}}}}}{\parhorseshoe{\Al}{\universal{\variable{y_{\integer{2}}}}\Dpp{\variable{z_{\integer{3}}}}{\variable{y_{\integer{2}}}}}}$, what is
\begin{enumerate}
\item $\CAPPHI\constant{a}/\variable{z_{\integer{3}}}$
\item $\CAPPHI\constant{a}/\variable{y_{\integer{2}}}$
\item $\CAPPHI\variable{z_{\integer{3}}}/\variable{x}$
\item $\CAPPHI\constant{a}/\variable{x}$
\end{enumerate} 
\item If $\CAPPHI\variable{x}/\variable{y}$ is $\Dpp{\variable{x}}{\variable{x}}$, can you determine what $\CAPPHI$ is? If so, what is it? If not, why not?
\end{enumerate}

\notocsubsection{Dragnet Practice}{ex:Dragnet Practice} For each of the following statements, show whether it is true or false. 
Use Dragnet or Free Choice whenever it is helpful to do so.

Each of $\CAPTHETA$, $\CAPPSI$, and $\CAPPHI$ is a formula of \GQL{} whose only free variable is $\variable{x}$, and none contain the variables $\variable{y}$, $\variable{z}$ or $\variable{w}$. 

\begin{enumerate}
\item {$\universal{\variable{w}}\parconjunction{\CAPPHI{\variable{w}/\variable{x}}}{\CAPPSI{\variable{w}/\variable{x}}}\sdtstile{}{}\horseshoe{\universal{\variable{x}}\CAPPHI}{\universal{\variable{x}}\CAPPSI}$}
\item {$\horseshoe{\universal{\variable{x}}\CAPPHI}{\universal{\variable{x}}\CAPPSI}\sdtstile{}{}\horseshoe{\universal{\variable{y}}\CAPPHI{\variable{y}/\variable{x}}}{\universal{\variable{z}}\CAPPSI{\variable{z}/\variable{x}}}$}
\item {$\horseshoe{\universal{\variable{x}}\CAPPHI}{\universal{\variable{x}}\CAPPSI}\sdtstile{}{}\existential{\variable{x}}\parconjunction{\CAPPHI}{\CAPPSI}$}
\item {$\universal{\variable{x}}\parhorseshoe{\CAPPHI}{\CAPPSI}\sdtstile{}{}\horseshoe{\universal{\variable{y}}\CAPPHI{\variable{y}/\variable{x}}}{\universal{\variable{z}}\CAPPSI{\variable{z}/\variable{x}}}$}
\item {$\horseshoe{\universal{\variable{y}}\CAPPHI{\variable{y}/\variable{x}}}{\universal{\variable{z}}\CAPPSI{\variable{z}/\variable{x}}}\sdtstile{}{}\universal{\variable{x}}\parhorseshoe{\CAPPHI}{\CAPPSI}$}
\item {$\horseshoe{\universal{\variable{y}}\CAPPHI{\variable{y}/\variable{x}}}{\universal{\variable{z}}\CAPPSI{\variable{z}/\variable{x}}}\text{, }\existential{\variable{x}}\parconjunction{\CAPPHI}{\CAPPSI}\sdtstile{}{}\universal{\variable{x}}\parhorseshoe{\CAPPHI}{\CAPPSI}$}
\item {$\text{If }\sdtstile{}{}\universal{\variable{x}}\parhorseshoe{\CAPPHI}{\CAPPSI}\text{, then }\sdtstile{}{}\universal{\variable{x}}\cparhorseshoe{\parconjunction{\CAPPHI}{\CAPTHETA}}{\parconjunction{\CAPPSI}{\CAPTHETA}}$}
\item {$\existential{\variable{y}}\parconjunction{\CAPPHI{\variable{y}/\variable{x}}}{\negation{\CAPPSI{\variable{y}/\variable{x}}}}\sdtstile{}{}\universal{\variable{y}}\parhorseshoe{\CAPPHI{\variable{y}/\variable{x}}}{\negation{\CAPPSI{\variable{y}/\variable{x}}}}$}
\item {$\negation{\universal{\variable{x}}\CAPPHI}\sdtstile{}{}\existential{\variable{x}}\negation{\CAPPHI}$}
\item {$\sdtstile{}{}\disjunction{\universal{\variable{x}}\parhorseshoe{\CAPPHI}{\CAPPSI}}{\universal{\variable{y}}\parhorseshoe{\CAPPHI{\variable{y}/\variable{x}}}{\negation{\CAPPSI{\variable{y}/\variable{x}}}}}$}
\end{enumerate}

%\theendnotes


%%%%%%%%%%%%%%%%%%%%%%%%%%%%%%%%%%%%%%%%%%%%%%%%%%
\chapter{Translations}\label{Translations}
%%%%%%%%%%%%%%%%%%%%%%%%%%%%%%%%%%%%%%%%%%%%%%%%%%
% \AddToShipoutPicture*{\BackgroundPicC}

%%%%%%%%%%%%%%%%%%%%%%%%%%%%%%%%%%%%%%%%%%%%%%%%%%
\section{SL Applications}\label{SLApplications}
%%%%%%%%%%%%%%%%%%%%%%%%%%%%%%%%%%%%%%%%%%%%%%%%%%


Sentences in English have meanings.
Declarative sentences say something about the world, and in virtue of their various meanings they are either true or false.
By contrast, the sentences of \GSL{} and \GQL{} are, in themselves, meaningless strings of symbols.
Any meaning ascribed to \GSL{} and \GQL{} sentences must come from a model.

The models in Chapter \ref{sententiallogic} do not distinguish between different sentences with the same truth value, because for the purposes of assessing logical truth in \GSL{}, truth value is all that matters.
But if we want to translate English into \GSL{} we must specify the models less directly: by associating English sentences with sentence letters.

\subsection{Using \GSL{} Models for Translation}\label{GSLTranslationModels}

Let's say we have some declarative sentences of English that we would like to translate and model in \GSL{}.
First, we look for the smallest declarative clauses, i.e. the \mention{atomic} parts, and we associate each of these with a sentence letter.
We decide whether each of these atoms is true or false, and then define a model $\IntA$ that makes the appropriate assignment to the corresponding sentence letter.
If truth values are assigned according to what we know about reality, then $\IntA$ is a \mention{model} of reality, or at least that part of reality described by the English sentences.
Alternatively, we can assign truth values so that they do not match reality, in which case $\IntA$ is a model of a non-actual state of affairs.
Finally, we can assign values to sentences about which the facts are unknown, in which case $\IntA$ is a model of a hypothetical state of affairs.

English sentences are not all equally suitable for translation to \GQL{}.
Ideal English sentences are unambiguous, not vague, and do not have a truth value that changes over time.
More practically, we can also model sentences that are unambiguous in the appropriate context, specific enough to have determinate truth values, and whose truth values do not change during some period of interest.
One source of well-behaved sentences is mathematics.
For other sentences we can usually supply enough context to meet our criteria.

To illustrate, we might say that \mention{Texas is a US state,} is a true sentence of English.
We mean that Texas is a US state at the time the sentence is uttered.
By convention, English sentences in the present tense are assessed relative to the time of utterance.
But without such context matters aren't so clear.
The sentence is false of Texas in 1844 and true of Texas now, so its truth value has varied over time.
The sentence \mention{Texas was a US state at noon CDT on May 1, 1983,} is less ambiguous, and does not require any context-sensitive judgment about time to assess.

In practice, however, we make background assumptions about what is relevant, who is being discussed, the time of assessment, and so on.
E.g., \mention{Washington was elected President,} has an obvious and natural interpretation, one that's distinct from \mention{Joan Washington was elected President of the Montrose Dog Walkers Association.}

Note that we speak of translating \emph{sentences} and not, e.g., propositions.
There is a controversy in the philosophy of logic over which objects are the most fundamental bearers of truth value: sentences, propositions, judgments, statements, etc.
Without weighing in on that debate, we prefer sentences over the alternatives for a logic textbook.
We argued in Chapter \ref{introduction} that there is no uncontroversial definition of a sentence of English.
Even so, there is less dispute about what a sentence is than what propositions, judgments, or statements are.

There is difference between sentence tokens and sentence types.
We prefer not to deal with the difficulties associated with that distinction in this text.
We assume that all sentence tokens of a type have the same truth value, but that isn't true in general.\footnote{
	See \citealt{Grandy1993} for details.
}

\subsection{\GSL{} Translation Keys}\label{GSLTranslationKeys}

We use a \idf{translation key} to associate atomic English sentences with \GSL{} sentence letters.
A translation key is different from a model, but it determines which sentence letter assignments a \GSL{} model has to make.
A complete model also needs the truth value of each sentence, which isn't supplied by the translation key.
Once we've mapped atomic sentences of English to sentence letters we can use the key to translate complex English sentences into \GSL{}.
Complex sentences are composed of (i) atomic sentences and (ii) logical connectives of English.

When is a sentence of English simple enough to be considered \mention{atomic}?
In putting together a key we should capture as much of the logical structure of the English as is practical and useful.  
All else being equal, translation keys that hide logical structure in a sentence letter are deficient.
For example, the following is not ordinarily a good candidate for sentence letter assignment:

\begin{smenumerate}
	\item Mares eat oats and does eat oats and little lambs eat ivy.
\end{smenumerate}

\noindent{}The word \mention{and} plays a truth-functional role in this sentence, analogous to the role that \mention{$\WEDGE$} plays in \GSL{}.
A better translation key would assign each of \mention{Mares eat oats}, \mention{Does eat oats}, and \mention{Little lambs eat ivy} its own sentence letter.
The original sentence of English can then be expressed as a conjunction in \GSL{}.
In general, the goal of translation to \GSL{} is to replace the truth-functional connectives of English sentences with appropriate logical connectives from \GSL{}.


\begin{description}[itemsep=0em]
	\item[Translation Key:] \hfill{} 
	\begin{description}[itemsep=0em]
		\item $\Ml$: Mares eat oats.
		\item $\Dl$: Does eat oats.
		\item $\Ll$: Little lambs eat ivy.
	\end{description} 
\end{description}

\subsection{Connectives}\label{GSLConnectives and Trans}

Before considering other examples we need to say a little more about connectives.\index{logical connectives} 
In Chapter \ref{sententiallogic} (in definition \pmvref{Basic Symbols of GSL}) we listed the \sq{logical connectives} of \GSL{}.
We've seen the role they play in constructing sentences of \GSL{} and in fixing their truth values in a model. 
But we haven't said anything about what connectives are in general. 

The essential idea of a sentential \idf{connective}---whether a \sq{logical} one in some formal language or an expression in a natural language like English---is that it's a part of language that can be used to combine one or more sentences of the language into a new sentence.\footnote{Other connectives play a subtler role, as with the quantifiers of QL.
In QL quantifiers can be used to combine predicates to make a more complex predicate.} 
We say that one connective is within the \idf{scope} of a second connective iff the first connective is within a sentence directly connected with the second connective.
The \idf{main connective} is the one connective of the sentence that's not within the scope of any other connectives. 

A connective is \niidf{truth-functional}\index{connective!truth-functional} iff the truth value of every new sentence formed by that connective depends solely on the truth value of the constituent sentences. 
(If we're talking about a connective in a formal language like \GSL{}, we mean the truth value of a sentence \emph{in a model}.)
Not all connectives of English are truth-functional.
In particular, there are modal connectives---such as \mention{necessarily} and \mention{possibly}---that defy truth-functional characterization.
For example, the sentence

\begin{menumerate}
	\item Necessarily, bachelors are unmarried.
\end{menumerate}

\noindent{}seems plausibly true.  If we replace \mention{bachelors are unmarried} with another true claim, however, the result is not so plausible:

\begin{menumerate}
	\item Necessarily, there is life on earth.
\end{menumerate}

\noindent{}Both \mention{bachelors are unmarried} and \mention{There is life on earth} are true.
But it is easier to imagine all life on earth perishing than for bachelors to be married.
That there is life on earth is contingent.
The truth of a sentence with a modal main connective depends on more than the truth value of the part of the sentence governed by the modal connective.  

By contrast, the five logical connectives of \GSL{} are all truth-functional.
This disparity means that \GSL{} is not capable of capturing all of the logical structure of English.
The best it can do is to approximate certain features of English.
Some formal languages have the resources to capture the logical structure of modal claims.
We introduce one such language in Chapter \ref{furtherdirections}.
We hasten to add, however, that \GSL{} has sufficient resources to render many English sentences without significantly distorting their basic meaning.

\subsection{Conjunctions and Negations}\label{Translating GSLConjunctionsAndNegations}

A logical connective of \GSL{} is a suitable translation of a (truth-functional) connective of English iff they express the same truth function (see section \pmvref{Truth Functions Truth Tables and Boolean Operators}).
That is, iff the new sentences formed by those connectives share the same truth value whenever the constituent sentences that make them up share the same truth values.
We've already mentioned one such pair of corresponding connectives: \mention{and} and \mention{$\WEDGE$}.
The sentence

\begin{menumerate}
	\item\label{GSLTransConjunction1} The sun is shining \emph{and} the birds are chirping.
\end{menumerate}

\noindent{}is true \Iff the following two sentences are also true:

\begin{menumerate}
	\item The sun is shining.
	\item The birds are chirping.
\end{menumerate}

\noindent{}The same holds of other conjunctions in English.  The \GSL{} connective \mention{$\WEDGE$} is a good translation of \mention{and} most of the time.
Sometimes the structure of English conjunctions is more hidden, as in the following example:

\begin{menumerate}
	\item\label{GSLTransConjunction2} Nathanael Green \emph{and} `Light Horse Harry' Lee were officers in the Continental Army.
\end{menumerate}

\noindent{}A translation key should not break the sentence at the word \mention{and} and give the \CAPS{lhs}---\mention{Nathanael Green}---its own sentence letter.
The \CAPS{lhs} is not a declarative sentence.
It's clear that \ref{GSLTransConjunction2} expresses the conjunction of the following two sentences:

\begin{menumerate}
	\item Nathanael Green was an officer in the Continental Army.
	\item `Light Horse Harry' Lee was an officer in the Continental Army.
\end{menumerate}

\noindent{}It is appropriate to map each of these latter sentences to a sentence letter---say, $\Nl$ and $\Ll$, respectively---and express \mention{Nathanael Green and `Light Horse Harry' Lee were officers in the Continental Army} as a conjunction of these sentence letters---e.g., $\parconjunction{\Nl}{\Ll}$.

Other words of English also play the role of conjunction. Consider:

\begin{menumerate}
	\item\label{GSLTransConjunction3} Mary wants to drive her mother's car, \emph{but} she is \emph{not} old enough.
\end{menumerate}

\noindent{}The word \mention{but} connects two independent claims.
The sentence as a whole is true \Iff the \CAPS{lhs} and the \CAPS{rhs} are each true.
So \mention{but} plays the same truth-functional role as \mention{$\WEDGE$} in \GSL{}.

The word \mention{not} is another connective; it corresponds to the \mention{$\NEGATION$} of \GSL{}. 
The words ``not'' and ``no'' are typically translated with $\NEGATION$. 

Negations in English are not always so obvious.
``Innumerable'' is not numerable, ``unmistakable'' is not mistakable, ``never'' is not ever, ``immoral'' is not moral, ``clueless'' is to have no clue, and so on. 
Yet while the prefix ``in-'' is often negation, as in ``inevitable'', it is also often a direction, as in ``influx''.
The latter case is not one of negation.

Notice that in the \CAPS{rhs} of \ref{GSLTransConjunction3}, context indicates that \mention{she} is Mary, and that mention of her being not \mention{old enough} is about the fact that she is not old enough \emph{to drive her mother's car}.
English is, in this respect, more efficient than \GSL{}.
To translate this sentence into \GSL{} we need something like:

\begin{description}[itemsep=0em]
	\item[Translation Key:] \hfill{} 
	\begin{description}[itemsep=0em]
		\item $\Al$: Mary wants to drive her mother's car.
		\item $\Ol$: Mary is old enough to drive her mother's car.
	\end{description} 
\end{description}

\noindent{}We take out the \mention{not} because we can represent that with \mention{$\NEGATION$}.
We render the original sentence in \GSL{} as $\parconjunction{\Al}{\negation{\Ol}}$.
This is indistinguishable from the result of translating

\begin{menumerate}
	\item Mary wants to drive her mother's car \emph{and} she is not old enough to drive her mother's car.
\end{menumerate}

\noindent{}In English we typically say \mention{$\PHI$ but $\THETA$} instead of \mention{$\PHI$ and $\THETA$} to signal to listeners that they are about to hear something surprising.
They may expect that $\PHI$ and $\THETA$ are an unlikely pairing.
This difference from the word \mention{and} cannot be captured in \GSL{}---it is lost in translation.
\GSL{} translations are approximations of English, and as such they can't always preserve the original meaning.

The word \mention{and} can do more than serve as a conjunction.
Sometimes the order of the conjuncts affects meaning.
Compare:

\begin{menumerate}
	\item\label{GSLTransConjunction4} John took off his clothes and went to bed.
\end{menumerate}
\begin{menumerate}
	\item\label{GSLTransConjunction5} John went to bed and took off his clothes.
\end{menumerate}

The order of the conjuncts suggests that John performed these actions in a different order.
Yet if we translate \ref{GSLTransConjunction4} and \ref{GSLTransConjunction5} into \GSL{} the results are truth-functionally equivalent.
We think this equivalence is defensible, because \ref{GSLTransConjunction4} and \ref{GSLTransConjunction5} don't explicitly describe the order of events.
We tend to read order into them because it's usual in conversation to recount events in chronological order.
This is not a requirement, however.
There is nothing contradictory about the following:

\begin{menumerate}
	\item John went to bed and took off his clothes, but I don't know in which order.
\end{menumerate}

\begin{table}
	\renewcommand{\arraystretch}{1.5}%
	\begin{center}
		\begin{tabular}{ l l } %p{2.2in} p{2in}
			\toprule
			\textbf{English} & \textbf{\GSL{}} \\ 
			\midrule
			$\PHI$ and $\THETA$ & $\conjunction{\CAPPHI}{\CAPTHETA}$ \\
			both $\PHI$ and $\THETA$ & $\conjunction{\CAPPHI}{\CAPTHETA}$ \\
			$\PHI$, but $\THETA$ & $\conjunction{\CAPPHI}{\CAPTHETA}$ \\
			$\PHI$, but not $\THETA$ & $\conjunction{\CAPPHI}{\negation{\CAPTHETA}}$ \\
			not $\PHI$, but $\THETA$ & $\conjunction{\negation{\CAPPHI}}{\CAPTHETA}$ \\
			\bottomrule
		\end{tabular} 
		\caption{Translations for Common English Conjunctions}
		\label{TransTableD} 
	\end{center}
\end{table}

\subsection{Disjunctions}\label{Translating GSLDisjunctions}

Another connective of English is the word \mention{or}:

\begin{menumerate}
	\item \emph{Either} the tigers will get us \emph{or} the lions will.
\end{menumerate}

\noindent{}The sentence as a whole is true \Iff a clause on either side of the \mention{or} is true.
This matches the truth-function of the \mention{$\VEE$} connective in \GSL{}.

\begin{description}[itemsep=0em]
	\item[Translation Key:] \hfill{} 
	\begin{description}[itemsep=0em]
		\item $\Il$: The tigers will get us.
		\item $\Ll$: The lions will get us.
	\end{description} 
\end{description}

\noindent{}The resulting translation: $\pardisjunction{\Il}{\Ll}$.
We again use context to fill out the sentence \mention{the lions will [get us]}.  

It might not seem quite right to translate \mention{or} as \mention{$\VEE$}. 
In \GSL{} $\pardisjunction{\CAPPHI}{\CAPTHETA}$ is true when both $\CAPPHI$ and $\CAPTHETA$ are true on some model. 
But sometimes in English when we say \mention{$\PHI$ or $\THETA$} we mean that one or the other of $\PHI$ and $\THETA$ is true---but not both. 
A disjunction in which both disjuncts can be true is \niidf{inclusive}\index{inclusive disjunction}, while a disjunction in which only one disjunct can be true is \niidf{exclusive}.\index{exclusive disjunction}
Should \mention{$\PHI$ or $\THETA$} be considered exclusive?
Is it incorrect to translate it as $\pardisjunction{\CAPPHI}{\CAPTHETA}$?
 
These are complicated questions that we cannot fully answer here. 
We believe, however, that \mention{or} is typically inclusive in English and so best translated as \mention{$\VEE$}.
Only in certain conversational contexts is the exclusive disjunction clearly intended.
We motivate our stance with a few examples.\footnote{This discussion borrows from Smith \citeyear[117--22]{Smith2012}.}
 	
One case in which \mention{or} sounds exclusive involves disjuncts that cannot possibly both be true. 
I might say \q{Lydia either took a boat or a plane}. 
It's not physically possible for Lydia to travel by distinct modes of travel at the same time.
However, we must distinguish the meaning of the sentence from what we otherwise know to be possible.
We know that it's impossible to travel by boat and by plane at the same time.
It's tempting to assimilate that fact into the sentence's meaning, but that leap is unjustified. 
Someone could understand \q{Lydia either took a boat or a plane} and agree that it's true, but also think that both disjuncts are true.
Perhaps Lydia's journey was disjointed and involved both modes of travel.
To conclude that this disjunction is exclusive requires more information than the sentence itself conveys.

There are two cases in which it's more plausible that \mention{or} is exclusive.  
The first case involves commands or rules. 
If you are at a fancy restaurant and a waiter says, \q{You may have the soup or a salad}, it's typically understood that you may have one or the other, but not both.

\begin{table}
	\renewcommand{\arraystretch}{1.5}%
	\begin{center}
		\begin{tabular}{ l l } %p{2.2in} p{2in}
			\toprule
			\textbf{English} & \textbf{\GSL{}} \\ 
			\midrule
			$\PHI$ or $\THETA$ & $\disjunction{\CAPPHI}{\CAPTHETA}$ \\
			either $\PHI$ or $\THETA$ & $\disjunction{\CAPPHI}{\CAPTHETA}$ \\
			$\PHI$ or $\THETA$, but not both & $\conjunction{\pardisjunction{\CAPPHI}{\CAPTHETA}}{\negation{\parconjunction{\CAPPHI}{\CAPTHETA}}}$ \\
			\bottomrule
		\end{tabular} 
		\caption{Translations for Common English Disjunctions}
		\label{TransTableC} 
	\end{center}
\end{table}

The second case involves elliptical clauses. 
The phrase \mention{$\PHI$ or $\THETA$} can be short for \mention{$\PHI$ or $\THETA$, but not both}. 
If so, then the intended message is exclusive. 
But this isn't a case in which the \mention{or} itself expresses exclusive disjunction. 
The \mention{or} plus the tacit phrase \mention{but not both} together express exclusive disjunction.

Special cases like these, which are rare but salient, might lead you to think that disjunctions in English are sometimes exclusive.
However, in these cases there are context clues indicating their exclusive character, separate from the meaning of the relevant sentences.
We know as a matter of cultural familiarity that restaurants don't offer \emph{both} soup \emph{and} salad unless we pay extra.  
When intergalactic visitors discover Earth and visit our restaurants, we recommend that waiters make the exclusivity of the disjunction explicit.
Otherwise there could be an interstellar diplomatic incident!
They can even use \GSL{} to do so:
 	
 \begin{menumerate}
 	\item The intergalactic visitor may have the soup \emph{or} a salad, \emph{but} \emph{not} both.
 \end{menumerate}
 	
\noindent{}This sentence has several truth-functional connectives, so we must be careful to understand which connectives govern which.
The \mention{but} governs everything else, so the whole sentence is a conjunction.
The visitor may have the soup or a salad, \emph{and} he may not have both the soup and a salad.
The translation key:

\begin{description}[itemsep=0em]
	\item[Translation Key:] \hfill{} 
	\begin{description}[itemsep=0em]
		\item $\Pl$: The visiting intergalactic visitor may have the soup.
		\item $\Dl$: The visiting intergalactic visitor may have a salad.
	\end{description} 
\end{description}

\noindent{}First we translate the \CAPS{lhs} of the \mention{but}: $\pardisjunction{\Pl}{\Dl}$.
If we translate the \CAPS{rhs} without the \mention{not}---i.e., \mention{the intergalactic visitor may have both soup and a salad}---we get: $\parconjunction{\Pl}{\Dl}$.
We negate the result to account for the \mention{not}: $\negation{\parconjunction{\Pl}{\Dl}}$.
Putting it all together, the resulting translation is: $\parconjunction{\pardisjunction{\Pl}{\Dl}}{\negation{\parconjunction{\Pl}{\Dl}}}$.

\subsection{Conditionals and Biconditionals}\label{Translating GSLConditionals}

Another connective of English is the \emph{conditional}.
Conditionals are expressible in many ways, but perhaps the most distinctive way is \mention{if $\ldots$ then $\ldots$.}

\begin{menumerate}
	\item If George Washington crosses the Delaware River then the Hessians will be defeated.
\end{menumerate}

\noindent{}Is the conditional of English a truth-functional connective?
This is a disputed topic that has yet to be resolved.
SL has only truth-functional connectives, so if the conditional is not truth-functional then \mention{$\HORSESHOE$} cannot fully capture its meaning.
We can show, however, that \mention{$\HORSESHOE$} is the best truth-functional representation of the conditional.
As we discuss translations the reader will get a sense of how well the model fits English.

Let's consider a few examples.
An \GSL{} sentence of the form $\parhorseshoe{\CAPPHI}{\CAPTHETA}$ is false on a model $\IntA$ \Iff $\IntA$ makes $\CAPPHI$ true and $\CAPTHETA$ false.
That gives us the correct assessment of the English sentence:

\begin{menumerate}
	\item If $2+2=4$ then $4+4=6$.
\end{menumerate}

\noindent{}On all other combinations of truth values for $\CAPPHI$ and $\CAPTHETA$, \GSL{} sentences of the form $\parhorseshoe{\CAPPHI}{\CAPTHETA}$ are true.
Consider a case in which both the \CAPS{lhs} and the \CAPS{rhs} of a conditional are false.
Some such conditionals are true:

\begin{menumerate}
	\item\label{GSLTransGW1} If George Washington landed on the moon then George Washington landed on the moon.
\end{menumerate}

\noindent{}This sentence is not merely true. It's a logical truth.
Other instances are more dubious: 

\begin{menumerate}
	\item\label{GSLTransGW2} If Soviet cosmonauts landed on the moon in 1968, then George Washington was never elected President.
\end{menumerate}

\noindent{}No truth-functional connective allows us to distinguish the last two examples.
In both cases we can only refer to the truth values of each side: the \CAPS{lhs} and the \CAPS{rhs} are false.
In order to admit \ref{GSLTransGW1} as true, we must also admit \ref{GSLTransGW2}.
This is one price we must pay for truth-functional conditionals.

Here is a sentence in which both the \CAPS{lhs} and the \CAPS{rhs} are true:

\begin{menumerate}
	\item\label{GSLTransGW3} If George Washington crossed the Delaware River then George Washington crossed the Delaware River.
\end{menumerate}

\noindent{}As with \ref{GSLTransGW1} this is not merely true, it's a logical truth.
Consider an \GSL{} model that makes $\CAPPHI$ false and $\CAPTHETA$ true, and hence makes $\parhorseshoe{\CAPPHI}{\CAPTHETA}$ true.
We can replace the \CAPS{lhs} of \ref{GSLTransGW3} with a false conjunction and the result is still true:

\begin{menumerate}
	\item If George Washington crossed the Delaware River and Thomas Jefferson invented bifocals, then George Washington crossed the Delaware River.
\end{menumerate}

\noindent{}The true inventor of bifocals was probably Benjamin Franklin.
Even though the conjunction on the \CAPS{lhs} is false, the sentence is still a logical truth.
The corresponding \GSL{} translation is \CAPS{tft}: $\parhorseshoe{\parconjunction{\Gl}{\Bl}}{\Gl}$.
So, while the \mention{$\HORSESHOE$} is not a perfect match for the English language conditional, it is the best truth-functional translation.

\begin{table}
	\renewcommand{\arraystretch}{1.5}%
	\begin{center}
		\begin{tabular}{ l l } %p{2.2in} p{2in}
			\toprule
			\textbf{English} & \textbf{\GSL{}} \\ 
			\midrule
			if $\PHI$, then $\THETA$ & $\horseshoe{\CAPPHI}{\CAPTHETA}$ \\
			$\PHI$ only if $\THETA$ & $\horseshoe{\CAPPHI}{\CAPTHETA}$ \\
			$\PHI$ if $\THETA$ & $\horseshoe{\CAPTHETA}{\CAPPHI}$ \\
			$\PHI$ provided that $\THETA$ & $\horseshoe{\CAPTHETA}{\CAPPHI}$ \\
			provided $\PHI$, $\THETA$ & $\horseshoe{\CAPPHI}{\CAPTHETA}$ \\
			$\PHI$ assuming that $\THETA$ & $\horseshoe{\CAPTHETA}{\CAPPHI}$ \\
			assuming $\PHI$, $\THETA$ & $\horseshoe{\CAPPHI}{\CAPTHETA}$ \\
			for $\PHI$, it's necessary that $\THETA$ & $\horseshoe{\CAPPHI}{\CAPTHETA}$ \\
			for $\PHI$, it's sufficient that $\THETA$ & $\horseshoe{\CAPTHETA}{\CAPPHI}$ \\
			$\PHI$ if and only if $\THETA$ & $\triplebar{\CAPPHI}{\CAPTHETA}$ \\
			for $\PHI$ it's necessary and sufficient that $\THETA$ & $\triplebar{\CAPPHI}{\CAPTHETA}$ \\
			$\PHI$ just when $\THETA$ & $\triplebar{\CAPPHI}{\CAPTHETA}$ \\
			$\PHI$ just in case $\THETA$ & $\triplebar{\CAPPHI}{\CAPTHETA}$ \\
			\bottomrule
		\end{tabular}% 
		\caption{Translations for Common English Conditionals and Biconditionals}
		\label{TransTableA}
	\end{center}
\end{table}

Biconditionals in English are closely related to conditionals, so they're subject to some of the same worries.
Nevertheless, we treat them as truth-functional connectives for the purpose of translating them into \GSL{}:

\begin{menumerate}
	\item Ruth may play outside if and only if she cleans her room.
\end{menumerate}

	\begin{description}[itemsep=0em]
		\item[Translation Key:] \hfill{} 
		\begin{description}[itemsep=0em]
			\item $\Hl$: Ruth may play outside.
			\item $\Cl$: Ruth cleans her room.
		\end{description} 
	\end{description}

\noindent{}With this key we translate the sentence as: $\partriplebar{\Hl}{\Cl}$.
We use \mention{$\TRIPLEBAR$} to translate English biconditionals.

Sometimes the word \mention{if} is used to express a biconditional:

\begin{menumerate}
	\item\label{RuthIf} Ruth may play outside if she cleans her room.
\end{menumerate}

\noindent{}But what if she doesn't clean her room?
Presumably she will not be allowed to play outside.
Yet this is not the literal meaning of the word \mention{if}.
We use context to supply the implicit \mention{but not otherwise} to the sentence.
It is well-known that parents use punishments and rewards to shape child behavior, so we rightly interpret \ref{RuthIf} as shorthand.

In general it is our policy is to translate what each sentence says literally.
If it is important for an argument to articulate what is being added by context, that should be specified explicitly.

\subsection{Further Examples}\label{GSLTransExamples}

Let's look at a few more translations.

\begin{majorILnc}{\LnpEC{GSLTranslationExampleB}} %implicature
	\begin{menumerate}
		\item\label{GSLTransSentenceE} Either Lydia flew or both Jackson and Helen took off early.
	\end{menumerate} 
	The first step is to identify the main connective of the sentence: \mention{either $\ldots$ or}.
	We translate \mention{either $\ldots$ or} as $\VEE$: 
	\begin{menumerate}
		\item\label{GSLTransSentenceF} (Lydia flew) $\VEE$ (both Jackson and Helen took off early).
	\end{menumerate}
	Notice that the word ``both'' clarifies the logical structure of the sentence.
	Without it, the sentence is, ``Either Lydia flew or Jackson and Helen took off early.''
	This is ambiguous, and could be read as saying that either Lydia or Jackson flew. 
	
	Next we identify and translate the main connectives in the connected clauses. 
	There are no connectives in \mention{Lydia flew}, so there is nothing to do with it. 
	There is one connective in \mention{both Jackson and Helen took off early}, \mention{both $\ldots$ and}. 
	We translate this as \mention{$\WEDGE$}:  
	\begin{menumerate}
		\item\label{GSLTransSentenceG} (Lydia flew) $\VEE$ ((Jackson took off early) $\WEDGE$ (Helen took off early)).
	\end{menumerate}
	\begin{description}[itemsep=0em]
		\item[Translation Key:] \hfill{} 
		\begin{description}[itemsep=0em]
			\item[] $\Nl$: Lydia flew.
			\item[] $\Jl$: Jackson took off early. 
			\item[] $\Hl$: Helen took off early. 
		\end{description}
	\end{description}
	\begin{menumerate}
		\item\label{GSLTransSentenceH} $\disjunction{\Nl}{\parconjunction{\Jl}{\Hl}}$
	\end{menumerate}
\end{majorILnc}

\begin{majorILnc}{\LnpEC{GSLTranslationExampleC}}
	\begin{menumerate}
		\item\label{GSLTransSentenceI} If Lydia got the job and Jackson didn't, then Lydia will take off tomorrow and Helen will have to come in.
	\end{menumerate}
	We identify the main connective, \mention{if $\ldots$ then}. 
	This is translated as \mention{$\rightarrow$}: 
	\begin{menumerate}
		\item\label{GSLTransSentenceJ} (Lydia got the job and Jackson didn't) $\HORSESHOE$ (Lydia will take off tomorrow and Helen will have to come in).
	\end{menumerate}
	Next we look at the \CAPS{LHS} of the conditional, \mention{Lydia got the job and Jackson didn't}. 
	The main (and only) connective of this sentence is \mention{and}, which we translate as \mention{$\WEDGE$}: 
	\begin{menumerate}
		\item\label{GSLTransSentenceK} ((Lydia got the job) $\WEDGE$ (Jackson didn't get the job)) $\HORSESHOE$ (Lydia will take off tomorrow and Helen will have to come in).
	\end{menumerate}

	\noindent{}The right-hand conjunct \mention{Jackson didn't} is an elliptical clause. 
	Lydia got the job and Jackson didn't \emph{get the job}.
	We make this tacit phrase explicit by putting it in brackets. 
	The clause \mention{Jackson didn't} has a negation in it. It combines \mention{not} with \mention{Jackson got the job}. 
	We finish the translation by translating the conjunction on the \CAPS{RHS} of the sentence: 
	\begin{menumerate}
		\item\label{GSLTransSentenceL} ((Lydia got the job) $\WEDGE$ ( $\NEGATION$ (Jackson got the job))) $\HORSESHOE$ ((Lydia will take off tomorrow) $\WEDGE$ (Helen will have to come in)).
	\end{menumerate}
	There are no more connectives, so we are ready to construct a translation key. 
	\begin{description}[itemsep=0em]
		\item[Translation Key:] \hfill{} 
		\begin{description}[itemsep=0em]
			\item[] $\Nl$: Lydia got the job.
			\item[] $\Jl$: Jackson got the job. 
			\item[] $\Il$: Lydia will take off tomorrow.
			\item[] $\Hl$: Helen will have to come in late.  
		\end{description}
	\end{description}
	\begin{menumerate}
		\item\label{GSLTransSentenceM} $\horseshoe{\parconjunction{\Nl}{\negation{\Jl}}}{\parconjunction{\Il}{\Hl}}$
	\end{menumerate}
\end{majorILnc}


\begin{majorILnc}{\LnpEC{GSLTranslationExampleE}}
	Some connectives in English do not correspond to any single \GSL{} connective.
	However, we can express any truth-function using some combination of the \GSL{} connectives that we \emph{do} have.
	For example: 
	\begin{menumerate}
		\item\label{GSLTransSentenceU} Neither Lydia nor Helen were late.
	\end{menumerate} 
	The connective is \mention{neither $\ldots$ nor}, and it joins \mention{Lydia [was late]} and \mention{Helen [was] late}. 
	There is no single \GSL{} connective that has the same truth-function as \mention{neither $\ldots$ nor}.\footnote{%
		But recall the logical connective \CAPS{NOR}, discussed at the end of section \mvref{Disjunctive Normal Form}. 
		\CAPS{NOR} correctly translates \mention{neither $\ldots$ nor}, but isn't part of \GSL{}. 
	} 
	Recall theorem \mvref{Truth-functional Expressive Completeness of GSL}, which says that any truth-functional connective can be expressed in \GSL{}. 
	It follows that there is some combination of \GSL{} connectives we can use to translate \mention{neither $\PHI$ nor $\THETA$}.
	There are two equivalent translations: $\negation{\pardisjunction{\CAPPHI}{\CAPTHETA}}$ and $\conjunction{\negation{\CAPPHI}}{\negation{\CAPTHETA}}$.
	\begin{table}
		\renewcommand{\arraystretch}{1.5}%
		\begin{center}
			\begin{tabular}{ l l } %p{2.2in} p{2in}
				\toprule
				\textbf{English} & \textbf{\GSL{}} \\ 
				\midrule
				not $\PHI$ & $\negation{\CAPPHI}$ \\
				it's not the case that $\PHI$ & $\negation{\CAPPHI}$ \\
				$\PHI$ unless $\THETA$ & $\horseshoe{\negation{\CAPTHETA}}{\CAPPHI}$ \\
				$\PHI$ unless $\THETA$ & $\disjunction{\CAPTHETA}{\CAPPHI}$ \\
				unless $\PHI$, $\THETA$ & $\horseshoe{\negation{\CAPPHI}}{\CAPTHETA}$ \\
				$\PHI$ if not $\THETA$ & $\horseshoe{\negation{\CAPTHETA}}{\CAPPHI}$ \\
				neither $\PHI$ nor $\THETA$ & $\negation{\pardisjunction{\CAPPHI}{\CAPTHETA}}$ \\
				& $\conjunction{\negation{\CAPPHI}}{\negation{\CAPTHETA}}$ \\
				not both $\PHI$ and $\THETA$ & $\negation{\parconjunction{\CAPPHI}{\CAPTHETA}}$ \\
				& $\disjunction{\negation{\CAPPHI}}{\negation{\CAPTHETA}}$ \\
				\bottomrule
			\end{tabular}
			\caption{Translations for Common English Negations and Complex Connectives}
			\label{TransTableF} 
		\end{center}
	\end{table}  
	So we have: 
	\begin{menumerate}
		\item\label{GSLTransSentenceV} $\NEGATION$ ((Lydia was late) $\VEE$ (Helen was late)).
	\end{menumerate} 
	With the key: 
	\begin{description}[itemsep=0em]
		\item[Translation Scheme:] \hfill{} 
		\begin{description}[itemsep=0em]
			\item $\Nl$: Lydia was late.
			\item $\Hl$: Helen was late. 
		\end{description} 
	\end{description}
	The final translation of sentence \ref{GSLTransSentenceU} is: 
	\begin{menumerate}
		\item\label{GSLTransSentenceW} $\negation{\pardisjunction{\Nl}{\Hl}}$
	\end{menumerate} 
\end{majorILnc}

\noindent{}The tables in this chapter should be thought of as rough-and-ready guides. 
Although many particular uses of \mention{and} express conjunction, not all do. 
Sometimes \mention{and} doesn't function as a connective at all, e.g. as in \mention{it will be years and years before the trees bear fruit} \citep[107]{Smith2012}.
Other times \mention{and} functions as a connective, but expresses a conditional instead of a conjunction, e.g. \mention{study hard, and you will pass the exam} \citep[107]{Smith2012}.

To translate a connective from English appropriately you must first determine what is being expressed, drawing on context as necessary to resolve ambiguities.
For example, ``Maria and Paul are married,'' can convey the conjunction ``Maria is married and Paul is married,'' or it can express ``Maria and Paul are married to each other,'' in which case the `and' is not a conjunction, but is serving a different logical purpose to be discussed in the next section.

%%%%%%%%%%%%%%%%%%%%%%%%%%%%%%%%%%%%%%%%%%%%%%%%%%
\section{QL Applications}
%%%%%%%%%%%%%%%%%%%%%%%%%%%%%%%%%%%%%%%%%%%%%%%%%%

\subsection{Constants and Predicates}
Let's say we want to translate the following sentence into \emph{\GSL{}}:

\begin{smenumerate}
	\item\label{GQLTrans1} Mary is happy, smart, adorable, and a child.
\end{smenumerate}

\noindent{}The only connective we can translate is the \mention{and}.
The translation key:

\begin{description}[itemsep=0em]
	\item[Translation Key:] \hfill{} 
	\begin{description}[itemsep=0em]
		\item[] $\Hl$: Mary is happy.
		\item[] $\Rl$: Mary is smart. 
		\item[] $\Al$: Mary is adorable.
		\item[] $\Cl$: Mary is a child.  
	\end{description}
\end{description}

\noindent{}The \GSL{} result is: $\parconjunction{\conjunction{\Hl}{\Rl}}{\conjunction{\Al}{\Cl}}$.
This translation might work for certain purposes, but the conjuncts have no logical connection with each other.
Nothing in \GQL{} indicates that each conjunct is about the same person.

In \GQL{} we can express that shared logical connection with constants and predicates.
Translation keys of \GQL{} connect nouns to constants and English predicates to \GQL{} predicates.
Let's use the following \GQL{} key to translate \ref{GQLTrans1}:

\begin{description}[itemsep=0em]
	\item[Translation Key:] \hfill{} 
	\begin{description}[itemsep=0em]
		\item[] $\constant{m}$: Mary
		\item[] $\Hl\variable{t}$: $\variable{t}$ a child. 
		\item[] $\Rl\variable{t}$: $\variable{t}$ is smart. 
		\item[] $\Al\variable{t}$: $\variable{t}$ is adorable.
		\item[] $\Cl\variable{t}$: $\variable{t}$ is a child.  
	\end{description}
\end{description}

\noindent{}The result is: $\parconjunction{\conjunction{\Hp{\constant{m}}}{\Rp{\constant{m}}}}{\conjunction{\Ap{\constant{m}}}{\Cp{\constant{m}}}}$.
This translation allows us to see that we are predicating several things about the same person.
On to another example:

\begin{menumerate}
	\item\label{GQLTrans2} Ronnie and Demaryius are athletic, but Peyton isn't.
\end{menumerate}

\noindent{}We \emph{could} translate this as a conjunction with three conjuncts in \GSL{}.
However, this would not make clear that the same predicate either applies, or doesn't, to each of the three.
Instead, let's use the following \GQL{} key:

\begin{description}[itemsep=0em]
	\item[Translation Key:] \hfill{} 
	\begin{description}[itemsep=0em]
		\item[] $\constant{r}$: Ronnie
		\item[] $\constant{d}$: Demaryius
		\item[] $\constant{p}$: Peyton
		\item[] $\Al\variable{t}$: $\variable{t}$ is athletic.
	\end{description}
\end{description}

\noindent{}The result: $\parconjunction{\conjunction{\Ap{\constant{r}}}{\Ap{\constant{d}}}}{\negation{\Ap{\constant{p}}}}$.
We see the common predicate in each conjunct.

The translation keys here resemble the \GQL{} models we provided earlier.\footnote{
	See table \ref{table:Example Interpretations} in Chapter \ref{quantifierlogic}.
}
One difference is that we don't have a domain assigned in either of the \GQL{} keys above.
We could add a domain to each of the above, and interpret the constant and predicate lines of the key as making assignments from the domain.
Hence, we can usually treat informal model assignments as translation keys.

\subsection{Quantifiers}

The quantifier \mention{$\forall$} corresponds to the English phrases \mention{all} or \mention{every}.
Let's say we want to claim that every member of some set is also a member of some other set:

\begin{menumerate}
	\item\label{GQLTrans3} All dogs are furry.
\end{menumerate}

\noindent{}There are two sets: the set of all dogs and the set of all furry things.
Sentence \ref{GQLTrans3} effectively claims that the set of dogs is a subset of the set of furry things.
How do we translate this into \GQL{}?
The word \mention{all} typically calls for a universal quantifier.

It isn't obvious from the sentence itself, but whenever we want to make a claim of the form \mention{All $\PHI$ are $\THETA$}, we nearly always want to translate it as: $\universal{\ALPHA}\parhorseshoe{\CAPPHI}{\CAPTHETA}$.
That is, with a \mention{$\forall$} governing an \mention{$\HORSESHOE$}.
This makes more sense if we paraphrase \ref{GQLTrans3} in MathEnglish as: \mention{For all $\variable{x}$, if $\variable{x}$ is a dog then $\variable{x}$ is furry.}

For the rest of the translations in this section we use the following model description as our translation key:

\begin{description}[itemsep=0em]
	\item[Animals model:] \hfill{} 
	\begin{description}[itemsep=0em]
		\item[] $\emph{Animals}(\variable{U})$: All animals.
		\item[] $\emph{Animals}(\Ap{'})$: is a mammal.
		\item[] $\emph{Animals}(\Cp{'})$: is a cat.
		\item[] $\emph{Animals}(\Dp{'})$: is a dog.
		\item[] $\emph{Animals}(\Ep{'})$: is energetic.
		\item[] $\emph{Animals}(\Hp{'})$: is a happy.
		\item[] $\emph{Animals}(\Rp{'})$: is furry.
		\item[] $\emph{Animals}(\Ap{''})$: is smarter than.
	\end{description}
\end{description}

\noindent{}We translate \ref{GQLTrans3} as: $\universal{\variable{x}}\parhorseshoe{\Dp{\variable{x}}}{\Rp{\variable{x}}}$.

Other English words can be translated with the universal quantifier.
The following uses the word \mention{no} to make a universal claim:

\begin{menumerate}
	\item\label{GQLTrans4} No dogs are furry.
\end{menumerate}

\noindent{}Again, we can think of this as a claim about two sets: the set of dogs and the set of furry things.
This sentence is tantamount to a claim that these sets are disjoint, i.e., that nothing is a member of both.
We usually translate such claims into the following form in \GQL{}: $\universal{\ALPHA}\parhorseshoe{\CAPPHI}{\negation{\CAPTHETA}}$.
To understand why, consider the following MathEnglish paraphrase: \mention{For all $\variable{x}$, if $\variable{x}$ is a dog then $\variable{x}$ is not furry.}
The resulting translation is: $\universal{\variable{x}}\parhorseshoe{\Dp{\variable{x}}}{\negation{\Rp{\variable{x}}}}$.

The word \mention{only} can also be used for universal claims:

\begin{menumerate}
	\item\label{GQLTrans5} Only dogs are furry.
\end{menumerate}

\noindent{}This is translated in the same way as sentence \ref{GQLTrans3}, except that we reverse the order of the LHS and the RHS of the conditional governed by the \mention{$\forall$}.
It's equivalent to the claim that \mention{All furry things are dogs.}
So: $\universal{\variable{x}}\parhorseshoe{\Rp{\variable{x}}}{\Dp{\variable{x}}}$.

The \GQL{} symbol \mention{$\exists$} corresponds to the English phrases \mention{there exists}, \mention{there is}, or \mention{some}.
Consider the following existential sentences.

\begin{menumerate}
	\item\label{GQLTrans6} Some dogs are furry.
	\item\label{GQLTrans7} Some dogs are not furry.
\end{menumerate}

\noindent{}If we again think of the set of dogs and the set of furry things, \ref{GQLTrans6} is a claim that there is at least one element that is a member of each set.
Notice that we are interpreting \ref{GQLTrans6} as a claim about at least one object.
In English we typically think that \ref{GQLTrans6} is a claim about at least two dogs.
For now we ignore certain plural/singular distinctions in the way we interpret existential claims.
For our purposes, \mention{some} means \mention{at least one}.
We handle \mention{some} in a more satisfactory way when we add to \GQL{} in Chapter \ref{furtherdirections}.

For now, we translate sentences like \ref{GQLTrans6} into the form: $\existential{\ALPHA}\parconjunction{\CAPPHI}{\CAPTHETA}$.
So, for \ref{GQLTrans6} itself: $\existential{\variable{x}}\parconjunction{\Dp{\variable{x}}}{\Rp{\variable{x}}}$.
And we account for the \mention{not} in sentence \ref{GQLTrans7} as follows: $\existential{\variable{x}}\parconjunction{\Dp{\variable{x}}}{\negation{\Rp{\variable{x}}}}$.

We can also translate sentence \ref{GQLTrans4}, \mention{No dogs are furry}, as an existential governed by a negation.
We could translate \mention{There is a furry dog} as: $\existential{\variable{x}}\parconjunction{\Rp{\variable{x}}}{\Dp{\variable{x}}}$.
We can negate the result to capture the meaning of \ref{GQLTrans4}: $\negation{\existential{\variable{x}}\parconjunction{\Rp{\variable{x}}}{\Dp{\variable{x}}}}$.
In fact, this latter translation is logically equivalent to the one we gave earlier: $\universal{\variable{x}}\parhorseshoe{\Dp{\variable{x}}}{\negation{\Rp{\variable{x}}}}$.
To see this, join these two sentences together with a biconditional, \mention{$\TRIPLEBAR$}, and prove that the result is \CAPS{qt}.

Let's translate more complicated sentences into \GQL{}:

\begin{menumerate}
	\item\label{GQLTrans8} All happy dogs are furry and energetic.
\end{menumerate}

\noindent{}We want to translate this as a universal quantifier governing a conditional, but we must be careful to translate each side of the conditional correctly.
We paraphrase \ref{GQLTrans8} in MathEnglish as: For all $\variable{x}$, if ($\variable{x}$ is happy and $\variable{x}$ is a dog) then ($\variable{x}$ is furry and $\variable{x}$ is energetic).
So we can consider each side of the conditional as a conjunction: $\universal{\variable{x}}\parhorseshoe{\parconjunction{\Hp{\variable{x}}}{\Dp{\variable{x}}}}{\parconjunction{\Rp{\variable{x}}}{\Ep{\variable{x}}}}$.

\begin{menumerate}
	\item\label{GQLTrans9} All cats and dogs are mammals.
\end{menumerate}

\noindent{}This sentence is also going to be translated as a universal quantifier governing a conditional, but the word \mention{and} can be tricky.  Here `and' seems to conjoin predicates, not sentences.  We may be tempted to translate \ref{GQLTrans9} as: $\universal{\variable{x}}\parhorseshoe{\parconjunction{\Cp{\variable{x}}}{\Dp{\variable{x}}}}{\Ap{\variable{x}}}$.  But this \GQL{} sentence can be translated into MathEnglish as: \mention{For every $\variable{x}$, if ($\variable{x}$ is a cat and $\variable{x}$ is a dog) then $\variable{x}$ is a mammal.}  But that's silly.  Unless mad scientists are involved, nothing is both a cat and a dog.  Instead, we should translate the \mention{and} in \ref{GQLTrans9} as a \mention{$\VEE$}:  $\universal{\variable{x}}\parhorseshoe{\pardisjunction{\Cp{\variable{x}}}{\Dp{\variable{x}}}}{\Ap{\variable{x}}}$.  Let's translate this \GQL{} sentence into MathEnglish: \mention{For all $\variable{x}$, if ($\variable{x}$ is a cat or $\variable{x}$ is a dog), then $\variable{x}$ is a mammal.}  Take a moment to see how this better expresses the meaning of \ref{GQLTrans9}.

Consider a sentence with multiple quantifiers:

\begin{menumerate}
	\item\label{GQLTrans10} All dogs are smarter than all cats.
\end{menumerate}

\noindent{}There are two instances of the word \mention{all}, so we use two universal quantifiers in the translation.
Consider a paraphrase into MathEnglish: \mention{For every $\variable{x}$, if $\variable{x}$ is a dog then (for all $\variable{y}$, if $\variable{y}$ is a cat then $\variable{x}$ is smarter than $\variable{y}$).}
So we translate \ref{GQLTrans10} as: $\universal{\variable{x}}\parhorseshoe{\Dp{\variable{x}}}{\universal{\variable{y}}\parhorseshoe{\Cp{\variable{y}}}{\App{\variable{x}}{\variable{y}}}}$.

We have not said much about the role of domains in translations.
Because the point of translations, other than the sheer joy of doing it, is to evaluate arguments, it is appropriate to choose a domain suitable for the arguments in question.
Often a suitable choice of domain simplifies the translations.
If we are translating arguments that mention both dogs and natural numbers, we need to include both in the domain and to have a predicate for each.
If the arguments deal only with dogs then we can take the domain to be dogs.
You may be able to do without a predicate for `dog', depending on the nature of the argument.

As we observed before many English sentences are ambiguous.
One systematic ambiguity is in sentences of the form \mention{All As are not Bs}, which can either mean that it is not true that \mention{All As are Bs}, or that all \mention{As are not-Bs}.
Context or content usually indicate what is meant:  \mention{All sheep are not good pets} likely has the first meaning: $\negation{\universal{\variable{x}}\parhorseshoe{\Sl\variable{x}}{\parconjunction{\Gp{\variable{x}}}{\Pp{\variable{x}}}}}$.
But \mention{All sharks are not good pets} has the second: $\universal{\variable{x}}\parhorseshoe{\Sl\variable{x}}{\negation{\parconjunction{\Gp{\variable{x}}}{\Pp{\variable{x}}}}}$.
One of the values of formalization is that we can clearly and unambiguously express the structure of both.

It is important to distinguish ambiguous sentences---English sentences that have more than one meaning---from sentences for which there is more than one good translation, but which are equivalent.
We see this with \mention{No sharks are good pets}: $\universal{\variable{x}}\parhorseshoe{\Sl\variable{x}}{\negation{\parconjunction{\Gp{\variable{x}}}{\Pp{\variable{x}}}}}$ and $\negation{\existential{\variable{x}}\parconjunction{\Sl\variable{x}}{\parconjunction{\Gp{\variable{x}}}{\Pp{\variable{x}}}}}$ are equally good (and equivalent) translations.

%%%%%%%%%%%%%%%%%%%%%%%%%%%%%%%%%%%%%%%%%%%%%%%%%%
\section{Exercises}
%%%%%%%%%%%%%%%%%%%%%%%%%%%%%%%%%%%%%%%%%%%%%%%%%%

\notocsubsection{\GSL{} to English Translations}{ex:GSL to English Translations}
Given the following translation key, translate the following \GSL{} sentences into English. 
\begin{description}[itemsep=0em]
	\item[Translation Key:] \hfill{} 
	\begin{description}[itemsep=0em]
		\item $\Cl$: Cindy the Capybara is a picky eater.
		\item $\Ol$: Oscar the Ocelot sleeps all day.
		\item $\Rl$: Ralph the Rhinoceros goes for a swim.
		\item $\Al$: France is east of Spain. 
	\end{description} 
\end{description}
\begin{multicols}{2}
	\begin{enumerate}
		\item $\horseshoe{\Cl}{\Ol}$
		\item $\horseshoe{\Ol}{\Cl}$
		\item $\negation{\parhorseshoe{\Rl}{\Al}}$
		\item $\horseshoe{\negation{\Rl}}{\Al}$
		\item $\disjunction{\parconjunction{\Cl}{\Ol}}{\negation{\partriplebar{\Cl}{\Al}}}$
		\item $\conjunction{\Cl}{\pardisjunction{\Ol}{\negation{\partriplebar{\Cl}{\Al}}}}$
		\item $\horseshoe{\Al}{\negation{\parconjunction{\Cl}{\Rl}}}$
		\item $\horseshoe{\parconjunction{\Cl}{\Rl}}{\negation{\Al}}$
	\end{enumerate}
\end{multicols}

%\begin{table}[!ht]
%\renewcommand{\arraystretch}{1.5}
%\begin{center}
\begin{figure}
\begin{longtable}[c]{ l l l } %p{2.2in} p{2in}
	\toprule
	&\textbf{Symbol} & \textbf{Model Assignment} \\
	\midrule 
	\endfirsthead
	\multicolumn{3}{c}{\emph{Continued from Previous Page}}\\
	\toprule
	&\textbf{Symbol} & \textbf{Model Assignment} \\
	\midrule 
	\endhead
	\bottomrule
	\caption{Model for Translations in Section \ref{Translation Problems}}\\ %[-.15in]
	\multicolumn{3}{c}{\emph{Continued next Page}}\\
	\endfoot
	\bottomrule
	\caption{Model for Translations in Section \ref{Translation Problems}}\\%
	\endlastfoot%
	\label{Trans Int Table}% 
	Universe:& & The set of states \\ \addlinespace[.25cm]
	Constants:& $\constant{c}$& CA\\
	& $\constant{m}$& MT\\
	& $\constant{h}$& RI\\
	& $\constant{e}$& TX\\ \addlinespace[.25cm]
	1 place predicates: &$\Pp{'}$& Pacific states\\
	&$\Ap{'}$& Atlantic states\\
	&$\Gp{'}$& Gulf states\\
	&$\Mp{'}$& Mountainous states\\
	&$\Cp{'}$& Coastal states\\ \addlinespace[.25cm]
	2 place predicates:&$\Lp{''}$& is larger than (area)\\
	&$\Bp{''}$& borders\\
\end{longtable}
\caption{Model for Translations in Section \ref{Translation Problems}}
\end{figure}

\notocsubsection{English to \GSL{} Translations \#1}{ex:English to GSL Translations 1}
Using some sensible translation key translate the following English sentences into \GSL{}. 
\begin{enumerate}
	\item If the sprockets come in on time, then we can fill the order.
	\item Only if the sprockets come in on time can we fill the order. 
	\item Either the order gets filled, or the cogs come in late and the sprockets never show up. 
	\item It's not the case that the sprockets need to come in for the order to be filled. 
	\item While filling the order is important, getting the sprockets in is more so. 
	\item The sprockets and cogs are late, but it's still not the case that we can't fill the order on time. 
	\item Assuming the order gets out on time, the sprockets will fail to arrive only if the cogs are either late or defective. 
	\item The spork is the least appreciated utensil. 
	\item They dined on mince, and slices of quince, [which] they ate with a runcible spoon. (1871, Edward Lear, “Owl \& Pussy-Cat” in \emph{Nonsense Songs})
	\item You eat with a spork if and only if you eat with a foon. 
	\item Although Jan will be amused, if you eat with a spork Jill will leave or at least not laugh.
	\item If you have a runcible spoon, then you don't need a fork, knife, or spoon. 
\end{enumerate}

\notocsubsection{English to \GSL{} Translations \#2}{ex:English to GSL Translations 2}
Using some sensible translation key translate the following English sentences into \GSL{}. 
\begin{enumerate}
	\item If 14-year-olds had the vote, I'd be president. (Evel Knievel)
	\item If Miami beats Cornell today and Penn State defeats Michigan State Miami will win the tournament.
	\item Should senator Ervin run again, he would be a formidable opponent. 
	\item If Congress does not find a way to force the banking industry to lower interest rates, and if the Securities and Exchange Commission does not stop authorizing unjustified new financing by corporations, we will face an unbearable depression. 
	\item Provided, but only provided, that the French Fleet is sailed forthwith for British harbors, His Majesty's Government give their full consent to an armistice for France. (Churchill, June 1940)
	\item For the tenability of the thesis that mathematics is logic it is not only sufficient but also necessary that all mathematical expressions be capable of definition on the basis solely of logical ones. (W.V.O. Quine)
\end{enumerate}


\notocsubsection{Translations}{Translation Problems} Translate each of the following English sentences into \GQL{} sentences about the model $\IntA$ given in figure \mvref{Trans Int Table}.
\begin{multicols}{2}
	\begin{enumerate}
		\item {All Pacific states that border a mountainous state are coastal.}
		\item {Some Atlantic state and some mountainous state both share a border with a state that is neither.}
		\item {All states are coastal and mountainous if and only if they are Pacific.}
		\item {All Atlantic states smaller than Montana share a border with Rhode Island.}
		\item {Only mountainous Pacific states are coastal.}
		\item {A Pacific state is mountainous.}
		\item {No state is larger than itself.}
		\item {Every non-mountainous state borders a state that is larger.}
		\item {Some Pacific states are mountainous.}
		\item {All Pacific states are mountainous.}
		\item {All Pacific states are larger than all Atlantic states.}
		\item {No Gulf state is mountainous.}
		\item {All Atlantic states are not mountainous.}
		\item {Some Gulf state is larger than all states that border it.}
		\item {Some Gulf state is an Atlantic state.}
		\item {All states that border a Pacific state are mountainous.}
		\item {Any state that is mountainous is larger than Rhode Island.}
		\item {Any state that is mountainous is larger than all Atlantic
			states.}
		\item {If any state is mountainous, California is.}
		\item {If any state is mountainous, it is larger than Rhode Island.}
		\item {Any state that has no bordering states is mountainous.}
		\item {All states that are bigger than all mountainous states are
			coastal.}
		\item {No state is bigger than Montana unless it is coastal.}
	\end{enumerate}
\end{multicols}

\notocsubsection{More Translations}{ex:More Translations}
Translate each of the following English sentences into \GQL{} sentences.
\begin{multicols}{2}
	\begin{enumerate}
		\item {All beavers avoid some kangaroo.}
		\item {All beavers avoid all kangaroos.}
		\item {Some beaver avoids all kangaroos.}
		\item {Every kangaroo is avoided by some beaver.}
		\item {All beavers avoid any kangaroo that frightens them.}
		\item {Some beavers avoid any kangaroo that frightens them.}
		\item {No kangaroo frightens any beaver.}
		\item {No beaver is frightened by any kangaroo.}
		\item {No beaver avoids a kangaroo unless the kangaroo frightens it.}
		\item {Some kangaroo frightens itself.}
		\item {No beaver avoids a kangaroo unless the beaver frightens the kangaroo.}
		\item {Any kangaroo that is frightened of itself is frightened by any beaver.}
		\item {Beavers avoid kangaroos only if they frighten them.}
		\item {Kangaroos that frighten beavers frighten themselves.}
		\item {All kangaroos avoid any kangaroo that avoids them.}
		\item {When a kangaroo frightens a beaver, the beaver avoids it.}
		\item {Beavers only avoid kangaroos.}
		\item {Beavers are frightened of all kangaroos unless they avoid them.}
		\item {Some beavers avoid only kangaroos that frighten them.}
		\item {No beaver that avoids all kangaroos frightens itself.}
	\end{enumerate}
\end{multicols}



%\theendnotes

%%%%%%%%%%%%%%%%%%%%%%%%%%%%%%%%%%%%%%%%%%%%%%%%%%
\chapter{Derivations}\label{Derivations}
%%%%%%%%%%%%%%%%%%%%%%%%%%%%%%%%%%%%%%%%%%%%%%%%%%
%\AddToShipoutPicture*{\BackgroundPicC}

%%%%%%%%%%%%%%%%%%%%%%%%%%%%%%%%%%%%%%%%%%%%%%%%%%
\section{Introduction}\label{Derivation Preliminaries}
%%%%%%%%%%%%%%%%%%%%%%%%%%%%%%%%%%%%%%%%%%%%%%%%%%


We made sense of semantic notions like truth, logical truth, and entailment in previous chapters by the use of \GSL{} and \GQL{} models.
But proving that a sentence of \GSL{} or \GQL{} is a logical truth, or proving that an entailment holds, is often difficult and involves informal reasoning in MathEnglish. 
We would like to prove such things more easily and with less reliance on intuitive judgment. 
Recall from section \ref{Formal Languages} that \GSL{} and \GQL{} are \emph{formal}.
As a result, one can determine whether a strings of symbols is a sentence in a purely mechanical way.
Are there similarly formal methods for establishing logical truths and entailments?

Yes.
For this purpose we define rules for manipulating \GSL{} and \GQL{} sentences, allowing us to derive other sentences by formal steps.
We call a finite sequence of such steps a \emph{derivation}. 
Every sentence is justified by a rule which permits us to write it down, usually on the basis of previous sentences in the derivation.%
\footnote{%
The terms \mention{derivation} and \mention{proof} (or \mention{formal proof}) are often used interchangeably in other texts. 
The subfield of logic that studies derivations is even called proof theory.
We always use \mention{derivation} to refer to the sequences of formal steps defined in this chapter and \mention{proof} to refer to the more informal mathematical proofs written in MathEnglish.
} Here's an example derivation:

\begin{gproofnn}
\gaproof{
\galine{1}{$\conjunction{\Bl}{\Cl}$}{\Rule{Assume}}
\galine{2}{$\Bl$}{\Rule{$\WEDGE$-Elim}, 1}
\galine{3}{$\Cl$}{\Rule{$\WEDGE$-Elim}, 1}
\galine{4}{$\conjunction{\Cl}{\Bl}$}{\Rule{$\WEDGE\!$-Intro}, 2,3}
}
\gline{5}{$\horseshoe{\parconjunction{\Bl}{\Cl}}{\parconjunction{\Cl}{\Bl}}$}{\Rule{$\HORSESHOE$-Intro}, 1--4\footnote{This kind of introduction and elimination rule-based derivation system is called a \idf{natural deduction} system.
Natural deduction systems were invented by Stanislaw Jaskowski in 1926, though he did not publish until Jaskowski \citeyearpar{Jaskowski1934}.
Other important developments are due to Gerhard Gentzen \citeyearpar{Gentzen1934}.
Unlike many systems, ours uses boxes to discharge assumptions.
Jaskowski mentions in the 1934 article that he used boxes in 1926, but they aren't part of his 1934 system.}}
\end{gproofnn}

Each step contains three parts: a line number, a sentence, and a rule.
Derivations are formal because the rules they are constructed with are formal.
That is, the rules are specified entirely in terms of the shape, or form, of the sentences in the derivation.
So there are no rules like: 
\begin{RESTARTmenumerate}
\item If $\CAPPHI\sdtstile{}{}\CAPPSI$ and you have $\CAPPHI$ on a previous line then you may write down $\CAPPSI$. 
\end{RESTARTmenumerate}
It can be arbitrarily difficult to determine whether $\CAPPHI\sdtstile{}{}\CAPPSI$, so it isn't always obvious whether a given application of this rule is correct.
By contrast, formal rules are transparent and easy to check for correctness: 
\begin{menumerate}
\item If you have $\conjunction{\CAPPHI}{\CAPPSI}$ on a previous line then you may write down $\CAPPSI$.
\end{menumerate} 

There is no explicit connection between the derivation rules and the semantic notions of truth, logical truth, and entailment. 
Even so, the rules are intended to give us the same results that were established about these notions in previous chapters.
For example, the rules are carefully defined to be truth-preserving. 
\begin{description}
\item[Truth-preservation:] If a rule allows the derivation of sentence $\CAPPHI$ from sentences $\CAPPSI_1,\ldots,\CAPPSI_{\integer{n}}$, then $\CAPPSI_1,\ldots,\CAPPSI_{\integer{n}}\sdtstile{}{}\CAPPHI$.\index{truth-preserving}\index{derivation!rule!truth-preserving} 
\end{description} 
Derivations should be correct.
That is, any sentence that can be derived without assumptions should be a logical truth, and any sentence derived from other sentences should be a logical consequence of them:
\begin{description}
\item[Soundness:] If $\CAPPHI$ can be derived from $\CAPPSI_1,\ldots,\CAPPSI_{\integer{n}}$, then $\CAPPSI_1,\ldots,\CAPPSI_{\integer{n}}\sdtstile{}{}\CAPPHI$.\index{soundness}
\end{description} 
\noindent{}We want derivations to work in the other direction too.
If a sentence is a logical truth, or if some sentence is a logical consequence of some other sentences, there should be a derivation showing as much:
\begin{description}
\item[Completeness:] If $\CAPPSI_1,\ldots,\CAPPSI_{\integer{n}}\sdtstile{}{}\CAPPHI$, then $\CAPPHI$ can be derived from $\CAPPSI_1,\ldots,\CAPPSI_{\integer{n}}$.\index{completeness}
\end{description}

We prove these two results (and variations of them) for both \GSL{} and \GQL{} in the next chapter.
For now it is enough to remember that derivations are intended to mirror many of the properties of the proofs of earlier chapters.

It's useful to think of writing a derivation as playing a game or solving a puzzle.
Like a game there are rules determining what moves you can make next, or like a puzzle there's a sequence of steps you can discover to make the derivation fit together. 
Unlike ordinarily puzzles, however, there are many ways they can fit together.

%%%%%%%%%%%%%%%%%%%%%%%%%%%%%%%%%%%%%%%%%%%%%%%%%%
\section{The Basic System \GSD{}}
%%%%%%%%%%%%%%%%%%%%%%%%%%%%%%%%%%%%%%%%%%%%%%%%%%

\subsection{Introduction and Elimination Rules} 
There are two kinds of rules: basic and shortcut.
The basic rules\index{derivation!rule!basic}\index{basic rule} are the minimum rules needed to define derivations.
The shortcut rules are not necessary but they make derivations easier and shorter.\index{derivation!rule!shortcut} 
Anything we can derive with the shortcut rules can be derived from the basic ones alone (see thm. \pmvref{GSD Shortcut Theorem3}). 

The rules for sentences of \GSL{} make up \idf{Sentential Derivation System}, or \GSD{}. 
\GSD{} consists of the following rules: \Rule{Assume}, \Rule{Repetition}, and an introduction rule\index{derivation!rule!introduction}\index{introduction rule} and elimination rule\index{derivation!rule!elimination}\index{elimination rule} for each of the \GSL{} connectives.
Derivations for the sentences of \GQL{} also have introduction and elimination rules for the logical connectives unique to \GQL{}, i.e. the quantifiers.

We begin by considering the introduction and elimination rules for the conjunction, $\WEDGE$.
If we have the sentences $\CAPPHI$ and $\CAPPSI$ in a derivation, then we may write $\conjunction{\CAPPHI}{\CAPPSI}$ on a new line, by the $\WEDGE$ introduction rule. 
On the other hand, if we have the sentence $\conjunction{\CAPPHI}{\CAPPSI}$, then the $\WEDGE$ elimination rule permits us to write either $\CAPPHI$ or $\CAPPSI$ on a new line.
Table \ref{GSD} gives all the basic \GSD{} rules.

\renewcommand{\arraystretch}{1.5}
\begin{longtable}[c]{ p{1in} l l }
\toprule
\textbf{Name} & \textbf{Given} & \textbf{May Add} \\ 
\midrule
\endfirsthead
\multicolumn{3}{c}{\emph{Continued from Previous Page}}\\
\toprule
\textbf{Name} & \textbf{Given} & \textbf{May Add} \\ 
\midrule
\endhead
\bottomrule
\caption{Basic Rules of \GSD{}}\\[-.15in]
\multicolumn{3}{c}{\emph{Continued next Page}}\\
\endfoot
\bottomrule
\caption{Basic Rules of \GSD{}}\\%
\endlastfoot%
\label{GSD}%
\Rule{Assume} & & | $\CAPPHI$ \\
\Rule{Rep.} & $\CAPPHI$ & $\CAPPHI$ \\
\Rule{$\HORSESHOE$-Elim} & $\horseshoe{\CAPTHETA}{\CAPPSI}$, $\CAPTHETA$ & $\CAPPSI$ \\
\Rule{$\HORSESHOE$-Intro} &  | $\CAPTHETA$ &  \\
 &  | $\vdots$ &  \\
 &  | $\CAPPSI$ & $\horseshoe{\CAPTHETA}{\CAPPSI}$, Draw box\footnote{When using the rule $\HORSESHOE$-Intro, you must draw a box around all lines from the $\CAPTHETA$ to the $\CAPPSI$.  The line with $\CAPTHETA$ must be sanctioned by the rule \mention{Assume}.} \\
\Rule{$\!\WEDGE\!$-Elim} &{}$\conjunction{\CAPTHETA_1}{\conjunction{\CAPTHETA_2}{\conjunction{\ldots}{\CAPTHETA_{\integer{n}}}}}$&{}Any one of the conjuncts\\[-.25cm]
 & &{}i.e., $\CAPTHETA_{\integer{i}}$\\
\Rule{$\!\WEDGE\!$-Intro} & $\CAPTHETA_1$, $\CAPTHETA_2$, $\ldots$ $\CAPTHETA_{\integer{n}}$ & $\conjunction{\CAPTHETA_1}{\conjunction{\CAPTHETA_2}{\conjunction{\ldots}{\CAPTHETA_{\integer{n}}}}}$ \\
\Rule{$\VEE$-Elim} & $\disjunction{\CAPTHETA_1}{\disjunction{\CAPTHETA_2}{\disjunction{\ldots}{\CAPTHETA_{\integer{n}}}}}$, &  \\
 &  $\horseshoe{\CAPTHETA_1}{\CAPPSI}$,  &  \\
 &  $\horseshoe{\CAPTHETA_2}{\CAPPSI}$,  &  \\
 &  $\vdots$  &  \\
 &  $\horseshoe{\CAPTHETA_{\integer{n}}}{\CAPPSI}$ & $\CAPPSI$ \\
\Rule{$\VEE$-Intro} & $\CAPTHETA$ & $\disjunction{\CAPPSI_1}{\disjunction{\CAPPSI_2}{\disjunction{\ldots}{\CAPPSI_{\integer{n}}}}}$, \\[-.25cm]
 \nopagebreak
 &  & where $\CAPTHETA$ is $\CAPPSI_i$ for some $i$. \\
\Rule{$\NEGATION$-Intro} & $\horseshoe{\CAPTHETA}{\parconjunction{\CAPPSI}{\negation{\CAPPSI}}}$ & $\negation{\CAPTHETA}$ \\
\Rule{$\NEGATION$-Elim} & $\horseshoe{\negation{\CAPTHETA}}{\parconjunction{\CAPPSI}{\negation{\CAPPSI}}}$ & $\CAPTHETA$ \\
\Rule{$\TRIPLEBAR$-Intro} & $\horseshoe{\CAPTHETA}{\CAPPSI}$, $\horseshoe{\CAPPSI}{\CAPTHETA}$ & $\triplebar{\CAPTHETA}{\CAPPSI}$ \\
\Rule{$\TRIPLEBAR$-Elim} & $\triplebar{\CAPTHETA}{\CAPPSI}$, $\CAPPSI$ & $\CAPTHETA$ \\
\nopagebreak
\Rule{$\TRIPLEBAR$-Elim} & $\triplebar{\CAPTHETA}{\CAPPSI}$, $\CAPTHETA$ & $\CAPPSI$ \\
\end{longtable}
\index{derivation!rule!basic}\index{basic rule}
\index{derivation!rule!introduction}\index{introduction rule}
\index{derivation!rule!elimination}\index{elimination rule}

\subsection{Boxes as an Accounting Device for Assumptions} 

Let's look again at the first example:

\begin{gproofnn}
\gaproof{
\galine{1}{$\conjunction{\Bl}{\Cl}$}{\Rule{Assume}}
\galine{2}{$\Bl$}{\Rule{$\WEDGE$-Elim}, 1}
\galine{3}{$\Cl$}{\Rule{$\WEDGE$-Elim}, 1}
\galine{4}{$\conjunction{\Cl}{\Bl}$}{\Rule{$\WEDGE\!$-Intro}, 2,3}
}
\gline{5}{$\horseshoe{\parconjunction{\Bl}{\Cl}}{\parconjunction{\Cl}{\Bl}}$}{\Rule{$\HORSESHOE$-Intro}, 1--4}
\end{gproofnn}

You will notice a box around the sentences of the first four steps.
That box is the product of two rules: \Rule{Assume} and \Rule{$\HORSESHOE$-intro}.
The \Rule{Assume} rule allows you to write any sentence you like, under one condition.
You must draw a vertical line to the left of the sentence you assume, and that line must be extended to the left of all sentences of each step below until you the \Rule{$\HORSESHOE$-intro} rule is applied.
When \Rule{$\HORSESHOE$-intro} is applied three more lines must be drawn, forming a box around all the sentences from the assumption to the step just before the \Rule{$\HORSESHOE$-intro}.

Every use of the \Rule{Assume} rule starts a box, and every use of \Rule{$\HORSESHOE$-intro} closes a box.
The vertical line to the left represents an assumption.
Every sentence that is derived on that line represents a conclusion reached using that assumption.
Closing the box with \Rule{$\HORSESHOE$-intro} represents a \emph{discharge} of that assumption.
In other words that assumption is no longer being made.
Neither the assumption nor the lines derived from that assumption may be used in the rest of the derivation.
A closed the box is a visual reminder that those lines cannot be used any longer.

To understand better how assumptions work in derivations we state their use in more general terms.
At any step you may assume any sentence you like, so long as you write a vertical line to the left of it: \mbox{| $\CAPTHETA$}.
Assumptions are discharged using the \Rule{$\HORSESHOE$-Intro} rule.
To apply \Rule{$\HORSESHOE$-Intro} you need a vertical line from an unboxed assumption \mbox{| $\CAPTHETA$} that extends down to an unboxed \mbox{| $\CAPPSI$}. 
After you've added $\horseshoe{\CAPTHETA}{\CAPPSI}$ to your derivation, you \emph{must} box off the part of the derivation that begins with | $\CAPTHETA$ and ends with | $\CAPPSI$.
That is, if you have the following:

\begin{gproofnn}[\label{conditionalintro}]
	\galineNC{1}{$\CAPTHETA$}{\Rule{Assume}}
	\galineNC{2}{}{}
	\galineNC{3}{$\qquad\vdots$}{}
	\galineNC{4}{}{}
	\galineNC{5}{$\CAPPSI$}{}
\end{gproofnn}

\noindent{}you may then use the rule \Rule{$\HORSESHOE$-Intro} to get:

\begin{gproofnn}[\label{conditionalintroclosed}]
	\gaproof{
		\galine{1}{$\CAPTHETA$}{\Rule{Assume}}
		\galine{2}{}{}
		\galine{3}{$\qquad\vdots$}{}
		\galine{4}{}{}
		\galine{5}{$\CAPPSI$}{}
	}
	\gline{6}{$\horseshoe{\CAPTHETA}{\CAPPSI}$}{\Rule{$\HORSESHOE$-Intro}, 1--5}
\end{gproofnn}

\noindent{}You may even assume more than one sentence at a time, in which case you may have several vertical lines to the left.
Just remember that these assumptions will typically need to be discharged.\footnote{Leaving an assumption undischarged is tantamount to treating it as a premise in an argument.}
Vertical assumption lines are like your credit card balance.
It can be convenient to run a high balance for a time, as long as you can pay it down eventually.

\subsection{Writing Derivations}

Let's look at a few derivations in line-by-line detail.
Consider the derivation of $\conjunction{\Cl}{\Bl}$ from $\conjunction{\Bl}{\Cl}$:
\begin{gproof}[\label{onelinederivation}]
\galineNC{1}{$\conjunction{\Bl}{\Cl}$}{\Rule{Assume}}
\end{gproof}
\noindent{}Note that \Rule{$\WEDGE$-Elim} allows us to write down one of the conjuncts of a conjunction we already have. 
The conjuncts of $\conjunction{\Bl}{\Cl}$ are $\Bl$ and $\Cl$. 
Accordingly, we may write down $\Bl$ and $\Cl$ on new lines. 
We continue:
\begin{gproof}[\label{threelinederivation}]
\galineNC{1}{$\conjunction{\Bl}{\Cl}$}{\Rule{Assume}}
\galineNC{2}{$\Bl$}{\Rule{$\WEDGE$-Elim}, 1}
\galineNC{3}{$\Cl$}{\Rule{$\WEDGE$-Elim}, 1}
\end{gproof}
\noindent{}Finally, using \Rule{$\WEDGE\!$-Intro} we put the sentences from lines 2 and 3 together in a conjunction:
\begin{gproof}[\label{simpleconjunction}]
\galineNC{1}{$\conjunction{\Bl}{\Cl}$}{\Rule{Assume}}
\galineNC{2}{$\Bl$}{\Rule{$\WEDGE$-Elim}, 1}
\galineNC{3}{$\Cl$}{\Rule{$\WEDGE$-Elim}, 1}
\galineNC{4}{$\conjunction{\Cl}{\Bl}$}{\Rule{$\WEDGE\!$-Intro}, 2,3}
\end{gproof}
\noindent{}This is a four-line derivation of $\conjunction{\Cl}{\Bl}$ from $\conjunction{\Bl}{\Cl}$.
Notice that steps 2 and 3 could have been done in opposite order.
In many cases the order of sentences in a derivation doesn't matter, but in other cases it is crucial.
Learning the difference is an important skill.

Just as we used the double turnstile to represent when one sentence (or a set of sentences) entailed another, we use what's called the \emph{single turnstile}, \mention{\:$\sststile{}{}\:$}, to represent when one sentence is derivable from another, or from another (finite) set of sentences. 
The four-line derivation shows that $\conjunction{\Bl}{\Cl}\sststile{}{}\conjunction{\Cl}{\Bl}$. 
It's also worth noting that each of the two partially completed pieces of the derivation just given are derivations themselves. 
The first (\ref{onelinederivation}) is a one-line derivation that shows that $\conjunction{\Bl}{\Cl}\sststile{}{}\conjunction{\Bl}{\Cl}$, while the second (\ref{threelinederivation}) is a three-line derivation that shows that $\conjunction{\Bl}{\Cl}\sststile{}{}\Cl$.

We can now ``close off'' the initial assumption by drawing a box around that part of the proof and writing a conditional on the next line:
\begin{gproof}[\label{simpleconjunctionclosed}]
\gaproof{
\galine{1}{$\conjunction{\Bl}{\Cl}$}{\Rule{Assume}}
\galine{2}{$\Bl$}{\Rule{$\WEDGE$-Elim}, 1}
\galine{3}{$\Cl$}{\Rule{$\WEDGE$-Elim}, 1}
\galine{4}{$\conjunction{\Cl}{\Bl}$}{\Rule{$\WEDGE\!$-Intro}, 2,3}
}
\gline{5}{$\horseshoe{\parconjunction{\Bl}{\Cl}}{\parconjunction{\Cl}{\Bl}}$}{\Rule{$\HORSESHOE$-Intro}, 1--4}
\end{gproof}
\noindent{}So we have derived $\horseshoe{\parconjunction{\Bl}{\Cl}}{\parconjunction{\Cl}{\Bl}}$ without any assumptions remaining.
When we have a derivation from no assumptions (or, rather, with all assumptions boxed), we represent that with a single turnstile with no formulas on the left.
This derivation shows that $\sststile{}{}\horseshoe{\parconjunction{\Bl}{\Cl}}{\parconjunction{\Cl}{\Bl}}$.\footnote{If $\sststile{}{}\CAPPHI$, it's often said that $\CAPPHI$ is a theorem of the derivation system.
``Theorem'' is used ambiguously between the derivation of an \GSL (or \GQL) sentence from the empty set of assumptions and things we prove in MathEnglish.
We follow that standard practice.}

Next, as an example of how assumption lines stack up in a derivation, consider the following derivation of $\Dl$ from $\conjunction{\Al}{\Bl}$ and $\horseshoe{\Bl}{\parconjunction{\Cl}{\Dl}}$.
\begin{gproof}[\label{secondexample}]
\galineNC{1}{$\conjunction{\Al}{\Bl}$}{\Rule{Assume}}
\gaalineNC{2}{$\horseshoe{\Bl}{\parconjunction{\Cl}{\Dl}}$}{\Rule{Assume}}
\gaalineNC{3}{$\Bl$}{\Rule{$\WEDGE$-Elim}, 1}
\gaalineNC{4}{$\conjunction{\Cl}{\Dl}$}{\Rule{$\HORSESHOE$-Elim}, 2,3}
\gaalineNC{5}{$\Dl$}{\Rule{$\WEDGE$-Elim}, 4}
\end{gproof}
\noindent{}As is shown, any time a new assumption line is added it must be placed to the right of the previous (unboxed) assumption lines.

We can only discharge one assumption at a time: the most recent unboxed assumption.
So we get

\begin{gproof}
\galineNC{1}{$\conjunction{\Al}{\Bl}$}{\Rule{Assume}}
	\gaaproof{
		\gaalineNCndS{2}{$\horseshoe{\Bl}{\parconjunction{\Cl}{\Dl}}$}{\Rule{Assume}}
		\gaalineNCndS{3}{$\Bl$}{\Rule{$\WEDGE$-Elim}, 1}
		\gaalineNCndS{4}{$\conjunction{\Cl}{\Dl}$}{\Rule{$\HORSESHOE$-Elim}, 2,3}
		\gaalineNCndS{5}{$\Dl$}{\Rule{$\WEDGE$-Elim}, 4}
	}
\galineNC{6}{$\horseshoe{\parhorseshoe{\Bl}{\bparconjunction{\Cl}{\Dl}}}{\Dl}$}{\Rule{$\HORSESHOE$-Intro}, 2--5}
\end{gproof}
\noindent{}Now that the second assumption is boxed, we are free to discharge the first:

\begin{gproof}[\label{secondexamplefinished}]
\gaproof{
\galine{1}{$\conjunction{\Al}{\Bl}$}{\Rule{Assume}}
\gaaproof{
\gaaline{2}{$\horseshoe{\Bl}{\parconjunction{\Cl}{\Dl}}$}{\Rule{Assume}}
\gaaline{3}{$\Bl$}{\Rule{$\WEDGE$-Elim}, 1}
\gaaline{4}{$\conjunction{\Cl}{\Dl}$}{\Rule{$\HORSESHOE$-Elim}, 2,3}
\gaaline{5}{$\Dl$}{\Rule{$\WEDGE$-Elim}, 4}
}
\galine{6}{$\horseshoe{\parhorseshoe{\Bl}{\bparconjunction{\Cl}{\Dl}}}{\Dl}$}{\Rule{$\HORSESHOE$-Intro}, 2--5}
}
\gline{7}{$\horseshoe{\parconjunction{\Al}{\Bl}}{\cparhorseshoe{\parhorseshoe{\Bl}{\bparconjunction{\Cl}{\Dl}}}{\Dl}}$}{\Rule{$\HORSESHOE$-Intro}, 1--6}
\end{gproof}
\noindent{}We have derived $\horseshoe{\parconjunction{\Al}{\Bl}}{\cparhorseshoe{\parhorseshoe{\Bl}{\bparconjunction{\Cl}{\Dl}}}{\Dl}}$ without any outstanding assumptions.
So we have $\sststile{}{}\horseshoe{\parconjunction{\Al}{\Bl}}{\cparhorseshoe{\parhorseshoe{\Bl}{\bparconjunction{\Cl}{\Dl}}}{\Dl}}$.

It's worth reiterating that once sentences have been boxed they can't be used for the rest of the derivation.
These sentences are no longer \mention{Given}, to use the term in the Basic Rules chart.
If their use were allowed then \GSD{} would not be sound.
That is, there would be some sentences $\CAPPHI$ and $\CAPPSI_1,\ldots,\CAPPSI_{\integer{n}}$ such that $\CAPPSI_1,\ldots,\CAPPSI_{\integer{n}}\text{ }\cancel{\sdtstile{}{}}\text{ }\CAPPHI$ but $\CAPPSI_1,\ldots,\CAPPSI_{\integer{n}}\sststile{}{}\CAPPHI$.
So $\sststile{}{}$ and $\sdtstile{}{}$ wouldn't agree with each other---an unfortunate result!
To see the problem, consider:
\begin{gproof}
\galineNC{1}{$\horseshoe{\Al}{\Bl}$}{\Rule{Assume}}
\gaaproof{
\gaalineNCS{2}{$\Al$}{\Rule{Assume}}
\gaalineNCS{3}{$\Bl$}{\Rule{$\HORSESHOE$-Elim}, 1,2}
}
\galineNC{4}{$\horseshoe{\Al}{\Bl}$}{\Rule{$\HORSESHOE$-Intro}, 2--3}
\galineNC{5}{$\Bl$}{\Rule{$\HORSESHOE$-Elim}, 2,4}
\end{gproof}
\noindent{}Line 5 cites line 2 to use $\Al$ to get $\Bl$ from line 4 using \Rule{$\HORSESHOE$-Intro}. 
But clearly $\horseshoe{\Al}{\Bl}\sdtstile{}{}\Bl$ does not hold. 
Consider a model $\IntA$ such that $\IntA(\Al)=\FalseB$ and $\IntA(\Bl)=\FalseB$.

\subsection{The Recursive Definition of a Derivation}\label{RecDefOfDerv}
We have not yet given a precise definition of \mention{derivation}.
But before we do so some prerequisite definitions are needed.

\begin{majorILnc}{\LnpDC{RuleSchemas}}
	A \nidf{formal derivation rule}\index{derivation!rule|textbf} is a sequence of sentence schemas divided into two parts, the first part being the \df{given schemas} and the second part being the \df{may-add schema}. 
\end{majorILnc}
\noindent{}Table \mvref{GSD} lists rules by putting the ``Given'' schemas in the left column, and the ``May Add'' schema in the right. 
We spell out these schemas so the next definition is easier to state.
\begin{majorILnc}{\LnpDC{RuleSanctioning}}
	Let there be unboxed lines $\integer{m}_1,\ldots,\integer{m}_{\integer{j}}$ with sentences $\CAPPSI_1,\ldots,\CAPPSI_{\integer{j}}$, respectively.
	A rule \Rule{R}, applied to lines $\integer{m}_1,\ldots,\integer{m}_{\integer{j}}$ \df{sanctions} writing the sentence $\CAPPHI$ \Iff there's some substitution of \GSL{} sentences that, for the given schemas of \Rule{R}, results in $\CAPPSI_1,\ldots,\CAPPSI_{\integer{j}}$ and, for the may-add schema, results in $\CAPPHI$. 
\end{majorILnc}
\noindent{}As an example, consider again derivation \pmvref{secondexamplefinished}. 
The rule \Rule{$\HORSESHOE$-Elim} was applied to line 2, which had sentence $\horseshoe{\Bl}{\parconjunction{\Cl}{\Dl}}$, and line 3, which had sentence $\Bl$, to get line 4, which had sentence $\conjunction{\Cl}{\Dl}$. 
Using definition \ref{RuleSanctioning} we can show that this move is sanctioned by \Rule{$\HORSESHOE$-Elim} by noting, from table \mvref{GSD}, that \Rule{$\HORSESHOE$-Elim} has two given schemas, $\horseshoe{\CAPTHETA}{\CAPPSI}$ and $\CAPTHETA$, and the may-add scheme $\CAPPSI$. 
Substituting $\CAPTHETA=\Bl$ and $\CAPPSI=\conjunction{\Cl}{\Dl}$ in the given schemas gets us lines 2 and 3, while making this same substitution in the may-add schema gets us line 4. 

We now give a precise definition of a derivation. 
The explicit definition is especially useful for proving soundness in the next chapter.
Although the actual definition is complicated, the basic idea is straightforward: 
A single \GSL{} sentence that's an assumption is a derivation (e.g., derivation \ref{onelinederivation}), and any finite sequence of \GSL{} sentences every one of which is either an assumption or sanctioned by some rule of \GSD{} is a derivation. 
This idea must be revised to handle the rules \Rule{$\HORSESHOE$-Intro} and \Rule{Assume}, neither of which quite work like the other rules.
To understand these rules we must first define what an \emph{open assumption} is. An assumption is open \Iff it appears on an assumption line and is not in a box.
\begin{majorILnc}{\LnpDC{Recursive definition of Derivation}} The following recursive clauses fix which finite sequences of derivation lines are \nidf{derivations}\index{derivation|textbf} in \GSD{}:
\begin{description}
\item[Base Clause:] For any \GSL{} sentence $\CAPPHI$, the following single derivation line (i.e., sequence of derivation lines of length 1) is a derivation: 
\begin{gproofnn}
\galineNC{1}{$\CAPPHI$\qquad}{\Rule{Assume}}
\end{gproofnn}
\item[Generating Clause:] \hfill{}
\begin{description}
\item[Case 1:] If you have some $\integer{n}$-line derivation with $\integer{k}$ assumptions still open at line $\integer{n}$, and in which the sentence $\CAPPSI$ on $\integer{n}$ is sanctioned by rule $\Rule{R}$ when applied to a subset of lines $\integer{j}_1,\ldots,\integer{j}_k$, 
\begin{gproofnn}
\gline{1}{}{}
\glinend{}{}{}
\glinend{}{$\qquad\vdots$}{}
\glinend{}{}{}
\gline{$\integer{n}$}{$\CAPPSI$}{\Rule{R}, $\integer{j}_1,\ldots,\integer{j}_k$}
\end{gproofnn}
then the sequence of derivation lines you get by adding a new line $\integer{n}+1$ with $\integer{k}$ open assumptions and the sentence $\CAPTHETA$ is a derivation,
\begin{gproofnn}
\gline{1}{}{}
\glinend{}{}{}
\glinend{}{$\qquad\vdots$}{}
\glinend{}{}{}
\gline{$\integer{n}$}{$\CAPPSI$}{\Rule{R}, $\integer{j}_1,\ldots,\integer{j}_k$}
\gline{$\integer{n}+1$}{$\CAPTHETA$}{\Rule{R$'$}, $\integer{h}_1,\ldots,\integer{h}_l$}
\end{gproofnn}
so long as $\CAPTHETA$ is sanctioned by some rule \Rule{R$'$} (other than \Rule{$\HORSESHOE$-Intro}) when applied to a subset of previous lines $\integer{h}_1,\ldots,\integer{h}_l$ already in the derivation. 

\item[Case 2:] If you have some $\integer{n}$-line derivation with $\integer{k}$ assumptions still open at line $\integer{n}$, and in which the sentence $\CAPPSI$ on $\integer{n}$ is sanctioned by rule $\Rule{R}$ when applied to a subset of lines $\integer{j}_1,\ldots,\integer{j}_k$, 
\begin{gproofnn}
\gline{1}{}{}
\glinend{}{}{}
\glinend{}{$\qquad\vdots$}{}
\glinend{}{}{}
\galineNC{$\integer{n}$}{$\CAPPSI$\hfill{}}{\Rule{R}, $\integer{j}_1,\ldots,\integer{j}_k$}
\end{gproofnn}
then, for any sentence $\CAPTHETA$, the sequence of derivation lines you get by adding a new line $\integer{n}+1$ with $\integer{k}+1$ open assumptions and $\CAPTHETA$ sanctioned by \Rule{Assume} is a derivation,
\begin{gproofnn}
\gline{1}{}{}
\glinend{}{}{}
\glinend{}{$\qquad\vdots$}{}
\glinend{}{}{}
\galineNC{$\integer{n}$}{$\CAPPSI$}{\Rule{R}, $\integer{j}_1,\ldots,\integer{j}_k$}
\gaalineNC{$\integer{n}+1$}{$\CAPTHETA$\qquad}{\Rule{Assume}}
\end{gproofnn}

\item[Case 3:] If you have some $\integer{n}$-line derivation with $\integer{k}$ assumptions still open at line $\integer{n}$, and in which the sentence $\CAPPSI$ on $\integer{n}$ is sanctioned by rule $\Rule{R}$ when applied to a subset of lines $\integer{j}_1,\ldots,\integer{j}_k$, and in which the $\integer{k}$th assumption was opened on line $\integer{m}$, 
\begin{gproofnn}
\gline{1}{}{}
\glinend{}{}{}
\glinend{}{$\qquad\vdots$}{}
\glinend{}{}{}
\galineNC{$\integer{m}$}{$\CAPPHI$}{\Rule{Assume}}
\galineNCnd{}{}{}
\galineNCnd{}{$\qquad\vdots$}{}
\galineNCnd{}{}{}
\galineNC{$\integer{n}$}{$\CAPPSI$}{\Rule{R}, $\integer{j}_1,\ldots,\integer{j}_k$}
\end{gproofnn}
then the sequence of derivation lines you get by adding a new line $\integer{n}+1$ with $\integer{k}-1$ open assumptions and the sentence $\horseshoe{\CAPPHI}{\CAPPSI}$ is a derivation,
\begin{gproofnn}
\gline{1}{}{}
\glinend{}{}{}
\glinend{}{$\qquad\vdots$}{}
\glinend{}{}{}
\gaproof{
\galine{$\integer{m}$}{$\CAPPHI$}{\Rule{Assume}}
\galinend{}{}{}
\galinend{}{$\quad\vdots$}{}
\galinend{}{}{}
\galine{$\integer{n}$}{$\CAPPSI$}{\Rule{R}, $\integer{j}_1,\ldots,\integer{j}_k$}
}
\gline{$\integer{n}+1$}{$\horseshoe{\CAPPHI}{\CAPPSI}$}{\Rule{$\HORSESHOE$-Intro}, $\integer{m}$--$\integer{n}$}
\end{gproofnn}
so long as you close the assumption opened on line $\integer{m}$ by drawing a box around lines $\integer{m}$--$\integer{n}$ and write down \Rule{$\HORSESHOE$-Intro}, $\integer{m}$--$\integer{n}$ as the rule which sanctions line $\integer{n}+1$. 
\end{description}

\item[Closure Clause:] Nothing else is a derivation of \GSL{}.
\end{description}
\end{majorILnc}
\noindent{}Note that in the generating clauses we specify that we start with a derivation with $\integer{k}$ open assumptions on the last line. 
In cases 1 and 2 the schematic drawings given don't explicitly depict the $\integer{k}$ vertical lines that should be running between the line numbers and the sentences, but the drawings are not intended to suggest that you can write derivations without the running vertical lines which track open assumptions. 
The schematic drawing in case 3 similarly only depicts the last vertical assumption line (the one for the assumption opened on line $\integer{m}$), but that's not to suggest the others aren't there, or that they can be left off.   

\subsection{Restrictions on Applying Rules}\label{Restrictions on Applying Rules}
It's important to be clear on precisely when a rule sanctions\index{derivation!rule}\index{sanctions} writing down a sentence. 
The restriction, which we quietly followed in previous examples, is that a rule can be applied only if the connectives mentioned in the rule are the main connectives of the sentences to which that rule is being applied.
That is, a rule can only be applied to whole sentences on a line---it can't be applied to proper subsentences of a line.

For example, we can only apply \Rule{$\HORSESHOE$-Elim} on two lines of a derivation if the sentence on one of the lines is a conditional $\horseshoe{\CAPPHI}{\CAPTHETA}$ and the sentence on the other line is $\CAPPHI$. 
If $\horseshoe{\CAPPHI}{\CAPTHETA}$ is the sentence on one line and $\CAPPHI$ is merely contained as a subsentence in the sentence on the other (say the sentence has the form $\disjunction{\CAPPHI}{\CAPPSI}$ or $\conjunction{\CAPPHI}{\CAPPSI}$), then we cannot apply \Rule{$\HORSESHOE$-Elim} to those lines. 
Likewise, if a line has a sentence $\CAPPHI$ and the conditional $\horseshoe{\CAPPHI}{\CAPTHETA}$ is merely contained as a subsentence in the sentence on another line (say the sentence has the form $\disjunction{\bparhorseshoe{\CAPPHI}{\CAPTHETA}}{\CAPPSI}$), then we cannot apply \Rule{$\HORSESHOE$-Elim} to those lines.

For a more concrete example, consider derivation \pmvref{secondexamplefinished}. 
Even though $\Bl$ appears on line 1 (as the conjunct of $\conjunction{\Al}{\Bl}$), we cannot apply \Rule{$\HORSESHOE$-Elim} to lines 1 and 2 to get $\conjunction{\Cl}{\Dl}$. 
The fact that we can easily get $\Bl$ on its own line through \Rule{$\WEDGE$-Elim} doesn't matter. 
The rule \Rule{$\HORSESHOE$-Elim} will only sanction writing $\conjunction{\Cl}{\Dl}$ on a line, given $\horseshoe{\Bl}{\parconjunction{\Cl}{\Dl}}$ on line 2, if we have another line with the \CAPS{lhs} of the conditional by itself. 
In just the same way we cannot apply \Rule{$\WEDGE$-Elim} to line 2 of derivation \ref{secondexamplefinished} to get $\Cl$ or $\Dl$, since the conjunction in line 2 is not the main connective. 
Instead, it's the main connective of the \CAPS{rhs} subsentence of $\horseshoe{\Bl}{\parconjunction{\Cl}{\Dl}}$. 
Again it doesn't matter that we can get the conjunct by itself using \Rule{$\HORSESHOE$-Elim} (after using \Rule{$\WEDGE$-Elim} on line 1), the rule \Rule{$\WEDGE$-Elim} cannot be applied to a line unless the sentence on that line is a conjunction. 

\subsection{Decidability}\label{Section:Intro to Decidability}
There are multiple algorithms to follow for applying the rules which, if there does exist a derivation, will halt when the last line written down is the sentence to be derived. 
For \GSD{} the algorithms are intuitive and straightforward (see Sec. \pmvref{Section:Completeness for GSD}), while for \GQD{} (the basic derivation system for \GQL{} we define in Sec. \pmvref{Section GQD}) they are much more complicated.

But although these algorithms are guaranteed to end in a derivation if the sentence can be derived, there are at least two reasons why you don't want to do most of your derivations using them. 
First, the derivations produced by them tend to be much longer and more complicated than is necessary. 
You will almost always be able to come up with a much shorter and more direct proof on your own. 
Second, at least for \GQD{} (but not \GSD{}), although \emph{if} there is a derivation the algorithms will ``find it'', if there is \emph{not} a derivation then the algorithms may never ``find out''. 
That is, if there is not a derivation then the algorithms (any one you pick) do essentially one of two things: either they halt in a way that indicates there is no derivation, or they never halt. 
If you happen to be working on a problem in the latter case and you're only following the algorithm, then you'll never find out whether there is a derivation. 
If a derivation system, like \GQD{}, has this feature, then it's said to be \idf{undecidable}. 
If there is an algorithm that always halts either in a derivation or with an indication that there's no derivation (as in the case of \GSD{}), then the system is said to be \idf{decidable} and the algorithm is said to be a \niidf{decision procedure}\index{decision procedure}.
\GQD{} is undecidable, while \GSD{} is decidable.\footnote{Enderton \citeyearpar{Enderton2010} provides a general, contemporary introduction to computability and decision procedures. Kleene \citeyearpar[ch.~5]{Kleene1967} provides a lucid and concise discussion within roughly the framework devolved here.}
Note that if we restrict \GQL{} to just 1-place predicates, then \GQD{} is decidable.
See section \pmvref{Decidability and Churchs Theorem} for more details and discussion.

\subsection{Some Strategies}\label{Sec:Some Strategies}
We don't want to use these sorts of algorithms to find derivations, if we can avoid it. 
But there are general strategies we do use. 
For each logical connective there are two types of strategies: those for what to do if you already have sentences with that as their main connective, and those for what to do if you want to get a sentence with that as its main connective. 
We call the first top-down strategies and the second bottom-up strategies.
In many cases, doing a derivation is like planning a plane trip. The top-down method is like figuring out the nearest convenient airport from your current location, and the bottom-up method is like figuring out which airport is convenient for getting to your destination. In derivations, sometimes you have to go through intermediate sentences (as with intermediate cities in travel).

We begin with some basic top-down and bottom-up strategies for each connective, adding more later in section \ref{Sec:Shortcut Rule Strategies} when we introduce shortcut rules.

\subsubsection*{Conjunction} 
We start with the basic top-down and bottom-up strategies for conjunction. They are straightforward.
\begin{description}
\item[$\WEDGE\!$ Top-down:] If you have a sentence of the form $\conjunction{\conjunction{\conjunction{\CAPPHI_1}{\CAPPHI_2}}{\ldots}}{\CAPPHI_n}$, then break it apart using \Rule{$\WEDGE$-Elim} to get each of the conjuncts $\CAPPHI_1$ through $\CAPPSI_n$, each on a new line.
\item[$\WEDGE\!$ Bottom-up:] If you want to get a sentence of the form $\conjunction{\conjunction{\conjunction{\CAPPHI_1}{\CAPPHI_2}}{\ldots}}{\CAPPHI_n}$, then derive each of $\CAPPHI_1$ through $\CAPPHI_n$ individually and use \Rule{$\WEDGE$-Intro} to derive it from them. 
\end{description} 
Both strategies are exemplified in example derivation \pmvref{simpleconjunction}. 
There we wanted to derive the sentence $\conjunction{\Cl}{\Bl}$, so in line with the bottom-up strategy for $\!\WEDGE\!$ we first derived both $\Cl$ and $\Bl$ and then used \Rule{$\WEDGE$-Intro} to derive $\conjunction{\Cl}{\Bl}$. 
In line with the top-down strategy, we took our assumption $\conjunction{\Bl}{\Cl}$ and broke it apart using \Rule{$\WEDGE$-Elim} (which got us the sentences, $\Cl$ and $\Bl$, we were looking to derive). 

\subsubsection*{Conditionals}
The basic strategies for conditionals are also straightforward and have already been exemplified. 
\begin{description}
\item[$\HORSESHOE$ Top-down:] If you have a sentence of the form $\horseshoe{\CAPPHI}{\CAPPSI}$, then first derive the \CAPS{lhs} $\CAPPHI$ and then break it apart using \Rule{$\HORSESHOE$-Elim} to get the \CAPS{rhs} $\CAPPSI$ on a new line.
\item[$\HORSESHOE$ Bottom-up:] If you want to get a sentence of the form $\horseshoe{\CAPPHI}{\CAPPSI}$, then assume the \CAPS{lhs} $\CAPPHI$, derive the \CAPS{rhs} $\CAPPSI$, and then use \Rule{$\HORSESHOE$-Intro} to write the conditional on the next line.
\end{description} 
In derivation \pmvref{secondexample}, we wanted to derive $\Dl$. 
We saw that we had a conditional, $\horseshoe{\Bl}{\parconjunction{\Cl}{\Dl}}$. 
In line with the top-down strategy, we derived its \CAPS{lhs} $\Bl$ (in the process using the top-down strategy for $\!\WEDGE\!$), then used \Rule{$\HORSESHOE$-Elim} to get the \CAPS{rhs} $\parconjunction{\Cl}{\Dl}$. 
We wanted $\parconjunction{\Cl}{\Dl}$, of course, because from it we could use \Rule{$\WEDGE$-Elim} to get $\Dl$. 
In derivation \pmvref{secondexamplefinished}, we wanted to derive $\horseshoe{\parconjunction{\Al}{\Bl}}{\cparhorseshoe{\parhorseshoe{\Bl}{\bparconjunction{\Cl}{\Dl}}}{\Dl}}$. 
In line with the bottom-up strategy, we assumed the \CAPS{lhs} $\parconjunction{\Al}{\Bl}$, derived the \CAPS{rhs} $\cparhorseshoe{\parhorseshoe{\Bl}{\bparconjunction{\Cl}{\Dl}}}{\Dl}$, and then used \Rule{$\HORSESHOE$-Intro} to write the conditional on the next line.

\subsubsection*{Biconditionals}
The basic strategies for biconditionals are similar to those for conditionals, as one might expect.
\begin{description}
\item[$\TRIPLEBAR$ Top-down:] If you have a sentence of the form $\triplebar{\CAPPHI}{\CAPPSI}$, then either 
\begin{enumerate}
\item first derive the \CAPS{lhs} $\CAPPHI$ and then break it apart using \Rule{$\TRIPLEBAR$-Elim} to get the \CAPS{rhs} $\CAPPSI$ on a new line,
\item first derive the \CAPS{rhs} $\CAPPSI$ and then break it apart using \Rule{$\TRIPLEBAR$-Elim} to get the \CAPS{lhs} $\CAPPHI$ on a new line,
\item or do both.
\end{enumerate}
\item[$\TRIPLEBAR$ Bottom-up:] If you want to get a sentence of the form $\triplebar{\CAPPHI}{\CAPPSI}$, then first derive both $\horseshoe{\CAPPHI}{\CAPPSI}$ and $\horseshoe{\CAPPSI}{\CAPPHI}$ and then use \Rule{$\TRIPLEBAR$-Intro} to write the biconditional on the next line.
\end{description}

\subsubsection*{Negations}
In the case of negations, we don't have a basic top-down strategy, only a basic bottom-up.  Later in this chapter we develop shortcut rules which allow us to provide top-down strategies for negation.
\begin{description}
\item[$\NEGATION$ Bottom-up:] If you want to get a sentence of the form $\negation{\CAPPHI}$, then first assume $\CAPPHI$, derive a contradiction $\conjunction{\CAPPSI}{\negation{\CAPPSI}}$, and then in two separate steps use \Rule{$\HORSESHOE$-Intro} and \Rule{$\NEGATION$-Intro} to write the negation on the next line.
\end{description}
For example, say we want to derive the sentence $\negation{\parconjunction{\bparhorseshoe{\Al}{\negation{\Bl}}}{\bparconjunction{\Al}{\Bl}}}$. In line with the basic bottom-up strategy for $\NEGATION$, we first assume $\parconjunction{\bparhorseshoe{\Al}{\negation{\Bl}}}{\bparconjunction{\Al}{\Bl}}$ and try to derive a contradiction:
\begin{gproof}
\galineNC{1}{$\parconjunction{\bparhorseshoe{\Al}{\negation{\Bl}}}{\bparconjunction{\Al}{\Bl}}$}{\Rule{Assume}}
\galineNCnd{}{}{}
\galineNCnd{}{$\qquad\vdots$}{}
\galineNCnd{}{}{}
\galineNC{$\integer{n}$}{$\conjunction{\CAPPSI}{\negation{\CAPPSI}}$}{}
\end{gproof}
\noindent{}It should be clear that we can derive $\conjunction{\Bl}{\negation{\Bl}}$, so that is our goal:
\begin{gproof}
\galineNC{1}{$\parconjunction{\bparhorseshoe{\Al}{\negation{\Bl}}}{\bparconjunction{\Al}{\Bl}}$}{\Rule{Assume}}
\galineNCnd{}{}{}
\galineNCnd{}{$\qquad\vdots$}{}
\galineNCnd{}{}{}
\galineNC{$\integer{n}$}{$\conjunction{\Bl}{\negation{\Bl}}$}{}
\end{gproof}
The bottom-up strategy for $\!\WEDGE\!$ says to get this we should derive both $\Bl$ and $\negation{\Bl}$.
\begin{gproof}
\galineNC{1}{$\parconjunction{\bparhorseshoe{\Al}{\negation{\Bl}}}{\bparconjunction{\Al}{\Bl}}$}{\Rule{Assume}}
\galineNCnd{}{}{}
\galineNCnd{}{$\qquad\vdots$}{}
\galineNCnd{}{}{}
\galineNC{$\integer{n}-2$}{$\Bl$}{}
\galineNC{$\integer{n}-1$}{$\negation{\Bl}$}{}
\galineNC{$\integer{n}$}{$\conjunction{\Bl}{\negation{\Bl}}$}{\Rule{$\WEDGE$-Intro}, $\integer{n}-$1,$\integer{n}-$2}
\end{gproof}
\noindent{}The top-down strategy for $\!\WEDGE\!$ is our only option at this point, so we break line 1 apart using \Rule{$\WEDGE$-Elim}.
\begin{gproof}
\galineNC{1}{$\parconjunction{\bparhorseshoe{\Al}{\negation{\Bl}}}{\bparconjunction{\Al}{\Bl}}$}{\Rule{Assume}}
\galineNC{2}{$\bparhorseshoe{\Al}{\negation{\Bl}}$}{\Rule{$\WEDGE$-Elim}, 1}
\galineNC{3}{$\bparconjunction{\Al}{\Bl}$}{\Rule{$\WEDGE$-Elim}, 1}
\galineNCnd{}{}{}
\galineNCnd{}{$\qquad\vdots$}{}
\galineNCnd{}{}{}
\galineNC{$\integer{n}-2$}{$\Bl$}{}
\galineNC{$\integer{n}-1$}{$\negation{\Bl}$}{}
\galineNC{$\integer{n}$}{$\conjunction{\Bl}{\negation{\Bl}}$}{\Rule{$\WEDGE$-Intro}, $\integer{n}-$1,$\integer{n}-$2}
\end{gproof}
\noindent{}Now we continue to work top-down, using \Rule{$\WEDGE$-Elim} to break apart line 3. Note that in this step we've partway joined up the top and bottom of the proof, since breaking apart line 3 gets us what we were calling line $\integer{n}-$2.
\begin{gproof}
\galineNC{1}{$\parconjunction{\bparhorseshoe{\Al}{\negation{\Bl}}}{\bparconjunction{\Al}{\Bl}}$}{\Rule{Assume}}
\galineNC{2}{$\bparhorseshoe{\Al}{\negation{\Bl}}$}{\Rule{$\WEDGE$-Elim}, 1}
\galineNC{3}{$\bparconjunction{\Al}{\Bl}$}{\Rule{$\WEDGE$-Elim}, 1}
\galineNC{4}{$\Al$}{\Rule{$\WEDGE$-Elim}, 3}
\galineNC{5}{$\Bl$}{\Rule{$\WEDGE$-Elim}, 3}
\galineNCnd{}{}{}
\galineNCnd{}{$\qquad\vdots$}{}
\galineNCnd{}{}{}
\galineNC{$\integer{n}-1$}{$\negation{\Bl}$}{}
\galineNC{$\integer{n}$}{$\conjunction{\Bl}{\negation{\Bl}}$}{\Rule{$\WEDGE$-Intro}, $\integer{n}-$1,$\integer{n}-$2}
\end{gproof}
\noindent{}Next we see that we can work bottom-down from lines 2 and 4, breaking the conditional on line 2 apart. In doing so we finish the proof, since the result of doing this is $\negation{\Bl}$, which is all that was left to get the contradiction. 
\begin{gproof}[\label{helpful1}]
\galineNC{1}{$\parconjunction{\bparhorseshoe{\Al}{\negation{\Bl}}}{\bparconjunction{\Al}{\Bl}}$}{\Rule{Assume}}
\galineNC{2}{$\bparhorseshoe{\Al}{\negation{\Bl}}$}{\Rule{$\WEDGE$-Elim}, 1}
\galineNC{3}{$\bparconjunction{\Al}{\Bl}$}{\Rule{$\WEDGE$-Elim}, 1}
\galineNC{4}{$\Al$}{\Rule{$\WEDGE$-Elim}, 3}
\galineNC{5}{$\Bl$}{\Rule{$\WEDGE$-Elim}, 3}
\galineNC{6}{$\negation{\Bl}$}{\Rule{$\HORSESHOE$-Elim}, 2,4}
\galineNC{7}{$\conjunction{\Bl}{\negation{\Bl}}$}{\Rule{$\WEDGE$-Intro}, 5,6}
\end{gproof}
Of course, we haven't yet derived $\negation{\!\parconjunction{\bparhorseshoe{\Al}{\negation{\Bl}}}{\bparconjunction{\Al}{\Bl}}}$, but we can now do so by discharging the assumption through \Rule{$\HORSESHOE$-Intro} and then applying \Rule{$\NEGATION$-Intro}.
\begin{gproof}
\gaproof{
\galine{1}{$\parconjunction{\bparhorseshoe{\Al}{\negation{\Bl}}}{\bparconjunction{\Al}{\Bl}}$}{\Rule{Assume}}
\galine{2}{$\bparhorseshoe{\Al}{\negation{\Bl}}$}{\Rule{$\WEDGE$-Elim}, 1}
\galine{3}{$\bparconjunction{\Al}{\Bl}$}{\Rule{$\WEDGE$-Elim}, 1}
\galine{4}{$\Al$}{\Rule{$\WEDGE$-Elim}, 3}
\galine{5}{$\Bl$}{\Rule{$\WEDGE$-Elim}, 3}
\galine{6}{$\negation{\Bl}$}{\Rule{$\HORSESHOE$-Elim}, 2,4}
\galine{7}{$\conjunction{\Bl}{\negation{\Bl}}$}{\Rule{$\WEDGE$-Intro}, 5,6}
}
\gline{8}{$\horseshoe{\parconjunction{\bparhorseshoe{\Al}{\negation{\Bl}}}{\bparconjunction{\Al}{\Bl}}}{\cparconjunction{\Bl}{\negation{\Bl}}}$}{$\HORSESHOE$-Intro, 1--7}
\gline{9}{$\negation{\parconjunction{\bparhorseshoe{\Al}{\negation{\Bl}}}{\bparconjunction{\Al}{\Bl}}}$}{\Rule{$\NEGATION$-Intro}, 8}
\end{gproof}

\subsubsection*{Disjunctions}
Our last pair of strategies is for disjunctions. As with conjunctions, we give the strategies for the case where there are only two disjuncts. Generalizing the strategies for disjunctions with more than two disjuncts is left to the reader.
\begin{description}
\item[$\VEE$ Top-down:] If you have a sentence of the form $\disjunction{\CAPPHI}{\CAPPSI}$ and you want to derive a sentence $\CAPTHETA$, first derive the conditionals $\horseshoe{\CAPPHI}{\CAPTHETA}$ and $\horseshoe{\CAPPSI}{\CAPTHETA}$, and then use \Rule{$\VEE$-Elim} to write down $\CAPTHETA$ on the next line. The order in which you derive the intermediate conditionals doesn't matter.
\item[$\VEE$ Bottom-up:] If you want a sentence of the form $\disjunction{\CAPPHI}{\CAPPSI}$, then first derive either $\CAPPHI$ or derive $\CAPPSI$, and then use \Rule{$\VEE$-Intro} to write it down.
\end{description}
The basic bottom-up strategy isn't always the right tool for deriving disjunctions, because usually you can't derive one of the disjuncts. 
This is an important point: it might be that a disjunction is derivable even if neither disjunct is. 
It's important that we can derive disjunctions without first deriving one or the other disjunct, since $\disjunction{\Bl}{\negation{\Bl}}$ is a logical truth. We would like to be able to derive it, though neither $\Bl$ nor $\negation{\Bl}$ is a logical truth.

Later on we get more useful bottom-up strategies for disjunctions. For now we focus on the top-down strategy. 

As an example, say we want to derive $\negation{\parconjunction{\bpardisjunction{\negation{\Al}}{\negation{\Bl}}}{\bparconjunction{\Al}{\Bl}}}$. 
The whole sentence itself is a negation, so we start just as in the last example. 
So we need some contradiction to aim for. 
We try to derive $\conjunction{\Bl}{\negation{\Bl}}$.
\begin{gproof}
\galineNC{1}{$\parconjunction{\bpardisjunction{\negation{\Al}}{\negation{\Bl}}}{\bparconjunction{\Al}{\Bl}}$}{\Rule{Assume}}
\galineNCnd{}{}{}
\galineNCnd{}{$\qquad\vdots$}{}
\galineNCnd{}{}{}
\galineNC{$\integer{n}$}{$\conjunction{\Bl}{\negation{\Bl}}$}{}
\end{gproof}
As in the last example the bottom-up strategy for $\!\WEDGE\!$ has us try to derive both $\Bl$ and $\negation{\Bl}$.
\begin{gproof}
\galineNC{1}{$\parconjunction{\bpardisjunction{\negation{\Al}}{\negation{\Bl}}}{\bparconjunction{\Al}{\Bl}}$}{\Rule{Assume}}
\galineNCnd{}{}{}
\galineNCnd{}{$\qquad\vdots$}{}
\galineNCnd{}{}{}
\galineNC{$\integer{n}-2$}{$\Bl$}{}
\galineNC{$\integer{n}-1$}{$\negation{\Bl}$}{}
\galineNC{$\integer{n}$}{$\conjunction{\Bl}{\negation{\Bl}}$}{\Rule{$\WEDGE$-Intro}, $\integer{n}-2$,$\integer{n}-1$}
\end{gproof}
And, the top-down strategy leads us to break apart the conjunction on line 1, which leads to another conjunction to break apart as well. This gets us one of the conjuncts of line $\integer{n}$ in the process.
\begin{gproof}
\galineNC{1}{$\parconjunction{\bpardisjunction{\negation{\Al}}{\negation{\Bl}}}{\bparconjunction{\Al}{\Bl}}$}{\Rule{Assume}}
\galineNC{2}{$\bpardisjunction{\negation{\Al}}{\negation{\Bl}}$}{\Rule{$\WEDGE$-Elim}, 1}
\galineNC{3}{$\bparconjunction{\Al}{\Bl}$}{\Rule{$\WEDGE$-Elim}, 1}
\galineNC{4}{$\Bl$}{\Rule{$\WEDGE$-Elim}, 3}
\galineNCnd{}{}{}
\galineNCnd{}{$\qquad\vdots$}{}
\galineNCnd{}{}{}
\galineNC{$\integer{n}-1$}{$\negation{\Bl}$}{\Rule{$\NEGATION$-Intro}, $\integer{n}-2$}
\galineNC{$\integer{n}$}{$\conjunction{\Bl}{\negation{\Bl}}$}{\Rule{$\WEDGE$-Intro}, $\integer{n}-2$,$\integer{n}-1$}
\end{gproof}
Now we only need to get $\negation{\Bl}$. 
To do this, we follow the bottom-up strategy for negation. 
We assume $\Bl$ and try to get a contradiction.
\begin{gproof}
\galineNC{1}{$\parconjunction{\bpardisjunction{\negation{\Al}}{\negation{\Bl}}}{\bparconjunction{\Al}{\Bl}}$}{\Rule{Assume}}
\galineNC{2}{$\bpardisjunction{\negation{\Al}}{\negation{\Bl}}$}{\Rule{$\WEDGE$-Elim}, 1}
\galineNC{3}{$\bparconjunction{\Al}{\Bl}$}{\Rule{$\WEDGE$-Elim}, 1}
\galineNC{4}{$\Bl$}{\Rule{$\WEDGE$-Elim}, 3}
\gaaproof{
\gaalineNCS{5}{$\Bl$}{\Rule{Assume}}
\gaalineNCndS{}{}{}
\gaalineNCndS{}{$\qquad\vdots$}{}
\gaalineNCndS{}{}{}
\gaalineNCS{$\integer{n}-3$}{$\conjunction{\CAPPSI}{\negation{\CAPPSI}}$}{}
}
\galineNC{$\integer{n}-2$}{$\horseshoe{\Bl}{\parconjunction{\CAPPSI}{\negation{\CAPPSI}}}$}{\Rule{$\HORSESHOE$-Intro}, 5--$\integer{n}-3$}
\galineNC{$\integer{n}-1$}{$\negation{\Bl}$}{\Rule{$\NEGATION$-Intro}, $\integer{n}-2$}
\galineNC{$\integer{n}$}{$\conjunction{\Bl}{\negation{\Bl}}$}{\Rule{$\WEDGE$-Intro}, $\integer{n}-2$,$\integer{n}-1$}
\end{gproof}
Note that although it may look wrong to have an assumption line for $\Bl$ when we have already derived it, there's nothing wrong with this derivation. 
You can always assume whatever you want, even if you already have it. 
Now we have two questions to think about: what contradiction could we get on line $\integer{n}-3$? 
And, how do we get it? 
The second question is straightforward. 
We want to get a contradiction, and at this point it's going to have to come from the disjunction on line 2. 
So we follow the top-down strategy for $\VEE$ which tells us how to get a sentence from a disjunction.
\begin{gproof}
\galineNC{1}{$\parconjunction{\bpardisjunction{\negation{\Al}}{\negation{\Bl}}}{\bparconjunction{\Al}{\Bl}}$}{\Rule{Assume}}
\galineNC{2}{$\bpardisjunction{\negation{\Al}}{\negation{\Bl}}$}{\Rule{$\WEDGE$-Elim}, 1}
\galineNC{3}{$\bparconjunction{\Al}{\Bl}$}{\Rule{$\WEDGE$-Elim}, 1}
\galineNC{4}{$\Bl$}{\Rule{$\WEDGE$-Elim}, 3}
\gaaproof{
\gaalineNCS{5}{$\Bl$}{\Rule{Assume}}
\gaalineNCndS{}{}{}
\gaalineNCndS{}{$\qquad\vdots$}{}
\gaalineNCndS{}{}{}
\gaalineNCS{$\integer{n}-5$}{$\horseshoe{\negation{\Bl}}{\parconjunction{\CAPPSI}{\negation{\CAPPSI}}}$}{}
\gaalineNCS{$\integer{n}-4$}{$\horseshoe{\negation{\Al}}{\parconjunction{\CAPPSI}{\negation{\CAPPSI}}}$}{}
\gaalineNCS{$\integer{n}-3$}{$\conjunction{\CAPPSI}{\negation{\CAPPSI}}$}{\Rule{$\VEE$-Elim}, 2,$\integer{n}-5$,$\integer{n}-4$}
}
\galineNC{$\integer{n}-2$}{$\horseshoe{\Bl}{\parconjunction{\CAPPSI}{\negation{\CAPPSI}}}$}{\Rule{$\HORSESHOE$-Intro}, 5--$\integer{n}-3$}
\galineNC{$\integer{n}-1$}{$\negation{\Bl}$}{\Rule{$\NEGATION$-Intro}, $\integer{n}-2$}
\galineNC{$\integer{n}$}{$\conjunction{\Bl}{\negation{\Bl}}$}{\Rule{$\WEDGE$-Intro}, $\integer{n}-2$,$\integer{n}-1$}
\end{gproof}
To get the conditionals on lines $\integer{n}-5$ and $\integer{n}-4$, we have to use \Rule{$\HORSESHOE$-Into}. So the setup is:
\begin{gproof}
\galineNC{1}{$\parconjunction{\bpardisjunction{\negation{\Al}}{\negation{\Bl}}}{\bparconjunction{\Al}{\Bl}}$}{\Rule{Assume}}
\galineNC{2}{$\bpardisjunction{\negation{\Al}}{\negation{\Bl}}$}{\Rule{$\WEDGE$-Elim}, 1}
\galineNC{3}{$\bparconjunction{\Al}{\Bl}$}{\Rule{$\WEDGE$-Elim}, 1}
\galineNC{4}{$\Bl$}{\Rule{$\WEDGE$-Elim}, 3}
\gaaproof{
\gaalineNCS{5}{$\Bl$}{\Rule{Assume}}
\gaaaproof{
\gaaalineS{6}{$\negation{\Bl}$}{\Rule{Assume}}
\gaaalinendS{}{}{}
\gaaalinendS{}{$\qquad\vdots$}{}
\gaaalinendS{}{}{}
\gaaalineS{$\integer{m}$}{$\parconjunction{\CAPPSI}{\negation{\CAPPSI}}$}{}
}
\gaalineNCS{$\integer{m}+1$}{$\horseshoe{\negation{\Bl}}{\parconjunction{\CAPPSI}{\negation{\CAPPSI}}}$}{\Rule{$\HORSESHOE$-Intro}, 6--$\integer{m}$}
\gaaaproof{
\gaaalineS{$\integer{m}+2$}{$\negation{\Al}$}{\Rule{Assume}}
\gaaalinendS{}{}{}
\gaaalinendS{}{$\qquad\vdots$}{}
\gaaalinendS{}{}{}
\gaaalineS{$\integer{n}-5$}{$\parconjunction{\CAPPSI}{\negation{\CAPPSI}}$}{}
}
\gaalineNCS{$\integer{n}-4$}{$\horseshoe{\negation{\Al}}{\parconjunction{\CAPPSI}{\negation{\CAPPSI}}}$}{\Rule{$\HORSESHOE$-Intro}, $\integer{m}+2$--$\integer{n}-5$}
\gaalineNCS{$\integer{n}-3$}{$\conjunction{\CAPPSI}{\negation{\CAPPSI}}$}{\Rule{$\VEE$-Elim}, 2,$\integer{m}+1$,$\integer{n}-4$}
}
\galineNC{$\integer{n}-2$}{$\horseshoe{\Bl}{\parconjunction{\CAPPSI}{\negation{\CAPPSI}}}$}{\Rule{$\HORSESHOE$-Intro}, 5--$\integer{n}-3$}
\galineNC{$\integer{n}-1$}{$\negation{\Bl}$}{\Rule{$\NEGATION$-Intro}, $\integer{n}-2$}
\galineNC{$\integer{n}$}{$\conjunction{\Bl}{\negation{\Bl}}$}{\Rule{$\WEDGE$-Intro}, $\integer{n}-2$,$\integer{n}-1$}
\end{gproof}
Now we can decide what contradiction $\parconjunction{\CAPPSI}{\negation{\CAPPSI}}$ to aim for. If we choose $\parconjunction{\Bl}{\negation{\Bl}}$ again, then the first conditional is easy. We are able to get it by using \Rule{$\WEDGE$-Intro} on lines 5 and 6.
\begin{gproof}
\galineNC{1}{$\parconjunction{\bpardisjunction{\negation{\Al}}{\negation{\Bl}}}{\bparconjunction{\Al}{\Bl}}$}{\Rule{Assume}}
\galineNC{2}{$\bpardisjunction{\negation{\Al}}{\negation{\Bl}}$}{\Rule{$\WEDGE$-Elim}, 1}
\galineNC{3}{$\bparconjunction{\Al}{\Bl}$}{\Rule{$\WEDGE$-Elim}, 1}
\galineNC{4}{$\Bl$}{\Rule{$\WEDGE$-Elim}, 3}
\gaaproof{
\gaalineNCS{5}{$\Bl$}{\Rule{Assume}}
\gaaaproof{
\gaaalineS{6}{$\negation{\Bl}$}{\Rule{Assume}}
\gaaalineS{7}{$\parconjunction{\Bl}{\negation{\Bl}}$}{\Rule{$\WEDGE$-Intro}, 5,6}
}
\gaalineNCS{8}{$\horseshoe{\negation{\Bl}}{\parconjunction{\Bl}{\negation{\Bl}}}$}{\Rule{$\HORSESHOE$-Intro}, 6--7}
\gaaaproof{
\gaaalineS{9}{$\negation{\Al}$}{\Rule{Assume}}
\gaaalinendS{}{}{}
\gaaalinendS{}{$\qquad\vdots$}{}
\gaaalinendS{}{}{}
\gaaalineS{$\integer{n}-5$}{$\parconjunction{\Bl}{\negation{\Bl}}$}{}
}
\gaalineNCS{$\integer{n}-4$}{$\horseshoe{\negation{\Al}}{\parconjunction{\Bl}{\negation{\Bl}}}$}{\Rule{$\HORSESHOE$-Intro}, 9--$\integer{n}-5$}
\gaalineNCS{$\integer{n}-3$}{$\conjunction{\Bl}{\negation{\Bl}}$}{\Rule{$\VEE$-Elim}, 2,8,$\integer{n}-4$}
}
\galineNC{$\integer{n}-2$}{$\horseshoe{\Bl}{\parconjunction{\Bl}{\negation{\Bl}}}$}{\Rule{$\HORSESHOE$-Intro}, 5--$\integer{n}-3$}
\galineNC{$\integer{n}-1$}{$\negation{\Bl}$}{\Rule{$\NEGATION$-Intro}, $\integer{n}-2$}
\galineNC{$\integer{n}$}{$\conjunction{\Bl}{\negation{\Bl}}$}{\Rule{$\WEDGE$-Intro}, $\integer{n}-2$,$\integer{n}-1$}
\end{gproof}
This leaves us with just the second conditional, deriving $\parconjunction{\Bl}{\negation{\Bl}}$ from $\negation{\Al}$. At first glance this may seem impossible. 
We were able to derive $\parconjunction{\Bl}{\negation{\Bl}}$ from $\negation{\Bl}$ because, given that we already had $\Bl$, assuming $\negation{\Bl}$ allowed us to use \Rule{$\WEDGE$-Intro}. 
But $\negation{\Al}$ obviously isn't sufficient to allow us to use \Rule{$\WEDGE$-Intro}, and there's no clear way to get what we need for \Rule{$\WEDGE$-Intro}, $\negation{\Bl}$, from $\negation{\Al}$. 
So we won't be able to get $\parconjunction{\Bl}{\negation{\Bl}}$ by using \Rule{$\WEDGE$-Intro}. 

Now so far we only have one strategy for getting a conjunction and that's to derive the conjuncts and use \Rule{$\WEDGE$-Intro}. 
But there's another strategy, not tied to any particular connective, which we can use here. 
This strategy is to use \Rule{$\NEGATION$-Elim}. 
We assume $\negation{\parconjunction{\Bl}{\negation{\Bl}}}$, derive a contradiction (any will do), and then use \Rule{$\HORSESHOE$-Intro} to get what we need. 
It should be reasonably clear that this strategy will work, since we obviously can derive the contradiction $\parconjunction{\Al}{\negation{\Al}}$.
\begin{gproof}[\label{bycontradiction}]
\galineNC{1}{$\parconjunction{\bpardisjunction{\negation{\Al}}{\negation{\Bl}}}{\bparconjunction{\Al}{\Bl}}$}{\Rule{Assume}}
\galineNC{2}{$\bpardisjunction{\negation{\Al}}{\negation{\Bl}}$}{\Rule{$\WEDGE$-Elim}, 1}
\galineNC{3}{$\bparconjunction{\Al}{\Bl}$}{\Rule{$\WEDGE$-Elim}, 1}
\galineNC{4}{$\Bl$}{\Rule{$\WEDGE$-Elim}, 3}
\gaaproof{
\gaalineNCS{5}{$\Bl$}{\Rule{Assume}}
\gaaaproof{
\gaaalineS{6}{$\negation{\Bl}$}{\Rule{Assume}}
\gaaalineS{7}{$\parconjunction{\Bl}{\negation{\Bl}}$}{\Rule{$\WEDGE$-Intro}, 5,6}
}
\gaalineNCS{8}{$\horseshoe{\negation{\Bl}}{\parconjunction{\Bl}{\negation{\Bl}}}$}{\Rule{$\HORSESHOE$-Intro}, 6--7}
\gaaaproof{
\gaaalineS{9}{$\negation{\Al}$}{\Rule{Assume}}
\gaaaaproof{
\gaaaalineS{10}{$\negation{\parconjunction{\Bl}{\negation{\Bl}}}$}{\Rule{Assume}}
\gaaaalineS{11}{$\Al$}{\Rule{$\WEDGE$-Elim}, 3}
\gaaaalineS{12}{$\conjunction{\Al}{\negation{\Al}}$}{\Rule{$\WEDGE$-Intro}, 9,11}
}
\gaaalineS{13}{$\horseshoe{\negation{\parconjunction{\Bl}{\negation{\Bl}}}}{\parconjunction{\Al}{\negation{\Al}}}$}{\Rule{$\HORSESHOE$-Intro}, 10--12}
\gaaalineS{14}{$\parconjunction{\Bl}{\negation{\Bl}}$}{\Rule{$\NEGATION$-Elim}, 13}
}
\gaalineNCS{15}{$\horseshoe{\negation{\Al}}{\parconjunction{\Bl}{\negation{\Bl}}}$}{\Rule{$\HORSESHOE$-Intro}, 9--14}
\gaalineNCS{16}{$\conjunction{\Bl}{\negation{\Bl}}$}{\Rule{$\VEE$-Elim}, 2,8,15}
}
\galineNC{17}{$\horseshoe{\Bl}{\parconjunction{\Bl}{\negation{\Bl}}}$}{\Rule{$\HORSESHOE$-Intro}, 5--16}
\galineNC{18}{$\negation{\Bl}$}{\Rule{$\NEGATION$-Intro}, 17}
\galineNC{19}{$\conjunction{\Bl}{\negation{\Bl}}$}{\Rule{$\WEDGE$-Intro}, 4,18}
\end{gproof}
Note that we didn't actually use the assumption $\negation{\parconjunction{\Bl}{\negation{\Bl}}}$ from line 10 in deriving the contradiction we end with on line 12, $\conjunction{\Al}{\negation{\Al}}$. There is nothing wrong with this, since \Rule{$\HORSESHOE$-Intro} doesn't require that the sentence $\CAPTHETA$ with which you started is actually used in deriving the sentence $\CAPPSI$ with which you finished. 

Now that we've derived a contradiction from $\parconjunction{\bpardisjunction{\negation{\Al}}{\negation{\Bl}}}{\bparconjunction{\Al}{\Bl}}$ we can finish the derivation by discharging the assumption with \Rule{$\HORSESHOE$-Intro} and then use \Rule{$\NEGATION$-Intro}.
\begin{gproof}[\label{cangetlong}]
\gaproof{
\galine{1}{$\parconjunction{\bpardisjunction{\negation{\Al}}{\negation{\Bl}}}{\bparconjunction{\Al}{\Bl}}$}{\Rule{Assume}}
\galine{2}{$\bpardisjunction{\negation{\Al}}{\negation{\Bl}}$}{\Rule{$\WEDGE$-Elim}, 1}
\galine{3}{$\bparconjunction{\Al}{\Bl}$}{\Rule{$\WEDGE$-Elim}, 1}
\galine{4}{$\Bl$}{\Rule{$\WEDGE$-Elim}, 3}
\gaaproof{
\gaaline{5}{$\Bl$}{\Rule{Assume}}
\gaaaproof{
\gaaaline{6}{$\negation{\Bl}$}{\Rule{Assume}}
\gaaaline{7}{$\parconjunction{\Bl}{\negation{\Bl}}$}{\Rule{$\WEDGE$-Intro}, 5,6}
}
\gaaline{8}{$\horseshoe{\negation{\Bl}}{\parconjunction{\Bl}{\negation{\Bl}}}$}{\Rule{$\HORSESHOE$-Intro}, 6--7}
\gaaaproof{
\gaaaline{9}{$\negation{\Al}$}{\Rule{Assume}}
\gaaaaproof{
\gaaaaline{10}{$\negation{\parconjunction{\Bl}{\negation{\Bl}}}$}{\Rule{Assume}}
\gaaaaline{11}{$\Al$}{\Rule{$\WEDGE$-Elim}, 3}
\gaaaaline{12}{$\conjunction{\Al}{\negation{\Al}}$}{\Rule{$\WEDGE$-Intro}, 9,11}
}
\gaaaline{13}{$\horseshoe{\negation{\parconjunction{\Bl}{\negation{\Bl}}}}{\parconjunction{\Al}{\negation{\Al}}}$}{\Rule{$\HORSESHOE$-Intro}, 10--12}
\gaaaline{14}{$\parconjunction{\Bl}{\negation{\Bl}}$}{\Rule{$\NEGATION$-Elim}, 13}
}
\gaaline{15}{$\horseshoe{\negation{\Al}}{\parconjunction{\Bl}{\negation{\Bl}}}$}{\Rule{$\HORSESHOE$-Intro}, 9--14}
\gaaline{16}{$\conjunction{\Bl}{\negation{\Bl}}$}{\Rule{$\VEE$-Elim}, 2,8,15}
}
\galine{17}{$\horseshoe{\Bl}{\parconjunction{\Bl}{\negation{\Bl}}}$}{\Rule{$\HORSESHOE$-Intro}, 5--16}
\galine{18}{$\negation{\Bl}$}{\Rule{$\NEGATION$-Intro}, 17}
\galine{19}{$\conjunction{\Bl}{\negation{\Bl}}$}{\Rule{$\WEDGE$-Intro}, 4,18}
}
\gline{20}{$\horseshoe{\parconjunction{\bpardisjunction{\negation{\Al}}{\negation{\Bl}}}{\bparconjunction{\Al}{\Bl}}}{\parconjunction{\Bl}{\negation{\Bl}}}$}{\Rule{$\HORSESHOE$-Intro}, 1--19}
\gline{21}{$\negation{\parconjunction{\bpardisjunction{\negation{\Al}}{\negation{\Bl}}}{\bparconjunction{\Al}{\Bl}}}$}{\Rule{$\NEGATION$-Intro}, 20}
\end{gproof}

\subsubsection*{Proof by Contradiction}
A general strategy involving negation was used in derivation \pmvref{cangetlong}. 
This strategy formalizes an informal proof method often called proof by contradiction. 
In proof by contradiction, one proves that some sentence $\CAPPHI$ is true by assuming it's false and showing that a contradiction follows. 
The corresponding strategy in \GSD{} is:
\begin{description}
\item[Proof by Contradiction:] If you want to get some sentence $\CAPPHI$, then first derive $\horseshoe{\negation{\CAPPHI}}{\parconjunction{\CAPPSI}{\negation{\CAPPSI}}}$ and then use \Rule{$\NEGATION$-Elim} to get $\CAPPHI$ from this conditional.  
\end{description}
This strategy has the virtue that if you can derive $\CAPPHI$ (without any assumptions), then you are able to derive a contradiction $\parconjunction{\CAPPSI}{\negation{\CAPPSI}}$ from $\negation{\CAPPHI}$ as an assumption. In other words, the strategy will always work. But despite this, there are usually much faster and better ways to derive a sentence. 
For example, write a derivation of $\conjunction{\Cl}{\Bl}$ from $\conjunction{\Bl}{\Cl}$ using proof by contradiction and then compare it to derivation \pmvref{simpleconjunction}. 
There are some cases where proof by contradiction is the \emph{only} strategy that will work. 
So there are two cases where it is a good idea to use the strategy: cases where all other available strategies haven't worked (or where there aren't any other available strategies), and cases where you can see a straightforward way to use it. 

\subsubsection*{Top-down and Bottom-up} Finally, the reader should reflect on the general method we have been using. In slogan form, the method goes: work top-down \emph{and} bottom-up. When writing derivation it's important to work not only from the assumptions, seeing how you can move down from them to the conclusion using the rules, but also to work up from the sentence you're trying to derive, looking for the different ways you can get there. Hence the grouping of strategies into top-down and bottom-up. The top-down strategies help guide what moves you can make as you work from the assumptions to the conclusion, while the bottom-up strategies help see what different paths there are to get to that conclusion. 

As an aside, the top-down and bottom-up method is something useful outside of writing formal derivations. While most arguments, including those found in philosophy, mathematical proofs, and formal derivations, are presented as a series of steps from premises to conclusion, they are seldom devised that way. Usually a mathematician starts with a conjecture and tries to work ``up'' from it to results that have already been proven. Rarely does one from the start know what premises are needed to prove a conjecture. Something similar goes for philosophy and other disciplines that rely on argument. So, the reader can think of the top-down, bottom-up method used in derivations as a reflection of how informal arguments and proofs are constructed in philosophy, mathematics, and other disciplines.

%%%%%%%%%%%%%%%%%%%%%%%%%%%%%%%%%%%%%%%%%%%%%%%%%%
\section{Shortcut Rules for \GSD{}}
%%%%%%%%%%%%%%%%%%%%%%%%%%%%%%%%%%%%%%%%%%%%%%%%%%

\subsection{Standard Shortcut Rules}\label{Standard Shortcut Rules GSD}
From derivation \pmvref{cangetlong}, we can see that derivations in \GSD{} of even simple looking sentences can be long and involve roundabout strategies. 
This is the main reason why we want to introduce shortcut rules. 
Shortcut rules can be thought of as ways of cutting out parts of derivations that we find ourselves doing repeatedly. 
The basic rules of \GSD{} plus the shortcut rules (both those in table \pncmvref{GSDplus1} and table \pncmvref{GSDplus2}) make up the derivation system which we call \GSDP{}.

For example, consider lines 10--14 of proof \pmvref{cangetlong}. Here we have two sentences, $\Al$ and $\negation{\Al}$, and we used them to derive the sentence $\conjunction{\Bl}{\negation{\Bl}}$. 
We can rewrite the relevant parts of the proof here, putting them in a less idiosyncratic order:
\begin{gproof}
\galineNC{1}{$\Al$}{}
\galineNC{2}{$\negation{\Al}$}{}
\gaaproof{
\gaalineNCS{3}{$\negation{\parconjunction{\Bl}{\negation{\Bl}}}$}{\Rule{Assume}}
\gaalineNCS{4}{$\conjunction{\Al}{\negation{\Al}}$}{\Rule{$\WEDGE\!$-Intro, 1,2}}
}
\galineNC{5}{$\horseshoe{\negation{\parconjunction{\Bl}{\negation{\Bl}}}}{\parconjunction{\Al}{\negation{\Al}}}$}{\Rule{$\HORSESHOE$-Intro}, 3--4}
\galineNC{6}{$\conjunction{\Bl}{\negation{\Bl}}$}{\Rule{$\NEGATION$-Elim}}
\end{gproof}
It should be clear from looking at this derivation that we can replace $\conjunction{\Bl}{\negation{\Bl}}$ with any sentence $\CAPPSI$ and the string of sentences that results from this replacement will also be a derivation. 
It should also be clear that we can replace $\Al$ and $\negation{\Al}$ with any pair $\CAPPHI$ and $\negation{\CAPPHI}$ and the resulting string of sentences is a derivation. 
That is,
\begin{gproof}[\label{anycontradictionSC}]
\galineNC{1}{$\CAPPHI$}{}
\galineNC{2}{$\negation{\CAPPHI}$}{}
\gaaproof{
\gaalineNCS{3}{$\negation{\CAPPSI}$}{\Rule{Assume}}
\gaalineNCS{4}{$\conjunction{\CAPPHI}{\negation{\CAPPHI}}$}{\Rule{$\WEDGE\!$-Intro, 1,2}}
}
\galineNC{5}{$\horseshoe{\negation{\CAPPSI}}{\parconjunction{\CAPPHI}{\negation{\CAPPHI}}}$}{\Rule{$\HORSESHOE$-Intro}, 3--4}
\galineNC{6}{$\CAPPSI$}{\Rule{$\NEGATION$-Elim}}
\end{gproof}
is a derivation of $\CAPPSI$ for any sentences $\CAPPHI$ and $\CAPPSI$. 
More importantly, any time we're doing a derivation and we have two lines, one with a sentence $\CAPPHI$ and the other with its negation $\negation{\CAPPHI}$, we could insert a derivation of just this form to get a sentence $\CAPPSI$. 
So we can introduce a new rule which says that given two sentences $\CAPPHI$ and $\negation{\CAPPHI}$, we can add any sentence $\CAPPSI$. 
We call this new rule \Rule{Any Contradiction}, or \Rule{A.C.} for short. 
The key feature of this rule is that in any derivation where we use the rule, we could have gotten the same results without it. 
All we'd have to do is insert the appropriate instance of \ref{anycontradictionSC} into the derivation. (We leave it to the reader to rewrite derivation \pncmvref{cangetlong} using \Rule{A.C.})

%\begin{table}[!ht]
%\renewcommand{\arraystretch}{1.5}
%\begin{center}
%\begin{tabular}{ p{1in} l l } %p{2.2in} p{2in}
%\toprule
%\textbf{Name} & \textbf{Given} & \textbf{May Add} \\ 
%\midrule
%\begin{table}[!ht]
\renewcommand{\arraystretch}{1.5}
\begin{longtable}[c]{ p{1in} l l } %p{2.2in} p{2in}
	\toprule
	\textbf{Name} & \textbf{Given} & \textbf{May Add} \\ 
	\midrule
	\endfirsthead
	\multicolumn{3}{c}{\emph{Continued from Previous Page}}\\
	\toprule
	\textbf{Name} & \textbf{Given} & \textbf{May Add} \\ 
	\midrule
	\endhead
	\bottomrule
	\caption{Standard Shortcut Rules for \GSD{}}\\[-.15in]
	\multicolumn{3}{c}{\emph{Continued next Page}}\\
	\endfoot
	\bottomrule
	\caption{Standard Shortcut Rules for \GSD{}}\\
	\endlastfoot
	\label{GSDplus1}\Rule{M.T.} & $\horseshoe{\CAPPHI}{\CAPTHETA}$, $\negation{\CAPTHETA}$ & $\negation{\CAPPHI}$ \\
	\Rule{D.S.} & $\disjunction{\CAPPHI_1}{\disjunction{\ldots}{\disjunction{\CAPPHI_i}{\disjunction{\ldots}{\CAPPHI_{\integer{n}}}}}}$, $\negation{\CAPPHI_i}$ & $\disjunction{\CAPPHI_1}{\disjunction{\ldots}{\disjunction{\CAPPHI_{i-1}}{\disjunction{\CAPPHI_{i+1}}{\disjunction{\ldots}{\CAPPHI_{\integer{n}}}}}}}$ \\
	\nopagebreak
	& $\disjunction{\CAPPHI_1}{\disjunction{\ldots}{\disjunction{\negation{\CAPPHI_i}}{\disjunction{\ldots}{\CAPPHI_{\integer{n}}}}}}$, ${\CAPPHI_i}$ & $\disjunction{\CAPPHI_1}{\disjunction{\ldots}{\disjunction{\CAPPHI_{i-1}}{\disjunction{\CAPPHI_{i+1}}{\disjunction{\ldots}{\CAPPHI_{\integer{n}}}}}}}$ \\
	\Rule{A.C.} & ${\CAPPHI},{\negation{\CAPPHI}}$ & $\CAPPSI$ \\
	\Rule{$\NEGATION$/$\TRIPLEBAR$-Intro} & $\triplebar{\CAPPHI}{\CAPPSI}$ & $\triplebar{\negation{\CAPPHI}}{\negation{\CAPPSI}}$ \\
	\Rule{Ext. $\WEDGE$-Elim} &{}$\conjunction{\CAPTHETA_1}{\conjunction{\CAPTHETA_2}{\conjunction{\ldots}{\CAPTHETA_{\integer{n}}}}}$&{}Conjunction of any\\[-.25cm]
	\nopagebreak
	& &{}subset of the conjuncts\\
\end{longtable}
%\bottomrule
%\end{tabular}
%\end{center}
%\caption{Standard Short-Cut Rules for \GSD{} (\GSD{})}
% \label{GSDplus1}
%\end{table}
 
The idea is the same for all the shortcut rules we introduce here. These come in two types, standard and exchange, and are listed in table \pmvref{GSDplus1} and \pmvref{GSDplus2}. The idea is, (i) for each shortcut rule \Rule{R}, for given any application of \Rule{R} we can derive, using only the basic rules, the sentence $\CAPPHI$ (which \Rule{R} allows us to write down) from the sentences $\CAPPSI_1,\ldots,\CAPPSI_{\integer{n}}$ to which we applied \Rule{R}. 
And, (ii) in general, anything we can derive using a rule, any application of which can be derived using only basic rules, can itself be derived using only basic rules.  
So, (iii) anything we can derive using the basic rules of \GSD{} and any of the shortcut rules can be derived from just the basic rules alone.
We have restated (ii) more precisely below as theorem \mvref{GSD Shortcut Theorem}, (i) as theorem \mvref{GSD Shortcut Theorem2}, and (iii) as theorem \mvref{GSD Shortcut Theorem3}. 
Exchange rules are short cut rules that work in both directions.

%\begin{table}[!ht]
%\renewcommand{\arraystretch}{1.5}
%\begin{center}
%\begin{tabular}{ p{1in} l l } %p{2.2in} p{2in}
%\toprule
%\textbf{Name} & \textbf{Given} & \textbf{May Add} \\ 
%\midrule
\renewcommand{\arraystretch}{1.5}
\begin{longtable}[c]{ p{1in} l l } %p{2.2in} p{2in}
\toprule
\textbf{Name} & \textbf{Given} & \textbf{May Add} \\ 
\midrule
\endfirsthead
\multicolumn{3}{c}{\emph{Continued from Previous Page}}\\
\toprule
\textbf{Name} & \textbf{Given} & \textbf{May Add} \\ 
\midrule
\endhead
\bottomrule
\caption{Exchange Short-Cut Rules for \GSD{}}\\[-.15in]
\multicolumn{3}{c}{\emph{Continued next Page}}\\
\endfoot
\bottomrule
\caption{Exchange Short-Cut Rules for \GSD{}}\\
\endlastfoot
\label{GSDplus2}\Rule{DeM} & $\negation{\parconjunction{\CAPPHI_1}{\conjunction{\ldots}{\CAPPHI_{\integer{n}}}}}$ & $\disjunction{\negation{\CAPPHI_1}}{\disjunction{\ldots}{\negation{\CAPPHI_{\integer{n}}}}}$\\
 & $\disjunction{\negation{\CAPPHI_1}}{\disjunction{\ldots}{\negation{\CAPPHI_{\integer{n}}}}}$ & $\negation{\parconjunction{\CAPPHI_1}{\conjunction{\ldots}{\CAPPHI_{\integer{n}}}}}$\\
 & $\negation{\pardisjunction{\CAPPHI_1}{\disjunction{\ldots}{\CAPPHI_{\integer{n}}}}}$ & $\conjunction{\negation{\CAPPHI_1}}{\conjunction{\ldots}{\negation{\CAPPHI_{\integer{n}}}}}$ \\
 & $\conjunction{\negation{\CAPPHI_1}}{\conjunction{\ldots}{\negation{\CAPPHI_{\integer{n}}}}}$ & $\negation{\pardisjunction{\CAPPHI_1}{\disjunction{\ldots}{\CAPPHI_{\integer{n}}}}}$ \\
\Rule{$\NEGATION\NEGATION$-Elim} & $\negation{\negation{\CAPPHI}}$ & $\CAPPHI$ \\
\Rule{$\NEGATION\NEGATION$-Intro} & $\CAPPHI$ & $\negation{\negation{\CAPPHI}}$ \\
\Rule{$\HORSESHOE$/$\VEE$-Exch.} & $\horseshoe{\CAPPHI}{\CAPTHETA}$ & $\disjunction{\negation{\CAPPHI}}{\CAPTHETA}$ \\
\nopagebreak
 & $\disjunction{\negation{\CAPPHI}}{\CAPTHETA}$ & $\horseshoe{\CAPPHI}{\CAPTHETA}$  \\
\Rule{Contraposition} & $\horseshoe{\CAPPHI}{\CAPTHETA}$ & $\horseshoe{\negation{\CAPTHETA}}{\negation{\CAPPHI}}$ \\
 & $\horseshoe{\negation{\CAPTHETA}}{\negation{\CAPPHI}}$ & $\horseshoe{\CAPPHI}{\CAPTHETA}$ \\
\Rule{$\NEGATION$/$\HORSESHOE$-Exch.} & $\negation{\parhorseshoe{\CAPPHI}{\CAPTHETA}}$ & $\conjunction{\CAPPHI}{\negation{\CAPTHETA}}$ \\
\nopagebreak
 & $\conjunction{\CAPPHI}{\negation{\CAPTHETA}}$ & $\negation{\parhorseshoe{\CAPPHI}{\CAPTHETA}}$ \\
\Rule{Distribution} & $\conjunction{\CAPTHETA}{\pardisjunction{\CAPPHI_1}{\disjunction{\ldots}{\CAPPHI_{\integer{n}}}}}$ & $\disjunction{\parconjunction{\CAPTHETA}{\CAPPHI_1}}{\disjunction{\ldots}{\parconjunction{\CAPTHETA}{\CAPPHI_{\integer{n}}}}}$\\
\nopagebreak
 & $\disjunction{\parconjunction{\CAPTHETA}{\CAPPHI_1}}{\disjunction{\ldots}{\parconjunction{\CAPTHETA}{\CAPPHI_{\integer{n}}}}}$ & $\conjunction{\CAPTHETA}{\pardisjunction{\CAPPHI_1}{\disjunction{\ldots}{\CAPPHI_{\integer{n}}}}}$\\
\nopagebreak 
 & $\conjunction{\pardisjunction{\CAPPHI_1}{\disjunction{\ldots}{\CAPPHI_{\integer{n}}}}}{\CAPTHETA}$ & $\disjunction{\parconjunction{\CAPPHI_1}{\CAPTHETA}}{\disjunction{\ldots}{\parconjunction{\CAPPHI_{\integer{n}}}{\CAPTHETA}}}$\\
\nopagebreak 
 & $\disjunction{\parconjunction{\CAPPHI_1}{\CAPTHETA}}{\disjunction{\ldots}{\parconjunction{\CAPPHI_{\integer{n}}}{\CAPTHETA}}}$  & $\conjunction{\pardisjunction{\CAPPHI_1}{\disjunction{\ldots}{\CAPPHI_{\integer{n}}}}}{\CAPTHETA}$\\
\nopagebreak 
 & $\disjunction{\CAPTHETA}{\parconjunction{\CAPPHI_1}{\conjunction{\ldots}{\CAPPHI_{\integer{n}}}}}$ & $\conjunction{\pardisjunction{\CAPTHETA}{\CAPPHI_1}}{\conjunction{\ldots}{\pardisjunction{\CAPTHETA}{\CAPPHI_{\integer{n}}}}}$\\
\nopagebreak 
 & $\conjunction{\pardisjunction{\CAPTHETA}{\CAPPHI_1}}{\conjunction{\ldots}{\pardisjunction{\CAPTHETA}{\CAPPHI_{\integer{n}}}}}$ & $\disjunction{\CAPTHETA}{\parconjunction{\CAPPHI_1}{\conjunction{\ldots}{\CAPPHI_{\integer{n}}}}}$\\
\nopagebreak 
 & $\disjunction{\parconjunction{\CAPPHI_1}{\conjunction{\ldots}{\CAPPHI_{\integer{n}}}}}{\CAPTHETA}$ & $\conjunction{\pardisjunction{\CAPPHI_1}{\CAPTHETA}}{\conjunction{\ldots}{\pardisjunction{\CAPPHI_{\integer{n}}}{\CAPTHETA}}}$\\
\nopagebreak
 & $\conjunction{\pardisjunction{\CAPPHI_1}{\CAPTHETA}}{\conjunction{\ldots}{\pardisjunction{\CAPPHI_{\integer{n}}}{\CAPTHETA}}}$ & $\disjunction{\parconjunction{\CAPPHI_1}{\conjunction{\ldots}{\CAPPHI_{\integer{n}}}}}{\CAPTHETA}$\\
\end{longtable}
%\bottomrule
%\end{tabular}
%\end{center}
%\caption{Exchange Short-Cut Rules for \GSD{} (\GSD{})}
%\label{GSDplus2}
%\end{table}

Theorem \mvref{GSD Shortcut Theorem} is a general claim that doesn't require a different proof for each new rule we introduce. 
The proof for it fills out the quick argument given when \Rule{A.C.} was introduced. There we argued that since any application of \Rule{A.C.} can be derived using just the basic rules, we can eliminate that application of the rule from the proof by cutting and pasting the derivation of that application into the original proof. 
\begin{majorILnc}{\LnpDC{RuleInstanceDerivability}}
Every application of a rule \Rule{R$_1$} is derivable using the rules \Rule{R$_2$}, $\ldots$, \Rule{R$_\integer{p}$} and the basic rules of \GSD{} \Iff for all \GSL{} sentences $\CAPPHI_1,\ldots,\CAPPHI_{\integer{m}}$ and $\CAPPSI$, if \Rule{R$_1$} sanctions writing down $\CAPPSI$ when applied to $\CAPPHI_1,\ldots,\CAPPHI_{\integer{m}}$ on previous unboxed lines, then $\CAPPSI$ can be derived from $\CAPPHI_1,\ldots,\CAPPHI_{\integer{m}}$ using only rules \Rule{R$_2$}$,\ldots,$\Rule{R$_\integer{p}$} and the basic rules of \GSD{}.
\end{majorILnc}
\begin{THEOREM}{\LnpTC{GSD Shortcut Theorem}}
For all \GSL{} sentences $\CAPTHETA_1,\ldots,\CAPTHETA_{\integer{n}},\DELTA$ and rules \Rule{R$_1$}$,\ldots,$\Rule{R$_\integer{p}$}, if
\begin{cenumerate}
\item $\DELTA$ can be derived from $\CAPTHETA_1,\ldots,\CAPTHETA_{\integer{n}}$ using rules \Rule{R$_1$}$,\ldots,$\Rule{R$_\integer{p}$} and the basic rules of \GSD{}, and
\item every application of a rule \Rule{R$_1$} is derivable using the rules \Rule{R$_2$}, $\ldots$, \Rule{R$_\integer{p}$} and the basic rules of \GSD{},
\end{cenumerate}
then $\DELTA$ can be derived from $\CAPTHETA_1,\ldots,\CAPTHETA_{\integer{n}}$ using only rules \Rule{R$_2$}$,\ldots,$\Rule{R$_\integer{p}$} and the basic rules of \GSD{}.
%Any sentence $\CAPPHI$ that can be derived from the sentences $\CAPPSI_1,\ldots,\CAPPSI_{\integer{n}}$ using the basic rules plus some of the shortcut rules in tables \ref{GSDplus1} and \ref{GSDplus2} can be derived from $\CAPPSI_1,\ldots,\CAPPSI_{\integer{n}}$ using the basic rules alone. 
\end{THEOREM}
\begin{PROOF}
Call the original derivation of $\DELTA$ from $\CAPTHETA_1,\ldots,\CAPTHETA_{\integer{n}}$ using rules \Rule{R$_1$}$,\ldots,$\Rule{R$_\integer{p}$} derivation $\Derivation{D}_1$. 
Say the first use of \Rule{R$_1$} happens in $\Derivation{D}_1$ on line $\integer{q}$. 
Say in that case sentence $\CAPPSI$ was written down (on line $\integer{q}$) and the application of the rule used previous lines $\integer{r}_1,\ldots,\integer{r}_{\integer{m}}$ with sentences $\CAPPHI_1,\ldots,\CAPPHI_{\integer{m}}$, respectively, on them. 
By assumption, $\CAPPSI$ can be derived from sentences $\CAPPHI_1,\ldots,\CAPPHI_{\integer{m}}$ using only \Rule{R$_2$}$,\ldots,$\Rule{R$_\integer{p}$} and the basic rules of \GSD{}. Call this derivation $\Derivation{D}^*$ and assume $\CAPPHI_1,\ldots,\CAPPHI_{\integer{m}}$ are, respectively, on lines 1 through $\integer{m}$ in $\Derivation{D}^*$. Assume there are $\integer{s}$ more lines (call these the middle lines of $\Derivation{D}^*$), and finally on line $\integer{m}+\integer{s}+1$ of $\Derivation{D}^*$ is $\CAPPSI$.

Now in between lines $\integer{q}-1$ and $\integer{q}$ of $\Derivation{D}_1$ insert $\integer{s}$ new lines. 
On these lines put the appropriate sentence from the $\integer{s}$ middle line of $\Derivation{D}^*$. 
(So, put the sentence on the first middle line of $\Derivation{D}^*$ on the first new line inserted into $\Derivation{D}_1$, the second on the second, etc.) 
Write the same rules as justifications on these new lines as were on the middle lines of $\Derivation{D}^*$ and for each justification if $\integer{t}$ was the number of a line cited by that justification in $\Derivation{D}^*$, cite line number $\integer{r}_\integer{t}$ if $\integer{t}\leq\integer{m}$ and cite line number $(\integer{t}-\integer{m})+(\integer{q}-1)$ if $\integer{t}>\integer{m}$. 
Next, on the line originally numbered $\integer{q}$ (now numbered $\integer{q}+\integer{s}$) erase rule \Rule{R$_1$} as the justification and whatever line numbers $\integer{t}$ were cited and in place of that put whatever rule was used to justify the last line of $\Derivation{D}^*$, citing instead lines $\integer{r}_\integer{t}$ if $\integer{t}\leq\integer{m}$ and lines $(\integer{t}-\integer{m})+(\integer{q}-1)$ if $\integer{t}>\integer{m}$.
Finally, on all the lines of $\Derivation{D}_1$ that were originally numbered $\integer{q}$ or higher (and now are numbered $\integer{q}+\integer{s}$ and higher), if the justification cites line $\integer{t}$ for $\integer{t}<\integer{q}$, do nothing. 
If it cites line $\integer{t}$ for $\integer{t}\geq\integer{q}$, replace $\integer{t}$ with $\integer{t}+\integer{s}$. 

Note that we've now produced a derivation $\Derivation{D}_2$ of $\DELTA$ from $\CAPTHETA_1,\ldots,\CAPTHETA_{\integer{n}}$ that has one less application of \Rule{R$_1$} than $\Derivation{D}_1$ had. 
Now if there is an application of rule \Rule{R$_1$} in $\Derivation{D}_2$, repeat exactly this procedure for $\Derivation{D}_2$. 
Repeating the procedure will lead to a derivation $\Derivation{D}_3$ with one less application of \Rule{R$_1$} than $\Derivation{D}_2$ had. 
We can continue this, producing some series $\Derivation{D}_1,\Derivation{D}_2,\ldots,\Derivation{D}_{\integer{l}}$ of derivations which will eventually end in a derivation $\Derivation{D}_{\integer{l}}$ that has no applications of \Rule{R$_1$}. 
(This procedure must end, since there could only have been a finite number of applications of \Rule{R$_1$} in $\Derivation{D}_1$.) 
Thus, $\Derivation{D}_{\integer{l}}$ is a derivation of $\DELTA$ from $\CAPTHETA_1,\ldots,\CAPTHETA_{\integer{n}}$ that uses only rules \Rule{R$_2$}$,\ldots,$\Rule{R$_\integer{p}$} and the basic rules of \GSD{}.
\end{PROOF}
\begin{THEOREM}{\LnpTC{GSD Shortcut Theorem2}}
For all standard and exchange shortcut rules \Rule{R} (see tables \ref{GSDplus1} and \ref{GSDplus2}), every application of \Rule{R} is derivable using the basic rules of \GSD{}.
\end{THEOREM}
\begin{PROOF}
See the discussion immediately following the proof of theorem \ref{GSD Shortcut Theorem3}.
\end{PROOF}
\begin{THEOREM}{\LnpTC{GSD Shortcut Theorem3} Shortcut Rule Elimination Theorem:}
For all \GSL{} sentences $\CAPPHI_1,\ldots,\CAPPHI_{\integer{m}}$ and $\CAPPSI$, if $\CAPPSI$ can be derived from $\CAPPHI_1,\ldots,\CAPPHI_{\integer{m}}$ in \GSDP{} (that is, using the basic rules of \GSD{} and any of the standard and exchange shortcut rules), then $\CAPPSI$ can be derived from $\CAPPHI_1,\ldots,\CAPPHI_{\integer{m}}$ in \GSD{} (that is, using only the basic rules).
\end{THEOREM}
\begin{PROOF}
Assume that $\CAPPSI$ can be derived from $\CAPPHI_1,\ldots,\CAPPHI_{\integer{m}}$ using the basic rules of \GSD{} and the standard and exchange shortcut rules. 
Consider any of the shortcut rules, say \Rule{M.T.} 
Let \Rule{R$_1$} be \Rule{M.T.} and rules \Rule{R$_2$} through \Rule{R$_{25}$} be the other standard and exchange rules.
By assumption, condition (1) in theorem \mvref{GSD Shortcut Theorem} holds for these sentences, while by theorem \mvref{GSD Shortcut Theorem2} condition (2) in theorem \ref{GSD Shortcut Theorem} holds for \Rule{M.T.} 
So, it follows from theorem \ref{GSD Shortcut Theorem} that $\CAPPSI$ can be derived from $\CAPPHI_1,\ldots,\CAPPHI_{\integer{m}}$ using the basic rules of \GSD{} and all the standard and exchange shortcut rules besides \Rule{M.T.} By reapplying theorem \ref{GSD Shortcut Theorem} in just the same way to all the standard and exchange shortcut rules, we get that $\CAPPSI$ can be derived from $\CAPPHI_1,\ldots,\CAPPHI_{\integer{m}}$ using only the basic rules of \GSD{}. 
(That is, we reapply theorem \ref{GSD Shortcut Theorem} twenty four more times, each time showing that another shortcut rule wasn't needed.)
\end{PROOF}

\bigskip
\noindent{}Unlike theorem \ref{GSD Shortcut Theorem}, for theorem \ref{GSD Shortcut Theorem2} we need a separate argument for each shortcut rule (both standard and exchange). 
We've already done one rule, \Rule{Any Contradiction}. 
Our argument in this case was that any application of the rule will involve writing some sentence $\CAPPSI$ on a new line from sentences $\CAPPHI$ and $\negation{\CAPPHI}$. 
But whatever sentences $\CAPPSI$ and $\CAPPHI$ we pick, if we substitute them into \ref{anycontradictionSC}, the result is a derivation of $\CAPPSI$ from $\CAPPHI$ and $\negation{\CAPPHI}$. 

Before continuing, it's important to note that, strictly speaking, \ref{anycontradictionSC} is \emph{not} a derivation. 
This is because a derivation, as we've defined it, is a series of \GSL{} sentences. 
The strings of symbols on each line of \ref{anycontradictionSC} are not \GSL{} sentences because they contain MathEnglish variables for \GSL{} sentences. 
Instead, they are sentence schemas.\index{sentence schema}
But, as we've done in \ref{anycontradictionSC}, nothing stops us from treating sentence schemas like the ones in \ref{anycontradictionSC} as \GSL{} sentences and applying rules to them. 
The key fact---why this is useful---is that what we get when we do this becomes a derivation whenever we substitute \GSL{} sentences in for the MathEnglish variables. 
For this reason we call these \niidf{derivation} \underidf{schemas}{derivation} instead of derivations. 

Returning to theorem \mvref{GSD Shortcut Theorem2}, we can handle the other rules in just the same way we handled \Rule{Any Contradiction}. 
For example, if we write a derivation schema of $\negation{\CAPPHI}$ from $\horseshoe{\CAPPHI}{\CAPTHETA}$ and $\negation{\CAPTHETA}$ using only basic rules of \GSD{}, then this is sufficient to show that any application of the rule \Rule{M.T.} (\Rule{Modus Tollens}) can be derived using only basic rules of \GSD{}.\footnote{For
those keeping track of the \distinction{use}{mention} distinction, here we have mentioned the MathEnglish variables \mention{$\CAPPHI$} and \mention{$\CAPTHETA$} as well as the strings of symbols \mention{$\negation{\CAPPHI}$}, \mention{$\horseshoe{\CAPPHI}{\CAPTHETA}$}, and \mention{$\negation{\CAPTHETA}$}. 
So, strictly speaking, we should have put them all in quotes.
This is different how we normally use these symbols, since we're normally actually using them (as variables) instead of mentioning them (as the objects of derivation schemas).}
(Recall def. \pmvref{RuleInstanceDerivability}: Saying that an application of a rule can be derived using only basic rules is a shorthand way of saying that the sentence written down on a new line, in some application of the rule, can be derived using only basic rules from the sentences to which the rule was applied.)
So, in order to complete the proof for theorem \ref{GSD Shortcut Theorem2}, we need to write derivation schemas for all the standard and exchange rules (for all the rules in tables \ref{GSDplus1} and \ref{GSDplus2}).
That is, for each rule we need to write a derivation schema that has the given schemas of the rule as premises and the may-add schema of the rule as conclusion. 

Note that once we have shown that a shortcut rule can be eliminated from a proof (i.e., once we've shown that theorem \ref{GSD Shortcut Theorem2}, holds at least for that rule), then we can use that shortcut rule in derivation schemas for new shortcut rules we haven't yet shown can be eliminated. 
If we write a derivation schema for some new shortcut rule we're trying to show can be eliminated and that schema uses a previous shortcut rule, then any derivation got from the schema will contain an application of the previous rule.
But we already know that these applications of the previous rule can be eliminated, so that's not a problem.  
%But, we can use rules, for which we've already completed a derivation schema, in our derivation schemas for rules we haven't yet completed. 

Most of the derivation schemas for the shortcut rules (both standard and exchange) are left to the reader as exercises. 
(See section \ref{exercisesGSDshortcutrules}.) 
There is one complication though. 
We cannot actually write a single derivation schema for \Rule{D.S.} (\Rule{Disjunctive Syllogism}), \Rule{DeM} (\Rule{DeMorgans}), or \Rule{Distribution}. 
This is because these rules mention arbitrarily long conjunctions and disjunctions and we can only write a derivation schema involving conjunctions and disjunctions of definite, fixed (and finite) length. 
A less rigorous option is to write a few derivation schemas for \Rule{D.S.}, \Rule{DeM}, and \Rule{Distribution} for the cases when the conjunctions and disjunctions are small (say 2- or 3-place) and convince ourselves that we can keep writing similar schemas no matter how large the conjunctions and disjunctions get.
A more rigorous option is to write the derivation schema for the 2-place case and then use mathematical induction on the length of the conjunctions and disjunctions to show derivation schemas can be written for all lengths. 
In the exercises, in section \ref{exercisesGSDshortcutrules}, we only ask the reader to write the derivation schemas for these three rules for the cases where the conjunctions and disjunctions are 2-place.

Here we write the derivation schema needed for the last of the four \Rule{DeMorgans} rules in table \mvref{GSDplus2}, assuming that the conjunction and disjunction are only 2-place.
So we need to derive $\negation{\pardisjunction{\CAPPHI}{\CAPTHETA}}$ from $\conjunction{\negation{\CAPPHI}}{\negation{\CAPTHETA}}$. 
The sentence we want is a negation, so we use our basic bottom-up strategy for negation:
\begin{gproof}
\galineNC{1}{$\conjunction{\negation{\CAPPHI}}{\negation{\CAPTHETA}}$}{\Rule{Assume}}
\gaaproof{
\gaalineNCS{2}{$\pardisjunction{\CAPPHI}{\CAPTHETA}$}{\Rule{Assume}}
\gaalineNCndS{}{}{}
\gaalineNCndS{}{$\qquad\vdots$}{}
\gaalineNCndS{}{}{}
\gaalineNCS{$\integer{n}-2$}{$\conjunction{\CAPPSI}{\negation{\CAPPSI}}$}{ }
}
\galineNC{$\integer{n}-1$}{$\horseshoe{\pardisjunction{\CAPPHI}{\CAPTHETA}}{\parconjunction{\CAPPSI}{\negation{\CAPPSI}}}$}{\Rule{$\HORSESHOE$-Intro}, 2--$\integer{n}-2$}
\galineNC{$\integer{n}$}{$\negation{\pardisjunction{\CAPPHI}{\CAPTHETA}}$}{\Rule{$\NEGATION$-Intro, $\integer{n}-1$}}
\end{gproof}
\noindent{}As always with \Rule{$\NEGATION$-Intro}, we need some contradiction $\conjunction{\CAPPSI}{\negation{\CAPPSI}}$. 
Before we had to be careful about setting up the right contradiction (recall derivation \ref{cangetlong}). %\pncmvref{cangetlong}). 
But now that we have rule \Rule{A.C.}, we don't need to be so careful. 
So long as we can get a contradiction, we can always use \Rule{A.C.} to get whatever other contradiction we need to make the derivation work.
 
Returning to the proof, we need to work top-down from the disjunction on line 2 to get to a contradiction. So we use the basic strategy:
\begin{gproof}
\galineNC{1}{$\conjunction{\negation{\CAPPHI}}{\negation{\CAPTHETA}}$}{\Rule{Assume}}
\gaaproof{
\gaalineNCS{2}{$\pardisjunction{\CAPPHI}{\CAPTHETA}$}{\Rule{Assume}}
\gaalineNCndS{}{}{}
\gaalineNCndS{}{$\qquad\vdots$}{}
\gaalineNCndS{}{}{}
\gaalineNCS{$\integer{n}-4$}{$\horseshoe{\CAPPHI}{\parconjunction{\CAPPSI}{\negation{\CAPPSI}}}$}{ }
\gaalineNCS{$\integer{n}-3$}{$\horseshoe{\CAPTHETA}{\parconjunction{\CAPPSI}{\negation{\CAPPSI}}}$}{ }
\gaalineNCS{$\integer{n}-2$}{$\conjunction{\CAPPSI}{\negation{\CAPPSI}}$}{\Rule{$\VEE$-Elim}, 2,$\integer{n}-4$,$\integer{n}-3$}
}
\galineNC{$\integer{n}-1$}{$\horseshoe{\pardisjunction{\CAPPHI}{\CAPTHETA}}{\parconjunction{\CAPPSI}{\negation{\CAPPSI}}}$}{\Rule{$\HORSESHOE$-Intro}, 2--$\integer{n}-2$}
\galineNC{$\integer{n}$}{$\negation{\pardisjunction{\CAPPHI}{\CAPTHETA}}$}{\Rule{$\NEGATION$-Intro, $\integer{n}-1$}}
\end{gproof}
\noindent{}And from here we need to work bottom-up from the conditionals on lines $\integer{n}-4$ and $\integer{n}-3$. 
\begin{gproof}
\galineNC{1}{$\conjunction{\negation{\CAPPHI}}{\negation{\CAPTHETA}}$}{\Rule{Assume}}
\gaaproof{
\gaalineNCS{2}{$\pardisjunction{\CAPPHI}{\CAPTHETA}$}{\Rule{Assume}}

\gaaaproof{
\gaaalineS{3}{$\CAPPHI$}{\Rule{Assume}}
\gaaalinendS{}{}{}
\gaaalinendS{}{$\qquad\vdots$}{}
\gaaalinendS{}{}{}
\gaaalineS{$\integer{m}$}{$\parconjunction{\CAPPSI}{\negation{\CAPPSI}}$}{}
}
\gaalineNCS{$\integer{m}+1$}{$\horseshoe{\CAPPHI}{\parconjunction{\CAPPSI}{\negation{\CAPPSI}}}$}{\Rule{$\HORSESHOE$-Intro}, 3-$\integer{m}$}

\gaaaproof{
\gaaalineS{$\integer{m}+2$}{$\CAPTHETA$}{\Rule{Assume}}
\gaaalinendS{}{}{}
\gaaalinendS{}{$\qquad\vdots$}{}
\gaaalinendS{}{}{}
\gaaalineS{$\integer{n}-4$}{$\parconjunction{\CAPPSI}{\negation{\CAPPSI}}$}{}
}
\gaalineNCS{$\integer{n}-3$}{$\horseshoe{\CAPTHETA}{\parconjunction{\CAPPSI}{\negation{\CAPPSI}}}$}{\Rule{$\HORSESHOE$-Intro}, $\integer{m}+2$--$\integer{n}-4$}

\gaalineNCS{$\integer{n}-2$}{$\conjunction{\CAPPSI}{\negation{\CAPPSI}}$}{\Rule{$\VEE$-Elim}, 2,$\integer{m}+1$,$\integer{n}-3$}
}
\galineNC{$\integer{n}-1$}{$\horseshoe{\pardisjunction{\CAPPHI}{\CAPTHETA}}{\parconjunction{\CAPPSI}{\negation{\CAPPSI}}}$}{\Rule{$\HORSESHOE$-Intro}, 2--$\integer{n}-2$}
\galineNC{$\integer{n}$}{$\negation{\pardisjunction{\CAPPHI}{\CAPTHETA}}$}{\Rule{$\NEGATION$-Intro, $\integer{n}-1$}}
\end{gproof}
And now we can finish the proof by working top-down, breaking apart the conjunction on line 1 and using \Rule{A.C.}
\begin{gproof}[\label{DeMDerivationSchema}]
\galineNC{1}{$\conjunction{\negation{\CAPPHI}}{\negation{\CAPTHETA}}$}{\Rule{Assume}}
\gaaproof{
\gaalineNCS{2}{$\pardisjunction{\CAPPHI}{\CAPTHETA}$}{\Rule{Assume}}

\gaaaproof{
\gaaalineS{3}{$\CAPPHI$}{\Rule{Assume}}
\gaaalinendS{4}{$\negation{\CAPPHI}$}{\Rule{$\WEDGE$-Elim}, 1}
\gaaalinendS{5}{$\conjunction{\negation{\CAPPHI}}{\CAPPHI}$}{\Rule{$\WEDGE$-Intro}, 3,4}
\gaaalineS{6}{$\parconjunction{\CAPPSI}{\negation{\CAPPSI}}$}{\Rule{A.C.}, 5}
}
\gaalineNCS{7}{$\horseshoe{\CAPPHI}{\parconjunction{\CAPPSI}{\negation{\CAPPSI}}}$}{\Rule{$\HORSESHOE$-Intro}, 3--6}

\gaaaproof{
\gaaalineS{8}{$\CAPTHETA$}{\Rule{Assume}}
\gaaalinendS{9}{$\negation{\CAPTHETA}$}{\Rule{$\WEDGE$-Elim}, 1}
\gaaalinendS{10}{$\conjunction{\negation{\CAPTHETA}}{\CAPTHETA}$}{\Rule{$\WEDGE$-Intro}, 8,9}
\gaaalineS{11}{$\parconjunction{\CAPPSI}{\negation{\CAPPSI}}$}{\Rule{A.C.}, 10}
}
\gaalineNCS{12}{$\horseshoe{\CAPTHETA}{\parconjunction{\CAPPSI}{\negation{\CAPPSI}}}$}{\Rule{$\HORSESHOE$-Intro}, 8--11}

\gaalineNCS{13}{$\conjunction{\CAPPSI}{\negation{\CAPPSI}}$}{\Rule{$\VEE$-Elim}, 2,7,12}
}
\galineNC{14}{$\horseshoe{\pardisjunction{\CAPPHI}{\CAPTHETA}}{\parconjunction{\CAPPSI}{\negation{\CAPPSI}}}$}{\Rule{$\HORSESHOE$-Intro}, 2--13}
\galineNC{15}{$\negation{\pardisjunction{\CAPPHI}{\CAPTHETA}}$}{\Rule{$\NEGATION$-Intro}, 14}
\end{gproof}
And so we now have a derivation schema showing how to derive a sentence of the form $\negation{\pardisjunction{\CAPPHI}{\CAPTHETA}}$ from one of the form $\conjunction{\negation{\CAPPHI}}{\negation{\CAPTHETA}}$. Note that we could have slightly shortened the derivation schema by doing \Rule{$\HORSESHOE$-Intro} on lines 3--5 instead of 3--6, using $\conjunction{\negation{\CAPPHI}}{\CAPPHI}$ as our contradiction instead of $\horseshoe{\CAPPHI}{\parconjunction{\CAPPSI}{\negation{\CAPPSI}}}$. In this case would have then used \Rule{A.C.} on line 11 to get $\conjunction{\negation{\CAPPHI}}{\CAPPHI}$. Similarly, we could have also used $\conjunction{\negation{\CAPTHETA}}{\CAPTHETA}$ as our contradiction. We leave it to the reader to show what slight modifications would be needed to the derivation schema in this case. 
%$\sststile{}{}\horseshoe{\negation{\pardisjunction{\CAPPHI}{\CAPTHETA}}}{\parconjunction{\negation{\CAPPHI}}{\negation{\CAPTHETA}}}$

\bigskip
\noindent{}Finally, we should note that every application of every shortcut rule (whether a standard or exchange shortcut rule) is truth-preserving. 
(Recall def. \pmvref{Derivation Rule Soundness}.) 
Although you might be tempted to try to prove this as a corollary to theorem \pmvref{Soundess of Basic GSD Rules}, and theorem \pmvref{GSD Shortcut Theorem2}, there's no easy way to get from (i) the claim that every application of every basic rule of \GSD{} is truth-preserving and (ii) the claim that every application of every rule of \GSDP{} is derivable in \GSD{}, to the further claim that every application of every rule of \GSDP{} is truth-preserving. 
Instead, the easiest way to prove this result more or less follows the lines of the proof for theorem \mvref{Soundess of Basic GSD Rules}.
\begin{THEOREM}{\LnpTC{Soundness of Std Shortcut Applications}}
Every application of every rule of \GSDP{}, including both standard and exchange shortcut rules, is truth-preserving.
\end{THEOREM}
\begin{PROOF}
Adjusting the proof of theorem \ref{Soundess of Basic GSD Rules} to work for the rules of \GSDP{} is left for the reader. 
The key is extending the truth-preservation lemma mentioned there to the rules of \GSDP{}.
We asked the reader to show this in exercises \mvref{exercises:truth-preservation lemma} and \mvref{exercises:GSDTFETheorem}.
\end{PROOF}

\subsection{Exchange Shortcut Rules}\label{Exchange Shortcut Rules GSD}
Recall our restriction in section \ref{Restrictions on Applying Rules} on the basic rules of \GSD{}, which said that a rule only sanctions writing down a sentence if the connectives it mentions are the main connectives of sentences on the lines to which it's applied. 
We put this restriction in place because not every application of the basic rules is truth-preserving without it. 
But with the restriction in place, every application of the basic rules becomes truth-preserving (theorem \pmvref{Soundess of Basic GSD Rules}). 
When we introduced the shortcut rules we carried over the restriction because, again, without it there are some shortcut rules with nontruth-preserving applications. 
But for some of our shortcut rules, the exchange shortcut rules (table \pmvref{GSDplus2}), every application is truth-preserving even without the restriction. 
%(The shortcut rules we've called standard, in table \pmvref{GSDplus1}, still need the restriction.)
To be explicit, for the exchange shortcut rules we can use the following definition of sanctioning (while still using def. \pncmvref{RuleSanctioning} for the basic rules and the standard exchange rules):
\begin{majorILnc}{\LnpDC{ExchangeRuleSanctioning}}
An exchange shortcut rule \Rule{R} from \GSDP{}, applied to a line with sentence $\CAPPSI$, \emph{sanctions} writing down sentence $\CAPPSI^*$ \Iff
\begin{cenumerate}
\item there is some substitution of \GSL{} sentences that, for the given schema of \Rule{R}, results in a sentence $\CAPPHI$ and, for the may-add schema, results in a sentence $\CAPPHI^*$,
\item $\CAPPHI$ is a subsentence of $\CAPPSI$, and
\item $\CAPPSI^*$ is the \GSL{} sentence you get when you replace one instance (token) of $\CAPPHI$ with an instance (token) of $\CAPPHI^*$ in $\CAPPSI$. 
\end{cenumerate}
\end{majorILnc}
\noindent{}Note that right below derivation \mvref{cangetlong3} is a detailed example of how the definition works in practice. 
With a moments thought it should be clear that if $\CAPPHI$ actually is $\CAPPSI$, then (recalling that all the exchange rules only have one given schema) this definition lines up with the original definition \mvref{RuleSanctioning}. This is just what we would expect.
\begin{THEOREM}{\LnpTC{ExchangeRuleGSDSoundness}}
Every application of every exchange shortcut rule from \GSDP{} is truth-preserving, even if we extend the notion of sanctioning for them with definition \ref{ExchangeRuleSanctioning}. 
\end{THEOREM} 
\noindent{}Note that this result was not already stated in theorem \mvref{Soundness of Std Shortcut Applications}. 
Although theorem \ref{Soundness of Std Shortcut Applications} says that every application of every rule of \GSDP{} is truth-preserving, truth-preservation is defined in terms of sanctioning (def. \ref{Derivation Rule Soundness}).
And, in theorem \ref{Soundness of Std Shortcut Applications} it was assumed that the exchange shortcut rules only sanctioned writing down a sentence if they were applied to whole sentences on lines. 
But now we're extending the notion of sanctioning for the exchange shortcut rules with definition \ref{ExchangeRuleSanctioning}. 
Theorem \ref{ExchangeRuleGSDSoundness} notes that even if we do this, the exchange shortcut rules are still truth-preserving.

We use theorem \mvref{TFE Replacement} and the following theorem to prove theorem \ref{ExchangeRuleGSDSoundness}. This theorem will play the same rule as the truth-preservation lemma from the proof of theorem \mvref{Soundess of Basic GSD Rules}.
\begin{THEOREM}{\LnpTC{ExchangeRuleGSDSoundnessLemma}}
For all exchange shortcut rules \Rule{R} from \GSDP{}, if $\CAPPHI$ and $\CAPPHI^*$ are the sentences you get after substituting \GSL{} sentences into the given and may-add schemas of \Rule{R}, respectively, then $\CAPPHI$ and $\CAPPHI^*$ are truth functionally equivalent. 
\end{THEOREM}
\begin{PROOF}
This theorem follows immediately from what the reader showed in exercises \mvref{exercises:GSDTFETheorem}. 
\end{PROOF}
\begin{PROOFOF}{Thm. \ref{ExchangeRuleGSDSoundness}}
Consider some arbitrary application of some exchange shortcut rule \Rule{R} from \GSDP{}. 
Say that in this application \Rule{R} is applied to sentence $\CAPPSI$ and permits, or sanctions, you to write down $\CAPPSI^*$. 
By definition \mvref{ExchangeRuleSanctioning}, (i) there is some substitution of \GSL{} sentences that, for the given schema of \Rule{R}, results in a sentence $\CAPPHI$ and, for the may-add schema, results in a sentence $\CAPPHI^*$, (ii) $\CAPPHI$ is a subsentence of $\CAPPSI$, and (iii) $\CAPPSI^*$ is the \GSL{} sentence you get when you replace one instance (token) of $\CAPPHI$ with an instance (token) of $\CAPPHI^*$ in $\CAPPSI$. 
From (i) and theorem \ref{ExchangeRuleGSDSoundnessLemma}, $\CAPPHI$ and $\CAPPHI^*$ are truth functionally equivalent.
So, from theorem \mvref{TFE Replacement} it follows that $\CAPPSI$ and $\CAPPSI^*$ are also truth functionally equivalent.
From the definition \mvref{GSL TFE} of truth functional equivalence it follows that $\CAPPSI\sdtstile{}{}\CAPPSI^*$.
So, by definition \mvref{Derivation Rule Soundness}, this application is truth-preserving. 
\end{PROOFOF}

\bigskip
\noindent{}The intuitive idea behind definition \mvref{ExchangeRuleSanctioning} is that the exchange shortcut rules can be applied to subsentences on lines (while the basic rules and standard shortcut rules can only be applied to whole sentences on lines). 
Some examples will hopefully make things more clear. 
Recall derivation \pmvref{cangetlong}, which showed that \mbox{$\sststile{}{}\;\negation{\parconjunction{\bpardisjunction{\negation{\Al}}{\negation{\Bl}}}{\bparconjunction{\Al}{\Bl}}}$}. 
First, consider how we might rewrite this proof using \Rule{DeMorgans}, in addition to the basic rules of \GSD{}, but only applying it to the whole sentence on a line. 
\begin{gproof}[\label{cangetlong2}]
\gaproof{
\galine{1}{$\conjunction{\bpardisjunction{\negation{\Al}}{\negation{\Bl}}}{\bparconjunction{\Al}{\Bl}}$}{\Rule{Assume}}
\galine{2}{$\bpardisjunction{\negation{\Al}}{\negation{\Bl}}$}{\Rule{$\WEDGE$-Elim}, 1}
\galine{3}{$\bparconjunction{\Al}{\Bl}$}{\Rule{$\WEDGE$-Elim}, 1}
\galine{4}{$\negation{\bparconjunction{\Al}{\Bl}}$}{\Rule{DeM}, 2}
\galine{5}{$\conjunction{\bparconjunction{\Al}{\Bl}}{\negation{\bparconjunction{\Al}{\Bl}}}$}{\Rule{$\WEDGE$-Intro}, 3,4}
}
\gline{6}{$\horseshoe{\parconjunction{\bpardisjunction{\negation{\Al}}{\negation{\Bl}}}{\bparconjunction{\Al}{\Bl}}}{\cparconjunction{\bparconjunction{\Al}{\Bl}}{\negation{\bparconjunction{\Al}{\Bl}}}}$}{\Rule{$\HORSESHOE$-Intro}, 1--5}
\gline{7}{$\negation{\parconjunction{\bpardisjunction{\negation{\Al}}{\negation{\Bl}}}{\bparconjunction{\Al}{\Bl}}}$}{\Rule{$\NEGATION$-Intro}, 6}
\end{gproof}
Obviously this is much shorter than \ref{cangetlong}, and much shorter than \ref{cangetlong} would be even if we rewrote it using \Rule{A.C.} (as we suggested the reader do above).
Here we have applied \Rule{DeMorgans} to line 2. 
But if we allow ourselves to apply \Rule{DeMorgans} to subsentences on lines in addition to whole sentences, then the proof can be made even shorter. 
\begin{gproof}[\label{cangetlong3}]
\gaproof{
\galine{1}{$\conjunction{\bpardisjunction{\negation{\Al}}{\negation{\Bl}}}{\bparconjunction{\Al}{\Bl}}$}{\Rule{Assume}}
\galine{2}{$\conjunction{\negation{\bparconjunction{\Al}{\Bl}}}{\bparconjunction{\Al}{\Bl}}$}{\Rule{DeM}, 1}
}
\gline{3}{$\horseshoe{\parconjunction{\bpardisjunction{\negation{\Al}}{\negation{\Bl}}}{\bparconjunction{\Al}{\Bl}}}{\parconjunction{\negation{\bparconjunction{\Al}{\Bl}}}{\bparconjunction{\Al}{\Bl}}}$}{\Rule{$\HORSESHOE$-Intro}, 1--2}
\gline{4}{$\negation{\parconjunction{\bpardisjunction{\negation{\Al}}{\negation{\Bl}}}{\bparconjunction{\Al}{\Bl}}}$}{\Rule{$\NEGATION$-Intro}, 3}
\end{gproof}
Unlike in \ref{cangetlong2}, here we did not need to break line 1 apart into its conjuncts. We simply applied \Rule{DeMorgans} to the first conjunct of line 1 and wrote the result down as line 2. To see how definition \mvref{ExchangeRuleSanctioning} captures this intuitive idea, consider how we would show, directly from the definition, that \Rule{DeMorgans} sanctions writing $\CAPPSI^*=\conjunction{\negation{\bparconjunction{\Al}{\Bl}}}{\bparconjunction{\Al}{\Bl}}$ on line 2 when applied to $\CAPPSI=\conjunction{\bpardisjunction{\negation{\Al}}{\negation{\Bl}}}{\bparconjunction{\Al}{\Bl}}$ on line 1. The given schema for \Rule{DeMorgans} relevant to line 1 is $\disjunction{\negation{\CAPPHI_1}}{\negation{\CAPPHI_2}}$, while the relevant may-add schema is $\negation{\parconjunction{\CAPPHI_1}{\CAPPHI_2}}$. The substitution we want for clause (1) of definition \ref{ExchangeRuleSanctioning} is $\CAPPHI_1=\Al$ and $\CAPPHI_2=\Bl$. This substitution results in $\CAPPHI=\bpardisjunction{\negation{\Al}}{\negation{\Bl}}$ and $\CAPPHI^*=\;\negation{\bparconjunction{\Al}{\Bl}}$. In line with clause (2) of the definition, $\CAPPHI=\bpardisjunction{\negation{\Al}}{\negation{\Bl}}$ is a subsentence of $\CAPPSI=\conjunction{\bpardisjunction{\negation{\Al}}{\negation{\Bl}}}{\bparconjunction{\Al}{\Bl}}$. And, in line with clause (3), $\CAPPSI^*=\conjunction{\negation{\bparconjunction{\Al}{\Bl}}}{\bparconjunction{\Al}{\Bl}}$ is the \GSL{} sentence you get when you replace one instance (token) of $\CAPPHI=\bpardisjunction{\negation{\Al}}{\negation{\Bl}}$ with an instance (token) of $\CAPPHI^*=\;\negation{\bparconjunction{\Al}{\Bl}}$ in $\CAPPSI=\conjunction{\bpardisjunction{\negation{\Al}}{\negation{\Bl}}}{\bparconjunction{\Al}{\Bl}}$.

\bigskip
\noindent{}We have shown that the exchange shortcut rules, more liberally allowed to be applied to subsentences on a line, still have truth-preserving applications (Thm. \ref{ExchangeRuleGSDSoundness}).
In the last section, \mvref{Standard Shortcut Rules GSD}, we showed that anything we can derive using the standard and exchange shortcut rules from \GSDP{} can be derived using only the basic rules of \GSD{} (Thm. \pmvref{GSD Shortcut Theorem3}). 
But to prove this we used theorem \mvref{GSD Shortcut Theorem} and theorem \ref{GSD Shortcut Theorem2} and we need to think about how more liberally allowing the exchange shortcut rules to be applied to subsentences on a line affects the proofs of these two theorems. 
That is, if we want to show that theorem \ref{GSD Shortcut Theorem3} still holds, we need to show that the theorems we used to prove it still hold.

It should not be hard to see that the change from definition \mvref{RuleSanctioning} to \mvref{ExchangeRuleSanctioning}, i.e. the move to more liberal applications of exchange shortcut rules, does not affect the proof of \mvref{GSD Shortcut Theorem}. 
But it does affect the proof of \ref{GSD Shortcut Theorem2}. 
We proved this theorem (or, rather, asked the reader to prove most of the cases) by giving, for each rule \Rule{R}, a derivation schema using only the basic rules of \GSD{} that has the given schemas of \Rule{R} as the premises and the may-add schema of \Rule{R} as the conclusion (as the last line). 
This was sufficient to show that every application of the rules of \GSDP{}, if restricted to whole sentences on lines, is derivable from the basic rules of \GSD{} because if the rules are only being applied to whole sentences, then these derivation schema will yield derivations that use only basic rules for every application. 
But this will not work if we're more liberally allowing the exchange rules to be applied to subsentences on a line.

For example, consider derivation schema \mvref{DeMDerivationSchema} for \Rule{DeMorgans}. 
Say that we have the sentence $\disjunction{\Al}{\parconjunction{\negation{\Bl}}{\negation{\Cl}}}$ on some line of a derivation on which we're working and we want to apply \Rule{DeMorgans} to the right disjunct. 
Under the more liberal definition \mvref{ExchangeRuleSanctioning} we can do this, and \Rule{DeMorgans} will permit, or sanction, us to write down $\disjunction{\Al}{\negation{\pardisjunction{\Bl}{\Cl}}}$. 
But it should be clear that substituting $\CAPPHI=\Bl$ and $\CAPTHETA=\Cl$ into derivation schema \ref{DeMDerivationSchema} will not result in a derivation of $\disjunction{\Al}{\negation{\pardisjunction{\Bl}{\Cl}}}$ from $\disjunction{\Al}{\parconjunction{\negation{\Bl}}{\negation{\Cl}}}$. 

To show that theorem \ref{GSD Shortcut Theorem2} still holds for our more liberal use of exchange shortcut rules we rely on two facts. 
First, definition \ref{ExchangeRuleSanctioning} ensures that if an exchange shortcut rule, applied to a sentence $\CAPPSI$, sanctions writing down $\CAPPSI^*$, then there's a specific relationship between $\CAPPSI$ and $\CAPPSI^*$. 
Specifically, there are two sentences $\CAPPHI$ and $\CAPPHI^*$, got by substituting sentences into the given and may-add schemas of the rules, and $\CAPPSI^*$ is $\CAPPSI$ with $\CAPPHI$ replaced with $\CAPPHI^*$. 
The second fact is that these sentences $\CAPPHI$ and $\CAPPHI^*$ are always \niidf{provably equivalent}, or as we might say \niidf{derivationally equivalent}\index{derivationally equivalent|see{provabbly equivalent}}.\footnote{Compare 
this definition with definition \mvref{GQL Provably Equivalent}, which generalizes it for formulas of \GQL{}.} 
(This is proved in exercise \pmvref{exercisesGSDshortcutrules}.)
\begin{majorILnc}{\LnpDC{GSDprovablyequivalent}}
Two sentences of \GSL{} are \nidf{provably equivalent}\index{provably equivalent!sentences of \GSL{}|textbf} \Iff one of the following two equivalent conditions holds:
\begin{cenumerate}
\item both $\CAPPHI\sststile{}{}\CAPPSI$ and $\CAPPSI\sststile{}{}\CAPPHI$, or
\item $\sststile{}{}\triplebar{\CAPPHI}{\CAPPSI}$.
\end{cenumerate}
\end{majorILnc}

These two facts, along with the following theorem,\footnote{Compare 
this theorem to theorem \mvref{GQD Replacement Theorem}, which is a stronger version of it generalized to \GQD{}. 
Note that the Restricted Replacement Theorem for \GSD{} follows immediately from the generalized \GQD{} version (but we provide a separate proof here). 
Also note that the proof of theorem \ref{GQD Replacement Theorem} uses the One-step Replacement Lemmas (Thm. \pmvref{OneStepReplacementLemmas}), and the proofs for these involve constructing derivation schemas. The proof given here for the Restricted Replacement Theorem for \GSD{} also involves the construction of derivation schemas, but does so by giving instructions in the inheritance step for how to write the relevant derivations instead of explicitly writing them out in a separate lemma.} 
are enough to show that theorem \ref{GSD Shortcut Theorem2} still holds for our more liberal use of exchange shortcut rules. (The reader should make sure they are convinced of that.)
\begin{THEOREM}{\LnpTC{ExchangeRuleTheorem} Restricted Replacement Theorem for \GSD{}:}
For all sentences $\CAPPSI$ of \GSL{}: if
\begin{cenumerate}
\item $\CAPPHI$ and $\CAPPHI^*$ are \GSL{} sentences such that $\CAPPHI\sststile{}{}\CAPPHI^*$ and $\CAPPHI^*\sststile{}{}\CAPPHI$, and
\item if $\CAPPHI$ is a subsentence of $\CAPPSI$, then $\CAPPSI^*$ is the \GSL{} sentence you get when you replace one instance (token) of $\CAPPHI$ with an instance (token) of $\CAPPHI^*$, and $\CAPPSI^*$ is $\CAPPSI$ if not, 
\end{cenumerate}
then $\CAPPSI^*$ can be derived from $\CAPPSI$ using only the basic rules of \GSD{}, i.e. $\CAPPSI\sststile{}{}\CAPPSI^*$.
\end{THEOREM}
\begin{PROOF}
Assume that $\CAPPHI$ and $\CAPPHI^*$ are two \GSL{} sentences such that $\CAPPHI\sststile{}{}\CAPPHI^*$ and $\CAPPHI^*\sststile{}{}\CAPPHI$.
\begin{description}
\item[Base Step:] If $\CAPPSI$ is atomic, then it's just a sentence letter. 
So, if $\CAPPHI$ is a subsentence of $\CAPPSI$, it itself must just be that same sentence letter. 
So, $\CAPPHI$ is the same sentence as $\CAPPSI$, and $\CAPPSI^*$ is the same as $\CAPPHI^*$. 
Since $\CAPPHI\sststile{}{}\CAPPHI^*$, it follows immediately that $\CAPPSI\sststile{}{}\CAPPSI^*$.
\item[Inheritance Step:] For the recursive hypothesis, assume that the theorem holds for the sentences $\CAPTHETA,\CAPTHETA_1,\ldots,\CAPTHETA_n,\DELTA$.
\begin{description}
\item[Conjunction:] Assume that $\CAPPSI$ is the conjunction $\conjunction{\CAPTHETA_1}{\conjunction{\ldots}{\CAPTHETA_{\integer{n}}}}$. 
Either $\CAPPHI$ is not a subsentence of $\CAPPSI$, it's a subsentence of $\CAPPSI$ but not the same as $\CAPPSI$, or is the same as $\CAPPSI$. 
If it's not a subsentence of $\CAPPSI$ or it's the same as $\CAPPSI$, then just as in the base step it follows immediately that $\CAPPSI\sststile{}{}\CAPPSI^*$.

So assume that $\CAPPHI$ is a subsentence of $\CAPPSI$ but not the same as it. 
Then $\CAPPHI$ is a subsentence of one of the conjuncts $\CAPTHETA_{\integer{i}}$ of $\CAPPSI$ and $\CAPPSI^*$ is the conjunction $\conjunction{\CAPTHETA_1}{\conjunction{\ldots}{\conjunction{\CAPTHETA_{\integer{i}}^*}{\conjunction{\ldots}{\CAPTHETA_{\integer{n}}}}}}$. 
By the recursive hypothesis, $\CAPTHETA_{\integer{i}}\sststile{}{}\CAPTHETA_{\integer{i}}^*$. 
It should be clear that by using \Rule{$\WEDGE$-Elim} we have that $\CAPPSI\sststile{}{}\CAPTHETA_1$, $\ldots$, $\CAPPSI\sststile{}{}\CAPTHETA_{\integer{i}}$, $\ldots$, $\CAPPSI\sststile{}{}\CAPTHETA_{\integer{n}}$. 
By transitivity, $\CAPPSI\sststile{}{}\CAPTHETA_{\integer{i}}^*$.
And, it should be clear that if each of $\CAPTHETA_1,\ldots,\CAPTHETA_{\integer{i}}^*,\ldots,\CAPTHETA_{\integer{n}}$ can be derived from $\CAPPSI$, then by using \Rule{$\WEDGE$-Intro} we can derive their conjunction $\conjunction{\CAPTHETA_1}{\conjunction{\ldots}{\conjunction{\CAPTHETA_{\integer{i}}^*}{\conjunction{\ldots}{\CAPTHETA_{\integer{n}}}}}}$ from $\CAPPSI$. 
But this conjunction just is $\CAPPSI^*$, so $\CAPPSI\sststile{}{}\CAPPSI^*$.

\item[Disjunction:] Assume that $\CAPPSI$ is the disjunction $\disjunction{\CAPTHETA_1}{\disjunction{\ldots}{\CAPTHETA_{\integer{n}}}}$. 
Either $\CAPPHI$ is not a subsentence of $\CAPPSI$, it's a subsentence of $\CAPPSI$ but not the same as $\CAPPSI$, or is the same as $\CAPPSI$. 
If it's not a subsentence of $\CAPPSI$ or it's the same as $\CAPPSI$, then just as in he base step it follows immediately that $\CAPPSI\sststile{}{}\CAPPSI^*$.

So assume that $\CAPPHI$ is a subsentence of $\CAPPSI$ but not the same as it. 
Then $\CAPPHI$ is a subsentence of one of the disjuncts $\CAPTHETA_{\integer{i}}$ of $\CAPPSI$ and $\CAPPSI^*$ is the disjunction $\disjunction{\CAPTHETA_1}{\disjunction{\ldots}{\disjunction{\CAPTHETA_{\integer{i}}^*}{\disjunction{\ldots}{\CAPTHETA_{\integer{n}}}}}}$.

We want to show that $\CAPPSI\sststile{}{}\CAPPSI^*$, i.e. that $\disjunction{\CAPTHETA_1}{\disjunction{\ldots}{\CAPTHETA_{\integer{n}}}}\sststile{}{}\disjunction{\CAPTHETA_1}{\disjunction{\ldots}{\disjunction{\CAPTHETA_{\integer{i}}^*}{\disjunction{\ldots}{\CAPTHETA_{\integer{n}}}}}}$. 
Using the proof by contradiction strategy, we write $\CAPPSI$ on the first line and write $\negation{\CAPPSI^*}$ on line 2 as an assumption with the goal of deriving a contradiction. 
Now we apply \Rule{DeMorgans} to line 2, getting $\conjunction{\negation{\CAPTHETA_1}}{\conjunction{\ldots}{\conjunction{\negation{\CAPTHETA_{\integer{i}}^*}}{\conjunction{\ldots}{\negation{\CAPTHETA_{\integer{n}}}}}}}$. 
Now using \Rule{$\WEDGE$-Elim} we break apart this conjunction, getting each conjunct $\negation{\CAPTHETA_1},\ldots,\negation{\CAPTHETA_{\integer{i}}^*},\ldots,\negation{\CAPTHETA_{\integer{n}}}$ on a separate line. 
Then using these conjuncts and \Rule{Disjunctive Syllogism} on line 1 to get $\CAPTHETA_{\integer{i}}$ on its own line. 
By the recursive hypothesis, $\CAPTHETA_{\integer{i}}\sststile{}{}\CAPTHETA_{\integer{i}}^*$, so from the line with $\CAPTHETA_{\integer{i}}$ we can derive $\CAPTHETA_{\integer{i}}^*$. 
Since we already have $\negation{\CAPTHETA_{\integer{i}}^*}$ on its own line, we can use \Rule{$\WEDGE$-Intro} to get $\conjunction{\CAPTHETA_{\integer{i}}^*}{\negation{\CAPTHETA_{\integer{i}}^*}}$. 
We then close the assumption on line 2 by using \Rule{$\HORSESHOE$-Intro} to get $\horseshoe{\negation{\CAPPSI^*}}{\parconjunction{\CAPTHETA_{\integer{i}}^*}{\negation{\CAPTHETA_{\integer{i}}^*}}}$. 
We finally use \Rule{$\NEGATION$-Elim} to get $\CAPPSI^*$.

\item[Negation:] Assume that $\CAPPSI$ is the negation $\negation{\CAPTHETA}$. 
Either $\CAPPHI$ is not a subsentence of $\CAPPSI$, it's a subsentence of $\CAPPSI$ but not the same as $\CAPPSI$, or is the same as $\CAPPSI$. 
If it's not a subsentence of $\CAPPSI$ or it's the same as $\CAPPSI$, then just as in he base step it follows immediately that $\CAPPSI\sststile{}{}\CAPPSI^*$.

So assume that $\CAPPHI$ is a subsentence of $\CAPPSI$ but not the same as it. 
Then $\CAPPHI$ is a subsentence of $\CAPTHETA$ and $\CAPPSI^*$ is the negation $\negation{\CAPTHETA^*}$.

We want to show that $\CAPPSI\sststile{}{}\CAPPSI^*$, i.e. that $\negation{\CAPTHETA}\sststile{}{}\;\negation{\CAPTHETA^*}$. Using our usual basic bottom-up strategy for negation, we set up this proof by putting $\negation{\CAPTHETA}$ on the first line and assuming $\CAPTHETA^*$ with the goal of deriving a contradiction. But, by the recursive hypothesis, $\CAPTHETA^*\sststile{}{}\CAPTHETA$. So we know that from assumption $\CAPTHETA^*$ we can derive $\CAPTHETA$, and at that point we have $\negation{\CAPTHETA}$ on one line and $\CAPTHETA$ on another. Using \Rule{$\WEDGE$-Intro} we can get $\conjunction{\CAPTHETA}{\negation{\CAPTHETA}}$. Then with \Rule{$\HORSESHOE$-Intro} we get $\horseshoe{\CAPTHETA^*}{\parconjunction{\CAPTHETA}{\negation{\CAPTHETA}}}$, and applying \Rule{$\NEGATION$-Intro} to this sentence will get us $\negation{\CAPTHETA^*}$. 

\item[Conditional:] Left to the reader as an exercise.

\item[Biconditional:] Also left to the reader as an exercise.

\end{description}
\item[Closure Step:] Since the inheritance step covers all the ways to generate \GSL{} sentences, we've shown that the theorem holds for all \GSL{} sentences $\CAPPSI$.  
\end{description}
\end{PROOF}

\noindent{}Thus we have shown that anything we can derive in \GSDP{} can be derived in \GSD{}.

\subsection{Shortcut Rule Strategies}\label{Sec:Shortcut Rule Strategies}
As we discussed in section \ref{Sec:Some Strategies}, for each logical connective there are two types of strategies: those for what to do if you already have sentences with that as their main connective (top-down strategies), and those for what to do if you want to get a sentence with that as its main connective (bottom-up strategies). 
In section \ref{Sec:Some Strategies} we covered basic top-down and bottom-up strategies for each connective.
We now add new strategies based on shortcut rules to this basic stock.
(Since shortcut rules don't divide nicely by connective, and often involve schemas with multiple connectives, some of the groupings here a bit arbitrary; but this isn't a substantial issue.)

\subsubsection*{Conjunction} 
\begin{description}
\item[\Rule{DeM} Top-down:] If you have a sentence of the form $\conjunction{\negation{\CAPPHI_1}}{\conjunction{\ldots}{\negation{\CAPPHI_{\integer{n}}}}}$, then convert it using \Rule{DeM} to get $\negation{\pardisjunction{\CAPPHI_1}{\disjunction{\ldots}{\CAPPHI_{\integer{n}}}}}$ on a new line.
\item[\Rule{$\NEGATION/\HORSESHOE$-Exchange} Top-down:] If you have a sentence of the form $\conjunction{\CAPPHI}{\negation{\CAPPSI}}$, then convert it using \Rule{$\NEGATION/\HORSESHOE$-Exchange} to get $\negation{\parhorseshoe{\CAPPHI}{\CAPPSI}}$ on a new line. 
\end{description} 
\subsubsection*{Disjunction} 
\begin{description}
\item[\Rule{D.S.} Top-down:] If you have a sentence of the form $\disjunction{\CAPPHI_1}{\disjunction{\ldots}{\disjunction{\CAPPHI_i}{\disjunction{\ldots}{\CAPPHI_{\integer{n}}}}}}$, and another sentence of the form $\negation{\CAPPHI_i}$, then eliminate one of the disjuncts using \Rule{D.S.} to get $\disjunction{\CAPPHI_1}{\disjunction{\ldots}{\disjunction{\CAPPHI_{i-1}}{\disjunction{\CAPPHI_{i+1}}{\disjunction{\ldots}{\CAPPHI_{\integer{n}}}}}}}$ on a new line.
\item[\Rule{DeM} Top-down:] If you have a sentence of the form $\disjunction{\negation{\CAPPHI_1}}{\disjunction{\ldots}{\negation{\CAPPHI_{\integer{n}}}}}$, then convert it using \Rule{DeM} to get $\negation{\parconjunction{\CAPPHI_1}{\conjunction{\ldots}{\CAPPHI_{\integer{n}}}}}$ on a new line.
\item[\Rule{$\HORSESHOE/\NEGATION$-Exchange} Top-down:] If you have a sentence of the form $\disjunction{\negation{\CAPPHI}}{\CAPPSI}$, then convert it using \Rule{$\HORSESHOE/\NEGATION$-Exchange} to get $\horseshoe{\CAPPHI}{\CAPPSI}$ on a new line.
\end{description} 
\subsubsection*{Negation} 
\begin{description}
\item[\Rule{DeM} Top-down:] If you have a sentence of the form $\negation{\parconjunction{\CAPPHI_1}{\conjunction{\ldots}{\CAPPHI_{\integer{n}}}}}$, then convert it using \Rule{DeM} to get $\disjunction{\negation{\CAPPHI_1}}{\disjunction{\ldots}{\negation{\CAPPHI_{\integer{n}}}}}$ on a new line.
\item[\Rule{DeM} Top-down:] If you have a sentence of the form $\negation{\pardisjunction{\CAPPHI_1}{\disjunction{\ldots}{\CAPPHI_{\integer{n}}}}}$, then convert it using \Rule{DeM} to get $\conjunction{\negation{\CAPPHI_1}}{\conjunction{\ldots}{\negation{\CAPPHI_{\integer{n}}}}}$ on a new line.
\item[\Rule{$\NEGATION\NEGATION$-Elim} Top-down:] If you have a sentence of the form $\negation{\negation{\CAPPHI}}$, then convert it using \Rule{$\NEGATION\NEGATION$-Elim} to get $\CAPPHI$ on a new line.
\item[\Rule{$\NEGATION/\HORSESHOE$-Exchange} Top-down:] If you have a sentence of the form $\negation{\parhorseshoe{\CAPPHI}{\CAPPSI}}$, then convert it using \Rule{$\NEGATION/\HORSESHOE$-Exchange} to get $\conjunction{\CAPPHI}{\negation{\CAPPSI}}$ on a new line.
\item[\Rule{$\NEGATION\NEGATION$-Intro} Bottom-up:] If you want to get a sentence of the form $\negation{\negation{\CAPPHI}}$, then first derive $\CAPPHI$ and then use \Rule{$\NEGATION\NEGATION$-Intro} to derive it. 
\end{description} 
\subsubsection*{Conditionals} 
\begin{description}
\item[\Rule{M.T.} Top-down:] If you have a sentence of the form $\horseshoe{\CAPPHI}{\CAPPSI}$, and another $\negation{\CAPPSI}$, then break the conditional apart using \Rule{M.T.} to get $\negation{\CAPPHI}$ on a new line.
\item[\Rule{$\HORSESHOE/\NEGATION$-Exchange} Top-down:] If you have a sentence of the form $\horseshoe{\CAPPHI}{\CAPPSI}$, then convert it using \Rule{$\HORSESHOE/\NEGATION$-Exchange} to get $\disjunction{\negation{\CAPPHI}}{\CAPPSI}$ on a new line.
\item[\Rule{Contraposition} Top-down:] If you have a sentence of the form $\horseshoe{\CAPPHI}{\CAPPSI}$, then convert it using \Rule{Contraposition} to get $\horseshoe{\negation{\CAPPSI}}{\negation{\CAPPHI}}$ on a new line.
\end{description} 
\subsubsection*{Biconditionals} 
\begin{description}
\item[\Rule{$\NEGATION/\TRIPLEBAR$-Intro} Top-down:] If you have a sentence of the form $\triplebar{\CAPPHI}{\CAPPSI}$, then convert it using \Rule{$\NEGATION/\TRIPLEBAR$-Intro} to get $\triplebar{\negation{\CAPPHI}}{\negation{\CAPPSI}}$ on a new line.
\end{description} 
\subsubsection*{Misc} 
\begin{description}
\item[\Rule{A.C.} Bottom-up:] If you want to get a sentence of the form $\CAPPSI$, then first derive both $\CAPPHI$ and $\negation{\CAPPHI}$ and then use \Rule{A.C.} to derive it from them. 
\end{description} 
\noindent{}Note that we've left strategies derived from \Rule{Distribution}. 
This isn't because they're not useful (quite the contrary). 
Instead, we leave it to the reader to place the appropriate Top-down strategies from \Rule{Distribution} under conjunction and disjunction. 

Also note that we've presented most of the strategies as top-down. 
For any of the strategies based on an exchange shortcut rule, it should be clear that you can ``reverse'' the strategy and read it as bottom-up. 

%%%%%%%%%%%%%%%%%%%%%%%%%%%%%%%%%%%%%%%%%%%%%%%%%%
\section{\GQD{}}\label{Section GQD}
%%%%%%%%%%%%%%%%%%%%%%%%%%%%%%%%%%%%%%%%%%%%%%%%%%

\subsection{Introduction and Elimination Rules}

Now our goal is to extend \GSD{} to  natural deduction system that will allow us to write derivations for \GQL{}. 
This system, which we call \idf{Quantificational Derivation System}, or \GQD{}, consists of all the rules of \GSD{}, plus an introduction and elimination rule for each quantifier (given in table \ref{GQD}). 
Like \GSD{}, we extend \GQD{} by adding shortcut rules. 
First, we can extend \GQD{} by adding all the shortcut rules we gave for \GSD{}. 
Second, we can extend \GQD{} by adding shortcut rules specifically for the quantifiers (see table \ref{GQDplus}).
The system with all the shortcut rules from \GSD{} (tables \ref{GSDplus1} and \ref{GSDplus1}) and the new shortcut rules for the quantifiers (table \ref{GQDplus}) is called \GQDP{}.
Note that some of the rules for \GQD{} have special restrictions.
We explain these in the examples below. 
%Since there are only two new connectives, introduce rules, strategies and do examples all at once.

%\begin{table}[!ht]
%\renewcommand{\arraystretch}{1.5}
%\begin{center}
%\begin{tabular}{ p{1in} l l } %p{2.2in} p{2in}
%\toprule
%\textbf{Name} & \textbf{Given} & \textbf{May Add} \\ 
%\midrule 
\renewcommand{\arraystretch}{1.5}
\begin{longtable}[c]{ p{1in} l l } %p{2.2in} p{2in}
\toprule
\textbf{Name} & \textbf{Given} & \textbf{May Add} \\ 
\midrule
\endfirsthead
\multicolumn{3}{c}{\emph{Continued from Previous Page}}\\
\toprule
\textbf{Name} & \textbf{Given} & \textbf{May Add} \\ 
\midrule
\endhead
\bottomrule
\caption{(New) Basic Rules for \GQD{}}\\[-.15in]
\multicolumn{3}{c}{\emph{Continued next Page}}\\
\endfoot
\bottomrule
\caption{(New) Basic Rules for \GQD{}}\\
\endlastfoot
\label{GQD}\Rule{$\forall$-Elim} & $\universal{\BETA}\CAPPHI$ & $\CAPPHI\constant{a}/\BETA$, for \mention{a} any  \\[-.25cm]
\nopagebreak
 &   &   individual constant \\
\Rule{$\forall$-Intro} & $\CAPPHI\constant{a}/\BETA$ & $\universal{\BETA}\CAPPHI$, iff \mention{a} does  \\[-.25cm]
\nopagebreak
 &  &  not occur in $\CAPPHI$  \\[-.25cm]
 \nopagebreak
 &  & nor in any unboxed line \\[-.25cm]
 \nopagebreak
& &  justified by Assumption\\
\Rule{$\exists$-Intro} & $\CAPPHI\constant{a}/\BETA$ & $\existential{\BETA}\CAPPHI$ \\
\Rule{$\exists$-Elim} & $\existential{\BETA}\CAPPHI$, $\horseshoe{\CAPPHI{\constant{a}/\BETA}}{\CAPTHETA}$ & $\CAPTHETA$, \Iff \mention{a} does \\[-.25cm]
\nopagebreak
 &  &  not occur in $\CAPPHI$ or $\CAPTHETA$, \\[-.25cm]
\nopagebreak
 & &  nor in any unboxed line\\[-.25cm]
 \nopagebreak
 & &  justified by Assumption\\
\end{longtable}
%\bottomrule
%\end{tabular}
%\end{center}
%\caption{(New) Basic Rules for \GQD{}}
%\label{GQD}
%\end{table}

The mechanics of writing proofs in \GQD{} are no different than the mechanics of writing proofs in \GSD{}, but we do have to slightly adapt the definition for sanctioning (Def. \pmvref{RuleSanctioning}).
\begin{majorILnc}{\LnpDC{GQDRuleSanctioning}}
A rule \Rule{R}, applied to unboxed lines $\integer{m}_1,\ldots,\integer{m}_{\integer{j}}$ with, respectively, sentences $\CAPPSI_1,\ldots,\CAPPSI_{\integer{j}}$, \df{sanctions} writing the sentence $\CAPPHI$ \Iff there's some substitution of \GQL{} \emph{formulas} that, for the given schemas of \Rule{R}, results in $\CAPPSI_1,\ldots,\CAPPSI_{\integer{j}}$ and, for the may-add schema, results in $\CAPPHI$. 
\end{majorILnc}
Another difference is that there are four new rules available. 
We now look at four examples which highlight each rule.

Our first example demonstrates \Rule{$\forall$-Elim}. 
Say we want to show that $\universal{\variable{x}}\parhorseshoe{\Ap{\variable{x}}}{\Bp{\variable{x}}},\Ap{\constant{d}}\sststile{}{}\Bp{\constant{d}}$.
We start by setting the two sentences on the \CAPS{lhs} of the turnstile as assumptions:
\begin{gproof}[\label{GQDExampleA}]
\galineNC{1}{$\universal{\variable{x}}\parhorseshoe{\Ap{\variable{x}}}{\Bp{\variable{x}}}$}{\Rule{Assume}}
\gaalineNC{2}{$\Ap{\constant{d}}$}{\Rule{Assume}}
\end{gproof}
Next we use \Rule{$\forall$-Elim} on line 1. 
Note that there are no restrictions on the use of \Rule{$\forall$-Elim}.
\begin{gproof}[\label{GQDExampleB}]
\galineNC{1}{$\universal{\variable{x}}\parhorseshoe{\Ap{\variable{x}}}{\Bp{\variable{x}}}$}{\Rule{Assume}}
\gaalineNC{2}{$\Ap{\constant{d}}$}{\Rule{Assume}}
\gaalineNC{3}{$\horseshoe{\Ap{\constant{d}}}{\Bp{\constant{d}}}$}{\Rule{$\forall$-Elim}, 1}
\end{gproof}
Although we could have substituted any constant for $\variable{x}$ in line 3, we chose $\constant{d}$ so that we can next apply \Rule{$\HORSESHOE$-Elim}. 
\begin{gproof}[\label{GQDExampleC}]
\galineNC{1}{$\universal{\variable{x}}\parhorseshoe{\Ap{\variable{x}}}{\Bp{\variable{x}}}$}{\Rule{Assume}}
\gaalineNC{2}{$\Ap{\constant{d}}$}{\Rule{Assume}}
\gaalineNC{3}{$\horseshoe{\Ap{\constant{d}}}{\Bp{\constant{d}}}$}{\Rule{$\forall$-Elim}, 1}
\gaalineNC{4}{$\Bp{\constant{d}}$}{\Rule{$\HORSESHOE$-Elim}, 2,3}
\end{gproof}
And this completes the derivation. 

The next example demonstrates \Rule{$\exists$-Intro}. Here we show that $\universal{\variable{y}}\bparhorseshoe{\existential{\variable{x}}\Dpp{\variable{x}}{\variable{y}}}{\Bp{\variable{y}}},\Dpp{\constant{a}}{\constant{b}}\sststile{}{}\Bp{\constant{b}}$.
As before, we start with the assumptions.
\begin{gproof}[\label{GQDExampleD}]
\galineNC{1}{$\universal{\variable{y}}\bparhorseshoe{\existential{\variable{x}}\Dpp{\variable{x}}{\variable{y}}}{\Bp{\variable{y}}}$}{\Rule{Assume}}
\gaalineNC{2}{$\Dpp{\constant{a}}{\constant{b}}$}{\Rule{Assume}}
\end{gproof}
Again as before, we need to use \Rule{$\forall$-Elim} so we can eventually use \Rule{$\HORSESHOE$-Elim} to finish. 
\begin{gproof}[\label{GQDExampleE}]
\galineNC{1}{$\universal{\variable{y}}\bparhorseshoe{\existential{\variable{x}}\Dpp{\variable{x}}{\variable{y}}}{\Bp{\variable{y}}}$}{\Rule{Assume}}
\gaalineNC{2}{$\Dpp{\constant{a}}{\constant{b}}$}{\Rule{Assume}}
\gaalineNC{3}{$\horseshoe{\existential{\variable{x}}\Dpp{\variable{x}}{\constant{b}}}{\Bp{\variable{\constant{b}}}}$}{\Rule{$\forall$-Elim}, 1}
\end{gproof}
Again we strategically chose the constant we did, $\constant{b}$, so that using \Rule{$\HORSESHOE$-Elim} will get us the right result. 
But we can't apply \Rule{$\HORSESHOE$-Elim} yet, since the \CAPS{lhs} of the horseshoe in line 3, $\existential{\variable{x}}\Dpp{\variable{x}}{\constant{b}}$, is an existential sentence, while what we have on line 2, $\Dpp{\constant{a}}{\constant{b}}$, is not. 
But, by using \Rule{$\exists$-Intro} we can easily get what we need:
\begin{gproof}[\label{GQDExampleF}]
\galineNC{1}{$\universal{\variable{y}}\bparhorseshoe{\existential{\variable{x}}\Dpp{\variable{x}}{\variable{y}}}{\Bp{\variable{y}}}$}{\Rule{Assume}}
\gaalineNC{2}{$\Dpp{\constant{a}}{\constant{b}}$}{\Rule{Assume}}
\gaalineNC{3}{$\horseshoe{\existential{\variable{x}}\Dpp{\variable{x}}{\constant{b}}}{\Bp{\variable{\constant{b}}}}$}{\Rule{$\forall$-Elim}, 1}
\gaalineNC{4}{$\existential{\variable{x}}\Dpp{\variable{x}}{\constant{b}}$}{\Rule{$\exists$-Intro}, 2}
\end{gproof}
Note that just as with \Rule{$\forall$-Elim} there are no restrictions on \Rule{$\HORSESHOE$-Elim}.
Also note that we could have used \Rule{$\HORSESHOE$-Elim} to generalize the constant $\constant{b}$, getting $\existential{\variable{x}}\Dpp{\constant{a}}{\variable{x}}$, but that wouldn't have helped us.
Also, we could have generalized using a different variable, getting, say $\existential{\variable{z}}\Dpp{\variable{z}}{\constant{b}}$.
But again that wouldn't have helped us. 
Any constant is legal to instantiate with \Rule{$\forall$-Elim}, but often only one is a wise choice.

We now have the right setup for \Rule{$\HORSESHOE$-Elim}:
\begin{gproof}[\label{GQDExampleG}]
\galineNC{1}{$\universal{\variable{y}}\bparhorseshoe{\existential{\variable{x}}\Dpp{\variable{x}}{\variable{y}}}{\Bp{\variable{y}}}$}{\Rule{Assume}}
\gaalineNC{2}{$\Dpp{\constant{a}}{\constant{b}}$}{\Rule{Assume}}
\gaalineNC{3}{$\horseshoe{\existential{\variable{x}}\Dpp{\variable{x}}{\constant{b}}}{\Bp{\variable{\constant{b}}}}$}{\Rule{$\forall$-Elim}, 1}
\gaalineNC{4}{$\existential{\variable{x}}\Dpp{\variable{x}}{\constant{b}}$}{\Rule{$\exists$-Intro}, 2}
\gaalineNC{5}{$\Bp{\constant{b}}$}{\Rule{$\HORSESHOE$-Elim}, 3,4}
\end{gproof}
And this completes the derivation. 

The next example demonstrates \Rule{$\exists$-Elim}. 
This is our first rule with restrictions.
We show that $\universal{\variable{x}}\negation{\Qpp{\variable{x}}{\constant{b}}},\existential{\variable{z}}\pardisjunction{\Qpp{\variable{z}}{\constant{b}}}{\Gp{\variable{z}}}\sststile{}{}\existential{\variable{x}}\Gp{\variable{x}}$.
As before, we start by setting out the assumptions. 
\begin{gproof}[\label{GQDExampleH}]
\galineNC{1}{$\universal{\variable{x}}\negation{\Qpp{\variable{x}}{\constant{b}}}$}{\Rule{Assume}}
\gaalineNC{2}{$\existential{\variable{z}}\pardisjunction{\Qpp{\variable{z}}{\constant{b}}}{\Gp{\variable{z}}}$}{\Rule{Assume}}
\end{gproof}

When using both \Rule{$\exists$-Elim} and \Rule{$\forall$-Elim}, it’s generally best to take the instance of the existential for \Rule{$\exists$-Elim} first and use \Rule{$\forall$-Elim} after. In some cases
you want the \Rule{$\forall$-Elim} instance to match the \Rule{$\exists$-Elim} constant, but in other cases you want it to be different.
Now, according to \Rule{$\exists$-Elim}, we can get a sentence $\CAPTHETA$ in this case by showing that $\horseshoe{\pardisjunction{\Qpp{\variable{t}}{\constant{b}}}{\Gp{\variable{t}}}}{\CAPTHETA}$, for some constant $\variable{t}$ that fits the restriction on \Rule{$\exists$-Elim}.
(Just what constants will work and what sentence $\CAPTHETA$ we want will take some thought.)
So, we need to use \Rule{$\HORSESHOE$-Intro}. 
\begin{gproof}[\label{GQDExampleI}]
\galineNC{1}{$\universal{\variable{x}}\negation{\Qpp{\variable{x}}{\constant{b}}}$}{\Rule{Assume}}
\gaalineNC{2}{$\existential{\variable{z}}\pardisjunction{\Qpp{\variable{z}}{\constant{b}}}{\Gp{\variable{z}}}$}{\Rule{Assume}}
\gaaaproof{
\gaaalineSS{3}{$\disjunction{\Qpp{\constant{a}}{\constant{b}}}{\Gp{\constant{a}}}$}{\Rule{Assume}}
\gaaalinendSS{}{}{}
\gaaalinendSS{}{$\qquad\vdots$}{}
\gaaalinendSS{}{}{}
\gaaalineSS{$\integer{n}$}{$\CAPTHETA$}{}
}
\gaalineNC{$\integer{n}+1$}{$\horseshoe{\pardisjunction{\Qpp{\constant{a}}{\constant{b}}}{\Gp{\constant{a}}}}{\CAPTHETA}$}{\Rule{$\HORSESHOE$-Intro}, 3--$\integer{n}$}
\gaalineNC{$\integer{n}+2$}{$\CAPTHETA$}{\Rule{$\exists$-Elim}, 2,$\integer{n}+1$}
\end{gproof}
Here we have chosen $\constant{a}$ to substitute for $\variable{z}$ in line 3 because (so long as we pick $\CAPTHETA$ correctly) it will meet our restriction. 
According to the restriction on \Rule{$\exists$-Elim}, the constant we pick can't appear in any open assumptions (the assumption used to derive the needed conditional, in this case line 3, is not open by the time we apply the rule). 
It also can't appear in the scope of the existential quantifier, which in this case is $\pardisjunction{\Qpp{\variable{z}}{\constant{b}}}{\Gp{\variable{z}}}$. 
Since $\constant{a}$ does not appear in any open assumptions and does not appear in $\pardisjunction{\Qpp{\variable{z}}{\constant{b}}}{\Gp{\variable{z}}}$, it will meet our restriction.
(Clearly other constants would have also met the restriction.)
Now we have to decide what sentence $\CAPTHETA$ enables us to finish the derivation. 
Note that if we derive $\Gp{\variable{t}}$ for any constant $\variable{t}\neq\constant{a}$, then we can use \Rule{$\exists$-Intro} to get the needed sentence $\existential{\variable{x}}\Gp{\variable{x}}$. 
(We will set this up, but we see in a moment that this first guess won't work.)
We can't pick $\variable{t}=\constant{a}$ because then we couldn't apply \Rule{$\exists$-Elim} on line $\integer{n}+2$.
\begin{gproof}[\label{GQDExampleJ}]
\galineNC{1}{$\universal{\variable{x}}\negation{\Qpp{\variable{x}}{\constant{b}}}$}{\Rule{Assume}}
\gaalineNC{2}{$\existential{\variable{z}}\pardisjunction{\Qpp{\variable{z}}{\constant{b}}}{\Gp{\variable{z}}}$}{\Rule{Assume}}
\gaaaproof{
\gaaalineSS{3}{$\disjunction{\Qpp{\constant{a}}{\constant{b}}}{\Gp{\constant{a}}}$}{\Rule{Assume}}
\gaaalinendSS{}{}{}
\gaaalinendSS{}{$\qquad\vdots$}{}
\gaaalinendSS{}{}{}
\gaaalineSS{$\integer{n}$}{$\Gp{\constant{c}}$}{}
}
\gaalineNC{$\integer{n}+1$}{$\horseshoe{\pardisjunction{\Qpp{\constant{a}}{\constant{b}}}{\Gp{\constant{a}}}}{\Gp{\constant{c}}}$}{\Rule{$\HORSESHOE$-Intro}, 3--$\integer{n}$}
\gaalineNC{$\integer{n}+2$}{$\Gp{\constant{c}}$}{\Rule{$\exists$-Elim}, 2,$\integer{n}+1$}
\gaalineNC{$\integer{n}+3$}{$\existential{\variable{x}}\Gp{\variable{x}}$}{\Rule{$\exists$-Intro}, $\integer{n}+2$}
\end{gproof}
Now we only need to complete the derivation by deriving $\Gp{\constant{c}}$ from $\pardisjunction{\Qpp{\constant{a}}{\constant{b}}}{\Gp{\constant{a}}}$. 
We can \emph{try} do this by using \Rule{$\forall$-Elim} and \Rule{D.S.}, but it becomes clear at once that this won't work.
\begin{gproof}[\label{GQDExampleK}]
\galineNC{1}{$\universal{\variable{x}}\negation{\Qpp{\variable{x}}{\constant{b}}}$}{\Rule{Assume}}
\gaalineNC{2}{$\existential{\variable{z}}\pardisjunction{\Qpp{\variable{z}}{\constant{b}}}{\Gp{\variable{z}}}$}{\Rule{Assume}}
\gaaaproof{
\gaaalineSS{3}{$\disjunction{\Qpp{\constant{a}}{\constant{b}}}{\Gp{\constant{a}}}$}{\Rule{Assume}}
\gaaalineSS{4}{$\negation{\Qpp{\constant{a}}{\constant{b}}}$}{\Rule{$\forall$-Elim}, 1}
\gaaalinendSS{5}{$\Gp{\constant{a}}$}{\Rule{D.S.}, 3,4}
\gaaalinendSS{}{}{}
\gaaalinendSS{}{$\qquad\vdots$}{}
\gaaalinendSS{}{}{}
\gaaalineSS{$\integer{n}$}{$\Gp{\constant{c}}$}{}
}
\gaalineNC{$\integer{n}+1$}{$\horseshoe{\pardisjunction{\Qpp{\constant{a}}{\constant{b}}}{\Gp{\constant{a}}}}{\Gp{\constant{c}}}$}{\Rule{$\HORSESHOE$-Intro}, 3--$\integer{n}$}
\gaalineNC{$\integer{n}+2$}{$\Gp{\constant{c}}$}{\Rule{$\exists$-Elim}, 2,$\integer{n}+1$}
\gaalineNC{$\integer{n}+3$}{$\existential{\variable{x}}\Gp{\variable{x}}$}{\Rule{$\exists$-Intro}, $\integer{n}+2$}
\end{gproof}
It should be clear that this \Rule{$\forall$-Elim}/\Rule{D.S.} strategy won't work, since for it to work we'd need the token of $\GG$ that appears in line 3 to be followed by the \emph{same} constant that follows the token of $\GG$ that appears in line $\integer{n}$. 
But that constant is the constant we substitute in for \Rule{$\exists$-Elim}, and we would then violate the restriction for the rule. 

So we need to go back to $\CAPTHETA$ and consider another strategy. 
This time we try letting $\CAPTHETA=\existential{\variable{x}}\Gp{\variable{x}}$.
\begin{gproof}[\label{GQDExampleL}]
\galineNC{1}{$\universal{\variable{x}}\negation{\Qpp{\variable{x}}{\constant{b}}}$}{\Rule{Assume}}
\gaalineNC{2}{$\existential{\variable{z}}\pardisjunction{\Qpp{\variable{z}}{\constant{b}}}{\Gp{\variable{z}}}$}{\Rule{Assume}}
\gaaaproof{
\gaaalineSS{3}{$\disjunction{\Qpp{\constant{a}}{\constant{b}}}{\Gp{\constant{a}}}$}{\Rule{Assume}}
\gaaalinendSS{}{}{}
\gaaalinendSS{}{$\qquad\vdots$}{}
\gaaalinendSS{}{}{}
\gaaalineSS{$\integer{n}$}{$\existential{\variable{x}}\Gp{\variable{x}}$}{}
}
\gaalineNC{$\integer{n}+1$}{$\horseshoe{\pardisjunction{\Qpp{\constant{a}}{\constant{b}}}{\Gp{\constant{a}}}}{\existential{\variable{x}}\Gp{\variable{x}}}$}{\Rule{$\HORSESHOE$-Intro}, 3--$\integer{n}$}
\gaalineNC{$\integer{n}+2$}{$\existential{\variable{x}}\Gp{\variable{x}}$}{\Rule{$\exists$-Elim}, 2,$\integer{n}+1$}
\end{gproof}
Note that we still keep our choice of $\constant{a}$ on line 3 within the restrictions of \Rule{$\exists$-Elim}. 
Now we can try again at finishing the proof. 
This time we can:
\begin{gproof}[\label{GQDExampleM}]
\galineNC{1}{$\universal{\variable{x}}\negation{\Qpp{\variable{x}}{\constant{b}}}$}{\Rule{Assume}}
\gaalineNC{2}{$\existential{\variable{z}}\pardisjunction{\Qpp{\variable{z}}{\constant{b}}}{\Gp{\variable{z}}}$}{\Rule{Assume}}
\gaaaproof{
\gaaalineSS{3}{$\disjunction{\Qpp{\constant{a}}{\constant{b}}}{\Gp{\constant{a}}}$}{\Rule{Assume}}
\gaaalineSS{4}{$\negation{\Qpp{\constant{a}}{\constant{b}}}$}{\Rule{$\forall$-Elim}, 1}
\gaaalineSS{5}{$\Gp{\constant{a}}$}{\Rule{D.S.}, 3,4}
\gaaalineSS{6}{$\existential{\variable{x}}\Gp{\variable{x}}$}{\Rule{$\exists$-Intro}, 5}
}
\gaalineNC{7}{$\horseshoe{\pardisjunction{\Qpp{\constant{a}}{\constant{b}}}{\Gp{\constant{a}}}}{\existential{\variable{x}}\Gp{\variable{x}}}$}{\Rule{$\HORSESHOE$-Intro}, 3--6}
\gaalineNC{8}{$\existential{\variable{x}}\Gp{\variable{x}}$}{\Rule{$\exists$-Elim}, 2,7}
\end{gproof}
And this completes the derivation. 

The final example demonstrates \Rule{$\forall$-Intro}. 
This rule also has restrictions. 
We show that $\universal{\variable{x}}\parhorseshoe{\Ap{\variable{x}}}{\Bp{\variable{x}}},\universal{\variable{x}}\parhorseshoe{\Bp{\variable{x}}}{\Hp{\variable{x}}}\sststile{}{}\universal{\variable{x}}\parhorseshoe{\Ap{\variable{x}}}{\Hp{\variable{x}}}$.
It's worth mentioning that the fact that all the examples so far have two sentences on the \CAPS{lhs} of the turnstile is just a coincidence.
As before, we start by setting out the assumptions.
\begin{gproof}[\label{GQDExampleN}]
\galineNC{1}{$\universal{\variable{x}}\parhorseshoe{\Ap{\variable{x}}}{\Bp{\variable{x}}}$}{\Rule{Assume}}
\gaalineNC{2}{$\universal{\variable{x}}\parhorseshoe{\Bp{\variable{x}}}{\Hp{\variable{x}}}$}{\Rule{Assume}}
\end{gproof}
Before moving forward, we need to think about how \Rule{$\forall$-Intro} works. 
According to the rule, if we want to get $\universal{\variable{x}}\parhorseshoe{\Ap{\variable{x}}}{\Hp{\variable{x}}}$ by using \Rule{$\forall$-Intro}, we need to first get $\horseshoe{\Ap{\variable{t}}}{\Hp{\variable{t}}}$ on some line, where $\variable{t}$ is some constant that does not appear in any unboxed assumptions, or in $\horseshoe{\Ap{\variable{x}}}{\Hp{\variable{x}}}$.
Since no constants appear in $\horseshoe{\Ap{\variable{x}}}{\Hp{\variable{x}}}$, it provides no constraints. 
Further, there are no constants in either unboxed assumption in derivation \ref{GQDExampleN}, so we are free to chose any constant we like. 
Pick $\constant{a}$. 
So we want to derive $\horseshoe{\Ap{\constant{a}}}{\Hp{\constant{a}}}$.
\begin{gproof}[\label{GQDExampleO}]
\galineNC{1}{$\universal{\variable{x}}\parhorseshoe{\Ap{\variable{x}}}{\Bp{\variable{x}}}$}{\Rule{Assume}}
\gaalineNC{2}{$\universal{\variable{x}}\parhorseshoe{\Bp{\variable{x}}}{\Hp{\variable{x}}}$}{\Rule{Assume}}
\gaaaproof{
\gaaalineSS{3}{$\Ap{\constant{a}}$}{\Rule{Assume}}
\gaaalinendSS{}{}{}
\gaaalinendSS{}{$\qquad\vdots$}{}
\gaaalinendSS{}{}{}
\gaaalineSS{$\integer{n}$}{$\Hp{\constant{a}}$}{}
}
\gaalineNC{$\integer{n}+1$}{$\horseshoe{\Ap{\constant{a}}}{\Hp{\constant{a}}}$}{\Rule{$\HORSESHOE$-Intro}, 3--$\integer{n}$}
\gaalineNC{$\integer{n}+2$}{$\universal{\variable{x}}\parhorseshoe{\Ap{\variable{x}}}{\Hp{\variable{x}}}$}{\Rule{$\forall$-Intro}, $\integer{n}+1$}
\end{gproof}
Now we just have to finish the derivation of $\Hp{\constant{a}}$ from $\Ap{\constant{a}}$ without introducing any new unboxed assumptions with constant $\constant{a}$. 
(Of course, if we did the bottom half of this proof wouldn't work, since we coudn't close the assumption on line 3 with \Rule{$\HORSESHOE$-Intro} if unboxed assumptions appeared after line 3.)
A natural next step is to use \Rule{$\forall$-Elim} on lines 1 and 2.
\begin{gproof}[\label{GQDExampleP}]
\galineNC{1}{$\universal{\variable{x}}\parhorseshoe{\Ap{\variable{x}}}{\Bp{\variable{x}}}$}{\Rule{Assume}}
\gaalineNC{2}{$\universal{\variable{x}}\parhorseshoe{\Bp{\variable{x}}}{\Hp{\variable{x}}}$}{\Rule{Assume}}
\gaaaproof{
\gaaalineSS{3}{$\Ap{\constant{a}}$}{\Rule{Assume}}
\gaaalineSS{4}{$\horseshoe{\Ap{\constant{a}}}{\Bp{\constant{a}}}$}{\Rule{$\forall$-Elim}, 1}
\gaaalineSS{5}{$\horseshoe{\Bp{\constant{a}}}{\Hp{\constant{a}}}$}{\Rule{$\forall$-Elim}, 2}
\gaaalinendSS{}{}{}
\gaaalinendSS{}{$\qquad\vdots$}{}
\gaaalinendSS{}{}{}
\gaaalineSS{$\integer{n}$}{$\Hp{\constant{a}}$}{}
}
\gaalineNC{$\integer{n}+1$}{$\horseshoe{\Ap{\constant{a}}}{\Hp{\constant{a}}}$}{\Rule{$\HORSESHOE$-Intro}, 3--$\integer{n}$}
\gaalineNC{$\integer{n}+2$}{$\universal{\variable{x}}\parhorseshoe{\Ap{\variable{x}}}{\Hp{\variable{x}}}$}{\Rule{$\forall$-Intro}, $\integer{n}+1$}
\end{gproof}
Now getting to $\Hp{\constant{a}}$ is just a matter of using \Rule{$\HORSESHOE$-Elim}:
\begin{gproof}[\label{GQDExampleQ}]
\galineNC{1}{$\universal{\variable{x}}\parhorseshoe{\Ap{\variable{x}}}{\Bp{\variable{x}}}$}{\Rule{Assume}}
\gaalineNC{2}{$\universal{\variable{x}}\parhorseshoe{\Bp{\variable{x}}}{\Hp{\variable{x}}}$}{\Rule{Assume}}
\gaaaproof{
\gaaalineSS{3}{$\Ap{\constant{a}}$}{\Rule{Assume}}
\gaaalineSS{4}{$\horseshoe{\Ap{\constant{a}}}{\Bp{\constant{a}}}$}{\Rule{$\forall$-Elim}, 1}
\gaaalineSS{5}{$\horseshoe{\Bp{\constant{a}}}{\Hp{\constant{a}}}$}{\Rule{$\forall$-Elim}, 2}
\gaaalineSS{6}{$\Bp{\constant{a}}$}{\Rule{$\HORSESHOE$-Elim}, 3,4}
\gaaalineSS{7}{$\Hp{\constant{a}}$}{\Rule{$\HORSESHOE$-Elim}, 5,6}
}
\gaalineNC{8}{$\horseshoe{\Ap{\constant{a}}}{\Hp{\constant{a}}}$}{\Rule{$\HORSESHOE$-Intro}, 3--7}
\gaalineNC{9}{$\universal{\variable{x}}\parhorseshoe{\Ap{\variable{x}}}{\Hp{\variable{x}}}$}{\Rule{$\forall$-Intro}, 8}
\end{gproof}
And this completes the derivation. Notice that since there are infinitely many constants in \GQL{}, and any of them would have worked in the derivation, there are infinitely many derivations of this
sentence.

\subsection{Shortcut Rules for \GQD{}}

All of the shortcut rules for \GSD{}, both the standard and exchange rules, can be carried over as shortcut rules for \GQD{}. 
In addition, there are four new shortcut rules for \GQD{}, the quantifier negation rules. 
These are found in table \ref{GQDplus}.
%\begin{table}[!ht]
%\renewcommand{\arraystretch}{1.5}
%\begin{center}
%\begin{tabular}{ p{1in} l l } %p{2.2in} p{2in}
%\toprule
%\textbf{Name} & \textbf{Given} & \textbf{May Add} \\ 
%\midrule
\renewcommand{\arraystretch}{1.5}
\begin{longtable}[c]{ p{1in} l l } %p{2.2in} p{2in}
\toprule
\textbf{Name} & \textbf{Given} & \textbf{May Add} \\ 
\midrule
\endfirsthead
\multicolumn{3}{c}{\emph{Continued from Previous Page}}\\
\toprule
\textbf{Name} & \textbf{Given} & \textbf{May Add} \\ 
\midrule
\endhead
\bottomrule
\caption{Exchange Short-Cut Rules for \GQD{}}\\[-.15in]
\multicolumn{3}{c}{\emph{Continued next Page}}\\
\endfoot
\bottomrule
\caption{Exchange Short-Cut Rules for \GQD{}}\\
\endlastfoot
\label{GQDplus}\Rule{QN} & $\negation{\universal{\BETA}{\CAPPHI}}$ & $\existential{\BETA}\negation{{\CAPPHI}}$ \\
 & $\existential{\BETA}\negation{{\CAPPHI}}$ & $\negation{\universal{\BETA}{\CAPPHI}}$  \\
 & $\negation{\existential{\BETA}{\CAPPHI}}$ & $\universal{\BETA}\negation{{\CAPPHI}}$ \\
 &  $\universal{\BETA}\negation{{\CAPPHI}}$ & $\negation{\existential{\BETA}{\CAPPHI}}$ \\
\end{longtable}
%\bottomrule
%\end{tabular}
%\end{center}
%\caption{Exchange Short-Cut Rules for \GQD{} (\GQDP{})}
%\label{GQDplus}
%\end{table}

Just as we had to slightly modify the definition of sanctioning used in \GSD{} for the basic (intro and elimination) rules and standard shortcut rules for \GQD{}, we also have to slightly modify the definition of sanctioning used in \GSDP{} (Def. \pmvref{ExchangeRuleSanctioning}) for the exchange shortcut rules for \GQD{} that make up \GQDP{}.
\begin{majorILnc}{\LnpDC{GQDExchangeRuleSanctioning}}
An exchange shortcut rule \Rule{R} of \GQDP{} (a rule from table \pmvref{GSDplus2}, or table \pmvref{GQDplus}), applied to a line with a \GQL{} formula $\CAPPSI$, \emph{sanctions} writing down sentence $\CAPPSI^*$ \Iff
\begin{cenumerate}
\item there is some substitution of \GQL{} formulas that, for the given schema of \Rule{R}, results in a formula $\CAPPHI$ and, for the may-add schema, results in a formula $\CAPPHI^*$,
\item $\CAPPHI$ is a subformula of $\CAPPSI$, and
\item $\CAPPSI^*$ is the \GSL{} sentence you get when you replace one instance (token) of $\CAPPHI$ with an instance (token) of $\CAPPHI^*$ in $\CAPPSI$. 
\end{cenumerate}
\end{majorILnc}

\subsection{Shortcut Rule Elimination Theorem for \GQD{}}\label{Shortcut Rule Elimination Theorem Section}

% we want to show that theorem \ref{GSD Shortcut Theorem3} still holds; or rather, that: if you can derive it in GQD+, then you can derive it in GSD. I think the proof of \ref{GSD Shortcut Theorem3} will carry over, so long as we make sure \ref{GSD Shortcut Theorem2} and \ref{GSD Shortcut Theorem} still hold. The proof for \ref{GSD Shortcut Theorem3} does need to be addressed again, since the derivation schemas for the shortcut rules for GQD+ need to be careful about variable substitutions. For \ref{GSD Shortcut Theorem2}, we need to show that the shortcut rules for GQD+ are provably equivalent (make hw?) and we need to show that theorem \ref{ExchangeRuleTheorem} holds for all GQL sentences too. To do this we just need to extend the recursive proof with two new clauses for univ and ext quantifiers. I think that's all we have to do to extend \ref{GSD Shortcut Theorem3}.

In this section we want to extend the Shortcut Rule Elimination Theorem (Thm. \pmvref{GSD Shortcut Theorem3}) to \GQD{}.
\begin{THEOREM}{\LnpTC{GQD Shortcut Theorem3} Shortcut Rule Elimination Theorem for \GQDP{}:}
For all \GQL{} sentences $\CAPPHI_1,\ldots,\CAPPHI_{\integer{m}}$ and $\CAPPSI$, if $\CAPPSI$ can be derived from $\CAPPHI_1,\ldots,\CAPPHI_{\integer{m}}$ in \GQDP{}, then $\CAPPSI$ can be derived from $\CAPPHI_1,\ldots,\CAPPHI_{\integer{m}}$ in \GQD{}.
\end{THEOREM}
\noindent{}The same proof we used for Shortcut Rule Elimination Theorem for \GSD{} (Thm. \ref{GSD Shortcut Theorem3}) will work for this version, so long as appropriate versions of theorems \mvref{GSD Shortcut Theorem} and \mvref{GSD Shortcut Theorem2} hold for the shortcut rules of \GQD{}.
Specifically:
\begin{THEOREM}{\LnpTC{GQD Shortcut Theorem}}
For all \GQL{} sentences $\CAPTHETA_1,\ldots,\CAPTHETA_{\integer{n}},\DELTA$ and rules \Rule{R$_1$}$,\ldots,$\Rule{R$_\integer{p}$}, if
\begin{cenumerate}
\item $\DELTA$ can be derived from $\CAPTHETA_1,\ldots,\CAPTHETA_{\integer{n}}$ using rules \Rule{R$_1$}$,\ldots,$\Rule{R$_\integer{p}$} and the basic rules of \GSD{}, and
\item every application of a rule \Rule{R$_1$} is derivable using the rules \Rule{R$_2$}, $\ldots$, \Rule{R$_\integer{p}$} and the basic rules of \GQD{} (recall Def. \pmvref{RuleInstanceDerivability}),
\end{cenumerate}
then $\DELTA$ can be derived from $\CAPTHETA_1,\ldots,\CAPTHETA_{\integer{n}}$ using only rules \Rule{R$_2$}$,\ldots,$\Rule{R$_\integer{p}$} and the basic rules of \GQD{}.
\end{THEOREM}
\begin{THEOREM}{\LnpTC{GQD Shortcut Theorem2}}
For all standard and exchange shortcut rules \Rule{R} (see tables \ref{GSDplus1}, \ref{GSDplus2}, and \ref{GQDplus}), every application of \Rule{R} is derivable using the basic rules of \GQD{} (see tables \ref{GSD} and \ref{GQD}).
\end{THEOREM}
\noindent{}We leave it to the reader to prove the Shortcut Rule Elimination Theorem (Thm. \ref{GQD Shortcut Theorem3}) using theorems \ref{GQD Shortcut Theorem} and \ref{GQD Shortcut Theorem2}.

Turning to the proofs for theorems \ref{GQD Shortcut Theorem} and \ref{GQD Shortcut Theorem2}, note that nothing in the proof of \ref{GSD Shortcut Theorem} depended on any special features of \GSD{}. 
Thus, the proof of theorem \ref{GSD Shortcut Theorem} can be adapted to \ref{GQD Shortcut Theorem} just by changing all the references to \GSD{} to references to \GQD{}. 
But unfortunately theorem \ref{GQD Shortcut Theorem2} is not nearly as straightforward.  

It is helpful to break theorem \ref{GQD Shortcut Theorem2} into two parts: (i) the claim that, for all standard shortcut rules (table \ref{GSDplus1}), every application is derivable using the basic rules of \GQD{}, and (ii) the claim that, for all the exchange shortcut rules (tables \ref{GSDplus2} and \ref{GQDplus}), every application is derivable using the basic rules of \GQD{}. 
Proving part (i) of theorem \ref{GQD Shortcut Theorem2} is no different from proving it for theorem \ref{GSD Shortcut Theorem2}; the same arguments using the derivation schemas written for theorem \ref{GSD Shortcut Theorem2} will work.
But nothing done so far will help with part (ii).
This is because applications of exchange shortcut rules in \GQDP{}, even those shared with \GSDP{} (see table \ref{GSDplus2}), use a different definition of sanctioning than was used in \GSDP{} (compare Def. \ref{ExchangeRuleSanctioning} and \ref{GQDExchangeRuleSanctioning}). 
According to definition \mvref{GQDExchangeRuleSanctioning}, in \GQDP{} exchange rules can be applied not only to subsentences, but also to subformulas. 
We have to show that allowing exchange rules to be applied not only to subsentences, but also to subformulas, doesn't prevent us from deriving their applications using only the basic rules.

To do this we use (1) an extended (and generalized) version of the Restricted Replacement Theorem for \GSD{} (Thm. \pmvref{ExchangeRuleTheorem}) and (2) the fact that any two formulas got by substituting \GQL{} formulas into the may-add and given schemas of the exchange shortcut rules for \GQD{} are provably equivalent. 
(We ask the reader to prove this second fact in exercise \pmvref{exer:GQDSCprovablyequiv}.)
\begin{THEOREM}{\LnpTC{GQD Replacement Theorem} The Replacement Theorem for \GQD{}:}
If $\CAPPHI$ and $\CAPPHI^*$ are provably equivalent formulas of \GQL{}, and $\CAPTHETA$ and $\CAPTHETA^*$ differ only in that $\CAPTHETA$ contains the subformula $\CAPPHI$ in one place where $\CAPTHETA^*$ contains the subformula $\CAPPHI^*$, then $\CAPTHETA$ and $\CAPTHETA^*$ are provably equivalent.
\end{THEOREM}
\noindent{}Before proving this theorem, we need to define when two \emph{formulas} of \GQL{} are provably equivalent. 
The definition given for \GSL{} sentences (def. \pmvref{GSDprovablyequivalent}) will not carry over to \GQL{} formulas, since formulas just aren't the sort of thing which can be derived. 
For example, we would like to be able to say that $\parhorseshoe{\Qp{\variable{x}}}{\Gp{\variable{y}}}$ and $\pardisjunction{\negation{\Qp{\variable{x}}}}{\Gp{\variable{y}}}$ are provably equivalent even though we cannot derive the formula $\triplebar{\parhorseshoe{\Qp{\variable{x}}}{\Gp{\variable{y}}}}{\pardisjunction{\negation{\Qp{\variable{x}}}}{\Gp{\variable{y}}}}$ (because it's not a sentence). 

The way we extend the notion of provable equivalence is through the universal closure of a formula. 
\begin{majorILnc}{\LnpDC{Universal Closure}}
The \df{universal closure} of a formula $\CAPTHETA$, written $\forall\CAPTHETA$, is the sentence that results by prefixing universal quantifiers in alphabetical order for all free variables of $\CAPTHETA$. 
\end{majorILnc}
\noindent{}E.g., $\forall\bpartriplebar{\parhorseshoe{\Qp{\variable{x}}}{\Gp{\variable{y}}}}{\pardisjunction{\negation{\Qp{\variable{x}}}}{\Gp{\variable{y}}}}$, the universal closure of $\bpartriplebar{\parhorseshoe{\Qp{\variable{x}}}{\Gp{\variable{y}}}}{\pardisjunction{\negation{\Qp{\variable{x}}}}{\Gp{\variable{y}}}}$, is $\universal{\variable{x}}\universal{\variable{y}}\bpartriplebar{\parhorseshoe{\Qp{\variable{x}}}{\Gp{\variable{y}}}}{\pardisjunction{\negation{\Qp{\variable{x}}}}{\Gp{\variable{y}}}}$.

We can now define provable equivalence for \GQL{} formulas using the universal closure:
\begin{majorILnc}{\LnpDC{GQL Provably Equivalent}}
Two \GQL{} formulas $\CAPTHETA$ and $\CAPPHI$ are \nidf{provably equivalent}\index{provably equivalent!formulas of \GQL{}|textbf} \Iff the universal closure of the formula that has a biconditional as main connective and $\CAPTHETA$ and $\CAPPHI$ as immediate constituents is derivable in \GQD{}; in other words, \Iff $\sststile{}{}\forall\bpartriplebar{\CAPTHETA}{\CAPPHI}$ in \GQD{}. 
\end{majorILnc}
\noindent{}It's important to note that this definition really is a generalization of definition \mvref{GSDprovablyequivalent}.
If $\CAPTHETA$ and $\CAPPHI$ are sentences of \GSL{} (recall that every sentence of \GSL{} is also a formula of \GQL{}), then they are provably equivalent on definition \ref{GSDprovablyequivalent} \Iff they are provably equivalent on definition \ref{GQL Provably Equivalent}.

Before moving to the proof of the Replacement Theorem for \GQD{} (Thm. \pncmvref{GQD Replacement Theorem}), it is convenient to prove the following one-step Replacement Lemmas:
\begin{THEOREM}{\LnpTC{OneStepReplacementLemmas} One-step Replacement Lemmas:}
If $\sststile{}{}\forall\partriplebar{\CAPPHI}{\CAPPHI^*}$, then:
\begin{cenumerate}
\item $\sststile{}{}\forall\partriplebar{\negation{\CAPPHI}}{\negation{\CAPPHI^*}}$
\item\label{exampleonesteplemma}
$\sststile{}{}\forall\partriplebar{\parconjunction{\CAPPHI}{\conjunction{\CAPPHI_1}{\conjunction{\ldots}{\CAPPHI_{\integer{p}}}}}}{\parconjunction{\CAPPHI^*}{\conjunction{\CAPPHI_1}{\conjunction{\ldots}{\CAPPHI_{\integer{p}}}}}}$
\item[] \hspace{1in} $\vdots$
\item $\sststile{}{}\forall\partriplebar{\parconjunction{\CAPPHI_1}{\conjunction{\ldots}{\conjunction{\CAPPHI_{\integer{p}}}{\CAPPHI}}}}{\parconjunction{\CAPPHI_1}{\conjunction{\ldots}{\conjunction{\CAPPHI_{\integer{p}}}{\CAPPHI^*}}}}$
\item 
$\sststile{}{}\forall\partriplebar{\pardisjunction{\CAPPHI}{\disjunction{\CAPPHI_1}{\disjunction{\ldots}{\CAPPHI_{\integer{p}}}}}}{\pardisjunction{\CAPPHI^*}{\disjunction{\CAPPHI_1}{\disjunction{\ldots}{\CAPPHI_{\integer{p}}}}}}$
\item[] \hspace{1in} $\vdots$
\item $\sststile{}{}\forall\partriplebar{\pardisjunction{\CAPPHI_1}{\disjunction{\ldots}{\disjunction{\CAPPHI_{\integer{p}}}{\CAPPHI}}}}{\pardisjunction{\CAPPHI_1}{\disjunction{\ldots}{\disjunction{\CAPPHI_{\integer{p}}}{\CAPPHI^*}}}}$
\item $\sststile{}{}\forall\bpartriplebar{\parhorseshoe{\CAPPHI}{\CAPPSI}}{\parhorseshoe{\CAPPHI^*}{\CAPPSI}}$
\item $\sststile{}{}\forall\bpartriplebar{\parhorseshoe{\CAPPSI}{\CAPPHI}}{\parhorseshoe{\CAPPSI}{\CAPPHI^*}}$
\item $\sststile{}{}\forall\bpartriplebar{\partriplebar{\CAPPHI}{\CAPPSI}}{\partriplebar{\CAPPHI^*}{\CAPPSI}}$
\item $\sststile{}{}\forall\bpartriplebar{\partriplebar{\CAPPSI}{\CAPPHI}}{\partriplebar{\CAPPSI}{\CAPPHI^*}}$
\end{cenumerate}
And, if $\sststile{}{}\forall\universal{\BETA}\bpartriplebar{\CAPPHI}{\CAPPHI^*}$, then:
\begin{enumerate}[label=(\arabic*), leftmargin=1.85\parindent,
labelindent=.35\parindent, labelsep=*, itemsep=0pt, start=10]%
\item $\sststile{}{}\forall\bpartriplebar{\universal{\BETA}\CAPPHI}{\universal{\BETA}\CAPPHI^*}$
\item $\sststile{}{}\forall\bpartriplebar{\existential{\BETA}\CAPPHI}{\existential{\BETA}\CAPPHI^*}$
\end{enumerate}
\end{THEOREM}
\noindent{}We prove \ref{exampleonesteplemma} for the case of a 2-place conjunction and leave the rest to the reader to prove in a similar way.
We use the following notation. 
If $\CAPPHI$ is a \GQL{} formula, then let $\variable{x}_1,\ldots,\variable{x}_{\integer{m}}$ be the complete list of free variables in $\CAPPHI$. 
Further, let $\CAPPHI\constant{c_{\integer{1}}}\ldots\constant{c_{\integer{\integer{m}}}}/\variable{x}_1\ldots\variable{x}_{\integer{m}}$ be the formula you get by substituting $\constant{c_1}$ for $\variable{x}_1$, $\ldots$, and $\constant{c_{\integer{m}}}$ for $\variable{x}_{\integer{m}}$.
\begin{PROOFOF}{Thm. \ref{OneStepReplacementLemmas}, \ref{exampleonesteplemma}, for 2-place Conjunctions}
Assume that $\sststile{}{}\forall\partriplebar{\CAPPHI}{\CAPPHI^*}$. 
Then consider some derivation $\Derivation{D}$ in \GQD{} of $\forall\partriplebar{\CAPPHI}{\CAPPHI^*}$.
The basic idea is to extend this derivation to a derivation of $\forall\partriplebar{\parconjunction{\CAPPHI}{\CAPPSI}}{\parconjunction{\CAPPHI^*}{\CAPPSI}}$ by first stripping away the initial quantifiers, then manipulating the truth functional connectives, and, finally, restoring the quantifiers. In detail, the new extended derivation should go (to save space when numbering lines, let $\integer{q}=\integer{n}+\integer{m}$):
\begin{gproofnn}
\glinend{ }{$\qquad\vdots$}{ }
\gline{$\integer{n}$}{$\forall\partriplebar{\CAPPHI}{\CAPPHI^*}$}{last line of $\Derivation{D}$}
\gline{$\integer{n}+1$}{$\forall[\partriplebar{\CAPPHI}{\CAPPHI^*}\constant{c}_1/\variable{x}_1]$}{\Rule{$\forall$-Elim}, $\integer{n}$}
\glinend{ }{$\qquad\vdots$}{ }
\gline{$\integer{n}+\integer{m}$}{$\partriplebar{\CAPPHI}{\CAPPHI^*}\constant{c_{\integer{1}}}\ldots\constant{c_{\integer{\integer{m}}}}/\variable{x}_1\ldots\variable{x}_{\integer{m}}$}{\Rule{$\forall$-Elim}, $\integer{n}+\integer{m}-1$}
\gaproof{
\galine{$\integer{q}+1$}{$\parconjunction{\CAPPHI}{\CAPPSI}\constant{c_{\integer{1}}}\ldots\constant{c_{\integer{\integer{m}}}}/\variable{x}_1\ldots\variable{x}_{\integer{m}}$}{\Rule{Assume}}
\galine{$\integer{q}+2$}{$\CAPPHI\constant{c_{\integer{1}}}\ldots\constant{c_{\integer{\integer{m}}}}/\variable{x}_1\ldots\variable{x}_{\integer{m}}$}{\Rule{$\WEDGE$-Elim}, $\integer{q}+1$}
\galine{$\integer{q}+3$}{$\CAPPHI^*\constant{c_{\integer{1}}}\ldots\constant{c_{\integer{\integer{m}}}}/\variable{x}_1\ldots\variable{x}_{\integer{m}}$}{\Rule{$\TRIPLEBAR$-Elim}, $\integer{q}$, $\integer{q}+2$}
\galine{$\integer{q}+4$}{$\CAPPSI\constant{c_{\integer{1}}}\ldots\constant{c_{\integer{\integer{m}}}}/\variable{x}_1\ldots\variable{x}_{\integer{m}}$}{\Rule{$\WEDGE$-Elim}, $\integer{q}+1$}
\galine{$\integer{q}+5$}{$\parconjunction{\CAPPHI^*}{\CAPPSI}\constant{c_{\integer{1}}}\ldots\constant{c_{\integer{\integer{m}}}}/\variable{x}_1\ldots\variable{x}_{\integer{m}}$}{\Rule{$\WEDGE$-Intro}, $\integer{q}+3$, $\integer{q}+4$}
}
\gline{$\integer{q}+6$}{${\parconjunction{\CAPPHI}{\CAPPSI}\constant{c_{\integer{1}}}\ldots\constant{c_{\integer{\integer{m}}}}/\variable{x}_1\ldots\variable{x}_{\integer{m}}}\HORSESHOE$}{ }
\glinend{}{$\qquad{\parconjunction{\CAPPHI^*}{\CAPPSI}\constant{c_{\integer{1}}}\ldots\constant{c_{\integer{\integer{m}}}}/\variable{x}_1\ldots\variable{x}_{\integer{m}}}$}{\Rule{$\HORSESHOE$-Intro}, $\integer{q}+1$--$\integer{q}+5$}

\gaproof{
\galine{$\integer{q}+7$}{$\parconjunction{\CAPPHI^*}{\CAPPSI}\constant{c_{\integer{1}}}\ldots\constant{c_{\integer{\integer{m}}}}/\variable{x}_1\ldots\variable{x}_{\integer{m}}$}{\Rule{Assume}}
\galine{$\integer{q}+8$}{$\CAPPHI^*\constant{c_{\integer{1}}}\ldots\constant{c_{\integer{\integer{m}}}}/\variable{x}_1\ldots\variable{x}_{\integer{m}}$}{\Rule{$\WEDGE$-Elim}, $\integer{q}+7$}
\galine{$\integer{q}+9$}{$\CAPPSI\constant{c_{\integer{1}}}\ldots\constant{c_{\integer{\integer{m}}}}/\variable{x}_1\ldots\variable{x}_{\integer{m}}$}{\Rule{$\WEDGE$-Elim}, $\integer{q}+7$}
\galine{$\integer{q}+10$}{$\CAPPHI\constant{c_{\integer{1}}}\ldots\constant{c_{\integer{\integer{m}}}}/\variable{x}_1\ldots\variable{x}_{\integer{m}}$}{\Rule{$\TRIPLEBAR$-Elim}, $\integer{q}$, $\integer{q}+8$}
\galine{$\integer{q}+11$}{$\parconjunction{\CAPPHI}{\CAPPSI}\constant{c_{\integer{1}}}\ldots\constant{c_{\integer{\integer{m}}}}/\variable{x}_1\ldots\variable{x}_{\integer{m}}$}{\Rule{$\WEDGE$-Intro}, $\integer{q}+9$, $\integer{q}+10$}
}

\gline{$\integer{q}+12$}{${\parconjunction{\CAPPHI^*}{\CAPPSI}\constant{c_{\integer{1}}}\ldots\constant{c_{\integer{\integer{m}}}}/\variable{x}_1\ldots\variable{x}_{\integer{m}}}\HORSESHOE$}{ }
\glinend{}{$\qquad{\parconjunction{\CAPPHI}{\CAPPSI}\constant{c_{\integer{1}}}\ldots\constant{c_{\integer{\integer{m}}}}/\variable{x}_1\ldots\variable{x}_{\integer{m}}}$}{\Rule{$\HORSESHOE$-Intro}, $\integer{q}+7$--$\integer{q}+11$}

\gline{$\integer{q}+13$}{$[{\parconjunction{\CAPPHI}{\CAPPSI}}\TRIPLEBAR$}{ }
\glinend{}{$\qquad{\parconjunction{\CAPPHI^*}{\CAPPSI}]\constant{c_{\integer{1}}}\ldots\constant{c_{\integer{\integer{m}}}}/\variable{x}_1\ldots\variable{x}_{\integer{m}}}$}{\Rule{$\TRIPLEBAR$-Intro}, $\integer{q}+6$, $\integer{q}+12$}

\gline{$\integer{q}+14$}{$\forall[{\parconjunction{\CAPPHI}{\CAPPSI}}\TRIPLEBAR$}{ }
\glinend{}{$\qquad{\parconjunction{\CAPPHI^*}{\CAPPSI}]\constant{c_{\integer{1}}}\ldots\constant{c_{\integer{\integer{m}-1}}}/\variable{x}_1\ldots\variable{x}_{\integer{m}-1}}$}{\Rule{$\forall$-Intro}, $\integer{q}+13$}

\glinend{ }{$\qquad\vdots$}{ }

\gline{$\integer{n}+2\integer{m}$}{$\forall\bpartriplebar{\parconjunction{\CAPPHI}{\CAPPSI}}{\parconjunction{\CAPPHI^*}{\CAPPSI}}$}{\Rule{$\forall$-Intro}, $\integer{n}+2\integer{m}-13$}

\end{gproofnn}
\noindent{}It is important to note that all the constants introduced on lines $\integer{n}+1$ through $\integer{n}+\integer{m}$ need to be new constants that do not appear in any previous lines. 
If not, then there's no guarantee that we be able to do \Rule{$\forall$-Intro} on the end lines. 
\end{PROOFOF}

\begin{PROOFOF}{Thm. \ref{GQD Replacement Theorem}}
Just as with the proof of the Restricted Replacement Theorem for \GSD{} (Thm. \pmvref{ExchangeRuleTheorem}), the proof for the Replacement Theorem for \GQD{} is a recursive proof. But, since the definition of provably equivalent is different (we've extended it to formulas of \GQL{}) we can't simply extend the proof of theorem \ref{ExchangeRuleTheorem} by adding new cases to the inheritance step for the quantifiers. 

Assume that $\CAPPHI$ and $\CAPPHI^*$ are provably equivalent formulas of \GQL{} (assume that $\sststile{}{}\forall\partriplebar{\CAPPHI}{\CAPPHI^*}$), that $\CAPPHI$ is a subformula of $\CAPTHETA$, and that $\CAPTHETA^*$ is the result of replacing $\CAPPHI$ with $\CAPPHI^*$ in $\CAPTHETA$.
\begin{description}
\item[Base Step:]
Similar to the base step in the proof of theorem \ref{ExchangeRuleTheorem}, in the base case $\CAPTHETA$ is atomic and so has no subformula other than itself.
So, if $\CAPPHI$ is a subformula of $\CAPTHETA$, then $\CAPPHI=\CAPTHETA$. Hence $\CAPTHETA^*=\CAPPHI^*$. Since $\CAPPHI$ and $\CAPPHI^*$ are provably equivalent, it follows immediately that $\CAPTHETA$ and $\CAPTHETA^*$ are provably equivalent. 

\item[Inheritance Step:] \hfill 

\begin{description}
\item[Recursive Assumption:] 
Assume that the theorem holds for formulas $\CAPPSI$, $\CAPPSI_1$, $\ldots$, $\CAPPSI_{\integer{k}}$; that is, assume that if $\CAPPSI^*$ is the result of replacing $\CAPPHI$ with $\CAPPHI^*$, then $\sststile{}{}\forall\partriplebar{\CAPPSI}{\CAPPSI^*}$, and similarly for the others.

\item[Negation:]
Assume that $\CAPTHETA=\;\negation{\CAPPSI}$.
Either $\CAPPHI=\CAPTHETA$, in which case it trivially follows that $\sststile{}{}\forall\partriplebar{\CAPTHETA}{\CAPTHETA^*}$, or $\CAPPHI$ is a subformula of $\CAPPSI$ (and hence $\CAPTHETA^*=\;\negation{\CAPPSI^*}$).
By the recursive assumption, $\sststile{}{}\forall\partriplebar{\CAPPSI}{\CAPPSI^*}$.
It follows by the One-step Replacement Lemma (Thm. \ref{OneStepReplacementLemmas}) that $\sststile{}{}\forall\partriplebar{\negation{\CAPPSI}}{\negation{\CAPPSI^*}}$. 

\item[Conjunction:]
Assume that $\CAPTHETA=\conjunction{\CAPPSI_1}{\conjunction{\ldots}{\CAPPSI_{\integer{k}}}}$.
Either $\CAPPHI=\CAPTHETA$, in which case it trivially follows that $\sststile{}{}\forall\partriplebar{\CAPTHETA}{\CAPTHETA^*}$, or $\CAPPHI$ is a subformula of one of the conjuncts $\CAPPSI_{\integer{i}}$ (and hence $\CAPTHETA^*=\conjunction{\CAPPSI_1}{\conjunction{\ldots}{\conjunction{\CAPPSI_{\integer{i}}^*}{\conjunction{\ldots}{\CAPPSI_{\integer{k}}}}}}$).
By the recursive assumption, $\sststile{}{}\forall\partriplebar{\CAPPSI_{\integer{i}}}{\CAPPSI_{\integer{i}}^*}$.
It follows by the One-step Replacement Lemma (Thm. \ref{OneStepReplacementLemmas}) that $\sststile{}{}\partriplebar{\parconjunction{\CAPPSI_1}{\conjunction{\ldots}{\conjunction{\CAPPSI_{\integer{i}}}{\conjunction{\ldots}{\CAPPSI_{\integer{k}}}}}}}{\parconjunction{\CAPPSI_1}{\conjunction{\ldots}{\conjunction{\CAPPSI_{\integer{i}}^*}{\conjunction{\ldots}{\CAPPSI_{\integer{k}}}}}}}$.

\item[Disjunction:]
This case is left to the reader.

\item[Conditional]
This case is also left to the reader.

\item[Biconditional:]
This case is also left to the reader.

\item[Universal:]
Assume that $\CAPTHETA=\universal{\BETA}\CAPPSI$. 
Either $\CAPPHI=\CAPTHETA$, in which case it trivially follows that $\sststile{}{}\forall\partriplebar{\CAPTHETA}{\CAPTHETA^*}$, or $\CAPPHI$ is a subformula of $\CAPPSI$ (and hence $\CAPTHETA^*=\universal{\BETA}\CAPPSI^*$).
By the recursive assumption, $\sststile{}{}\forall\partriplebar{\CAPPSI}{\CAPPSI^*}$.
To derive $\forall\partriplebar{\universal{\BETA}\CAPPSI}{\universal{\BETA}\CAPPSI^*}$, start with the derivation of $\forall\partriplebar{\CAPPSI}{\CAPPSI^*}$. 
Extend it by adding as many steps of \Rule{$\forall$-Elim} as needed to get to the sentence $\partriplebar{\CAPPSI}{\CAPPSI^*}\constant{c_{\integer{1}}}\ldots\constant{c_{\integer{\integer{m}}}}/\variable{x}_1\ldots\variable{x}_{\integer{m}}$.
Then using \Rule{$\forall$-Intro} first on $\BETA$, then on the others we can get $\forall\universal{\BETA}\partriplebar{\CAPPSI}{\CAPPSI^*}$ on the line. 
Hence $\sststile{}{}\forall\universal{\BETA}\partriplebar{\CAPPSI}{\CAPPSI^*}$,
so by the One-step Replacement Lemma (Thm. \ref{OneStepReplacementLemmas}) we get that $\sststile{}{}\forall\bpartriplebar{\universal{\BETA}\CAPPSI}{\universal{\BETA}\CAPPSI^*}$.

\item[Existential:] This case is exactly the same, except that a different result from the One-step Replacement Lemma is used.

\end{description}

\item[Closure Step:]
Since the inheritance step covers all the ways to generate \GQL{} formulas, we've shown that the theorem holds for all \GQL{} formulas $\CAPTHETA$.
\end{description}
\end{PROOFOF}
\begin{PROOFOF}{Thm. \ref{GQD Shortcut Theorem2}, Part (ii)}
Since any two formulas $\CAPPHI$ and $\CAPPHI^*$ got by substituting \GQL{} formulas into the may-add and given schemas of the exchange shortcut rules from \GQDP{} (tables \ref{GSDplus2} and \ref{GQDplus}) are provably equivalent, it follows from the Replacement Theorem for \GQD{} (Thm. \pncmvref{GQD Replacement Theorem}) that if $\CAPTHETA^*$ is a sentence sanctioned by an exchange rule applied to some sentence $\CAPTHETA$, then $\CAPTHETA$ and $\CAPTHETA^*$ are provably equivalent. That is, $\sststile{}{}\forall\partriplebar{\CAPTHETA}{\CAPTHETA^*}$. Since $\CAPTHETA$ and $\CAPTHETA^*$ are sentences, the universal closure of their biconditional $\triplebar{\CAPTHETA}{\CAPTHETA^*}$ is just the biconditional itself, so  $\sststile{}{}\partriplebar{\CAPTHETA}{\CAPTHETA^*}$.
But since  $\sststile{}{}\partriplebar{\CAPTHETA}{\CAPTHETA^*}$, it should be clear that $\CAPTHETA\sststile{}{}\CAPTHETA^*$. 
Thus, any application of an exchange rule from \GQDP{} is derivable using the basic rules of \GQD{} alone. 
\end{PROOFOF}


%%%%%%%%%%%%%%%%%%%%%%%%%%%%%%%%%%%%%%%%%%%%%%%%%%
\section{Exercises}
%%%%%%%%%%%%%%%%%%%%%%%%%%%%%%%%%%%%%%%%%%%%%%%%%%

%\notocsubsection{Section Review Exercises}{Section Review Exercises}
%\begin{enumerate}
%\item Rewrite derivation \ref{cangetlong} using \Rule{A.C.}
%\item Definition \mvref{GSDprovablyequivalent} assumes/claims that: both $\CAPPHI\sststile{}{}\CAPPSI$ and $\CAPPSI\sststile{}{}\CAPPHI$ \Iff $\sststile{}{}\triplebar{\CAPPHI}{\CAPPSI}$. Prove that this is true.
%\item Give the needed arguments for conditionals and biconditionals in the inheritance step of the proof for theorem \mvref{ExchangeRuleTheorem}. 
%\end{enumerate}

\notocsubsection{\GSD{} Practice Problems}{GSD Practice Problems} 
Write derivations for each of the following using only the rules specified by the instructor. 
It is probably a good idea to do the problems in order, as the earlier ones tend to be easier than the later ones. 
\begin{multicols}{2}
\begin{enumerate}
\item $\sststile{}{}\horseshoe{\Al}{\parhorseshoe{\Bl}{\Al}}$
\item $\sststile{}{}\horseshoe{\parhorseshoe{\Al}{\Bl}}{\bparhorseshoe{\parhorseshoe{\Bl}{\Cl}}{\parhorseshoe{\Al}{\Cl}}}$
\item $\sststile{}{}\horseshoe{\parhorseshoe{\Al}{\Bl}}{\bparhorseshoe{\pardisjunction{\Al}{\Cl}}{\pardisjunction{\Bl}{\disjunction{\Cl}{\Dl}}}}$
\item $\sststile{}{}\horseshoe{\parconjunction{\conjunction{\bparhorseshoe{\Al}{\Bl}}{\Cl}}{\bpartriplebar{\Cl}{\Al}}}{\Bl}$
\item $\sststile{}{}\disjunction{\negation{\Al}}{\parhorseshoe{\Bl}{\Al}}$
\item $\sststile{}{}\disjunction{\parhorseshoe{\Al}{\Bl}}{\parhorseshoe{\Bl}{\Cl}}$
\end{enumerate}
\end{multicols}
\begin{enumerate}[start=7]
\item $\sststile{}{}\horseshoe{\cparconjunction{\negation{\Cl}}{\bpardisjunction{\parhorseshoe{\Al}{\Cl}}{\parhorseshoe{\Bl}{\Cl}}}}{\negation{\parconjunction{\Al}{\Bl}}}$
\end{enumerate}

\notocsubsection{\GSD{} Shortcut Rules}{exercisesGSDshortcutrules} 
Finish the proof of theorem \mvref{GSD Shortcut Theorem2}. 
To do this, write derivation schemas for each of the following using only the basic rules of \GSD{} or shortcut rules for which you've already done a derivation schema. 
Note that this is also sufficient to show that sentences got by substituting into the given and may-add schemas of the \GSD{} exchange shortcut rules are provably equivalent (Def. \pmvref{GSDprovablyequivalent}).
Note that a star $^*$ has been placed next to the ones that have already been done in the text above. 
(They are left in the list for completeness.) 
Hint: derivations \ref{helpful1} and \ref{bycontradiction} should be helpful in doing two of the problems below.
\begin{multicols}{2}
\begin{description}
\item[M.T.]\hfill{}
\begin{enumerate}
\item $\horseshoe{\CAPPHI}{\CAPTHETA},\negation{\CAPTHETA}\sststile{}{}\;\negation{\CAPPHI}$
\end{enumerate}
\item[D.S.]\hfill{}
\begin{enumerate}[start=2]
\item $\disjunction{\CAPPHI}{\CAPTHETA}, \negation{\CAPPHI}\sststile{}{}\CAPTHETA$
\item $\disjunction{\negation{\CAPPHI}}{\CAPTHETA},\CAPPHI\sststile{}{}\CAPTHETA$
\end{enumerate}
\item[A.C.]\hfill{}
\begin{enumerate}[start=4]
\item $\CAPPHI,\negation{\CAPPHI}\sststile{}{}\CAPPSI$ $^*$ (derivation \ref{anycontradictionSC})
\end{enumerate}
\item[$\NEGATION/\TRIPLEBAR$-Intro]\hfill{}
\begin{enumerate}[start=5]
\item $\triplebar{\CAPPHI}{\CAPPSI}\sststile{}{}\triplebar{\negation{\CAPPHI}}{\negation{\CAPPSI}}$
\end{enumerate}
\item[DeM]\hfill{}
\begin{enumerate}[start=6]
\item ${\negation{\parconjunction{\CAPPHI}{\CAPTHETA}}}\sststile{}{}{\pardisjunction{\negation{\CAPPHI}}{\negation{\CAPTHETA}}}$
\item ${\disjunction{\negation{\CAPPHI}}{\negation{\CAPTHETA}}}\sststile{}{}{\negation{\parconjunction{\CAPPHI}{\CAPTHETA}}}$
\item ${\negation{\pardisjunction{\CAPPHI}{\CAPTHETA}}}\sststile{}{}{\parconjunction{\negation{\CAPPHI}}{\negation{\CAPTHETA}}}$ 
\item ${\conjunction{\negation{\CAPPHI}}{\negation{\CAPTHETA}}}\sststile{}{}{\negation{\pardisjunction{\CAPPHI}{\CAPTHETA}}}$ $^*$ (derivation \ref{DeMDerivationSchema})
\end{enumerate}
\item[$\NEGATION\NEGATION$-Elim]\hfill{}
\begin{enumerate}[start=10]
\item $\negation{\negation{\CAPPHI}}\sststile{}{}\CAPPHI$
\end{enumerate}
\item[$\NEGATION\NEGATION$-Intro]\hfill{}
\begin{enumerate}[start=11]
\item $\CAPPHI\sststile{}{}\;\negation{\negation{\CAPPHI}}$
\end{enumerate}
\item[$\HORSESHOE/\:\VEE$-Exchange]\hfill{}
\begin{enumerate}[start=12]
\item ${\horseshoe{\CAPPHI}{\CAPTHETA}}\sststile{}{}{\disjunction{\negation{\CAPPHI}}{\CAPTHETA}}$
\item ${\disjunction{\negation{\CAPPHI}}{\CAPTHETA}}\sststile{}{}{\horseshoe{\CAPPHI}{\CAPTHETA}}$
\end{enumerate}
\item[Contraposition]\hfill{}
\begin{enumerate}[start=14]
\item $\horseshoe{\CAPPHI}{\CAPTHETA}\sststile{}{}\horseshoe{\negation{\CAPTHETA}}{\negation{\CAPPHI}}$
\item $\horseshoe{\negation{\CAPTHETA}}{\negation{\CAPPHI}}\sststile{}{}\horseshoe{\CAPPHI}{\CAPTHETA}$
\end{enumerate}
\item[$\NEGATION/\HORSESHOE$-Exchange]\hfill{}
\begin{enumerate}[start=16]
\item ${\negation{\parhorseshoe{\CAPPHI}{\CAPTHETA}}}\sststile{}{}{\conjunction{\CAPPHI}{\negation{\CAPTHETA}}}$
\item ${\conjunction{\CAPPHI}{\negation{\CAPTHETA}}}\sststile{}{}{\negation{\parhorseshoe{\CAPPHI}{\CAPTHETA}}}$
\end{enumerate}
\end{description}
\end{multicols}
\begin{description} 
\item[Distribution]\hfill{}
\begin{enumerate}[start=18]
\item $\conjunction{\CAPTHETA}{\pardisjunction{\CAPPHI_1}{\CAPPHI_2}}\sststile{}{}\disjunction{\parconjunction{\CAPTHETA}{\CAPPHI_1}}{\parconjunction{\CAPTHETA}{\CAPPHI_2}}$
\item $\disjunction{\parconjunction{\CAPTHETA}{\CAPPHI_1}}{\parconjunction{\CAPTHETA}{\CAPPHI_2}}\sststile{}{}\conjunction{\CAPTHETA}{\pardisjunction{\CAPPHI_1}{\CAPPHI_2}}$

\item $\conjunction{\pardisjunction{\CAPPHI_1}{\CAPPHI_2}}{\CAPTHETA}\sststile{}{}\disjunction{\parconjunction{\CAPPHI_1}{\CAPTHETA}}{\parconjunction{\CAPPHI_2}{\CAPTHETA}}$
\item $\disjunction{\parconjunction{\CAPPHI_1}{\CAPTHETA}}{\parconjunction{\CAPPHI_2}{\CAPTHETA}}\sststile{}{}\conjunction{\pardisjunction{\CAPPHI_1}{\CAPPHI_2}}{\CAPTHETA}$

\item $\disjunction{\CAPTHETA}{\parconjunction{\CAPPHI_1}{\CAPPHI_2}}\sststile{}{}\conjunction{\pardisjunction{\CAPTHETA}{\CAPPHI_1}}{\pardisjunction{\CAPTHETA}{\CAPPHI_2}}$
\item $\conjunction{\pardisjunction{\CAPTHETA}{\CAPPHI_1}}{\pardisjunction{\CAPTHETA}{\CAPPHI_2}}\sststile{}{}\disjunction{\CAPTHETA}{\parconjunction{\CAPPHI_1}{\CAPPHI_2}}$

\item $\disjunction{\parconjunction{\CAPPHI_1}{\CAPPHI_2}}{\CAPTHETA}\sststile{}{}\conjunction{\pardisjunction{\CAPPHI_1}{\CAPTHETA}}{\pardisjunction{\CAPPHI_2}{\CAPTHETA}}$
\item $\conjunction{\pardisjunction{\CAPPHI_1}{\CAPTHETA}}{\pardisjunction{\CAPPHI_2}{\CAPTHETA}}\sststile{}{}\disjunction{\parconjunction{\CAPPHI_1}{\CAPPHI_2}}{\CAPTHETA}$

\item $\triplebar{\CAPTHETA}{\CAPPSI}\sststile{}{}\disjunction{\parconjunction{\CAPTHETA}{\CAPPSI}}{\parconjunction{\negation{\CAPTHETA}}{\negation{\CAPPSI}}}$

\item $\disjunction{\parconjunction{\CAPTHETA}{\CAPPSI}}{\parconjunction{\negation{\CAPTHETA}}{\negation{\CAPPSI}}}\sststile{}{}\triplebar{\CAPTHETA}{\CAPPSI}$
\end{enumerate}
\end{description} 

%\notocsubsection{Provably Equivalence of GSD Exchange Shortcut Rules}{exercisesGSDPEshortcutrules} 
%Show that any sentences got by substituting into the given and may-add schemas of the \GSD{} exchange shortcut rules are provably equivalent (Def. \pmvref{GSDprovablyequivalent}).
%That is, find derivation schemas showing that each sentence schema in the following pairs is derivable from the other.
%\begin{enumerate}

%\item $\negation{\parconjunction{\CAPPHI_1}{\CAPPHI_2}}$, $\disjunction{\negation{\CAPPHI_1}}{\negation{\CAPPHI_2}}$

% %%\item $\disjunction{\negation{\CAPPHI_1}}{\disjunction{\ldots}{\negation{\CAPPHI_{\integer{n}}}}}$, $\negation{\parconjunction{\CAPPHI_1}{\conjunction{\ldots}{\CAPPHI_{\integer{n}}}}}$
 
%\item $\negation{\pardisjunction{\CAPPHI_1}{\CAPPHI_2}}$, $\conjunction{\negation{\CAPPHI_1}}{\negation{\CAPPHI_2}}$ 
 
% %%\item $\conjunction{\negation{\CAPPHI_1}}{\conjunction{\ldots}{\negation{\CAPPHI_{\integer{n}}}}}$, $\negation{\pardisjunction{\CAPPHI_1}{\disjunction{\ldots}{\CAPPHI_{\integer{n}}}}}$ 
 
%\item $\negation{\negation{\CAPPHI}}$, $\CAPPHI$

% %%\item $\CAPPHI$, $\negation{\negation{\CAPPHI}}$ 

%\item $\horseshoe{\CAPPHI}{\CAPTHETA}$, $\disjunction{\negation{\CAPPHI}}{\CAPTHETA}$ 

% %%\item $\disjunction{\negation{\CAPPHI}}{\CAPTHETA}$, $\horseshoe{\CAPPHI}{\CAPTHETA}$
 
%\item $\horseshoe{\CAPPHI}{\CAPTHETA}$, $\horseshoe{\negation{\CAPTHETA}}{\negation{\CAPPHI}}$ 

% %%\item $\horseshoe{\negation{\CAPTHETA}}{\negation{\CAPPHI}}$, $\horseshoe{\CAPPHI}{\CAPTHETA}$ 
 
%\item $\negation{\parhorseshoe{\CAPPHI}{\CAPTHETA}}$, $\conjunction{\CAPPHI}{\negation{\CAPTHETA}}$

% %%\item $\conjunction{\CAPPHI}{\negation{\CAPTHETA}}$, $\negation{\parhorseshoe{\CAPPHI}{\CAPTHETA}}$
 
%\item $\conjunction{\CAPTHETA}{\pardisjunction{\CAPPHI_1}{\CAPPHI_2}}$, $\disjunction{\parconjunction{\CAPTHETA}{\CAPPHI_1}}{\parconjunction{\CAPTHETA}{\CAPPHI_2}}$

% %%\item $\disjunction{\parconjunction{\CAPTHETA}{\CAPPHI_1}}{\disjunction{\ldots}{\parconjunction{\CAPTHETA}{\CAPPHI_{\integer{n}}}}}$, $\conjunction{\CAPTHETA}{\pardisjunction{\CAPPHI_1}{\disjunction{\ldots}{\CAPPHI_{\integer{n}}}}}$
 

%\item $\conjunction{\pardisjunction{\CAPPHI_1}{\CAPPHI_2}}{\CAPTHETA}$, $\disjunction{\parconjunction{\CAPPHI_1}{\CAPTHETA}}{\parconjunction{\CAPPHI_2}{\CAPTHETA}}$
 
% %%\item $\disjunction{\parconjunction{\CAPPHI_1}{\CAPTHETA}}{\disjunction{\ldots}{\parconjunction{\CAPPHI_{\integer{n}}}{\CAPTHETA}}}$, $\conjunction{\pardisjunction{\CAPPHI_1}{\disjunction{\ldots}{\CAPPHI_{\integer{n}}}}}{\CAPTHETA}$
 
 
%\item $\disjunction{\CAPTHETA}{\parconjunction{\CAPPHI_1}{\CAPPHI_2}}$, $\conjunction{\pardisjunction{\CAPTHETA}{\CAPPHI_1}}{\pardisjunction{\CAPTHETA}{\CAPPHI_2}}$
 
% %%\item $\conjunction{\pardisjunction{\CAPTHETA}{\CAPPHI_1}}{\conjunction{\ldots}{\pardisjunction{\CAPTHETA}{\CAPPHI_{\integer{n}}}}}$, $\disjunction{\CAPTHETA}{\parconjunction{\CAPPHI_1}{\conjunction{\ldots}{\CAPPHI_{\integer{n}}}}}$

%\item $\disjunction{\parconjunction{\CAPPHI_1}{\CAPPHI_2}}{\CAPTHETA}$, $\conjunction{\pardisjunction{\CAPPHI_1}{\CAPTHETA}}{\pardisjunction{\CAPPHI_2}{\CAPTHETA}}$

% %%\item $\conjunction{\pardisjunction{\CAPPHI_1}{\CAPTHETA}}{\conjunction{\ldots}{\pardisjunction{\CAPPHI_{\integer{n}}}{\CAPTHETA}}}$, $\disjunction{\parconjunction{\CAPPHI_1}{\conjunction{\ldots}{\CAPPHI_{\integer{n}}}}}{\CAPTHETA}$

%\item $\triplebar{\CAPTHETA}{\CAPPSI}$, $\disjunction{\parconjunction{\CAPTHETA}{\CAPPSI}}{\parconjunction{\negation{\CAPTHETA}}{\negation{\CAPPSI}}}$

%\end{enumerate}

\notocsubsection{\GQD{} Shortcut Rules}{exer:GQDSCprovablyequiv}
Prove that any two formulas got by substituting \GQL{} formulas into the may-add and given schemas of the exchange shortcut rules for \GQD{} are provably equivalent (Def. \pmvref{GQL Provably Equivalent}).
That is, show that the following hold for all \GQL{} formulas $\CAPPHI,\CAPPHI_1,\CAPPHI_2,\CAPTHETA,\CAPPSI$ by writing the appropriate derivation schemas.
Note that all but (7) and (8) deal with exchange shortcut rules from \GSDP{}.
For these virtually all the work has been done in exercise \ref{exercisesGSDshortcutrules}; all you need to do is show how to put the derivation schemas done there together and how to remove and put back on the quantifiers needed to make the universal closure.
\begin{multicols}{2}
\begin{enumerate}
\item $\sststile{}{}\forall[\negation{\parconjunction{\CAPPHI_1}{\CAPPHI_2}}\TRIPLEBAR\pardisjunction{\negation{\CAPPHI_1}}{\negation{\CAPPHI_2}}]$

% %%\item $\disjunction{\negation{\CAPPHI_1}}{\disjunction{\ldots}{\negation{\CAPPHI_{\integer{n}}}}}$, $\negation{\parconjunction{\CAPPHI_1}{\conjunction{\ldots}{\CAPPHI_{\integer{n}}}}}$
 
\item $\sststile{}{}\forall[\negation{\pardisjunction{\CAPPHI_1}{\CAPPHI_2}}\TRIPLEBAR\parconjunction{\negation{\CAPPHI_1}}{\negation{\CAPPHI_2}}]$ 
 
% %%\item $\conjunction{\negation{\CAPPHI_1}}{\conjunction{\ldots}{\negation{\CAPPHI_{\integer{n}}}}}$, $\negation{\pardisjunction{\CAPPHI_1}{\disjunction{\ldots}{\CAPPHI_{\integer{n}}}}}$ 
 
\item $\sststile{}{}\forall[\negation{\negation{\CAPPHI}}\TRIPLEBAR\CAPPHI]$

% %%\item $\CAPPHI$, $\negation{\negation{\CAPPHI}}$ 

\item $\sststile{}{}\forall[\parhorseshoe{\CAPPHI}{\CAPTHETA}\TRIPLEBAR\pardisjunction{\negation{\CAPPHI}}{\CAPTHETA}]$ 

% %%\item $\disjunction{\negation{\CAPPHI}}{\CAPTHETA}$, $\horseshoe{\CAPPHI}{\CAPTHETA}$
 
\item $\sststile{}{}\forall[\parhorseshoe{\CAPPHI}{\CAPTHETA}\TRIPLEBAR\parhorseshoe{\negation{\CAPTHETA}}{\negation{\CAPPHI}}]$ 

% %%\item $\horseshoe{\negation{\CAPTHETA}}{\negation{\CAPPHI}}$, $\horseshoe{\CAPPHI}{\CAPTHETA}$ 
 
\item $\sststile{}{}\forall[\negation{\parhorseshoe{\CAPPHI}{\CAPTHETA}}\TRIPLEBAR\parconjunction{\CAPPHI}{\negation{\CAPTHETA}}]$

% %%\item $\conjunction{\CAPPHI}{\negation{\CAPTHETA}}$, $\negation{\parhorseshoe{\CAPPHI}{\CAPTHETA}}$

\item $\sststile{}{}\forall[\negation{\universal{\BETA}{\CAPPHI}}\TRIPLEBAR\existential{\BETA}\negation{{\CAPPHI}}]$

\item $\sststile{}{}\forall[\negation{\existential{\BETA}{\CAPPHI}}\TRIPLEBAR\universal{\BETA}\negation{{\CAPPHI}}]$

\end{enumerate}
\end{multicols}
\begin{enumerate}[start=9]
\item $\sststile{}{}\forall[\parconjunction{\CAPTHETA}{\pardisjunction{\CAPPHI_1}{\CAPPHI_2}}\TRIPLEBAR\pardisjunction{\parconjunction{\CAPTHETA}{\CAPPHI_1}}{\parconjunction{\CAPTHETA}{\CAPPHI_2}}]$

% %%\item $\disjunction{\parconjunction{\CAPTHETA}{\CAPPHI_1}}{\disjunction{\ldots}{\parconjunction{\CAPTHETA}{\CAPPHI_{\integer{n}}}}}$, $\conjunction{\CAPTHETA}{\pardisjunction{\CAPPHI_1}{\disjunction{\ldots}{\CAPPHI_{\integer{n}}}}}$

\item $\sststile{}{}\forall[\parconjunction{\pardisjunction{\CAPPHI_1}{\CAPPHI_2}}{\CAPTHETA}\TRIPLEBAR\pardisjunction{\parconjunction{\CAPPHI_1}{\CAPTHETA}}{\parconjunction{\CAPPHI_2}{\CAPTHETA}}]$
 
% %%\item $\disjunction{\parconjunction{\CAPPHI_1}{\CAPTHETA}}{\disjunction{\ldots}{\parconjunction{\CAPPHI_{\integer{n}}}{\CAPTHETA}}}$, $\conjunction{\pardisjunction{\CAPPHI_1}{\disjunction{\ldots}{\CAPPHI_{\integer{n}}}}}{\CAPTHETA}$
 
\item $\sststile{}{}\forall[\pardisjunction{\CAPTHETA}{\parconjunction{\CAPPHI_1}{\CAPPHI_2}}\TRIPLEBAR\parconjunction{\pardisjunction{\CAPTHETA}{\CAPPHI_1}}{\pardisjunction{\CAPTHETA}{\CAPPHI_2}}]$
 
% %%\item $\conjunction{\pardisjunction{\CAPTHETA}{\CAPPHI_1}}{\conjunction{\ldots}{\pardisjunction{\CAPTHETA}{\CAPPHI_{\integer{n}}}}}$, $\disjunction{\CAPTHETA}{\parconjunction{\CAPPHI_1}{\conjunction{\ldots}{\CAPPHI_{\integer{n}}}}}$

\item $\sststile{}{}\forall[\pardisjunction{\parconjunction{\CAPPHI_1}{\CAPPHI_2}}{\CAPTHETA}\TRIPLEBAR\parconjunction{\pardisjunction{\CAPPHI_1}{\CAPTHETA}}{\pardisjunction{\CAPPHI_2}{\CAPTHETA}}]$

% %%\item $\conjunction{\pardisjunction{\CAPPHI_1}{\CAPTHETA}}{\conjunction{\ldots}{\pardisjunction{\CAPPHI_{\integer{n}}}{\CAPTHETA}}}$, $\disjunction{\parconjunction{\CAPPHI_1}{\conjunction{\ldots}{\CAPPHI_{\integer{n}}}}}{\CAPTHETA}$

\item $\sststile{}{}\forall[\partriplebar{\CAPTHETA}{\CAPPSI}\TRIPLEBAR\pardisjunction{\parconjunction{\CAPTHETA}{\CAPPSI}}{\parconjunction{\negation{\CAPTHETA}}{\negation{\CAPPSI}}}]$
\end{enumerate}

\notocsubsection{\GQD{} Practice Problems}{GQD Practice Problems} 
Write derivations for each of the following using only the rules specified by the instructor. 
It is probably a good idea to do the problems in order, as the earlier ones tend to be easier than the later ones. 
\begin{multicols}{2}
\begin{enumerate}
\item $\sststile{}{}\horseshoe{\universal{\variable{x}}\universal{\variable{y}}\Qpp{\variable{x}}{\variable{y}}}{\universal{\variable{z}}\Qpp{\variable{z}}{\variable{z}}}$
\item $\sststile{}{}\horseshoe{\universal{\variable{x}}\universal{\variable{y}}\Qpp{\variable{x}}{\variable{y}}}{\universal{\variable{x}}\universal{\variable{y}}\Qpp{\variable{y}}{\variable{x}}}$
\item $\sststile{}{}\horseshoe{\universal{\variable{x}}\parconjunction{\Qp{\variable{x}}}{\Gp{\variable{x}}}}{\bparconjunction{\universal{\variable{x}}\Qp{\variable{x}}}{\universal{\variable{x}}\Gp{\variable{x}}}}$
\item $\sststile{}{}\horseshoe{\bparconjunction{\universal{\variable{x}}\Qp{\variable{x}}}{\universal{\variable{x}}\Gp{\variable{x}}}}{\universal{\variable{x}}\parconjunction{\Qp{\variable{x}}}{\Gp{\variable{x}}}}$
\item $\sststile{}{}\horseshoe{\bpardisjunction{\universal{\variable{x}}\Qp{\variable{x}}}{\universal{\variable{x}}\Gp{\variable{x}}}}{\universal{\variable{x}}\pardisjunction{\Qp{\variable{x}}}{\Gp{\variable{x}}}}$
\item $\sststile{}{}\horseshoe{\universal{\variable{x}}\parhorseshoe{\Qp{\variable{x}}}{\Gp{\variable{x}}}}{\bparhorseshoe{\universal{\variable{x}}\Qp{\variable{x}}}{\universal{\variable{x}}\Gp{\variable{x}}}}$
\item $\sststile{}{}\horseshoe{\universal{\variable{x}}\parconjunction{\Pl}{\Qp{\variable{x}}}}{\parconjunction{\Pl}{\universal{\variable{x}}\Qp{\variable{x}}}}$
\item $\sststile{}{}\horseshoe{\parconjunction{\Pl}{\universal{\variable{x}}\Qp{\variable{x}}}}{\universal{\variable{x}}\parconjunction{\Pl}{\Qp{\variable{x}}}}$

\item $\sststile{}{}\horseshoe{\universal{\variable{x}}\pardisjunction{\Pl}{\Qp{\variable{x}}}}{\pardisjunction{\Pl}{\universal{\variable{x}}\Qp{\variable{x}}}}$
\item $\sststile{}{}\horseshoe{\pardisjunction{\Pl}{\universal{\variable{x}}\Qp{\variable{x}}}}{\universal{\variable{x}}\pardisjunction{\Pl}{\Qp{\variable{x}}}}$

\item $\sststile{}{}\horseshoe{\universal{\variable{x}}\parhorseshoe{\Pl}{\Qp{\variable{x}}}}{\parhorseshoe{\Pl}{\universal{\variable{x}}\Qp{\variable{x}}}}$
\item $\sststile{}{}\horseshoe{\parhorseshoe{\Pl}{\universal{\variable{x}}\Qp{\variable{x}}}}{\universal{\variable{x}}\parhorseshoe{\Pl}{\Qp{\variable{x}}}}$

\item $\sststile{}{}\horseshoe{\existential{\variable{x}}\universal{\variable{y}}\Qpp{\variable{x}}{\variable{y}}}{\universal{\variable{y}}\existential{\variable{x}}\Qpp{\variable{x}}{\variable{y}}}$

\item $\sststile{}{}\horseshoe{\universal{\variable{x}}\parhorseshoe{\Qp{\variable{x}}}{\Gp{\variable{x}}}}{\bparhorseshoe{\existential{\variable{x}}\Qp{\variable{x}}}{\existential{\variable{x}}\Gp{\variable{x}}}}$

\item $\sststile{}{}\horseshoe{\existential{\variable{x}}\parconjunction{\Qp{\variable{x}}}{\Gp{\variable{x}}}}{\bparconjunction{\existential{\variable{x}}\Qp{\variable{x}}}{\existential{\variable{x}}\Gp{\variable{x}}}}$

\item $\sststile{}{}\horseshoe{\bpardisjunction{\existential{\variable{x}}\Qp{\variable{x}}}{\existential{\variable{x}}\Gp{\variable{x}}}}{\existential{\variable{x}}\pardisjunction{\Qp{\variable{x}}}{\Gp{\variable{x}}}}$

\item $\sststile{}{}\horseshoe{\existential{\variable{x}}\parconjunction{\Pl}{\Qp{\variable{x}}}}{\parconjunction{\Pl}{\existential{\variable{x}}\Qp{\variable{x}}}}$
\item $\sststile{}{}\horseshoe{\parconjunction{\Pl}{\existential{\variable{x}}\Qp{\variable{x}}}}{\existential{\variable{x}}\parconjunction{\Pl}{\Qp{\variable{x}}}}$

\item $\sststile{}{}\horseshoe{\existential{\variable{x}}\pardisjunction{\Pl}{\Qp{\variable{x}}}}{\pardisjunction{\Pl}{\existential{\variable{x}}\Qp{\variable{x}}}}$
\item $\sststile{}{}\horseshoe{\pardisjunction{\Pl}{\existential{\variable{x}}\Qp{\variable{x}}}}{\existential{\variable{x}}\pardisjunction{\Pl}{\Qp{\variable{x}}}}$

\item $\sststile{}{}\horseshoe{\existential{\variable{x}}\parhorseshoe{\Pl}{\Qp{\variable{x}}}}{\parhorseshoe{\Pl}{\existential{\variable{x}}\Qp{\variable{x}}}}$
\item $\sststile{}{}\horseshoe{\parhorseshoe{\Pl}{\existential{\variable{x}}\Qp{\variable{x}}}}{\existential{\variable{x}}\parhorseshoe{\Pl}{\Qp{\variable{x}}}}$

\item $\sststile{}{}\horseshoe{\universal{\variable{x}}\parhorseshoe{\Qp{\variable{x}}}{\Pl}}{\bparhorseshoe{\existential{\variable{x}}\Qp{\variable{x}}}{\Pl}}$
\item $\sststile{}{}\horseshoe{\bparhorseshoe{\existential{\variable{x}}\Qp{\variable{x}}}{\Pl}}{\universal{\variable{x}}\parhorseshoe{\Qp{\variable{x}}}{\Pl}}$
\end{enumerate}
\end{multicols}
\begin{enumerate}[start=25]
\item $\sststile{}{}\horseshoe{\universal{\variable{x}}\existential{\variable{y}}\parconjunction{\Qp{\variable{x}}}{\Gp{\variable{y}}}}{\bparconjunction{\universal{\variable{x}}\Qp{\variable{x}}}{\existential{\variable{y}}\Gp{\variable{y}}}}$

\item $\sststile{}{}\horseshoe{\bparconjunction{\universal{\variable{x}}\Qp{\variable{x}}}{\existential{\variable{y}}\Gp{\variable{y}}}}{\universal{\variable{x}}\existential{\variable{y}}\parconjunction{\Qp{\variable{x}}}{\Gp{\variable{y}}}}$

\item $\sststile{}{}\horseshoe{\universal{\variable{x}}\existential{\variable{y}}\parconjunction{\Qp{\variable{x}}}{\Gp{\variable{y}}}}{\existential{\variable{y}}\universal{\variable{x}}\parconjunction{\Qp{\variable{x}}}{\Gp{\variable{y}}}}$

\item $\sststile{}{}\horseshoe{\universal{\variable{x}}\existential{\variable{y}}\pardisjunction{\Qp{\variable{x}}}{\Gp{\variable{y}}}}{\bpardisjunction{\universal{\variable{x}}\Qp{\variable{x}}}{\existential{\variable{y}}\Gp{\variable{y}}}}$

\item $\sststile{}{}\horseshoe{\bpardisjunction{\existential{\variable{y}}\Gp{\variable{y}}}{\universal{\variable{x}}\Qp{\variable{x}}}}{\universal{\variable{x}}\existential{\variable{y}}\pardisjunction{\Qp{\variable{x}}}{\Gp{\variable{y}}}}$

\item $\sststile{}{}\horseshoe{\existential{\variable{y}}\universal{\variable{x}}\pardisjunction{\Qp{\variable{x}}}{\Gp{\variable{y}}}}{\universal{\variable{x}}\existential{\variable{y}}\pardisjunction{\Qp{\variable{x}}}{\Gp{\variable{y}}}}$

\item $\sststile{}{}\horseshoe{\existential{\variable{y}}\universal{\variable{x}}\parhorseshoe{\Qp{\variable{x}}}{\Gp{\variable{y}}}}{\universal{\variable{x}}\existential{\variable{y}}\parhorseshoe{\Qp{\variable{x}}}{\Gp{\variable{y}}}}$

\item $\sststile{}{}\horseshoe{\universal{\variable{x}}\existential{\variable{y}}\pardisjunction{\Qp{\variable{x}}}{\Gp{\variable{y}}}}{\existential{\variable{y}}\universal{\variable{x}}\pardisjunction{\Qp{\variable{x}}}{\Gp{\variable{y}}}}$

\item $\sststile{}{}\horseshoe{\bparhorseshoe{\existential{\variable{y}}\Qp{\variable{y}}}{\existential{\variable{x}}\Gp{\variable{x}}}}{\existential{\variable{y}}\universal{\variable{x}}\parhorseshoe{\Qp{\variable{x}}}{\Gp{\variable{y}}}}$

\item $\sststile{}{}\horseshoe{\existential{\variable{y}}\universal{\variable{x}}\parhorseshoe{\Qp{\variable{x}}}{\Gp{\variable{y}}}}{\bparhorseshoe{\existential{\variable{x}}\Qp{\variable{x}}}{\existential{\variable{x}}\Gp{\variable{x}}}}$

\item 
$\sststile{}{}\horseshoe{\universal{\variable{x}}\existential{\variable{y}}\parhorseshoe{\Qp{\variable{x}}}{\Gp{\variable{y}}}}{\existential{\variable{y}}\universal{\variable{x}}\parhorseshoe{\Qp{\variable{x}}}{\Gp{\variable{y}}}}$

\item $\sststile{}{}\horseshoe{\existential{\variable{x}}\existential{\variable{y}}\parconjunction{\Qp{\variable{x}}}{\negation{\Qp{\variable{y}}}}}{\bparconjunction{\existential{\variable{x}}\Qp{\variable{x}}}{\existential{\variable{x}}\negation{\Qp{\variable{x}}}}}$

\item $\sststile{}{}\horseshoe{\existential{\variable{x}}\universal{\variable{y}}\parhorseshoe{\Qp{\variable{x}}}{\Gp{\variable{y}}}}{\bparhorseshoe{\universal{\variable{x}}\Qp{\variable{x}}}{\universal{\variable{x}}\Gp{\variable{x}}}}$

\item $\sststile{}{}\horseshoe{\bparconjunction{\existential{\variable{x}}\Qp{\variable{x}}}{\existential{\variable{x}}\negation{\Qp{\variable{x}}}}}{\existential{\variable{x}}\existential{\variable{y}}\parconjunction{\Qp{\variable{x}}}{\negation{\Qp{\variable{y}}}}}$

\item $\sststile{}{}\horseshoe{\bparhorseshoe{\universal{\variable{x}}\Qp{\variable{x}}}{\universal{\variable{x}}\Gp{\variable{x}}}}{\existential{\variable{x}}\universal{\variable{y}}\parhorseshoe{\Qp{\variable{x}}}{\Gp{\variable{y}}}}$

\item $\sststile{}{}\horseshoe{\bpardisjunction{\negation{\existential{\variable{x}}\Qp{\variable{x}}}}{\universal{\variable{x}}\Qp{\variable{x}}}}{\universal{\variable{x}}\universal{\variable{y}}\parhorseshoe{\Qp{\variable{x}}}{\Qp{\variable{y}}}}$
\end{enumerate}

%\theendnotes


%%%%%%%%%%%%%%%%%%%%%%%%%%%%%%%%%%%%%%%%%%%%%%%%%%
\chapter{Soundness and Completeness}\label{completenesschapter}
%%%%%%%%%%%%%%%%%%%%%%%%%%%%%%%%%%%%%%%%%%%%%%%%%%

%%%%%%%%%%%%%%%%%%%%%%%%%%%%%%%%%%%%%%%%%%%%%%%%%%
\section{Introduction}
%%%%%%%%%%%%%%%%%%%%%%%%%%%%%%%%%%%%%%%%%%%%%%%%%%

In the previous chapter, we stipulated restrictions for the rule applications of \GSD{} (and \GQD{}) so that the rules would be \emph{truth-preserving}.

\begin{majorILnc}{\LnpDC{Derivation Rule Soundness}}
	A rule is \df{truth-preserving}\index{derivation!rule!truth-preserving}\index{truth-preserving} \Iff the sentence or sentences to which the rule is applied entail any sentence which the rule  sanctions you to write as the next step. 
\end{majorILnc}


\begin{majorILnc}{\LnpDC{RuleSchemas}}
	A \nidf{formal derivation rule}\index{derivation!rule|textbf} is a sequence of sentence schemas, the first through the second last of which is called the \df{given schemas} and the last is called the \df{may-add schema}. 
\end{majorILnc}
\noindent{}As the names suggest, table \mvref{GSD} lists rules by putting the first through second last schemas in the left column, labeled ``Given'', and the last schema in the right column, labeled ``May Add''. 
We only bring out that we can think of rules as sequences of sentence schemas, and call them the given schemas and the may-add schema so the next definition is easier to state.
\begin{majorILnc}{\LnpDC{RuleSanctioning}}
	A rule \Rule{R}, applied to unboxed lines $\integer{m}_1,\ldots,\integer{m}_{\integer{j}}$ with, respectively, sentences $\CAPPSI_1,\ldots,\CAPPSI_{\integer{j}}$, \df{sanctions} writing the sentence $\CAPPHI$ \Iff there's some substitution of \GSL{} sentences that, for the given schemas of \Rule{R}, results in $\CAPPSI_1,\ldots,\CAPPSI_{\integer{j}}$ and, for the may-add schema, results in $\CAPPHI$. 
\end{majorILnc}
\noindent{}As an example, consider again derivation \pmvref{secondexamplefinished}. 
The rule \Rule{$\HORSESHOE$-Elim} was applied to line 2, which had sentence $\horseshoe{\Bl}{\parconjunction{\Cl}{\Dl}}$, and line 3, which had sentence $\Bl$, to get line 4, which had sentence $\conjunction{\Cl}{\Dl}$. 
Using definition \ref{RuleSanctioning} we can show that this move is sanctioned by \Rule{$\HORSESHOE$-Elim} by noting, from table \mvref{GSD}, that \Rule{$\HORSESHOE$-Elim} has two given schemas, $\horseshoe{\CAPTHETA}{\CAPPSI}$ and $\CAPTHETA$, and the may-add scheme $\CAPPSI$. 
Substituting $\CAPTHETA=\Bl$ and $\CAPPSI=\conjunction{\Cl}{\Dl}$ in the given schemas gets us lines 2 and 3, while making this same substitution in the may-add schema gets us line 4. 

Lastly, we end with the following theorem:
\begin{THEOREM}{\LnpTC{Soundess of Basic GSD Rules}}
	Every application of every basic rule of \GSD{} is truth-preserving.
\end{THEOREM}
\begin{PROOF}
	It can be shown that: for any basic rule \Rule{R} of \GSD{}, if some substitution of \GSL{} sentences into the given schema of \Rule{R} results in \GSL{} sentences $\CAPPSI_1,\ldots,\CAPPSI_{\integer{n}}$ and that same substitution into the may-add schema of \Rule{R} results in the \GSL{} sentence $\CAPPHI$, then $\CAPPSI_1,\ldots,\CAPPSI_{\integer{n}}\sdtstile{}{}\CAPPHI$.
	Call this the truth-preservation lemma.
	(We asked the reader to show that the lemma is true in exercises \pmvref{exercises:truth-preservation lemma}.)
	Now consider some arbitrary application of some basic rule \Rule{R} of \GSD{}. 
	Say that in this application \Rule{R} is applied to sentences $\CAPTHETA_1,\ldots,\CAPTHETA_{\integer{m}}$ and permits, or sanctions, you to write down $\DELTA$. 
	By definition \mvref{RuleSanctioning}, there's some substitution of \GSL{} sentences that, for the given schemas of \Rule{R}, results in $\CAPTHETA_1,\ldots,\CAPTHETA_{\integer{m}}$ and, for the may-add schema, results in $\DELTA$. 
	By the truth preservation lemma, $\CAPTHETA_1,\ldots,\CAPTHETA_{\integer{m}}\sdtstile{}{}\DELTA$.
	By definition \mvref{Derivation Rule Soundness}, this application is truth-preserving. 
	
	This proof doesn't cover \Rule{$\HORSESHOE$-Intro} or \Rule{Assume}.  We will have to give these rules special treatment.  The former is the only rule that eliminates an assumption, and the latter is the only rule that adds an assumption, so they each have a special role.
\end{PROOF}

Recall from section \mvref{Derivation Preliminaries} that we want to use derivations as a way of showing that a sentence is a logical truth, or of showing that some set of sentences entails some other sentence.
Specifically, we want to use derivations in \GSD{} and \GQD{} to show that sentences of \GSL{} and \GQL{} are \CAPS{tft} and \CAPS{qt}, or to show entailments between sentences of \GSL{} or between sentences of \GQL{}.  
But derivations can only fill this role if our derivation systems are sound.  Gnerally speaking, a derivation system is only totally satisfactory if it is also complete.
Let \Language{L} be some formal language for which we have defined some kind of models.
\begin{majorILnc}{\LnpDC{LSoundness}}
A derivation system \DerivationSystem{D} for \Language{L} is \nidf{sound}\index{soundness|textbf} \Iff for every set $\Delta$ of sentences of \Language{L} and every sentence $\CAPPHI$ of \Language{L}, if $\Delta\sststile{}{}\CAPPHI$, then $\Delta\sdtstile{}{}\CAPPHI$.
\end{majorILnc} 


%%%%%%%%%%%%%%%%%%%%%%%%%%%%%%%%%%%%%%%%%%%%%%%%%%
\section{Soundness}
%%%%%%%%%%%%%%%%%%%%%%%%%%%%%%%%%%%%%%%%%%%%%%%%%%

\subsection{Soundness of \GSD{}}
We begin by proving the soundness of \GSD{}.\index{soundness!of \GSD{}}
\begin{THEOREM}{\LnpTC{Soundness of Sentential Logic} \GSD{} Soundness Theorem:}
\GSD{} is sound; i.e., for every set $\Delta$ of sentences of \GSL{} and every sentence $\CAPPHI$ of \GSL{}, if $\Delta\sststile{}{}\CAPPHI$ in \GSD{}, then $\Delta\sdtstile{}{}\CAPPHI$.
\end{THEOREM}
\noindent{}To prove that \GSD{} is sound we will first prove the following result about derivations:
\begin{THEOREM}{\LnpTC{Main GSL Soundness Lemma} Soundness Lemma:}
For any sequence of derivation lines that is a derivation (see definition \pmvref{Recursive definition of Derivation}), the sentence $\CAPPHI$ on the last line is entailed by the set $\Delta$ of sentences that are on unboxed lines and are sanctioned by \Rule{Assumption}. 
\end{THEOREM}
\noindent{}Since the definition of a derivation (def. \pmvref{Recursive definition of Derivation}) is a recursive definition, the most natural way to prove theorem \ref{Main GSL Soundness Lemma} is through a recursive proof. 
The recursive proof we give will use three easily proved lemmas (proofs are left to the reader; the first lemma, on monotonicity, is also used to prove thm. \ref{Soundness of Sentential Logic}). 
Two are facts about entailment and one is about derivation.
\begin{THEOREM}{\LnpTC{Monotonicity of Entailment} Monotonicity of Entailment:}
For all \GSL{} sentences $\CAPPHI_1,\ldots,\CAPPHI_{\integer{n}},\CAPTHETA,\CAPPSI$:
\begin{center}
If $\CAPPHI_1,\CAPPHI_2,\ldots,\CAPPHI_{\integer{n}}\sdtstile{}{}\CAPPSI$, then $\CAPPHI_1,\CAPPHI_2,\ldots,\CAPPHI_{\integer{n}},\CAPTHETA\sdtstile{}{}\CAPPSI$
\end{center}
\end{THEOREM}
\begin{THEOREM}{\LnpTC{Transitivity of Entailment} Transitivity of Entailment:}
For all \GSL{} sentences $\CAPPHI_1,\ldots,\CAPPHI_{\integer{n}}$, $\CAPTHETA$, and $\CAPPSI_1,\ldots,\CAPPSI_{\integer{k}}$:
\begin{center}
\begin{tabular}{ l@{\hspace{.25em}}l@{\hspace{.25em}}l }
If & $\CAPPHI_1,\CAPPHI_2,\ldots,\CAPPHI_{\integer{n}}\sdtstile{}{}\CAPPSI_1$ & and \\
   & $\CAPPHI_1,\CAPPHI_2,\ldots,\CAPPHI_{\integer{n}}\sdtstile{}{}\CAPPSI_2$ & and \\
   & \hspace{.5in} $\vdots$ &  \\
   & $\CAPPHI_1,\CAPPHI_2,\ldots,\CAPPHI_{\integer{n}}\sdtstile{}{}\CAPPSI_{\integer{k}}$ & and \\
   & $\CAPPSI_1,\CAPPSI_2,\ldots,\CAPPSI_{\integer{k}}\sdtstile{}{}\CAPTHETA$ & then: \\
   & & $\CAPPHI_1,\CAPPHI_2,\ldots,\CAPPHI_{\integer{n}}\sdtstile{}{}\CAPTHETA$   \\
\end{tabular}
\end{center}
\end{THEOREM}
\begin{THEOREM}{\LnpTC{Non-decreasing Assumption Principle} Non-decreasing Assumption Principle (NDAP):}
If $\Delta_1$ is the set of assumptions of an unboxed line and $\Delta_2$ is the set of assumptions of a later unboxed line, then $\Delta_1$ is a subset of $\Delta_2$, i.e., $\Delta_1\subseteq\Delta_2$.
\end{THEOREM}
\begin{PROOFOF}{Thm. \ref{Main GSL Soundness Lemma}, Soundness Lemma}
\begin{description}

\item[Base Step:] 
The base case is a single-line derivation $\Derivation{D}$ sanctioned by the rule \Rule{Assumption}. 
Say the sentence on that line is $\CAPPHI$.
We have to show that the sentence on the last line is entailed by all the sentences, on unboxed lines, that are sanctioned by \Rule{Assumption}. 
But in this case the sentence on the last line is $\CAPPHI$, and the set of unboxed sentences sanctioned by \Rule{Assumption} only contains $\CAPPHI$. 
Obviously $\CAPPHI\sdtstile{}{}\CAPPHI$, so the theorem holds in the base case. 

\item[Inheritance Step:] 
In the inheritance step we start with a derivation $\Derivation{D}$.
Say $\Delta$ is the set of unboxed assumptions occurring in $\Derivation{D}$, and $\Delta_\integer{i}$ is the set of unboxed assumptions occurring in $\Derivation{D}$ up to (and including) line number $\integer{i}$. 
%We then want to show that if some rule \Rule{R} of \GSD{} applied to unboxed lines of $\Derivation{D}$ sanctions writing down sentence $\CAPPHI$, then $Delta^*\sdtstile{}{}\CAPPHI$, where $\Delta^*$ is the set of unboxed assumptions for the new line with $\CAPPHI$.\footnote{Note 
We then want to show that if we add another line to $\Derivation{D}$ with sentence $\CAPPHI$ sanctioned by rule \Rule{R}, then $\Delta^*\sdtstile{}{}\CAPPHI$, where $\Delta^*$ is the set of unboxed assumptions for the new line. 
(Notice that this requires more than the fact that the rules are truth preserving; we also have to attend to how we define a derivation. The fact that the rules are truth preserving is essential of course.) 
We need to consider each rule \Rule{R} of \GSD{} as its own case.

\begin{description}

\item[Recursive Assumption:]  
The recursive assumption is that for all lines $\Derivation{L}_\integer{i}$ in the derivation $\Derivation{D}$, if $\CAPPHI$ is the sentence on the line, then $\Delta_{\integer{i}}\sdtstile{}{}\CAPPHI$. 

\item[\Rule{Assumption}:] 
Say we add another line to $\Derivation{D}$ with sentence $\CAPPHI$ sanctioned by \Rule{Assumption}. 
Note that the set $\Delta^*$ of unboxed assumptions for this new line are those in $\Delta$ plus $\CAPPHI$. 
We know $\CAPPHI\sdtstile{}{}\CAPPHI$, and $\Delta,\CAPPHI\sdtstile{}{}\CAPPHI$ follows from this by monotonicity.

\item[\Rule{Repetition}:] 
Say $\CAPPHI$ already occurs somewhere in $\Derivation{D}$, say on line number $\integer{i}$. 
Then $\Delta_{\integer{i}}\sdtstile{}{}\CAPPHI$ by the recursive assumption. 
Suppose we add another line to $\Derivation{D}$ with $\CAPPHI$ sanctioned by \Rule{Repetition}. 
By NDAP, $\Delta_{\integer{i}}\subseteq\Delta^*$. 
So by monotonicity, $\Delta^*\sdtstile{}{}\CAPPHI$.

\item[\Rule{$\VEE$-Intro}:]
Assume we add another line to $\Derivation{D}$ with sentence $\CAPPHI$ sanctioned by \Rule{$\VEE$-Intro}. 
Then there's some earlier line $\integer{i}$ with the sentence $\CAPTHETA$ and $\CAPPHI$ is a disjunction with $\CAPTHETA$ as one disjunct. 
We have that $\Delta_{\integer{i}}$ is the set of unboxed assumptions of line $\integer{i}$, and by NDAP $\Delta_{\integer{i}}\subseteq\Delta^*$.
By the recursive assumption, $\Delta_{\integer{i}}\sdtstile{}{}\CAPTHETA$.
So by monotonicity, $\Delta^*\sdtstile{}{}\CAPTHETA$.
Because the rule is truth preserving we know that $\CAPPHI$ is a disjunction with $\CAPTHETA$ as one disjunct, $\CAPTHETA\sdtstile{}{}\CAPPHI$. 
So by transitivity, $\Delta^*\sdtstile{}{}\CAPPHI$. 

\item[\Rule{$\WEDGE\!$-Elim}:]
Say we add another line to $\Derivation{D}$ with sentence $\CAPPHI$ sanctioned by \Rule{$\WEDGE\!$-Elim}.
Then there's some earlier line $\integer{i}$ with the sentence $\conjunction{\CAPTHETA_\integer{1}}{\conjunction{\ldots}{\CAPTHETA_{\integer{m}}}}$ and $\CAPPHI$ is one of the conjuncts. 
As before, by NDAP $\Delta_{\integer{i}}\subseteq\Delta^*$.
By the recursive assumption, $\Delta_{\integer{i}}\sdtstile{}{}\conjunction{\CAPTHETA_\integer{1}}{\conjunction{\ldots}{\CAPTHETA_{\integer{m}}}}$. 
So by monotonicity, $\Delta^*\sdtstile{}{}\conjunction{\CAPTHETA_\integer{1}}{\conjunction{\ldots}{\CAPTHETA_{\integer{m}}}}$.
And $\CAPPHI$ is one of the conjuncts of $\conjunction{\CAPTHETA_\integer{1}}{\conjunction{\ldots}{\CAPTHETA_{\integer{m}}}}$, so it follows that $\conjunction{\CAPTHETA_\integer{1}}{\conjunction{\ldots}{\CAPTHETA_{\integer{m}}}}\sdtstile{}{}\CAPPHI$. 
So by transitivity, $\Delta^*\sdtstile{}{}\CAPPHI$.

\item[\Rule{$\NEGATION$-Elim}:] 
Suppose we add another line to $\Derivation{D}$ with sentence $\CAPPHI$ sanctioned by \Rule{$\NEGATION$-Elim}.
Then there's some earlier line $\integer{i}$ with the sentence $\horseshoe{\negation{\CAPPHI}}{\parconjunction{\CAPPSI}{\negation{\CAPPSI}}}$.
As before, by NDAP $\Delta_{\integer{i}}\subseteq\Delta^*$.
By the recursive assumption, $\Delta_{\integer{i}}\sdtstile{}{}\horseshoe{\negation{\CAPPHI}}{\parconjunction{\CAPPSI}{\negation{\CAPPSI}}}$.
So by monotonicity, $\Delta^*\sdtstile{}{}\horseshoe{\negation{\CAPPHI}}{\parconjunction{\CAPPSI}{\negation{\CAPPSI}}}$.
Since the \CAPS{rhs} of $\horseshoe{\negation{\CAPPHI}}{\parconjunction{\CAPPSI}{\negation{\CAPPSI}}}$ is false in all models (it's \CAPS{tff}), the conditional is true in a model $\IntA{}$ only if the \CAPS{lhs} is false in $\IntA{}$.
So if the conditional is true in a model $\IntA{}$, $\CAPPHI$ is true in $\IntA{}$.
In other words, $\horseshoe{\negation{\CAPPHI}}{\parconjunction{\CAPPSI}{\negation{\CAPPSI}}}\sdtstile{}{}\CAPPHI$. 
So by transitivity, $\Delta^*\sdtstile{}{}\CAPPHI$.

\item[\Rule{$\NEGATION$-Intro}:] 
This case is very similar to the last and is left to the reader. 

\item[\Rule{$\HORSESHOE$-Elim}:]
Assume we add another line to $\Derivation{D}$ with sentence $\CAPPHI$ sanctioned by \Rule{$\HORSESHOE$-Elim}.
Then there's two earlier lines $\integer{i}$ and $\integer{j}$, and (say) line $\integer{i}$ has a sentence $\horseshoe{\CAPTHETA}{\CAPPHI}$ and line $\integer{j}$ has sentence $\CAPTHETA$. 
By NDAP we have that $\Delta_{\integer{i}}\subseteq\Delta^*$ and $\Delta_{\integer{j}}\subseteq\Delta^*$.
By the recursive assumption, $\Delta_{\integer{i}}\sdtstile{}{}\horseshoe{\CAPTHETA}{\CAPPHI}$ and $\Delta_{\integer{j}}\sdtstile{}{}\CAPTHETA$.
By monotonicity, 
$\Delta^*\sdtstile{}{}\horseshoe{\CAPTHETA}{\CAPPHI}$ and $\Delta^*\sdtstile{}{}\CAPTHETA$.
Because the rule is truth preserving we know that $\CAPTHETA,\horseshoe{\CAPTHETA}{\CAPPHI}\sdtstile{}{}\CAPPHI$.
So by transitivity, $\Delta^*\sdtstile{}{}\CAPPHI$.

\item[\Rule{$\TRIPLEBAR$-Elim}:] The argument for each of the two versions of \Rule{$\TRIPLEBAR$-Elim} is the same as that for \Rule{$\HORSESHOE$-Elim}.

\item[\Rule{$\TRIPLEBAR$-Intro}:]
Say we add another line to $\Derivation{D}$ with sentence $\CAPPHI=\triplebar{\CAPPHI}{\CAPTHETA}$ sanctioned by \Rule{$\TRIPLEBAR$-Intro}.
Then there's two earlier lines $\integer{i}$ and $\integer{j}$, and (say) line $\integer{i}$ has a sentence $\horseshoe{\CAPTHETA}{\CAPPHI}$ and line $\integer{j}$ has sentence $\horseshoe{\CAPPHI}{\CAPTHETA}$. 
By NDAP we have that $\Delta_{\integer{i}}\subseteq\Delta^*$ and $\Delta_{\integer{j}}\subseteq\Delta^*$.
By the recursive assumption, $\Delta_{\integer{i}}\sdtstile{}{}\horseshoe{\CAPTHETA}{\CAPPHI}$ and $\Delta_{\integer{j}}\sdtstile{}{}\horseshoe{\CAPPHI}{\CAPTHETA}$.
By monotonicity, $\Delta^*\sdtstile{}{}\horseshoe{\CAPTHETA}{\CAPPHI}$ and $\Delta^*\sdtstile{}{}\horseshoe{\CAPPHI}{\CAPTHETA}$.
Because the rule is truth preserving we know that $\horseshoe{\CAPPHI}{\CAPTHETA},\horseshoe{\CAPTHETA}{\CAPPHI}\sdtstile{}{}\triplebar{\CAPPHI}{\CAPTHETA}$.
So by transitivity, $\Delta^*\sdtstile{}{}\CAPPHI$.

\item[\Rule{$\WEDGE\!$-Intro}:]
Suppose we add another line to $\Derivation{D}$ with sentence $\CAPPHI=\conjunction{\CAPPHI_1}{\conjunction{\ldots}{\CAPPHI_{\integer{m}}}}$ sanctioned by \Rule{$\WEDGE\!$-Intro}.
Then there are $\integer{m}$ earlier lines numbered $\integer{i}_{1},\ldots,\integer{i}_{\integer{m}}$ with, respectively, sentences $\CAPPHI_1,\ldots,\CAPPHI_{\integer{m}}$. 
By NDAP we have that $\Delta_{\integer{i}_1}\subseteq\Delta^*,\ldots,\Delta_{\integer{i}_\integer{m}}\subseteq\Delta^*$.
By the recursive assumption, $\Delta_{\integer{i}_1}\sdtstile{}{}\CAPPHI_1,\ldots,\Delta_{\integer{i}_\integer{m}}\sdtstile{}{}\CAPPHI_{\integer{m}}$.
So by monotonicity, $\Delta^*\sdtstile{}{}\CAPPHI_1,\ldots,\Delta^*\sdtstile{}{}\CAPPHI_{\integer{m}}$.
We now observe that $\CAPPHI_1,\ldots,\CAPPHI_{\integer{m}}\sdtstile{}{}\conjunction{\CAPPHI}{\conjunction{\ldots}{\CAPPHI_{\integer{m}}}}$.
So by transitivity, $\Delta^*\sdtstile{}{}\CAPPHI$.

\item[\Rule{$\VEE$-Elim}:]
Assume we add another line to $\Derivation{D}$ with sentence $\CAPPHI$ sanctioned by \Rule{$\VEE$-Elim}.
Then there are $\integer{m}+1$ earlier lines numbered $\integer{i}_{1},\ldots,\integer{i}_{\integer{m}},\integer{i}_{\integer{m}+1}$ with, respectively, sentences $\horseshoe{\CAPTHETA_1}{\CAPPHI}$, $\ldots$, $\horseshoe{\CAPTHETA_{\integer{m}}}{\CAPPHI}$, and  $\disjunction{\CAPTHETA_1}{\disjunction{\ldots}{\CAPTHETA_{\integer{m}}}}$.
By NDAP we have that $\Delta_{\integer{i}_1}\subseteq\Delta^*,\ldots,\Delta_{\integer{i}_\integer{m}}\subseteq\Delta^*$ and $\Delta_{\integer{i}_{\integer{m}+1}}\subseteq\Delta^*$.
By the recursive assumption, $\Delta_{\integer{i}_1}\sdtstile{}{}\horseshoe{\CAPTHETA_1}{\CAPPHI}$, $\ldots$, $\Delta_{\integer{i}_\integer{m}}\sdtstile{}{}\horseshoe{\CAPTHETA_{\integer{m}}}{\CAPPHI}$ and $\Delta_{\integer{i}_{\integer{m}+1}}\sdtstile{}{}\disjunction{\CAPTHETA_1}{\disjunction{\ldots}{\CAPTHETA_{\integer{m}}}}$.
By monotonicity, $\Delta^*\sdtstile{}{}\horseshoe{\CAPTHETA_1}{\CAPPHI}$, $\ldots$, $\Delta^*\sdtstile{}{}\horseshoe{\CAPTHETA_{\integer{m}}}{\CAPPHI}$ and $\Delta^*\sdtstile{}{}\disjunction{\CAPTHETA_1}{\disjunction{\ldots}{\CAPTHETA_{\integer{m}}}}$.
We now observe that $\horseshoe{\CAPTHETA_1}{\CAPPHI},\ldots,\horseshoe{\CAPTHETA_{\integer{m}}}{\CAPPHI},\disjunction{\CAPTHETA_1}{\disjunction{\ldots}{\CAPTHETA_{\integer{m}}}}\sdtstile{}{}\CAPPHI$.
So by transitivity, $\Delta^*\sdtstile{}{}\CAPPHI$.

\item[\Rule{$\HORSESHOE$-Intro}:]
Like \Rule{Assumption}, the assumptions change in \Rule{$\HORSESHOE$-Intro}. 
If the new line sanctioned by \Rule{$\HORSESHOE$-Intro} has sentence $\horseshoe{\CAPPHI}{\CAPTHETA}$ and unboxed assumptions $\Delta^*$, then earlier we have an assumption line (now in a box) that starts with $\CAPPHI$ and $\Delta^*$ as its other assumptions, and we have a line (now at the bottom of the box) with $\CAPTHETA$ on it with assumptions $\CAPPHI,\Delta^*$. 
By the recursive assumption we have that $\CAPPHI,\Delta^*\sdtstile{}{}\CAPTHETA$. 
Consider any model $\IntA{}$ that makes $\Delta_{\integer{i}}$ true;
if it also makes $\CAPPHI$ true, then $\CAPTHETA$ is true in $\IntA{}$ as well and so is $\horseshoe{\CAPPHI}{\CAPTHETA}$. 
If $\IntA{}$ makes $\CAPPHI$ false, then $\horseshoe{\CAPPHI}{\CAPTHETA}$ is true. 
(Notice that this step only works because we defined the conditional to be true when the \CAPS{lhs} is false.)
So if $\IntA{}$ makes $\Delta^*$ true, it makes $\horseshoe{\CAPPHI}{\CAPTHETA}$ true too. 
So, $\Delta^*\sdtstile{}{}\horseshoe{\CAPPHI}{\CAPTHETA}$.  

\end{description}
\item[Closure Step:] We have now covered all the generating cases for derivations. By the closure clause of the definition, we have proved soundness for all derivations. 
\end{description}
\end{PROOFOF} 

\begin{PROOFOF}{Thm. \ref{Soundness of Sentential Logic}, SL Soundness Theorem}
Assume that $\Delta$ is a set of \GSL{} sentences. 
Assume $\Delta\sststile{}{}\CAPPHI$ and consider some derivation $\Derivation{D}$ of $\CAPPHI$ from $\Delta$. 
Let $\Delta'$ be the set of sentences in $\Delta$ that appear as unboxed assumptions in $\Derivation{D}$. 
By the soundness lemma (Thm. \pmvref{Main GSL Soundness Lemma}), $\Delta'\sdtstile{}{}\CAPPHI$. 
It follows immediately by monotonicity that $\Delta\sdtstile{}{}\CAPPHI$.  
\end{PROOFOF} 

\subsection{Soundness of \GQD{}}
In this section we prove that \GQD{} is also sound.\index{soundness!of \GQD{}}
\begin{THEOREM}{\LnpTC{Soundness of Quantifier Logic} \GQD{} Soundness Theorem:}
\GQD{} is sound; i.e., for every set $\Delta$ of sentences of \GQL{} and every sentence $\CAPPHI$ of \GQL{}, if $\Delta\sststile{}{}\CAPPHI$ in \GSD{}, then $\Delta\sdtstile{}{}\CAPPHI$.
\end{THEOREM}
\noindent{}The proof given in the last section of the \GSL{} Soundness Theorem (Thm. \pmvref{Soundness of Sentential Logic}) can be carried over to the \GQL{} Soundness Theorem. 
That proof relied on the monotonicity of entailment and the soundness lemma (Them. \pmvref{Main GSL Soundness Lemma}). 
It should be clear that entailment is also monotonic in the case of \GQL{}. 
Since \GQD{} is just an extension of \GSD{} (it's just \GSD{} plus the rules for the quantifiers in table \pncmvref{GQD}), all we need to do to show that the soundness lemma holds for \GQD{} is add a case, for each new rule of \GQD{}, to the inheritance step of the proof of the soundness lemma for \GSD{}.
\begin{PROOFOF}{Thm. \ref{Main GSL Soundness Lemma} for GQD}
\begin{description}

\item[Base Step:] 
The base case has been covered in the proof for \GSD{}. 

\item[Inheritance Step:] 
Just as in the proof for \GSD{}, in the inheritance step we start with a derivation $\Derivation{D}$.
Say $\Delta$ is the set of unboxed assumptions occurring in $\Derivation{D}$, and $\Delta_\integer{i}$ is the set of unboxed assumptions occurring in $\Derivation{D}$ up to (and including) line number $\integer{i}$. 
%We then want to show that if some rule \Rule{R} of \GSD{} applied to unboxed lines of $\Derivation{D}$ sanctions writing down sentence $\CAPPHI$, then $Delta^*\sdtstile{}{}\CAPPHI$, where $\Delta^*$ is the set of unboxed assumptions for the new line with $\CAPPHI$.\footnote{Note 
We then want to show that if we add another line to $\Derivation{D}$ with sentence $\CAPPHI$ sanctioned by rule \Rule{R}, then $\Delta^*\sdtstile{}{}\CAPPHI$, where $\Delta^*$ is the set of unboxed assumptions for the new line. %\footnote{Note 
%that this is not the same as showing that the rule \Rule{R} is truth-preserving (see def. \pncmvref{Derivation Rule Soundness}).
%} 
Again we need to consider each rule \Rule{R} of \GQD{} as its own case.
Most of the rules have already been covered in the proof of \GSD{}, so we only need to cover the introduction and elimination rules for the quantifiers. 

\begin{description}

\item[Recursive Assumption:]  
The recursive assumption, as in the proof for \GSD{}, is that for all lines $\Derivation{L}_\integer{i}$ in the derivation $\Derivation{D}$, if $\CAPPHI$ is the sentence on the line, then $\Delta_{\integer{i}}\sdtstile{}{}\CAPPHI$. 

\item[\Rule{$\forall$-Elim}:]
Say we add another line to $\Derivation{D}$ with sentence $\CAPPHI\variable{s}/\BETA$ sanctioned by \Rule{$\forall$-Elim}.
Then there's some earlier line $\integer{i}$ with the sentence $\universal{\BETA}\CAPPHI$. 
We have that $\Delta_{\integer{i}}$ is the set of unboxed assumptions of line $\integer{i}$, and by NDAP $\Delta_{\integer{i}}\subseteq\Delta^*$.
By the recursive assumption, $\Delta_{\integer{i}}\sdtstile{}{}\universal{\BETA}\CAPPHI$.
So by monotonicity, $\Delta^*\sdtstile{}{}\universal{\BETA}\CAPPHI$.

Assume some model $\IntA$ such that $\universal{\BETA}\CAPPHI$ is true.  By the def. of truth for $\forall$, $\CAPPHI{\variable{t}/\BETA}$ is true on all $\variable{t}$-variants of $\IntA$.  Notice that $\CAPPHI{\variable{t}/\BETA}$ and $\CAPPHI{\variable{s}/\BETA}$ are exactly the same, except that the latter has $\variable{s}$ substituted for $\variable{t}$.  These sentences satisfy condition (1) of Dragnet.

Now let's consider, in particular, the $\variable{t}$-variant that assigns to $\variable{t}$ what $\IntA$ assigns to $\variable{s}$.  Name that $\variable{t}$-variant $\As{\variable{t}}{}$.  The models $\IntA$ and $\As{\variable{t}}{}$ meet Dragnet condition (2).  Thus, by Dragnet, $\CAPPHI{\variable{t}/\BETA}$ is true on $\As{\variable{t}}{}$ iff $\CAPPHI{\variable{s}/\BETA}$ is true on $\IntA$.  Therefore, $\CAPPHI{\variable{s}/\BETA}$ is true on $\IntA$.

Any model such that $\universal{\BETA}\CAPPHI$ is true also makes $\CAPPHI{\variable{s}/\BETA}$ true.  Thus, $\universal{\BETA}\CAPPHI\sdtstile{}{}\CAPPHI\variable{s}/\BETA$.  So by transitivity, $\Delta^*\sdtstile{}{}\CAPPHI\variable{s}/\BETA$.   

\item[\Rule{$\exists$-Intro}:]
Say we add another line to $\Derivation{D}$ with sentence $\existential{\BETA}\CAPPHI$ sanctioned by \Rule{$\exists$-Intro}.
Then there's some earlier line $\integer{i}$ with the sentence $\CAPPHI\variable{s}/\BETA$.
Again we have that $\Delta_{\integer{i}}$ is the set of unboxed assumptions of line $\integer{i}$, and by NDAP $\Delta_{\integer{i}}\subseteq\Delta^*$.
By the recursive assumption, $\Delta_{\integer{i}}\sdtstile{}{}\CAPPHI\variable{s}/\BETA$.
By monotonicity, $\Delta^*\sdtstile{}{}\CAPPHI\variable{s}/\BETA$.

Assume some model $\IntA$ such that $\existential{\BETA}\CAPPHI$ is false.  By the def. of truth for $\exists$, there is no $\variable{t}$-variant of $\IntA$ that makes $\CAPPHI{\variable{t}/\BETA}$ true.  Notice that $\CAPPHI{\variable{t}/\BETA}$ and $\CAPPHI{\variable{s}/\BETA}$ are exactly the same, except that the latter has $\variable{s}$ substituted for $\variable{t}$.  These sentences satisfy condition (1) of Dragnet.

Now let's consider, in particular, the $\variable{t}$-variant that assigns to $\variable{t}$ what $\IntA$ assigns to $\variable{s}$.  Name that $\variable{t}$-variant $\As{\variable{t}}{}$.  The models $\IntA$ and $\As{\variable{t}}{}$ meet Dragnet condition (2).  Thus, by Dragnet, $\CAPPHI{\variable{t}/\BETA}$ is true on $\As{\variable{t}}{}$ iff $\CAPPHI{\variable{s}/\BETA}$ is true on $\IntA$.  Therefore, $\CAPPHI{\variable{s}/\BETA}$ is false on $\IntA$.

Any model that makes $\existential{\BETA}\CAPPHI$ false also makes $\CAPPHI{\variable{s}/\BETA}$ false.  Thus,  $\CAPPHI\variable{s}/\BETA\sdtstile{}{}\existential{\BETA}\CAPPHI$. So by transitivity, $\Delta^*\sdtstile{}{}\existential{\BETA}\CAPPHI$.

\item[\Rule{$\forall$-Intro}:]
Say we add another line to $\Derivation{D}$ with sentence $\universal{\BETA}\CAPPHI$ sanctioned by \Rule{$\forall$-Intro}.
Then there's some earlier line $\integer{i}$ with the sentence $\CAPPHI\variable{s}/\BETA$. 
Again we have that $\Delta_{\integer{i}}$ is the set of unboxed assumptions of line $\integer{i}$, and by NDAP $\Delta_{\integer{i}}\subseteq\Delta^*$.
By the recursive assumption, $\Delta_{\integer{i}}\sdtstile{}{}\CAPPHI\variable{s}/\BETA$. 
By monotonicity, $\Delta^*\sdtstile{}{}\CAPPHI\variable{s}/\BETA$.
However we know that $\CAPPHI\variable{s}/\BETA$ does not entail $\universal{\BETA}\CAPPHI$, so we have to do some extra work and make use of the restrictions on the rule \Rule{$\forall$-Intro}. 

Let $\IntA$ be some model that makes all of $\Delta^*$ true; and let's assume for \emph{reductio} that $\IntA$ that makes $\universal{\BETA}\CAPPHI$ false.  One of the restrictions for \Rule{$\forall$-Intro} is that $\variable{s}$ must not occur in $\universal{\BETA}\CAPPHI$.  Hence, by the Free Choice Theorem, $\CAPPHI\variable{s}/\BETA$ is false on some $\variable{s}$-variant of $\IntA$.  Let's name that $\variable{s}$-variant $\As{\variable{s}}{}$.

The other rule restriction for \Rule{$\forall$-Intro} is that $\variable{s}$ must not occur in $\Delta^*$.  The variant $\As{\variable{s}}{}$ differs from $\IntA$ only on the assignment to $\variable{s}$; otherwise, they make all the same assignments.  Because $\variable{s}$ doesn't occur in $\Delta^*$ and $\IntA$ makes all of $\Delta^*$ true, $\As{\variable{s}}{}$ also makes all of $\Delta^*$ true.  The assignment $\As{\variable{s}}{}$ makes to $\variable{s}$ doesn't matter for this result.

Because $\Delta^*\sdtstile{}{}\CAPPHI\variable{s}/\BETA$, $\As{\variable{s}}{}$ makes $\CAPPHI\variable{s}/\BETA$ true.  But we had concluded that $\CAPPHI\variable{s}/\BETA$ is false on $\As{\variable{s}}{}$.  We've inferred a contradiction.  Our assumption that $\IntA$ makes $\universal{\BETA}\CAPPHI$ false must be wrong.

Therefore, if $\IntA$ is a model that makes all of $\Delta^*$ true, then $\IntA$ makes $\universal{\BETA}\CAPPHI$ true as well. 
So, $\Delta^*\sdtstile{}{}\universal{\BETA}\CAPPHI$.

\item[\Rule{$\exists$-Elim}:]
Say we add another line to $\Derivation{D}$ with sentence $\CAPTHETA$ sanctioned by \Rule{$\exists$-Elim}.
Then there's some earlier line $\integer{i}$ with the sentence $\horseshoe{\CAPPHI\variable{s}/\BETA}{\CAPTHETA}$ and an earlier line $\integer{j}$ with the sentence $\existential{\BETA}\CAPPHI$. 
Again we have that $\Delta_{\integer{i}}$ is the set of unboxed assumptions of line $\integer{i}$ and $\Delta_{\integer{j}}$ the unboxed assumptions of line $\integer{j}$.
By NDAP $\Delta_{\integer{i}}\subseteq\Delta^*$ and $\Delta_{\integer{j}}\subseteq\Delta^*$.
By the recursive assumption, $\Delta_{\integer{i}}\sdtstile{}{}\horseshoe{\CAPPHI\variable{s}/\BETA}{\CAPTHETA}$ and $\Delta_{\integer{j}}\sdtstile{}{}\existential{\BETA}\CAPPHI$.
By monotonicity, $\Delta^*\sdtstile{}{}\horseshoe{\CAPPHI\variable{s}/\BETA}{\CAPTHETA}$ and $\Delta^*\sdtstile{}{}\existential{\BETA}\CAPPHI$.
Again we have to do some extra work and make use of the restrictions on the rule \Rule{$\exists$-Elim} to show that $\Delta^*\sdtstile{}{}\CAPTHETA$. 

Let $\IntA$ be some model that makes all of $\Delta^*$ true. 
Because $\Delta^*\sdtstile{}{}\existential{\BETA}\CAPPHI$, $\IntA$ also makes $\existential{\BETA}\CAPPHI$ true.  One of the rule restrictions for \Rule{$\exists$-Elim} is that $\variable{s}$ must not occur in $\existential{\BETA}\CAPPHI$.  Hence, by the Free Choice theorem, $\CAPPHI{\variable{s}/\BETA}$ is true on some $\variable{s}$-variant of $\IntA$.  Name that $\variable{s}$-variant $\As{\variable{s}}{}$.  

Another of the rule restrictions for \Rule{$\exists$-Elim} is that $\variable{s}$ must not occur in $\Delta^*$.  The variant $\As{\variable{s}}{}$ makes all the same assignments as $\IntA$ except in what it assigns to $\variable{s}$.  Because $\IntA$ makes $\Delta^*$ true and $\Delta^*$ doesn't contain $\variable{s}$, $\As{\variable{s}}{}$ also makes $\Delta^*$ true.  The assignment $\As{\variable{s}}{}$ makes to $\variable{s}$ doesn't make any difference.

Thus, because $\Delta^*\sdtstile{}{}\horseshoe{\CAPPHI\variable{s}/\BETA}{\CAPTHETA}$, $\As{\variable{s}}{}$ makes $\horseshoe{\CAPPHI\variable{s}/\BETA}{\CAPTHETA}$ true.  We saw earlier that $\As{\variable{s}}{}$ makes $\CAPPHI\variable{s}/\BETA$ true, so $\As{\variable{s}}{}$ makes $\CAPTHETA$ true as well (def. of truth, $\HORSESHOE$).  

According to the third rule restriction for \Rule{$\exists$-Elim}, $\variable{s}$ must not occur in $\CAPTHETA$.  Because $\As{\variable{s}}{}$ makes $\CAPTHETA$ true and $\variable{s}$ isn't in $\CAPTHETA$, $\IntA$ also makes $\CAPTHETA$ true.  The assignment that $\As{\variable{s}}{}$ makes to $\variable{s}$ is irrelevant.

So, we have shown that $\Delta^*\sdtstile{}{}\CAPTHETA$.

\end{description}

\item[Closure Step:] We have now covered all the generating cases for derivations. By the closure clause of the definition, we have proved soundness for all derivations in \GQD{}. 

\end{description}
\end{PROOFOF} 

%%%%%%%%%%%%%%%%%%%%%%%%%%%%%%%%%%%%%%%%%%%%%%%%%%
\section{Completeness}\label{Section:Completeness for GSD}
%%%%%%%%%%%%%%%%%%%%%%%%%%%%%%%%%%%%%%%%%%%%%%%%%%

In this section we first prove the completeness of \GSD{}.

\begin{majorILnc}{\LnpDC{LRCompleteness}}
	A derivation system \DerivationSystem{D} for \Language{L} is \nidf{complete}\index{completeness|textbf} \Iff for every finite set $\Delta$ of sentences of \Language{L} and every sentence $\CAPPHI$ of \Language{L}, if $\Delta\sdtstile{}{}\CAPPHI$, then $\Delta\sststile{}{}\CAPPHI$.
\end{majorILnc} 
\noindent{}If we limit $\Delta$ to just the empty set, we get weak completeness:
\begin{majorILnc}{\LnpDC{LWCompleteness}}
	A derivation system \DerivationSystem{D} for \Language{L} is \nidf{weakly complete}\index{completeness!weak|textbf} \Iff for every sentence $\CAPPHI$ of \Language{L}, if $\sdtstile{}{}\CAPPHI$, then $\sststile{}{}\CAPPHI$.
\end{majorILnc} 
\noindent{}The following theorem can be proved using basic results we already have.  For other systems of logic, what we are calling completeness and weak completeness are not equivalent.  Because they equivalent are in our systems, we will not always distinguish them in what follows.  Strong completeness also holds for our systems, but is not trivially equivalent to completeness.  As in the first cases, there are systems that are complete but not strongly complete.
\begin{THEOREM}{\LnpTC{RegWeakCompletenessEquiv}}
	\GSD{} is weakly complete \Iff it's complete; and likewise for \GQD{}.
\end{THEOREM}
\begin{PROOF}
	$(\Leftarrow)$ This direction of the biconditional is trivial. 
	Assume that \GSD{}/\GQD{} is complete. 
	Then for any finite set $\Delta$, if $\Delta\sdtstile{}{}\CAPPHI$, then $\Delta\sststile{}{}\CAPPHI$. 
	By definition, this includes the case when $\Delta$ is the empty set. 
	Hence \GSD{}/\GQD{} is weakly complete. 
	
	$(\Rightarrow)$ Assume that \GSD{}/\GQD{} is weakly complete. 
	Hence for any sentence $\CAPPSI$, if $\sdtstile{}{}\CAPPSI$, then $\sststile{}{}\CAPPSI$. 
	Now assume that, for some finite set $\Delta$ of sentences and sentence $\CAPPSI$, $\Delta\sdtstile{}{}\CAPPHI$.
	Since $\Delta$ is finite, we can consider the conjunction of all the sentences in $\Delta$.
	Let $\DELTA$ be this conjunction. 
	%From the \GQL{} Entailment-Exponentiation Theorem (Thm. \pmvref{Exponentiation of Entailment GQL}) we know that $\DELTA\sdtstile{}{}\CAPPHI$ \Iff $\sdtstile{}{}\horseshoe{\DELTA}{\CAPPHI}$.
	We want to show that $\sdtstile{}{}\horseshoe{\DELTA}{\CAPPHI}$;
	to do so, assume that there's some model $\IntA$ that makes $\horseshoe{\DELTA}{\CAPPHI}$ false.
	By the definition of truth for $\HORSESHOE$ and $\WEDGE$, it follows that $\IntA$ makes all the conjuncts of $\DELTA$ true and $\CAPPHI$ false. 
	But that would mean that $\IntA$ makes all the sentences in $\Delta$ true and $\CAPPHI$ false.
	But we assumed that $\Delta\sdtstile{}{}\CAPPHI$, so there's no model $\IntA$ that makes $\horseshoe{\DELTA}{\CAPPHI}$ false.
	Hence $\sdtstile{}{}\horseshoe{\DELTA}{\CAPPHI}$, and so by weak completeness, $\sststile{}{}\horseshoe{\DELTA}{\CAPPHI}$.
	It should be clear to the reader that if $\sststile{}{}\horseshoe{\DELTA}{\CAPPHI}$, then $\DELTA\sststile{}{}\CAPPHI$.
	Hence, $\DELTA\sststile{}{}\CAPPHI$.
	Finally, it should be clear that $\Delta\sststile{}{}\DELTA$, and since $\sststile{}{}$ is transitive, $\Delta\sststile{}{}\CAPPHI$.
	%To show that $\Delta\sststile{}{}\CAPPHI$, first write each sentence in $\Delta$ (which are the conjuncts of $\DELTA$) as an assumption on a line in a derivation.
	%Then derive $\horseshoe{\DELTA}{\CAPPHI}$, which we know can be done without any assumptions. 
	%Next, use \Rule{$\WEDGE\!$-Intro} to conjoin all the assumptions from $\Delta$; this will result in $\DELTA$ on a line in the derivation. 
	%Then use \Rule{$\HORSESHOE$-Elim} on $\horseshoe{\DELTA}{\CAPPHI}$ and $\DELTA$. 
	%This will get us $\CAPPHI$ on a line with all and only the sentences of $\Delta$ as assumptions, thus showing that $\Delta\sststile{}{}\CAPPHI$. 
\end{PROOF}
\begin{majorILnc}{\LnpDC{LCompleteness}}
	A derivation system \DerivationSystem{D} for \Language{L} is \nidf{strongly complete}\index{completeness!strong|textbf} \Iff for every set $\Delta$ of sentences of \Language{L} and every sentence $\CAPPHI$ of \Language{L}, if $\Delta\sdtstile{}{}\CAPPHI$, then $\Delta\sststile{}{}\CAPPHI$.
\end{majorILnc} 
\noindent{}Note that in the definitions the \CAPS{rhs} of the biconditionals must hold even in the special case when $\Delta$ is the empty set and the special case when $\Delta$ is infinite. 
If we limit $\Delta$ so that it must be finite (but still allow it to be empty), we get (regular) completeness.
\noindent{}Note that both \GSD{} and \GQD{} are strongly complete, but there is no simple theorem that uses results we already have which extends weak completeness to strong completeness in the way this theorem (Thm. \ref{RegWeakCompletenessEquiv}) extends weak completeness to (regular) completeness.

Returning to strong completeness, letting $\Delta$ be infinite may seem problematic, since as we've defined them (def. \pmvref{Recursive definition of Derivation}) derivations can only have finitely many lines. 
Hence, a derivation can only have finitely many assumptions. 
And, as we've defined the single turnstile, $\Delta\sststile{}{}\CAPPHI$ iff there's a derivation of $\CAPPHI$ from the sentences in $\Delta$. 
But there's nothing problematic about letting $\Delta$ be infinite, because showing that there's a derivation of $\CAPPHI$ from the sentences in $\Delta$ doesn't require that the derivation use \emph{all} the sentences in $\Delta$ as assumptions. 
In general, even when $\Delta$ is finite, any derivation of $\CAPPHI$ from some subset of sentences in $\Delta$ will show that $\Delta\sststile{}{}\CAPPHI$. 
So, if $\Delta$ is infinite and $\Delta\sdtstile{}{}\CAPPHI$, if the derivation system \DerivationSystem{D} is complete we'll know that $\CAPPHI$ can be derived from some finite subset of sentences of $\Delta$.\footnote{Before turning to the proofs of these theorems, some historical background might be of interest. 
	As mentioned above (Sec. \ref{Sec:GQLSymbols}), quantificational languages were first developed by Frege, Peirce and Mitchell in the 1870's and 1880's. 
	But it wasn't until David Hilbert and Wilhelm Ackermann published their hugely influential text \emph{Grundz\"uge der theoretischen Logik} (Principles of Mathematical Logic) in \citeyear{Hilbert1928} that the question of completeness was clearly formulated. 
	While Kurt G\"odel, in his \citeyear{Godel1929} doctorial dissertation (republished in \citeyear{Godel1930}), is widely accepted as the first person to prove that quantificational logic is strongly complete, Church \citeyearpar[291,~fn.464]{Church1956} reports that the Jacques Herbrand's dissertation in 1930 had the essential material for the same proof.  
	Further, completeness follows from results of Skolem \citeyearpar{Skolem1928}, but since the question of completeness hadn't been clearly raised yet no one seems to have noticed. 
	Leon Henkin \citeyearpar{Henkin1949} later developed a method of proving completeness different from G\"odels. 
	Henkin's approach is probably the most common one used today in logic textbooks, but the proof we give here is a constructive proof closer to G\"odel's original.
	(Ours owes much to Willard Quine's completeness proof \citeyearpar{Quine1982}.)}



We'll prove that \GSD{} is weakly complete and then show that, for \GSD{}, completeness and weak completeness are equivalent. 
By theorem \mvref{RegWeakCompletenessEquiv}, this will be sufficient to show that \GSD{} is complete.
The equivalent statement is:
\begin{THEOREM}{\LnpTC{GSDCompletenessLemma} The \GSD{} Weak Completeness Lemma:}
For\index{completeness!weak \GSD{}} any sentence $\CAPPHI$ of \GSD{}, either $\CAPPHI\sststile{}{}\conjunction{\Al}{\negation{\Al}}$, or $\CAPPHI$ is true in some model $\IntA$.
\end{THEOREM}

%\noindent{}We need a systematic way of looking for derivations.  Our method will be to assume the opposite of the sentence of interest and then derive a sentence in DNF that is provably equivalent.  The advantage of DNF is that it's simple and transparent.

%Once we have a DNF sentence, it is easy to either extract a contradiction or to read off a model that makes the original sentence true.  This means that we can show either $\negation{\CAPPHI}\sdtstile{}{}\parconjunction{\Al}{\negation{\Al}}$ or define a model that makes $\negation{\CAPPHI}$ true.
%So we either have a derivation of $\CAPPHI$ in a few more steps, or a model that shows $\CAPPHI$ is not a logical truth.

Before proving the theorem, it will be useful to introduce a new exchange rule for \GQD{} 
and then show that anything we can derive using \GQD{} and this rule can be derived using \GQD{} alone. We call the rule \Rule{$\TRIPLEBAR$-Exchange}.
(Note that every application of \Rule{$\TRIPLEBAR$-Exchange} is truth preserving, as the last problem in exercise \pmvref{exercises:GSDTFETheorem}, extends theorem \pmvref{ExchangeRuleGSDSoundnessLemma}, to it.)
It's given in table \ref{GSDplusDNF}.
\begin{table}[!ht]
\renewcommand{\arraystretch}{1.5}
\begin{center}
\begin{tabular}{ p{1in} l l } %p{2.2in} p{2in}
\toprule
\textbf{Name} & \textbf{Given} & \textbf{May Add} \\ 
\midrule
\Rule{$\TRIPLEBAR$-Exchange} &  $\triplebar{\CAPTHETA}{\CAPPSI}$ & $\disjunction{\parconjunction{\CAPTHETA}{\CAPPSI}}{\parconjunction{\negation{\CAPTHETA}}{\negation{\CAPPSI}}}$ \\
\nopagebreak
 & $\disjunction{\parconjunction{\CAPTHETA}{\CAPPSI}}{\parconjunction{\negation{\CAPTHETA}}{\negation{\CAPPSI}}}$ &  $\triplebar{\CAPTHETA}{\CAPPSI}$ \\
\bottomrule
\end{tabular}
\end{center}
\caption{\Rule{$\TRIPLEBAR$-Exchange}}
\label{GSDplusDNF}%
\end{table}
\index{derivation!rule!for DNF}\index{DNF}
\noindent{}Recall from section \ref{Shortcut Rule Elimination Theorem Section} that all we need to do to show that anything that can be derived using \GQD{} and this rule can be derived using just \GQD{} is to prove the following:
\begin{THEOREM}{\LnpTC{GQD NDF Rule}}
Any two \GQL{} formulas got by substituting other \GQL{} formulas into the may-add and given schemas of \Rule{$\TRIPLEBAR$-Exchange} are provably equivalent; that is, $\sststile{}{}\forall\bpartriplebar{\partriplebar{\CAPTHETA}{\CAPPSI}}{\pardisjunction{\parconjunction{\CAPTHETA}{\CAPPSI}}{\parconjunction{\negation{\CAPTHETA}}{\negation{\CAPPSI}}}}$.
\end{THEOREM}
\begin{PROOF}
We show that $\sststile{}{}\forall\bpartriplebar{\partriplebar{\CAPTHETA}{\CAPPSI}}{\pardisjunction{\parconjunction{\CAPTHETA}{\CAPPSI}}{\parconjunction{\negation{\CAPTHETA}}{\negation{\CAPPSI}}}}$ by giving a derivation schema, which for any two formulas $\CAPTHETA$ and $\CAPPSI$ will result in the needed derivation. 
(Note that to save space $\integer{q}=\integer{n}+\integer{m}$.)
\begin{gproofnn}
\gaproof{
\galine{1}{$\partriplebar{\CAPTHETA}{\CAPPSI}\constant{c_{\integer{1}}}\ldots\constant{c_{\integer{\integer{m}}}}/\variable{x}_1\ldots\variable{x}_{\integer{m}}$}{\Rule{Assume}}
\galine{2}{$\partriplebar{\negation{\CAPTHETA}}{\negation{\CAPPSI}}\constant{c_{\integer{1}}}\ldots\constant{c_{\integer{\integer{m}}}}/\variable{x}_1\ldots\variable{x}_{\integer{m}}$}{\Rule{$\NEGATION$/$\TRIPLEBAR$-Intro}, 1}
\gaaproof{
\gaaline{3}{$\negation{\parconjunction{\CAPTHETA}{\CAPPSI}}\constant{c_{\integer{1}}}\ldots\constant{c_{\integer{\integer{m}}}}/\variable{x}_1\ldots\variable{x}_{\integer{m}}$}{\Rule{Assume}}
\gaaline{4}{$\pardisjunction{\negation{\CAPTHETA}}{\negation{\CAPPSI}}\constant{c_{\integer{1}}}\ldots\constant{c_{\integer{\integer{m}}}}/\variable{x}_1\ldots\variable{x}_{\integer{m}}$}{\Rule{DeM}, 3}
\gaaaproof{
\gaaaline{5}{$\negation{\CAPTHETA}\constant{c_{\integer{1}}}\ldots\constant{c_{\integer{\integer{m}}}}/\variable{x}_1\ldots\variable{x}_{\integer{m}}$}{\Rule{Assume}}
\gaaaline{6}{$\negation{\CAPPSI}\constant{c_{\integer{1}}}\ldots\constant{c_{\integer{\integer{m}}}}/\variable{x}_1\ldots\variable{x}_{\integer{m}}$}{\Rule{$\TRIPLEBAR$-Elim}, 2, 5}
\gaaaline{7}{$\parconjunction{\negation{\CAPTHETA}}{\negation{\CAPPSI}}\constant{c_{\integer{1}}}\ldots\constant{c_{\integer{\integer{m}}}}/\variable{x}_1\ldots\variable{x}_{\integer{m}}$}{\Rule{$\WEDGE\!$-Intro}, 5, 6}
}
\gaaline{8}{$\parhorseshoe{\negation{\CAPTHETA}}{\parconjunction{\negation{\CAPTHETA}}{\negation{\CAPPSI}}}\constant{c_{\integer{1}}}\ldots\constant{c_{\integer{\integer{m}}}}/\variable{x}_1\ldots\variable{x}_{\integer{m}}$}{\Rule{$\HORSESHOE$-Intro}, 5--7}

\gaaaproof{
\gaaaline{9}{$\negation{\CAPPSI}\constant{c_{\integer{1}}}\ldots\constant{c_{\integer{\integer{m}}}}/\variable{x}_1\ldots\variable{x}_{\integer{m}}$}{\Rule{Assume}}
\gaaaline{10}{$\negation{\CAPTHETA}\constant{c_{\integer{1}}}\ldots\constant{c_{\integer{\integer{m}}}}/\variable{x}_1\ldots\variable{x}_{\integer{m}}$}{\Rule{$\TRIPLEBAR$-Elim}, 2, 9}
\gaaaline{11}{$\parconjunction{\negation{\CAPTHETA}}{\negation{\CAPPSI}}\constant{c_{\integer{1}}}\ldots\constant{c_{\integer{\integer{m}}}}/\variable{x}_1\ldots\variable{x}_{\integer{m}}$}{\Rule{$\WEDGE\!$-Intro}, 9, 10}
}
\gaaline{12}{$\parhorseshoe{\negation{\CAPPSI}}{\parconjunction{\negation{\CAPTHETA}}{\negation{\CAPPSI}}}\constant{c_{\integer{1}}}\ldots\constant{c_{\integer{\integer{m}}}}/\variable{x}_1\ldots\variable{x}_{\integer{m}}$}{\Rule{$\HORSESHOE$-Intro}, 9--11}
\gaaline{13}{$\parconjunction{\negation{\CAPTHETA}}{\negation{\CAPPSI}}\constant{c_{\integer{1}}}\ldots\constant{c_{\integer{\integer{m}}}}/\variable{x}_1\ldots\variable{x}_{\integer{m}}$}{\Rule{$\VEE$-Intro}, 4, 8, 12}
}
\galine{14}{$\parhorseshoe{\negation{\parconjunction{\CAPTHETA}{\CAPPSI}}}{\parconjunction{\negation{\CAPTHETA}}{\negation{\CAPPSI}}}\constant{c_{\integer{1}}}\ldots\constant{c_{\integer{\integer{m}}}}/\variable{x}_1\ldots\variable{x}_{\integer{m}}$}{\Rule{$\HORSESHOE$-Intro}, 3--13}
\galine{15}{$\pardisjunction{\negation{\negation{\parconjunction{\CAPTHETA}{\CAPPSI}}}}{\parconjunction{\negation{\CAPTHETA}}{\negation{\CAPPSI}}}\constant{c_{\integer{1}}}\ldots\constant{c_{\integer{\integer{m}}}}/\variable{x}_1\ldots\variable{x}_{\integer{m}}$}{\Rule{$\HORSESHOE$/$\VEE$-Exch.}, 14}
\galine{16}{$\pardisjunction{\parconjunction{\CAPTHETA}{\CAPPSI}}{\parconjunction{\negation{\CAPTHETA}}{\negation{\CAPPSI}}}\constant{c_{\integer{1}}}\ldots\constant{c_{\integer{\integer{m}}}}/\variable{x}_1\ldots\variable{x}_{\integer{m}}$}{\Rule{$\NEGATION\NEGATION$-Elim}, 15}
}
\gline{17}{$[\partriplebar{\CAPTHETA}{\CAPPSI}\HORSESHOE$}{ }
\nopagebreak
\glinend{ }{$\qquad\pardisjunction{\parconjunction{\CAPTHETA}{\CAPPSI}}{\parconjunction{\negation{\CAPTHETA}}{\negation{\CAPPSI}}}]\constant{c_{\integer{1}}}\ldots\constant{c_{\integer{\integer{m}}}}/\variable{x}_1\ldots\variable{x}_{\integer{m}}$}{\Rule{$\HORSESHOE$-Intro}, 1--16}
\gaproof{
\galine{18}{$\pardisjunction{\parconjunction{\CAPTHETA}{\CAPPSI}}{\parconjunction{\negation{\CAPTHETA}}{\negation{\CAPPSI}}}\constant{c_{\integer{1}}}\ldots\constant{c_{\integer{\integer{m}}}}/\variable{x}_1\ldots\variable{x}_{\integer{m}}$}{\Rule{Assume}}
\galinend{ }{ }{ }
\galinend{ }{$\qquad\vdots$}{ }
\galinend{ }{ }{ }
\galine{$\integer{n}$}{$\partriplebar{\CAPTHETA}{\CAPPSI}\constant{c_{\integer{1}}}\ldots\constant{c_{\integer{\integer{m}}}}/\variable{x}_1\ldots\variable{x}_{\integer{m}}$}{ }
}
\gline{$\integer{n}+1$}{$[\pardisjunction{\parconjunction{\CAPTHETA}{\CAPPSI}}{\parconjunction{\negation{\CAPTHETA}}{\negation{\CAPPSI}}}\HORSESHOE$}{ }
\glinend{ }{$\qquad\partriplebar{\CAPTHETA}{\CAPPSI}]\constant{c_{\integer{1}}}\ldots\constant{c_{\integer{\integer{m}}}}/\variable{x}_1\ldots\variable{x}_{\integer{m}}$}{\Rule{$\HORSESHOE$-Intro}, 18--$\integer{n}$}
\gline{$\integer{n}+2$}{$[\partriplebar{\CAPTHETA}{\CAPPSI}\TRIPLEBAR$}{\Rule{$\TRIPLEBAR$-Intro}, 17,}
\glinend{ }{$\qquad\pardisjunction{\parconjunction{\CAPTHETA}{\CAPPSI}}{\parconjunction{\negation{\CAPTHETA}}{\negation{\CAPPSI}}}]\constant{c_{\integer{1}}}\ldots\constant{c_{\integer{\integer{m}}}}/\variable{x}_1\ldots\variable{x}_{\integer{m}}$}{$\integer{n}+1$}
\glinend{ }{ }{ }
\glinend{ }{$\qquad\vdots$}{ }
\glinend{ }{ }{ }
\gline{$\integer{q}+2$}{$\forall\bpartriplebar{\partriplebar{\CAPTHETA}{\CAPPSI}}{\pardisjunction{\parconjunction{\CAPTHETA}{\CAPPSI}}{\parconjunction{\negation{\CAPTHETA}}{\negation{\CAPPSI}}}}$}{\Rule{$\forall$-Intro}, $\integer{q}+1$}
\end{gproofnn}
\noindent{}Note that we have left steps $18$--$\integer{n}$ for the reader; 
this is just the derivation of the other conditional needed for \Rule{$\TRIPLEBAR$-Intro} on line $\integer{n}+2$. 
Also note that the last steps, lines $\integer{n}+3$ to the end, are all \Rule{$\forall$-Intro} meant to eliminate the constants $\constant{c_{\integer{1}}},\ldots,\constant{c_{\integer{\integer{m}}}}$.
\end{PROOF}
%Now we turn to the proof of the \GSD{} Completeness Lemma.
\begin{PROOFOF}{Thm. \ref{GSDCompletenessLemma}}
To prove the theorem, we shall describe an algorithm for applying the rules of \GSDP{} and \Rule{$\TRIPLEBAR$-Exchange} that takes a \GSL{} sentence $\CAPPHI$ and either halts in a derivation of $\conjunction{\Al}{\negation{\Al}}$, or halts with a sentence in \CAPS{dnf} for which there is some model $\IntA$ that makes $\CAPPHI$ true.
Since a sentence can be derived using the rules of \GSDP{} and \Rule{$\TRIPLEBAR$-Exchange} \Iff it can be derived using the basic rules of \GSD{}, this will be sufficient to prove the theorem. 

The algorithm begins with $\CAPPHI$ as an assumption on line 1. 
The algorithm then applies the method studied earlier in section \mvref{Disjunctive Normal Form} to produce a sentence $\CAPPHI'$ in \CAPS{dnf} that's \CAPS{tfe} to $\CAPPHI$.
We have to show that each step of the earlier method can be carried out in steps using the rules of \GSDP{} and \Rule{$\TRIPLEBAR$-Exchange}.
The earlier method proceeded in three stages. 
\begin{description}
\item[Step A:] \hfill
\begin{cenumerate}
\item If a subsentence of $\CAPPHI$ has $\HORSESHOE$ as its main connective, i.e. if $\CAPPHI=\horseshoe{\CAPTHETA}{\CAPPSI}$, replace the subsentence by $\disjunction{\negation{\CAPTHETA}}{\CAPPSI}$.
Repeat as necessary to obtain a sentence $\CAPPHI^*$ without conditionals. 
Each of these steps are sanctioned by \Rule{$\HORSESHOE$/$\VEE$-Exchange}.

\item If a subsentence of $\CAPPHI$ has $\TRIPLEBAR$ as its main connective, i.e. if $\CAPPHI=\triplebar{\CAPTHETA}{\CAPPSI}$, it is replaced with the subsentence $\disjunction{\parconjunction{\CAPTHETA}{\CAPPSI}}{\parconjunction{\negation{\CAPTHETA}}{\negation{\CAPPSI}}}$.
Repeat as necessary to obtain a sentence $\CAPPHI^{**}$ without biconditionals.
Each of these steps are sanctioned by \Rule{$\TRIPLEBAR$-Exchange}.
\end{cenumerate}

\item[Step B:]
In the case where $\CAPPHI^{**}$ contains a subsentence whose main connective is negation and which contains other connectives, we replace that subsentence by the following steps:
\begin{cenumerate}
\item Replace $\negation{\negation{\CAPTHETA}}$ by $\CAPTHETA$; this step is sanctioned by \Rule{$\NEGATION\NEGATION$-Elim}.
\item Replace $\negation{\parconjunction{\CAPTHETA}{\CAPPSI}}$ by $\disjunction{\negation{\CAPTHETA}}{\negation{\CAPPSI}}$; this step is sanctioned by \Rule{DeM}.
\item Replace $\negation{\pardisjunction{\CAPTHETA}{\CAPPSI}}$ by $\conjunction{\negation{\CAPTHETA}}{\negation{\CAPPSI}}$; this step is sanctioned by \Rule{DeM}.
\end{cenumerate}
Repeat as necessary to obtain a sentence $\CAPPHI^{***}$ in which negations govern nothing but sentence letters. 

\item[Step C:]
The only thing that could prevent $\CAPPHI^{***}$ from being in \CAPS{dnf} is that some conjunctions govern some disjunctions, i.e., there is a subsentence of the form $\conjunction{\CAPTHETA}{\pardisjunction{\CAPPSI_1}{\disjunction{\ldots}{\CAPPSI_{\integer{n}}}}}$, or the reverse $\conjunction{\pardisjunction{\CAPPSI_1}{\disjunction{\ldots}{\CAPPSI_{\integer{n}}}}}{\CAPTHETA}$.
Those subsentences can each be replaced by the equivalent sentence $\disjunction{\parconjunction{\CAPTHETA}{\CAPPSI_1}}{\disjunction{\ldots}{\parconjunction{\CAPTHETA}{\CAPPSI_{\integer{n}}}}}$ or $\disjunction{\parconjunction{\CAPPSI_1}{\CAPTHETA}}{\disjunction{\ldots}{\parconjunction{\CAPPSI_{\integer{n}}}{\CAPTHETA}}}$.
These steps are sanctioned by \Rule{Distribution}.
\end{description}
\noindent{}Applying the above steps A, B, and C will provide us a derivation starting with $\CAPPHI$ as an assumption (and no other assumptions) and ending with a \CAPS{dnf} sentence that's \CAPS{tfe} to $\CAPPHI$. 
We now have two possibilities:
\begin{description}
\item[Case 1:] 
Every disjunct contains a sentence letter and the negation of that sentence letter. 
That is, each disjunction has the form $\parconjunction{\CAPPSI_1}{\conjunction{\ldots}{\conjunction{\CAPPSI_{\integer{i}}}{\conjunction{\ldots}{\conjunction{\negation{\CAPPSI_{\integer{i}}}}{\conjunction{\ldots}{\CAPPSI_{\integer{n}}}}}}}}$; for example: $\parconjunction{\Al}{\conjunction{B}{\conjunction{\Cl}{\conjunction{\negation{\El}}{\conjunction{\negation{\Bl}}{\negation{\Kl}}}}}}$.

\item[Case 2:]
At least one disjunct contains no sentence letter such that the negation of the sentence letter is also in the disjunct. 
\end{description}
\noindent{}We can show in case $1$ that the original sentence leads to a contradiction.
First, we observe that any conjunction that contains a sentence letter and its negation leads to a contradiction by repeated steps of \Rule{$\WEDGE\!$-Elim}. 
Thus we can derive the negation of any such conjunction using \Rule{$\NEGATION$-Intro}.
So, if the last line of our derivation so far is of the form $\pardisjunction{\CAPPSI_1}{\disjunction{\ldots}{\CAPPSI_{\integer{n}}}}$ and each $\CAPPSI_{\integer{i}}$ contains a sentence letter and the negation of that sentence letter, then we can add to the derivation lines that establish the negation of each $\CAPPSI_{\integer{i}}$. 
Thus by $\integer{n}-1$ steps of \Rule{D.S.} we get a single $\CAPPSI_{\integer{i}}$ by itself with only the first line as an assumption.
Since by hypothesis in this case $\CAPPSI_{\integer{i}}$ leads to a contradiction, we can show the initial assumption leads to a contradiction.  %Here is an example of case $1$, where $\CAPPHI=$


%\begin{gproofnn}
%	\gaproof{%
%		\galine{1}{$\conjunction{\Dp{\text{a}}}{\universal{\variable{z}}\Hp{\variable{z}}}$}{A}%
%		\galine{2}{$\Dp{\text{a}}$}{$\WEDGE$-Elim 1}%
%		\galine{3}{$\universal{\variable{z}}\Hp{\variable{z}}$}{$\WEDGE$-Elim 1}%
%		\galine{4}{$\Hp{\text{a}}$}{$\forall$-Elim 3}%
%		\galine{5}{$\conjunction{\Dp{\text{a}}}{\Hp{\text{a}}}$}{$\WEDGE$-Intro 2,4}%
%		\galine{6}{$\existential{\variable{y}}\parconjunction{\Dp{\variable{y}}}{\Hp{\variable{y}}}$}{$\exists$-Intro 5}%
%	}%
%	\gline{7}{$\horseshoe{\parconjunction{\Dp{\text{a}}}{\universal{\variable{z}}\Hp{\variable{z}}}}{\existential{\variable{y}}\parconjunction{\Dp{\variable{y}}}{\Hp{\variable{y}}}}$}{$\HORSESHOE$-Intro 1--6}%
%	\gline{8}{$\universal{\variable{x}}\bparhorseshoe{\parconjunction{\Dp{\variable{x}}}{\universal{\variable{z}}\Hp{\variable{z}}}}{\existential{\variable{y}}\parconjunction{\Dp{\variable{y}}}{\Hp{\variable{y}}}}$}{$\forall$-Intro 7 (a)}%
%\end{gproofnn}















Generally, the procedure will look like this:

\begin{gproofnn}
\glinend{ }{$\CAPPHI$}{\Rule{Assume}} %\marginnote{\scriptsize{}The original sentence}[0cm]
\glinend{ }{ }{ }
\glinend{ }{$\qquad\vdots$}{ }
\glinend{ }{ }{ }
\glinend{ }{$\disjunction{\CAPPSI_1}{\disjunction{\ldots}{\CAPPSI_{\integer{n}}}}$}{ } %\marginnote{\scriptsize{}The \CAPS{dnf} sentence after steps A, B, and C}[0cm]
\gaproof{
\galinend{ }{$\CAPPSI_1$}{\Rule{Assume}}
\galinend{ }{ }{ }
\galinend{ }{$\qquad\vdots$}{ }
\galinend{ }{ }{ }
\galinend{ }{$\conjunction{\CAPTHETA_1}{\negation{\CAPTHETA_1}}$}{ }
}
\glinend{ }{$\horseshoe{\CAPPSI_1}{\parconjunction{\CAPTHETA_1}{\negation{\CAPTHETA_1}}}$}{\Rule{$\HORSESHOE$-Intro}} %\marginnote{\scriptsize{}We start deriving the negation of each disjunct}[0cm]
\glinend{ }{$\negation{\CAPPSI_1}$}{\Rule{$\NEGATION$-Intro}}
\glinend{ }{ }{ }
\glinend{ }{$\qquad\vdots$}{ }
\glinend{ }{ }{ }
\gaproof{
\galinend{ }{$\CAPPSI_{\integer{n}}$}{\Rule{Assume}}
\galinend{ }{ }{ }
\galinend{ }{$\qquad\vdots$}{ }
\galinend{ }{ }{ }
\galinend{ }{$\conjunction{\CAPTHETA_{\integer{n}}}{\negation{\CAPTHETA_{\integer{n}}}}$}{ }
}
\glinend{ }{$\horseshoe{\CAPPSI_{\integer{n}}}{\parconjunction{\CAPTHETA_{\integer{n}}}{\negation{\CAPTHETA_{\integer{n}}}}}$}{\Rule{$\HORSESHOE$-Intro}}
\glinend{ }{$\negation{\CAPPSI_{\integer{n}}}$}{\Rule{$\NEGATION$-Intro}}
\glinend{ }{$\disjunction{\CAPPSI_1}{\disjunction{\ldots}{\CAPPSI_{\integer{n}-1}}}$}{\Rule{D.S.}} %\marginnote{\scriptsize{}Start applying \Rule{D.S.}}[0cm]
\glinend{ }{$\disjunction{\CAPPSI_1}{\disjunction{\ldots}{\CAPPSI_{\integer{n}-2}}}$}{\Rule{D.S.}}
\glinend{ }{$\disjunction{\CAPPSI_1}{\disjunction{\ldots}{\CAPPSI_{\integer{n}-3}}}$}{\Rule{D.S.}}
\glinend{ }{ }{ }
\glinend{ }{$\qquad\vdots$}{ }
\glinend{ }{ }{ }
\glinend{ }{$\disjunction{\CAPPSI_1}{\CAPPSI_2}$}{\Rule{D.S.}}
\glinend{ }{$\CAPPSI_1$}{\Rule{D.S.}}
\glinend{ }{$\negation{\CAPPSI_1}$}{\Rule{Rep.}}
\glinend{ }{$\conjunction{\Al}{\negation{\Al}}$}{\Rule{A.C.}} %\marginnote{\scriptsize{}Finally we reach a contradiction}[0cm]
\end{gproofnn}
We can show in case $2$ that we can find a model that makes all sentences in the derivation true, starting with the last. 
We choose the disjunction that does not contain a sentence letter and its negation (if there is more than one, it doesn't matter which we choose), and we construct a model $\IntA$ by assigning $\TrueB$ to each sentence letter that occurs positively (without a negation in front) in the conjunction and $\FalseB$ to each sentence letter that occurs negatively (with a negation in front).
We can do this since none occur in both modes. 

This model makes each element of the conjunction true and thus makes the entire conjunction true. 
Since the sentence containing it is a disjunction, this is sufficient to make the entire sentence true.
Thus we can make the last line of the derivation true.
Observe now that all of the steps we used in the derivation were replacement of provably equivalence steps;
that is, they used exchange shortcut rules.
Thus, we know that we could also construct a derivation by \mention{turning this proof upside down}, so to speak.  In other words, we could construct a new derivation, with the last step of the original derivation as the initial assumption step.  Then we could use the exchange shortcut rules to work back to $\CAPPHI$ of the original derivation.

Thus, by soundness, we know that if the first sentence of the \mention{upside-down derivation} (the sentence in \CAPS{dnf} that was at the bottom) is true in a model, then so is everything that can be derived from it, including our original sentence $\CAPPHI$ that is now at the end of the inverted derivation. 
Therefore, $\CAPPHI$ is true in some model. 
\end{PROOFOF}
\begin{THEOREM}{\LnpTC{GSDWCompleteness} Weak \GSD{} Completeness Theorem:}
For all \GSL{} sentences $\CAPPHI$: if $\sdtstile{}{}\CAPPHI$, then $\sststile{}{}\CAPPHI$ in \GSD{}.
\end{THEOREM}
\begin{PROOF}
We apply the method above from the \GSD{} Completeness Lemma to the negation of $\CAPPHI$. 
This either produces a derivation of a contradiction from $\negation{\CAPPHI}$, in which case we can prove $\CAPPHI$ by adding two more steps justified by \Rule{$\HORSESHOE$-Intro} and \Rule{$\NEGATION$-Elim}, or it produces a model that makes $\negation{\CAPPHI}$ true and that therefore makes $\CAPPHI$ false. So, $\CAPPHI$ is either false in some model or is derivable in \GSD{}. 
\end{PROOF}
\noindent{}Finally, as a corollary we get:
\begin{THEOREM}{\LnpTC{GSDCompleteness} \GSD{} Completeness Theorem:}
For every finite set $\Delta$ of sentences of \GSL{} and every sentence $\CAPPHI$ of \GSL{}, if $\Delta\sdtstile{}{}\CAPPHI$, then $\Delta\sststile{}{}\CAPPHI$ in \GSD{}.
\end{THEOREM}
\begin{PROOF}
This follows immediately from the Weak \GSD{} Completeness Theorem and theorem \mvref{RegWeakCompletenessEquiv}.
\end{PROOF}

%%%%%%%%%%%%%%%%%%%%%%%%%%%%%%%%%%%%%%%%%%%%%%%%%%
\section{Completeness of \GQD{}}\label{Sec:Completeness of GQD}
%%%%%%%%%%%%%%%%%%%%%%%%%%%%%%%%%%%%%%%%%%%%%%%%%%

In this section we shall prove that \GQD{} is complete.  We would like to use the same kind of strategy for \GQD{} we did for \GSD{}, so we have deal with the quantifiers.
Unfortunately, the quantifiers prevent us from proving the analogue of DNF for \GQL{}. For example:  $\universal{\variable{x}}\pardisjunction{\Hp{\variable{x}}}{\Gp{\variable{x}}}$ is not equivalent to $\disjunction{\universal{\variable{x}}\Hp{\variable{x}}}{\universal{\variable{x}}\Gp{\variable{x}}}$.

To get around problems like this, we must show we can move all of the quantifiers to the front of the sentence.  This has the advantage of separating the quantifier parts of the logical structure from the SL parts.


\subsection{Prenex Definition and Steps}\label{Prenex Definition and Steps}
\begin{majorILnc}{\LnpDC{PrenexNF}}
A sentence $\CAPPHI$ of \GQL{} is in \df{prenex normal form} \Iff every quantifier is in initial position, or, in other words, the scope of all quantifiers is greater than that of any non-quantifier connective.
\end{majorILnc}
\begin{THEOREM}{\LnpTC{PrenexNFTheorem} Prenex Normal Form Theorem:}
For all sentence $\CAPTHETA$ of \GQL{}, there is a provably equivalent sentence $\CAPTHETA^*$ in prenex normal form; that is, $\CAPTHETA^*$ is in prenex normal form and $\sststile{}{}\triplebar{\CAPTHETA}{\CAPTHETA^*}$ in \GQD{}.
\end{THEOREM}
\begin{PROOF}
As with \CAPS{dnf} we have a set of steps for turning sentence $\CAPTHETA$ into a sentence $\CAPTHETA^*$ in prenex normal form. 
First we give the steps, and then show that each step can be sanctioned either by \Rule{QN}, \Rule{$\TRIPLEBAR$-Exchange}, or an exchange rule that can be introduced.
(We'll call these new exchange rules the \niidf{Prenex Exchange Rules}.\index{Exchange Rules!Prenex}) 
Because all the steps in the process are justified by exchange rules, we can either read the resulting series of steps top-down as a derivation of $\CAPTHETA^*$ from $\CAPTHETA$, or bottom-up as a derivation of $\CAPTHETA$ from $\CAPTHETA^*$. 
So, we'll have shown that $\sststile{}{}\triplebar{\CAPTHETA}{\CAPTHETA^*}$ in the derivation system consisting of \Rule{$\TRIPLEBAR$-Exchange} and the Prenex Exchange Rules.
But, as with all the other exchange rules anything that can be derived using the Prenex Exchange Rules can be derived in \GQD{} alone;
so, this will be sufficient to show that $\sststile{}{}\triplebar{\CAPTHETA}{\CAPTHETA^*}$ in \GQD{}. First, the steps are:
\begin{cenumerate}
\item Replace biconditionals with disjunctions of conjunctions; i.e. replace $\triplebar{\CAPPHI}{\CAPPSI}$ with $\disjunction{\parconjunction{\CAPPHI}{\CAPPSI}}{\parconjunction{\negation{\CAPPHI}}{\negation{\CAPPSI}}}$.
\item Rewrite any variables that occur bound by more than one quantifier.
\item Move the first quantifier not in prenex position one step towards the front by the following principles. Repeat this step as often as necessary.  Keep in mind that you can't move forward a quantifier that binds the variable \mention{$\variable{x}$} if it will now have within its scope a new subformula that has a free \mention{$\variable{x}$}.  But we have prevented that problem by eliminating potentially clashing variables in Step 2.
\begin{longtable}[c]{ l l }
\toprule
\textbf{Replace} & \textbf{by} \\
\midrule
$\parconjunction{(\#\variable{x})\CAPTHETA}{\CAPPSI}$ & $(\#\variable{x})\parconjunction{\CAPTHETA}{\CAPPSI}$ \\
$\parconjunction{\CAPTHETA}{(\#\variable{x})\CAPPSI}$ & $(\#\variable{x})\parconjunction{\CAPTHETA}{\CAPPSI}$ \\

$\pardisjunction{(\#\variable{x})\CAPTHETA}{\CAPPSI}$ & $(\#\variable{x})\pardisjunction{\CAPTHETA}{\CAPPSI}$ \\
$\pardisjunction{\CAPTHETA}{(\#\variable{x})\CAPPSI}$ & $(\#\variable{x})\pardisjunction{\CAPTHETA}{\CAPPSI}$ \\

$\parhorseshoe{\CAPTHETA}{(\#\variable{x})\CAPPSI}$ & $(\#\variable{x})\parhorseshoe{\CAPTHETA}{\CAPPSI}$ \\

$\parhorseshoe{\existential{\variable{x}}\CAPTHETA}{\CAPPSI}$ & $\universal{\variable{x}}\parhorseshoe{\CAPTHETA}{\CAPPSI}$ \\
$\parhorseshoe{\universal{\variable{x}}\CAPTHETA}{\CAPPSI}$ & $\existential{\variable{x}}\parhorseshoe{\CAPTHETA}{\CAPPSI}$ \\

$\negation{\existential{\variable{x}}\CAPTHETA}$ & $\universal{\variable{x}}\negation{\CAPTHETA}$ \\
$\negation{\universal{\variable{x}}\CAPTHETA}$ & $\existential{\variable{x}}\negation{\CAPTHETA}$ \\
\bottomrule
\end{longtable}
Note that $(\#\variable{x})$ is just a dummy quantifier standing for either; 
replacement is the same for both quantifiers.  Also, we use \mention{$\variable{x}$} in the chart above, but the same principles hold for quantifiers with any other variable.
\end{cenumerate}
After applying these steps to a sentence $\CAPTHETA$ we will get a sentence $\CAPTHETA^*$ that is in prenex normal form.\footnote{For more discussion of Prenex Form, see \citealt[132]{Kleene1967}, \citealt[54]{Hodges2001}, \citeyear[30]{Hodges2001b}.} 
We now have to show that each step can be sanctioned by an exchange rule.
Step (1) is straightforward, since obviously it will be sanctioned by \Rule{$\TRIPLEBAR$-Exchange}.
But steps (2) and (3) we need new rules (although the replacements involving negations in (3) can be handled with \Rule{QN}).
The most straightforward strategy is to read the needed exchange rules right off the steps. 
Thus, the Prenex Exchange Rules are given in the following chart.
%\begin{table}[!ht]
%\renewcommand{\arraystretch}{1.5}
%\begin{center}
%\begin{tabular}{ p{1in} l l } %p{2.2in} p{2in}
%\toprule
%\textbf{Name} & \textbf{Given} & \textbf{May Add} \\ 
%\midrule
\renewcommand{\arraystretch}{1.5}
\begin{longtable}[c]{ p{1in} l l } %p{2.2in} p{2in}
\toprule
\textbf{Name} & \textbf{Given} & \textbf{May Add} \\ 
\midrule
\endfirsthead
\multicolumn{3}{c}{\emph{Continued from Previous Page}}\\
\toprule
\textbf{Name} & \textbf{Given} & \textbf{May Add} \\ 
\midrule
\endhead
\bottomrule
\caption{Prenex Exchange Short-Cut Rules for \GQD{}}\\[-.15in]
\multicolumn{3}{c}{\emph{Continued next Page}}\\
\endfoot
\bottomrule
\caption{Prenex Exchange Shortcut Rules for \GQD{}}\\
\endlastfoot
\label{GSDplusPrenex}\Rule{$\ALPHA$/$\BETA$-Exch} & $(\#\ALPHA)\CAPPHI$ & $(\#\BETA)\CAPPHI\BETA/\ALPHA$ \\
\Rule{Q Shuffling} & $\parconjunction{(\#\variable{x})\CAPTHETA}{\CAPPSI}$ & $(\#\variable{x})\parconjunction{\CAPTHETA}{\CAPPSI}$ \\
& $\parconjunction{\CAPTHETA}{(\#\variable{x})\CAPPSI}$ & $(\#\variable{x})\parconjunction{\CAPTHETA}{\CAPPSI}$ \\

& $\pardisjunction{(\#\variable{x})\CAPTHETA}{\CAPPSI}$ & $(\#\variable{x})\pardisjunction{\CAPTHETA}{\CAPPSI}$ \\
& $\pardisjunction{\CAPTHETA}{(\#\variable{x})\CAPPSI}$ & $(\#\variable{x})\pardisjunction{\CAPTHETA}{\CAPPSI}$ \\

& $\parhorseshoe{\CAPTHETA}{(\#\variable{x})\CAPPSI}$ & $(\#\variable{x})\parhorseshoe{\CAPTHETA}{\CAPPSI}$ \\

& $\parhorseshoe{\existential{\variable{x}}\CAPTHETA}{\CAPPSI}$ & $\universal{\variable{x}}\parhorseshoe{\CAPTHETA}{\CAPPSI}$ \\
& $\parhorseshoe{\universal{\variable{x}}\CAPTHETA}{\CAPPSI}$ & $\existential{\variable{x}}\parhorseshoe{\CAPTHETA}{\CAPPSI}$ \\
\end{longtable}
%\bottomrule
%\end{tabular}
%\end{center}
%\caption{Exchange Short-Cut Rules for \GSD{} (\GSD{})}
%\label{GSDplus2}
%\end{table}
\noindent{}Now all that's left to show is that anything that can be derived using the Prenex Exchange Rules can be derived using the basic rules of \GQD{} alone.
Recall from section \ref{Shortcut Rule Elimination Theorem Section} that all we need to do to show this is to prove the following:
\begin{THEOREM}{\LnpTC{GQD NDF Rule2}}
For all Prenex Exchange Rules \Rule{R}, any two \GQL{} formulas got by substituting other \GQL{} formulas into the may-add and given schemas of \Rule{R} are provably equivalent.
\end{THEOREM}
\noindent{}We leave the proof of this theorem to the reader, since as with the other exchange rules it just involves writing down the appropriate derivation schemas. 
\end{PROOF}

\subsection{The Strategy for Proving \GQD{} Completeness}
Our goal is to prove the strong completeness of \GQD{}; 
that is, we want to prove that for any set $\Delta$ of \GQL{} sentences and \GQL{} sentence $\CAPPHI$, if $\Delta\sdtstile{}{}\CAPPHI$, then $\Delta\sststile{}{}\CAPPHI$. 
Our strategy will be to first prove the completeness of \GQD{} and then show how to modify the method to prove the strong completeness.
Our strategy for proving  completeness will be to show that for any sentence we can either find a derivation of it or we can construct a model that makes it false. 
(This part of the strategy is more or less the same as what we did to show that \GSD{} is  complete.)
In other words, $\CAPPHI$ is either derivable or not quantificationally true, from which it immediately follows that if $\CAPPHI$ is quantificationally true, then it is derivable. 

The method,\index{method, the} in brief, is to negate the sentence $\CAPPHI$ and begin a derivation.
Then we transform the negation of the sentence into prenex normal form, using the steps outlined in section \ref{Prenex Definition and Steps}. 
Next we transform the inner part of he sentence (remember the quantifiers are all up front) into \CAPS{dnf} form, using our standard method for that (see Sec. \pmvref{Disjunctive Normal Form}).
This will not introduce any new assumptions. 
We then systematically take instances of the bound variables and try to derive a contradiction.
If we can derive a contradiction we can then (assuming all goes well) use \Rule{$\NEGATION$-Elim} to obtain a derivation of $\CAPPHI$.

We must be very systematic since we have to be sure that if we get a contradiction we can derive it from the initial sentence $\CAPPHI$, and that if we do not get a contradiction we have not overlooked anything and that we can show the existence of a model making the sentence on the first line, $\CAPPHI$, true. 

It is important that we know the form of the sentence we have reached and are able to prescribe a uniform systematic method.
The sentence has been highly standardized; 
there are no biconditionals or conditionals (these have been eliminated in early steps of the transformation), negations govern only atomic sentences, and conjunctions govern only atomic sentences or their negations. 
These last, atomic sentence and their negations, are called \idf{ions}. 
We say an ion occurs \niidf{positively} \Iff it's an atomic sentence without a negation, and it occurs \niidf{negatively} \Iff it's a negated atomic sentence. 
We will call the quantifier free part of the original sentence the \idf{matrix}. 
It is usually not a sentence since it may have free variables.
We will call the sentences that are obtained from the matrix by substitution in the process of constructing the derivation \niidf{matrix instances}\index{matrix!instances|textbf}. 
To put some of our jargon together, the matrix of the sentence will consist of disjunctions of conjunctions of ions. 

\subsection{The Method and Completeness Lemmas}\label{The Method Section}
In this section we describe the method sketched above.\index{method, the} 
Say we're given a sentence $\CAPTHETA$. The Method either produces a derivation of $\CAPTHETA$ or indicates a model that makes it false:

\begin{description}
\item[Step 0:] Write $\negation{\CAPTHETA}$ on line 1 as an assumption.
Then first apply the prenex steps to put $\negation{\CAPTHETA}$ in Prenex Normal Form (\CAPS{pnf}). 
Next, apply the disjunctive normal form steps to the inner, quantifier-free part of the sentence until it's in \CAPS{dnf}. 
At this point we'll have a sentence $(\negation{\CAPTHETA})^*$ that's in what we'll call \idf{prenex disjunctive normal form}\index{disjunctive normal form!prenex} (\CAPS{pdnf}).  

\item[Step 1:] We continue the derivation operating on $(\negation{\CAPTHETA})^*$, the \CAPS{pdnf} of the sentence we're concerned with. 
If this \CAPS{pdnf} is a universal statement and contains no constants we write as the next line the instance of it we obtain by eliminating the quantifier and substituting the constant $\constant{a}$ for the previously bound variable;
these steps are sanctioned by \Rule{$\forall$-Elim}.
This step is only done once, whereas the next three steps generally require repeated recursive applications. 

\item[Step 2:] For every universal sentence that appears thus far in the derivation, we add \emph{all new instances} that can be formed with constants that occur earlier in the derivation;
these steps are sanctioned by \Rule{$\forall$-Elim}.
E.g., if $\universal{\variable{x}}\CAPPHI$ appears on a line and the constant $\constant{c}$ appears anywhere (earlier) in the derivation, then if we have not taken an instance of $\CAPPHI$ with $\constant{c}$ yet (i.e., $\CAPPHI\constant{c}/\variable{x}$), we do so.
As a practical matter, this means that it is useful in following the method to keep track somewhere of the constants used at each stage and of which constants have been used to instantiate which universal statements.
Note that in this step we are taking new instances with old constants and that we are not adding any new assumptions. 
We may, however, be adding new existentials. 

\item[Step 3:] For every existential sentence that appears in the derivation for which no instance has been added yet, add an instance using the first constant which \emph{does not occur in any previous assumption}. 
Note that \mention{instance} is to be taken very strictly here. 
The fact that we instantiated $\existential{\variable{x}}\Kpp{\variable{x}}{\constant{a}}$ with $\Kpp{\constant{b}}{\constant{a}}$ takes care of that existential, but if we later add the sentence $\existential{\variable{x}}\Kpp{\variable{x}}{\constant{b}}$ then we must add an instance of it. 
The rule which sanctions these steps will be \Rule{Assumption}. 
We will eventually discharge these premises by \Rule{$\HORSESHOE$-Elim} and \Rule{$\exists$-Elim} if we get to a contradiction.
It is in anticipation of this eventuality that we carefully chose a constant which does not occur in any previous assumption.
Note that with this step we are adding new instances with new constants in new assumptions. 

\item[Step 4:] Determine whether the conjunction of the \emph{instances} of the matrix in the derivation thus far are contradictory. 
Officially, the way to do this is to take the conjunction of them all by \Rule{$\WEDGE\!$-Intro}, use \Rule{Distribution} to get the conjunction into \CAPS{dnf} and check whether every disjunct contains a contradiction. 
If so, then by a process of \Rule{$\VEE$-Elim} and \Rule{Any Contradiction} we can eventually produce the line $\conjunction{\Al}{\negation{\Al}}$. 

\item[Step 5:] \hfill
\begin{cenumerate}
\item If the matrices are contradictory we stop.
\item Or if the conjunction of the matrix instances is consistent and the last applications of Steps 2 and 3 produce no new sentences, we stop.
\item Or if the conjunction of the matrix instance is consistent and the last applications of Steps 2 and 3 produced new sentences, then we return to Step 2 and reapply those steps.
\end{cenumerate}
\end{description}
Thus, there are three possible outcomes of applying this method to a sentence:
\begin{cenumerate}
\item The method might reach a contradiction.
\item The method might stop without a contradiction.
\item The method might generate new sentences perpetually without contradiction.
\end{cenumerate}
We will show first that if a contradiction is reached we can construct a derivation of $\CAPTHETA$. 

\begin{THEOREM}{\LnpTC{Derivational Lemma} Derivational Lemma:}
If the Method starts with $\negation{\CAPTHETA}$ and produces a contradiction, then there is a derivation of $\CAPTHETA$.
\end{THEOREM}
\begin{PROOF}
Step 3 left us with $\conjunction{\Al}{\negation{\Al}}$ on a line with its assumptions being those of the matrices. 
We want to shift those assumptions so that we end up with the contradiction from the first assumption, $\negation{\CAPTHETA}$, alone.
We know by considering our method that other assumptions entered only by Step 2, where we added instances of existentials using new constants. 
We eliminate the last assumption by a \Rule{$\HORSESHOE$-Intro}. 
We know that our last assumption introduced a \emph{new} constant, and therefore we know that that constant did not appear in any earlier assumption or in the existential of which we are taking an instance.
It also (obviously) does not occur in $\conjunction{\Al}{\negation{\Al}}$.
Thus the \Rule{$\exists$-Elim} step in legitimate. 

Thus we can repeat the contradiction $\conjunction{\Al}{\negation{\Al}}$, sanctioning it by \Rule{$\exists$-Elim}. 
We continue this process, repeating $\conjunction{\Al}{\negation{\Al}}$ as often as necessary to shift the dependence back to the assumption on line 1. 
This gives us a derivation of $\conjunction{\Al}{\negation{\Al}}$ from the first assumption, $\negation{\CAPTHETA}$, only. 

We then add two more lines: $\horseshoe{\negation{\CAPTHETA}}{\parconjunction{\Al}{\negation{\Al}}}$, sanctioned by \Rule{$\HORSESHOE$-Intro}, and $\CAPTHETA$, sanctioned by \Rule{$\NEGATION$-Elim}. 
Thus we have a derivation of $\CAPTHETA$ from no assumptions. 
\end{PROOF}

We have shown that if we obtain a contradiction in the derivation process we can derive the original sentence that interests us.


We must now show that if we do not obtain a contradiction (whether or not the method stops), then there is a model that makes $\negation{\CAPTHETA}$ true (and hence makes $\CAPTHETA$ false).

Before giving the rigorous version of the construction of the model, we will present some of the ideas in a more concrete context. 
If we consider a sentence such as $\disjunction{\parconjunction{\Kp{\constant{a}}}{\negation{\Gp{\constant{b}}}}}{\parconjunction{\negation{\Kp{\constant{b}}}}{\Hp{\constant{c}}}}$ we can observe several things. 
First, each disjunct is satisfiable \Iff no ion occurs both positively and negatively in it. 
It is obvious that a conjunction that includes a sentence and its negation cannot be satisfied, but we can show for a conjunction of ions that that is the only way in which it can fail to be satisfiable. 
For example, we can make $\Kp{\constant{a}}$ and $\negation{\Gp{\constant{b}}}$ true by letting $\KK$ be interpreted as the set of even numbers, $\GG$ the set of numbers divisible by $10$ and letting \mention{$\constant{a}$} be assigned $2$ and \mention{$\constant{b}$} be assigned $7$. 

Of course several such sentences taken together produce different results. E.g., as we saw above $\disjunction{\parconjunction{\Kp{\constant{a}}}{\conjunction{\negation{\Kp{\constant{b}}}}{\negation{\Gp{\constant{c}}}}}}{\parconjunction{\Gp{\constant{b}}}{\negation{\Gp{\constant{c}}}}}$ is satisfiable, as is $\disjunction{\parconjunction{\Kp{\constant{b}}}{\conjunction{\negation{\Kp{\constant{c}}}}{\negation{\Gp{\constant{b}}}}}}{\parconjunction{\Gp{\constant{c}}}{\negation{\Gp{\constant{d}}}}}$, but the two together (taken as a conjunction) are not.
The reason is that while each disjunct of the first sentence is self-consistent, it cannot be true simultaneously with either of the disjuncts of the second sentence. If we have a series of disjunctions then they are simultaneously satisfiable only if we can find a way of picking a disjunct from each one in such a way that all the chosen disjuncts can be true together.

This is relevant to the task at hand because we know that all of the non-quantified sentences in our derivation are in \CAPS{dnf} and are thus disjunctions of conjunctions of ions.
We are calling the quantifier free part of the original sentence the matrix.
It is usually not a sentence since it may have free variables. 
The sentences that are obtained from the matrix by substitution in the process of constructing the derivation are the matrix instances. 
We will use the notation $M_{i,j}$ for the disjuncts of the matrix instances, specifically the disjuncts of the first matrix instance will be $M_{1,1},M_{1,2},\ldots,M_{1,m}$.
Thus the first matrix instance is $\disjunction{M_{1,1}}{\disjunction{M_{1,2}}{\disjunction{\ldots}{M_{1,m}}}}$.
The matrix instances that appear in the derivation can be listed in an array:
\begin{center}
\begin{tabular}{ c }
$\disjunction{M_{1,1}}{\disjunction{M_{1,2}}{\disjunction{\ldots}{M_{1,m}}}}$ \\
$\disjunction{M_{2,1}}{\disjunction{M_{2,2}}{\disjunction{\ldots}{M_{2,m}}}}$ \\
\\
\hspace{.5in} $\vdots$ \\
\\
$\disjunction{M_{n,1}}{\disjunction{M_{n,2}}{\disjunction{\ldots}{M_{n,m}}}}$ \\
\end{tabular}
\end{center}
Note that if the the method never stops, then this array will be infinitely long. 

The matrices are jointly consistent \Iff there is a way of picking an $M_{i,j}$ from each matrix instance so that the conjunction of those $M_{i,j}$ contains no atomic sentence and its negation. 
In one direction this is easy to see: if there is no way of choosing a disjunct from each matrix instance that does not end up with an atomic sentence and its negation among the chosen sentences then the set of instances is inconsistent. 

To show that all instances are satisfiable when such a selection can be made without choosing a sentence and its negation will take some proving.
In order to do this we will need to define the \idf{master matrix list} $M$.\index{matrix!master list} 
We will first choose (if there is more than one) a set of disjuncts $M_{i,j}$ (including one from each matrix instance $M_i$) that does not contain any atomic sentence and its negation.
This will be a set of conjunctions of atomic sentences and negations of atomic sentences.
Our master matrix list $M$ simply consists of all these atomic sentences and negated atomic sentences.
Note that since the $M_{i,j}$ selections must be consistent no atomic sentence that appears unnegated also appears negated.
\begin{majorILnc}{\LnpDC{MatrixModel}}
Given a master matrix list $M$, the \nidf{matrix model of $M$}\index{matrix!model} is the model $\IntA_M$ such that:
\begin{cenumerate}
\item The universe of $\IntA_M$ contains one natural number for each constant that appears in $M$, and $\IntA_M(\constant{a})=1$, $\IntA_M(\constant{b})=2$, $\IntA_M(\constant{c})=3$, $\IntA_M(\constant{d})=4$, and so on; $\IntA_M(\variable{t})=1$ for any constant $\variable{t}$ that doesn't appear in $M$. 
\item For each $\integer{m}$-place predicate $\PP$, $\IntA_M(\PP)$ is the set of $\integer{m}$-tuples of natural numbers $\langle\integer{n}_1,\ldots,\integer{n}_\integer{m}\rangle$ such that $\IntA_M(\variable{t}_1)=\integer{n}_1,\ldots,\IntA_M(\variable{t}_\integer{m})=\integer{n}_\integer{m}$ and $\Pp{\variable{t}_1\ldots\variable{t}_\integer{m}}$ appears on the list $M$.
\item Assignments are only made if justified by these principles.
\end{cenumerate}
\end{majorILnc}
A bit more informally, we list the constants that occur on the master matrix list. $\IntA_M$ has a universe that contains as many natural numbers as constants used.
We assign to each constant that occurs on the list the natural number that indicates its place in the order, i.e. $1$ to $\constant{a}$, $2$ to $\constant{b}$, and so on.
%Any constants not occurring on the master matrix list $M$ will be assigned $1$. 
Note that this produces a \idf{census}.
Each $1$-place predicate is assigned the set of numbers associated with the constants such that an instance of the predicate followed by that constant appears on the master matrix list $M$. 
Each $2$-place predicate is assigned the set of pairs of numbers associated with constants such that an instance of the predicate followed by that pair of constants appears on the master matrix list $M$. 
E.g., if $\Kpp{\constant{a}}{\constant{b}}$, $\Kpp{\constant{b}}{\constant{c}}$, and $\Kpp{\constant{d}}{\constant{e}}$ appear on $M$, then $\IntA_M(\KK)$ is assigned $\{\langle1,2\rangle,\langle2,3\rangle,\langle4,5\rangle\}$.
Assignments are made in a similar fashion for $\integer{n}$-placed predicates for $\integer{n}>2$.\footnote{Note 
that if the method never stops, then the master matrix list $M$ will be infinite and we won't actually be able to write down the matrix model $\IntA_M$. 
But this isn't a problem, the matrix model $\IntA_M$ still exists, even if we can't write it down.} 
\begin{THEOREM}{\LnpTC{MethodLemmaA} The Method Lemma 1:}
The matrix model $\IntA_M$ makes true all sentences on the master matrix list $M$.
\end{THEOREM}
\begin{PROOF}
By construction, if an atomic sentence appears on the list we decided to put the relevant pair, triple, or whatever, of numbers in the set assigned to the predicate letter. 
For each negated atomic sentence on the list we know that we would not put the relevant pair, triple, or whatever, in the model of the predicate letter unless the atomic sentence which is being negated also appeared. 
But that never happened because $M$ is consistent by hypothesis.
\end{PROOF}
\begin{THEOREM}{\LnpTC{MethodLemmaB} The Method Lemma 2:}
All matrix instances in the derivation are true in the matrix model $\IntA_M$.
\end{THEOREM}
\begin{PROOF}
By Lemma 1 (Thm. \ref{MethodLemmaA}), all sentences on the master matrix list $M$ are true, and we included all the conjuncts of at least one disjunct $M_{i,j}$ from each matrix instance in forming the master list.
\end{PROOF}
\begin{THEOREM}{\LnpTC{MethodLemmaC} The Method Lemma 3:}
All quantified sentences in the derivation are true in the matrix model $\IntA_M$.
\end{THEOREM}
\begin{PROOF}
We prove this lemma using a recursive proof on the number of quantifiers in each sentence.
\begin{description}
\item[Base Case:] 
The base case is the case of the sentences with $\integer{k}=0$ quantifiers. 
But these sentences are just the matrix instances in the derivation. 
We already proved in The Method Lemma 2 (Thm. \ref{MethodLemmaB}) that all these sentences are true the matrix model $\IntA_M$, so the base case is complete.

\item[Inheritance Step:] \hfill
\begin{description}
\item[Recursive Assumption:]
Our recursive assumption is that all sentences in the derivation with less than $\integer{k}$ quantifiers are true in the matrix model $\IntA_M$.

\item[Existential Quantifier:]
Say $\CAPTHETA$ is a sentence appearing in the derivation of the form $\existential{\ALPHA}\CAPPHI$, where $\CAPPHI$ is a formula with $\integer{k}-1$ quantifiers. 
Step 3 of the method guarantees that the sentence $\CAPPHI\variable{t}/\ALPHA$, for some constant $\variable{t}$, appears somewhere in the derivation. 
This sentence $\CAPPHI\variable{t}/\ALPHA$ has $\integer{k}-1$ quantifiers, so by the recursive assumption it is true in the matrix model $\IntA_M$. 
But then the existentially quantified sentence $\existential{\ALPHA}\CAPPHI$ is true in $\IntA_M$ as well.
(To show this rigorously, consider the $\variable{s}$-variant of $\IntA_M$, $\As{\variable{s}_1}{}$, that assigns the same element of the universe of $\IntA_M$ to $\variable{s}$ as $\IntA_M$ assigns to the constant $\variable{t}$.
Now by the Dragnet Theorem, Thm. \pncmvref{The Dragnet Theorem}, since $\IntA_M$ makes $\CAPPHI\variable{t}/\ALPHA$ true, the sentence $\CAPPHI\variable{s}/\ALPHA$ is true on $\As{\variable{s}_1}{}$.
It follows from this that $\existential{\ALPHA}\CAPPHI$ is true on $\IntA_M$.)

\item[Universal Quantifier:]
Say $\CAPTHETA$ is a sentence appearing in the derivation of the form $\universal{\ALPHA}\CAPPHI$, where $\CAPPHI$ is a formula with $\integer{k}-1$ quantifiers. 
All instances $\CAPPHI\variable{t}/\ALPHA$ which appear in the derivation have $\integer{k}-1$ quantifiers, and so by the recursive hypothesis are true in the matrix model $\IntA_M$. 
If we consider any $\variable{s}$-variant of $\IntA_M$, we know that what it assigns to $\variable{s}$ must be a number from the universe of $\IntA_M$;
we also know from the way that we constructed the matrix model $\IntA_M$ that a number was included in the universe of $\IntA_M$ only if it was assigned to some constant that occurred in the derivation.  
Let what's assigned to $\variable{s}$ by some $\variable{s}$-variant, $\As{\variable{s}_2}{}$, be the number associated with the constant $\constant{c}$.
Because the universal statement $\universal{\ALPHA}\CAPPHI$ occurred in the derivation, we know that we took all instances of it, including $\CAPPHI\constant{c}/\ALPHA$. 
As already stated, all instances $\CAPPHI\variable{t}/\ALPHA$ are true in $\IntA_M$, including $\CAPPHI\constant{c}/\ALPHA$.
By the Dragnet Theorem (Thm. \pncmvref{The Dragnet Theorem}), since $\CAPPHI\constant{c}/\ALPHA$ is true in $\IntA_M$ it follows that $\CAPPHI\variable{s}/\ALPHA$ is true on $\As{\variable{s}_2}{}$. 
But the exact same argument will work for every $\variable{s}$-variant of $\IntA_M$; so $\CAPPHI\variable{s}/\ALPHA$ is true on every $\variable{s}$-variant of $\IntA_M$. 
It follows that $\universal{\ALPHA}\CAPPHI$ is true on the matrix model $\IntA_M$. 
\end{description}

\item[Closure Step:]
Every sentence in the derivation is true in the matrix model $\IntA_M$, which is what was to be shown. 
\end{description}
\end{PROOF}

\subsection{Proving Completeness}
In\index{completeness!weak \GQD{}} this section we put together all the pieces from the last section to prove that \GQD{} is complete. 
\begin{THEOREM}{\LnpTC{MainGQDWCompletenessLemma} Main Weak \GQD{} Completeness Lemma:}
For all sentences $\CAPTHETA$ of \GQL{}, if the method is applied to $\negation{\CAPTHETA}$ then either: (a) the method produces a derivation of $\CAPTHETA$ in \GQDP{}, or (b) there is some model $\IntA$ that makes $\CAPTHETA$ false.
\end{THEOREM}
\begin{PROOF}
If the method is applied to $\negation{\CAPTHETA}$, then either (1) it will produce a contradiction $\conjunction{\Al}{\negation{\Al}}$, (2) the method halts without a contradiction, or (3) the method never halts (and hence never halts in a contradiction). 
If (1), then by the Derivational Lemma (Thm. \pmvref{Derivational Lemma}) there is a derivation of $\CAPTHETA$. 

If either (2) or (3) is the case, then by the Method Lemma 3 (Thm. \pmvref{MethodLemmaC}) we know that all sentences in the derivation starting with $(\negation{\CAPTHETA})^*$, the prenex disjunctive normal form sentence produced in Step 0 of the method from the sentence $\negation{\CAPTHETA}$ on line 1, are true in the matrix model $\IntA_M$. 
Note that all the steps in the derivation of $(\negation{\CAPTHETA})^*$ from $\negation{\CAPTHETA}$ are sanctioned by exchange rules;
therefore those steps can be turned upside down to produce a derivation in \GQDP{} of $\negation{\CAPTHETA}$ from $(\negation{\CAPTHETA})^*$. 
So by theorem \mvref{GQD Shortcut Theorem3} there's a derivation in \GQD{} of $\negation{\CAPTHETA}$ from $(\negation{\CAPTHETA})^*$.
Since \GQD{} is sound (Thm. \pmvref{Soundness of Quantifier Logic}), it follows that $(\negation{\CAPTHETA})^*\sdtstile{}{}\;\negation{\CAPTHETA}$.
Since we know that $(\negation{\CAPTHETA})^*$ is true in the matrix model $\IntA_M$, it follows that $\negation{\CAPTHETA}$ is true in $\IntA_M$ too.
So it follows that $\CAPTHETA$ is false in $\IntA_M$. 
\end{PROOF}
\begin{THEOREM}{\LnpTC{GQDWeakCompletenessTheorem} Weak \GQD{} Completeness Theorem:}
For all sentences $\CAPTHETA$ of \GQL{}, if $\sdtstile{}{}\CAPTHETA$, then $\sststile{}{}\CAPTHETA$ in \GQD{}.
\end{THEOREM}
\begin{PROOF}
Assume that $\sdtstile{}{}\CAPTHETA$. Then there are no models $\IntA$ which makes $\CAPTHETA$ false. 
Thus, if the method is applied to $\negation{\CAPTHETA}$ it can't be that some model $\IntA$ makes $\CAPTHETA$ false. 
By the Main \GQD{} Weak Completeness Lemma (Thm. \ref{MainGQDWCompletenessLemma}), it follows that when the method is applied to $\negation{\CAPTHETA}$ it produces a derivation of $\CAPTHETA$ in \GQDP{}. 
Hence there is a derivation of $\CAPTHETA$ in \GQD{}.
\end{PROOF}
\begin{THEOREM}{\LnpTC{GQDCompletenessTheorem} \GQD{} Completeness Theorem:}
For all finite sets $\Delta$ of \GQL{} sentences and \GQL{} sentence $\CAPPHI$, if $\Delta\sdtstile{}{}\CAPPHI$, then $\Delta\sststile{}{}\CAPPHI$.
\end{THEOREM}
\begin{PROOF}
The theorem follows immediately from the Weak \GQD{} Completeness Theorem and theorem \mvref{RegWeakCompletenessEquiv}.
\end{PROOF}

A consequence of our Strong Method  is that if $\CAPPHI$ is  entailed by an infinite set of sentences $\Delta$, it is entailed by and derivable from  a finite subset of $\Delta$.  If the Strong Method does not go on forever, then we get a contradiction at a finite stage and we have only assumed a finite subset of $\Delta$.

\subsection{Shortcut Rules for the Method}
The method discussed in section \ref{The Method Section} becomes practically unwieldy.
For example, if the matrix has three disjuncts with two sentences each, then combining two instances gives 9 disjuncts with 4 elements each, and combining three gives 27 disjuncts with 8 elements each. 
Thus we will use some additional short cut rules to speed the process of detecting contradictions. 
(But note that \emph{two} of the shortcut Rules we add here are not \emph{exchange} shortcut rules.  Greg's rule is the exception.)

Our first shortcut rule is \Rule{Greg's Rule}. We\index{Greg's Rule} know that if a conjunction contains an atomic formula and the negation of that atomic formula then we can derive the negation of the conjunction.
E.g., we can derive the negation of $\parconjunction{\Kp{\constant{a}}}{\conjunction{\Gp{\constant{b}}}{\conjunction{\Kp{\constant{c}}}{\negation{\Gp{\constant{b}}}}}}$.
So if we have a disjunction, one disjunct of which contains a contradiction of this kind, we can derive he negation of that disjunct and use disjunctive syllogism to prune that disjunct.
For example, given $\disjunction{\parconjunction{\Kp{\constant{a}}}{\conjunction{\Gp{\constant{b}}}{\conjunction{\Kp{\constant{c}}}{\negation{\Gp{\constant{b}}}}}}}{\parconjunction{\Kp{\constant{a}}}{\conjunction{\Gp{\constant{b}}}{\conjunction{\Kp{\constant{c}}}{\negation{\Gp{\constant{a}}}}}}}$ on a line we can derive $\negation{\parconjunction{\Kp{\constant{a}}}{\conjunction{\Gp{\constant{b}}}{\conjunction{\Kp{\constant{c}}}{\negation{\Gp{\constant{b}}}}}}}$ and then $\parconjunction{\Kp{\constant{a}}}{\conjunction{\Gp{\constant{b}}}{\conjunction{\Kp{\constant{c}}}{\negation{\Gp{\constant{a}}}}}}$. 
Greg's Rule lets us accomplish those steps by crossing out the contradictory and writing down the remaining ones.

It\index{$\VEE$/$\WEDGE$-Elim} is helpful to have a short cut rule which combines \Rule{$\WEDGE\!$-Elim} steps with \Rule{$\VEE$-Elim} steps to go from a disjunction of which each disjunct contains a particular sentence to that sentence itself on a later line;
we will call it \Rule{$\VEE$/$\WEDGE\!$-Elim} and it sanctions the step from $\disjunction{\parconjunction{\Kp{\constant{a}}}{\conjunction{\Gp{\constant{b}}}{\conjunction{\Kp{\constant{c}}}{\negation{\Gp{\constant{c}}}}}}}{\parconjunction{\Kp{\constant{a}}}{\conjunction{\Gp{\constant{b}}}{\conjunction{\Kp{\constant{c}}}{\negation{\Gp{\constant{a}}}}}}}$ to $\Gp{\constant{b}}$. 
In addition to citing the justification, if the sentence is at all complex you should circle the repeated subsentence.

Finally,\index{One Bad Apple} given the opposite of even one conjunct in a conjunction, we can derive the negation of the conjunction.
E.g., from $\Gp{\constant{a}}$ we can derive $\negation{\parconjunction{\Kp{\constant{a}}}{\conjunction{\Gp{\constant{b}}}{\conjunction{\Kp{\constant{c}}}{\negation{\Gp{\constant{a}}}}}}}$.
We will call this rule \Rule{One Bad Apple}, or \Rule{OBA}.
%\begin{table}[!ht]
%\renewcommand{\arraystretch}{1.5}
%\begin{center}
%\begin{tabular}{ p{1in} l l } %p{2.2in} p{2in}
%\toprule
%\textbf{Name} & \textbf{Given} & \textbf{May Add} \\ 
%\midrule
\renewcommand{\arraystretch}{1.5}
\begin{longtable}[c]{ p{1in} l l } %p{2.2in} p{2in}
\toprule
\textbf{Name} & \textbf{Given} & \textbf{May Add} \\ 
\midrule
\endfirsthead
\multicolumn{3}{c}{\emph{Continued from Previous Page}}\\
\toprule
\textbf{Name} & \textbf{Given} & \textbf{May Add} \\ 
\midrule
\endhead
\bottomrule
\caption{Short-Cut Rules for the Method}\\[-.15in]
\multicolumn{3}{c}{\emph{Continued next Page}}\\
\endfoot
\bottomrule
\caption{Short-Cut Rules for the Method}\\
\endlastfoot
\label{GSDplusMethod}\Rule{Greg's Rule} & $\disjunction{\CAPPSI_1}{\disjunction{\ldots}{\CAPPSI_{\integer{n}}}}$, where some & $\disjunction{\CAPPSI_1}{\disjunction{\ldots}{\disjunction{\CAPPSI_{\integer{i}-1}}{\disjunction{\CAPPSI_{\integer{i}+1}}{\disjunction{\ldots}{\CAPPSI_{\integer{n}}}}}}}$ \\[-.25cm]
 & $\CAPPSI_{\integer{i}}=\conjunction{\CAPPHI_1}{\conjunction{\ldots}{\conjunction{\CAPPHI_{\integer{j}}}{\ldots}}}$ & \\[-.25cm]
\nopagebreak
 & $\WEDGE\conjunction{\negation{\CAPPHI_{\integer{j}}}}{\conjunction{\ldots}{\CAPPHI_{\integer{m}}}}$ & \\
 
\Rule{$\VEE$/$\WEDGE\!$-Elim} & $\disjunction{\CAPPSI_{1}}{\disjunction{\ldots}{\CAPPSI_{\integer{n}}}}$, where & $\CAPPHI$ \\[-.25cm]
 & each $\CAPPSI_{\integer{i}}$ contains $\CAPPHI$ & \\
 
\Rule{OBA} &  $\conjunction{\CAPPHI_1}{\conjunction{\ldots}{\conjunction{\CAPPHI_i}{\conjunction{\ldots}{\CAPPHI_{\integer{n}}}}}}$, $\negation{\CAPPHI_i}$ & $\negation{\parconjunction{\CAPPHI_1}{\conjunction{\ldots}{\conjunction{\CAPPHI_i}{\conjunction{\ldots}{\CAPPHI_{\integer{n}}}}}}}$ \\
\nopagebreak
 & $\conjunction{\CAPPHI_1}{\conjunction{\ldots}{\conjunction{\negation{\CAPPHI_i}}{\conjunction{\ldots}{\CAPPHI_{\integer{n}}}}}}$, ${\CAPPHI_i}$ & $\negation{\parconjunction{\CAPPHI_1}{\conjunction{\ldots}{\conjunction{\negation{\CAPPHI_i}}{\conjunction{\ldots}{\CAPPHI_{\integer{n}}}}}}}$ \\
\end{longtable}
%\bottomrule
%\end{tabular}
%\end{center}
%\caption{Exchange Short-Cut Rules for \GSD{} (\GSDP{})}
%\label{GSDplus2}
%\end{table}

\section{Strong Completeness and Other Results}\label{Sec:Proving Strong Completeness}
In this section we want to extend our results and show that \GQD{} is strongly complete.
Note that the method we used to extend the Weak Completeness Theorem to the Completeness Theorem will not work here.
To do that, we used theorem \mvref{RegWeakCompletenessEquiv}, the proof of which depending on $\Delta$ being finite.  
To show that \GQD{} is strongly complete, we have modify the method we used to show that it's weakly complete.
\begin{THEOREM}{\LnpTC{GQDStrongCompletenessTheorem} Strong \GQD{} Completeness Theorem:}
For any set $\Delta$ of \GSL{} sentences and any \GSL{} sentence $\CAPPHI$, if $\Delta\sdtstile{}{}\CAPPHI$, then $\Delta\sststile{}{}\CAPPHI$.
\end{THEOREM}
\noindent{}To show that \GQD{} is weakly complete, we gave a method that, given a sentence $\CAPTHETA$, either produces a derivation of $\CAPTHETA$ or produces a model $\IntA$ which makes $\CAPTHETA$ false. 
To show that \GQD{} is strongly complete, what we want is a method that, given a (possibly infinite) set $\Delta$ of sentences and another sentence $\CAPPHI$, either produces a derivation of a contradiction $\conjunction{\Al}{\negation{\Al}}$ from $\negation{\CAPPHI}$ and some finite subset of $\Delta$ or produces a model $\IntA$ that makes $\negation{\CAPPHI}$ and every sentence in $\Delta$ true.

The method we'll give is a modification of the original method given in section \ref{The Method Section}. 
Since it is just a modification of the the original method, we'll only sketch the changes needed. 
We'll call the modified method the \niidf{strong method}\index{strong method, the}\index{method, the!strong}. 
Given some possibly countably infinite set $\Delta$ and sentence $\CAPPHI$, the strong method is:
\begin{description}
\item[Step 1:] Let $\Delta^*=\Delta\cup\{\negation{\CAPPHI}\}$. 
Then pair each sentence of $\Delta^*$ with a natural number and use that to determine the order in which they are assumed. 
The only constraint on this ordering is that $\negation{\CAPPHI}$ should be first.

\item[Step 2:] Put the first sentence of $\Delta^*$ on line 1 and put it in \CAPS{pdnf}, just as was done in Step 0 of the method.

\item[Step 3:] Apply Step 1 of the method.

\item[Step 4:] Apply Steps 2 and 3 of the method to the whole derivation thus far.

\item[Step 5:] Check for contradictions, just as in Step 4 of the method.

\item[Step 6:] \hfill
\begin{cenumerate}
\item If there's a contradiction, stop.
\item If there's no contradiction, write the next sentence of $\Delta^*$ on the next line of the derivation, put that sentence in \CAPS{pdnf}, and go back into Step 4. 
\end{cenumerate}
\end{description}
The strong method will either halt in a contradiction, or not. 
\begin{THEOREM}{\LnpTC{DerivationalLemmaS} Strong Derivational Lemma:}
If the strong method halts in a contradiction, then $\Delta\sststile{}{}\CAPPHI$.
\end{THEOREM}
\begin{PROOF}
If the strong method halts in a contradiction, then it will have produced a derivation of a contradiction $\conjunction{\Al}{\negation{\Al}}$ from $\negation{\CAPPHI}$ and some subset $\Delta'$ of $\Delta$.  We want to show that $\Delta\sststile{}{}\CAPPHI$; to do this, we first want to show that $\Delta'\sststile{}{}\CAPPHI$.

We might think that $\negation{\CAPPHI},\Delta'\sststile{}{}\conjunction{\Al}{\negation{\Al}}$, but in fact, there are additional open assumptions that we made; the Strong Method tells us to make an additional assumption for each line containing an existentially quantified sentence. For these lines, we assume an instance of the existentially quantified sentence.  We want to discharge these assumptions by using \Rule{$\exists$-Elim}, but these assumptions may come prior to some of the assumptions we made from $\Delta'$.  We want to keep the sentences of $\Delta'$ as assumptions, because the contradiction we reached depends on them. 

We can discharge the assumed instances of existentially quantified sentences only by additionally discharging all the assumptions that come later in the derivation.  Accordingly, we want to extend our derivation so that we discharge \emph{all} our assumptions and then repeat all the assumptions \emph{except} for the instances of existentially quantified sentences. 

Let $\CAPTHETA_1, \CAPTHETA_2, \CAPTHETA_3, \ldots, \CAPTHETA_{\integer{n}}$ be the sentences of $\Delta'$ assumed in our derivation.  The last open assumption is either (a) the last sentence of $\Delta'$, i.e., $\CAPTHETA_{\integer{n}}$; (b) an instance of an existentially quantified sentence on an earlier line of the derivation which we'll call $\CAPPSI$; or (c) $\negation{\CAPPHI}$.  If (a) is the case, then discharge the assumption by applying the rule \Rule{$\HORSESHOE$-Intro} to get $\horseshoe{\CAPTHETA_{\integer{n}}}{\parconjunction{\Al}{\negation{\Al}}}$.  If (b) is the case, then first apply \Rule{$\HORSESHOE$-Intro} to get $\horseshoe{\CAPPSI}{\parconjunction{\Al}{\negation{\Al}}}$, and then apply \Rule{$\exists$-Elim} to get $\conjunction{\Al}{\negation{\Al}}$.  When (b) is the case, we know that the assumption introduced a \emph{new} constant, and therefore we know that that constant did not appear in any earlier assumption or in the existential of which we are taking an instance.
It also (obviously) does not occur in $\conjunction{\Al}{\negation{\Al}}$.
Thus the \Rule{$\exists$-Elim} step is legitimate. If (c) is the case then we don't have to worry about instances of existentially quantified sentences, and we can skip this part of the process.  Let us disregard case (c) for now.

Now we must continue to discharge the rest of the assumptions.  For any assumption of an instance of an existentially quantified sentence, we may do the same thing we did in (b) above---first use \Rule{$\HORSESHOE$-Intro} to derive a conditional, and then use \Rule{$\exists$-Elim} to derive the RHS of that conditional.  For any assumption that is a sentence of $\Delta'$ (i.e., $\CAPTHETA_{\integer{i}}$), we will use \Rule{$\HORSESHOE$-Intro} to derive a conditional, as in (a) above.  So, after discharging the assumption with the second to last sentence from $\Delta'$ we get $\horseshoe{\CAPTHETA_{\integer{n-1}}}{\parhorseshoe{\CAPTHETA_{\integer{n}}}{\parconjunction{\Al}{\negation{\Al}}}}$.  After the third instance, we derive $\horseshoe{\CAPTHETA_{\integer{n-2}}}{\parhorseshoe{\CAPTHETA_{\integer{n-1}}}{\parhorseshoe{\CAPTHETA_{\integer{n}}}{\parconjunction{\Al}{\negation{\Al}}}}}$.  And so on, so that we eventually get a sentence of the form $\horseshoe{\CAPTHETA_1}{\parhorseshoe{\CAPTHETA_2}{\parhorseshoe{\CAPTHETA_3}{\ldots \parhorseshoe{\CAPTHETA_{\integer{n}}}{\parconjunction{\Al}{\negation{\Al}}}}}}$.

After discharging all such assumptions, the only open assumption left is $\negation{\CAPPHI}$.  Apply \Rule{$\HORSESHOE$-Intro} once more to get something like the following: $\horseshoe{\negation{\CAPPHI}}{\parhorseshoe{\CAPTHETA_1}{\parhorseshoe{\CAPTHETA_2}{\ldots \parhorseshoe{\CAPTHETA_{\integer{n}}}{\parconjunction{\Al}{\negation{\Al}}}}}}$.  Now we have no open assumptions remaining.  (Note that we have effectively covered case (c) above, since applying \Rule{$\HORSESHOE$-Intro} in this case would give us $\horseshoe{\negation{\CAPPHI}}{\parconjunction{\Al}{\negation{\Al}}}$ and no open assumptions.  In this case, we didn't have to assume any of the sentences of $\Delta$ to get a contradiction, so $\Delta'$ is the empty set.)

At this point we have a conditional, possibly a very long one.  The RHS of the last conditional (possibly embedded in several conditionals) is our contradiction, $\parconjunction{\Al}{\negation{\Al}}$.  We want to show that $\Delta'\sststile{}{}\CAPPHI$, so let us make a series of assumptions from the sentences of $\Delta'$.  That is, let us assume each of $\CAPTHETA_1, \CAPTHETA_2, \CAPTHETA_3, \ldots, \CAPTHETA_{\integer{n}}$.  Now that we've assumed all the sentences of $\Delta'$, let us assume $\negation{\CAPPHI}$.

Given our earlier conditional of the form $\horseshoe{\negation{\CAPPHI}}{\parhorseshoe{\CAPTHETA_1}{\parhorseshoe{\CAPTHETA_2}{\ldots, \parhorseshoe{\CAPTHETA_{\integer{n}}}{\parconjunction{\Al}{\negation{\Al}}}}}}$ and the assumption $\negation{\CAPPHI}$, we may now apply \Rule{$\HORSESHOE$-Elim} to get $\horseshoe{\CAPTHETA_1}{\parhorseshoe{\CAPTHETA_2}{\parhorseshoe{\CAPTHETA_3}{\ldots, \parhorseshoe{\CAPTHETA_{\integer{n}}}{\parconjunction{\Al}{\negation{\Al}}}}}}$.  Because we also have all the sentences of $\Delta'$ as assumptions (i.e., all of $\CAPTHETA_1, \CAPTHETA_2, \CAPTHETA_3, \ldots, \CAPTHETA_{\integer{n}}$), we may apply a series of \Rule{$\HORSESHOE$-Elim} steps until we eventually derive $\conjunction{\Al}{\negation{\Al}}$.  That is, given our earlier assumption $\CAPTHETA_1$ and the conditional $\horseshoe{\CAPTHETA_1}{\parhorseshoe{\CAPTHETA_2}{\parhorseshoe{\CAPTHETA_3}{\ldots, \parhorseshoe{\CAPTHETA_{\integer{n}}}{\parconjunction{\Al}{\negation{\Al}}}}}}$, we may apply \Rule{$\HORSESHOE$-Elim} to derive $\horseshoe{\CAPTHETA_2}{\parhorseshoe{\CAPTHETA_3}{\ldots, \parhorseshoe{\CAPTHETA_{\integer{n}}}{\parconjunction{\Al}{\negation{\Al}}}}}$.  And then because we have $\CAPTHETA_2$ as an assumption we may again apply \Rule{$\HORSESHOE$-Elim} to get $\horseshoe{\CAPTHETA_3}{\ldots, \parhorseshoe{\CAPTHETA_{\integer{n}}}{\parconjunction{\Al}{\negation{\Al}}}}$.  And so on, until we derive $\conjunction{\Al}{\negation{\Al}}$.

Remember that our last open assumption is $\negation{\CAPPHI}$.  We may now discharge that assumption and apply \Rule{$\HORSESHOE$-Intro} to get $\horseshoe{\negation{\CAPPHI}}{\parconjunction{\Al}{\negation{\Al}}}$.  Then we apply \Rule{$\NEGATION$-Elim} to derive $\CAPPHI$.

Now we have as our open assumptions only the sentences of $\Delta'$ and we have derived $\CAPPHI$.  We have thus shown that $\Delta'\sststile{}{}\CAPPHI$.  And because $\Delta'$ is a subset of $\Delta$, it follows that $\Delta\sststile{}{}\CAPPHI$.
\end{PROOF}
\noindent{}Next, note that if the strong method doesn't halt in a contradiction, then we will have a list of matrix instances from which we can construct a matrix model $\IntA_M$ in just the same way we did for the method (Def. \pmvref{MatrixModel}).
Similar to the method, we have the following three theorems.
\begin{THEOREM}{\LnpTC{MethodSLemmaA} The Strong Method Lemma 1:}
The matrix model $\IntA_M$ makes true all sentences on the master matrix list $M$.
\end{THEOREM}
\begin{PROOF}
The same proof used for the method (Thm. \pmvref{MethodLemmaA}) applies here too.
\end{PROOF}
\begin{THEOREM}{\LnpTC{MethodSLemmaB} The Strong Method Lemma 2:}
All matrix instances in the derivation are true in the matrix model $\IntA_M$.
\end{THEOREM}
\begin{PROOF}
The same proof used for the method (Thm. \pmvref{MethodLemmaB}) applies here too.
\end{PROOF}
\begin{THEOREM}{\LnpTC{MethodSLemmaC} The Strong Method Lemma 3:}
All quantified sentences in the derivation are true in the matrix model $\IntA_M$.
\end{THEOREM}
\begin{PROOF}
The same proof used for the method (Thm. \pmvref{MethodLemmaC}) applies here too, so long as we can show that if an existential $\existential{\ALPHA}\CAPPSI$ appears on a line, at least one instance $\CAPPSI\variable{t}/\ALPHA$ does, and if a universal $\universal{\ALPHA}\CAPPSI$ appears on a line, then every instance $\CAPPSI\variable{t}/\ALPHA$ of it with a constant $\variable{t}$ appearing somewhere in the derivation appears somewhere in the derivation. So, all we need to show is that the strong method will derive the appropriate instances of all quantified sentences that appear in our derivation. Let $\Delta'$ be those sentences of $\Delta^*$ that appear in our derivation as a result of the application of the strong method.

\begin{description}
\item[Existential Quantifier:] 
Say $\CAPTHETA$ is some sentence in the derivation of the form $\existential{\ALPHA}\CAPPHI$. 
Step 4 of the strong method uses step 3 of the method on $\CAPTHETA$, which guarantees that the sentence $\CAPPHI\variable{t}/\ALPHA$, for some constant $\variable{t}$, appears somewhere in the derivation.
By hypothesis, step 4 of the strong method is applied to all sentences in $\Delta'$, so we know that it will derive the appropriate instances of all existentially quantified statements in in $\Delta'$.

\item[Universal Quantifier:]
Say $\CAPTHETA$ is a sentence appearing in the derivation of the form $\universal{\ALPHA}\CAPPHI$. 
Step 4 of the strong method uses step 2 of the method on $\CAPTHETA$, which, for every constant $\variable{t}$ in the derivation, guarantees that the sentence $\CAPPHI\variable{t}/\ALPHA$ is derived.
By hypothesis, step 4 of the strong method is applied to all sentences in $\Delta'$, so we know that it will derive the appropriate instances of all universally quantified statements in $\Delta'$.
\end{description}

\noindent{}So, the strong method derives the appropriate instances of all quantified sentences in $\Delta'$.

%(Why? assume not. Then say that universal is PHI and that constant is b. Since the derivation is never ending, there must have been a pass of steps 1 and 2 that comes after both the pass that introduced b and the pass that introduced PHI. So contra assumption, this pass put that instance of PHI in the sequence.).
\end{PROOF}
\noindent{}Finally, we have one last lemma:
\begin{THEOREM}{\LnpTC{MainGQDSCompletenessLemma} Main Strong \GQD{} Completeness Lemma:}
For all sets of \GQL{} sentences $\Delta$ and \GQL{} sentences $\CAPPHI$, if the strong method is applied to $\Delta^*=\Delta\cup\{\negation{\CAPPHI}\}$ then either: (a) the strong method produces a derivation of $\CAPPHI$ from $\Delta$ in \GQDP{}, or (b) there is a model $\IntA$ which makes every sentence in $\Delta$ true and $\CAPPHI$ false.
\end{THEOREM}
\begin{PROOF}
If the method is applied to $\Delta^*=\Delta\cup\{\negation{\CAPPHI}\}$, then either it halts in a contradiction or not. 
By the Strong Derivational Lemma (\pmvref{DerivationalLemmaS}), if the strong method halts in a contradiction, then $\Delta\sststile{}{}\CAPPHI$ in \GQDP{}.

If the method does not halt in a contradiction, then by the Strong Method Lemma 3 (Thm. \pmvref{MethodSLemmaC}) the matrix model $\IntA_M$  makes all the sentences in the derivation true. 
But since the strong method did not halt in a contradiction, for every sentence $\CAPPSI$ in $\Delta$ and $\negation{\CAPPHI}$, there's some sentence in \CAPS{pdnf} that's quantificationally equivalent to $\CAPPSI$ and appears in the derivation. 
So $\IntA_M$ makes all the sentences in $\Delta$ and the sentence $\negation{\CAPPHI}$ true; 
hence $\IntA_M$ makes all the sentences in $\Delta$ true and $\CAPPHI$ false. 
\end{PROOF}
\begin{PROOFOF}{Thm. \ref{GQDStrongCompletenessTheorem}, The Strong Completeness Theorem for GQD}
Assume that $\Delta\sdtstile{}{}\CAPPHI$. 
Then there can be no model $\IntA$ which makes all of the sentences in $\Delta$ true and $\CAPPHI$ false;
so, application of the method can't produce such a model.
Thus by the Main \GQD{} Strong Completeness Lemma (Thm. \ref{MainGQDSCompletenessLemma}), $\Delta\sststile{}{}\CAPPHI$ in \GQDP{}. 
It follows by theorem \mvref{GQD Shortcut Theorem3} that $\Delta\sststile{}{}\CAPPHI$ in \GQD{}.
\end{PROOFOF}

Our Method for proving completeness for QD is short of being a decision procedure.  If $\CAPPHI$ is a logical truth, then The Method will produce a derivation of it.  And if $\CAPPHI$ is not a logical truth, then in many cases it produces a model that makes $\CAPPHI$ false.  But sometimes it just doesn't stop.  We know that if it doesn't stop there is a model that makes the original sentence false, but at each stage we don't know whether it will stop (soon?) or not.   And we know that it can't stop at a finite stage for some sentences because those sentences are only false in an infinite model.

All we know from our work so far is that The Method works as described above. We don't know that there isn't a better method that provides a decision procedure.  Church's Theorem, proved by more advanced methods that involve clarifying what counts as a ``method'' or ``algorithm'' tells us that our result is as good as we can do for all of \GQL{}.

However, we can do better for the language \GQL{}1 and a little more.

%%%%%%%%%%%%%%%%%%%%%%%%%%%%%%%%%%%%%%%%%%%%%%%%%%
\section{Decidability and Church's Theorem}\label{Decidability and Churchs Theorem}
%%%%%%%%%%%%%%%%%%%%%%%%%%%%%%%%%%%%%%%%%%%%%%%%%%

Next\index{decidable}\index{undecidable} we turn to refinements of the method to obtain what are called \niidf{decision procedures} for logical truth in a language \Language{L}. %\index{decision procedure}
We first introduced the idea of a decision procedure in section \mvref{Section:Intro to Decidability}; here we shall fill things out a bit further. 
\begin{majorILnc}{\LnpDC{Def:DecisionProcedure}}
A \df{decision procedure} for logical truth in a language (or sublanguage) \Language{L} is a completely specified method which produces, for any sentence $\CAPPHI$ of \Language{L} and in a finite number of steps, the answer YES if $\CAPPHI$ is a logical truth and the answer NO otherwise.
\end{majorILnc}
\begin{majorILnc}{\LnpEC{TruthTableDecisionProcedure}}
We have already seen two decision procedures for \CAPS{tft} in \GSL{}: truth tables and the method discussed in the completeness proof of \GSD{}. 
(Others include truth trees and Quine's ``fell swoop'', \citealt{Quine1950}, \citealt[23]{Hodges2001}.)
The truth-table decision procedure is simple:\index{decision procedure!truth table} take a sentence $\CAPPHI$ of \GSL{} and construct a truth table for it. If you get all $\TrueB{}$ in the column under $\CAPPHI$; answer YES. If you don't, answer NO. 
Likewise for the method discussed in the completeness proof of \GSD{}:\index{decision procedure!for \CAPS{tft} in \GSL{}} take a sentence $\CAPPHI$, negate it to get $\negation{\CAPPHI}$, and apply the method. 
If it results in a contradiction $\conjunction{\Al}{\negation{\Al}}$, answer YES. 
If no contradiction is reached, answer NO. 
\end{majorILnc}
\noindent{}Our method of proving completeness for \GQD{} is a little short of being a decision procedure for quantificational truth in \GQL{} because it does not always produce an answer in a finite amount of time.
It can be shown that there is no decision procedure for the whole language \GQL{} \citetext{\citealp[83--86]{Hodges2001}, \citeyear[31]{Hodges2001b}, \citealp[486]{Bergmann2003}}.
\begin{THEOREM}{\LnpTC{ChurchsTheorem} Church's Theorem:}
If\index{Church's Theorem}\index{decision procedure!for \CAPS{qt} in \GQL{}|see{Church's Theorem}} \Language{L} is a sublanguage of \GQL{} with (1) the same logical connectives as \GQL{}, and (2) at least one 2-place predicate symbol, then there is no decision procedure for the set of logical truths of \Language{L}.\footnote{Actually, 
Church's Theorem also says that if we also consider languages with function symbols, then if \Language{L} has at least two 1-place function symbols there is no decision procedure for the set of logical truths of \Language{L}.}
\end{THEOREM}
Church's Theorem at once tells us that there is no decision procedure for quantificational truth in \GQL{}, but we can show that with some modifications our method from section \ref{The Method Section} can be turned into a decision procedure for certain sublanguages of \GQL{}. 
As the thesis suggests, one such sublanguage \Language{L} of \GQL{} is the language that consists of just 1-place predicate symbols. 
We've been calling this language \GQL{}1.\index{decision procedure!for \CAPS{qt} in monadic \GQL{}}\index{GQL!monadic}

The basic insight for modifying the method is that infinite loops are created by having an existential quantifier inside a universal quantifier; 
the existential requires a new constant to be instantiated, and that creates a new potential instance for the universal, which then creates a new existential, and so on. 

Put more positively, if the method produces a sentence in standardized form which has only existential, or only universal quantifiers, then the method will stop. 
Moreover, if in the standardized form the existentials all precede the universals the method will also stop, for we will first take instances of all the existentials using $\integer{n}$ constants if there are $\integer{n}$ existentials, and afterwards we will instantiate the $\integer{n}$ new constants in the $\integer{m}$ universals giving $\integer{n}^\integer{m}$ instances.
But, we will be done then since no more new constants will be added. 

The general principle is that if one quantifier occurs within the scope of another then it cannot be moved in front of the other quantifier. 
You may remember that the scope of a quantifier, say $\forall$ in $\universal{\variable{x}}\CAPPHI$, is the subformula $\CAPPHI$ of which it is the main connective. 
We can say two quantifiers are \niidf{independent}\index{quantifier!s, independent} if neither is in the scope of the other.
With this terminology, we can notice that for any sentence all of whose quantifiers are independent of each other that they can be brought forward in any order, and thus we have a decision procedure for such sentences. 

The critical result to prove is that every sentence of \GQL{}1 is equivalent to a sentence whose quantifiers are independent. 
We can show this by appeal to the reverse of our procedure for putting sentences into prenex form, i.e. by moving quantifiers inward, but we use most of the same rules as for standard prenex (remember they are exchange rules).
\begin{THEOREM}{\LnpTC{MonadicGQLEquivTheorem}  \GQL{}1 Equivalence Theorem:}
Every sentence of \GQL{}1 is quantificationally equivalent to a sentence whose quantifiers are independent.
\end{THEOREM}
\begin{PROOF}
Proof here.
\end{PROOF}
\begin{THEOREM}{\LnpTC{MonadicDecisionTheorem} The \GQL{}1 Decision Theorem:}
The\index{Monadic Decision Theorem, The} modified method just described provides a decision procedure for quantificational truth in \GQL{}1.
\end{THEOREM}

%%%%%%%%%%%%%%%%%%%%%%%%%%%%%%%%%%%%%%%%%%%%%%%%%%
\section{L\"owenheim-Skolem and Compactness}
%%%%%%%%%%%%%%%%%%%%%%%%%%%%%%%%%%%%%%%%%%%%%%%%%%

A number of results follow directly from the completeness of \GQD{} or the method we used to prove completeness. 
\begin{THEOREM}{\LnpTC{LowenheimSkolemTheorem} The Downward L\"owenheim-Skolem Theorem:}
If a sentence of \GQL{} is true in any model, then it is true in one whose domain consists of all or some of the natural numbers.
\end{THEOREM}
\begin{PROOF}
If $\CAPPHI$ is true in some model, then $\negation{\CAPPHI}$ is not a quantificational truth.
Thus, applying the method to $\negation{\CAPPHI}$ will not produce a contradiction, but will produce a model of the natural numbers which falsifies $\negation{\CAPPHI}$ and hence makes $\CAPPHI$ true.
\end{PROOF}
It's important to note that there's really nothing special about the natural numbers.
When we devised the procedure for constructing a model of the ions in the master matrix list that results from the method (when no contradiction arises), we choose to use natural numbers for the universe. 
But it should be clear that we did this out of convenience (it's easy, after all, to associate constants with the natural numbers). 
We could have used any set of objects for the universe. 
What's important is that, whatever we used, the domain of the constructed model will at most be countably infinite (that is, it will at most be the size of the natural numbers and no larger). 
Hence a more abstract version of the downward Lowenheim-Skolem Theorem simply says: If a sentence of \GQL{} is true in any model, then either it's true in only models with finite domains, or, if it's true at all in models with infinite domains, then there's an model with a \emph{countably} infinite domain in which it's true. 

This thorem was proved in a weaker form originally by Leopold L\"owenheim \citeyearpar{Lowenheim1915}, and the proof was improved by Thoralf Skolem \citeyearpar{Skolem1920,Skolem1922}. 
Notice that the theorem talks only about models, and we have proved it via a detour through derivations. 
As you might imagine, there are more direct proofs, including Skolem's \citetext{\citealp{Tarski1956}, \citealp{Vaught1974}, \citealp[ch.~3.1]{Hodges1997}, \citeyear[63]{Hodges2001}}.

Notice also that after our work on \GQL{}1, we know that for monadic sentences the method will stop after a finite number of steps (if we arrange the prenex carefully) and so we can conclude that if a monadic sentence is true in any model then it is true in a finite one. 
\begin{THEOREM}{\LnpTC{MonadicIntSizeTheorem}}
If $\CAPPHI$ is a sentence of \GQL{}1 and has a model, then it has a finite model.
\end{THEOREM}

Our next corollary of completeness is the Compactness Theorem.
Although historically the completeness theorem was proved first and compactness followed as a corollary, today the compactness theorem takes center stage in many areas of logic (especially model theory). 
Like the L\"owenheim-Skolem Theorem, there are many different proofs of compactness that do not go through completeness or use any facts about derivations \citetext{see \citealt[321]{Kleene1967}, \citealt{Ebbinghaus1985}, \citealt[ch.~5.1]{Hodges1997}, \citealp[63]{Hodges2001}, \citeyear[29]{Hodges2001b}}. 
%answers the question of whether it's possible that an infinite set of sentences is intuitively contradictory but we cannot deduce the contradiction in our system because our derivations are finite?
%This is a specific version of the more general worry that if $\Delta$ is an infinite set then it might be that $\Delta\sdtstile{}{}\CAPPHI$ but not $\Delta\sststile{}{}\CAPPHI$. 
%We can prove that this doe not occur in our language by proving the following theorem.
\begin{THEOREM}{\LnpTC{Thm:CompactnessTheorem} The Compactness Theorem for \GQL{}:}
For all sets of sentences $\Delta$ of \GQL{}, if for every finite subset $\Delta'$ of $\Delta$ there exists a model $\IntA'$ that makes all the sentences in $\Delta'$ true, then there's some model $\IntA$ that makes all the sentences in $\Delta$ true. 
\end{THEOREM}
\begin{PROOF}
By the strong completeness theorem, for all sets $\Delta$ of \GQL{} sentences and \GQL{} sentence $\CAPPHI$, if $\Delta\sdtstile{}{}\CAPPHI$, then $\Delta\sststile{}{}\CAPPHI$.
Now assume that there's no model $\IntA$ that makes all the sentences in $\Delta$ true. 
Hence $\Delta\sdtstile{}{}\conjunction{\Al}{\negation{\Al}}$.
So by strong completeness, $\Delta\sststile{}{}\conjunction{\Al}{\negation{\Al}}$.
By definition, this implies that there's some finite subset $\Delta'$ of $\Delta$ such that $\conjunction{\Al}{\negation{\Al}}$ can be derived from $\Delta'$. 
Hence there is no model $\IntA$ that makes all the sentences in $\Delta'$ true. 
Hence it's not the case that for every finite subset $\Delta'$ of $\Delta$ there exists a model $\IntA'$ that makes all the sentences in $\Delta'$ true. 
\end{PROOF}

%%%%%%%%%%%%%%%%%%%%%%%%%%%%%%%%%%%%%%%%%%%%%%%%%%
\section{Exercises}
%%%%%%%%%%%%%%%%%%%%%%%%%%%%%%%%%%%%%%%%%%%%%%%%%%

\notocsubsection{Misc. Problems}{Misc Problems} 
\begin{enumerate}
\item Let's say that any derivation rule \Rule{R} that has the following property is \niidf{sound}:\index{derivation!rule!sound} if we add a line to a derivation $\Derivation{D}$ with sentence $\CAPPHI$ sanctioned by rule \Rule{R}, then $\Delta\sdtstile{}{}\CAPPHI$, where $\Delta$ is the set of unboxed assumptions for the new line. 
(Compare this with what it is for a rule to be truth-preserving, def. \pncmvref{Derivation Rule Soundness}, which is different.)
Then the proof of theorem \mvref{Main GSL Soundness Lemma} basically shows that \GSD{} is sound by showing that all the basic rules of \GSD{} are sound. 
We know that \GSDP{} is sound because \GSD{} is sound and (by theorem \pmvref{GSD Shortcut Theorem3}) anything you can derive in \GSDP{} can be derived in \GSD{}. 
But we could also show that \GSDP{} is sound directly (without appealing to theorem \ref{GSD Shortcut Theorem3}) by showing that the shortcut rules used in \GSDP{} themselves are sound. 
Of course, this follows from theorem \pmvref{GSD Shortcut Theorem2} and the fact that the basic rules are sound, but again we can show it directly. 
But again we can show it without going through the basic rules.
Show directly (without appealing to theorem \ref{GSD Shortcut Theorem2}) that the following rules are sound (see tables \pmvref{GSDplus1} and \pmvref{GSDplus2}): 
\begin{multicols}{2}
\begin{enumerate}
\item \Rule{M.T.}
\item \Rule{A.C.}
\item \Rule{$\HORSESHOE$/$\VEE$-Exch.}
\item \Rule{Contraposition}
\end{enumerate}
\end{multicols} 
\item Recall that $\HORSESHOE$ elimination can only be used on a conditional that is the main connective of a sentence. Show that if we do not make this restriction, then the rule is unsound. In other words, give a derivation which violates only that restriction (a derivation where you use $\HORSESHOE$ elimination on a horseshoe that's not the main connective) and which ends with a proof of a sentence that is \emph{not} a logical truth (not truth-functionally true) from the empty set of assumptions.
\end{enumerate}

%\theendnotes


%%%%%%%%%%%%%%%%%%%%%%%%%%%%%%%%%%%%%%%%%%%%%%%%%%
\chapter{Soundness and Completeness}\label{completenesschapter}
%%%%%%%%%%%%%%%%%%%%%%%%%%%%%%%%%%%%%%%%%%%%%%%%%%

%%%%%%%%%%%%%%%%%%%%%%%%%%%%%%%%%%%%%%%%%%%%%%%%%%
\section{Introduction}
%%%%%%%%%%%%%%%%%%%%%%%%%%%%%%%%%%%%%%%%%%%%%%%%%%

In this chapter we establish various connections between the derivation rules of the systems defined in the last two chapters with the semantic notions of truth, logical truth, and entailment. 
We carefully defined the rules of \GSD{} and \GQD{} so that they are \emph{truth-preserving}.

\begin{majorILnc}{\LnpDC{Derivation Rule Soundness}}
	A rule is \df{truth-preserving}\index{derivation!rule!truth-preserving}\index{truth-preserving} \Iff the sentence or sentences to which the rule is applied entail any sentence which the rule sanctions you to write as the next step. 
\end{majorILnc}

To prove this we start with the following theorem:
\begin{THEOREM}{\LnpTC{Soundess of Basic GSD Rules}}
	Every application of every basic rule of \GSD{} is truth-preserving.
\end{THEOREM}
\begin{PROOF}
	It can be shown that: for any basic rule \Rule{R} of \GSD{}, if some substitution of \GSL{} sentences into the given schema of \Rule{R} results in \GSL{} sentences $\CAPPSI_1,\ldots,\CAPPSI_{\integer{n}}$ and that same substitution into the may-add schema of \Rule{R} results in the \GSL{} sentence $\CAPPHI$, then $\CAPPSI_1,\ldots,\CAPPSI_{\integer{n}}\sdtstile{}{}\CAPPHI$.
	Call this the truth-preservation lemma.\footnote{
	    See exercise \pmvref{exercises:truth-preservation lemma}.
	}
	Now consider some arbitrary application of some basic rule \Rule{R} of \GSD{}. 
	Say that in this application \Rule{R} is applied to sentences $\CAPTHETA_1,\ldots,\CAPTHETA_{\integer{m}}$ and permits, or sanctions, you to write down $\DELTA$. 
	By definition \mvref{RuleSanctioning}, there's some substitution of \GSL{} sentences that, for the given schemas of \Rule{R}, results in $\CAPTHETA_1,\ldots,\CAPTHETA_{\integer{m}}$ and, for the may-add schema, results in $\DELTA$. 
	By the truth preservation lemma, $\CAPTHETA_1,\ldots,\CAPTHETA_{\integer{m}}\sdtstile{}{}\DELTA$.
	By definition \mvref{Derivation Rule Soundness}, this application is truth-preserving. 
\end{PROOF}
\begin{commentary}
	This proof doesn't cover \Rule{$\HORSESHOE$-Intro} or \Rule{Assume}.  These rules require special treatment.  The former is the only rule that eliminates an assumption, and the latter is the only rule that adds an assumption, so they each have a unique role.
\end{commentary}

Recall from section \mvref{Derivation Preliminaries} that we want to use derivations as a way of showing that a sentence is a logical truth, or of showing that some set of sentences entails some other sentence.
Specifically, we want to use derivations in \GSD{} and \GQD{} to show that sentences of \GSL{} and \GQL{} are \CAPS{tft} and \CAPS{qt}, or to show entailments between sentences of \GSL{} or between sentences of \GQL{}.  
But derivations can only fill this role if the derivation system in question is sound.
Let \Language{L} be some formal language for which we have defined some kind of models.
\begin{majorILnc}{\LnpDC{LSoundness}}
A derivation system \DerivationSystem{D} for \Language{L} is \nidf{sound}\index{soundness|textbf} \Iff for every set $\Delta$ of sentences of \Language{L} and every sentence $\CAPPHI$ of \Language{L}, if $\Delta\sststile{}{}\CAPPHI$, then $\Delta\sdtstile{}{}\CAPPHI$.
\end{majorILnc} 


%%%%%%%%%%%%%%%%%%%%%%%%%%%%%%%%%%%%%%%%%%%%%%%%%%
\section{Soundness}
%%%%%%%%%%%%%%%%%%%%%%%%%%%%%%%%%%%%%%%%%%%%%%%%%%

\subsection{Soundness of \GSD{}}
We prove the soundness of \GSD{} by way of a lemma.\index{soundness!of \GSD{}}
\begin{THEOREM}{\LnpTC{Soundness of Sentential Logic} \GSD{} Soundness Theorem:}
\GSD{} is sound; i.e., for every set $\Delta$ of sentences of \GSL{} and every sentence $\CAPPHI$ of \GSL{}, if $\Delta\sststile{}{}\CAPPHI$ in \GSD{}, then $\Delta\sdtstile{}{}\CAPPHI$.
\end{THEOREM}
\begin{THEOREM}{\LnpTC{Main GSL Soundness Lemma} Soundness Lemma:}
For any sequence of derivation lines that is a derivation, the sentence $\CAPPHI$ on the last line is entailed by the set $\Delta$ of sentences that are on unboxed lines and are sanctioned by \Rule{Assume}.
\end{THEOREM}
\noindent{}Since the definition of a derivation is a recursive definition\footnote{See definition \pmvref{Recursive definition of Derivation}.}, the most natural way to prove theorem \ref{Main GSL Soundness Lemma} is through a recursive proof. 
The recursive proof uses three easily proved lemmas (proofs are left to the reader; the first lemma, on monotonicity, is also used to prove thm. \ref{Soundness of Sentential Logic}). 
Two are facts about entailment and one is about derivation.
\begin{THEOREM}{\LnpTC{Monotonicity of Entailment} Monotonicity of Entailment:}
For all \GSL{} sentences $\CAPPHI_1,\ldots,\CAPPHI_{\integer{n}},\CAPTHETA,\CAPPSI$:
\begin{center}
If $\CAPPHI_1,\CAPPHI_2,\ldots,\CAPPHI_{\integer{n}}\sdtstile{}{}\CAPPSI$, then $\CAPPHI_1,\CAPPHI_2,\ldots,\CAPPHI_{\integer{n}},\CAPTHETA\sdtstile{}{}\CAPPSI$
\end{center}
\end{THEOREM}
\begin{THEOREM}{\LnpTC{Transitivity of Entailment} Transitivity of Entailment:}
For all \GSL{} sentences $\CAPPHI_1,\ldots,\CAPPHI_{\integer{n}}$, $\CAPTHETA$, and $\CAPPSI_1,\ldots,\CAPPSI_{\integer{k}}$:
\begin{center}
\begin{tabular}{ l@{\hspace{.25em}}l@{\hspace{.25em}}l }
If & $\CAPPHI_1,\CAPPHI_2,\ldots,\CAPPHI_{\integer{n}}\sdtstile{}{}\CAPPSI_1$ & and \\
   & $\CAPPHI_1,\CAPPHI_2,\ldots,\CAPPHI_{\integer{n}}\sdtstile{}{}\CAPPSI_2$ & and \\
   & \hspace{.5in} $\vdots$ &  \\
   & $\CAPPHI_1,\CAPPHI_2,\ldots,\CAPPHI_{\integer{n}}\sdtstile{}{}\CAPPSI_{\integer{k}}$ & and \\
   & $\CAPPSI_1,\CAPPSI_2,\ldots,\CAPPSI_{\integer{k}}\sdtstile{}{}\CAPTHETA$ & then: \\
   & & $\CAPPHI_1,\CAPPHI_2,\ldots,\CAPPHI_{\integer{n}}\sdtstile{}{}\CAPTHETA$   \\
\end{tabular}
\end{center}
\end{THEOREM}
\begin{THEOREM}{\LnpTC{Non-decreasing Assumption Principle} Non-decreasing Assumption Principle (NDAP):}
If $\Delta_1$ is the set of assumptions of an unboxed line and $\Delta_2$ is the set of assumptions of a later unboxed line, then $\Delta_1$ is a subset of $\Delta_2$, i.e., $\Delta_1\subseteq\Delta_2$.
\end{THEOREM}
\begin{PROOFOF}{Thm. \ref{Main GSL Soundness Lemma}, Soundness Lemma}
\begin{description}

\item[Base Step:] 
Consider a single-line derivation $\Derivation{D}$ of the sentence $\CAPPHI$ sanctioned by the rule \Rule{Assume}. 
Since the sentence on the last line is $\CAPPHI$, sanctioned by \Rule{Assume}, and $\CAPPHI\sdtstile{}{}\CAPPHI$, it follows that the set of all unboxed sentences sanctioned by \Rule{Assume} entails $\CAPPHI$. 

\item[Inheritance Step:]
Let $\Derivation{D}$ be an arbitrary derivation consisting of $k$ lines, where $k>1$.
Let $\Delta_\integer{i}$ be the set of unboxed assumptions occurring in a derivation $\Derivation{D}$ up to (and including) line number $\integer{i}$.
Consider the case in which a new line with sentence $\CAPPHI$ is added to derivation $\Derivation{D}$, sanctioned by rule \Rule{R}.

\begin{commentary}
	The resulting derivation has $k+1$ lines.
	We want to show, for each rule \Rule{R} of \GSD{}, that $\Delta_{k+1}\sdtstile{}{}\CAPPHI$.
	Each rule is considered separately in the following.
\end{commentary}

\begin{description}

\item[Recursive Assumption:]
Assume for each line $i$ of derivation $\Derivation{D}$, where $i\leq k$ and $\CAPPHI_{i}$ is the sentence on line $i$, that $\Delta_{\integer{i}}\sdtstile{}{}\CAPPHI_{i}$.

\item[\Rule{Assume}:] 
Let line $k+1$ of $\Derivation{D}$ be sanctioned by \Rule{Assume}.
Note that the set $\Delta_{k+1}$ of unboxed assumptions are those in $\Delta_{k}$ plus $\CAPPHI$. 
Clearly $\CAPPHI\sdtstile{}{}\CAPPHI$.
Then $\Delta,\CAPPHI\sdtstile{}{}\CAPPHI$ follows by monotonicity.

\item[\Rule{Repetition}:] 
Let line $k+1$ of $\Derivation{D}$ be sanctioned by \Rule{Repetition}.
Since $\CAPPHI$ is sanctioned at line $k+1$, then it must already occur on some line $i$, where $i<k+1$.
By the recursive assumption $\Delta_{\integer{i}}\sdtstile{}{}\CAPPHI$.
By NDAP, $\Delta_{\integer{i}}\subseteq\Delta_{k+1}$. 
So by monotonicity, $\Delta_{k+1}\sdtstile{}{}\CAPPHI$.

\item[\Rule{$\VEE$-Intro}:]
Let line $k+1$ of $\Derivation{D}$ be sanctioned by \Rule{$\VEE$-Intro}.
Then $\CAPPHI$ must be a disjunction with some disjunct, $\CAPTHETA$, that is already present on some line $i$, where $i<k+1$.
$\Delta_{\integer{i}}$ is the set of unboxed assumptions at that line, and by NDAP $\Delta_{\integer{i}}\subseteq\Delta_{k+1}$.
By the recursive assumption, $\Delta_{\integer{i}}\sdtstile{}{}\CAPTHETA$.
So by monotonicity, $\Delta_{k+1}\sdtstile{}{}\CAPTHETA$.
Since \Rule{$\VEE$-Intro} is truth preserving, $\CAPTHETA\sdtstile{}{}\CAPPHI$. 
So by transitivity of entailment, $\Delta_{k+1}\sdtstile{}{}\CAPPHI$. 

\item[\Rule{$\WEDGE\!$-Elim}:]
Let line $k+1$ of $\Derivation{D}$ be sanctioned by \Rule{$\WEDGE\!$-Elim}.
Then there's some earlier line $\integer{i}$ with the sentence $\conjunction{\CAPTHETA_\integer{1}}{\conjunction{\ldots}{\CAPTHETA_{\integer{m}}}}$ and $\CAPPHI$ is one of the conjuncts. 
As before, by NDAP $\Delta_{\integer{i}}\subseteq\Delta_{k+1}$.
By the recursive assumption, $\Delta_{\integer{i}}\sdtstile{}{}\conjunction{\CAPTHETA_\integer{1}}{\conjunction{\ldots}{\CAPTHETA_{\integer{m}}}}$. 
So by monotonicity, $\Delta_{k+1}\sdtstile{}{}\conjunction{\CAPTHETA_\integer{1}}{\conjunction{\ldots}{\CAPTHETA_{\integer{m}}}}$.
And $\CAPPHI$ is one of the conjuncts of $\conjunction{\CAPTHETA_\integer{1}}{\conjunction{\ldots}{\CAPTHETA_{\integer{m}}}}$, so it follows that $\conjunction{\CAPTHETA_\integer{1}}{\conjunction{\ldots}{\CAPTHETA_{\integer{m}}}}\sdtstile{}{}\CAPPHI$. 
So by transitivity, $\Delta_{k+1}\sdtstile{}{}\CAPPHI$.

\item[\Rule{$\NEGATION$-Elim}:] 
Let line $k+1$ of $\Derivation{D}$ be sanctioned by \Rule{$\NEGATION$-Elim}.
Then there's some earlier line $\integer{i}$ with the sentence $\horseshoe{\negation{\CAPPHI}}{\parconjunction{\CAPPSI}{\negation{\CAPPSI}}}$.
As before, by NDAP $\Delta_{\integer{i}}\subseteq\Delta_{k+1}$.
By the recursive assumption, $\Delta_{\integer{i}}\sdtstile{}{}\horseshoe{\negation{\CAPPHI}}{\parconjunction{\CAPPSI}{\negation{\CAPPSI}}}$.
So by monotonicity, $\Delta_{k+1}\sdtstile{}{}\horseshoe{\negation{\CAPPHI}}{\parconjunction{\CAPPSI}{\negation{\CAPPSI}}}$.
Since the \CAPS{rhs} of $\horseshoe{\negation{\CAPPHI}}{\parconjunction{\CAPPSI}{\negation{\CAPPSI}}}$ is false in all models (it's \CAPS{tff}), the conditional is true in a model $\IntA{}$ only if the \CAPS{lhs} is false in $\IntA{}$.
So if the conditional is true in a model $\IntA{}$, $\CAPPHI$ is true in $\IntA{}$.
In other words, $\horseshoe{\negation{\CAPPHI}}{\parconjunction{\CAPPSI}{\negation{\CAPPSI}}}\sdtstile{}{}\CAPPHI$. 
So by transitivity, $\Delta_{k+1}\sdtstile{}{}\CAPPHI$.

\item[\Rule{$\NEGATION$-Intro}:] 
This case is similar to the previous one and is left as an exercise for the reader. 

\item[\Rule{$\HORSESHOE$-Elim}:]
Let line $k+1$ of $\Derivation{D}$ be sanctioned by \Rule{$\HORSESHOE$-Elim}.
Then there are two earlier lines $\integer{i}$ and $\integer{j}$, and (say) line $\integer{i}$ has a sentence $\horseshoe{\CAPTHETA}{\CAPPHI}$ and line $\integer{j}$ has sentence $\CAPTHETA$. 
By NDAP we have that $\Delta_{\integer{i}}\subseteq\Delta_{k+1}$ and $\Delta_{\integer{j}}\subseteq\Delta_{k+1}$.
By the recursive assumption, $\Delta_{\integer{i}}\sdtstile{}{}\horseshoe{\CAPTHETA}{\CAPPHI}$ and $\Delta_{\integer{j}}\sdtstile{}{}\CAPTHETA$.
By monotonicity, 
$\Delta_{k+1}\sdtstile{}{}\horseshoe{\CAPTHETA}{\CAPPHI}$ and $\Delta_{k+1}\sdtstile{}{}\CAPTHETA$.
Since the rule is truth preserving, $\CAPTHETA,\horseshoe{\CAPTHETA}{\CAPPHI}\sdtstile{}{}\CAPPHI$.
So by transitivity, $\Delta_{k+1}\sdtstile{}{}\CAPPHI$.

\item[\Rule{$\TRIPLEBAR$-Elim}:] The argument for each of the two versions of \Rule{$\TRIPLEBAR$-Elim} is the same as that for \Rule{$\HORSESHOE$-Elim}.

\item[\Rule{$\TRIPLEBAR$-Intro}:]
Let line $k+1$ of $\Derivation{D}$ with sentence $\CAPPHI=\triplebar{\CAPPHI}{\CAPTHETA}$ be sanctioned by \Rule{$\TRIPLEBAR$-Intro}.
Then there are two earlier lines $\integer{i}$ and $\integer{j}$, and (say) line $\integer{i}$ has a sentence $\horseshoe{\CAPTHETA}{\CAPPHI}$ and line $\integer{j}$ has sentence $\horseshoe{\CAPPHI}{\CAPTHETA}$. 
By NDAP we have that $\Delta_{\integer{i}}\subseteq\Delta_{k+1}$ and $\Delta_{\integer{j}}\subseteq\Delta_{k+1}$.
By the recursive assumption, $\Delta_{\integer{i}}\sdtstile{}{}\horseshoe{\CAPTHETA}{\CAPPHI}$ and $\Delta_{\integer{j}}\sdtstile{}{}\horseshoe{\CAPPHI}{\CAPTHETA}$.
By monotonicity, $\Delta_{k+1}\sdtstile{}{}\horseshoe{\CAPTHETA}{\CAPPHI}$ and $\Delta_{k+1}\sdtstile{}{}\horseshoe{\CAPPHI}{\CAPTHETA}$.
Since the rule is truth preserving, $\horseshoe{\CAPPHI}{\CAPTHETA},\horseshoe{\CAPTHETA}{\CAPPHI}\sdtstile{}{}\triplebar{\CAPPHI}{\CAPTHETA}$.
So by transitivity, $\Delta_{k+1}\sdtstile{}{}\CAPPHI$.

\item[\Rule{$\WEDGE\!$-Intro}:]
Let line $k+1$ of $\Derivation{D}$ with sentence $\CAPPHI=\conjunction{\CAPPHI_1}{\conjunction{\ldots}{\CAPPHI_{\integer{m}}}}$ be sanctioned by \Rule{$\WEDGE\!$-Intro}.
Then there are $\integer{m}$ earlier lines numbered $\integer{i}_{1},\ldots,\integer{i}_{\integer{m}}$ with, respectively, sentences $\CAPPHI_1,\ldots,\CAPPHI_{\integer{m}}$. 
By NDAP we have that $\Delta_{\integer{i}_1}\subseteq\Delta_{k+1},\ldots,\Delta_{\integer{i}_\integer{m}}\subseteq\Delta_{k+1}$.
By the recursive assumption, $\Delta_{\integer{i}_1}\sdtstile{}{}\CAPPHI_1,\ldots,\Delta_{\integer{i}_\integer{m}}\sdtstile{}{}\CAPPHI_{\integer{m}}$.
So by monotonicity, $\Delta_{k+1}\sdtstile{}{}\CAPPHI_1,\ldots,\Delta_{k+1}\sdtstile{}{}\CAPPHI_{\integer{m}}$.
Observe that $\CAPPHI_1,\ldots,\CAPPHI_{\integer{m}}\sdtstile{}{}\conjunction{\CAPPHI}{\conjunction{\ldots}{\CAPPHI_{\integer{m}}}}$.
So by transitivity, $\Delta_{k+1}\sdtstile{}{}\CAPPHI$.

\item[\Rule{$\VEE$-Elim}:]
Let line $k+1$ of $\Derivation{D}$ be sanctioned by \Rule{$\VEE$-Elim}.
Then there are $\integer{m}+1$ earlier lines numbered $\integer{i}_{1},\ldots,\integer{i}_{\integer{m}},\integer{i}_{\integer{m}+1}$ with, respectively, sentences $\horseshoe{\CAPTHETA_1}{\CAPPHI}$, $\ldots$, $\horseshoe{\CAPTHETA_{\integer{m}}}{\CAPPHI}$, and  $\disjunction{\CAPTHETA_1}{\disjunction{\ldots}{\CAPTHETA_{\integer{m}}}}$.
By NDAP we have that $\Delta_{\integer{i}_1}\subseteq\Delta_{k+1},\ldots,\Delta_{\integer{i}_\integer{m}}\subseteq\Delta_{k+1}$ and $\Delta_{\integer{i}_{\integer{m}+1}}\subseteq\Delta_{k+1}$.
By the recursive assumption, $\Delta_{\integer{i}_1}\sdtstile{}{}\horseshoe{\CAPTHETA_1}{\CAPPHI}$, $\ldots$, $\Delta_{\integer{i}_\integer{m}}\sdtstile{}{}\horseshoe{\CAPTHETA_{\integer{m}}}{\CAPPHI}$ and $\Delta_{\integer{i}_{\integer{m}+1}}\sdtstile{}{}\disjunction{\CAPTHETA_1}{\disjunction{\ldots}{\CAPTHETA_{\integer{m}}}}$.
By monotonicity, $\Delta_{k+1}\sdtstile{}{}\horseshoe{\CAPTHETA_1}{\CAPPHI}$, $\ldots$, $\Delta_{k+1}\sdtstile{}{}\horseshoe{\CAPTHETA_{\integer{m}}}{\CAPPHI}$ and $\Delta_{k+1}\sdtstile{}{}\disjunction{\CAPTHETA_1}{\disjunction{\ldots}{\CAPTHETA_{\integer{m}}}}$.
Observe that $\horseshoe{\CAPTHETA_1}{\CAPPHI},\ldots,\horseshoe{\CAPTHETA_{\integer{m}}}{\CAPPHI},\disjunction{\CAPTHETA_1}{\disjunction{\ldots}{\CAPTHETA_{\integer{m}}}}\sdtstile{}{}\CAPPHI$.
So by transitivity, $\Delta_{k+1}\sdtstile{}{}\CAPPHI$.

\item[\Rule{$\HORSESHOE$-Intro}:]
The rule \Rule{$\HORSESHOE$-Intro} changes the number of unboxed assumptions.
If line $k+1$ sanctioned by \Rule{$\HORSESHOE$-Intro} has sentence $\horseshoe{\CAPPHI}{\CAPTHETA}$ and unboxed assumptions $\Delta_{k+1}$, then there must be an assumption line (now in a box) that starts with $\CAPPHI$ and $\Delta_{k+1}$ as its other assumptions, and we have a line (now at the bottom of the box) with $\CAPTHETA$ on it with assumptions $\CAPPHI,\Delta_{k+1}$. 
By the recursive assumption we have that $\CAPPHI,\Delta_{k+1}\sdtstile{}{}\CAPTHETA$. 
Consider any model $\IntA{}$ that makes $\Delta_{k+1}$ true;
if it also makes $\CAPPHI$ true, then $\CAPTHETA$ is true in $\IntA{}$ as well and so is $\horseshoe{\CAPPHI}{\CAPTHETA}$. 
If $\IntA{}$ makes $\CAPPHI$ false, then $\horseshoe{\CAPPHI}{\CAPTHETA}$ is true. 
(Notice that this step only works because we defined the conditional to be true when the \CAPS{lhs} is false.)
So if $\IntA{}$ makes $\Delta_{k+1}$ true, it makes $\horseshoe{\CAPPHI}{\CAPTHETA}$ true too. 
So, $\Delta_{k+1}\sdtstile{}{}\horseshoe{\CAPPHI}{\CAPTHETA}$.  

\end{description}
\item[Closure Step:] We have now covered all the generating cases for derivations. By the closure clause of the definition, we have proved soundness for all derivations. 
\end{description}
\end{PROOFOF} 

\begin{PROOFOF}{Thm. \ref{Soundness of Sentential Logic}, SL Soundness Theorem}
Assume that $\Delta$ is a set of \GSL{} sentences. 
Assume $\Delta\sststile{}{}\CAPPHI$ and consider some derivation $\Derivation{D}$ of $\CAPPHI$ from $\Delta$. 
Let $\Delta'$ be the set of sentences in $\Delta$ that appear as unboxed assumptions in $\Derivation{D}$. 
By the soundness lemma (Thm. \ref{Main GSL Soundness Lemma}), $\Delta'\sdtstile{}{}\CAPPHI$. 
It follows immediately by monotonicity that $\Delta\sdtstile{}{}\CAPPHI$.  
\end{PROOFOF} 

\subsection{Soundness of \GQD{}}
In this section we prove that \GQD{} is also sound.\index{soundness!of \GQD{}}
\begin{THEOREM}{\LnpTC{Soundness of Quantifier Logic} \GQD{} Soundness Theorem:}
\GQD{} is sound; i.e., for every set $\Delta$ of sentences of \GQL{} and every sentence $\CAPPHI$ of \GQL{}, if $\Delta\sststile{}{}\CAPPHI$ in \GSD{}, then $\Delta\sdtstile{}{}\CAPPHI$.
\end{THEOREM}
\noindent{}The proof given in the last section of the \GSL{} Soundness Theorem \ref{Soundness of Sentential Logic} can be carried over to the \GQL{} Soundness Theorem. 
That proof relied on the monotonicity of entailment and the soundness lemma (Thm. \ref{Main GSL Soundness Lemma}). 
It should be clear that entailment is also monotonic in the case of \GQL{}. 
Since \GQD{} is an extension of \GSD{} (it's just \GSD{} plus the rules for the quantifiers in table \pncmvref{GQD}), all we need to do to show that the soundness lemma holds for \GQD{} is add a case, for each new rule of \GQD{}, to the inheritance step of the proof of the soundness lemma for \GSD{}.
\begin{PROOFOF}{Thm. \ref{Main GSL Soundness Lemma} for GQD}
\begin{description}

\item[Base Step:] 
The base case is covered in Thm. \ref{Main GSL Soundness Lemma}.

\item[Inheritance Step:]
Let $\Derivation{D}$ be an arbitrary derivation consisting of $k$ lines, where $k>1$.
Let $\Delta_\integer{i}$ be the set of unboxed assumptions occurring in a derivation $\Derivation{D}$ up to (and including) line number $\integer{i}$.
Consider the case in which a new line with sentence $\CAPPHI$ is added to derivation $\Derivation{D}$, sanctioned by rule \Rule{R}.

\begin{commentary}
	The resulting derivation has $k+1$ lines.
	We want to show, for each rule \Rule{R} of \GQD{}, that $\Delta_{k+1}\sdtstile{}{}\CAPPHI$.
	It was shown already that the property holds of each \GSD{} rule, so we only need to consider the quantifier introduction and elimination rules.
\end{commentary}

\begin{description}

\item[Recursive Assumption:]  
Assume for each line $i$ of derivation $\Derivation{D}$, where $i\leq k$ and $\CAPPHI_{i}$ is the sentence on line $i$, that $\Delta_{\integer{i}}\sdtstile{}{}\CAPPHI_{i}$.

\item[\Rule{$\forall$-Elim}:]
Let line $k+1$ of $\Derivation{D}$ with sentence $\CAPPHI\variable{s}/\BETA$ be sanctioned by \Rule{$\forall$-Elim}.
Then there's some earlier line $\integer{i}$ with the sentence $\universal{\BETA}\CAPPHI$. 
$\Delta_{\integer{i}}$ is the set of unboxed assumptions of line $\integer{i}$, and by NDAP $\Delta_{\integer{i}}\subseteq\Delta_{k+1}$.
By the recursive assumption, $\Delta_{\integer{i}}\sdtstile{}{}\universal{\BETA}\CAPPHI$.
So by monotonicity, $\Delta_{k+1}\sdtstile{}{}\universal{\BETA}\CAPPHI$.

Assume some model $\IntA$ such that $\universal{\BETA}\CAPPHI$ is true.
By the def. of truth of $\forall$, $\CAPPHI{\variable{t}/\BETA}$ is true on all $\variable{t}$-variants of $\IntA$.
Notice that $\CAPPHI{\variable{t}/\BETA}$ and $\CAPPHI{\variable{s}/\BETA}$ are exactly the same, except that the latter has $\variable{s}$ substituted for $\variable{t}$.
These sentences satisfy condition (1) of Dragnet.

Consider the $\variable{t}$-variant that assigns to $\variable{t}$ what $\IntA$ assigns to $\variable{s}$.
Name that $\variable{t}$-variant $\As{\variable{t}}{}$.
The models $\IntA$ and $\As{\variable{t}}{}$ meet Dragnet condition (2).
Thus, by Dragnet, $\CAPPHI{\variable{t}/\BETA}$ is true on $\As{\variable{t}}{}$ iff $\CAPPHI{\variable{s}/\BETA}$ is true on $\IntA$.
Therefore, $\CAPPHI{\variable{s}/\BETA}$ is true on $\IntA$.

Any model such that $\universal{\BETA}\CAPPHI$ is true also makes $\CAPPHI{\variable{s}/\BETA}$ true.
Thus, $\universal{\BETA}\CAPPHI\sdtstile{}{}\CAPPHI\variable{s}/\BETA$.
So by transitivity, $\Delta_{k+1}\sdtstile{}{}\CAPPHI\variable{s}/\BETA$.   

\item[\Rule{$\exists$-Intro}:]
Let line $k+1$ of $\Derivation{D}$ with sentence $\existential{\BETA}\CAPPHI$ be sanctioned by \Rule{$\exists$-Intro}.
Then there's some earlier line $\integer{i}$ with the sentence $\CAPPHI\variable{s}/\BETA$.
$\Delta_{\integer{i}}$ is the set of unboxed assumptions of line $\integer{i}$, and by NDAP $\Delta_{\integer{i}}\subseteq\Delta_{k+1}$.
By the recursive assumption, $\Delta_{\integer{i}}\sdtstile{}{}\CAPPHI\variable{s}/\BETA$.
By monotonicity, $\Delta_{k+1}\sdtstile{}{}\CAPPHI\variable{s}/\BETA$.

Assume some model $\IntA$ such that $\existential{\BETA}\CAPPHI$ is false.
By the def. of truth of $\exists$, there is no $\variable{t}$-variant of $\IntA$ that makes $\CAPPHI{\variable{t}/\BETA}$ true.
Notice that $\CAPPHI{\variable{t}/\BETA}$ and $\CAPPHI{\variable{s}/\BETA}$ are exactly the same, except that the latter has $\variable{s}$ substituted for $\variable{t}$.
These sentences satisfy condition (1) of Dragnet.

Consider the $\variable{t}$-variant that assigns to $\variable{t}$ what $\IntA$ assigns to $\variable{s}$.
Name that $\variable{t}$-variant $\As{\variable{t}}{}$.
The models $\IntA$ and $\As{\variable{t}}{}$ meet Dragnet condition (2).
Thus, by Dragnet, $\CAPPHI{\variable{t}/\BETA}$ is true on $\As{\variable{t}}{}$ iff $\CAPPHI{\variable{s}/\BETA}$ is true on $\IntA$.
Therefore, $\CAPPHI{\variable{s}/\BETA}$ is false on $\IntA$.

Any model that makes $\existential{\BETA}\CAPPHI$ false also makes $\CAPPHI{\variable{s}/\BETA}$ false.
Thus, $\CAPPHI\variable{s}/\BETA\sdtstile{}{}\existential{\BETA}\CAPPHI$.
So by transitivity, $\Delta_{k+1}\sdtstile{}{}\existential{\BETA}\CAPPHI$.

\item[\Rule{$\forall$-Intro}:]
Let line $k+1$ of $\Derivation{D}$ with sentence $\universal{\BETA}\CAPPHI$ be sanctioned by \Rule{$\forall$-Intro}.
Then there's some earlier line $\integer{i}$ with the sentence $\CAPPHI\variable{s}/\BETA$. 
$\Delta_{\integer{i}}$ is the set of unboxed assumptions of line $\integer{i}$, and by NDAP $\Delta_{\integer{i}}\subseteq\Delta_{k+1}$.
By the recursive assumption, $\Delta_{\integer{i}}\sdtstile{}{}\CAPPHI\variable{s}/\BETA$. 
By monotonicity, $\Delta_{k+1}\sdtstile{}{}\CAPPHI\variable{s}/\BETA$.
However $\CAPPHI\variable{s}/\BETA$ does not entail $\universal{\BETA}\CAPPHI$, so we have to do some extra work and make use of the restrictions on the rule \Rule{$\forall$-Intro}. 

Let $\IntA$ be some model that makes all of $\Delta_{k+1}$ true; and let's assume for \emph{reductio} that $\IntA$ that makes $\universal{\BETA}\CAPPHI$ false.
One of the restrictions for \Rule{$\forall$-Intro} is that $\variable{s}$ must not occur in $\universal{\BETA}\CAPPHI$.
Hence, by the Free Choice Theorem, $\CAPPHI\variable{s}/\BETA$ is false on some $\variable{s}$-variant of $\IntA$.
Let's name that $\variable{s}$-variant $\As{\variable{s}}{}$.

The other rule restriction for \Rule{$\forall$-Intro} is that $\variable{s}$ must not occur in $\Delta_{k+1}$.
The variant $\As{\variable{s}}{}$ differs from $\IntA$ only on the assignment to $\variable{s}$; otherwise, they make all the same assignments.
Since $\variable{s}$ doesn't occur in $\Delta_{k+1}$ and $\IntA$ makes all of $\Delta_{k+1}$ true, $\As{\variable{s}}{}$ also makes all of $\Delta_{k+1}$ true.
The assignment $\As{\variable{s}}{}$ makes to $\variable{s}$ doesn't matter for this result.

Since $\Delta_{k+1}\sdtstile{}{}\CAPPHI\variable{s}/\BETA$, $\As{\variable{s}}{}$ makes $\CAPPHI\variable{s}/\BETA$ true.
But we already showed that $\CAPPHI\variable{s}/\BETA$ is false on $\As{\variable{s}}{}$.
From the assumption that $\IntA$ makes $\universal{\BETA}\CAPPHI$ is false we have proven a contradiction.
So $\IntA$ makes $\universal{\BETA}\CAPPHI$ true.

Therefore, if $\IntA$ is a model that makes all of $\Delta_{k+1}$ true, then $\IntA$ makes $\universal{\BETA}\CAPPHI$ true as well. 
So, $\Delta_{k+1}\sdtstile{}{}\universal{\BETA}\CAPPHI$.

\item[\Rule{$\exists$-Elim}:]
Let line $k+1$ of $\Derivation{D}$ with sentence $\CAPTHETA$ be sanctioned by \Rule{$\exists$-Elim}.
Then there's some earlier line $\integer{i}$ with the sentence $\horseshoe{\CAPPHI\variable{s}/\BETA}{\CAPTHETA}$ and an earlier line $\integer{j}$ with the sentence $\existential{\BETA}\CAPPHI$. 
$\Delta_{\integer{i}}$ is the set of unboxed assumptions of line $\integer{i}$ and $\Delta_{\integer{j}}$ the unboxed assumptions of line $\integer{j}$.
By NDAP $\Delta_{\integer{i}}\subseteq\Delta_{k+1}$ and $\Delta_{\integer{j}}\subseteq\Delta_{k+1}$.
By the recursive assumption, $\Delta_{\integer{i}}\sdtstile{}{}\horseshoe{\CAPPHI\variable{s}/\BETA}{\CAPTHETA}$ and $\Delta_{\integer{j}}\sdtstile{}{}\existential{\BETA}\CAPPHI$.
By monotonicity, $\Delta_{k+1}\sdtstile{}{}\horseshoe{\CAPPHI\variable{s}/\BETA}{\CAPTHETA}$ and $\Delta_{k+1}\sdtstile{}{}\existential{\BETA}\CAPPHI$.
We have make use of the restrictions on the rule \Rule{$\exists$-Elim} to show that $\Delta_{k+1}\sdtstile{}{}\CAPTHETA$. 

Let $\IntA$ be some model that makes all of $\Delta_{k+1}$ true. 
Since $\Delta_{k+1}\sdtstile{}{}\existential{\BETA}\CAPPHI$, $\IntA$ also makes $\existential{\BETA}\CAPPHI$ true.
One of the rule restrictions for \Rule{$\exists$-Elim} is that $\variable{s}$ must not occur in $\existential{\BETA}\CAPPHI$.
Hence, by the Free Choice theorem, $\CAPPHI{\variable{s}/\BETA}$ is true on some $\variable{s}$-variant of $\IntA$.
Name that $\variable{s}$-variant $\As{\variable{s}}{}$.  

Another of the rule restrictions for \Rule{$\exists$-Elim} is that $\variable{s}$ must not occur in $\Delta_{k+1}$.
The variant $\As{\variable{s}}{}$ makes all the same assignments as $\IntA$ except in what it assigns to $\variable{s}$.
Since $\IntA$ makes $\Delta_{k+1}$ true and $\Delta_{k+1}$ doesn't contain $\variable{s}$, $\As{\variable{s}}{}$ also makes $\Delta_{k+1}$ true.
The assignment $\As{\variable{s}}{}$ makes to $\variable{s}$ doesn't make any difference.

Thus, because $\Delta_{k+1}\sdtstile{}{}\horseshoe{\CAPPHI\variable{s}/\BETA}{\CAPTHETA}$, $\As{\variable{s}}{}$ makes $\horseshoe{\CAPPHI\variable{s}/\BETA}{\CAPTHETA}$ true.
We saw earlier that $\As{\variable{s}}{}$ makes $\CAPPHI\variable{s}/\BETA$ true, so $\As{\variable{s}}{}$ makes $\CAPTHETA$ true as well (def. of truth, $\HORSESHOE$).  

According to the third rule restriction for \Rule{$\exists$-Elim}, $\variable{s}$ must not occur in $\CAPTHETA$.
Because $\As{\variable{s}}{}$ makes $\CAPTHETA$ true and $\variable{s}$ isn't in $\CAPTHETA$, $\IntA$ also makes $\CAPTHETA$ true.
The assignment that $\As{\variable{s}}{}$ makes to $\variable{s}$ is irrelevant.

So, we have shown that $\Delta_{k+1}\sdtstile{}{}\CAPTHETA$.

\end{description}

\item[Closure Step:] We have covered all the generating cases for derivations. By the closure clause of the definition, we have proved soundness for all derivations in \GQD{}. 

\end{description}
\end{PROOFOF} 

%%%%%%%%%%%%%%%%%%%%%%%%%%%%%%%%%%%%%%%%%%%%%%%%%%
\section{Completeness}\label{Section:Completeness for GSD}
%%%%%%%%%%%%%%%%%%%%%%%%%%%%%%%%%%%%%%%%%%%%%%%%%%

Next we prove the completeness of \GSD{}.

\begin{majorILnc}{\LnpDC{LRCompleteness}}
	A derivation system \DerivationSystem{D} for \Language{L} is \nidf{complete}\index{completeness|textbf} \Iff for every finite set $\Delta$ of sentences of \Language{L} and every sentence $\CAPPHI$ of \Language{L}, if $\Delta\sdtstile{}{}\CAPPHI$, then $\Delta\sststile{}{}\CAPPHI$.
\end{majorILnc} 
\noindent{}When $\Delta$ is limited to the empty set, the result is weak completeness:
\begin{majorILnc}{\LnpDC{LWCompleteness}}
	A derivation system \DerivationSystem{D} for \Language{L} is \nidf{weakly complete}\index{completeness!weak|textbf} \Iff for every sentence $\CAPPHI$ of \Language{L}, if $\sdtstile{}{}\CAPPHI$, then $\sststile{}{}\CAPPHI$.
\end{majorILnc} 
\noindent{}The following theorem can be proved using basic results already shown.
In other systems of logic, what we call completeness and weak completeness are not equivalent.
They \emph{are} equivalent in our systems, so we do not always distinguish them.
There is also a notion of `strong' completeness, which we define shortly.
\GSD{} and \GQD{} are strongly complete, but this result is not trivially equivalent to completeness.
There are other systems that are complete but not strongly complete.
\begin{THEOREM}{\LnpTC{RegWeakCompletenessEquiv}}
	\GSD{} is weakly complete \Iff it's complete; and likewise for \GQD{}.
\end{THEOREM}
\begin{PROOF}
	$(\Leftarrow)$ Assume that \GSD{}/\GQD{} is complete. 
	Then for any finite set $\Delta$, if $\Delta\sdtstile{}{}\CAPPHI$, then $\Delta\sststile{}{}\CAPPHI$. 
	This includes the case when $\Delta$ is the empty set. 
	So if $\sdtstile{}{}\CAPPHI$, then $\sststile{}{}\CAPPHI$.
	Hence \GSD{}/\GQD{} is weakly complete. 
	
	$(\Rightarrow)$ Assume that \GSD{}/\GQD{} is weakly complete: for any sentence $\CAPPHI$, if $\sdtstile{}{}\CAPPHI$, then $\sststile{}{}\CAPPHI$. 
	Assume that, for some finite set $\Delta$ of sentences and sentence $\CAPPHI$, $\Delta\sdtstile{}{}\CAPPHI$.
	Since $\Delta$ is finite, there is a sentence $\DELTA$ that is a conjunction of all the sentences in $\Delta$.
	We want to show that $\sdtstile{}{}\horseshoe{\DELTA}{\CAPPHI}$.
	So, assume for \emph{indirect proof} there's some model $\IntA$ that makes $\horseshoe{\DELTA}{\CAPPHI}$ false.
	By the definition of truth for $\HORSESHOE$ and $\WEDGE$, it follows that $\IntA$ makes all the conjuncts of $\DELTA$ true and $\CAPPHI$ false. 
	Then $\IntA$ makes all the sentences in $\Delta$ true and $\CAPPHI$ false.
	But we assumed that $\Delta\sdtstile{}{}\CAPPHI$.
	It follows that there's no model $\IntA$ that makes $\horseshoe{\DELTA}{\CAPPHI}$ false.
	Hence $\sdtstile{}{}\horseshoe{\DELTA}{\CAPPHI}$, and so by weak completeness, $\sststile{}{}\horseshoe{\DELTA}{\CAPPHI}$.
	It should be clear to the reader that if $\sststile{}{}\horseshoe{\DELTA}{\CAPPHI}$, then $\DELTA\sststile{}{}\CAPPHI$.
	Hence, $\DELTA\sststile{}{}\CAPPHI$.
	Finally, since $\Delta\sststile{}{}\DELTA$ and $\sststile{}{}$ is transitive, $\Delta\sststile{}{}\CAPPHI$.
\end{PROOF}
\begin{majorILnc}{\LnpDC{LCompleteness}}
	A derivation system \DerivationSystem{D} for \Language{L} is \nidf{strongly complete}\index{completeness!strong|textbf} \Iff for every set $\Delta$ of sentences of \Language{L} and every sentence $\CAPPHI$ of \Language{L}, if $\Delta\sdtstile{}{}\CAPPHI$, then $\Delta\sststile{}{}\CAPPHI$.
\end{majorILnc} 
\noindent{}Strong completeness differs from (regular) completeness in that $\Delta$ is allowed to be infinite.
If we limit $\Delta$ so that it must be finite (but still allow it to be empty), we get completeness.
\noindent{}Note that both \GSD{} and \GQD{} are strongly complete, but there is no simple theorem that uses results we already have which extends weak completeness to strong completeness in the way this theorem (Thm. \ref{RegWeakCompletenessEquiv}) extends weak completeness to (regular) completeness.

Letting $\Delta$ be infinite may seem excessive, since we have defined derivations (def. \pmvref{Recursive definition of Derivation}) to have only finitely many lines. 
A derivation can only have finitely many assumptions.
And, as we've defined the single turnstile, $\Delta\sststile{}{}\CAPPHI$ iff there's a derivation of $\CAPPHI$ from the sentences in $\Delta$. 

As it turns out, however, there's nothing wrong with letting $\Delta$ be infinite.
Showing that there's a derivation of $\CAPPHI$ from the sentences in $\Delta$ doesn't require that the derivation use \emph{all} the sentences in $\Delta$ as assumptions.
In general, even when $\Delta$ is finite, any derivation of $\CAPPHI$ from some subset of sentences in $\Delta$ shows that $\Delta\sststile{}{}\CAPPHI$. 
So, if $\Delta$ is infinite and $\Delta\sdtstile{}{}\CAPPHI$, if the derivation system \DerivationSystem{D} is complete it follows that $\CAPPHI$ can be derived from some finite subset of sentences of $\Delta$.\footnote{Before turning to the proofs of these theorems, some historical background might be of interest. 
	As mentioned above (Sec. \ref{Sec:GQLSymbols}), quantificational languages were first developed by Frege, Peirce and Mitchell in the 1870's and 1880's. 
	But it wasn't until David Hilbert and Wilhelm Ackermann published their hugely influential text \emph{Grundz\"uge der theoretischen Logik} (Principles of Mathematical Logic) in \citeyear{Hilbert1928} that the question of completeness was clearly formulated. 
	While Kurt G\"odel, in his \citeyear{Godel1929} doctorial dissertation (republished in \citeyear{Godel1930}), is widely accepted as the first person to prove that quantificational logic is strongly complete, Church \citeyearpar[291,~fn.464]{Church1956} reports that the Jacques Herbrand's dissertation in 1930 had the essential material for the same proof.  
	Further, completeness follows from results of Skolem \citeyearpar{Skolem1928}, but since the question of completeness hadn't been clearly raised yet no one seems to have noticed. 
	Leon Henkin \citeyearpar{Henkin1949} later developed a method of proving completeness different from G\"odels. 
	Henkin's approach is probably the most common one used today in logic textbooks, but the proof we give here is a constructive proof closer to G\"odel's original.
	(Ours owes much to Willard Quine's completeness proof \citeyearpar{Quine1982}.)}

By theorem \mvref{RegWeakCompletenessEquiv}, if we show that \GSD{} is weakly complete, it follows that \GSD{} is complete.
To demonstrate the weak completeness of \GSD{} we prove this intermediate result:
\begin{THEOREM}{\LnpTC{GSDCompletenessLemma} The \GSD{} Weak Completeness Lemma:}
For\index{completeness!weak \GSD{}} any sentence $\CAPPHI$ of \GSD{}, either $\CAPPHI\sststile{}{}\conjunction{\Al}{\negation{\Al}}$, or $\CAPPHI$ is true in some model $\IntA$.
\end{THEOREM}

As an aid to prove this result, we introduce a new exchange rule for \GQD{} and then show that anything derivable with \GQD{} and this rule can be derived using \GQD{} alone.
We call the rule \Rule{$\TRIPLEBAR$-Exchange}.
(Note that every application of \Rule{$\TRIPLEBAR$-Exchange} is truth preserving, as the last problem in exercise \pmvref{exercises:GSDTFETheorem}, extends theorem \pmvref{ExchangeRuleGSDSoundnessLemma}, to it.)
It's given in table \ref{GSDplusDNF}.
\begin{table}[!ht]
\renewcommand{\arraystretch}{1.5}
\begin{center}
\begin{tabular}{ p{1in} l l } %p{2.2in} p{2in}
\toprule
\textbf{Name} & \textbf{Given} & \textbf{May Add} \\ 
\midrule
\Rule{$\TRIPLEBAR$-Exchange} &  $\triplebar{\CAPTHETA}{\CAPPSI}$ & $\disjunction{\parconjunction{\CAPTHETA}{\CAPPSI}}{\parconjunction{\negation{\CAPTHETA}}{\negation{\CAPPSI}}}$ \\
\nopagebreak
 & $\disjunction{\parconjunction{\CAPTHETA}{\CAPPSI}}{\parconjunction{\negation{\CAPTHETA}}{\negation{\CAPPSI}}}$ &  $\triplebar{\CAPTHETA}{\CAPPSI}$ \\
\bottomrule
\end{tabular}
\end{center}
\caption{\Rule{$\TRIPLEBAR$-Exchange}}
\label{GSDplusDNF}%
\end{table}
\index{derivation!rule!for DNF}\index{DNF}
\noindent{}All we need to show that anything that can be derived using \GQD{} and \Rule{$\TRIPLEBAR$-Exchange} can be derived using just \GQD{}, is to prove the following:\footnote{
	See section \ref{Shortcut Rule Elimination Theorem Section}.
}
\begin{THEOREM}{\LnpTC{GQD NDF Rule}}
Any two \GQL{} formulas got by substituting other \GQL{} formulas into the may-add and given schemas of \Rule{$\TRIPLEBAR$-Exchange} are provably equivalent; that is, $\sststile{}{}\forall\bpartriplebar{\partriplebar{\CAPTHETA}{\CAPPSI}}{\pardisjunction{\parconjunction{\CAPTHETA}{\CAPPSI}}{\parconjunction{\negation{\CAPTHETA}}{\negation{\CAPPSI}}}}$.
\end{THEOREM}
\begin{PROOF}
We show that $\sststile{}{}\forall\bpartriplebar{\partriplebar{\CAPTHETA}{\CAPPSI}}{\pardisjunction{\parconjunction{\CAPTHETA}{\CAPPSI}}{\parconjunction{\negation{\CAPTHETA}}{\negation{\CAPPSI}}}}$ by giving a derivation schema, which for any two formulas $\CAPTHETA$ and $\CAPPSI$ results in the needed derivation. 
(Note that to save space $\integer{q}=\integer{n}+\integer{m}$.)
\begin{gproofnn}
\gaproof{
\galine{1}{$\partriplebar{\CAPTHETA}{\CAPPSI}\constant{c_{\integer{1}}}\ldots\constant{c_{\integer{\integer{m}}}}/\variable{x}_1\ldots\variable{x}_{\integer{m}}$}{\Rule{Assume}}
\galine{2}{$\partriplebar{\negation{\CAPTHETA}}{\negation{\CAPPSI}}\constant{c_{\integer{1}}}\ldots\constant{c_{\integer{\integer{m}}}}/\variable{x}_1\ldots\variable{x}_{\integer{m}}$}{\Rule{$\NEGATION$/$\TRIPLEBAR$-Intro}, 1}
\gaaproof{
\gaaline{3}{$\negation{\parconjunction{\CAPTHETA}{\CAPPSI}}\constant{c_{\integer{1}}}\ldots\constant{c_{\integer{\integer{m}}}}/\variable{x}_1\ldots\variable{x}_{\integer{m}}$}{\Rule{Assume}}
\gaaline{4}{$\pardisjunction{\negation{\CAPTHETA}}{\negation{\CAPPSI}}\constant{c_{\integer{1}}}\ldots\constant{c_{\integer{\integer{m}}}}/\variable{x}_1\ldots\variable{x}_{\integer{m}}$}{\Rule{DeM}, 3}
\gaaaproof{
\gaaaline{5}{$\negation{\CAPTHETA}\constant{c_{\integer{1}}}\ldots\constant{c_{\integer{\integer{m}}}}/\variable{x}_1\ldots\variable{x}_{\integer{m}}$}{\Rule{Assume}}
\gaaaline{6}{$\negation{\CAPPSI}\constant{c_{\integer{1}}}\ldots\constant{c_{\integer{\integer{m}}}}/\variable{x}_1\ldots\variable{x}_{\integer{m}}$}{\Rule{$\TRIPLEBAR$-Elim}, 2, 5}
\gaaaline{7}{$\parconjunction{\negation{\CAPTHETA}}{\negation{\CAPPSI}}\constant{c_{\integer{1}}}\ldots\constant{c_{\integer{\integer{m}}}}/\variable{x}_1\ldots\variable{x}_{\integer{m}}$}{\Rule{$\WEDGE\!$-Intro}, 5, 6}
}
\gaaline{8}{$\parhorseshoe{\negation{\CAPTHETA}}{\parconjunction{\negation{\CAPTHETA}}{\negation{\CAPPSI}}}\constant{c_{\integer{1}}}\ldots\constant{c_{\integer{\integer{m}}}}/\variable{x}_1\ldots\variable{x}_{\integer{m}}$}{\Rule{$\HORSESHOE$-Intro}, 5--7}

\gaaaproof{
\gaaaline{9}{$\negation{\CAPPSI}\constant{c_{\integer{1}}}\ldots\constant{c_{\integer{\integer{m}}}}/\variable{x}_1\ldots\variable{x}_{\integer{m}}$}{\Rule{Assume}}
\gaaaline{10}{$\negation{\CAPTHETA}\constant{c_{\integer{1}}}\ldots\constant{c_{\integer{\integer{m}}}}/\variable{x}_1\ldots\variable{x}_{\integer{m}}$}{\Rule{$\TRIPLEBAR$-Elim}, 2, 9}
\gaaaline{11}{$\parconjunction{\negation{\CAPTHETA}}{\negation{\CAPPSI}}\constant{c_{\integer{1}}}\ldots\constant{c_{\integer{\integer{m}}}}/\variable{x}_1\ldots\variable{x}_{\integer{m}}$}{\Rule{$\WEDGE\!$-Intro}, 9, 10}
}
\gaaline{12}{$\parhorseshoe{\negation{\CAPPSI}}{\parconjunction{\negation{\CAPTHETA}}{\negation{\CAPPSI}}}\constant{c_{\integer{1}}}\ldots\constant{c_{\integer{\integer{m}}}}/\variable{x}_1\ldots\variable{x}_{\integer{m}}$}{\Rule{$\HORSESHOE$-Intro}, 9--11}
\gaaline{13}{$\parconjunction{\negation{\CAPTHETA}}{\negation{\CAPPSI}}\constant{c_{\integer{1}}}\ldots\constant{c_{\integer{\integer{m}}}}/\variable{x}_1\ldots\variable{x}_{\integer{m}}$}{\Rule{$\VEE$-Intro}, 4, 8, 12}
}
\galine{14}{$\parhorseshoe{\negation{\parconjunction{\CAPTHETA}{\CAPPSI}}}{\parconjunction{\negation{\CAPTHETA}}{\negation{\CAPPSI}}}\constant{c_{\integer{1}}}\ldots\constant{c_{\integer{\integer{m}}}}/\variable{x}_1\ldots\variable{x}_{\integer{m}}$}{\Rule{$\HORSESHOE$-Intro}, 3--13}
\galine{15}{$\pardisjunction{\negation{\negation{\parconjunction{\CAPTHETA}{\CAPPSI}}}}{\parconjunction{\negation{\CAPTHETA}}{\negation{\CAPPSI}}}\constant{c_{\integer{1}}}\ldots\constant{c_{\integer{\integer{m}}}}/\variable{x}_1\ldots\variable{x}_{\integer{m}}$}{\Rule{$\HORSESHOE$/$\VEE$-Exch.}, 14}
\galine{16}{$\pardisjunction{\parconjunction{\CAPTHETA}{\CAPPSI}}{\parconjunction{\negation{\CAPTHETA}}{\negation{\CAPPSI}}}\constant{c_{\integer{1}}}\ldots\constant{c_{\integer{\integer{m}}}}/\variable{x}_1\ldots\variable{x}_{\integer{m}}$}{\Rule{$\NEGATION\NEGATION$-Elim}, 15}
}
\gline{17}{$[\partriplebar{\CAPTHETA}{\CAPPSI}\HORSESHOE$}{ }
\nopagebreak
\glinend{ }{$\qquad\pardisjunction{\parconjunction{\CAPTHETA}{\CAPPSI}}{\parconjunction{\negation{\CAPTHETA}}{\negation{\CAPPSI}}}]\constant{c_{\integer{1}}}\ldots\constant{c_{\integer{\integer{m}}}}/\variable{x}_1\ldots\variable{x}_{\integer{m}}$}{\Rule{$\HORSESHOE$-Intro}, 1--16}
\gaproof{
\galine{18}{$\pardisjunction{\parconjunction{\CAPTHETA}{\CAPPSI}}{\parconjunction{\negation{\CAPTHETA}}{\negation{\CAPPSI}}}\constant{c_{\integer{1}}}\ldots\constant{c_{\integer{\integer{m}}}}/\variable{x}_1\ldots\variable{x}_{\integer{m}}$}{\Rule{Assume}}
\galinend{ }{ }{ }
\galinend{ }{$\qquad\vdots$}{ }
\galinend{ }{ }{ }
\galine{$\integer{n}$}{$\partriplebar{\CAPTHETA}{\CAPPSI}\constant{c_{\integer{1}}}\ldots\constant{c_{\integer{\integer{m}}}}/\variable{x}_1\ldots\variable{x}_{\integer{m}}$}{ }
}
\gline{$\integer{n}+1$}{$[\pardisjunction{\parconjunction{\CAPTHETA}{\CAPPSI}}{\parconjunction{\negation{\CAPTHETA}}{\negation{\CAPPSI}}}\HORSESHOE$}{ }
\glinend{ }{$\qquad\partriplebar{\CAPTHETA}{\CAPPSI}]\constant{c_{\integer{1}}}\ldots\constant{c_{\integer{\integer{m}}}}/\variable{x}_1\ldots\variable{x}_{\integer{m}}$}{\Rule{$\HORSESHOE$-Intro}, 18--$\integer{n}$}
\gline{$\integer{n}+2$}{$[\partriplebar{\CAPTHETA}{\CAPPSI}\TRIPLEBAR$}{\Rule{$\TRIPLEBAR$-Intro}, 17,}
\glinend{ }{$\qquad\pardisjunction{\parconjunction{\CAPTHETA}{\CAPPSI}}{\parconjunction{\negation{\CAPTHETA}}{\negation{\CAPPSI}}}]\constant{c_{\integer{1}}}\ldots\constant{c_{\integer{\integer{m}}}}/\variable{x}_1\ldots\variable{x}_{\integer{m}}$}{$\integer{n}+1$}
\glinend{ }{ }{ }
\glinend{ }{$\qquad\vdots$}{ }
\glinend{ }{ }{ }
\gline{$\integer{q}+2$}{$\forall\bpartriplebar{\partriplebar{\CAPTHETA}{\CAPPSI}}{\pardisjunction{\parconjunction{\CAPTHETA}{\CAPPSI}}{\parconjunction{\negation{\CAPTHETA}}{\negation{\CAPPSI}}}}$}{\Rule{$\forall$-Intro}, $\integer{q}+1$}
\end{gproofnn}
\noindent{}Note that we have left steps $18$--$\integer{n}$ for the reader; 
this is just the derivation of the other conditional needed for \Rule{$\TRIPLEBAR$-Intro} on line $\integer{n}+2$. 
Also note that the last steps, lines $\integer{n}+3$ to the end, are all \Rule{$\forall$-Intro} meant to eliminate the constants $\constant{c_{\integer{1}}},\ldots,\constant{c_{\integer{\integer{m}}}}$.
\end{PROOF}
%Now we turn to the proof of the \GSD{} Completeness Lemma.
\begin{PROOFOF}{Thm. \ref{GSDCompletenessLemma}}
To prove the theorem, we shall describe an algorithm for applying the rules of \GSDP{} and \Rule{$\TRIPLEBAR$-Exchange} that takes a \GSL{} sentence $\CAPPHI$ and either halts in a derivation of $\conjunction{\Al}{\negation{\Al}}$, or halts with a sentence in \CAPS{dnf} for which there is some model $\IntA$ that makes $\CAPPHI$ true.
Since a sentence can be derived using the rules of \GSDP{} and \Rule{$\TRIPLEBAR$-Exchange} \Iff it can be derived using the basic rules of \GSD{}, this is sufficient to prove the theorem. 

The algorithm begins with $\CAPPHI$ as an assumption on line 1. 
The algorithm then applies the method studied earlier in section \mvref{Disjunctive Normal Form} to produce a sentence $\CAPPHI'$ in \CAPS{dnf} that's \CAPS{tfe} to $\CAPPHI$.
We have to show that each step of the earlier method can be carried out in steps using the rules of \GSDP{} and \Rule{$\TRIPLEBAR$-Exchange}.
The earlier method proceeded in three stages. 
\begin{description}
\item[Step A:] \hfill
\begin{cenumerate}
\item If a subsentence of $\CAPPHI$ has $\HORSESHOE$ as its main connective, i.e. if $\CAPPHI=\horseshoe{\CAPTHETA}{\CAPPSI}$, replace the subsentence by $\disjunction{\negation{\CAPTHETA}}{\CAPPSI}$.
Repeat as necessary to obtain a sentence $\CAPPHI^*$ without conditionals. 
Each of these steps are sanctioned by \Rule{$\HORSESHOE$/$\VEE$-Exchange}.

\item If a subsentence of $\CAPPHI$ has $\TRIPLEBAR$ as its main connective, i.e. if $\CAPPHI=\triplebar{\CAPTHETA}{\CAPPSI}$, it is replaced with the subsentence $\disjunction{\parconjunction{\CAPTHETA}{\CAPPSI}}{\parconjunction{\negation{\CAPTHETA}}{\negation{\CAPPSI}}}$.
Repeat as necessary to obtain a sentence $\CAPPHI^{**}$ without biconditionals.
Each of these steps are sanctioned by \Rule{$\TRIPLEBAR$-Exchange}.
\end{cenumerate}

\item[Step B:]
In the case where $\CAPPHI^{**}$ contains a subsentence whose main connective is negation and which contains other connectives, we replace that subsentence by the following steps:
\begin{cenumerate}
\item Replace $\negation{\negation{\CAPTHETA}}$ by $\CAPTHETA$; this step is sanctioned by \Rule{$\NEGATION\NEGATION$-Elim}.
\item Replace $\negation{\parconjunction{\CAPTHETA}{\CAPPSI}}$ by $\disjunction{\negation{\CAPTHETA}}{\negation{\CAPPSI}}$; this step is sanctioned by \Rule{DeM}.
\item Replace $\negation{\pardisjunction{\CAPTHETA}{\CAPPSI}}$ by $\conjunction{\negation{\CAPTHETA}}{\negation{\CAPPSI}}$; this step is sanctioned by \Rule{DeM}.
\end{cenumerate}
Repeat as necessary to obtain a sentence $\CAPPHI^{***}$ in which negations govern nothing but sentence letters. 

\item[Step C:]
The only thing that could prevent $\CAPPHI^{***}$ from being in \CAPS{dnf} is that some conjunctions govern some disjunctions, i.e., there is a subsentence of the form $\conjunction{\CAPTHETA}{\pardisjunction{\CAPPSI_1}{\disjunction{\ldots}{\CAPPSI_{\integer{n}}}}}$, or the reverse $\conjunction{\pardisjunction{\CAPPSI_1}{\disjunction{\ldots}{\CAPPSI_{\integer{n}}}}}{\CAPTHETA}$.
Those subsentences can each be replaced by the equivalent sentence $\disjunction{\parconjunction{\CAPTHETA}{\CAPPSI_1}}{\disjunction{\ldots}{\parconjunction{\CAPTHETA}{\CAPPSI_{\integer{n}}}}}$ or $\disjunction{\parconjunction{\CAPPSI_1}{\CAPTHETA}}{\disjunction{\ldots}{\parconjunction{\CAPPSI_{\integer{n}}}{\CAPTHETA}}}$.
These steps are sanctioned by \Rule{Distribution}.
\end{description}
\noindent{}Applying the above steps A, B, and C provides us a derivation starting with $\CAPPHI$ as an assumption (and no other assumptions) and ending with a \CAPS{dnf} sentence that's \CAPS{tfe} to $\CAPPHI$. 
There are two cases:
\begin{description}
\item[Case 1:] 
Every disjunct contains a sentence letter and the negation of that sentence letter. 
That is, each disjunction has the form $\parconjunction{\CAPPSI_1}{\conjunction{\ldots}{\conjunction{\CAPPSI_{\integer{i}}}{\conjunction{\ldots}{\conjunction{\negation{\CAPPSI_{\integer{i}}}}{\conjunction{\ldots}{\CAPPSI_{\integer{n}}}}}}}}$; for example: $\parconjunction{\Al}{\conjunction{B}{\conjunction{\Cl}{\conjunction{\negation{\El}}{\conjunction{\negation{\Bl}}{\negation{\Kl}}}}}}$.

\item[Case 2:]
At least one disjunct contains no sentence letter such that the negation of the sentence letter is also in the disjunct. 
\end{description}
\noindent{}It can be shown in case $1$ that the algorithm derives a contradiction from the original sentence.
First, observe that any conjunction that contains a sentence letter and its negation leads to a contradiction by repeated steps of \Rule{$\WEDGE\!$-Elim}. 
Thus we can derive the negation of any such conjunction using \Rule{$\NEGATION$-Intro}.
So if the last line of the derivation so far is of the form $\pardisjunction{\CAPPSI_1}{\disjunction{\ldots}{\CAPPSI_{\integer{n}}}}$ and each $\CAPPSI_{\integer{i}}$ contains a sentence letter and the negation of that sentence letter, then add to the derivation lines that establish the negation of each $\CAPPSI_{\integer{i}}$. 
Thus by $\integer{n}-1$ steps of \Rule{D.S.} the result is a single $\CAPPSI_{\integer{i}}$ by itself with only the first line as an assumption.
Since this remaining conjunct has both a sentence letter and its negation it is trivial to derive a contradiction.

Schematically, the procedure looks like:

\begin{gproofnn}
\glinend{ }{$\CAPPHI$}{\Rule{Assume}} %\marginnote{\scriptsize{}The original sentence}[0cm]
\glinend{ }{ }{ }
\glinend{ }{$\qquad\vdots$}{ }
\glinend{ }{ }{ }
\glinend{ }{$\disjunction{\CAPPSI_1}{\disjunction{\ldots}{\CAPPSI_{\integer{n}}}}$}{ } %\marginnote{\scriptsize{}The \CAPS{dnf} sentence after steps A, B, and C}[0cm]
\gaproof{
\galinend{ }{$\CAPPSI_1$}{\Rule{Assume}}
\galinend{ }{ }{ }
\galinend{ }{$\qquad\vdots$}{ }
\galinend{ }{ }{ }
\galinend{ }{$\conjunction{\CAPTHETA_1}{\negation{\CAPTHETA_1}}$}{ }
}
\glinend{ }{$\horseshoe{\CAPPSI_1}{\parconjunction{\CAPTHETA_1}{\negation{\CAPTHETA_1}}}$}{\Rule{$\HORSESHOE$-Intro}} %\marginnote{\scriptsize{}We start deriving the negation of each disjunct}[0cm]
\glinend{ }{$\negation{\CAPPSI_1}$}{\Rule{$\NEGATION$-Intro}}
\glinend{ }{ }{ }
\glinend{ }{$\qquad\vdots$}{ }
\glinend{ }{ }{ }
\gaproof{
\galinend{ }{$\CAPPSI_{\integer{n}}$}{\Rule{Assume}}
\galinend{ }{ }{ }
\galinend{ }{$\qquad\vdots$}{ }
\galinend{ }{ }{ }
\galinend{ }{$\conjunction{\CAPTHETA_{\integer{n}}}{\negation{\CAPTHETA_{\integer{n}}}}$}{ }
}
\glinend{ }{$\horseshoe{\CAPPSI_{\integer{n}}}{\parconjunction{\CAPTHETA_{\integer{n}}}{\negation{\CAPTHETA_{\integer{n}}}}}$}{\Rule{$\HORSESHOE$-Intro}}
\glinend{ }{$\negation{\CAPPSI_{\integer{n}}}$}{\Rule{$\NEGATION$-Intro}}
\glinend{ }{$\disjunction{\CAPPSI_1}{\disjunction{\ldots}{\CAPPSI_{\integer{n}-1}}}$}{\Rule{D.S.}} %\marginnote{\scriptsize{}Start applying \Rule{D.S.}}[0cm]
\glinend{ }{$\disjunction{\CAPPSI_1}{\disjunction{\ldots}{\CAPPSI_{\integer{n}-2}}}$}{\Rule{D.S.}}
\glinend{ }{$\disjunction{\CAPPSI_1}{\disjunction{\ldots}{\CAPPSI_{\integer{n}-3}}}$}{\Rule{D.S.}}
\glinend{ }{ }{ }
\glinend{ }{$\qquad\vdots$}{ }
\glinend{ }{ }{ }
\glinend{ }{$\disjunction{\CAPPSI_1}{\CAPPSI_2}$}{\Rule{D.S.}}
\glinend{ }{$\CAPPSI_1$}{\Rule{D.S.}}
\glinend{ }{$\negation{\CAPPSI_1}$}{\Rule{Rep.}}
\glinend{ }{$\conjunction{\Al}{\negation{\Al}}$}{\Rule{A.C.}} %\marginnote{\scriptsize{}Finally we reach a contradiction}[0cm]
\end{gproofnn}

\noindent{}So much for case 1.

In case $2$ it can be shown that there is a model that makes all sentences in the derivation true, starting with the last.
To construct this model, first choose the disjunction that does not contain a sentence letter and its negation.
If there is more than one, it doesn't matter which is chosen.
Then construct a model $\IntA$ by assigning $\TrueB$ to each sentence letter that occurs positively (without a negation in front) and $\FalseB$ to each sentence letter that occurs negatively (with a negation in front).

This model makes each element of the conjunction true and thus makes the entire conjunction true. 
Since the sentence containing it is a disjunction, this is sufficient to make the entire sentence true.
Thus it makes the last line of the derivation true.
Observe that all of the steps we used in the derivation were replacement of provably equivalence sentences;
that is, they used exchange shortcut rules.
Thus, we can also construct a derivation by \mention{turning this proof upside down}, so to speak.
In other words, we can construct a new derivation, with the last step of the original derivation as the initial assumption step, then use the exchange rules to work back to original $\CAPPHI$.

Thus, by soundness, if the first sentence of the \mention{upside-down derivation} (the sentence in \CAPS{dnf} that was at the bottom) is true in a model, then so is everything that can be derived from it, including our original sentence $\CAPPHI$ that is now at the end of the inverted derivation. 
Therefore, $\CAPPHI$ is true in some model. 
\end{PROOFOF}
\begin{THEOREM}{\LnpTC{GSDWCompleteness} Weak \GSD{} Completeness Theorem:}
For all \GSL{} sentences $\CAPPHI$: if $\sdtstile{}{}\CAPPHI$, then $\sststile{}{}\CAPPHI$ in \GSD{}.
\end{THEOREM}
\begin{PROOF}
We apply the method above from the \GSD{} Completeness Lemma to the negation of $\CAPPHI$. 
This either produces a derivation of a contradiction from $\negation{\CAPPHI}$, in which case we can prove $\CAPPHI$ by adding two more steps justified by \Rule{$\HORSESHOE$-Intro} and \Rule{$\NEGATION$-Elim}, or it produces a model that makes $\negation{\CAPPHI}$ true and that therefore makes $\CAPPHI$ false. So, $\CAPPHI$ is either false in some model or is derivable in \GSD{}. 
\end{PROOF}
\noindent{}Finally, as a corollary we get:
\begin{THEOREM}{\LnpTC{GSDCompleteness} \GSD{} Completeness Theorem:}
For every finite set $\Delta$ of sentences of \GSL{} and every sentence $\CAPPHI$ of \GSL{}, if $\Delta\sdtstile{}{}\CAPPHI$, then $\Delta\sststile{}{}\CAPPHI$ in \GSD{}.
\end{THEOREM}
\begin{PROOF}
This follows immediately from the Weak \GSD{} Completeness Theorem and theorem \mvref{RegWeakCompletenessEquiv}.
\end{PROOF}

%%%%%%%%%%%%%%%%%%%%%%%%%%%%%%%%%%%%%%%%%%%%%%%%%%
\section{Completeness of \GQD{}}\label{Sec:Completeness of GQD}
%%%%%%%%%%%%%%%%%%%%%%%%%%%%%%%%%%%%%%%%%%%%%%%%%%

Next we prove the completeness of \GQD{}.  We want to use a similar strategy for \GQD{} we did for \GSD{}, but we have to deal with the quantifiers somehow.
Unfortunately, the quantifiers prevent us from proving the analogue of DNF for \GQL{}.
Given an arbitrary sentence, there is no guarantee that we can reorder the quantifiers so that each has its scope contained in a single disjunct.
For example:  $\universal{\variable{x}}\pardisjunction{\Hp{\variable{x}}}{\Gp{\variable{x}}}$ is not equivalent to $\disjunction{\universal{\variable{x}}\Hp{\variable{x}}}{\universal{\variable{x}}\Gp{\variable{x}}}$.

To get around problems like this, we instead show we can move all the quantifiers to the front of the sentence.
This has the advantage of separating the quantifier parts of the logical structure from the SL parts.

\subsection{Prenex Definition and Steps}\label{Prenex Definition and Steps}
\begin{majorILnc}{\LnpDC{PrenexNF}}
A sentence $\CAPPHI$ of \GQL{} is in \df{prenex normal form} \Iff every quantifier is in initial position, or, in other words, the scope of all quantifiers is greater than that of any non-quantifier connective.
\end{majorILnc}
\begin{THEOREM}{\LnpTC{PrenexNFTheorem} Prenex Normal Form Theorem:}
For each sentence $\CAPTHETA$ of \GQL{}, there is a provably equivalent sentence $\CAPTHETA^*$ in prenex normal form; that is, $\CAPTHETA^*$ is in prenex normal form and $\sststile{}{}\triplebar{\CAPTHETA}{\CAPTHETA^*}$ in \GQD{}.
\end{THEOREM}
\begin{PROOF}
As with \CAPS{dnf} there are a set of steps for turning sentence $\CAPTHETA$ into a sentence $\CAPTHETA^*$ in prenex normal form. 
First we give the steps, and then show that each step can be sanctioned either by \Rule{QN}, \Rule{$\TRIPLEBAR$-Exchange}, or an exchange rule that can be introduced.
(We call these new exchange rules the \niidf{Prenex Exchange Rules}.\index{Exchange Rules!Prenex}) 
Because all the steps in the process are justified by exchange rules, we can either read the resulting series of steps top-down as a derivation of $\CAPTHETA^*$ from $\CAPTHETA$, or bottom-up as a derivation of $\CAPTHETA$ from $\CAPTHETA^*$. 
So, we'll have shown that $\sststile{}{}\triplebar{\CAPTHETA}{\CAPTHETA^*}$ in the derivation system consisting of \Rule{$\TRIPLEBAR$-Exchange} and the Prenex Exchange Rules.
But, as with all the other exchange rules anything that can be derived using the Prenex Exchange Rules can be derived in \GQD{} alone;
so, this is sufficient to show that $\sststile{}{}\triplebar{\CAPTHETA}{\CAPTHETA^*}$ in \GQD{}. First, the steps are:
\begin{cenumerate}
\item Replace biconditionals with disjunctions of conjunctions; i.e. replace $\triplebar{\CAPPHI}{\CAPPSI}$ with $\disjunction{\parconjunction{\CAPPHI}{\CAPPSI}}{\parconjunction{\negation{\CAPPHI}}{\negation{\CAPPSI}}}$.
\item Rewrite any variables that occur bound by more than one quantifier.
\item Move the first quantifier not in prenex position one step towards the front by the following principles. Repeat this step as often as necessary.  Keep in mind that you can't move forward a quantifier that binds the variable \mention{$\variable{x}$} if it has within its scope a new subformula that has a free \mention{$\variable{x}$}.  But we have prevented that problem by eliminating potentially clashing variables in Step 2.
\begin{longtable}[c]{ l l }
\toprule
\textbf{Replace} & \textbf{by} \\
\midrule
$\parconjunction{(\#\variable{x})\CAPTHETA}{\CAPPSI}$ & $(\#\variable{x})\parconjunction{\CAPTHETA}{\CAPPSI}$ \\
$\parconjunction{\CAPTHETA}{(\#\variable{x})\CAPPSI}$ & $(\#\variable{x})\parconjunction{\CAPTHETA}{\CAPPSI}$ \\

$\pardisjunction{(\#\variable{x})\CAPTHETA}{\CAPPSI}$ & $(\#\variable{x})\pardisjunction{\CAPTHETA}{\CAPPSI}$ \\
$\pardisjunction{\CAPTHETA}{(\#\variable{x})\CAPPSI}$ & $(\#\variable{x})\pardisjunction{\CAPTHETA}{\CAPPSI}$ \\

$\parhorseshoe{\CAPTHETA}{(\#\variable{x})\CAPPSI}$ & $(\#\variable{x})\parhorseshoe{\CAPTHETA}{\CAPPSI}$ \\

$\parhorseshoe{\existential{\variable{x}}\CAPTHETA}{\CAPPSI}$ & $\universal{\variable{x}}\parhorseshoe{\CAPTHETA}{\CAPPSI}$ \\
$\parhorseshoe{\universal{\variable{x}}\CAPTHETA}{\CAPPSI}$ & $\existential{\variable{x}}\parhorseshoe{\CAPTHETA}{\CAPPSI}$ \\

$\negation{\existential{\variable{x}}\CAPTHETA}$ & $\universal{\variable{x}}\negation{\CAPTHETA}$ \\
$\negation{\universal{\variable{x}}\CAPTHETA}$ & $\existential{\variable{x}}\negation{\CAPTHETA}$ \\
\bottomrule
\end{longtable}
Note that $(\#\variable{x})$ is just a dummy quantifier standing for either; 
replacement is the same for both quantifiers.  Also, we use \mention{$\variable{x}$} in the chart above, but the same principles hold for quantifiers with any other variable.
\end{cenumerate}
After applying these steps to a sentence $\CAPTHETA$ we get a sentence $\CAPTHETA^*$ that is in prenex normal form.\footnote{For more discussion of Prenex Form, see \citealt[132]{Kleene1967}, \citealt[54]{Hodges2001}, \citeyear[30]{Hodges2001b}.} 
We have to show that each step can be sanctioned by an exchange rule.
Step (1) is straightforward, since it is sanctioned by \Rule{$\TRIPLEBAR$-Exchange}.
But steps (2) and (3) we need new rules (although the replacements involving negations in (3) can be handled with \Rule{QN}).
The most straightforward strategy is to read the needed exchange rules right off the steps. 
Thus, the Prenex Exchange Rules are given in the following chart.
%\begin{table}[!ht]
%\renewcommand{\arraystretch}{1.5}
%\begin{center}
%\begin{tabular}{ p{1in} l l } %p{2.2in} p{2in}
%\toprule
%\textbf{Name} & \textbf{Given} & \textbf{May Add} \\ 
%\midrule
\renewcommand{\arraystretch}{1.5}
\begin{longtable}[c]{ p{1in} l l } %p{2.2in} p{2in}
\toprule
\textbf{Name} & \textbf{Given} & \textbf{May Add} \\ 
\midrule
\endfirsthead
\multicolumn{3}{c}{\emph{Continued from Previous Page}}\\
\toprule
\textbf{Name} & \textbf{Given} & \textbf{May Add} \\ 
\midrule
\endhead
\bottomrule
\caption{Prenex Exchange Short-Cut Rules for \GQD{}}\\[-.15in]
\multicolumn{3}{c}{\emph{Continued next Page}}\\
\endfoot
\bottomrule
\caption{Prenex Exchange Shortcut Rules for \GQD{}}\\
\endlastfoot
\label{GSDplusPrenex}\Rule{$\ALPHA$/$\BETA$-Exch} & $(\#\ALPHA)\CAPPHI$ & $(\#\BETA)\CAPPHI\BETA/\ALPHA$ \\
\Rule{Q Shuffling} & $\parconjunction{(\#\variable{x})\CAPTHETA}{\CAPPSI}$ & $(\#\variable{x})\parconjunction{\CAPTHETA}{\CAPPSI}$ \\
& $\parconjunction{\CAPTHETA}{(\#\variable{x})\CAPPSI}$ & $(\#\variable{x})\parconjunction{\CAPTHETA}{\CAPPSI}$ \\

& $\pardisjunction{(\#\variable{x})\CAPTHETA}{\CAPPSI}$ & $(\#\variable{x})\pardisjunction{\CAPTHETA}{\CAPPSI}$ \\
& $\pardisjunction{\CAPTHETA}{(\#\variable{x})\CAPPSI}$ & $(\#\variable{x})\pardisjunction{\CAPTHETA}{\CAPPSI}$ \\

& $\parhorseshoe{\CAPTHETA}{(\#\variable{x})\CAPPSI}$ & $(\#\variable{x})\parhorseshoe{\CAPTHETA}{\CAPPSI}$ \\

& $\parhorseshoe{\existential{\variable{x}}\CAPTHETA}{\CAPPSI}$ & $\universal{\variable{x}}\parhorseshoe{\CAPTHETA}{\CAPPSI}$ \\
& $\parhorseshoe{\universal{\variable{x}}\CAPTHETA}{\CAPPSI}$ & $\existential{\variable{x}}\parhorseshoe{\CAPTHETA}{\CAPPSI}$ \\
\end{longtable}
%\bottomrule
%\end{tabular}
%\end{center}
%\caption{Exchange Short-Cut Rules for \GSD{} (\GSD{})}
%\label{GSDplus2}
%\end{table}
\noindent{}Finally we show that anything that can be derived using the Prenex Exchange Rules can be derived using the basic rules of \GQD{} alone.
Recall from section \ref{Shortcut Rule Elimination Theorem Section} that all we need to do to show this is to prove the following:
\begin{THEOREM}{\LnpTC{GQD NDF Rule2}}
For all Prenex Exchange Rules \Rule{R}, any two \GQL{} formulas got by substituting other \GQL{} formulas into the may-add and given schemas of \Rule{R} are provably equivalent.
\end{THEOREM}
\noindent{}We leave the proof of this theorem to the reader, since as with the other exchange rules it just involves writing down the appropriate derivation schemas. 
\end{PROOF}

\subsection{The Strategy for Proving \GQD{} Completeness}
Our goal is to prove the strong completeness of \GQD{}: for any set $\Delta$ of \GQL{} sentences and \GQL{} sentence $\CAPPHI$, if $\Delta\sdtstile{}{}\CAPPHI$, then $\Delta\sststile{}{}\CAPPHI$. 
First we prove the completeness of \GQD{} and then show how to modify the method to prove strong completeness.
To prove completeness we show that for any sentence we can either (1) find a derivation of it or (2) prove that there is a model that makes it false. 
(This part of the strategy is more or less the same as what we did to show that \GSD{} is complete.)
In other words, $\CAPPHI$ is either derivable or not quantificationally true, from which it immediately follows that if $\CAPPHI$ is quantificationally true, then it is derivable. 

The method,\index{method, the} in brief, is to negate the sentence $\CAPPHI$ and begin a derivation.
Then we transform the negation of the sentence into prenex normal form, using the steps outlined in section \ref{Prenex Definition and Steps}. 
Next we transform the inner part of the sentence (remember the quantifiers are all up front) into \CAPS{dnf} form, using our standard method for that (see Sec. \pmvref{Disjunctive Normal Form}).
This does not introduce any new assumptions. 
We then systematically take instances of the bound variables and try to derive a contradiction.
If we can derive a contradiction we can then (assuming all goes well) use \Rule{$\NEGATION$-Elim} to obtain a derivation of $\CAPPHI$.

We must be very systematic since we have to be sure that if we get a contradiction we can derive it from the initial sentence $\CAPPHI$; and that if we do not get a contradiction we have not overlooked anything and that we can show the existence of a model making the sentence on the first line, $\CAPPHI$, true. 

It is important that we know the form of the sentence we have reached and are able to prescribe a uniform systematic method.
The sentence has been highly standardized; 
there are no biconditionals or conditionals (these have been eliminated in early steps of the transformation), negations govern only atomic sentences, and conjunctions govern only atomic sentences or their negations. 
These last, atomic sentence and their negations, are called \idf{ions}. 
We say an ion occurs \niidf{positively} \Iff it's an atomic sentence without a negation, and it occurs \niidf{negatively} \Iff it's a negated atomic sentence. 
We call the quantifier free part of the original sentence the \idf{matrix}. 
It is usually not a sentence since it may have free variables.
We call the sentences that are obtained from the matrix by substitution in the process of constructing the derivation \niidf{matrix instances}\index{matrix!instances|textbf}. 
To put some of our jargon together, the matrix of the sentence consists of disjunctions of conjunctions of ions. 

\subsection{The Method and Completeness Lemmas}\label{The Method Section}
In this section we describe the method sketched above.\index{method, the} 
Given a sentence $\CAPTHETA$, the Method either produces a derivation of $\CAPTHETA$ or indicates a model that makes it false:

\begin{description}
\item[Step 0:] Write $\negation{\CAPTHETA}$ on line 1 as an assumption.
Then first apply the prenex steps to put $\negation{\CAPTHETA}$ in Prenex Normal Form (\CAPS{pnf}). 
Next, apply the disjunctive normal form steps to the inner, quantifier-free part of the sentence until it's in \CAPS{dnf}. 
At this point we'll have a sentence $(\negation{\CAPTHETA})^*$ that's in what we'll call \idf{prenex disjunctive normal form}\index{disjunctive normal form!prenex} (\CAPS{pdnf}).  

\item[Step 1:] We continue the derivation operating on $(\negation{\CAPTHETA})^*$, the \CAPS{pdnf} of the sentence we're concerned with. 
If this \CAPS{pdnf} is a universal statement and contains no constants we write as the next line the instance of it we obtain by eliminating the quantifier and substituting the constant $\constant{a}$ for the previously bound variable;
these steps are sanctioned by \Rule{$\forall$-Elim}.
This step is only done once, whereas the next three steps generally require repeated recursive applications. 

\item[Step 2:] For every universal sentence that appears thus far in the derivation, we add \emph{all new instances} that can be formed with constants that occur earlier in the derivation;
these steps are sanctioned by \Rule{$\forall$-Elim}.
E.g., if $\universal{\variable{x}}\CAPPHI$ appears on a line and the constant $\constant{c}$ appears anywhere (earlier) in the derivation, then if we have not taken an instance of $\CAPPHI$ with $\constant{c}$ yet (i.e., $\CAPPHI\constant{c}/\variable{x}$), we do so.
As a practical matter, this means that it is useful in following the method to keep track somewhere of the constants used at each stage and of which constants have been used to instantiate which universal statements.
Note that in this step we are taking new instances with old constants and that we are not adding any new assumptions. 
We may, however, be adding new existentials. 

\item[Step 3:] For every existential sentence that appears in the derivation for which no instance has been added yet, add an instance using the first constant which \emph{does not occur in any previous assumption}. 
Note that \mention{instance} is to be taken very strictly here. 
The fact that we instantiated $\existential{\variable{x}}\Kpp{\variable{x}}{\constant{a}}$ with $\Kpp{\constant{b}}{\constant{a}}$ takes care of that existential, but if we later add the sentence $\existential{\variable{x}}\Kpp{\variable{x}}{\constant{b}}$ then we must add an instance of it. 
The rule which sanctions these steps is \Rule{Assume}. 
We eventually discharge these premises by \Rule{$\HORSESHOE$-Elim} and \Rule{$\exists$-Elim} if we get to a contradiction.
It is in anticipation of this eventuality that we carefully chose a constant which does not occur in any previous assumption.
Note that with this step we are adding new instances with new constants in new assumptions. 

\item[Step 4:] Determine whether the conjunction of the \emph{instances} of the matrix in the derivation thus far are contradictory. 
Officially, the way to do this is to take the conjunction of them all by \Rule{$\WEDGE\!$-Intro}, use \Rule{Distribution} to get the conjunction into \CAPS{dnf} and check whether every disjunct contains a contradiction. 
If so, then by a process of \Rule{$\VEE$-Elim} and \Rule{Any Contradiction} we can eventually produce the line $\conjunction{\Al}{\negation{\Al}}$. 

This step can be cumbersome in practice. We give some unofficial short cuts to make this more manageable soon.

\item[Step 5:] \hfill
\begin{cenumerate}
\item If the matrices are contradictory (see step 4) we stop.
\item Or if the conjunction of the matrix instances is consistent and the last applications of Steps 2 and 3 produce no new sentences, we stop.
\item Or if the conjunction of the matrix instance is consistent and the last applications of Steps 2 and 3 produced new sentences, then we return to Step 2 and reapply those steps.
\end{cenumerate}
\end{description}
There are three possible outcomes of applying this method to a sentence:
\begin{cenumerate}
\item The method reaches a contradiction.
\item The method stops without a contradiction.
\item The method generates new sentences perpetually without contradiction.
\end{cenumerate}
We show first that if a contradiction is reached we can construct a derivation of $\CAPTHETA$. 

\begin{THEOREM}{\LnpTC{Derivational Lemma} Derivational Lemma:}
If the Method starts with $\negation{\CAPTHETA}$ and produces a contradiction, then there is a derivation of $\CAPTHETA$.
\end{THEOREM}
\begin{PROOF}
Step 3 left us with $\conjunction{\Al}{\negation{\Al}}$ on a line with its assumptions being those of the matrices. 
We want to shift those assumptions so that we end up with the contradiction from the first assumption, $\negation{\CAPTHETA}$, alone.
We know by considering our method that other assumptions entered only by Step 2, where we added instances of existentials using new constants. 
We eliminate the last assumption by a \Rule{$\HORSESHOE$-Intro}. 
We know that this eliminated assumption introduced a \emph{new} constant from an existential. 
Therefore we know that this constant did not appear in any earlier assumption or in the existential of which we are taking an instance.
It also (obviously) does not occur in $\conjunction{\Al}{\negation{\Al}}$.
Thus we are allowed to use the rule \Rule{$\exists$-Elim} on the existential claim from which we assumed that instance. 

So we derive the contradiction $\conjunction{\Al}{\negation{\Al}}$ again, by \Rule{$\exists$-Elim}. 
We continue this process, repeating $\conjunction{\Al}{\negation{\Al}}$ as often as necessary to shift the dependence back to the assumption on line 1. 
This gives us a derivation of $\conjunction{\Al}{\negation{\Al}}$ from the first assumption, $\negation{\CAPTHETA}$, only. 

We then add two more lines: $\horseshoe{\negation{\CAPTHETA}}{\parconjunction{\Al}{\negation{\Al}}}$, sanctioned by \Rule{$\HORSESHOE$-Intro}, and $\CAPTHETA$, sanctioned by \Rule{$\NEGATION$-Elim}. 
Thus we have a derivation of $\CAPTHETA$ from no assumptions. 
\end{PROOF}

We have shown that if we obtain a contradiction in the derivation process we can derive the original sentence that interests us.


We must now show that if we do not obtain a contradiction (whether or not the method stops), then there is a model that makes $\negation{\CAPTHETA}$ true (and hence makes $\CAPTHETA$ false).

Before giving the rigorous version of the construction of the model, we present some of the ideas in a more concrete context. 
If we consider a sentence such as $\disjunction{\parconjunction{\Kp{\constant{a}}}{\negation{\Gp{\constant{b}}}}}{\parconjunction{\negation{\Kp{\constant{b}}}}{\Hp{\constant{c}}}}$ we can observe several things. 
First, each disjunct is satisfiable \Iff no ion occurs both positively and negatively in it. 
It is obvious that a conjunction that includes a sentence and its negation cannot be satisfied, but we can show for a conjunction of ions that that is the only way in which it can fail to be satisfiable. 
For example, we can make $\Kp{\constant{a}}$ and $\negation{\Gp{\constant{b}}}$ true by letting $\KK$ be interpreted as the set of even numbers, $\GG$ the set of numbers divisible by $10$ and letting \mention{$\constant{a}$} be assigned $2$ and \mention{$\constant{b}$} be assigned $7$. 

Of course several such sentences taken together produce different results. E.g., as we saw above $\disjunction{\parconjunction{\Kp{\constant{a}}}{\conjunction{\negation{\Kp{\constant{b}}}}{\negation{\Gp{\constant{c}}}}}}{\parconjunction{\Gp{\constant{b}}}{\negation{\Gp{\constant{c}}}}}$ is satisfiable, as is $\disjunction{\parconjunction{\Kp{\constant{b}}}{\conjunction{\negation{\Kp{\constant{c}}}}{\negation{\Gp{\constant{b}}}}}}{\parconjunction{\Gp{\constant{c}}}{\negation{\Gp{\constant{d}}}}}$, but the two together (taken as a conjunction) are not.
The reason is that while each disjunct of the first sentence is self-consistent, it cannot be true simultaneously with either of the disjuncts of the second sentence. If we have a series of disjunctions then they are simultaneously satisfiable only if we can find a way of picking a disjunct from each one in such a way that all the chosen disjuncts can be true together.

This is relevant to the task at hand because we know that all of the non-quantified sentences in our derivation are in \CAPS{dnf} and are thus disjunctions of conjunctions of ions.
We are calling the quantifier free part of the original sentence the matrix.
It is usually not a sentence since it may have free variables. 
The sentences that are obtained from the matrix by substitution in the process of constructing the derivation are the matrix instances. 
We use the notation $M_{i,j}$ for the disjuncts of the matrix instances, specifically the disjuncts of the first matrix instance are $M_{1,1},M_{1,2},\ldots,M_{1,m}$.
Thus the first matrix instance is $\disjunction{M_{1,1}}{\disjunction{M_{1,2}}{\disjunction{\ldots}{M_{1,m}}}}$.
The matrix instances that appear in the derivation can be listed in an array:
\begin{center}
\begin{tabular}{ c }
$\disjunction{M_{1,1}}{\disjunction{M_{1,2}}{\disjunction{\ldots}{M_{1,m}}}}$ \\
$\disjunction{M_{2,1}}{\disjunction{M_{2,2}}{\disjunction{\ldots}{M_{2,m}}}}$ \\
\\
\hspace{.5in} $\vdots$ \\
\\
$\disjunction{M_{n,1}}{\disjunction{M_{n,2}}{\disjunction{\ldots}{M_{n,m}}}}$ \\
\end{tabular}
\end{center}
Note that if the the method never stops, then this array is infinitely long. 

The matrices are jointly consistent \Iff there is a way of picking an $M_{i,j}$ from each matrix instance so that the conjunction of those $M_{i,j}$ contains no atomic sentence and its negation. 
In one direction this is easy to see: if there is no way of choosing a disjunct from each matrix instance that does not end up with an atomic sentence and its negation among the chosen sentences then the set of instances is inconsistent. 

To show that all instances are satisfiable when such a selection can be made without choosing a sentence and its negation takes some proving.
In order to do this we need to define the \idf{master matrix list} $M$.\index{matrix!master list} 
We first choose (if there is more than one) a set of disjuncts $M_{i,j}$ (including one from each matrix instance $M_i$) that does not contain any atomic sentence and its negation.
This is a set of conjunctions of atomic sentences and negations of atomic sentences.
Our master matrix list $M$ simply consists of all these atomic sentences and negated atomic sentences.
Note that since the $M_{i,j}$ selections must be consistent no atomic sentence that appears unnegated also appears negated.
\begin{majorILnc}{\LnpDC{MatrixModel}}
Given a master matrix list $M$, the \nidf{matrix model of $M$}\index{matrix!model} is the model $\IntA_M$ such that:
\begin{cenumerate}
\item The universe of $\IntA_M$ contains one natural number for each constant that appears in $M$, and $\IntA_M(\constant{a})=1$, $\IntA_M(\constant{b})=2$, $\IntA_M(\constant{c})=3$, $\IntA_M(\constant{d})=4$, and so on; $\IntA_M(\variable{t})=1$ for any constant $\variable{t}$ that doesn't appear in $M$. 
\item For each $\integer{m}$-place predicate $\PP$, $\IntA_M(\PP)$ is the set of $\integer{m}$-tuples of natural numbers $\langle\integer{n}_1,\ldots,\integer{n}_\integer{m}\rangle$ such that $\IntA_M(\variable{t}_1)=\integer{n}_1,\ldots,\IntA_M(\variable{t}_\integer{m})=\integer{n}_\integer{m}$ and $\Pp{\variable{t}_1\ldots\variable{t}_\integer{m}}$ appears on the list $M$.
\item Assignments are only made if justified by these principles.
\end{cenumerate}
\end{majorILnc}
A bit more informally, we list the constants that occur on the master matrix list. $\IntA_M$ has a universe that contains as many natural numbers as constants used.
We assign to each constant that occurs on the list the natural number that indicates its place in the order, i.e. $1$ to $\constant{a}$, $2$ to $\constant{b}$, and so on.
%Any constants not occurring on the master matrix list $M$ will be assigned $1$. 
Note that this produces a \idf{census}.
Each $1$-place predicate is assigned the set of numbers associated with the constants such that an instance of the predicate followed by that constant appears on the master matrix list $M$. 
Each $2$-place predicate is assigned the set of pairs of numbers associated with constants such that an instance of the predicate followed by that pair of constants appears on the master matrix list $M$. 
E.g., if $\Kpp{\constant{a}}{\constant{b}}$, $\Kpp{\constant{b}}{\constant{c}}$, and $\Kpp{\constant{d}}{\constant{e}}$ appear on $M$, then $\IntA_M(\KK)$ is assigned $\{\langle1,2\rangle,\langle2,3\rangle,\langle4,5\rangle\}$.
Assignments are made in a similar fashion for $\integer{n}$-placed predicates for $\integer{n}>2$.\footnote{Note 
that if the method never stops, then the master matrix list $M$ is infinite and we won't actually be able to write down the matrix model $\IntA_M$. 
But this isn't a problem, the matrix model $\IntA_M$ still exists, even if we can't write it down.} 
\begin{THEOREM}{\LnpTC{MethodLemmaA} The Method Lemma 1:}
The matrix model $\IntA_M$ makes true all sentences on the master matrix list $M$.
\end{THEOREM}
\begin{PROOF}
By construction, if an atomic sentence appears on the list we decided to put the relevant pair, triple, or whatever, of numbers in the set assigned to the predicate letter. 
For each negated atomic sentence on the list we know that we would not put the relevant pair, triple, or whatever, in the model of the predicate letter unless the atomic sentence which is being negated also appeared. 
But that never happened because $M$ is consistent by hypothesis.
\end{PROOF}
\begin{THEOREM}{\LnpTC{MethodLemmaB} The Method Lemma 2:}
All matrix instances in the derivation are true in the matrix model $\IntA_M$.
\end{THEOREM}
\begin{PROOF}
By Lemma 1 (Thm. \ref{MethodLemmaA}), all sentences on the master matrix list $M$ are true, and we included all the conjuncts of at least one disjunct $M_{i,j}$ from each matrix instance in forming the master list.
\end{PROOF}
\begin{THEOREM}{\LnpTC{MethodLemmaC} The Method Lemma 3:}
All quantified sentences in the derivation are true in the matrix model $\IntA_M$.
\end{THEOREM}

Informally, every universal has been instantiated with all the relevant constants, and every existential by at least one. 
Since $\IntA_M$ is a census, it follows that all the sentences in the derivation are true.

More rigorously...

\begin{PROOF}
We prove this lemma using a recursive proof on the number of quantifiers in each sentence.
\begin{description}
\item[Base Case:] 
The base case is the case of the sentences with $\integer{k}=0$ quantifiers. 
But these sentences are just the matrix instances in the derivation. 
We already proved in The Method Lemma 2 (Thm. \ref{MethodLemmaB}) that all these sentences are true the matrix model $\IntA_M$, so the base case is complete.

\item[Inheritance Step:] \hfill
\begin{description}
\item[Recursive Assumption:]
Our recursive assumption is that all sentences in the derivation with less than $\integer{k}$ quantifiers are true in the matrix model $\IntA_M$.

\item[Existential Quantifier:]
Say $\CAPTHETA$ is a sentence appearing in the derivation of the form $\existential{\ALPHA}\CAPPHI$, where $\CAPPHI$ is a formula with $\integer{k}-1$ quantifiers. 
Step 3 of the method guarantees that the sentence $\CAPPHI\variable{t}/\ALPHA$, for some constant $\variable{t}$, appears somewhere in the derivation. 
This sentence $\CAPPHI\variable{t}/\ALPHA$ has $\integer{k}-1$ quantifiers, so by the recursive assumption it is true in the matrix model $\IntA_M$. 
But then the existentially quantified sentence $\existential{\ALPHA}\CAPPHI$ is true in $\IntA_M$ as well.
(To show this rigorously, consider the $\variable{s}$-variant of $\IntA_M$, $\As{\variable{s}_1}{}$, that assigns the same element of the universe of $\IntA_M$ to $\variable{s}$ as $\IntA_M$ assigns to the constant $\variable{t}$.
Now by the Dragnet Theorem, Thm. \pncmvref{The Dragnet Theorem}, since $\IntA_M$ makes $\CAPPHI\variable{t}/\ALPHA$ true, the sentence $\CAPPHI\variable{s}/\ALPHA$ is true on $\As{\variable{s}_1}{}$.
It follows from this that $\existential{\ALPHA}\CAPPHI$ is true on $\IntA_M$.)

\item[Universal Quantifier:]
Say $\CAPTHETA$ is a sentence appearing in the derivation of the form $\universal{\ALPHA}\CAPPHI$, where $\CAPPHI$ is a formula with $\integer{k}-1$ quantifiers. 
All instances $\CAPPHI\variable{t}/\ALPHA$ which appear in the derivation have $\integer{k}-1$ quantifiers, and so by the recursive hypothesis are true in the matrix model $\IntA_M$. 
If we consider any $\variable{s}$-variant of $\IntA_M$, we know that what it assigns to $\variable{s}$ must be a number from the universe of $\IntA_M$;
we also know from the way that we constructed the matrix model $\IntA_M$ that a number was included in the universe of $\IntA_M$ only if it was assigned to some constant that occurred in the derivation.  
Let what's assigned to $\variable{s}$ by some $\variable{s}$-variant, $\As{\variable{s}_2}{}$, be the number associated with the constant $\constant{c}$.
Because the universal statement $\universal{\ALPHA}\CAPPHI$ occurred in the derivation, we know that we took all instances of it, including $\CAPPHI\constant{c}/\ALPHA$. 
As already stated, all instances $\CAPPHI\variable{t}/\ALPHA$ are true in $\IntA_M$, including $\CAPPHI\constant{c}/\ALPHA$.
By the Dragnet Theorem (Thm. \pncmvref{The Dragnet Theorem}), since $\CAPPHI\constant{c}/\ALPHA$ is true in $\IntA_M$ it follows that $\CAPPHI\variable{s}/\ALPHA$ is true on $\As{\variable{s}_2}{}$. 
But the exact same argument works for every $\variable{s}$-variant of $\IntA_M$; so $\CAPPHI\variable{s}/\ALPHA$ is true on every $\variable{s}$-variant of $\IntA_M$. 
It follows that $\universal{\ALPHA}\CAPPHI$ is true on the matrix model $\IntA_M$. 
\end{description}

\item[Closure Step:]
Every sentence in the derivation is true in the matrix model $\IntA_M$, which is what was to be shown. 
\end{description}
\end{PROOF}

\subsection{Proving Completeness}
In\index{completeness!weak \GQD{}} this section we put together all the pieces from the last section to prove that \GQD{} is complete. 
\begin{THEOREM}{\LnpTC{MainGQDWCompletenessLemma} Main Weak \GQD{} Completeness Lemma:}
For all sentences $\CAPTHETA$ of \GQL{}, if the method is applied to $\negation{\CAPTHETA}$ then either: (a) the method produces a derivation of $\CAPTHETA$ in \GQDP{}, or (b) there is some model $\IntA$ that makes $\CAPTHETA$ false.
\end{THEOREM}
\begin{PROOF}
If the method is applied to $\negation{\CAPTHETA}$, then either (1) it produces a contradiction $\conjunction{\Al}{\negation{\Al}}$, (2) the method halts without a contradiction, or (3) the method never halts (and hence never halts in a contradiction). 
If (1), then by the Derivational Lemma (Thm. \pmvref{Derivational Lemma}) there is a derivation of $\CAPTHETA$. 

If either (2) or (3) is the case, then by the Method Lemma 3 (Thm. \pmvref{MethodLemmaC}) we know that all sentences in the derivation starting with $(\negation{\CAPTHETA})^*$, the prenex disjunctive normal form sentence produced in Step 0 of the method from the sentence $\negation{\CAPTHETA}$ on line 1, are true in the matrix model $\IntA_M$. 
Note that all the steps in the derivation of $(\negation{\CAPTHETA})^*$ from $\negation{\CAPTHETA}$ are sanctioned by exchange rules;
therefore those steps can be turned upside down to produce a derivation in \GQDP{} of $\negation{\CAPTHETA}$ from $(\negation{\CAPTHETA})^*$. 
So by theorem \mvref{GQD Shortcut Theorem3} there's a derivation in \GQD{} of $\negation{\CAPTHETA}$ from $(\negation{\CAPTHETA})^*$.
Since \GQD{} is sound (Thm. \pmvref{Soundness of Quantifier Logic}), it follows that $(\negation{\CAPTHETA})^*\sdtstile{}{}\;\negation{\CAPTHETA}$.
Since we know that $(\negation{\CAPTHETA})^*$ is true in the matrix model $\IntA_M$, it follows that $\negation{\CAPTHETA}$ is true in $\IntA_M$ too.
So it follows that $\CAPTHETA$ is false in $\IntA_M$. 
\end{PROOF}
\begin{THEOREM}{\LnpTC{GQDWeakCompletenessTheorem} Weak \GQD{} Completeness Theorem:}
For all sentences $\CAPTHETA$ of \GQL{}, if $\sdtstile{}{}\CAPTHETA$, then $\sststile{}{}\CAPTHETA$ in \GQD{}.
\end{THEOREM}
\begin{PROOF}
Assume that $\sdtstile{}{}\CAPTHETA$. Then there are no models $\IntA$ which makes $\CAPTHETA$ false. 
Thus, if the method is applied to $\negation{\CAPTHETA}$ it can't be that some model $\IntA$ makes $\CAPTHETA$ false. 
By the Main \GQD{} Weak Completeness Lemma (Thm. \ref{MainGQDWCompletenessLemma}), it follows that when the method is applied to $\negation{\CAPTHETA}$ it produces a derivation of $\CAPTHETA$ in \GQDP{}. 
Hence there is a derivation of $\CAPTHETA$ in \GQD{}.
\end{PROOF}
\begin{THEOREM}{\LnpTC{GQDCompletenessTheorem} \GQD{} Completeness Theorem:}
For all finite sets $\Delta$ of \GQL{} sentences and \GQL{} sentence $\CAPPHI$, if $\Delta\sdtstile{}{}\CAPPHI$, then $\Delta\sststile{}{}\CAPPHI$.
\end{THEOREM}
\begin{PROOF}
The theorem follows immediately from the Weak \GQD{} Completeness Theorem and theorem \mvref{RegWeakCompletenessEquiv}.
\end{PROOF}

A consequence of our Strong Method  is that if $\CAPPHI$ is  entailed by an infinite set of sentences $\Delta$, it is entailed by and derivable from  a finite subset of $\Delta$.  If the Strong Method does not go on forever, then we get a contradiction at a finite stage and we have only assumed a finite subset of $\Delta$.

\subsection{Shortcut Rules for the Method}
The method discussed in section \ref{The Method Section} becomes practically unwieldy.
For example, if the matrix has three disjuncts with two sentences each, then combining two instances gives 9 disjuncts with 4 elements each, and combining three gives 27 disjuncts with 8 elements each. 
Thus we use some additional short cut rules to speed the process of detecting contradictions. 
(But note that \emph{two} of the shortcut Rules we add here are not \emph{exchange} shortcut rules.  Greg's rule is the exception.)

Our first shortcut rule is \Rule{Greg's Rule}. We\index{Greg's Rule} know that if a conjunction contains an atomic formula and the negation of that atomic formula then we can derive the negation of the conjunction.
E.g., we can derive the negation of $\parconjunction{\Kp{\constant{a}}}{\conjunction{\Gp{\constant{b}}}{\conjunction{\Kp{\constant{c}}}{\negation{\Gp{\constant{b}}}}}}$.
So if we have a disjunction, one disjunct of which contains a contradiction of this kind, we can derive the negation of that disjunct and use disjunctive syllogism to prune that disjunct.
For example, assume the Method ends at the sentence $\disjunction{\parconjunction{\Kp{\constant{a}}}{\conjunction{\Gp{\constant{b}}}{\conjunction{\Kp{\constant{c}}}{\negation{\Gp{\constant{b}}}}}}}{\parconjunction{\Kp{\constant{a}}}{\conjunction{\Gp{\constant{b}}}{\conjunction{\Kp{\constant{c}}}{\negation{\Gp{\constant{a}}}}}}}$. We can independently derive $\negation{\parconjunction{\Kp{\constant{a}}}{\conjunction{\Gp{\constant{b}}}{\conjunction{\Kp{\constant{c}}}{\negation{\Gp{\constant{b}}}}}}}$. From these two we derive  $\parconjunction{\Kp{\constant{a}}}{\conjunction{\Gp{\constant{b}}}{\conjunction{\Kp{\constant{c}}}{\negation{\Gp{\constant{a}}}}}}$. 
Greg's Rule lets us accomplish those steps by crossing out the contradictory part and writing down the remaining ones.

It\index{$\VEE$/$\WEDGE$-Elim} is helpful to have a short cut rule which combines \Rule{$\WEDGE\!$-Elim} steps with \Rule{$\VEE$-Elim} steps to go from a disjunction of which each disjunct contains a particular sentence to that sentence itself on a later line;
we call it \Rule{$\VEE$/$\WEDGE\!$-Elim} and it sanctions the step from $\disjunction{\parconjunction{\Kp{\constant{a}}}{\conjunction{\Gp{\constant{b}}}{\conjunction{\Kp{\constant{c}}}{\negation{\Gp{\constant{c}}}}}}}{\parconjunction{\Kp{\constant{a}}}{\conjunction{\Gp{\constant{b}}}{\conjunction{\Kp{\constant{c}}}{\negation{\Gp{\constant{a}}}}}}}$ to $\Gp{\constant{b}}$. 
In addition to citing the justification, if the sentence is at all complex you should circle the repeated subsentence.

Finally,\index{One Bad Apple} given the opposite of even one conjunct in a conjunction, we can derive the negation of the conjunction.
E.g., from $\Gp{\constant{a}}$ we can derive $\negation{\parconjunction{\Kp{\constant{a}}}{\conjunction{\Gp{\constant{b}}}{\conjunction{\Kp{\constant{c}}}{\negation{\Gp{\constant{a}}}}}}}$.
We call this rule \Rule{One Bad Apple}, or \Rule{OBA}.
%\begin{table}[!ht]
%\renewcommand{\arraystretch}{1.5}
%\begin{center}
%\begin{tabular}{ p{1in} l l } %p{2.2in} p{2in}
%\toprule
%\textbf{Name} & \textbf{Given} & \textbf{May Add} \\ 
%\midrule
\renewcommand{\arraystretch}{1.5}
\begin{longtable}[c]{ p{1in} l l } %p{2.2in} p{2in}
\toprule
\textbf{Name} & \textbf{Given} & \textbf{May Add} \\ 
\midrule
\endfirsthead
\multicolumn{3}{c}{\emph{Continued from Previous Page}}\\
\toprule
\textbf{Name} & \textbf{Given} & \textbf{May Add} \\ 
\midrule
\endhead
\bottomrule
\caption{Short-Cut Rules for the Method}\\[-.15in]
\multicolumn{3}{c}{\emph{Continued next Page}}\\
\endfoot
\bottomrule
\caption{Short-Cut Rules for the Method}\\
\endlastfoot
\label{GSDplusMethod}\Rule{Greg's Rule} & $\disjunction{\CAPPSI_1}{\disjunction{\ldots}{\CAPPSI_{\integer{n}}}}$, where some & $\disjunction{\CAPPSI_1}{\disjunction{\ldots}{\disjunction{\CAPPSI_{\integer{i}-1}}{\disjunction{\CAPPSI_{\integer{i}+1}}{\disjunction{\ldots}{\CAPPSI_{\integer{n}}}}}}}$ \\[-.25cm]
 & $\CAPPSI_{\integer{i}}=\conjunction{\CAPPHI_1}{\conjunction{\ldots}{\conjunction{\CAPPHI_{\integer{j}}}{\ldots}}}$ & \\[-.25cm]
\nopagebreak
 & $\WEDGE\conjunction{\negation{\CAPPHI_{\integer{j}}}}{\conjunction{\ldots}{\CAPPHI_{\integer{m}}}}$ & \\
 
\Rule{$\VEE$/$\WEDGE\!$-Elim} & $\disjunction{\CAPPSI_{1}}{\disjunction{\ldots}{\CAPPSI_{\integer{n}}}}$, where & $\CAPPHI$ \\[-.25cm]
 & each $\CAPPSI_{\integer{i}}$ contains $\CAPPHI$ & \\
 
\Rule{OBA} &  $\conjunction{\CAPPHI_1}{\conjunction{\ldots}{\conjunction{\CAPPHI_i}{\conjunction{\ldots}{\CAPPHI_{\integer{n}}}}}}$, $\negation{\CAPPHI_i}$ & $\negation{\parconjunction{\CAPPHI_1}{\conjunction{\ldots}{\conjunction{\CAPPHI_i}{\conjunction{\ldots}{\CAPPHI_{\integer{n}}}}}}}$ \\
\nopagebreak
 & $\conjunction{\CAPPHI_1}{\conjunction{\ldots}{\conjunction{\negation{\CAPPHI_i}}{\conjunction{\ldots}{\CAPPHI_{\integer{n}}}}}}$, ${\CAPPHI_i}$ & $\negation{\parconjunction{\CAPPHI_1}{\conjunction{\ldots}{\conjunction{\negation{\CAPPHI_i}}{\conjunction{\ldots}{\CAPPHI_{\integer{n}}}}}}}$ \\
\end{longtable}
%\bottomrule
%\end{tabular}
%\end{center}
%\caption{Exchange Short-Cut Rules for \GSD{} (\GSDP{})}
%\label{GSDplus2}
%\end{table}

\section{Strong Completeness and Other Results}\label{Sec:Proving Strong Completeness}
In this section we want to extend our results and show that \GQD{} is strongly complete.
Note that the method we used to extend the Weak Completeness Theorem to the Completeness Theorem does not work here.
To do that, we used theorem \mvref{RegWeakCompletenessEquiv}, the proof of which depending on $\Delta$ being finite.  
To show that \GQD{} is strongly complete, we have modify the method we used to show that it's weakly complete.
\begin{THEOREM}{\LnpTC{GQDStrongCompletenessTheorem} Strong \GQD{} Completeness Theorem:}
For any set $\Delta$ of \GSL{} sentences and any \GSL{} sentence $\CAPPHI$, if $\Delta\sdtstile{}{}\CAPPHI$, then $\Delta\sststile{}{}\CAPPHI$.
\end{THEOREM}
\noindent{}To show that \GQD{} is weakly complete, we gave a method that, given a sentence $\CAPTHETA$, either produces a derivation of $\CAPTHETA$ or produces a model $\IntA$ which makes $\CAPTHETA$ false. 
To show that \GQD{} is strongly complete, what we want is a method that, given a (possibly infinite) set $\Delta$ of sentences and another sentence $\CAPPHI$, either produces a derivation of a contradiction $\conjunction{\Al}{\negation{\Al}}$ from $\negation{\CAPPHI}$ and some finite subset of $\Delta$ or produces a model $\IntA$ that makes $\negation{\CAPPHI}$ and every sentence in $\Delta$ true.

The method we'll give is a modification of the original method given in section \ref{The Method Section}. 
Since it is just a modification of the the original method, we'll only sketch the changes needed. 
We'll call the modified method the \niidf{strong method}\index{strong method, the}\index{method, the!strong}. 
Given some possibly countably infinite set $\Delta$ and sentence $\CAPPHI$, the strong method is:
\begin{description}
\item[Step 1:] Let $\Delta^*=\Delta\cup\{\negation{\CAPPHI}\}$. 
Then pair each sentence of $\Delta^*$ with a natural number and use that to determine the order in which they are assumed. 
The only constraint on this ordering is that $\negation{\CAPPHI}$ should be first.

\item[Step 2:] Put the first sentence of $\Delta^*$ on line 1 and put it in \CAPS{pdnf}, just as was done in Step 0 of the method.

\item[Step 3:] Apply Step 1 of the method.

\item[Step 4:] Apply Steps 2 and 3 of the method to the whole derivation thus far.

\item[Step 5:] Check for contradictions, just as in Step 4 of the method.

\item[Step 6:] \hfill
\begin{cenumerate}
\item If there's a contradiction, stop.
\item If there's no contradiction, write the next sentence of $\Delta^*$ on the next line of the derivation, put that sentence in \CAPS{pdnf}, and go back into Step 4. 
\end{cenumerate}
\end{description}
The strong method either halts in a contradiction, or it does not. 
\begin{THEOREM}{\LnpTC{DerivationalLemmaS} Strong Derivational Lemma:}
If the strong method halts in a contradiction, then $\Delta\sststile{}{}\CAPPHI$.
\end{THEOREM}
\begin{PROOF}
If the strong method halts in a contradiction, then it has produced a derivation of a contradiction $\conjunction{\Al}{\negation{\Al}}$ from $\negation{\CAPPHI}$ and some subset $\Delta'$ of $\Delta$.  We want to show that $\Delta\sststile{}{}\CAPPHI$; to do this, we first want to show that $\Delta'\sststile{}{}\CAPPHI$.

We might think we already know that $\negation{\CAPPHI},\Delta'\sststile{}{}\conjunction{\Al}{\negation{\Al}}$, but in fact, there are additional open assumptions that we made; the Strong Method tells us to make an additional assumption for each line containing an existentially quantified sentence. For these lines, we assume an instance of the existentially quantified sentence.  We want to discharge these assumptions by using \Rule{$\exists$-Elim}, but these assumptions may come prior to some of the assumptions we made from $\Delta'$.  We want to keep the sentences of $\Delta'$ as assumptions, because the contradiction we reached depends on them. 

We can discharge the assumed instances of existentially quantified sentences only by additionally discharging all the assumptions that come later in the derivation.  Accordingly, we want to extend our derivation so that we discharge \emph{all} our assumptions and then repeat all the assumptions \emph{except} for the instances of existentially quantified sentences. 

Let $\CAPTHETA_1, \CAPTHETA_2, \CAPTHETA_3, \ldots, \CAPTHETA_{\integer{n}}$ be the sentences of $\Delta'$ assumed in our derivation.  The last open assumption is either (a) the last sentence of $\Delta'$, i.e., $\CAPTHETA_{\integer{n}}$; (b) an instance of an existentially quantified sentence on an earlier line of the derivation which we'll call $\CAPPSI$; or (c) $\negation{\CAPPHI}$.  If (a) is the case, then discharge the assumption by applying the rule \Rule{$\HORSESHOE$-Intro} to get $\horseshoe{\CAPTHETA_{\integer{n}}}{\parconjunction{\Al}{\negation{\Al}}}$.  If (b) is the case, then first apply \Rule{$\HORSESHOE$-Intro} to get $\horseshoe{\CAPPSI}{\parconjunction{\Al}{\negation{\Al}}}$, and then apply \Rule{$\exists$-Elim} to get $\conjunction{\Al}{\negation{\Al}}$.  When (b) is the case, we know that the assumption introduced a \emph{new} constant, and therefore we know that that constant did not appear in any earlier assumption or in the existential of which we are taking an instance.
It also (obviously) does not occur in $\conjunction{\Al}{\negation{\Al}}$.
Thus the \Rule{$\exists$-Elim} step is legitimate. If (c) is the case then we don't have to worry about instances of existentially quantified sentences, and we can skip this part of the process.  Let us disregard case (c) for now.

Now we must continue to discharge the rest of the assumptions.  For any assumption of an instance of an existentially quantified sentence, we may do the same thing we did in (b) above---first use \Rule{$\HORSESHOE$-Intro} to derive a conditional, and then use \Rule{$\exists$-Elim} to derive the RHS of that conditional.  For any assumption that is a sentence of $\Delta'$ (i.e., $\CAPTHETA_{\integer{i}}$), we use \Rule{$\HORSESHOE$-Intro} to derive a conditional, as in (a) above.  So, after discharging the assumption with the second to last sentence from $\Delta'$ we get $\horseshoe{\CAPTHETA_{\integer{n-1}}}{\parhorseshoe{\CAPTHETA_{\integer{n}}}{\parconjunction{\Al}{\negation{\Al}}}}$.  After the third instance, we derive $\horseshoe{\CAPTHETA_{\integer{n-2}}}{\parhorseshoe{\CAPTHETA_{\integer{n-1}}}{\parhorseshoe{\CAPTHETA_{\integer{n}}}{\parconjunction{\Al}{\negation{\Al}}}}}$.  And so on, so that we eventually get a sentence of the form $\horseshoe{\CAPTHETA_1}{\parhorseshoe{\CAPTHETA_2}{\parhorseshoe{\CAPTHETA_3}{\ldots \parhorseshoe{\CAPTHETA_{\integer{n}}}{\parconjunction{\Al}{\negation{\Al}}}}}}$.

After discharging all such assumptions, the only open assumption left is $\negation{\CAPPHI}$.  Apply \Rule{$\HORSESHOE$-Intro} once more to get something like the following: \\
$\horseshoe{\negation{\CAPPHI}}{\parhorseshoe{\CAPTHETA_1}{\parhorseshoe{\CAPTHETA_2}{\ldots \parhorseshoe{\CAPTHETA_{\integer{n}}}{\parconjunction{\Al}{\negation{\Al}}}}}}$.  Now we have no open assumptions remaining.  (Note that we have effectively covered case (c) above, since applying \Rule{$\HORSESHOE$-Intro} in this case would give us $\horseshoe{\negation{\CAPPHI}}{\parconjunction{\Al}{\negation{\Al}}}$ and no open assumptions.  In this case, we didn't have to assume any of the sentences of $\Delta$ to get a contradiction, so $\Delta'$ is the empty set.)

At this point we have a conditional, possibly a very long one.  The RHS of the last conditional (possibly embedded in several conditionals) is our contradiction, $\parconjunction{\Al}{\negation{\Al}}$.  We want to show that $\Delta'\sststile{}{}\CAPPHI$, so let us make a series of assumptions from the sentences of $\Delta'$.  That is, let us assume each of $\CAPTHETA_1, \CAPTHETA_2, \CAPTHETA_3, \ldots, \CAPTHETA_{\integer{n}}$.  Now that we've assumed all the sentences of $\Delta'$, let us assume $\negation{\CAPPHI}$.

Given our earlier conditional of the form $\horseshoe{\negation{\CAPPHI}}{\parhorseshoe{\CAPTHETA_1}{\parhorseshoe{\CAPTHETA_2}{\ldots, \parhorseshoe{\CAPTHETA_{\integer{n}}}{\parconjunction{\Al}{\negation{\Al}}}}}}$ and the assumption $\negation{\CAPPHI}$, we may now apply \Rule{$\HORSESHOE$-Elim} to get \\
$\horseshoe{\CAPTHETA_1}{\parhorseshoe{\CAPTHETA_2}{\parhorseshoe{\CAPTHETA_3}{\ldots, \parhorseshoe{\CAPTHETA_{\integer{n}}}{\parconjunction{\Al}{\negation{\Al}}}}}}$.  Because we also have all the sentences of $\Delta'$ as assumptions (i.e., all of $\CAPTHETA_1, \CAPTHETA_2, \CAPTHETA_3, \ldots, \CAPTHETA_{\integer{n}}$), we may apply a series of \Rule{$\HORSESHOE$-Elim} steps until we eventually derive $\conjunction{\Al}{\negation{\Al}}$.  That is, given our earlier assumption $\CAPTHETA_1$ and the conditional $\horseshoe{\CAPTHETA_1}{\parhorseshoe{\CAPTHETA_2}{\parhorseshoe{\CAPTHETA_3}{\ldots, \parhorseshoe{\CAPTHETA_{\integer{n}}}{\parconjunction{\Al}{\negation{\Al}}}}}}$, we may apply \Rule{$\HORSESHOE$-Elim} to derive $\horseshoe{\CAPTHETA_2}{\parhorseshoe{\CAPTHETA_3}{\ldots, \parhorseshoe{\CAPTHETA_{\integer{n}}}{\parconjunction{\Al}{\negation{\Al}}}}}$.  And then because we have $\CAPTHETA_2$ as an assumption we may again apply \Rule{$\HORSESHOE$-Elim} to get $\horseshoe{\CAPTHETA_3}{\ldots, \parhorseshoe{\CAPTHETA_{\integer{n}}}{\parconjunction{\Al}{\negation{\Al}}}}$.  And so on, until we derive $\conjunction{\Al}{\negation{\Al}}$.

Remember that our last open assumption is $\negation{\CAPPHI}$.  We may now discharge that assumption and apply \Rule{$\HORSESHOE$-Intro} to get $\horseshoe{\negation{\CAPPHI}}{\parconjunction{\Al}{\negation{\Al}}}$.  Then we apply \Rule{$\NEGATION$-Elim} to derive $\CAPPHI$.

Now we have as our open assumptions only the sentences of $\Delta'$ and we have derived $\CAPPHI$.  We have thus shown that $\Delta'\sststile{}{}\CAPPHI$.  And because $\Delta'$ is a subset of $\Delta$, it follows that $\Delta\sststile{}{}\CAPPHI$.
\end{PROOF}
\noindent{}Next, note that if the strong method doesn't halt in a contradiction, then we have a list of matrix instances from which we can construct a matrix model $\IntA_M$ in just the same way we did for the method (Def. \pmvref{MatrixModel}).
Similar to the method, we have the following three theorems.
\begin{THEOREM}{\LnpTC{MethodSLemmaA} The Strong Method Lemma 1:}
The matrix model $\IntA_M$ makes true all sentences on the master matrix list $M$.
\end{THEOREM}
\begin{PROOF}
The same proof used for the method (Thm. \pmvref{MethodLemmaA}) applies here too.
\end{PROOF}
\begin{THEOREM}{\LnpTC{MethodSLemmaB} The Strong Method Lemma 2:}
All matrix instances in the derivation are true in the matrix model $\IntA_M$.
\end{THEOREM}
\begin{PROOF}
The same proof used for the method (Thm. \pmvref{MethodLemmaB}) applies here too.
\end{PROOF}
\begin{THEOREM}{\LnpTC{MethodSLemmaC} The Strong Method Lemma 3:}
All quantified sentences in the derivation are true in the matrix model $\IntA_M$.
\end{THEOREM}
\begin{PROOF}
The same proof used for the method (Thm. \pmvref{MethodLemmaC}) applies here too, so long as we can show that if an existential $\existential{\ALPHA}\CAPPSI$ appears on a line, at least one instance $\CAPPSI\variable{t}/\ALPHA$ does, and if a universal $\universal{\ALPHA}\CAPPSI$ appears on a line, then every instance $\CAPPSI\variable{t}/\ALPHA$ of it with a constant $\variable{t}$ appearing somewhere in the derivation appears somewhere in the derivation. So, all we need to show is that the strong method derives the appropriate instances of all quantified sentences that appear in our derivation. Let $\Delta'$ be those sentences of $\Delta^*$ that appear in our derivation as a result of the application of the strong method.

\begin{description}
\item[Existential Quantifier:] 
Say $\CAPTHETA$ is some sentence in the derivation of the form $\existential{\ALPHA}\CAPPHI$. 
Step 4 of the strong method uses step 3 of the method on $\CAPTHETA$, which guarantees that the sentence $\CAPPHI\variable{t}/\ALPHA$, for some constant $\variable{t}$, appears somewhere in the derivation.
By hypothesis, step 4 of the strong method is applied to all sentences in $\Delta'$, so we know that it derives the appropriate instances of all existentially quantified statements in in $\Delta'$.

\item[Universal Quantifier:]
Say $\CAPTHETA$ is a sentence appearing in the derivation of the form $\universal{\ALPHA}\CAPPHI$. 
Step 4 of the strong method uses step 2 of the method on $\CAPTHETA$, which, for every constant $\variable{t}$ in the derivation, guarantees that the sentence $\CAPPHI\variable{t}/\ALPHA$ is derived.
By hypothesis, step 4 of the strong method is applied to all sentences in $\Delta'$, so we know that it derives the appropriate instances of all universally quantified statements in $\Delta'$.
\end{description}

\noindent{}So, the strong method derives the appropriate instances of all quantified sentences in $\Delta'$.

%(Why? assume not. Then say that universal is PHI and that constant is b. Since the derivation is never ending, there must have been a pass of steps 1 and 2 that comes after both the pass that introduced b and the pass that introduced PHI. So contra assumption, this pass put that instance of PHI in the sequence.).
\end{PROOF}
\noindent{}Finally, we have one last lemma:
\begin{THEOREM}{\LnpTC{MainGQDSCompletenessLemma} Main Strong \GQD{} Completeness Lemma:}
For all sets of \GQL{} sentences $\Delta$ and \GQL{} sentences $\CAPPHI$, if the strong method is applied to $\Delta^*=\Delta\cup\{\negation{\CAPPHI}\}$ then either: (a) the strong method produces a derivation of $\CAPPHI$ from $\Delta$ in \GQDP{}, or (b) there is a model $\IntA$ which makes every sentence in $\Delta$ true and $\CAPPHI$ false.
\end{THEOREM}
\begin{PROOF}
If the method is applied to $\Delta^*=\Delta\cup\{\negation{\CAPPHI}\}$, then either it halts in a contradiction or not. 
By the Strong Derivational Lemma (\pmvref{DerivationalLemmaS}), if the strong method halts in a contradiction, then $\Delta\sststile{}{}\CAPPHI$ in \GQDP{}.

If the method does not halt in a contradiction, then by the Strong Method Lemma 3 (Thm. \pmvref{MethodSLemmaC}) the matrix model $\IntA_M$  makes all the sentences in the derivation true. 
But since the strong method did not halt in a contradiction, for every sentence $\CAPPSI$ in $\Delta$ and $\negation{\CAPPHI}$, there's some sentence in \CAPS{pdnf} that's quantificationally equivalent to $\CAPPSI$ and appears in the derivation. 
So $\IntA_M$ makes all the sentences in $\Delta$ and the sentence $\negation{\CAPPHI}$ true; 
hence $\IntA_M$ makes all the sentences in $\Delta$ true and $\CAPPHI$ false. 
\end{PROOF}
\begin{PROOFOF}{Thm. \ref{GQDStrongCompletenessTheorem}, The Strong Completeness Theorem for GQD}
Assume that $\Delta\sdtstile{}{}\CAPPHI$. 
Then there can be no model $\IntA$ which makes all of the sentences in $\Delta$ true and $\CAPPHI$ false;
so, application of the method can't produce such a model.
Thus by the Main \GQD{} Strong Completeness Lemma (Thm. \ref{MainGQDSCompletenessLemma}), $\Delta\sststile{}{}\CAPPHI$ in \GQDP{}. 
It follows by theorem \mvref{GQD Shortcut Theorem3} that $\Delta\sststile{}{}\CAPPHI$ in \GQD{}.
\end{PROOFOF}

Our Method for proving completeness for QD is short of being a decision procedure.  If $\CAPPHI$ is a logical truth, then The Method produces a derivation of it.  And if $\CAPPHI$ is not a logical truth, then in many cases it produces a model that makes $\CAPPHI$ false.  But sometimes it just doesn't stop.  We know that if it doesn't stop there is a model that makes the original sentence false, but at each stage we may not know whether it stops (soon?) or not.   And we know that it can't stop at a finite stage for some sentences because those sentences are only false in an infinite model.

All we know from our work so far is that The Method works as described above. We don't know that there isn't a better method that provides a decision procedure.  Church's Theorem, proved by more advanced methods that involve clarifying what counts as a ``method'' or ``algorithm'' tells us that our result is as good as we can do for all of \GQL{}.

However, we can do better for the language \GQL{}1 and a little more.

%%%%%%%%%%%%%%%%%%%%%%%%%%%%%%%%%%%%%%%%%%%%%%%%%%
\section{Decidability and Church's Theorem}\label{Decidability and Churchs Theorem}
%%%%%%%%%%%%%%%%%%%%%%%%%%%%%%%%%%%%%%%%%%%%%%%%%%

Next\index{decidable}\index{undecidable} we turn to refinements of the method to obtain what are called \niidf{decision procedures} for logical truth in a language \Language{L}. %\index{decision procedure}
We first introduced the idea of a decision procedure in section \mvref{Section:Intro to Decidability}; here we shall fill things out a bit further. 
\begin{majorILnc}{\LnpDC{Def:DecisionProcedure}}
A \df{decision procedure} for logical truth in a language (or sublanguage) \Language{L} is a completely specified method which produces, for any sentence $\CAPPHI$ of \Language{L} and in a finite number of steps, the answer YES if $\CAPPHI$ is a logical truth and the answer NO otherwise.
\end{majorILnc}
\begin{majorILnc}{\LnpEC{TruthTableDecisionProcedure}}
We have already seen two decision procedures for \CAPS{tft} in \GSL{}: truth tables and the method discussed in the completeness proof of \GSD{}. 
(Others include truth trees and Quine's ``fell swoop'', \citealt{Quine1950}, \citealt[23]{Hodges2001}.)
The truth-table decision procedure is simple:\index{decision procedure!truth table} take a sentence $\CAPPHI$ of \GSL{} and construct a truth table for it. If you get all $\TrueB{}$ in the column under $\CAPPHI$; answer YES. If you don't, answer NO. 
Likewise for the method discussed in the completeness proof of \GSD{}:\index{decision procedure!for \CAPS{tft} in \GSL{}} take a sentence $\CAPPHI$, negate it to get $\negation{\CAPPHI}$, and apply the method. 
If it results in a contradiction $\conjunction{\Al}{\negation{\Al}}$, answer YES. 
If no contradiction is reached, answer NO. 
\end{majorILnc}
\noindent{}Our method of proving completeness for \GQD{} is a little short of being a decision procedure for quantificational truth in \GQL{} because it does not always produce an answer in a finite amount of time.
It can be shown that there is no decision procedure for the whole language \GQL{} \citetext{\citealp[83--86]{Hodges2001}, \citeyear[31]{Hodges2001b}, \citealp[486]{Bergmann2003}}.
\begin{THEOREM}{\LnpTC{ChurchsTheorem} Church's Theorem:}
If\index{Church's Theorem}\index{decision procedure!for \CAPS{qt} in \GQL{}|see{Church's Theorem}} \Language{L} is a sublanguage of \GQL{} with (1) the same logical connectives as \GQL{}, and (2) at least one 2-place predicate symbol, then there is no decision procedure for the set of logical truths of \Language{L}.\footnote{Actually, 
Church's Theorem also says that if we also consider languages with function symbols, then if \Language{L} has at least two 1-place function symbols there is no decision procedure for the set of logical truths of \Language{L}.}
\end{THEOREM}
Church's Theorem at once tells us that there is no decision procedure for quantificational truth in \GQL{}, but we can show that with some modifications our method from section \ref{The Method Section} can be turned into a decision procedure for certain sublanguages of \GQL{}. 
As the theorem suggests, one such sublanguage of \GQL{} consists of just 1-place predicate symbols. 
We've been calling this language \GQL{}1.\index{decision procedure!for \CAPS{qt} in monadic \GQL{}}\index{GQL!monadic}

Let's modify the method to prove that \GQL{}1 has a decision procedure.
The problem with the existing method is that infinite loops are created by having an existential quantifier inside a universal quantifier.
The existential requires a new constant to be instantiated, and that creates a new potential instance for the universal, which then creates a new existential, and so on.
We want a procedure that is guaranteed to halt.

If the method produces a sentence in standardized form which has only existential, or only universal quantifiers, then the method stops. 
Moreover, if in the standardized form the existentials all precede the universals the method also stops, for we first take instances of all the existentials using $\integer{n}$ constants if there are $\integer{n}$ existentials, and afterwards we instantiate the $\integer{n}$ new constants in the $\integer{m}$ universals giving $\integer{n}^\integer{m}$ instances.
We are done then, since no additional constants are added.

But what about sentences with an existential quantifier within the scope of a universal quantifier, e.g. $\universal{\variable{x}}\existential{\variable{y}}\pardisjunction{\parconjunction{\Bp{\variable{x}}}{\Cp{\variable{y}}}}{\parconjunction{\Dp{\variable{x}}}{\Gp{\variable{y}}}}$?
Since \GQL{} has many-place predicates, if one quantifier occurs within the scope of another then it often cannot be moved in front of the other quantifier. 
But since the predicates of \GQL{}1 are 1-place there is a procedure to switch quantifier order in all cases.
It takes some work to establish this result.

\begin{majorILnc}{\LnpDC{Independent}}
Two quantifiers are \df{independent}\index{quantifier!s, independent} \Iff neither is in the scope of the other.
\end{majorILnc}

Let's say we have a sentence all of whose quantifiers are independent of each other.
Then these quantifiers can be brought forward in any order, and so we have a decision procedure for such sentences. 
We claim that each \GQL{}1 sentence is equivalent to one whose quantifiers are all independent.
To simplify the proof for this we introduce some additional exchange rules.

\subsection{Additonal Exchange Rules}\label{Additonal Exchange Rules}

Our aim is to move the quantifiers of a \GQL{}1 sentence so that all are independent.
To simplify this task we introduce new quantifier exchange rules, as well as some rules that permit rearranging the order of conjunctions and disjunctions.
These rules allow us to provide a procedure to push each quantifier in far enough so that the scope of each covers the variables it binds and no others.

\renewcommand{\arraystretch}{1.5}
\begin{longtable}[c]{ p{1in} l l } %p{2.2in} p{2in}
\toprule
\textbf{Name} & \textbf{Given} & \textbf{May Add} \\ 
\midrule
\endfirsthead
\multicolumn{3}{c}{\emph{Continued from Previous Page}}\\
\toprule
\textbf{Name} & \textbf{Given} & \textbf{May Add} \\ 
\midrule
\endhead
\bottomrule
\caption{Additonal Exchange Rules for \GQD{}}\\[-.15in]
\multicolumn{3}{c}{\emph{Continued next Page}}\\
\endfoot
\bottomrule
\caption{Additonal Exchange Rules for \GQD{}}\\
\endlastfoot
\label{AdditionalQuantifierExRules}\Rule{$\universal$-Shuffle} & $\universal\variable{\alpha}\parconjunction{\CAPPHI_1}{\conjunction{\ldots}{\CAPPHI_n}}$ & $\conjunction{\universal\variable{\alpha}\CAPPHI_1}{\conjunction{\ldots}{\universal\variable{\alpha}\CAPPHI_n}}$ \\
\Rule{$\existential$-Shuffle} & $\existential\variable{\alpha}\pardisjunction{\CAPPHI_1}{\disjunction{\ldots}{\CAPPHI_n}}$ & $\par\disjunction{\existential\variable{\alpha}\CAPPHI_1}{\disjunction{\ldots}{\existential\variable{\alpha}\CAPPHI_n}}$ \\
\Rule{Q-Deletion} & $\#\variable{\alpha}\CAPPHI$ & $\CAPPHI$, iff $\alpha$ does  \\[-.25cm]
\nopagebreak
 &  &  not occur in $\CAPPHI$  \\
\Rule{$\wedge$-Shuffle} & $\conjunction{\CAPPHI_1}{\conjunction{\ldots}{\conjunction{\CAPPHI_k}{\conjunction{\ldots}{\CAPPHI_n}}}}$ & $\conjunction{\CAPPHI_k}{\conjunction{\CAPPHI_1}{\conjunction{\ldots}{\CAPPHI_n}}}$ \\
\Rule{$\vee$-Shuffle} & $\disjunction{\CAPPHI_1}{\disjunction{\ldots}{\disjunction{\CAPPHI_k}{\disjunction{\ldots}{\CAPPHI_n}}}}$ & $\disjunction{\CAPPHI_k}{\disjunction{\CAPPHI_1}{\disjunction{\ldots}{\CAPPHI_n}}}$ \\
\end{longtable}

It is left as an exercise for the reader to prove that these rules are sound.
Note that $\universal$-Shuffle is only sound when the universal quantifier is moved over a conjunction, and $\existential$-Shuffle is only sound when moved over a disjunction.
For example, the sentence $\universal\variable{x}\pardisjunction{\Bp{\variable{x}}}{\negation{\Bp{\variable{x}}}}$ is a logical truth but $\disjunction{\universal\variable{x}\Bp{\variable{x}}}{\universal\variable{x}\negation{\Bp{\variable{x}}}}$ isn't.
Similarly, $\negation{\existential\variable{x}\parconjunction{\Bp{\variable{x}}}{\negation{\Bp{\variable{x}}}}}$ is a logical truth but $\negation{\parconjunction{\existential\variable{x}\Bp{\variable{x}}}{\existential\variable{x}\negation{\Bp{\variable{x}}}}}$ isn't.

These rules enable us to push the quantifiers all the way into a suffiently well-behaved sentence.
To push universal quantifiers inward over conjunctions, we use \Rule{$\universal$-Shuffle}.
To push exstential quantifiers inward over disjunctions, we use \Rule{$\existential$-Shuffle}.
In cases when the universal quantifer governs a disjunction we must isolate the disjuncts with a matching variable, using \Rule{$\vee$-Shuffle} to move those disjuncts to the left.
Then we use Q shuffle to move the quantifier to just those disjuncts.
Similarly, when an existential quantifier governs a conjunction, we isolate the conjuncts with a matching variable, using \Rule{$\wedge$-Shuffle} to move those conjuncts to the left.
Then we use Q shuffle to move the quantifier to just those conjuncts.

\subsection{The QL1 Decision Procedure}\label{The QL1 Decision Procedure}

\begin{THEOREM}{\LnpTC{MonadicGQLEquivTheorem} \GQL{}1 Independent Quantifiers Theorem:}
Every sentence of \GQL{}1 is quantificationally equivalent to a sentence whose quantifiers are independent.
\end{THEOREM}
\begin{PROOF}
\begin{commentary}
	The basic strategy to prove this theorem is relatively straightforward even if some of the details aren't.
	\commentaryspace
	First, we start with a \GQL{}1 sentence and get its PDNF equivalent. 
	All the quantifiers are in the initial position of the latter sentence.
	Then we push each quantifier inward, using the new quantifier exchange rules, to shrink its scope.
	We show that after the quantifiers are pushed in they are all independent.
\end{commentary}
Let there be a sentence of \GQL{}1.
Then by theorem \ref{PrenexNFTheorem} there is an equivalent sentence, $\CAPPHI$, that is in prenex disjunctive normal form.
So $\CAPPHI$ is of the form $\#_1\#_2\ldots\#_n\CAPPHI'$, where for each $i\in\{1,\ldots,n\}$, $\#_i$ is a quantifier, and $\CAPPHI'$ is a DNF formula.
Then $\CAPPHI'$ is of the form $\pardisjunction{\disjunction{\CAPPSI_1}{\CAPPSI_2}}{\disjunction{\ldots}{\CAPPSI_k}}$, where for each $j\in\{1,\ldots,k\}$, $\CAPPSI_j$ is a conjunction of ions.

Let $\CAPPHI_{n+1}$ be $\CAPPHI'$, i.e., the DNF body of $\CAPPHI$.
To make the quantifiers independent of each other, each quantifier must be pushed in to govern the variables it binds and no others.
We start with $\#_n$, pushing the quantifier in, and work backward in sequence to $\#_1$.
The formula we start with, $\CAPPHI_{n+1}$, is transformed $n$ times, once for each of the $n$ quantifiers.
Let $\#_i$ be the the $i^{\text{th}}$ quantifier of the sequence in $\CAPPHI$, $\#_1\#_2\ldots\#_n$.
Then let $\CAPPHI_i$ for each $i\in\{1,\ldots,n\}$ be the result of having pushed in the $i^{\text{th}}$ quantifier, $\#_i$, into the formula $\CAPPHI_{i+1}$.
The final product of all these transformations is $\CAPPHI_{1}$, at which point all the quantifiers are moved in.
Each transformation uses exchange rules only, so $\CAPPHI_{1}$ is equivalent to $\CAPPHI$.

The following steps define the transformation to push in each quantifier, starting with $\#_n$ and working back to $\#_1$.
For each quantifer $\#_i$ to move there are two cases, handled separately:

\vspace*{0.15in}
\noindent{}\textbf{Case 1}: $\#_i$ is an existential quantifier.
\begin{commentary}
	To push an existential quantifier in is easier, since the body of the formula is a disjunction.
	We push it in using \Rule{$\existential$-Shuffle} so that each disjunct is governed by an existential quantifer.
	Then, for each disjunct, we reorder the conjuncts using \Rule{$\wedge$-Shuffle} to isolate the conjuncts with a matching variable, and use Q shuffle to move the quantifier to govern just those.
\end{commentary}

\begin{quote}
\begin{description}
\item[Step A:] By $\existential$-shuffle, $\#_i$ is moved to each disjunct in $\CAPPHI_{i+1}$, resulting in: $\pardisjunction{\disjunction{\#_i\CAPPSI_1}{\#_i\CAPPSI_2}}{\disjunction{\ldots}{\#_i\CAPPSI_k}}$. For any $\CAPPSI_j$ not containing the variable in $\#_i$, $\#_i$ may be removed from that disjunct, by Q-deletion.
\item[Step B:] Each disjunct with a quantifier $\#_i\CAPPSI_j$ is of the form \\ $\#_i\parconjunction{\conjunction{\CAPTHETA_1}{\CAPTHETA_2}}{\conjunction{\ldots}{\CAPTHETA_m}}$ where each $\CAPTHETA$ is an ion (or a quantified subsentence governing a conjunction or disjunction, as will be made clear shortly). An equivalent conjunction is obtained by applying \Rule{$\wedge$-shuffle} so that each $\CAPTHETA$ with a variable matching that of $\#_i$ is moved to the left.
\item[Step C:] By the Q shuffle exchange rule, the $\#_i$ is moved in to govern just the leftmost conjuncts with a matching variable, resulting in each $\#_i\CAPPSI_j$ becoming: $\conjunction{\#_i\parconjunction{\CAPTHETA_1}{\CAPTHETA_2}}{\conjunction{\ldots}{\CAPTHETA_m}}$. With this transformation $\#_i$ binds all variables within its scope, and no other variables are in its scope.
\end{description}
\noindent{}The result is a formula $\CAPPHI_{i}$ where $\#_i$ binds all and only its variables.
\end{quote}

\noindent{}\textbf{Case 2}: $\#_i$ is a universal quantifier.
\begin{commentary}
	To push a universal quantifier in, the distribution exchange rule is first used to make the body of the sentence a conjunction.
	Then we push the universal in using \Rule{$\universal$-Shuffle} so that each conjunct is governed by a universal quantifer.
	Then, for each conjunct, we reorder the disjuncts using \Rule{$\vee$-Shuffle} to isolate the disjuncts with a matching variable, and use Q shuffle to move the quantifier to govern just those.
	After the quantifier is moved in, the distribution exchange rule is applied again to make the body of the sentence a disjunction again.
\end{commentary}

\begin{quote}
\begin{description}
\item[Step A:] Apply the distribution rule to $\CAPPHI_{i+1}$ until no conjunction is governed by a disjunction (with the exception of any conjunction within the scope of an already moved quantifier). The resulting formula, $\CAPPHI_{i+1}^*$, is a conjunction.
\item[Step B:] By $\universal$-shuffle, $\#_i$ is moved to each conjunct in $\CAPPHI_{i+1}^*$, resulting in: $\parconjunction{\conjunction{\#_i\CAPPSI_1}{\#_i\CAPPSI_2}}{\conjunction{\ldots}{\#_i\CAPPSI_k}}$. For any $\CAPPSI_j$ not containing the variable bound by $\#_i$, the quantifier may be removed, by Q-deletion.
\item[Step C:] Each conjunct with a quantifier $\#_i\CAPPSI_j$ is of the form \\ $\#_i\pardisjunction{\disjunction{\CAPTHETA_1}{\CAPTHETA_2}}{\disjunction{\ldots}{\CAPTHETA_m}}$ where each $\CAPTHETA$ is an ion (or a quantifier subsentence governing a disjunction or conjunction). An equivalent disjunction is obtained by by applying \Rule{$\vee$-shuffle} so that each $\CAPTHETA$ with a variable matching that of $\#_i$ is moved to the left.
\item[Step D:] By the Q shuffle exchange rule, the $\#_i$ is moved in to govern just the leftmost disjuncts with a matching variable, resulting in $\#_i\CAPPSI_j$ becoming: $\disjunction{\#_i\pardisjunction{\CAPTHETA_1}{\CAPTHETA_2}}{\disjunction{\ldots}{\CAPTHETA_m}}$. With this transformation $\#_i$ binds all variables within its scope, and no other variables are in its scope.
\item[Step E:] Apply the distribution rule to the resulting formula until no disjunction is governed by a conjunction (with the exception of any disjunction within the scope of an already moved quantifier).
\end{description}
\noindent{}The result is a formula $\CAPPHI_{i}$ where $\#_i$ binds all and only its variables.
\end{quote}

\noindent{}The result of these $n$ transformation is a \GQL{}1 sentence $\CAPPHI_{1}$ with each moved quantifier governing all and only the variables it binds.
Therefore the quantifiers of $\CAPPHI_{1}$ are all independent.
\end{PROOF}

\begin{THEOREM}{\LnpTC{MonadicDecisionTheorem} The \GQL{}1 Decision Theorem:}
\end{THEOREM}
\begin{PROOF}
Let\index{Monadic Decision Theorem, The} $\CAPPHI$ be a \GQL{}1 sentence.
Then by theorem \ref{MonadicGQLEquivTheorem} there is an equivalent sentence $\CAPPHI^*$ such that all its quantifiers are independent.
Use Q-shuffle to move the existential quantifers to the prenex position, and then move the universal quantifiers as well, resulting in an equivalent sentence $\CAPPHI^{**}$.
Then put the body of $\CAPPHI^{**}$ into DNF, resulting in $\CAPPHI^{***}$.
$\CAPPHI^{***}$ is in PDNF.
So for the reasons described previously, the method is guaranteed to halt on $\CAPPHI^{***}$.
Therefore there is a decision procedure for $\CAPPHI^{***}$, and hence also for $\CAPPHI$.
\end{PROOF}

%%%%%%%%%%%%%%%%%%%%%%%%%%%%%%%%%%%%%%%%%%%%%%%%%%
\section{L\"owenheim-Skolem and Compactness}
%%%%%%%%%%%%%%%%%%%%%%%%%%%%%%%%%%%%%%%%%%%%%%%%%%

A number of results follow directly from the completeness of \GQD{} or the method we used to prove completeness. 
\begin{THEOREM}{\LnpTC{LowenheimSkolemTheorem} The Downward L\"owenheim-Skolem Theorem:}
If a sentence of \GQL{} is true in any model, then it is true in one whose domain consists of all or some of the natural numbers.
\end{THEOREM}
\begin{PROOF}
If $\CAPPHI$ is true in some model, then $\negation{\CAPPHI}$ is not a quantificational truth.
Thus, applying the method to $\negation{\CAPPHI}$ does not produce a contradiction, but produces a model of the natural numbers which falsifies $\negation{\CAPPHI}$ and hence makes $\CAPPHI$ true.
\end{PROOF}
It's important to note that there's really nothing special about the natural numbers.
When we devised the procedure for constructing a model of the ions in the master matrix list that results from the method (when no contradiction arises), we choose to use natural numbers for the universe. 
But it should be clear that we did this out of convenience (it's easy, after all, to associate constants with the natural numbers). 
We could have used any set of objects for the universe. 
What's important is that, whatever we used, the domain of the constructed model is at most countably infinite (i.e., it is at most the size of the natural numbers and no larger). 
Hence a more abstract version of the downward Lowenheim-Skolem Theorem simply says: If a sentence of \GQL{} is true in any model, then either it's true in only models with finite domains, or, if it's true at all in models with infinite domains, then there's an model with a \emph{countably} infinite domain in which it's true. 

This thorem was proved in a weaker form originally by Leopold L\"owenheim \citeyearpar{Lowenheim1915}, and the proof was improved by Thoralf Skolem \citeyearpar{Skolem1920,Skolem1922}. 
Notice that the theorem talks only about models, and we have proved it via a detour through derivations. 
As you might imagine, there are more direct proofs, including Skolem's \citetext{\citealp{Tarski1956}, \citealp{Vaught1974}, \citealp[ch.~3.1]{Hodges1997}, \citeyear[63]{Hodges2001}}.

Notice also that after our work on \GQL{}1, we know that for monadic sentences the method stops after a finite number of steps (if we arrange the prenex carefully) and so we can conclude that if a monadic sentence is true in any model then it is true in a finite one. 
\begin{THEOREM}{\LnpTC{MonadicIntSizeTheorem}}
If $\CAPPHI$ is a sentence of \GQL{}1 and has a model, then it has a finite model.
\end{THEOREM}

Our next corollary of completeness is the Compactness Theorem.
Although historically the completeness theorem was proved first and compactness followed as a corollary, today the compactness theorem takes center stage in many areas of logic (especially model theory). 
Like the L\"owenheim-Skolem Theorem, there are many different proofs of compactness that do not go through completeness or use any facts about derivations \citetext{see \citealt[321]{Kleene1967}, \citealt{Ebbinghaus1985}, \citealt[ch.~5.1]{Hodges1997}, \citealp[63]{Hodges2001}, \citeyear[29]{Hodges2001b}}. 
%answers the question of whether it's possible that an infinite set of sentences is intuitively contradictory but we cannot deduce the contradiction in our system because our derivations are finite?
%This is a specific version of the more general worry that if $\Delta$ is an infinite set then it might be that $\Delta\sdtstile{}{}\CAPPHI$ but not $\Delta\sststile{}{}\CAPPHI$. 
%We can prove that this doe not occur in our language by proving the following theorem.
\begin{THEOREM}{\LnpTC{Thm:CompactnessTheorem} The Compactness Theorem for \GQL{}:}
For all sets of sentences $\Delta$ of \GQL{}, if for every finite subset $\Delta'$ of $\Delta$ there exists a model $\IntA'$ that makes all the sentences in $\Delta'$ true, then there's some model $\IntA$ that makes all the sentences in $\Delta$ true. 
\end{THEOREM}
\begin{PROOF}
By the strong completeness theorem, for all sets $\Delta$ of \GQL{} sentences and \GQL{} sentence $\CAPPHI$, if $\Delta\sdtstile{}{}\CAPPHI$, then $\Delta\sststile{}{}\CAPPHI$.
Now assume that there's no model $\IntA$ that makes all the sentences in $\Delta$ true. 
Hence $\Delta\sdtstile{}{}\conjunction{\Al}{\negation{\Al}}$.
So by strong completeness, $\Delta\sststile{}{}\conjunction{\Al}{\negation{\Al}}$.
By definition, this implies that there's some finite subset $\Delta'$ of $\Delta$ such that $\conjunction{\Al}{\negation{\Al}}$ can be derived from $\Delta'$. 
Hence there is no model $\IntA$ that makes all the sentences in $\Delta'$ true. 
Hence it's not the case that for every finite subset $\Delta'$ of $\Delta$ there exists a model $\IntA'$ that makes all the sentences in $\Delta'$ true. 
\end{PROOF}

\newpage
%%%%%%%%%%%%%%%%%%%%%%%%%%%%%%%%%%%%%%%%%%%%%%%%%%
\section{Exercises}
%%%%%%%%%%%%%%%%%%%%%%%%%%%%%%%%%%%%%%%%%%%%%%%%%%

\notocsubsection{Misc. Problems}{Misc Problems}
\begin{enumerate}
\item Let's say that any derivation rule \Rule{R} that has the following property is \niidf{sound}:\index{derivation!rule!sound} if we add a line to a derivation $\Derivation{D}$ with sentence $\CAPPHI$ sanctioned by rule \Rule{R}, then $\Delta\sdtstile{}{}\CAPPHI$, where $\Delta$ is the set of unboxed assumptions for the new line. 
(Compare this with what it is for a rule to be truth-preserving, def. \pncmvref{Derivation Rule Soundness}, which is different.)
Then the proof of theorem \mvref{Main GSL Soundness Lemma} basically shows that \GSD{} is sound by showing that all the basic rules of \GSD{} are sound. 
We know that \GSDP{} is sound because \GSD{} is sound and (by theorem \pmvref{GSD Shortcut Theorem3}) anything you can derive in \GSDP{} can be derived in \GSD{}. 
But we could also show that \GSDP{} is sound directly (without appealing to theorem \ref{GSD Shortcut Theorem3}) by showing that the shortcut rules used in \GSDP{} themselves are sound. 
Of course, this follows from theorem \pmvref{GSD Shortcut Theorem2} and the fact that the basic rules are sound, but again we can show it directly. 
But again we can show it without going through the basic rules.
Show directly (without appealing to theorem \ref{GSD Shortcut Theorem2}) that the following rules are sound (see tables \pmvref{GSDplus1} and \pmvref{GSDplus2}): 
\begin{multicols}{2}
\begin{enumerate}
\item \Rule{M.T.}
\item \Rule{A.C.}
\item \Rule{$\HORSESHOE$/$\VEE$-Exch.}
\item \Rule{Contraposition}
\end{enumerate}
\end{multicols} 
\item Recall that $\HORSESHOE$ elimination can only be used on a conditional that is the main connective of a sentence. Show that if we do not make this restriction, then the rule is unsound. In other words, give a derivation which violates only that restriction (a derivation where you use $\HORSESHOE$ elimination on a horseshoe that's not the main connective) and which ends with a proof of a sentence that is \emph{not} a logical truth (not truth-functionally true) from the empty set of assumptions.
\end{enumerate}

\notocsubsection{Quantifier Exchange Rule Soundness}{Quantifier Exchange Rule Soundness}

Prove the soundness of the following exchange rules.

\begin{enumerate}
\item \Rule{$\universal$-Shuffle}
\item \Rule{$\existential$-Shuffle}
\item \Rule{Q-Deletion}
\item \Rule{$\wedge$-Shuffle}
\item \Rule{$\vee$-Shuffle}
\end{enumerate}

\renewcommand{\arraystretch}{1.5}
\begin{longtable}[c]{ p{1in} l l } %p{2.2in} p{2in}
\toprule
\textbf{Name} & \textbf{Given} & \textbf{May Add} \\ 
\midrule
\endfirsthead
\multicolumn{3}{c}{\emph{Continued from Previous Page}}\\
\toprule
\textbf{Name} & \textbf{Given} & \textbf{May Add} \\ 
\midrule
\endhead
\bottomrule
\caption{Additonal Exchange Rules for \GQD{}}\\[-.15in]
\multicolumn{3}{c}{\emph{Continued next Page}}\\
\endfoot
\bottomrule
\caption{Additonal Exchange Rules for \GQD{}}\\
\endlastfoot
\Rule{$\universal$-Shuffle} & $\universal\variable{\alpha}\parconjunction{\CAPPHI_1}{\conjunction{\ldots}{\CAPPHI_n}}$ & $\conjunction{\universal\variable{\alpha}\CAPPHI_1}{\conjunction{\ldots}{\universal\variable{\alpha}\CAPPHI_n}}$ \\
\Rule{$\existential$-Shuffle} & $\existential\variable{\alpha}\pardisjunction{\CAPPHI_1}{\disjunction{\ldots}{\CAPPHI_n}}$ & $\par\disjunction{\existential\variable{\alpha}\CAPPHI_1}{\disjunction{\ldots}{\existential\variable{\alpha}\CAPPHI_n}}$ \\
\Rule{Q-Deletion} & $\#\variable{\alpha}\CAPPHI$ & $\CAPPHI$, iff $\alpha$ does  \\[-.25cm]
\nopagebreak
 &  &  not occur in $\CAPPHI$  \\
\Rule{$\wedge$-Shuffle} & $\conjunction{\CAPPHI_1}{\conjunction{\ldots}{\conjunction{\CAPPHI_k}{\conjunction{\ldots}{\CAPPHI_n}}}}$ & $\conjunction{\CAPPHI_k}{\conjunction{\CAPPHI_1}{\conjunction{\ldots}{\CAPPHI_n}}}$ \\
\Rule{$\vee$-Shuffle} & $\disjunction{\CAPPHI_1}{\disjunction{\ldots}{\disjunction{\CAPPHI_k}{\disjunction{\ldots}{\CAPPHI_n}}}}$ & $\disjunction{\CAPPHI_k}{\disjunction{\CAPPHI_1}{\disjunction{\ldots}{\CAPPHI_n}}}$ \\
\end{longtable}

%\theendnotes

\include{ch9}

%%%%%%%%%%%%%%%%%%%%%%%%%%%%%%%%%%%%%%%%%%%%%%%%%%
\chapter*{Appendix A: List of Derivation Rules}
\addcontentsline{toc}{chapter}{Appendix A: List of Derivation Rules}
%%%%%%%%%%%%%%%%%%%%%%%%%%%%%%%%%%%%%%%%%%%%%%%%%%
%\fancyhead[RE,LO]{\textsf{Appendix A: List of Derivation Rules}}
\chead{\textsf{Appendix A: List of Derivation Rules}}
\fancyhead[LE,RO]{\textsf{\thepage}}
\setcounter{section}{0}

\begin{tabular}{ l l }
\multicolumn{2}{l}{\textbf{Derivation Systems:}}\\
\GSD{}& Set 1\\
\GSDP{}& Sets 1,2,3\\
\GQD{}& Sets 1,4\\
\GQDP{}& Sets 1,2,3,4,5\\
\GQDPP{}& Sets 1,2,3,4,5,6,7,8\\
\SF{}& Sets 1,9\\
\SFP{}& Sets 1,2,3,9,10\\
\GQDI{}& Sets 1,4,11\\
\GQDIP{}& Sets 1,2,3,4,5,11\\
\end{tabular}

%\begin{table}[!ht]
\renewcommand{\arraystretch}{1.5}
%\begin{center}
\begin{longtable}[c]{ p{1in} l l } %p{2.2in} p{2in}
\toprule
\textbf{Name} & \textbf{Given} & \textbf{May Add} \\ 
\midrule
\endfirsthead
%\multicolumn{3}{c}{\emph{Continued from Previous Page}}\\
\toprule
\textbf{Name} & \textbf{Given} & \textbf{May Add} \\ 
\midrule
\endhead
%\bottomrule
%\caption{Basic Rules of \GSD{}}\\[-.15in]
%\multicolumn{3}{c}{\emph{Continued next Page}}\\
\endfoot
\bottomrule
%\caption{Basic Rules of \GSD{}}\\%
\endlastfoot%
%\label{GSD}%
%\midrule
\multicolumn{3}{l}{\textbf{Set 1: Basic Rules of GSD, table \mvref{GSD}}}\\
%\midrule
\Rule{Ass.} & & | $\CAPPHI$ \\
\Rule{Rep.} & $\CAPPHI$ & $\CAPPHI$ \\
\Rule{$\HORSESHOE$-Elim} & $\horseshoe{\CAPTHETA}{\CAPPSI}$, $\CAPTHETA$ & $\CAPPSI$ \\
\Rule{$\HORSESHOE$-Intro} &  | $\CAPTHETA\Rightarrow\CAPPSI$ & $\horseshoe{\CAPTHETA}{\CAPPSI}$, Box $|\CAPTHETA\Rightarrow\CAPPSI$ \\
\Rule{$\!\WEDGE\!$-Elim} &{}$\conjunction{\CAPTHETA_1}{\conjunction{\CAPTHETA_2}{\conjunction{\ldots}{\CAPTHETA_{\integer{n}}}}}$&{}Conjunction of any proper\\[-.25cm]
 & &{}subset of the conjuncts\\
\Rule{$\!\WEDGE\!$-Intro} & $\CAPTHETA_1$, $\CAPTHETA_2$, $\ldots$ $\CAPTHETA_{\integer{n}}$ & $\conjunction{\CAPTHETA_1}{\conjunction{\CAPTHETA_2}{\conjunction{\ldots}{\CAPTHETA_{\integer{n}}}}}$ \\
\Rule{$\VEE$-Elim} & $\disjunction{\CAPTHETA_1}{\disjunction{\CAPTHETA_2}{\disjunction{\ldots}{\CAPTHETA_{\integer{n}}}}}$, &  \\
 &  $\horseshoe{\CAPTHETA_1}{\CAPPSI}$,  &  \\
 &  $\horseshoe{\CAPTHETA_2}{\CAPPSI}$,  &  \\
 &  $\vdots$  &  \\
 &  $\horseshoe{\CAPTHETA_{\integer{n}}}{\CAPPSI}$ & $\CAPPSI$ \\
\Rule{$\VEE$-Intro} & $\CAPTHETA$ & $\disjunction{\CAPPSI_1}{\disjunction{\CAPPSI_2}{\disjunction{\ldots}{\CAPPSI_{\integer{n}}}}}$, \\[-.25cm]
\nopagebreak
 &  & where $\CAPTHETA=\CAPPSI_i$ for some $i$. \\
\Rule{$\NEGATION$-Intro} & $\horseshoe{\CAPTHETA}{\parconjunction{\CAPPSI}{\negation{\CAPPSI}}}$ & $\negation{\CAPTHETA}$ \\
\Rule{$\NEGATION$-Elim} & $\horseshoe{\negation{\CAPTHETA}}{\parconjunction{\CAPPSI}{\negation{\CAPPSI}}}$ & $\CAPTHETA$ \\
\Rule{$\TRIPLEBAR$-Intro} & $\horseshoe{\CAPTHETA}{\CAPPSI}$, $\horseshoe{\CAPPSI}{\CAPTHETA}$ & $\triplebar{\CAPTHETA}{\CAPPSI}$ \\
\Rule{$\TRIPLEBAR$-Elim} & $\triplebar{\CAPTHETA}{\CAPPSI}$, $\CAPPSI$ & $\CAPTHETA$ \\
\Rule{$\TRIPLEBAR$-Elim} & $\triplebar{\CAPTHETA}{\CAPPSI}$, $\CAPTHETA$ & $\CAPPSI$ \\
%\bottomrule
%\midrule
\multicolumn{3}{l}{\textbf{Set 2: Standard Shortcut Rules of GSD, table \mvref{GSDplus1}}}\\
\nopagebreak
%\midrule
\Rule{M.T.} & $\horseshoe{\CAPPHI}{\CAPTHETA}$, $\negation{\CAPTHETA}$ & $\negation{\CAPPHI}$ \\
\Rule{D.S.} & $\disjunction{\CAPPHI_1}{\disjunction{\ldots}{\disjunction{\CAPPHI_i}{\disjunction{\ldots}{\CAPPHI_{\integer{n}}}}}}$, $\negation{\CAPPHI_i}$ & $\disjunction{\CAPPHI_1}{\disjunction{\ldots}{\disjunction{\CAPPHI_{i-1}}{\disjunction{\CAPPHI_{i+1}}{\disjunction{\ldots}{\CAPPHI_{\integer{n}}}}}}}$ \\
\nopagebreak
 & $\disjunction{\CAPPHI_1}{\disjunction{\ldots}{\disjunction{\negation{\CAPPHI_i}}{\disjunction{\ldots}{\CAPPHI_{\integer{n}}}}}}$, ${\CAPPHI_i}$ & $\disjunction{\CAPPHI_1}{\disjunction{\ldots}{\disjunction{\CAPPHI_{i-1}}{\disjunction{\CAPPHI_{i+1}}{\disjunction{\ldots}{\CAPPHI_{\integer{n}}}}}}}$ \\
\Rule{A.C.} & ${\CAPPHI},{\negation{\CAPPHI}}$ & $\CAPPSI$ \\
\Rule{$\NEGATION$/$\TRIPLEBAR$-Intro} & $\triplebar{\CAPPHI}{\CAPPSI}$ & $\triplebar{\negation{\CAPPHI}}{\negation{\CAPPSI}}$ \\
%\midrule
\multicolumn{3}{l}{\textbf{Set 3: Exchange Shortcut Rules of GSD, table \mvref{GSDplus2}}}\\
\nopagebreak
%\midrule
\Rule{DeM} & $\negation{\parconjunction{\CAPPHI_1}{\conjunction{\ldots}{\CAPPHI_{\integer{n}}}}}$ & $\disjunction{\negation{\CAPPHI_1}}{\disjunction{\ldots}{\negation{\CAPPHI_{\integer{n}}}}}$\\
 & $\disjunction{\negation{\CAPPHI_1}}{\disjunction{\ldots}{\negation{\CAPPHI_{\integer{n}}}}}$ & $\negation{\parconjunction{\CAPPHI_1}{\conjunction{\ldots}{\CAPPHI_{\integer{n}}}}}$\\
 & $\negation{\pardisjunction{\CAPPHI_1}{\disjunction{\ldots}{\CAPPHI_{\integer{n}}}}}$ & $\conjunction{\negation{\CAPPHI_1}}{\conjunction{\ldots}{\negation{\CAPPHI_{\integer{n}}}}}$ \\
 & $\conjunction{\negation{\CAPPHI_1}}{\conjunction{\ldots}{\negation{\CAPPHI_{\integer{n}}}}}$ & $\negation{\pardisjunction{\CAPPHI_1}{\disjunction{\ldots}{\CAPPHI_{\integer{n}}}}}$ \\
\Rule{$\NEGATION\NEGATION$-Elim} & $\negation{\negation{\CAPPHI}}$ & $\CAPPHI$ \\
\Rule{$\NEGATION\NEGATION$-Intro} & $\CAPPHI$ & $\negation{\negation{\CAPPHI}}$ \\
\Rule{$\HORSESHOE$/$\VEE$-Exchange} & $\horseshoe{\CAPPHI}{\CAPTHETA}$ & $\disjunction{\negation{\CAPPHI}}{\CAPTHETA}$ \\
\nopagebreak
 & $\disjunction{\negation{\CAPPHI}}{\CAPTHETA}$ & $\horseshoe{\CAPPHI}{\CAPTHETA}$  \\
\Rule{Contraposition} & $\horseshoe{\CAPPHI}{\CAPTHETA}$ & $\horseshoe{\negation{\CAPTHETA}}{\negation{\CAPPHI}}$ \\
 & $\horseshoe{\negation{\CAPTHETA}}{\negation{\CAPPHI}}$ & $\horseshoe{\CAPPHI}{\CAPTHETA}$ \\
\Rule{$\NEGATION$/$\HORSESHOE$-Exchange} & $\negation{\parhorseshoe{\CAPPHI}{\CAPTHETA}}$ & $\conjunction{\CAPPHI}{\negation{\CAPTHETA}}$ \\
\nopagebreak
 & $\conjunction{\CAPPHI}{\negation{\CAPTHETA}}$ & $\negation{\parhorseshoe{\CAPPHI}{\CAPTHETA}}$ \\
\Rule{Distribution} & $\conjunction{\CAPTHETA}{\pardisjunction{\CAPPHI_1}{\disjunction{\ldots}{\CAPPHI_{\integer{n}}}}}$ & $\disjunction{\parconjunction{\CAPTHETA}{\CAPPHI_1}}{\disjunction{\ldots}{\parconjunction{\CAPTHETA}{\CAPPHI_{\integer{n}}}}}$\\
\nopagebreak
 & $\disjunction{\parconjunction{\CAPTHETA}{\CAPPHI_1}}{\disjunction{\ldots}{\parconjunction{\CAPTHETA}{\CAPPHI_{\integer{n}}}}}$ & $\conjunction{\CAPTHETA}{\pardisjunction{\CAPPHI_1}{\disjunction{\ldots}{\CAPPHI_{\integer{n}}}}}$\\
\nopagebreak 
 & $\conjunction{\pardisjunction{\CAPPHI_1}{\disjunction{\ldots}{\CAPPHI_{\integer{n}}}}}{\CAPTHETA}$ & $\disjunction{\parconjunction{\CAPPHI_1}{\CAPTHETA}}{\disjunction{\ldots}{\parconjunction{\CAPPHI_{\integer{n}}}{\CAPTHETA}}}$\\
\nopagebreak 
 & $\disjunction{\parconjunction{\CAPPHI_1}{\CAPTHETA}}{\disjunction{\ldots}{\parconjunction{\CAPPHI_{\integer{n}}}{\CAPTHETA}}}$  & $\conjunction{\pardisjunction{\CAPPHI_1}{\disjunction{\ldots}{\CAPPHI_{\integer{n}}}}}{\CAPTHETA}$\\
\nopagebreak 
 & $\disjunction{\CAPTHETA}{\parconjunction{\CAPPHI_1}{\conjunction{\ldots}{\CAPPHI_{\integer{n}}}}}$ & $\conjunction{\pardisjunction{\CAPTHETA}{\CAPPHI_1}}{\conjunction{\ldots}{\pardisjunction{\CAPTHETA}{\CAPPHI_{\integer{n}}}}}$\\
\nopagebreak 
 & $\conjunction{\pardisjunction{\CAPTHETA}{\CAPPHI_1}}{\conjunction{\ldots}{\pardisjunction{\CAPTHETA}{\CAPPHI_{\integer{n}}}}}$ & $\disjunction{\CAPTHETA}{\parconjunction{\CAPPHI_1}{\conjunction{\ldots}{\CAPPHI_{\integer{n}}}}}$\\
\nopagebreak 
 & $\disjunction{\parconjunction{\CAPPHI_1}{\conjunction{\ldots}{\CAPPHI_{\integer{n}}}}}{\CAPTHETA}$ & $\conjunction{\pardisjunction{\CAPPHI_1}{\CAPTHETA}}{\conjunction{\ldots}{\pardisjunction{\CAPPHI_{\integer{n}}}{\CAPTHETA}}}$\\
\nopagebreak
 & $\conjunction{\pardisjunction{\CAPPHI_1}{\CAPTHETA}}{\conjunction{\ldots}{\pardisjunction{\CAPPHI_{\integer{n}}}{\CAPTHETA}}}$ & $\disjunction{\parconjunction{\CAPPHI_1}{\conjunction{\ldots}{\CAPPHI_{\integer{n}}}}}{\CAPTHETA}$\\
%\midrule
\multicolumn{3}{l}{\textbf{Set 4: Basic Rules of GQD, table \mvref{GQD}}}\\
\nopagebreak
%\midrule
\Rule{$\forall$-Elim} & $\universal{\BETA}\CAPPHI$ & $\CAPPHI\constant{a}/\BETA$, for \mention{a} any  \\[-.25cm]
\nopagebreak
 &   &   individual constant \\
\Rule{$\forall$-Intro} & $\CAPPHI\constant{a}/\BETA$ & $\universal{\BETA}\CAPPHI$, iff \mention{a} does  \\[-.25cm]
 &  &  not occur in $\CAPPHI$  \\[-.25cm]
 &  & nor in any unboxed assumption \\
\Rule{$\exists$-Intro} & $\CAPPHI\constant{a}/\BETA$ & $\existential{\BETA}\CAPPHI$ \\
\Rule{$\exists$-Elim} & $\existential{\BETA}\CAPPHI$, $\horseshoe{\CAPPHI{\constant{a}/\BETA}}{\CAPTHETA}$ & $\CAPTHETA$, \Iff \mention{a} does \\[-.25cm]
\nopagebreak
 &  &  not occur in $\CAPPHI$ or $\CAPTHETA$, \\[-.25cm]
\nopagebreak
 & &  nor in any unboxed assumption\\
%\midrule
\multicolumn{3}{l}{\textbf{Set 5: Exchange Shortcut Rules of GQD, table \mvref{GQDplus}}}\\
\nopagebreak
%\midrule
\Rule{QN} & $\negation{\universal{\BETA}{\CAPPHI}}$ & $\existential{\BETA}\negation{{\CAPPHI}}$ \\
 & $\existential{\BETA}\negation{{\CAPPHI}}$ & $\negation{\universal{\BETA}{\CAPPHI}}$  \\
 & $\negation{\existential{\BETA}{\CAPPHI}}$ & $\universal{\BETA}\negation{{\CAPPHI}}$ \\
 &  $\universal{\BETA}\negation{{\CAPPHI}}$ & $\negation{\existential{\BETA}{\CAPPHI}}$ \\
%\midrule
\multicolumn{3}{l}{\textbf{Set 6: DNF Exchange Shortcut Rules for GSD, table \mvref{GSDplusDNF}}}\\
\nopagebreak
%\midrule
\Rule{$\TRIPLEBAR$-Exchange} &  $\triplebar{\CAPTHETA}{\CAPPSI}$ & $\disjunction{\parconjunction{\CAPTHETA}{\CAPPSI}}{\parconjunction{\negation{\CAPTHETA}}{\negation{\CAPPSI}}}$ \\
\nopagebreak
 & $\disjunction{\parconjunction{\CAPTHETA}{\CAPPSI}}{\parconjunction{\negation{\CAPTHETA}}{\negation{\CAPPSI}}}$ &  $\triplebar{\CAPTHETA}{\CAPPSI}$ \\
%\midrule
\multicolumn{3}{l}{\textbf{Set 7: Prenex Exchange Shortcut Rules for GQD, table \mvref{GSDplusPrenex}}}\\
\nopagebreak
%\midrule
\Rule{$\ALPHA$/$\BETA$-Exch} & $(\#\ALPHA)\CAPPHI$ & $(\#\BETA)\CAPPHI\BETA/\ALPHA$ \\
\Rule{Q Shuffling} & $\parconjunction{(\#\variable{x})\CAPTHETA}{\CAPPSI}$ & $(\#\variable{x})\parconjunction{\CAPTHETA}{\CAPPSI}$ \\
& $\parconjunction{\CAPTHETA}{(\#\variable{x})\CAPPSI}$ & $(\#\variable{x})\parconjunction{\CAPTHETA}{\CAPPSI}$ \\

& $\pardisjunction{(\#\variable{x})\CAPTHETA}{\CAPPSI}$ & $(\#\variable{x})\pardisjunction{\CAPTHETA}{\CAPPSI}$ \\
& $\pardisjunction{\CAPTHETA}{(\#\variable{x})\CAPPSI}$ & $(\#\variable{x})\pardisjunction{\CAPTHETA}{\CAPPSI}$ \\

& $\parhorseshoe{\CAPTHETA}{(\#\variable{x})\CAPPSI}$ & $(\#\variable{x})\parhorseshoe{\CAPTHETA}{\CAPPSI}$ \\

& $\parhorseshoe{\existential{\variable{x}}\CAPTHETA}{\CAPPSI}$ & $\universal{\variable{x}}\parhorseshoe{\CAPTHETA}{\CAPPSI}$ \\
& $\parhorseshoe{\universal{\variable{x}}\CAPTHETA}{\CAPPSI}$ & $\existential{\variable{x}}\parhorseshoe{\CAPTHETA}{\CAPPSI}$ \\
%\midrule
\multicolumn{3}{l}{\textbf{Set 8: The Method Shortcut Rules for GQD, table \mvref{GSDplusMethod}}}\\
\nopagebreak
%\midrule
\Rule{Greg's Rule} & $\disjunction{\CAPPSI_1}{\disjunction{\ldots}{\CAPPSI_{\integer{n}}}}$, where some & $\disjunction{\CAPPSI_1}{\disjunction{\ldots}{\disjunction{\CAPPSI_{\integer{i}-1}}{\disjunction{\CAPPSI_{\integer{i}+1}}{\disjunction{\ldots}{\CAPPSI_{\integer{n}}}}}}}$ \\[-.25cm]
 & $\CAPPSI_{\integer{i}}=\conjunction{\CAPPHI_1}{\conjunction{\ldots}{\conjunction{\CAPPHI_{\integer{j}}}{\ldots}}}$ & \\[-.25cm]
 & $\WEDGE\conjunction{\negation{\CAPPHI_{\integer{j}}}}{\conjunction{\ldots}{\CAPPHI_{\integer{m}}}}$ & \\
 
\Rule{$\VEE$/$\WEDGE\!$-Elim} & $\disjunction{\CAPPSI_{1}}{\disjunction{\ldots}{\CAPPSI_{\integer{n}}}}$, where & $\CAPPHI$ \\[-.25cm]
 & each $\CAPPSI_{\integer{i}}$ contains $\CAPPHI$ & \\
 
\Rule{OBA} &  $\conjunction{\CAPPHI_1}{\conjunction{\ldots}{\conjunction{\CAPPHI_i}{\conjunction{\ldots}{\CAPPHI_{\integer{n}}}}}}$, $\negation{\CAPPHI_i}$ & $\negation{\parconjunction{\CAPPHI_1}{\conjunction{\ldots}{\conjunction{\CAPPHI_i}{\conjunction{\ldots}{\CAPPHI_{\integer{n}}}}}}}$ \\
\nopagebreak
 & $\conjunction{\CAPPHI_1}{\conjunction{\ldots}{\conjunction{\negation{\CAPPHI_i}}{\conjunction{\ldots}{\CAPPHI_{\integer{n}}}}}}$, ${\CAPPHI_i}$ & $\negation{\parconjunction{\CAPPHI_1}{\conjunction{\ldots}{\conjunction{\negation{\CAPPHI_i}}{\conjunction{\ldots}{\CAPPHI_{\integer{n}}}}}}}$ \\

\multicolumn{3}{l}{\textbf{Set 9: Additonal Quantifier Exchange Rules for \GQD{}, table \mvref{AdditionalQuantifierExRules}}}\\
\nopagebreak
\Rule{$\universal$-Shuffle} & $\universal\variable{\alpha}\parconjunction{\CAPPHI_1}{\conjunction{\ldots}{\CAPPHI_n}}$ & $\conjunction{\universal\variable{\alpha}\CAPPHI_1}{\conjunction{\ldots}{\universal\variable{\alpha}\CAPPHI_n}}$ \\
\Rule{$\existential$-Shuffle} & $\existential\variable{\alpha}\pardisjunction{\CAPPHI_1}{\disjunction{\ldots}{\CAPPHI_n}}$ & $\par\disjunction{\existential\variable{\alpha}\CAPPHI_1}{\disjunction{\ldots}{\existential\variable{\alpha}\CAPPHI_n}}$ \\
\Rule{Q-Deletion} & $\#\variable{\alpha}\CAPPHI$ & $\CAPPHI$, iff $\alpha$ does  \\[-.25cm]
\nopagebreak
 &  &  not occur in $\CAPPHI$  \\

%\midrule
\multicolumn{3}{l}{\textbf{Set 10: Basic Rules for S5, table \mvref{SF}}}\\
\nopagebreak
%\midrule
\Rule{$\BOX\!$-Elim} & $\BOX\CAPPHI$ & $\CAPPHI$ \\
\Rule{$\BOX\!$-Intro} & $\CAPPHI$ (*) & $\BOX\CAPPHI$ \\
\Rule{$\DIAMOND\!$-Elim} & $\DIAMOND\CAPPHI$, $\horseshoe{\CAPPHI}{\CAPPSI}$ (*), (**) & $\CAPPSI$ \\
\Rule{$\DIAMOND\!$-Intro} &  $\CAPPHI$ & $\DIAMOND\CAPPHI$ \\
%\midrule
\multicolumn{3}{l}{\textbf{Set 11: Modal Negation Exchange Shortcut Rules for S5, table \mvref{SFMN}}}\\
\nopagebreak
%\midrule
\Rule{MN} & $\negation{\BOX\CAPPHI}$ & $\DIAMOND\negation{\CAPPHI}$ \\
 & $\negation{\DIAMOND\CAPPHI}$ & $\BOX\negation{\CAPPHI}$ \\
 & $\negation{\DIAMOND\negation{\CAPPHI}}$ & $\BOX\CAPPHI$ \\
 &  $\negation{\BOX\negation{\CAPPHI}}$ & $\DIAMOND\CAPPHI$ \\ 
%\midrule
\multicolumn{3}{l}{\textbf{Set 12: Basic Rules for GQDI, \mvref{GQDI}}}\\
\nopagebreak
%\midrule
\Rule{$=$-Intro} &  & $\variable{t}=\variable{t}$ \\
\Rule{$=$-Elim} & $\CAPPHI$, $\variable{t}=\variable{s}$ & $\CAPPHI\variable{t}/\variable{s}$ \\
\end{longtable}
\noindent{}(*) in \Rule{$\BOX\!$-Intro} and \Rule{$\DIAMOND\!$-Elim}, the rule can only be applied if all the open assumptions have modal prefixes.

\noindent{}(**) In \Rule{$\DIAMOND\!$-Elim}, the rule can only be applied if $\CAPPSI$ has a modal prefix.

%\theendnotes


\clearemptydoublepage
\addcontentsline{toc}{chapter}{Works Cited}
%\fancyhead[RE,LO]{\sffamily{}Works Cited}
\chead{\sffamily{}Works Cited}
\fancyhead[LE,RO]{\sffamily{}\thepage}
\setcounter{section}{0}
%\bibliography{C:/Users/Michael/Documents/writings/papers_and_books/annotated_bib/Bibliography}{}
\bibliography{Bibliography}{}
\bibliographystyle{philreview-m}
%bibstyle options (in usual directory): kluwer, agms, dcu, plainnat, chicago, astron
%my bibstyle options (must be placed in directory with tex file): 
    %analysis (styled used in journal analysis)
    %philreview (style used in journal phil review)
    %philreview-m (my modified version, this is prefered)
    %plainnat2 (my modified version of plainnat, this is 2nd choice)

\clearemptydoublepage
\addcontentsline{toc}{chapter}{Index}
%\fancyhead[RE,LO]{\sffamily{}Index}
\chead{\sffamily{}Index}
\fancyhead[LE,RO]{\sffamily{}\thepage}
\setcounter{section}{0}
\printindex

\end{document}

