% *******                   Logic Textbook                         *******
% *******  by Richard Grandy & Michael Barkasi & Joshua Reagan       *******
%Note: \raggedbottom is set right before ch 1
\documentclass[11pt,fleqn,twoside,openright]{report}%leqno   
%\includeonly{ch1,ch4}  
\usepackage{etex}
%\usepackage{leading}
%\leading{15pt} % with caslon, use 14.4 (or 14.7)pt, use 15pt for Palatino and cardo
%\usepackage[firstpage]{draftwatermark}
% % % % % XeTeX and other font stuff:
\usepackage{amsmath,amssymb}
\usepackage{fontspec}
\usepackage{cancel}
\setmainfont[
Ligatures = {Required, Common, Contextual},
Numbers = {Monospaced}, %Proportional, OldStyle
SlantedFont = {* Slanted},
Scale = 1.00,
Mapping=tex-text
]{Latin Modern Roman}
\setsansfont[
Scale=MatchLowercase, %=MatchLowercase, =1.00
BoldFont = Source Sans Pro Semibold,
Mapping=tex-text
]{Source Sans Pro} 
\usepackage{unicode-math}
\setmathfont[math-style=TeX]{Latin Modern Math}
%unicode-math
\usepackage[factor=600]{microtype}
\newcommand{\uppercasespacing}[1]{{\addfontfeature{LetterSpace=2.0}#1}} 
\renewcommand{\textsc}[1]{\uppercasespacing{\MakeUppercase{#1}}}
\newcommand{\textSC}[1]{{\addfontfeatures{Letters={UppercaseSmallCaps,SmallCaps}}#1}}
\newcommand{\osf}[1]{{\addfontfeatures{Numbers={OldStyle}}{#1}}}
\newcommand{\proportional}[1]{{\addfontfeatures{Numbers={Proportional}}{#1}}}
\newcommand{\tabluarnumbers}[1]{{\addfontfeatures{Numbers=Monospaced}#1}}
\newcommand{\set}[1]{\left\{#1\right\}}
\usepackage{realscripts}
   % gives commands: 
      % \textsuperscript
	  % \textsubscript
	  % \textsubsuperscript
	  % \setlength\supsupersep{2pt}
   % But if the font has the right features but not the full set of glyphs, need:
      % \textsuperscript*
	  % \textsubscript*
   % which use the ``fake'' versions
\newcommand{\smallfaction}[1]{{\addfontfeatures{Fractions=On}#1}} 
   % Note: just type in fraction with slash, e.g.
      % \smallfaction{1/2}
\newcommand{\textfaction}[2]{%
    {\addfontfeatures{VerticalPosition={Numerator}}#1}{/}%
	{\addfontfeatures{VerticalPosition={Denominator}}#2}}
% %\newcommand*\rfrac[2]{{}^{#1}\!/_{#2}}
%% \rfrac is a command to allow slanted fractions, here used only sparingly in the many-valued logic section.
\usepackage{xfrac}
\newcommand{\rfrac}[2]{\sfrac{#1}{#2}}
\newcommand{\entails}{\sdtstile{}{}}
% % % % 
\usepackage{turnstile}
%for shorter turnstiles:
\renewcommand{\makever}[4]
{\ifthenelse{\equal{#1}{s}}{\rule[-0.4#3]{#2}{.8#3}}{}
\ifthenelse{\equal{#1}{d}}{\rule[-0.5#3]{#2}{#3}
\hspace{#4}
\rule[-0.5#3]{#2}{#3}}{}
\ifthenelse{\equal{#1}{t}}{\rule[-0.5#3]{#2}{#3}
\hspace{#4}
\rule[-0.5#3]{#2}{#3}
\hspace{#4}
\rule[-0.5#3]{#2}{#3}}{}}%   
\usepackage{multicol}
\usepackage{relsize}
\usepackage{makeidx}
\makeindex   
\usepackage[usenames,dvipsnames]{xcolor}
\definecolor{DarkBlue}{HTML}{000090} % Some of these defined colors may not actually be used below
\definecolor{DDarkBlue}{HTML}{003399}
\definecolor{RRed}{HTML}{812811}
\definecolor{LightGray}{HTML}{444444}               
\usepackage{tcolorbox}
\newcommand{\commentaryspace}{

    \vspace{5mm}
}
\usepackage[font=sf]{caption}
\usepackage{booktabs}
\usepackage{longtable}
\usepackage{varioref}
\usepackage{marginnote}
\usepackage[noload]{qtree}
\usepackage{tikz}
\usepackage{tikz-qtree-compat}
\usepackage[splitrule,bottom]{footmisc} % "hang" is another option to use 
   %\setlength{\footnotemargin}{0.65em}       %Manually set footnote margin, use with "hang"
%These next new commands are to define the arrows used in a few examples in chapter two that point from variables to the quantifiers that bind them.
\newcommand{\NextLineRef}[1][]{%
    \tikz [overlay,remember picture] \draw [->, out=-10, in=20, distance=0.25cm, thick, #1] 
        (-1ex,-0.25ex) to (-5.75ex,-0.75ex);
}
\newcommand{\NextLineRefB}[1][]{%
    \tikz [overlay,remember picture] \draw [->, out=-10, in=20, distance=0.25cm, thick, #1] 
        (-1ex,-0.25ex) to (-13.5ex,-1.5ex);
}
\newcommand{\NextLineRefC}[1][]{%
    \tikz [overlay,remember picture] \draw [->, out=-10, in=20, distance=0.25cm, thick, #1] 
        (-0.5ex,1.75ex) to (-9.1ex,2ex); %(-11.5ex,2ex);
}
\newcommand{\NextLineRefD}[1][]{%
    \tikz [overlay,remember picture] \draw [->, out=-10, in=20, distance=0.25cm, thick, #1] 
        (-1ex,-0.75ex) to (-7.25ex,-0.75ex);
}
\newcommand{\NextLineRefE}[1][]{%
    \tikz [overlay,remember picture] \draw [->, out=-10, in=20, distance=0.25cm, thick, #1] 
        (-1ex,-0.25ex) to (-17ex,-0.5ex); %(-18.5ex,-0.5ex);
}
\newcommand{\NextLineRefF}[1][]{%
    \tikz [overlay,remember picture] \draw [->, out=-10, in=20, distance=0.25cm, thick, #1] 
        (-1ex,1.75ex) to (-14.8ex,2ex); %(-17ex,2ex);
}
\newcommand{\NextLineRefG}[1][]{%
    \tikz [overlay,remember picture] \draw [->, out=-10, in=20, distance=0.25cm, thick, #1] 
        (-1ex,-0.25ex) to (-21ex,-0.5ex);
}
\newcommand{\NextLineRefH}[1][]{%
	\tikz [overlay,remember picture] \draw [->, out=-10, in=20, distance=0.25cm, thick, #1] 
	(-1ex,-0.25ex) to (-15.3ex,-0.5ex); %(-18.5ex,-0.5ex);
}
\newcommand{\NextLineRefI}[1][]{%
	\tikz [overlay,remember picture] \draw [->, out=-10, in=20, distance=0.25cm, thick, #1] 
	(-1ex,-0.25ex) to (-19ex,-0.5ex);
}
\newcommand{\NextLineRefJ}[1][]{%
	\tikz [overlay,remember picture] \draw [->, out=-10, in=20, distance=0.25cm, thick, #1] 
	(-1ex,-0.25ex) to (-6.75ex,-0.75ex);
}
\newcommand{\NextLineRefK}[1][]{%
	\tikz [overlay,remember picture] \draw [->, out=-10, in=20, distance=0.25cm, thick, #1] 
	(-1ex,-0.25ex) to (-4.75ex,-0.75ex);
}

\usepackage{eso-pic}
%The package above and the new commands below define what's needed for the upper right-hand pictures near the chapter headings.
\newcommand\BackgroundPicA{%
\put(0,0){%
\parbox[b][\paperheight]{\paperwidth}{%
\vspace{1.5in}
\hfill
\includegraphics[width=.38\paperwidth,height=\paperheight,%
keepaspectratio]{CLF3.png}\hspace*{1.43in}%
\vfill
}}}
\newcommand\BackgroundPicB{%
\put(0,0){%
\parbox[b][\paperheight]{\paperwidth}{%
\vspace{1.5in}
\hfill
\includegraphics[width=.38\paperwidth,height=\paperheight,%
keepaspectratio]{fj2.png}\hspace*{1.43in}%
\vfill
}}}
\newcommand\BackgroundPicC{%
\put(0,0){%
\parbox[b][\paperheight]{\paperwidth}{%
\vspace{1.5in}
\hfill
\includegraphics[width=.38\paperwidth,height=\paperheight,%
keepaspectratio]{Tur.png}\hspace*{1.43in}%
\vfill
}}}
\usepackage{wallpaper}
\usepackage[inline]{enumitem}  % for control over lists
\labelformat{enumi}{#1} %{(#1)}
%These commands define the custom running lists/counters used for many of the examples.
\newenvironment{smenumerate}{%
\begin{enumerate}[series=m, itemsep=0em]% 
}{%
\end{enumerate}%
}
\newenvironment{menumerate}{%
\begin{enumerate}[resume*=m]%,start=1
}{%
\end{enumerate}%
}
\newenvironment{RESTARTmenumerate}{%
\begin{enumerate}[resume*=m,start=1]%
}{%
\end{enumerate}%
}
% For self-contained lists, like clauses in a definition
\newenvironment{cenumerate}{%
\begin{enumerate}[label=(\arabic*), leftmargin=1.85\parindent,
labelindent=.35\parindent, labelsep=*, itemsep=0pt]%,start=1, parsep=0pt
}{%
\end{enumerate}%
}
%These define some things for the in-text citations
\renewcommand{\reftextfaceafter}{on the \reftextvario{next}{next} page}%
\renewcommand{\reftextfacebefore}{on the \reftextvario{previous}{previous} page}%
\renewcommand{\reftextbefore}{on the \reftextvario{previous}{previous} page}%
%\renewcommand{\reftextfaraway}[1]{p. \pageref{#1}}
%Some custom commands for intra-text hyperreferences
\newcommand{\mvref}[1]{\ref{#1} (\vpageref*{#1})}
\newcommand{\pmvref}[1]{\ref{#1}, \vpageref{#1}}
\newcommand{\pncmvref}[1]{\ref{#1} \vpageref{#1}}
%These control the header
\usepackage{fancyhdr}
\setlength{\headheight}{15.2pt}
\renewcommand{\headrulewidth}{0pt} 
\renewcommand{\footrulewidth}{0pt}
\newcommand{\CAPS}[1]{\uppercasespacing{\MakeUppercase{#1}}}% % %
% Any changes to the command \CAPS must manually be made to the title of the section "DNF and the TFE Replacement Theorem"
% was: {\textls[20]{#1}} % % % This controls the (extra) spacing between strings of capital letters, like TFT and QT
%\expandafter\index\expandafter{\CAPS}
%some custom commands (used to get these from the local "philosophy" package): 
\newcommand{\df}[1]{\textit{#1}\index{#1|textbf}}
\newcommand{\underdf}[2]{\emph{#1}\index{#2!#1|textbf}}
\newcommand{\nidf}[1]{\textit{#1}}
\newcommand{\mention}[1]{`#1'}
\newcommand{\Iff}{iff }
\newcommand{\IFF}{if, and only if, }
%Makes it easier to change some of the fonts on the definition/theorem/example enivornments: 
\newcommand{\descriptionfont}{\bfseries\sffamily} 
\newcommand{\theoremfont}{\bfseries\sffamily}
\newcommand{\definitionfont}{\bfseries\sffamily}
\newcommand{\examplefont}{\bfseries\sffamily}
\newcommand{\prooffont}{\bfseries\sffamily}
\setlist[description]{font=\descriptionfont{}}  % this command is part of enumitem; control look of description label.
\newcommand{\idf}[1]{\emph{#1}\index{#1|textbf}} % informal, or in-line, definition
\newcommand{\seealsoidf}[2]{\emph{#1}\index{#1|seealso{#2}}}
\newcommand{\underidf}[2]{\emph{#1}\index{#2!#1|textbf}}
\newcommand{\niidf}[1]{\textit{#1}}
\newcommand{\Rule}[1]{\textsl{#1}}  % for names of inference rules, was using \textsc{#1}
\newcommand{\sq}[1]{{``}#1{''}}
\newcommand{\distinction}[2]{#1/#2}
\newcommand{\q}[1]{{``}#1{''}}
%The counter for the definition/theorem/example environments
\newcounter{DefThm}[chapter]
\renewcommand\theDefThm{\thechapter.\arabic{DefThm}} % \arabic{chapter}.\arabic{DefThm}
\newcommand{\DC}[1]{\refstepcounter{DefThm}(Definition \arabic{DefThm}\label{#1})}
\newcommand{\TC}[1]{\refstepcounter{DefThm}(Theorem \arabic{DefThm}\label{#1})}
\newcommand{\EC}[1]{\refstepcounter{DefThm}(Example \arabic{DefThm}\label{#1})}
\newcommand{\npDC}[1]{\refstepcounter{DefThm}\arabic{DefThm}\label{#1}}
\newcommand{\npTC}[1]{\refstepcounter{DefThm}\arabic{DefThm}\label{#1}}
\newcommand{\npEC}[1]{\refstepcounter{DefThm}\arabic{DefThm}\label{#1}}
\newcommand{\LnpDC}[1]{\refstepcounter{DefThm}{\definitionfont{}Definition \theDefThm.\label{#1}}}
\newcommand{\LnpTC}[1]{\refstepcounter{DefThm}{\theoremfont{}Theorem \theDefThm.\label{#1}}}
\newcommand{\LnpEC}[1]{\refstepcounter{DefThm}{\examplefont{}Example \theDefThm.\label{#1}}}
% This package for setting page geometry and layout
\usepackage[paper=letterpaper,heightrounded=true,inner=1.43in,outer=1.43in,top=1.5in,bottom=1.5in]{geometry} %useful options not loaded: showcrop, showframe, marginparwidth=3.5cm, marginparsep=-3.5cm,
% Some other new environments used below:
\newenvironment{major}[1]
{

\medskip
\noindent{{#1}:}}%
{

\medskip}
\newenvironment{majorIL}[1]
{

\medskip
\noindent{{#1}:}}%
{ }
\newenvironment{majorILnc}[1]
{

\medskip
\noindent{\sffamily\bfseries{}#1}}%
{

\medskip}
\newenvironment{PROOF}
{

\medskip
\noindent{\prooffont{}Proof:}}%
{\hfill \ensuremath{\blacksquare}

\medskip}
\newenvironment{PROOFOF}[1]
{

\medskip
\noindent{\prooffont{}Proof of #1:}}%
{\hfill \ensuremath{\blacksquare}

\medskip}
\newenvironment{SUBPROOF}
{

\medskip
\noindent{\prooffont{}Subproof:}}%
{\hfill \ensuremath{\blacksquare} %\;\blacksquare}

\medskip}
\newenvironment{THEOREM}[1]
{

\medskip
\noindent{#1}}% \itshape{}
{ %\hfill \ensuremath{\blacktriangleleft}

\medskip}
\newenvironment{commentary}{\begin{tcolorbox}}{\end{tcolorbox}}
\newenvironment{minor}[1]
{\emph{{#1}:}}%
{\par}
% Section Heading Controls:
\usepackage[toctitles,nobottomtitles*]{titlesec}
\renewcommand{\bottomtitlespace}{.05\textheight}
\titleformat{\chapter}[display]{\huge\sffamily\bfseries}{\huge{}\sffamily\bfseries\color{DDarkBlue}Chapter \thechapter}{.1em}{\vspace{.075em}}[]
\titlespacing{\chapter}{0pt}{-.27in}{1.27in}
\titleformat{\section}[hang]{\Large\bf\sffamily\color{DDarkBlue}}{\thesection}{1em}{}[{\color{Black}\titlerule}]
\titlespacing*{\section}{0pt}{20pt}{0.2cm}
\titleformat{\subsection}[hang]{\large\bf\sffamily}{\thesubsection}{1em}{}
\titlespacing*{\subsection}{0pt}{10pt}{0.2cm}
\titleformat{\subsubsection}[runin]{\bf}{\thesubsubsection}{1em}{}
\titlespacing{\subsubsection}{0pt}{1em}{1em}
%\renewcommand\thesection{\arabic{section}}%   %this command redefines the representation of the section counter so it doesn't display the chapter number before the section number.
  \newcommand{\notocsubsection}[2]{%
    \refstepcounter{subsection}%
    \subsection*{\thesubsection \quad #1}\label{#2}}% 
% Table of Content Controls:
\usepackage{tocloft}%
\setcounter{tocdepth}{1}%                                         %Table of Contents depth
\renewcommand{\cfttoctitlefont}{\huge\sffamily\bfseries}% %make sure this matches chapter heading space
\renewcommand{\cftaftertoctitle}{}
\renewcommand{\cftchapfont}{\large\bfseries\sffamily\selectfont{}}
\renewcommand{\cftchappagefont}{\large\bfseries\sffamily\selectfont{}}
\renewcommand{\cftsecfont}{\normalsize}
\renewcommand{\cftsecpagefont}{\normalsize}
\renewcommand{\cftsubsecfont}{\normalsize}
\renewcommand{\cftsubsecpagefont}{\normalsize}
\setlength{\cftaftertoctitleskip}{1in}
\setlength{\cftchapindent}{0in}
\setlength{\cftsecindent}{0in} 
\setlength{\cftsubsecindent}{.15in} 
\setlength{\cftbeforesecskip}{0in} 
%this command is for doublepage clears with a blank page:
\let\origdoublepage\cleardoublepage
\newcommand{\clearemptydoublepage}{%
  \clearpage
  {\pagestyle{empty}\origdoublepage}%
}
%this might cause problems, but should be okay:
\let\cleardoublepage\clearemptydoublepage
% This is the start of the big "logic2" style file:
\usepackage[
%propositional constants
A=A,
B=B,
C=C,
D=D,
E=E,
F=F,
G=G,
H=H,
I=I,
J=J,
K=K,
L=L,
M=M,
N=N,
O=O,
P=P,
Q=Q,
R=R,
S=S,
T=T,
U=U,
V=V,
W=W,
X=X,
Y=Y,
Z=Z,
%perdicate letters%%%%%%%%%%%%%%
predparL=,
predparR=,
%PredArgComma=2, %1=comma, 2=no comma
Ap=A, %\text{A},
Bp=B, %\text{B},
Cp=C, %\text{C},
Dp=D, %\text{D},
Ep=E, %\text{E},
Fp=F, %\text{F},
Gp=G, %\text{G},
Hp=H, %\text{H},
Ip=I, %\text{I},
Jp=J, %\text{J},
Kp=K, %\text{K},
Lp=L, %\text{L},
Mp=M, %\text{M},
Np=N, %\text{N},
Op=O, %\text{O},
Pp=P, %\text{P},
Qp=Q, %\text{Q},
Rp=R, %\text{R},
Sp=S, %\text{S},
Tp=T, %\text{T},
Up=U, %\text{U},
Vp=V, %\text{V},
Wp=W, %\text{W},
Xp=X, %\text{X},
Yp=Y, %\text{Y},
Zp=Z, %\text{Z},
%object language variables%%%%%%%%%%%
w=w,
x=x,
y=y,
z=z,
%metalanguage constants for formula
alpha=\alpha,   %use macro \ALPHA, \BETA, etc
beta=\beta,
chi=\chi,
delta=\delta,
epsilon=\epsilon,
phi=\Phi,
gamma=\gamma,
eta=\eta,
iota=\iota,
kappa=\kappa,
lambda=\lambda,
mu=\mu,
nu=\nu,
Greeko=o,
pi=\pi,
theta=\Theta,
rho=\rho,
sigma=\sigma,
tau=\tau,
upsilon=\upsilon,
omega=\omega,
xi=\xi,
psi=\Psi,
zeta=\zeta,
capalpha=A,  %use macro \CAPALPHA, \CAPBETA, etc
capbeta=B,
capchi=X,
capdelta=\Delta,
capepsilon=E,
capphi=\phi,
capgamma=\Gamma,
capeta=H,
capiota=I,
capkappa=K,
caplambda=\Lambda,
capmu=M,
capnu=N,
capGreeko=O,
cappi=\Pi,
captheta=\theta,
caprho=P,
capsigma=\Sigma,
captau=T,
capupsilon=\Upsilon,
capomega=\Omega,
capxi=\Xi,
cappsi=\psi,
capzeta=Z,
%connectives  use macros \disjunction,\conjunction,\negation,\horseshoe, and \triplebar (plus
quantifierparL={},%  the proper number of arguments; affix as prefixs par, bpar, or cpar (i.e.
quantifierparR={},%  \pardisjunction to wrap in parentheses.
manybrackets=1, %1=yes, 2=no
IntA={I},
True=\text{true},
False=\text{false}
]{logic2}
%\usepackage{eucal} %changes \mathcal, so interpretations look better
%\usepackage{eufrak} %changes \mathfrak, so stuff looks better (TeX says redundant if amsfonts used?)
\newcommand{\constant}[1]{\mathrm{#1}} % for object-lang constants
\newcommand{\variable}[1]{#1} % for object-lang variables
\newcommand{\integer}[1]{#1}
%\DeclareMathAlphabet{\mathpzc}{OT1}{pzc}{m}{it}
\newcommand{\world}[1]{\mathit{#1}}
\newcommand{\TrueB}{\mathsf{T}}
\newcommand{\FalseB}{\mathsf{F}}
\renewcommand{\IntA}{\mathfrak{m}}
\newcommand{\m}{\mathfrak{m}}
\newcommand{\tfe}{\CAPS{tfe} }
\newcommand{\lhs}{\CAPS{lhs} }
\newcommand{\rhs}{\CAPS{rhs} }
\newcommand{\LP}{\text{LP}}
\newcommand{\HHH}{\text{AR}}
\newcommand{\ORD}[1]{\text{ORD}#1}
\newcommand{\Derivation}[1]{\mathit{#1}}
\newcommand{\e}{\text{e}}
\newcommand{\As}[2]{\mathfrak{m^{\mathrm{#1}}_{\mathrm{#2}}}}
\newcommand{\model}[2]{\mathfrak{m^{\mathrm{#1}}_{\mathrm{#2}}}}
%\newcommand{\As}{\mathfrak{a}}
\newcommand{\PosIntEM}[2]{Pos Int^{#1+}_{\mathit{#2}}}
\newcommand{\BWEDGE}{\mathlarger{\mathlarger{\WEDGE}}}
%\renewcommand{\baselinestretch}{1.5}
\newcommand{\BOX}{\ensuremath \raisebox{-.95pt}{$\Box\,$}} %.75pt
\newcommand{\DIAMOND}{\ensuremath \raisebox{-.6pt}{$\Diamond\,$}} %.75pt
\newcommand{\GSL}{\textsc{sl}} %{GSL} %
\newcommand{\GQL}{\textsc{ql}} %{GQL} %
\newcommand{\SL}{\textsc{sl}} %{SL} %
\newcommand{\QL}{\textsc{ql}} %{QL} %
\newcommand{\PL}{\textsc{pl}} %{QL} %
\newcommand{\MGSL}{\textsc{msl}} %
\newcommand{\GQLI}{\textsc{qli}} %
\newcommand{\GSD}{\textsc{sd}} %{GSD} %
\newcommand{\GSDP}{\textsc{sd}$^+$} %{GSD} %
\newcommand{\GQD}{\textsc{qd}} %{GQD}%
\newcommand{\GQDP}{\textsc{qd}$^+$} %{GQD} %
\newcommand{\GQDPP}{\textsc{qd}$^+_{m}$} %
\newcommand{\GQDI}{\textsc{qdi}} %{GQDI} %
\newcommand{\GQDIP}{\textsc{qdi}$^+$} %{GQDI} %
\newcommand{\SF}{$S5$}
\newcommand{\SFP}{$S5^+$}
\newcommand{\SO}{$S1$}
\newcommand{\Language}[1]{#1}
\newcommand{\DerivationSystem}[1]{#1}
%\renewcommand{\underline}[1]{{\color{Green}#1}}
%\usepackage{endnotes}
%\let\footnote=\endnote
\renewcommand{\v}{\variable{v}}
\renewcommand{\u}{\variable{u}}
\renewcommand{\a}{\constant{a}}
\renewcommand{\b}{\constant{b}}
\renewcommand{\c}{\constant{c}}
\renewcommand{\d}{\constant{d}}
% More Bib stuff:
\newcommand{\ICP}{;} %should match "\setcitestyle{citesep={;}}"
\usepackage[round]{natbib}
\setcitestyle{notesep={: }}% %punctuation seperating year from pages
\setcitestyle{aysep={}}% %punctuation seperating author and year
\setcitestyle{yysep={,}}% %punctuation seperating years for multiple
                               %%% works from same author
\setcitestyle{citesep={\ICP{}}}% %punctuation between citations
\def\bibfont{\normalsize }
\renewcommand{\bibname}{Works Cited}
%Examples:
%\citet{refname}   author, (year)
%\citep{refname}  (author, year)
%\citep[p.~5]{refname}   (author, year, p. 5)
%\citeyearpar{refname}  (year)
%\citetext{\citealp[24--26]{Burge1986}; \citeyear[68,~98--99]{Burge2010}}  
                        %(Burge, 1986: 24--25; 2010: 68, 98--99)
\usepackage[%
unicode,pdfencoding=auto,
    pagebackref=false,%
    linktoc=page,%
    pdfnewwindow=true,%               % links in new window
    colorlinks=true,%                 % false: boxed links; true: colored links
    linkcolor=Blue,%                   % color of internal links
    citecolor=Blue,%                  % color of links to bibliography
    filecolor=Black,%magenta,               % color of file links
    urlcolor=Black%cyan                    % color of external links
%    unicode,
%    pdfencoding=auto
]{hyperref}%
\begin{document}
\pagenumbering{roman} % Roman numerals
\setcounter{page}{1}%
\pagestyle{empty}
%\restoregeometry

\vspace*{.8in}
%\noindent{\makebox[4.75in][s]{\Huge\sffamily{}M a t h e m a t i c a l L o g i c}}
\noindent{\sffamily\fontsize{35}{35}\selectfont{}\color{DDarkBlue}{\color{Gray}Mathematical}Logic} %\textls[30]

%\medskip
%\noindent{\LARGE{}\fontencoding{T1}\sffamily\fontseries{b}\fontshape{it}\selectfont{}\textls[30]{***A subtitle would go here, if we had one***}}

\vspace{1.25in}
\noindent{}{\LARGE\sffamily\selectfont{}Richard Grandy}

\vspace*{12pt}
\noindent{}{\LARGE\sffamily\selectfont{}Michael Barkasi}

\vspace*{12pt}
\noindent{}{\LARGE\sffamily\selectfont{}Joshua Reagan}

\vfill
\noindent{\LARGE{}\sffamily\selectfont{}{Draft 0.9}} % could use \textls[30] for effect

\newpage
\pagestyle{empty}

\vspace*{.25in}
%\begin{center} % We can center this material, but my preference is to keep it ragged-right (mb)
{\small 
\noindent{}Copyright \textcopyright{} 2024 Richard Grandy, Michael Barkasi, and Joshua Reagan

\bigskip
\noindent{}This work is licensed under the Creative Commons Attribution-NonCommercial-NoDerivs 3.0 Unported License. 

\noindent{}To view a copy of the license, visit http://creativecommons.org/licenses/by-nc-nd/3.0/.

\bigskip
\noindent{}You are free {to Share} -- to copy, distribute and transmit the work, 

\noindent{}under the following conditions: 

\noindent{}{Attribution} -- You must attribute the work to the authors (but not in any way that suggests that they endorse you or your use of the work); 

\noindent{}{Noncommercial} -- You may not use this work for commercial purposes;

\noindent{}{No Derivative Works} -- You may not alter, transform, or build upon this work.

\vfill
\noindent{}This book is set in Latin Modern (for text and math fonts, released under the GUST Font License) and Adobe Source Sans Pro (for headings, released by Adobe under the OFL). 

\bigskip
\noindent{}Edition 0.9 (draft), 2024
}
%\end{center}

\newpage%
\clearemptydoublepage
\pagestyle{plain}
\tableofcontents

\newpage%
\clearemptydoublepage
\pagestyle{plain}

\chapter*{Preface}
\addcontentsline{toc}{chapter}{Preface}

\section*{What is mathematical logic?}

Logic is the study of logical consequence and, by extension, logical truth.
Logical consequence is a relation between a sentence $\PHI$ and a set of sentences $\Delta$. 
A common way to define logical consequence, modeled on Aristotle's discussion of syllogisms, is:
\begin{quote}
$\PHI$ is a \df{logical consequence} of $\Delta$ if and only if it is not possible for the members of $\Delta$ to be true and $\PHI$ to be false.
\end{quote}
Likewise, a sentence $\PHI$ is a logical truth if and only if it cannot possibly be false.
We can think of logical truth\index{logical!truth} as a special case: a sentence $\PHI$ is a \niidf{logical truth} if and only if it is a logical consequence of the empty set; so, i.e., $\Delta=\emptyset$.
However, the sense of \mention{possible} at work in this definition needs clarification.  Usually it's thought that a logical consequence holds in virtue of the meaning of \sq{logical} terms and the overall form of the sentences. 
For example, \mention{Fran is a Wallaby} is a logical consequence of \mention{If Fran was born last year, then Fran is a Wallaby} and \mention{Fran was born last year}.  That's because, in virtue of the meaning of \mention{If $\ldots$ then}, any sentence $\PSI$ must be true whenever both of the sentences \mention{If $\PHI$ then $\PSI$,} and $\PHI$ are true.  Because \mention{meaning} is a notoriously difficult concept to define precisely, we use a formal semantics, giving an exact specification for logical terms. 

In \emph{mathematical} logic\index{mathematical logic} we use mathematical methods to study logical consequence and logical truth. 
In this textbook in particular, we define abstract, formal languages which have as sentences strings of basic symbols.
For sentences of these formal languages, we define (i) the relation of entailment and (ii) formal derivation systems. 
Put roughly, both entailment and formal derivations provide a mathematical approximation of logical consequence.
The goal of this textbook is to give the reader an introduction to those mathematical methods.  
Questions about how logical consequence relates to entailment and derivation are largely set aside. 



\section*{What material is covered here?}
The main subject of this text is classical first-order logic (or as we call it, quantificational logic), with sentential logic (sometimes called \mention{propositional} logic) introduced first. 
We focus mostly on a quantificational logic that doesn't include identity and function symbols.  
Our conjunction and disjunction operators are of arbitrary (but finite) arity because that more closely mirrors typical English use.  
Many-valued logic, modal logic, and the identity operator are briefly introduced in the last chapter.  The main purpose there is to show how sentential and quantificational logic can be extended to treat other logical operators and to cover a wider range of applications. 

The main goal of the text is to prove that for quantificational logic, whether one uses formal derivations or proofs of logical entailment, one always gets the same results. 
There are various ways to define the semantics for quantificational logic.\footnote{For example, game-theoretic semantics and substitutional quantifiers; see \citealp{Dunn1968}, \citealp{Stevenson1973}, \citealp{Hintikka1996}, \citealp{Leblanc2001}, \citealp{Westerstahl2001}, \citealp{Jacquette2002}.}  We use Benson Mates' modification of the standard Tarskian model-theoretic semantics.\footnote{\citealp{Mates1972}} 
The derivation system in this text is a natural deduction style system.\footnote{This style of system was first developed by Gentzen in \citeyearpar{Gentzen1934}, \citealt[26]{Hodges2001}. If you are curious, derivation systems roughly divide into four types: natural deduction, Hilbert-style axiomatic systems, Gentzen-style sequent calculi, and proof tableaux; see \citealt[24]{Hodges2001b} for an overview.} 
In the system used here only conditional elimination can discharge assumptions; this tends to make the other rules simpler. 
We use Fitch-style derivations,\footnote{\citealt{Fitch1952}} but following Donald Kalish and Richard Montague\footnote{\citealp{Kalish1964}} we discharge assumptions by drawing a box around each completed subproof.
Rather than using the more commonly found Henkin-style proofs\footnote{\citealt{Henkin1949}. For one of many modern treatments, see \citealp[ch.~11.4]{Bergmann2003}.} to prove strong completeness, we prove it constructively. We believe this makes the connection between syntactic and semantic methods more intuitive.  In Chapter \ref{completenesschapter} we provide an algorithm (``The Method'') which, for any sentence $\PHI$ entailed by some set of sentences $\Delta$, produces a derivation of $\PHI$ from $\Delta$. 
The proof used here owes much to Willard Quine's completeness proof.\footnote{\citealp{Quine1982}}

\section*{How can I effectively use this book?}

Reading mathematical texts is difficult, especially for those not accustomed to the style of prose typical of the genre.
Following a mathematical argument requires significant cognitive effort and concentration, so prepare yourself.
The most fruitful way to approach this text is with pencil and paper. 
As you read, work out the examples, write the proofs, and find counterexamples on your own.
There's no reason to stick to the text---look for your own proofs and work out new examples. 
Effective use of the text requires active engagement. 

We have tried our best to prepare a helpful and clear text.
A large stock of examples is included and most intratext references include a page number.
Many of the proofs in the text, especially the recursive proofs, leave steps for the reader to complete.
These are always steps (or whole proofs) for which the reader will have seen something similar before.  
Many of the examples have been worked through in detail.
Our hope is that the examples show clearly all the ``moving parts''---that they show how the answers to problems follow directly from the definitions and theorems. 
You probably won't learn much logic by reading the text once or twice, so work through the examples and theorems. 

\section*{Issues in the Philosophy of Logic}\label{issuesinphillogic}
 
In mathematical logic, one begins with rigorous mathematical definitions of ``sentence'', truth, logical truth, entailment, and derivation and then proves theorems about them.
A primary concern in logic is how to make sense of reasoning in natural languages and the notion of logical consequence. 
How the two---the rigorously defined mathematical concepts like entailment and ``pretheoretic'' concepts like logical consequence---relate is a natural question, and one that's received a great deal of attention by philosophers.\footnote{See Blanchette's \citeyearpar{Blanchette2001} and Shapiro's \citeyearpar{Shapiro2005c} for an introduction.}

Because the aim of this text is to introduce mathematical logic and not philosophical logic, we largely set these questions aside. 
In some places, usually for the sake of exposition or motivation, we make claims that raise issues squarely in the domain of philosophy of logic.\footnote{For example, we say in section \ref{Formal Languages} that formal languages can be thought of as models of natural languages; and much of the discussion on identity in section \ref{Sec:Quantifier Logic with Identity} raises issues as well.} 
Given the aims and scope of this text, however, we aren't able to point out all the relevant philosophical concerns, let alone do much to motivate our potentially contentious claims.

\section*{History of Logic}

Logic has a long history going back to Aristotle. 
Aristotle claims that he was the first to work out any systematic treatment of logical consequence (\emph{Sophistical Refutations}, 34, 183b34--36; citation from \citealp[27]{Smith1995}), and as Robin Smith notes, ``we have no reason to dispute this'' \citeyearpar[27]{Smith1995}.
Aristotle's work gave rise to a tradition of work in logic that extends through the Stoics, Byzantine commentators, early Islamic philosophy, and European Scholasticism. 
Modern mathematical logic has roots in this Aristotelian tradition as well (though it has roots elsewhere too, e.g. foundational work in mathematics). 
However, the Aristotelian tradition is beyond the expertise the authors, and so we have not attempted to cite relevant work from it. 

\newpage%
\clearemptydoublepage
%\thispagestyle{empty}

\chapter*{Acknowledgments}
\addcontentsline{toc}{chapter}{Acknowledgments}

\noindent{}This book began as a series of a lecture handouts written for the yearly mathematical logic class in the philosophy department at Rice University. 
The handouts were primarily written by Prof. Richard Grandy from 2007 to 2011, with both Stan Husi and Jacob Mills (TAs for the class during that time) contributing. 
Michael Barkasi began the process of typesetting these handouts in \LaTeX{} for the fall of 2012 (at which time he was the TA).
By next summer he finished this process, and over the next year he compiled and reorganized the handouts into a draft textbook, adding in material as needed. 
Joshua Reagan (TA from 2014-17) has been collaborating with Prof. Grandy since that time, significantly revising and adding new material to the text.

As with most other textbooks, no theorems are original and their presentation is, for the most part, what you'll find elsewhere. 
The proofs for all the theorems---along with the actual writing and presentation---have been done from scratch by one or more of the authors; 
but the general methods and approaches used are just those the authors learned from others. 
Thus this textbook owes just as much to previous work in logic as most other modern textbooks. It passes on to students a body of results and a general conceptual framework developed by many logicians over the years. 
It does so in roughly the style and presentation in which those results were passed to the authors themselves, with some changes they think are improvements. 
We note these as they come up. 

A few more specific acknowledgments are called for. 
At the time the first author was preparing the original handouts he was using Merrie Bergmann, James Moor, and Jack Nelson's textbook \emph{The Logic Book} \citeyearpar{Bergmann2003}; hence some of the terminology and organization in this book reflects that. 
Benson Mates' classic \emph{Elementary Logic} \citeyearpar{Mates1972}, the previous text, and Church's \emph{Introduction to Elementary Logic} \citeyearpar{Church1956} also influenced the first author's approach. 
The second author has relied heavily on Wilfred Hodges's rich survey article ``Elementary Predicate Logic'' \citeyearpar{Hodges2001} for historical citations. 
The name ``Dragnet'' for theorem \ref{The Dragnet Theorem} comes from Michael Smith, another former TA for the first author. 
The problems in exercise \mvref{ex:English to GSL Translations 2} comes from Howard Pospesel \& David Marans, \emph{Arguments: Deductive Logic Exercises} \citeyearpar{Pospesel1978}, which Pospesel and Marans have generously released to the public.   


\newpage
\clearemptydoublepage
\pagestyle{headings}
\pagestyle{myheadings}
\markright{ }

% % % % % % % % Raggedbottom!!! HERE
\raggedbottom
% % % % % % % %
%Controls for fancy headers, if used.
\pagestyle{fancy}
%\renewcommand{\chaptermark}[1]{ \markboth{#1}{} }                       % can only redefine 2 of the 3 "marks"
\renewcommand{\sectionmark}[1]{ \markboth{}{\thesection. #1} }
\renewcommand{\subsectionmark}[1]{ \markboth{}{\thesubsection. #1} }
\fancyhf{}
\chead{\sffamily\rightmark}
\fancyhead[LE]{\sffamily\thepage}
\fancyhead[RO]{\sffamily\thepage}
%\rhead{\thepage}

\pagenumbering{arabic} % Roman numerals
\setcounter{page}{1}


\addtocontents{toc}{\protect\thispagestyle{empty}}
%%%%%%%%%%%%%%%%%%%%%%%%%%%%%%%%%%%%%%%%%%%%%%%%%%
\chapter{Introduction}\label{introduction}
%%%%%%%%%%%%%%%%%%%%%%%%%%%%%%%%%%%%%%%%%%%%%%%%%%
%\AddToShipoutPicture*{\BackgroundPicA}

%%%%%%%%%%%%%%%%%%%%%%%%%%%%%%%%%%%%%%%%%%%%%%%%%%
\section{What is Logic?}\label{What Is Logic}
%%%%%%%%%%%%%%%%%%%%%%%%%%%%%%%%%%%%%%%%%%%%%%%%%%
\subsection{Rational Creatures}\label{Rational Creatures}

Humans are rational creatures. 
As such, we reason about our circumstances and the world in general so we may understand them better. 
An improved understanding of the world helps us make informed decisions, which (ideally) tend to bring about preferable outcomes. 
If you find yourself lost in a labyrinth and hunted by a hungry minotaur, clever reasoning could help you escape. 
Poor reasoning could get you eaten! 

At least some of our reasoning is \emph{discursive}. 
Discursive reasoning is an iterative process: start with a set of initial claims and then infer a result from them, perhaps repeating the process until a certain conclusion is reached. 
Such a chain of inferences can be written down or otherwise recorded as an argument.

Knowledge of logic helps us exploit a particularly reliable kind of discursive reasoning. 
It enables us to judge an argument from any source according to independent, principled criteria.
Not only can we assess the strength of arguments presented to us; we can construct rigorous arguments of our own to share with others. 
Rationality thus has an important social aspect. 
By sharing arguments with each other we can, to some extent, coordinate beliefs and behaviors, and thereby complete tasks beyond the ability of any one person. 

The social role of argumentation provides a clue about the subject matter of logic. 
Arguments are shared by way of language. 
One can give an argument by writing it on paper, typing it in an e-mail, stating it in a speech, etc. 
Once inscribed or encoded in one linguistic medium or another, it can then be assessed by others. 
Accordingly, logicians must attend to certain public features of language. 

%%%%%%%%%%%%%%%%%%%%%%%%%%%%%%%%%%%%%%%%%%%%%%%%%%
\subsection{Logical Consequence}\label{Logical Consequence}

Logic is the study of \emph{logical consequence} and \emph{logical truth}. 
When we say that one sentence `logically follows' from another, we mean that the first sentence is a logical consequence of the other. 
But how can we identify when some claim is a logical consequence of another? 
As suggested previously, we must attend to certain features of the language in which the argument is given. 

One sentence $\CAPPSI$ is a logical consequence of another sentence $\CAPPHI$ if and only if the truth of $\CAPPHI$ guarantees, in virtue of its logical structure, the truth of $\CAPPSI$.

Three points are worth making about this provisional definition. First, don't be afraid of the Greek letters: they're just variables for sentences. Second, the phrase `if and only if' is one we use throughout the text, sometimes abbreviated as `iff'. `$\CAPPHI$ if and only if $\CAPPSI$' is equivalent to `If $\CAPPHI$ then $\CAPPSI$ and if $\CAPPSI$ then $\CAPPHI$'. Third, a solid understanding of this definition depends on having a clear notion of \emph{logical structure}. One of the central goals of this textbook is to provide such a notion.

Consider the following two sentences:
\begin{smenumerate}
	\item George Washington was in the Continental Army and Nathanael Green was in the Continental Army.
	\item Nathanael Green was in the Continental Army.
\end{smenumerate}
The second sentence is a logical consequence of the first. If the first is true the second must be too. The next two sentences share the same structure as the last two:
\begin{menumerate}
	\item Benjamin Franklin was a delegate to the Continental Congress and Thomas Jefferson was a delegate to the Continental Congress.
	\item Thomas Jefferson was a delegate to the Continental Congress.
\end{menumerate}
As above, the second sentence is a logical consequence of the first. We can show the common logical structure of the preceding pairs of sentences using a \emph{schema}:
\begin{menumerate}
	\item $\CAPPHI$ and $\CAPPSI$.
	\item $\CAPPSI$.
\end{menumerate}
We have replaced the non-logical content of each sentence with Greek letters and kept the logical word `and'. The Greek letters serve as variables which stand for clauses of the English language. Any substitution of appropriate English clauses for the letters $\CAPPHI$ and $\CAPPSI$ in the above schema generates a pair of sentences in which the latter follows from the former.\footnote{We will say more about what counts as an appropriate substitution later in the chapter.}

A sentence can also be the logical consequence of a set of sentences. For example, given the sentences,
\begin{menumerate}
	\item The sun is shining.
	\item\label{Birds} The birds are chirping.
\end{menumerate}
the following is a logical consequence:
\begin{menumerate}
	\item The sun is shining and the birds are chirping.
\end{menumerate}
We can represent the logical structure of these sentences in schematic form:
\begin{menumerate}
	\item $\CAPPHI$.
	\item $\CAPPSI$.
	\item $\CAPPHI$ and $\CAPPSI$.
\end{menumerate}
Any substitution of appropriate English clauses for the letters $\CAPPHI$ and $\CAPPSI$ in this schema generates three sentences in which the last follows logically from the first two.

%%%%%%%%%%%%%%%%%%%%%%%%%%%%%%%%%%%%%%%%%%%%%%%%%%
\subsection{Logical Truth}\label{Logical Truth}

A sentence is a \idf{logical truth} if and only if the logical structure of the sentence guarantees its truth.  For example,
\begin{menumerate}
	\item\label{Sun} If the sun is shining then the sun is shining.
\end{menumerate}
This sentence must be true, regardless of whether the sun is shining. Contrast that with sentence \ref{Birds}; if no birds exist, then \ref{Birds} can't be true.  There is nothing special about the non-logical content of \ref{Sun}.  We could replace `the sun is shining' with `it's raining outside' and get another logical truth:
\begin{menumerate}
	\item If it's raining outside then it's raining outside.
\end{menumerate}
The schema of these two logical truths is:
\begin{menumerate}
	\item If $\CAPPHI$ then $\CAPPHI$.
\end{menumerate}
The non-logical content is replaced with $\CAPPHI$ and the words `if\dots then' are retained. Any substitution of an appropriate clause of English for $\CAPPHI$ generates a logical truth.
More generally, logically true sentences have a structure such that, whatever appropriate clause(s) we substitute for the non-logical parts, the result is also a logical truth. 

To construct general schemas from particular examples, as we did above, requires distinguishing between the logical structure and the non-logical content of a sentence. 
We take for granted that certain ``logical'' words play a special role in constituting the logical structure of an English sentence---e.g., words such as `and', `or', `if\ldots  then', `not', `all', `some', among many others.

Logical consequence and logical truth are closely related concepts. In fact, each can be defined in terms of the other, though we won't specify the connection until we give a mathematically precise definition of each. Unfortunately, natural languages such as English have features that make these concepts difficult (or even impossible) to define adequately. 

%%%%%%%%%%%%%%%%%%%%%%%%%%%%%%%%%%%%%%%%%%%%%%%%%%
\section{Natural and Formal Languages}\label{Preliminaries}
%%%%%%%%%%%%%%%%%%%%%%%%%%%%%%%%%%%%%%%%%%%%%%%%%%
\subsection{Natural Languages}\label{Natural Languages}

Calling English a \niidf{natural}\index{language!natural} language is only meant to indicate that it developed gradually and informally over time, and wasn't the product of, say, scholars officially defining the grammar and vocabulary from scratch. 
English, Greek, German, Russian, Spanish, Mandarin, and Hindi are all natural in this sense. 

Let's examine some of the obstacles to achieving mathematical precision in natural language. 
First, it is indeterminate which collections of words and letters are genuine sentences. 
The following strings are, perhaps, neither clearly sentences nor clearly not sentences of English:\footnote{Sentences \ref{Chom1} and \ref{Chom2} are from \citealp{Chomsky1957}.}
\begin{menumerate}
\item\label{Chom1} Colorless green ideas sleep furiously. 
\item\label{Chom2} Furiously sleep ideas green colorless. 
\item Seventeen is purple.
\item Green is a prime number.
\item Someone someone admires admires someone.
\end{menumerate}
There are no principled, universally accepted rules that determine whether each is officially a sentence of English. 
Without a clear distinction between sentences and non-sentences, we cannot be certain which are appropriate for substitution in logical schemas like those given in the previous section.

A second problem with natural languages is that they contain paradoxical sentences. 
A \underidf{paradoxical}{sentence} sentence is one that's true if and only if it's false. 
The most famous examples are those that engender the liar paradox, often called \underidf{liar}{sentence} sentences.\footnote{See the following: \citealt{Tarski1983,Tarski1944}, \citealp{Kripke1975}, \citealp{Barwise1987}, and \citealp{Gupta2001}.}
The following are two examples.
\begin{menumerate}
\item\label{liar1} Sentence \ref{liar1} is false.
\item\label{liar2} The second sentence on this page with exactly twelve words is false.
\end{menumerate}
Paradoxical sentences seem to be simultaneously both true and false. Try assuming that \ref{liar1} is false, and it seems to follow that it's also true. Try assuming that \ref{liar1} is true, and it comes out false.  Both outcomes are impossible because they're each contradictory. 
Paradoxical sentences frustrate systematic and coherent logical analysis, so logicians prefer to exclude them altogether.

A third problem is that natural languages contain \underidf{ungrounded}{sentence} sentences. 
The following pair is a good example:
\begin{menumerate} 
\item\label{si} Sentence \ref{sii} is true.
\item\label{sii} Sentence \ref{si} is true.
\end{menumerate}
These sentences seem to be unhinged from reality. 
We can assume consistently that both are true, and we can assume consistently that both are false, regardless of any substantive fact about the world. 
Most logicians prefer to exclude ungrounded sentences along with the paradoxical ones.

An important note is in order. 
Paradoxical and ungrounded sentences are different from \underidf{contradictory}{sentence} (or ``self-contradictory'') ones. 
The following is a contradiction, not a paradox:
\begin{menumerate} 
	\item The door is open and it isn't open.
\end{menumerate}
Unlike a paradoxical sentence, which seems to take both truth values, a contradictory sentence must be false. 
There is nothing wrong with contradictory sentences from a logician's point of view. 
They are perfectly legitimate relatives of logical truths. 
Just as a logical truth must be true, a contradiction (i.e., a logical falsehood) must be false. 
Contradictory sentences are self-refuting, but they take a single truth value and so are acceptable for our purposes. 

A fourth problem with natural language is that many, if not most, sentences of natural languages have ambiguous or context-dependent meanings. 
Consider the following. 
\begin{menumerate}
\item\label{syn1} Wealthy Americans flee south in record numbers.
\item\label{syn2} We gave the bananas to the monkeys because they
were hungry.
\item\label{syn3} We gave the bananas to the monkeys because they
were tasty.
\item\label{syn4} We gave the bananas to the monkeys because they
were there.
\item\label{he} Paul insulted George because he thought he was
someone else. 
\item\label{car} Sally's car is red. 
\item\label{tall} He is tall. 
\item\label{here} He is the tallest one here. 
\item\label{indexical} You are tall. 
\item\label{bank} Sally went to the bank. 
\end{menumerate} 
We can often resolve ambiguous sentence meaning by considering the particular circumstances in which a sentence is spoken or written. 
For instance, the ambiguity may come from an indexical term like \mention{you} (e.g., \ref{indexical}), in which case the intended referent is clear once we figure out who is speaking or writing to whom. 
Sometimes the meaning of a term is context-dependent, as with \mention{tall} (e.g., \ref{tall}). 
Five feet would be tall for a seven-year old, but not for an adult. 
Other cases are more complicated. 
Does \mention{Sally's car} in \ref{car} mean the car Sally owns, leases, or something else? 
Is \mention{red} the color of the car's exterior or interior? 
Is it light red, dark red, or something in between? 

Pronoun reference can be difficult to resolve, even in context (e.g., `he' in \ref{he}, \ref{tall}, \ref{here}). 
Finally, sometimes a word in the sentence has multiple distinct meanings (e.g., \mention{bank} in \ref{bank}) or the overall syntax of the sentence is ambiguous (e.g., \ref{syn1}--\ref{syn4}). As a result the intended meaning of the sentence may not be recoverable from the context without additional information.

Ambiguous sentences can make careful logical analysis difficult or impossible. The following looks like a logical truth:
\begin{menumerate}
	\item If she is tall then she is tall.
\end{menumerate}
But what if the first `she' refers to one person and the second refers to another? 

%\footnote{%
%A natural response to ambiguity is that typically it's a feature of sentence \emph{types} (see \pmvref{Some Distinctions} of this chapter). 
%Often the ambiguity is solved at the level of tokens: the context of utterance or writing typically makes it so that ambiguities in a given token sentence's type are resolved. 
%But this isn't always the case. 
%\label{pragmaticsfootnote}
%}

\subsection{Formal Languages}\label{Formal Languages}
The logician avoids these difficulties by using carefully constructed formal languages.\footnote{Poorly (or deviously) constructed formal languages can have the same problems as natural ones.} 
A \underidf{formal}{sentence} language has the following features: 
\begin{cenumerate}
	\item\label{formal1} There is an explicit list of permitted symbols.
	\item\label{formal2} There is a definition of which strings of symbols count as sentences.   
\end{cenumerate} 
A \idf{string} of symbols is just a row or sequence of symbols. 
For example, \mention{fq3H7} is a string consisting of \mention{f} followed by \mention{q}, \mention{3}, \mention{H}, and \mention{7}. The order of the characters is important; \mention{f7q} is not the same string as \mention{7fq}. 

Formal languages have a precisely defined syntax, but no fixed semantics. 
Definitions of sentences in formal languages make no reference to meanings of the sentences or their parts. 
These definitions make reference only to the \emph{form}, or shape of the symbols and their arrangements in strings. 
We should think of the symbols and sentences of a formal language as lacking inherent meaning.

In a well-constructed formal language we can check whether any given string is a genuine sentence, solving the indeterminacy problem faced by natural languages.
We also exclude the features of natural languages that lead to paradoxical and ungrounded sentences. 
Sentences of formal languages have no inherent meaning, but there are ways to assign meaning to them while avoiding the difficulties of natural language. 
In chapter \ref{Translations} we provide methods of translating English sentences into constructed formal languages.

By turning to formal languages we are using a familiar (and often successful) strategy of replacing a hard messy problem with a more tractable mathematical one. 
For example, geometry doesn't deal with points or lines in the physical world; the points and lines of geometry have zero area and zero width, respectively. 
Nevertheless, geometry provides useful models of the physical world when physical points and lines approximate geometrical ones sufficiently.
When the logical features of our formal language sufficiently reflect the logical features of our natural language, mathematical idealization is a productive strategy.\footnote{%  
	Historically the move to formal languages in the study of logic (and mathematics) has been fruitful. 
	One example, important for the development of modern logic, comes from set theory. 
	By using formal methods logicians found that the na\"{i}ve axioms of set theory are inconsistent \citetext{see \citealp{Demopoulos2005} for a quick overview of Russell's paradox, or \citealp[ch~1]{Smullyan2010} for a more careful discussion}.
} 

How do our formal languages relate to natural languages like English?
We think of the formal languages as abstract models of important parts of their natural counterparts. 
We cover several formal languages in this text: \GSL{} (chapter \ref{sententiallogic}), \GQL{}1 (chapter \ref{quantifierlogic1}), \GQL{} (chapter \ref{quantifierlogic}), and later \MGSL{} and \GQLI{} (chapter \ref{furtherdirections}), will serve as progressively more detailed models of English. 
After learning how to define these formal languages and how to translate between them and English, the logic student should have a better understanding of the logical structure of English and of logical consequence more generally.
The student should keep in mind that these formal languages cannot perfectly capture all the desirable logical features of English. 
There is a limit to both the range of English sentences they model and the accuracy with which they model them. 

%\subsection{Further Thoughts on Languages}
%\sq{Natural} languages are not formal.  There is no explicit definition that, given the Latin alphabet, picks out all and only the strings of letters and characters which are English sentences, let alone a definition which makes reference only to the shape of the letters and characters.  (This is a contingent fact. We can imagine hypothetical histories or future developments in which the language of a society, in its natural development, comes to have one or both of these formal features.)
%
%
%The languages we'll be discussing, \GSL{} and \GQL{}, are by contrast \niidf{artificial}\index{language!artificial} in the sense of being created (more or less all at once) for some specific purpose. 
%The line between natural and artificial languages isn't sharp: the heavily technical argot of mathematicians and physicists falls somewhere in between natural languages like English and artificial languages like \GSL{} and \GQL{}. 
%The distinction between formal and informal languages is orthogonal to the distinction between natural and artificial languages, at least in principle. 
%Esperanto, for example, might count as an informal language that's artificial in our sense.

\section{Some Distinctions}\label{Some Distinctions} 
\subsection{Use and Mention}\label{usemention}
There are three distinctions that are important in the modern study of logic. 
The first is the \distinction{use}{mention} distinction\index{\distinction{use}{mention} distinction}. 
For any word or expression (string of words), we can either \emph{use} the word or expression, or instead \emph{mention} it. 
\emph{Using} words and expressions is what we normally do, while \emph{mentioning} a word or expression is one way to talk about that word or expression. 
For example, you could say (i) that a cow is a four-legged bovine, or you could instead say (ii) that \mention{cow} has three letters. 
In case (i) we are talking about a particular kind of animal, while in case (ii) we are talking about the \emph{word} for that animal. 
In this textbook words that are mentioned are put in single quotes.\index{single quotes} Here are some examples:
\begin{menumerate}
\item Houston has over 2 million inhabitants. [True]
\item Houston has over 10 million inhabitants. [False]
\item \mention{Houston} has more than 2 million inhabitants. [False]
\item \mention{Houston} has more than 4 letters. [True]
\item Austin has more than 4 letters. [False]
\item\label{twentyeight} \mention{Austin} has exactly 6 letters. [True]
\end{menumerate}

\subsection{Type and Token}\label{typetoken}
The next distinction is the \distinction{type}{token} distinction.\index{\distinction{type}{token} distinction} 
How many letters does \mention{Houston} have?  The answer depends on whether the asker means letter tokens or letter types.
A \idf{token} is a physical object or event, while a \idf{type} is an abstract \emph{kind} of physical object or event. 
Tokens are located in space and time, while types presumably are not. 
The word \mention{Houston} has 7 letter tokens and 6 letter types, because the letter \mention{o} occurs twice.
In the sentence \mention{Ron went on and on} there are five word tokens and four word types.\footnote{In the sentence \mention{Ron went on and on}, there are three occurrences of the string \mention{on} with one of them in the name \mention{Ron}.  However, this instance doesn't count as a word token in English, because it's only a piece of a word, not a complete word.  Other cases in English are more ambiguous.  In the compound word \mention{footnote} does \mention{note} count as a word token?} 
In \mention{radar} there are five letter tokens and three letter types. 

\subsection{Object Language and Metalanguage}\label{objectandmetalanguage}
The third distinction is that between an object language and a metalanguage. 
Returning to the \distinction{use}{mention} distinction, if one mentions a word, then the \idf{object language}\index{language!object} is the language of the word being mentioned and the \idf{metalanguage}\index{language!meta-} is the language in which that word is mentioned. 
For example, if I say \q{The German word for dog is \mention{Hund},} I've used English (the metalanguage) to talk about German (the object language). 
In this text the metalanguage will always be English augmented with some carefully selected mathematical notation. 
The object language will be a formal language. 

Usually there's a little more to being a metalanguage than the mere fact that it's the language used to talk about another. 
Often there's some specific purpose. 
For example, the metalanguage might be used to give definitions of words in the object language. Or, as in our case, the metalanguage might be used to define when a sentence of the object language is true, or to describe when the logical consequence relationship holds between sentences of the object language. 

%%%%%%%%%%%%%%%%%%%%%%%%%%%%%%%%%%%%%%%%%%%%%%%%%%
\section{MathEnglish}\label{MathEnglish}
%%%%%%%%%%%%%%%%%%%%%%%%%%%%%%%%%%%%%%%%%%%%%%%%%%

The various formal languages in this text are to be object languages.  However, before we can define our first formal language (\GSL{}) we need to familiarize the reader with certain mathematical concepts.  The metalanguage contains a mix of English, mathematical symbols (e.g., set theory operators), and other technical jargon.  We call our metalanguage \idf{MathEnglish}.   

\subsection{Metavariables}\label{metavariables}
One important tool for MathEnglish is the \emph{metavariable}, which we define as a variable for strings of the object language.\index{variables!MathEnglish}  
We use lowercase Greek letters as our metavariables, as we did in the schemas given at the beginning of the chapter.
The three we'll use most commonly are \mention{$\CAPPHI$} (phi), pronounced \q{fie}, \mention{$\CAPTHETA$} (theta), pronounced \q{theta}, and \mention{$\CAPPSI$} (psi), pronounced \q{sigh}.
These Greek letters are not in the object language, but instead are used in the metalanguage for talking about the object language. 

An example will make their role more clear.  Let's say that MathEnglish is our metalanguage and (regular) English is our object language.\footnote{As discussed earlier, in later chapters we use a formal language and not English as our object language, but for the purpose of illustration we temporarily ignore the problems of natural language.}  Recall the logically true sentence from the beginning of the chapter:

\begin{quote}
\noindent{}If the sun is shining then the sun is shining.
\end{quote}

\noindent{}We can replace `the sun is shining' with any other assertion and get another logically true sentence.  With this in mind, consider the following claim of MathEnglish: 

\noindent{}\begin{quote}All English sentences of the form \mention{If $\CAPPHI$ then $\CAPPHI$} are logical truths.\end{quote}

\noindent{}The use of `$\CAPPHI$' helps us talk about \emph{all} English sentences of a certain form as logically true, or at least all English sentences of a certain kind. 
As logicians we are concerned with truth and truth-preservation, so we are only going to address strings with plausible truth values, i.e. declarative sentences. 
A declarative sentence is one that makes an assertion. 
By restricting the substitutions to declaratives, we rule out substitutions such as `Ergle bergle barfle narfle,' 'nef34fwjh9ufh32,' `Kentucky fried chicken,' and `Are you cooking?'. 
We cannot legitimately affirm truth or falsity of these non-assertions. 
In the rest of the text, when we refer to \mention{all} sentences of a natural language we only mean the declarative ones.

In any case, metavariables are indispensable tools for making general claims about object languages.


\subsection{Sets}\label{sets}

MathEnglish makes use of set theory.  A \emph{set} is just a collection of items called \emph{elements}. Elements can be anything, like animals, people, planets, cartoon characters, or even abstract objects such as numbers.  A set can be specified using curly brackets, as in the following set of integers from 1 to 4:

\begin{center}
\noindent{}$\{1, 2, 3, 4\}$
\end{center}

\noindent{}The order of elements in set notation doesn't matter, so the following set is identical to the last:

\begin{center}
\noindent{}$\{2, 3, 1, 4\}$
\end{center}

\noindent{}Furthermore, repetition of an element in set notation is to be disregarded; an element can only be in the set once.  Accordingly, the following sets are the same set as the last two:

\begin{center}
\noindent{}$\{1, 2, 2, 3, 3, 3, 4, 4, 4, 4\}$\\
\noindent{}$\{2, 2, 3, 1, 4, 2, 3, 1, 4, 4, 3\}$
\end{center}

\noindent{}Sets don't have to be finite. Consider the set of positive integers:

\begin{center}
\noindent{}$\{1, 2, 3, 4, ..., 1002, 1003, 1004, ... \}$
\end{center}

There is one set that doesn't have any elements, and it's called either the \emph{null set} or the \emph{empty set}.  Symbolically the empty set is denoted by either $\{ \}$ or $\emptyset$.

Set membership is expressed with the `$\in$' symbol.  Let the Greek letters $\Delta$ and $\Gamma$ stand for arbitrary sets. (They are pronounced \mention{delta} and \mention{gamma}, respectively.)  We assert that $7$ is an element of set $\Delta$ and $3$ is an element of $\Gamma$ as follows: $7 \in \Delta$, $3 \in \Gamma$. On this notation,

\begin{center}
	\noindent{}$2\in\{1, 2, 3\}$
\end{center}

\noindent{}is true and,

\begin{center}
	\noindent{}$4\in\{1, 2, 3\}$ 
\end{center}

\noindent{}is false.

We use lowercase Greek letters $\alpha$ and $\beta$ (pronounced \mention{alpha} and \mention{beta}, respectively) as metavariables for elements. For example, if $\alpha$ and $\beta$ are odd numbers and $\Delta$ is the set of even numbers, then $\alpha + \beta \in \Delta$.

\begin{majorILnc}{\LnpDC{Subset}} For any two sets $\Delta$ and $\Gamma$, $\Delta$ is a \df{subset} of $\Gamma$ \Iff for each $\alpha$ such that $\alpha \in \Delta$, $\alpha \in \Gamma$.
\end{majorILnc} 

\noindent{}For example, $\{1, 2, 3\}$ is a subset of $\{1, 2, 3, 4\}$. The set $\{1, 2\}$ is a subset of each of the two previous sets.  The symbol `$\subseteq$' denotes the subset relation.  So if $\Delta$ is a subset of $\Gamma$, $\Delta \subseteq \Gamma$. Note that by the above definition every set is a subset of itself.

\begin{majorILnc}{\LnpDC{Proper Subset}} For any two sets $\Delta$ and $\Gamma$, $\Delta$ is a \df{proper subset} of $\Gamma$ \Iff 
	\begin{cenumerate}
		\item $\Delta \subseteq \Gamma$ and
		\item there is some $\alpha$ such that $\alpha \notin \Delta$ and $\alpha \in \Gamma$.
	\end{cenumerate}
\end{majorILnc}

\noindent{}As you can probably guess, the \mention{$\notin$} symbol means \mention{is not an element of}. The symbolic notation for \mention{$\Delta$ is a proper subset of $\Gamma$} is $\Delta \subset \Gamma$. No set is a proper subset of itself.

Take care to avoid confusing the membership, subset, and proper subset relations.

\begin{menumerate}
	\item $2\in\{1, 2, 3\}$ [True]
	\item $2\subseteq\{1, 2, 3\}$ [False]
	\item $\{2, 3\}\subseteq\{1, 2, 3\}$ [True]
	\item $\{2, 3\}\in\{1, 2, 3\}$ [False]
	\item $\{2, 3\}\subset\{1, 2, 3\}$ [True]
	\item $\{1, 2, 3\}\subset\{1, 2, 3\}$ [False]
\end{menumerate}

\subsection{More on Sets: Union and Intersection}\label{moreonsets}

Sometimes a set is defined by reference to other sets. Say that there is a set $\Delta$ that contains all the members of two other sets, $\Gamma_1$ and $\Gamma_2$, and nothing else. Then $\Delta$ is said to be the \emph{union} of $\Gamma_1$ and $\Gamma_2$. Let's give a strict definition:

\begin{majorILnc}{\LnpDC{Union}} $\Delta=\Gamma_1\cup\Gamma_2$ \Iff 
	\begin{cenumerate}
		\item for each $\alpha \in \Gamma_1$, $\alpha \in \Delta$ and
		\item for each $\alpha \in \Gamma_2$, $\alpha \in \Delta$ and
		\item for each $\alpha \in \Delta$, $\alpha \in \Gamma_1$ or $\alpha \in \Gamma_2$ (or both).
	\end{cenumerate}	
\end{majorILnc} 

\noindent{}The symbol \mention{$\cup$} means \mention{union}. Some examples:

\begin{center}
	\noindent{}$\{$Mercury, Jupiter, Mars, Saturn$\}=\{$Mercury, Jupiter$\}\cup\{$Mars, Saturn$\}$.
\end{center}

\begin{center}
	\noindent{}$\{$Newton, Bacon, Boyle$\}=\{$Newton, Bacon$\}\cup\{$Bacon, Boyle$\}$.
\end{center}

Another useful concept is \emph{intersection}.  The set $\Delta$ is the intersection of two sets $\Gamma_1$ and $\Gamma_2$ \Iff $\Delta$ contains all the elements shared by both $\Gamma_1$ and $\Gamma_2$ but nothing else. More strictly:

\begin{majorILnc}{\LnpDC{Intersection}} $\Delta=\Gamma_1\cap\Gamma_2$ \Iff 
	\begin{cenumerate}
		\item for each $\alpha$ such that $\alpha \in \Gamma_1$ and $\alpha \in \Gamma_2$, $\alpha \in \Delta$ and
		\item for each $\alpha \in \Delta$, $\alpha \in \Gamma_1$ and $\alpha \in \Gamma_2$.
	\end{cenumerate}	
\end{majorILnc} 

The symbol \mention{$\cap$} means \mention{intersection}.
The intersection of $\{$Mercury, Jupiter$\}$ and $\{$Mars, Saturn$\}$ is the empty set, $\{ \}$, because the former two sets have no members in common.

\begin{center}
	\noindent{}$\{$Mercury, Jupiter$\}\cap\{$Mars, Saturn$\}=\{ \}$.
\end{center}

\noindent{}By contrast, the intersection of $\{$1, 5, 6, 9$\}$ and $\{$2, 3, 5, 6$\}$ is $\{$5, 6$\}$.

\begin{center}
	\noindent{}$\{$1, 5, 6, 9$\}\cap\{$2, 3, 5, 6$\}=\{$5, 6$\}$.
\end{center}

\noindent{}Another example:

\begin{center}
	\noindent{}$\{$Carnap, Quine$\}=\{$Anscombe, Quine, Carnap$\}\cap\{$Carnap, Lewis, Quine$\}$.
\end{center}

\subsection{Ordered Pairs and Ordered n-tuples}\label{orderedpairs}

Sometimes we would like to describe a collection in which, unlike sets, the order does matter.  For that purpose we have \emph{ordered pairs} and \emph{ordered $n$-tuples}.  An ordered pair is a two-place collection in which the order matters.  To differentiate ordered pairs from sets, we indicate them using angled brackets rather than curly brackets.  So, while the sets $\{$1, 2$\}$ and $\{$2, 1$\}$ are identical, the ordered pairs $\langle$1, 2$\rangle$ and $\langle$2, 1$\rangle$ are not.

An ordered collection having more than two places is an $n$-tuple, where $n$ is the number of places needed.  So, $\langle$Mercury, Venus, Earth$\rangle$ is a 3-tuple, $\langle$Jupiter, Saturn, Uranus, Neptune$\rangle$ is a 4-tuple, and so on.

Also unlike sets, for ordered pairs and $n$-tuples, repetition makes a difference. For example, $\langle 1, 2, 3\rangle$ is different from $\langle 1, 2, 3, 3, 3\rangle$.

\subsection{Mathematical Proofs}\label{Mathematical Proofs}

We use proofs to establish that something has certain mathematical properties. 
A proof works from one or more definitions to some interesting result, called a \emph{theorem}. 
Much of this textbook is filled with proofs of theorems about the various formal languages defined in later chapters.
If we want to prove an intermediate result that isn't by itself particularly interesting but which is handy for use in one or more other proofs, we sometimes call it a \emph{lemma}.

To illustrate, let's look at a simple proof.

\begin{THEOREM}{\LnpTC{Transitivity of Subset}}
	If $\Delta_1\subseteq\Delta_2$ and $\Delta_2\subseteq\Delta_3$ then $\Delta_1\subseteq\Delta_3$.
\end{THEOREM}
\begin{PROOF}
	Let there be three sets, $\Delta_1$, $\Delta_2$, and $\Delta_3$, such that $\Delta_1\subseteq\Delta_2$ and $\Delta_2\subseteq\Delta_3$.
	Since $\Delta_1\subseteq\Delta_2$, then by the definition of $\subseteq$, any $\alpha$ in $\Delta_1$ is also in $\Delta_2$.
	And since $\Delta_2\subseteq\Delta_3$, any $\alpha$ in $\Delta_2$ is also in $\Delta_3$.
	It then follows that if $\alpha\in\Delta_1$ then $\alpha\in\Delta_3$. 
	Thus, by the definition of $\subseteq$, $\Delta_1\subseteq\Delta_3$.
\end{PROOF}

\noindent{}We have proved that \mention{$\subseteq$} is transitive. 
Having achieved this result, we permit ourselves to use it throughout the rest of the text without proving it again. 
The end of a completed proof is indicated with a black box: $\blacksquare$.

\subsection{Recursive Definitions}\label{Recursive Definitions}

In later chapters we use MathEnglish to define concepts such as \emph{sentence}, \emph{truth}, and \emph{entailment} for a given object language. 
Many of these definitions are recursive. 
Recursive definitions can be understood as defining which elements are in a given set.

\begin{majorILnc}{\LnpDC{Definition of Recursive Definition}}
A \df{recursive definition} of $\Delta$ has three clauses:
\begin{cenumerate}
\item the \df{base clause(s)}, which specifies one or more objects as elements of $\Delta$,
\item the \df{generating clause(s)}, which specifies one or more ways of generating (or finding) other objects that are elements of $\Delta$, and
\item the \df{closure clause}, which specifies that something is in $\Delta$ only if it can be shown to be so by applications of the first two clauses.
\end{cenumerate}
\end{majorILnc}

\noindent{}One concept that we can define recursively is \emph{natural number}.
\begin{majorILnc}{\LnpDC{Natural Number}} The set $\mathbb{N}$ of natural numbers:
	\begin{cenumerate}
		\item Base Clause: Let $0$ be a natural number. I.e. $0 \in \mathbb{N}$.
		\item Generating Clause: If $\integer{n} \in \mathbb{N}$ then $\integer{n}+1 \in \mathbb{N}$.
		\item Closure Clause: Nothing else is in $\mathbb{N}$.  
	\end{cenumerate}
\end{majorILnc}
This definition unambiguously identifies infinitely many numbers as natural numbers. 
Obviously $0$ is a natural number. 
Is $7$ a natural number? 
It is. We know $0$ is a natural number because the base clause says so. 
Thus, according to the generating clause, $0+1$ (i.e., $1$) is also a natural number. Apply the generating clause again to show that $1+1$ (i.e., $2$) is natural number. One can continue applying the generating clause until $7$ is reached. 
Is $-3$ a natural number? No. $-3$ is less than $0$ and the generating clause only defines larger numbers as natural. The closure clause rules out the inclusion of anything else.

Let's use a recursive definition to identify which people are ancestors of French mathematician Blaise Pascal.

\begin{cenumerate}
	\item Base Clause: Blaise Pascal's mother and father are ancestors of Blaise Pascal.
	\item Generating Clause: If $x$ is an ancestor of Blaise Pascal and $y$ is a parent of $x$, then $y$ is also an ancestor of Blaise Pascal.
	\item Closure Clause: No one else is an ancestor of Blaise Pascal.
\end{cenumerate}
\noindent{}This definition picks out the mother and father of Blaise Pascal, their respective mothers and fathers, the mothers and fathers of the latter, and so on.  It excludes everyone else. Recursive definitions are powerful, because in a few lines we can fix an unambiguous collection that is arbitrarily large. The closure clause is usually relatively trivial (e.g., ``Nothing else is an $x$.'').

\subsection{Conclusion}\label{Conclusion}

Now we have most of the mathematical tools we need to define our first formal language, \emph{Sentential Logic}. Anything else we need will be given along the way.

%%%%%%%%%%%%%%%%%%%%%%%%%%%%%%%%%%%%%%%%%%%%%%%%%%
\section{Exercises}
%%%%%%%%%%%%%%%%%%%%%%%%%%%%%%%%%%%%%%%%%%%%%%%%%%

\notocsubsection{Sets}{ex:Sets}

\begin{enumerate}
	\item True or false: $\{2, 3\}\subseteq\{1, 2, 3\}$
    \item True or false: $\{1, 2, 3\}\subset\{1, 2, 3\}$
	\item True or false: $\{2, 3\}\in\{1, 2, 3\}$
    \item True or false: $\{2, 3\}\in\{1, \{2, 3\}\}$
    \item True or false: $2\in\{1, \{2, 3\}\}$
    \item True or false: $\{1, 2, 3\}=\{1, \{2, 3\}\}$
    \item True or false: $\{1, 2, 3\}=\{3, 3, 1, 2, 3, 2, 2, 1\}$
\end{enumerate}

\notocsubsection{Unions and Intersections}{ex:Unions and Intersections}

\begin{enumerate}
	\item True or false: $\{2,4,6\}\cup\{1,2,3\}\subseteq\{1,2,3,4,5,6\}$
	\item True or false: $\{2,4,6\}\cap\{1,2,3\}=\{1,2,3,4,5,6\}$
    \item True or false: $\{1,2\}\cap\{3,4\}\subseteq\{1,2\}\cup\{3,4\}$
	\item True or false: $\{1\}\cup\{2, 3\}=\{1, \{2, 3\}\}$
\end{enumerate}

\notocsubsection{Proofs}{ex:Simple Proofs}
Prove each of the following.
\begin{enumerate}
	\item $\Delta_1\cap\Delta_2\subseteq\Delta_1\cup\Delta_2$.
	\item If $\Delta_1\subseteq\Delta_2$ then $\Delta_1\cup\Delta_3\subseteq\Delta_2\cup\Delta_3$.
    \item If $\Delta_1\subseteq\Delta_2$ then $\Delta_1\cap\Delta_3\subseteq\Delta_2\cap\Delta_3$.
\end{enumerate}

\notocsubsection{Recursive Definitions}{ex:Recursive Definitions}
Define the following recursively.
\begin{enumerate}
	\item The ancestors of George Washington.
	\item All integers.
    \item All even numbers.
\end{enumerate}

%\theendnotes



%\addtocontents{toc}{\protect\thispagestyle{empty}}
%%%%%%%%%%%%%%%%%%%%%%%%%%%%%%%%%%%%%%%%%%%%%%%%%%
\chapter{Sentential Logic}\label{sententiallogic}
%%%%%%%%%%%%%%%%%%%%%%%%%%%%%%%%%%%%%%%%%%%%%%%%%%
%\AddToShipoutPicture*{\BackgroundPicA}


%%%%%%%%%%%%%%%%%%%%%%%%%%%%%%%%%%%%%%%%%%%%%%%%%%
\section{The Language \GSL{}}\label{The Language GSL}
%%%%%%%%%%%%%%%%%%%%%%%%%%%%%%%%%%%%%%%%%%%%%%%%%%

\subsection{Sentences of \GSL{}}\label{Sentences of GSL}
Our first task is to define the syntax of the basic formal language \GSL{}.\footnote{Work on modern sentential logic originated with George Boole \citeyearpar{Boole1854} and Augustus De Morgan \citeyearpar{DeMorgan1847,DeMorgan1860}. 
	As Christine Ladd-Franklin notes \citeyearpar[17]{LaddFranklin1883} before giving her own, variations quickly followed from William S. Jevons, Ernst Schr\"oder, Hugh McColl, and Charles S. Peirce.} 
This language is variously called \idf{sentential logic} or \niidf{propositional logic}\index{propositional logic|see{sentential logic}}. 
We use the former name because it's easier to get a grasp on what a sentence is. What a proposition is is a matter of intense philosophical debate.\footnote{For historical and contemporary discussions of propositions, see \citealt{Frege1892}, \citealt[13,47]{Russell1903}, \citealt[26]{Church1956}, \citealt[ch.~1]{Quine1986}, \citealt[ch.~3]{Schiffer1987}, \citealt{Grandy1993}, \citealt{Bealer1998b}, \citealp{King2007}, \citealt{Soames2010}.}

As stated in the previous chapter, each formal language\index{language} requires (1) a list of basic symbols, and (2) a specification of which sequences of those symbols count as sentences\index{sentence}. Think of these as determining the proper grammar or syntax of the language. 

\begin{majorILnc}{\LnpDC{Basic Symbols of GSL}} The \df{basic symbols} of \GSL{} are of three kinds:\footnote{The commas and ellipses are \emph{not} symbols of \GSL{}.}
\begin{cenumerate}
\item Logical Connectives: $\NEGATION$, $\WEDGE$, $\VEE$, $\HORSESHOE$, $\TRIPLEBAR$
\item Punctuation Symbols: (, )
\item Sentence Letters: $\Al,\Bl, \ldots, \Tl, \Al_1,\Bl_1, \ldots, \Tl_1, \Al_2, \Bl_2, \ldots$  
\end{cenumerate}
\end{majorILnc} 
\noindent{}Logical connectives are sometimes called logical symbols, logical operators, logical terms, or logical functors.%
\footnote{%
In other textbooks there are sometimes different symbols used for these connectives. 
Along with $\NEGATION$, $\neg$ and $-$ are used for negation, $\&$ and $\cdot$ for conjunction, $\supset$ and $\Rightarrow$ for conditionals, and $\equiv$ for biconditionals. $\VEE$ is almost universally the symbol for disjunction.
} 
One could draw important distinctions between symbols, terms, and operators, but the names are just as often used interchangeably.
The \emph{sentence letters} are italicized capital letters of the Roman alphabet from \mention{$\Al$} to \mention{$\Tl$}.  To give ourselves an infinite supply, any of these letters with a subscripted positive integer is also a sentence letter; e.g. $\Cl$ and a subscripted $7$ can be combined to get the new sentence letter $\Cl_7$. Keep in mind that $\Cl_7$ and $\Cl$ are different sentences.

\begin{majorILnc}{\LnpDC{Recursive definition of Sentences of GSL}} The \nidf{sentences} \underdf{of \GSL{}}{sentence} are given by the following recursive definition:
\begin{description}
\item[Base Clause:] Every sentence letter is a sentence.
\item[Generating Clauses:] \hfill
\begin{cenumerate}
\item If $\CAPPHI$ is a sentence, then so is $\negation{\CAPPHI}$.\footnote{Remember from Chapter 1 that $\CAPPHI$ and $\CAPTHETA$ are used as metavariables. In this definition they stand for sentences of \GSL{}.}
\item If $\CAPPHI$ and $\CAPTHETA$ are sentences, then so are both $\parhorseshoe{\CAPPHI}{\CAPTHETA}$ and $\partriplebar{\CAPPHI}{\CAPTHETA}$.
\item If all of $\CAPPHI_1,\CAPPHI_2,\ldots,\CAPPHI_{\integer{n}}$ are sentences (where $\integer{n}$ is an integer $\geq2$), then so are $\parconjunction{\CAPPHI_1}{\conjunction{\CAPPHI_2}{\conjunction{\ldots}{\CAPPHI_{\integer{n}}}}}$ and $\pardisjunction{\CAPPHI_1}{\disjunction{\CAPPHI_2}{\disjunction{\ldots}{\CAPPHI_{\integer{n}}}}}$.
\end{cenumerate}
\item[Closure Clause:] No other string is an \GSL{} sentence.
\end{description}
\end{majorILnc}

\noindent{}Here are some example \GSL{} sentences:
	
\begin{multicols}{2}
	\begin{smenumerate}
	\item $\Bl$
	\item $\negation{\Bl}$
	\item $\parhorseshoe{\negation{\Bl}}{\Cl}$
	\item $\negation{\partriplebar{\negation{\Bl}}{\Cl}}$
	\item $\pardisjunction{\Al}{\disjunction{\Cl}{\Dl}}$
	\item $\pardisjunction{\parhorseshoe{\Al}{\El}}{\disjunction{\Cl}{\disjunction{\negation{\Dl}}{\Gl}}}$
	\item $\parconjunction{\pardisjunction{\Al}{\Bl}}{\pardisjunction{\Cl}{\Dl}}$
	\item $\negation{\parconjunction{\Al}{\negation{\parhorseshoe{\Bl}{\Cl}}}}$
	\end{smenumerate}
\end{multicols}

Sentence letters are sometimes called \underidf{atomic}{sentence} sentences.
Sentences of the form $\negation{\CAPPHI}$ are \idf{negations}.
Sentences of the form $\parhorseshoe{\CAPPHI}{\CAPTHETA}$ are \idf{conditionals}, and those of the form $\partriplebar{\CAPPHI}{\CAPTHETA}$ are \idf{biconditionals}. 
The left-hand side of the conditional, $\CAPPHI$, is traditionally called the \idf{antecedent}\index{LHS} and the right-hand side, $\CAPTHETA$, the \idf{consequent}\index{RHS}. 
We often use the alternative terminology \CAPS{lhs} (left-hand side) and \CAPS{rhs} (right-hand side).
Sentences of the form $\parconjunction{\CAPPHI_1}{\conjunction{\CAPPHI_2}{\conjunction{\ldots}{\CAPPHI_{\integer{n}}}}}$ are \idf{conjunctions} and their component sentences (e.g. $\CAPPHI_1$) are \idf{conjuncts}.
Sentences of the form $\pardisjunction{\CAPPHI_1}{\disjunction{\CAPPHI_2}{\disjunction{\ldots}{\CAPPHI_{\integer{n}}}}}$ are \idf{disjunctions} and their component sentences are \idf{disjuncts}.\footnote{
	Many logic books treat conjunction and disjunction as binary (2-place), e.g., \mention{$\pardisjunction{\Al}{\Bl}$}. According to our definition \mention{$\pardisjunction{\disjunction{\Al}{\Bl}}{\Cl}$} is also a perfectly good sentence. Textbooks that treat disjunctions as binary require an extra set of parentheses to generate an equivalent sentence: e.g., \mention{$\pardisjunction{\pardisjunction{\Al}{\Bl}}{\Cl}$}. This is also an acceptable sentence of \GSL{}, but the extra parentheses don't add interesting information. We prefer the definitions of conjunction and disjunction given above because they're closer to natural English and they allow us to avoid unnecessary parentheses.
}

The base and generating clauses tell us which strings are sentences, but they don't say which strings aren't.
How do we know, for example, that \mention{$(\Bl(\HORSESHOE{}\Al$} isn't an \GSL{} sentence?
That's what the closure clause is for.
It explicitly excludes strings that cannot be constructed with the base and generating clauses.

Note that $\negation{\CAPPHI}$, $\parhorseshoe{\CAPPHI}{\CAPTHETA}$, etc. in the generating clauses are \textbf{not \GSL{} sentences}.
That's because metavariables aren't included in the symbols of \GSL{}.
These \idf{sentence schemas} are strings that can be made into sentences by substituting sentences for metavariables.
For example, the substitution $\CAPPHI=\;$\mention{$\Al$}, $\CAPTHETA=\;$\mention{$\partriplebar{\Cl}{\Dl}$} in the schema $\parhorseshoe{\CAPPHI}{\CAPTHETA}$ results in the sentence \mention{$\parhorseshoe{\Al}{\partriplebar{\Cl}{\Dl}}$}.\footnote{When Greek letters \mention{$\CAPPHI$}, \mention{$\CAPPSI$}, \mention{$\CAPTHETA$}, etc. are used it should usually be assumed that they range over \GSL{} sentences and not mere strings of \GSL{} symbols. 
In exceptional cases we intend for them to range over all strings of \GSL{} symbols, as we did in the definition just given of \GSL{} sentences.}

For any sequence of \GSL{} symbols it is possible to \emph{prove} whether it is a sentence.

\begin{majorILnc}{\LnpEC{Example of Recursive definition of GSL sentences}}
$\parconjunction{\cpardisjunction{\Al}{\parhorseshoe{\Dl}{\Bl}}}{\negation{\Gl}}$ is a sentence of \GSL{}. 

\noindent{}The base clause of \ref{Recursive definition of Sentences of GSL} defines each of $\Al$, $\Dl$, $\Bl$, and $\Gl$ as sentences. 
From $\Dl$ and $\Bl$, and by generating clause (2), $\parhorseshoe{\Dl}{\Bl}$ is a sentence. 
From $\Gl$ and generating clause (1), $\negation{\Gl}$ is a sentence. 
From $\Al$ and $\parhorseshoe{\Dl}{\Bl}$, and generating clause (3), $\cpardisjunction{\Al}{\parhorseshoe{\Dl}{\Bl}}$ is a sentence. 
And, finally, from $\cpardisjunction{\Al}{\parhorseshoe{\Dl}{\Bl}}$ and $\negation{\Gl}$, and generating clause (3), $\parconjunction{\cpardisjunction{\Al}{\parhorseshoe{\Dl}{\Bl}}}{\negation{\Gl}}$ is a sentence. 
\end{majorILnc}

\subsection{Official and Unofficial Sentences of \GSL{}}\label{Unofficial Sentences of GSL}

Definition \ref{Recursive definition of Sentences of GSL} is the definition of an \emph{official} sentence of \GSL{}.
For convenience' sake we often work with \emph{un}official sentences. 
\begin{majorILnc}{\LnpDC{Unofficial Sentence of GSL}}
A string of symbols is an \nidf{unofficial} sentence\index{sentence!unofficial (of \GSL{})|textbf} \Iff we can obtain it from an official sentence by
\begin{cenumerate}
\item deleting outer parentheses, or
\item replacing one or more pairs of official round parentheses ( ) with square brackets [ ] or curly brackets \{ \}.
\end{cenumerate}
\end{majorILnc}
\noindent{}Thus \mention{$\conjunction{\Al}{\conjunction{\Bl}{\Cl}}$} is an unofficial sentence, as are
\begin{multicols}{2}
\begin{smenumerate}
\item\label{usex1} $\negation{\pardisjunction{[\Al\wedge\Bl]}{[\Cl\wedge\Dl]}}$
\item $\conjunction{\parconjunction{\Al}{\Bl}}{\Cl}$
\item $\horseshoe{\parconjunction{\Al}{\Bl}}{\Cl}$
\item $\parconjunction{\Al}{[\Bl\rightarrow\Cl]}$
\item $\disjunction{\negation{\Al}}{\Cl}$
\item $[\parconjunction{\Al}{\Bl}\wedge\Cl]$
\item $\conjunction{\{\Al\wedge\Bl\}}{\Cl}$
\item\label{usexL} $\conjunction{\Al}{[\Bl\rightarrow\Cl]}$
\end{smenumerate}
\end{multicols}
\noindent{}From an unofficial sentence we can unambiguously reconstruct the related official sentence. 
Throughout the rest of this text we usually drop outer parentheses, but we will consistently use the standard parentheses ( ). 
The reader should feel free to use brackets [ ] or curly parentheses \{ \} as is helpful. 

\subsection{A Comment on Use and Mention}\label{use mention comment}

Most of the time when you see Greek letters in the text, as in definition \mvref{Recursive definition of Sentences of GSL}, we are \emph{using} them, not mentioning them. 
Thus it's appropriate that in definition \ref{Recursive definition of Sentences of GSL} we did not put them in single quotes.\index{single quotes}
By contrast, we generally \emph{mention} official and unofficial \GSL{} sentences rather than use them.
Because we mention \GSL{} sentences so often it would be tedious to put them in quotes. 
Therefore we refrain from doing so unless there is some special reason to do so.  
We are also less strict when mentioning the basic symbols of \GSL{}.\footnote{This is the usual convention. See e.g. \citealt[7]{Hodges2001}.}

\subsection{Other Properties of Sentences}\label{Other Properties of GSL Sentences}
Next we define four important properties and related features of sentences: subsentence, order, main connective, and construction tree. 
\begin{majorILnc}{\LnpDC{Subsentences}}
The following clauses define when one sentence is a \df{subsentence} of another:
\begin{cenumerate}
\item\label{ss1} Every sentence is a subsentence of itself.
\item $\CAPPHI$ is a subsentence of $\negation{\CAPPHI}$.
\item $\CAPPHI$ and $\CAPTHETA$ are subsentences of $\parhorseshoe{\CAPPHI}{\CAPTHETA}$ and $\partriplebar{\CAPPHI}{\CAPTHETA}$.
\item\label{ss4} Each of $\CAPPHI_1,\CAPPHI_2,\ldots,\CAPPHI_{\integer{n}}$ is a subsentence of $\parconjunction{\CAPPHI_1}{\conjunction{\CAPPHI_2}{\conjunction{\ldots}{\CAPPHI_{\integer{n}}}}}$\\ and $\pardisjunction{\CAPPHI_1}{\disjunction{\CAPPHI_2}{\disjunction{\ldots}{\CAPPHI_{\integer{n}}}}}$.
\item\label{ss5} (Transitivity) If $\CAPPHI$ is a subsentence of $\CAPTHETA$ and $\CAPTHETA$ is a subsentence of $\CAPPSI$, then $\CAPPHI$ is a subsentence of $\CAPPSI$.
\item\label{ss6} That's all. 
\end{cenumerate}
\end{majorILnc}
\begin{majorILnc}{\LnpEC{SubSentenceExampleA}}
	The sentence $\parhorseshoe{\Bl}{\Cl}$ has 3 subsentences:
	\begin{cenumerate}
		\item $\parhorseshoe{\Bl}{\Cl}$
		\item $\Bl$
		\item $\Cl$
	\end{cenumerate}
\end{majorILnc}
\noindent{}Subsentences are counted by token, not type. Hence the similar $\parhorseshoe{\Bl}{\Bl}$ also has three subsentences; the two tokens of $\Bl$ are counted separately.
\begin{majorILnc}{\LnpEC{SubSentenceExampleB}}
$\conjunction{\cpardisjunction{\Al}{\parhorseshoe{\Dl}{\Bl}}}{\negation{\Gl}}$ has 8 subsentences:
\begin{multicols}{2}
\begin{cenumerate}
\item $\conjunction{\cpardisjunction{\Al}{\parhorseshoe{\Dl}{\Bl}}}{\negation{\Gl}}$
\item $\disjunction{\Al}{\parhorseshoe{\Dl}{\Bl}}$
\item $\horseshoe{\Dl}{\Bl}$
\item $\Al$
\item $\Dl$
\item $\Bl$
\item $\negation{\Gl}$
\item $\Gl$
\end{cenumerate}
\end{multicols}
\end{majorILnc}
\begin{majorILnc}{\LnpDC{Proper Subsentences}}
	A sentence $\CAPPHI$ is a \df{proper subsentence} of $\CAPPSI$ \Iff $\CAPPHI$ is a subsentence of but isn't identical to $\CAPPSI$.
\end{majorILnc}
\noindent{}Each sentence is a subsentence of itself, but no sentence is a proper subsentence of itself.
\begin{majorILnc}{\LnpDC{Order}}
The following clauses define the \df{order} of every \GSL{} sentence.\footnote{\citetext{\citealt{Post1921}, \citealt[11]{Hodges2001}}} Let $\ORD{\CAPPHI}$ be the order of $\CAPPHI$. Then: 
\begin{cenumerate}
\item If $\CAPPHI$ is an atomic sentence (a sentence letter), then $\ORD{\CAPPHI}=1$.
\item For any sentence $\CAPPHI$, $\ORD{\negation{\CAPPHI}}=\ORD{\CAPPHI}+1$.
\item For any sentences $\CAPPHI$ and $\CAPTHETA$, $\ORD{\parhorseshoe{\CAPPHI}{\CAPTHETA}}$ is one greater than the max of $\ORD{\CAPPHI}$ and $\ORD{\CAPTHETA}$. Likewise, $\ORD{\partriplebar{\CAPPHI}{\CAPTHETA}}$ is one greater than the max of $\ORD{\CAPPHI}$ and $\ORD{\CAPTHETA}$.
\item For any sentences $\CAPPHI_1,\ldots,\CAPPHI_\integer{n}$, $\ORD{\parconjunction{\CAPPHI_1}{\conjunction{\ldots}{\CAPPHI_\integer{n}}}}$ is one greater than the max of $\ORD{\CAPPHI_1}$, $\ldots$, $\ORD{\CAPPHI_\integer{n}}$.
\item For any sentences $\CAPPHI_1,\ldots,\CAPPHI_\integer{n}$, $\ORD{\pardisjunction{\CAPPHI_1}{\disjunction{\ldots}{\CAPPHI_\integer{n}}}}$ is one greater than the max of $\ORD{\CAPPHI_1}$, $\ldots$, $\ORD{\CAPPHI_\integer{n}}$. 
\item That's all.
\end{cenumerate}
\end{majorILnc}
\begin{majorILnc}{\LnpEC{OrderExampleA}}
What is the order of $\conjunction{\cpardisjunction{\Al}{\parhorseshoe{\Dl}{\Bl}}}{\negation{\Gl}}$?
The order of an atomic sentence is 1, so $\Al$, $\Dl$, $\Bl$, $\Gl$ each have order 1.
The order of $\horseshoe{\CAPPHI}{\CAPPSI}$ is $1$ plus the maximum of the orders of $\CAPPHI$ and $\CAPPSI$; thus the order of $\horseshoe{\Dl}{\Bl}$ is 2.  
Because $\ORD{\negation{\CAPPHI}}=\ORD{\CAPPHI}+1$, $\ORD{(\negation{\Gl})}=\ORD{(\Gl)}+1=2$. 
Because the order of a disjunction $\disjunction{\CAPPHI_\integer{1}}{\disjunction{\ldots}{\CAPPHI_{\integer{n}}}}$ is $1$ plus the maximum order of the disjuncts, the order of $\disjunction{\Al}{\parhorseshoe{\Dl}{\Bl}}$ is 3.
The same goes for conjunctions, so the order of $\conjunction{\cpardisjunction{\Al}{\parhorseshoe{\Dl}{\Bl}}}{\negation{\Gl}}$ is 4.
\end{majorILnc}
\begin{majorILnc}{\LnpDC{GSL Main connective}}
The \nidf{main connective}\index{main connective!of GSL|textbf} is the connective token (or tokens) that occur(s) in the sentence but in no proper subsentence.  
\end{majorILnc}
\begin{majorILnc}{\LnpEC{GSLMainConnectiveExampleA}}
The main connective in each of the following sentences has been underlined.
\begin{multicols}{2}
\begin{cenumerate}
\item $(\Al\VEE(\Dl\HORSESHOE\Bl))\, \underline{\WEDGE}\, \negation{\Gl}$
\item $\Ll\, \underline{\VEE}\, \Kl\, \underline{\VEE}\, \Hl $
\item $\Ll\, \underline{\VEE}\, \parhorseshoe{\Al}{\Bl}\, \underline{\VEE}\, \Hl $
\item $\underline{\NEGATION}(\Ll\VEE\Kl\VEE\Hl)$
\item $(((\Dl\!\HORSESHOE\!\El)\VEE\Al)\underline{\WEDGE}(\NEGATION\Bl\WEDGE\NEGATION(\Cl\!\TRIPLEBAR\!\Hl)))$
\item $(\NEGATION\Bl\, \underline{\WEDGE}\, \NEGATION(\Cl\TRIPLEBAR\Hl))$
\end{cenumerate}
\end{multicols}
\end{majorILnc}

\begin{majorILnc}{\LnpDC{Construction Tree}}
The \df{construction tree} for a sentence is a diagram of how the sentence is generated through the recursive clauses of the definition of \GSL{} sentences. We put atomic sentences as leaves at the top, and the generating clauses specify how we can join nodes of the tree together (starting with the leaves at the top) into new nodes. The complete sentence is the node at the base of the tree. 
\end{majorILnc}
\begin{majorILnc}{\LnpEC{ConstructionTreeExampleA}}
Give the construction tree for $\conjunction{\cpardisjunction{\Al}{\parhorseshoe{\Dl}{\Bl}}}{\negation{\Gl}}$.
\begin{center}
\begin{tikzpicture}[grow=up]
\tikzset{level distance=40pt}
\tikzset{level 1/.style={level distance=60pt}}
\tikzset{sibling distance=32pt}
\tikzset{every tree node/.style={align=center,anchor=north}}
	\Tree%http://angasm.org/papers/qtree/    http://www.ling.upenn.edu/advice/latex/qtree/qtreenotes.pdf
[.{$\conjunction{\cpardisjunction{\Al}{\parhorseshoe{\Dl}{\Bl}}}{\negation{\Gl}}$}
  [.{$\negation{\Gl}$} %!{\qsetw{3in}}
  [.{$\Gl$}
  ]
  ]
  [.{$\disjunction{\Al}{\parhorseshoe{\Dl}{\Bl}}$} 
    [.{$\horseshoe{\Dl}{\Bl}$} 
      [.{$\Bl$} 
      ]
      [.{$\Dl$} %!{\qsetw{2in}}
      ] 
    ]
    [.{$\Al$}
    ]    
  ]
]%
	%\caption{Example formula tree}
	%\label{fig:ExampleFormulaTree}
\end{tikzpicture}
\end{center}
\end{majorILnc}
\begin{majorILnc}{\LnpEC{ConstructionTreeExampleB}}
Give the construction tree for $((\Cl \WEDGE \Dl) \HORSESHOE \Al)\TRIPLEBAR (\Dl \VEE \Hl)$.
\begin{center}
\begin{tikzpicture}[grow=up]
\tikzset{level distance=42pt}
\tikzset{sibling distance=32pt}
\tikzset{every tree node/.style={align=center,anchor=north}}
	\Tree%http://angasm.org/papers/qtree/    http://www.ling.upenn.edu/advice/latex/qtree/qtreenotes.pdf
[.{$((\Cl \WEDGE \Dl) \HORSESHOE \Al)\TRIPLEBAR (\Dl \VEE \Hl)$}
  [.{$\Dl\VEE \Hl$} %!{\qsetw{3in}}
    [.{$\Hl$} %!{\qsetw{2in}}
    ]   
    [.{$\Dl$}
    ]  
  ]
  [.{$(\Cl\WEDGE \Dl)\HORSESHOE \Al$} 
    [.{$\Al$} %!{\qsetw{2in}}
    ]  
    [.{$\Cl\WEDGE \Dl$} 
      [.{$\Dl$} %!{\qsetw{2in}}
      ]     
      [.{$\Cl$} 
      ]
    ]
  ]
]%
	%\caption{Example formula tree}
	%\label{fig:ExampleFormulaTree}
\end{tikzpicture}
\end{center}
\end{majorILnc}
\noindent{}There are some helpful relationships between the construction tree of a sentence,  its order, its subsentences, and its main connective. 
The subsentences of a sentence are the nodes in the sentence's construction tree.
The order of a sentence is the number of nodes of its longest branch from root to leaf, i.e., the height of the tree. 
The main connective of a sentence is the connective added last (at the bottom) of the construction tree. 
\begin{majorILnc}{\LnpEC{ConstructionTreeExampleC}}
Consider again the construction tree for $\conjunction{\cpardisjunction{\Al}{\parhorseshoe{\Dl}{\Bl}}}{\negation{\Gl}}$. 
Find the order of the sentence by counting the height of the branches of the tree. 
(We count up as we work our way down the branches.)
Note that the answer we get agrees with that computed in example \mvref{OrderExampleA}.
\begin{center}
\begin{tikzpicture}[grow=up]
\tikzset{level distance=40pt}
\tikzset{level 1/.style={level distance=60pt}}
\tikzset{sibling distance=40pt}
\tikzset{every tree node/.style={align=center,anchor=north}}
	\Tree%http://angasm.org/papers/qtree/    http://www.ling.upenn.edu/advice/latex/qtree/qtreenotes.pdf
[.{$\conjunction{\cpardisjunction{\Al}{\parhorseshoe{\Dl}{\Bl}}}{\negation{\Gl}}$ \textbf{[4]}}
  [.{$\negation{\Gl}$ \textbf{[2]}} %!{\qsetw{3in}}
  [.{$\Gl$ \textbf{[1]}}
  ]
  ]
  [.{$\disjunction{\Al}{\parhorseshoe{\Dl}{\Bl}}$ \textbf{[3]}} 
    [.{$\horseshoe{\Dl}{\Bl}$ \textbf{[2]}} 
      [.{$\Bl$ \textbf{[1]}} 
      ]
      [.{$\Dl$ \textbf{[1]}} %!{\qsetw{2in}}
      ] 
    ]
    [.{$\Al$ \textbf{[1]}}
    ]    
  ]
]%
	%\caption{Example formula tree}
	%\label{fig:ExampleFormulaTree}
\end{tikzpicture}
\end{center}
\end{majorILnc}

\subsection{How Many \GSL{} Sentences are There?}
There are at least as many sentence letters as there are natural numbers, and the sentence letters are a proper subset of the set of sentences. 
Are there more sentences than natural numbers?  
Below we prove there are not by matching up sentences with natural numbers.
\begin{THEOREM}{\LnpTC{Number of sentences}}
The number of \GSL{} sentences is equal to the number of natural numbers.
\end{THEOREM}
\begin{PROOF}
First, assign each sentence letter a natural number that only contains the digit \mention{$1$}, for example:
\begin{center}
\begin{tabular}{ c c c c c }
$\Al$ & $\Bl$ & $\Cl$ & $\Dl$ & $\ldots$ \\
1 & 11 & 111 & 1111 & $\ldots$ \\
\end{tabular}
\end{center}
Next, assign numbers to the other symbols of \GSL{}, for example:
\begin{center}
\begin{tabular}{ c c c c c c c }
$\NEGATION$ & $\WEDGE$ & $\VEE$ & $\HORSESHOE$ & $\TRIPLEBAR$ & ( & ) \\
2 & 3 & 4 & 5 & 6 & 7 & 8 \\
\end{tabular}
\end{center}
Given any sentence, replace its symbols with the associated numbers.
For example, \mbox{$\negation{\parconjunction{\Al}{\conjunction{\Bl}{\negation{\Dl}}}}$} gets mapped to 2713113211118.
For any sentence of \GSL{}, there is a unique natural number defined by this process. 
For any natural number we can determine if it represents an \GSL{} sentence, and if so which one.
\end{PROOF}

\noindent{}This is not the most efficient way of representing sentences with numbers, but it is a simple one that avoids the use of special properties (e.g., being a prime number).


%%%%%%%%%%%%%%%%%%%%%%%%%%%%%%%%%%%%%%%%%%%%%%%%%%
\section{Models}\label{Interpretations}
%%%%%%%%%%%%%%%%%%%%%%%%%%%%%%%%%%%%%%%%%%%%%%%%%%

\GSL{} is a formal language, so its sentences are mere strings of symbols without any inherent meaning---they don't ``say anything'' about the world.\footnote{
	We mentioned this property of formal languages in \ref{Formal Languages}.
}
As a consequence, sentences of \GSL{} lack an inherent truth value. That is, they are neither true nor false. 
Even so, nothing stops us from \emph{assigning} truth values to \GSL{} sentences. 
In this section we explain how to do so consistently.\footnote{
	You might ask why we don't first assign \GSL{} sentences meanings, then determine whether they are true or false based on those meanings. 
	There are difficulties associated with assigning meanings that would complicate our project unnecessarily. (We show how to \emph{interpret} \GSL{} sentences as having meanings in chapter \ref{Translations}.)
	One of the most important discoveries in logic was that for \GSL{} only truth values matter; all other details of meaning are irrelevant.
}

\subsection{Truth in a Model}\label{Truth in an Interpretation} 

Truth values are assigned to sentences by models.

\begin{majorILnc}{\LnpDC{Definition of GSL interpretation}}
A \df{model of an \GSL{} sentence $\CAPPHI$} is an assignment of a truth value, either $\True$ or $\False$, to each sentence letter in $\CAPPHI$.
\end{majorILnc}

\noindent{}We can think of a model of $\CAPPHI$ as a function from the set of sentence letters of $\CAPPHI$ to the set of truth values: $\{\TrueB, \FalseB\}$. 
We use the letter \mention{$\IntA$} (a fraktur-style \mention{m}) to represent such a function.
If a model $\IntA$ assigns a sentence letter $\CAPPSI$ the value \mention{$\True$}, $\IntA(\CAPPSI)=\TrueB$.
For the value \mention{$\False$}, $\IntA(\CAPPSI)=\FalseB$.\footnote{
	The definiton of \mention{model} we use assigns one of two truth values to each sentence letter, but that's not the only way to define them. Other definitions assign more than two.
	The assumption that there are only two truth values simplifies analysis and is widely shared, but whether two is enough is a matter of philosophical debate.
	Nevertheless, even if our definition of a model is a simplification, it is a historically fruitful one and helps us better understand logical consequence.  In chapter \ref{furtherdirections} we discuss formal languages with additional truth values.
}

To illustrate, a model for $\pardisjunction{\disjunction{\Al}{\Bl}}{\Cl}$ assigns a truth value to each of the sentence letters $\Al$, $\Bl$, and $\Cl$.
Any model for $\pardisjunction{\disjunction{\Al}{\Bl}}{\Cl}$ is therefore also a model for $\parconjunction{\conjunction{\Al}{\Bl}}{\Cl}$ and $\parconjunction{\parhorseshoe{\Al}{\Bl}}{\negation{\Cl}}$; they all have the same sentence letters.

We often speak informally of \emph{models} without making reference to any particular \GSL{} sentence.  It's useful to talk this way because any given model of $\CAPPHI$ is a model of any other \GSL{} sentence with the same sentence letters, or a subset of them.  If $\IntA$ makes assignments to $\Al$, $\Bl$, and $\Cl$, then $\IntA$ is a model of all the sentences that only contain sentence letters from that list.  Accordingly, we define a model for a \emph{set} of \GSL{} sentences.

\begin{majorILnc}{\LnpDC{Definition of Model for Set}}
	$\IntA$ is a \df{model of a set of sentences $\Delta$} \Iff $\IntA$ is a model for each sentence in $\Delta$.
\end{majorILnc}

Consider a model $\IntA$ that makes a truth value assignment to every sentence letter of \GSL{}.  No sentence letter lacks an assignment, so it follows that $\IntA$ is a model for the set of all \GSL{} sentences.  The following definition characterizes such models as \emph{models of \GSL{}}.

\begin{majorILnc}{\LnpDC{Definition of Model for SL}}
	$\IntA$ is a \df{model of \GSL{}} \Iff $\IntA$ is a model of every sentence of \GSL{}.
\end{majorILnc}

Models can be uniform; for example, there is a model that assigns $\True$ to every sentence letter.
Or they can be given according to an arbitrary pattern, such as the model that alternately assigns $\True$ or $\False$ to a list of sentence letters.
A model can even assign truth values at random. 
\emph{Every} possible function from sentence letters to truth values is a model.

A model of $\CAPPHI$ only assigns truth values to the sentence letters of $\CAPPHI$.
It does not directly assign a truth value to any of the complex (i.e. non-atomic) sentences of \GSL{}.
The truth value of a complex sentence $\CAPPHI$ depends upon two things: (1) the main connective of $\CAPPHI$, and (2) the truth values of the proper subsentences of $\CAPPHI$.
Each logical connective is associated with a truth function.
While the truth assignments to the sentence letters vary by model, the truth functions of the connectives do not.\footnote{We discuss \emph{truth functions} further in section \ref{Truth Functions Truth Tables and Boolean Operators}.}

\begin{majorILnc}{\LnpDC{True on a GSL interpretation}} The following clauses define whether an \GSL{} sentence $\CAPTHETA$ is \nidf{$\True$} or \nidf{$\False$} on a model $\IntA$ for $\CAPTHETA$. The relevant clause is determined by which main connective $\CAPTHETA$ has, if any:
\begin{cenumerate}
\item $\CAPTHETA$ is a sentence letter. $\CAPTHETA$ is $\True$ on $\IntA$ \Iff $\IntA$ assigns $\True$ to it, i.e. $\IntA(\CAPTHETA)=\TrueB$.
\item $\CAPTHETA$ is of the form $\negation{\CAPPHI}$. $\CAPTHETA$ is $\True$ on $\IntA$ \Iff $\CAPPHI$ is $\False$ on $\IntA$.
\item\label{GSL true conjunction} $\CAPTHETA$ is of the form $\parconjunction{\CAPPHI_1}{\conjunction{\CAPPHI_2}{\conjunction{\ldots}{\CAPPHI_{\integer{n}}}}}$.   $\CAPTHETA$ is $\True$ on $\IntA$ \Iff each of the conjuncts $\CAPPHI_1, \CAPPHI_2, \ldots, \CAPPHI_n$ is $\True$ on $\IntA$.
\item $\CAPTHETA$ is of the form $\pardisjunction{\CAPPHI_1}{\disjunction{\CAPPHI_2}{\disjunction{\ldots}{\CAPPHI_{\integer{n}}}}}$. $\CAPTHETA$ is $\True$ on $\IntA$ \Iff at least one of the disjuncts $\CAPPHI_1, \CAPPHI_2, \ldots, \CAPPHI_n$ is $\True$ on $\IntA$.
\item\label{GSL true horseshoe} $\CAPTHETA$ is of the form $\horseshoe{\CAPPHI}{\CAPPSI}$. $\CAPTHETA$ is $\True$ on $\IntA$ \Iff the \CAPS{lhs} $\CAPPHI$ is $\False$ or the \CAPS{rhs} $\CAPPSI$ is $\True$ on $\IntA$ (or both).
\item $\CAPTHETA$ is of the form $\triplebar{\CAPPHI}{\CAPPSI}$. $\CAPTHETA$ is $\True$ on $\IntA$ \Iff $\CAPPHI$ and $\CAPPSI$ have the same truth value on $\IntA$.
\item A sentence is $\False$ on $\IntA$ \Iff it's not $\True$ on $\IntA$.
\end{cenumerate}
\end{majorILnc}

For every model $\IntA$ for sentence $\CAPPHI$ this definition (i.e. the definition of truth) fixes a unique truth value for $\CAPPHI$.\footnote{If a model $\IntA$ is \emph{not} a model for some \GSL{} sentence $\CAPPHI$ then $\IntA$ does \emph{not} fix a truth value for $\CAPPHI$.}

Although there are an infinite number of \GSL{} sentences letters to which a model can assign truth values, the only ones that matter for any given sentence are, unsurprisingly, the sentence letters that the sentence contains.\footnote{
	We prove this later in the chapter.
}
For example, when assessing the value of $\horseshoe{\Al}{\Bl}$ on $\IntA$ only the assignments to $\Al$ and $\Bl$ are relevant; assignments to other letters are irrelevant. It follows that if two models $\IntA_1$ and $\IntA_2$ assign the same truth values to $\Al$ and $\Bl$, respectively, then they fix the same truth value for $\horseshoe{\Al}{\Bl}$. 

If $\IntA$ makes assignments to $\Al$ and $\Bl$ but no other sentence letters, we say that $\IntA$ is a \emph{minimal model} of $\horseshoe{\Al}{\Bl}$. More generally:

\begin{majorILnc}{\LnpDC{Definition of Minimal SL Model}}
	$\IntA$ is a \df{minimal model of $\CAPPHI$} \Iff $\IntA$ makes assignments to every sentence letter in $\CAPPHI$ but to no other sentence letters.
\end{majorILnc}

\noindent{}There are only four minimal models for the sentence $\horseshoe{\Al}{\Bl}$, because there are only $4$ combinations of truth values that can be assigned to $\Al$ and $\Bl$. The number of distinct minimal models for a sentence $\CAPPHI$ is $2^n$, where $n$ is the number of sentence letters (counted by type) in $\CAPPHI$.

There are several ways to compute the truth value of an \GSL{} sentence in a model.
We demonstrate some informal ones in the following examples, and then develop a systematic method in section \mvref{Proceduresfortesting}. 

\begin{majorILnc}{\LnpEC{GSLTVExampleA}}
	Give the truth value of $\disjunction{\Al}{\negation{\Bl}}$ on a model $\IntA$ such that $\IntA(\Al)=\FalseB$ and $\IntA(\Bl)=\FalseB$.
	
	\begin{PROOF}	
		Given that $\IntA(\Bl)=\FalseB$, it follows by the negation clause of the definition of truth that $\negation{\Bl}$ is true on $\IntA$. 
		And since $\negation{\Bl}$ is true on $\IntA$, it follows by the disjunction clause of the definition of truth that $\disjunction{\Al}{\negation{\Bl}}$ is too.
	\end{PROOF}

	\begin{commentary}
		One way to compute the truth value of this sentence in $\IntA$ is to read off the values of the atomic subsentences and use definition \ref{True on a GSL interpretation} (the definition of truth in \GSL{}) to determine the value of successively larger subsentences, until finally you get the value of the whole sentence.
	\end{commentary}
		
\end{majorILnc}

\begin{majorILnc}{\LnpEC{GSLTVExampleB}}
	Give the truth value of $\conjunction{\cpardisjunction{\Al}{\parhorseshoe{\Dl}{\Bl}}}{\negation{\Gl}}$ on a model $\IntA$ such that $\IntA(\Al)=\FalseB$, $\IntA(\Dl)=\TrueB$, $\IntA(\Bl)=\TrueB$, and $\IntA(\Gl)=\TrueB$.
	
	\begin{PROOF}
		Since $\IntA(\Gl)=\TrueB$, it follows by the negation clause of the definition of truth that $\negation{\Gl}$ is $\False$ on $\IntA$. 
		So, by the conjunction clause of the definition of truth, $\conjunction{\cpardisjunction{\Al}{\parhorseshoe{\Dl}{\Bl}}}{\negation{\Gl}}$ is false too.
	\end{PROOF}

	\begin{commentary}
		When computing the truth value for a complicated sentence it can help to be strategic about which subsentences to look at first. Often you don't need to determine the value of every subsentence. The main connective can be an important clue about where to start. The sentence $\conjunction{\cpardisjunction{\Al}{\parhorseshoe{\Dl}{\Bl}}}{\negation{\Gl}}$ is a conjunction, which is false on a model if any one of its conjuncts is.
	\end{commentary}
	
	We can use a construction tree to make sure we compute the truth values of the subsentences in an appropriate order.
	The idea is to start at the top of the construction tree, the truth values of which are given by the model, and work our way down the branches. 
	Let's illustrate with $\conjunction{\cpardisjunction{\Al}{\parhorseshoe{\Dl}{\Bl}}}{\negation{\Gl}}$ on $\IntA$:
	\begin{center}
		\begin{tikzpicture}[grow=up]
		\tikzset{level distance=40pt}
		\tikzset{level 1/.style={level distance=60pt}}
		\tikzset{sibling distance=32pt}
		\tikzset{every tree node/.style={align=center,anchor=north}}
		\Tree%http://angasm.org/papers/qtree/    http://www.ling.upenn.edu/advice/latex/qtree/qtreenotes.pdf
		[.{$\conjunction{\cpardisjunction{\Al}{\parhorseshoe{\Dl}{\Bl}}}{\negation{\Gl}}$ \textbf{[$\FalseB$]}}
		[.{$\negation{\Gl}$ \textbf{[$\FalseB$]}} %!{\qsetw{3in}}
		[.{$\Gl$ \textbf{[$\TrueB$]}}
		]
		]
		%[.{$\Gl$ /$\TrueB$\\ $\negation{\Gl}$ /$\FalseB$} %!{\qsetw{3in}}
		%%   [.{$\Gl$ /$\TrueB$}
		%%   ]
		%]
		[.{$\disjunction{\Al}{\parhorseshoe{\Dl}{\Bl}}$ \textbf{[$\TrueB$]}} 
		[.{$\horseshoe{\Dl}{\Bl}$ \textbf{[$\TrueB$]}} 
		[.{$\Bl$ \textbf{[$\TrueB$]}} 
		]
		[.{$\Dl$ \textbf{[$\TrueB$]}} %!{\qsetw{2in}}
		] 
		]
		[.{$\Al$ \textbf{[$\FalseB$]}}
		]    
		]
		]%
		%\caption{Example formula tree}
		%\label{fig:ExampleFormulaTree}
		\end{tikzpicture}
	\end{center}
\end{majorILnc}
\begin{majorILnc}{\LnpEC{GSLTVExampleC}}
	Compute the truth value of the sentence $((\Cl \WEDGE \Dl) \HORSESHOE \Al)\TRIPLEBAR (\Dl \VEE \Hl)$ on a model $\IntA$ such that $\IntA(\Cl)=\TrueB$, $\IntA(\Dl)=\TrueB$, $\IntA(\Al)=\FalseB$, and $\IntA(\Hl)=\TrueB$.

	\begin{center}
		\begin{tikzpicture}[grow=up]
		\tikzset{level distance=42pt}
		\tikzset{sibling distance=32pt}
		\Tree%http://angasm.org/papers/qtree/    http://www.ling.upenn.edu/advice/latex/qtree/qtreenotes.pdf
		[.{$((\Cl \WEDGE \Dl) \HORSESHOE \Al)\TRIPLEBAR (\Dl \VEE \Hl)$ \textbf{[$\FalseB$]}}
		[.{$\Dl\VEE \Hl$ \textbf{[$\TrueB$]}} %!{\qsetw{3in}}
		[.{$\Hl$ \textbf{[$\TrueB$]}} %!{\qsetw{2in}}
		]  
		[.{$\Dl$ \textbf{[$\TrueB$]}}
		]
		]
		[.{$(\Cl\WEDGE \Dl)\HORSESHOE \Al$ \textbf{[$\FalseB$]}} 
		[.{$\Al$ \textbf{[$\FalseB$]}} %!{\qsetw{2in}}
		]   
		[.{$\Cl\WEDGE \Dl$ \textbf{[$\TrueB$]}}
		[.{$\Dl$ \textbf{[$\TrueB$]}} %!{\qsetw{2in}}
		]     
		[.{$\Cl$ \textbf{[$\TrueB$]}} 
		]
		]
		]
		]%
		%\caption{Example formula tree}
		%\label{fig:ExampleFormulaTree}
		\end{tikzpicture}
	\end{center}
\end{majorILnc}
%\begin{majorILnc}{\LnpEC{GSLTVExampleC}}
%
%\end{majorILnc}
 
\subsection{Truth Functions and Truth Tables}\label{Truth Functions Truth Tables and Boolean Operators}
A truth function is any function $f:\{\TrueB,\FalseB\}\times\ldots\times\{\TrueB,\FalseB\}\Rightarrow\{\TrueB,\FalseB\}$, i.e., from sequences of truth values to truth values.  The definition of truth in a model (\ref{True on a GSL interpretation}) associates each logical connective of \GSL{} with a truth function. 
For example, the truth function for $\horseshoe{}{}$ is:
\begin{center} 
	$f(\TrueB,\TrueB)=\TrueB$ \\
	$f(\TrueB,\FalseB)=\FalseB$ \\
	$f(\FalseB,\TrueB)=\TrueB$ \\
	$f(\FalseB,\FalseB)=\TrueB$ \\
\end{center}
Or, for short:
\begin{center} 
	$f(v_1,v_2)=
	\begin{cases}
	\FalseB{} & \text{ if } v_1=\TrueB\text{ and }v_2=\FalseB \\
	\TrueB{} & \text{ otherwise}
	\end{cases}$
\end{center}
A truth function can also be given as a \emph{truth table}.
The truth table for $\horseshoe{}{}$ is:
\begin{center}
	\begin{tabular}[t]{c | c c}
		$\HORSESHOE$ & $\TrueB$ & $\FalseB$ \\
		\hline
		& & \\[-.25cm]
		$\TrueB$ & $\TrueB$ & $\FalseB$ \\
		$\FalseB$ & $\TrueB$ & $\TrueB$  
	\end{tabular}
\end{center}
or can be written alternatively as:
\begin{center}
	\begin{tabular}[t]{c c c}
		$\CAPPHI$ & $\CAPPSI$ & $\horseshoe{\CAPPHI}{\CAPPSI}$ \\
		\hline 
		& & \\[-.25cm]
		$\TrueB$ & $\TrueB$ & $\TrueB$ \\
		$\TrueB$ & $\FalseB$ & $\FalseB$ \\
		$\FalseB$ & $\TrueB$ & $\TrueB$ \\
		$\FalseB$ & $\FalseB$ & $\TrueB$ \\
	\end{tabular}
\end{center}

\subsection{Logical Truth: TFT, TFF, \& TFC}\label{TFT TFF TFI}

A randomly chosen sentence will probably be true on some models and false on others. 
However, some sentences are true on all models. 
One example of such is the sentence $\disjunction{\Al}{\negation{\Al}}$.
Others are false on all models, e.g. $\conjunction{\Al}{\negation{\Al}}$.
\begin{majorILnc}{\LnpDC{GSL TFT}}
A sentence $\CAPPHI$ of \GSL{} is \nidf{truth functionally true}\index{truth!truth functional|textbf} (\CAPS{tft})\index{TFT|see{truth, truth functional}} \Iff it is $\True$ on all models for $\CAPPHI$.
\end{majorILnc}

\begin{majorILnc}{\LnpEC{TFTExampleA}}
	Prove that $\disjunction{\Al}{\negation{\Al}}$ is \CAPS{tft}.
	\begin{PROOF}
		A model $\IntA$ for $\disjunction{\Al}{\negation{\Al}}$ has to assign either $\TrueB$ or $\FalseB$ to $\Al$. 
		If it assigns $\TrueB$ to $\Al$, then $\disjunction{\Al}{\negation{\Al}}$ is true on $\IntA$.
		Otherwise it assigns $\FalseB$ to $\Al$, in which case $\negation{\Al}$ is true on $\IntA$.
		It follows that $\disjunction{\Al}{\negation{\Al}}$ is true on $\IntA$.
		Either way, the sentence is true on $\IntA$.
		This holds in all models $\IntA$, so $\disjunction{\Al}{\negation{\Al}}$ is \CAPS{tft}.		
	\end{PROOF}
\end{majorILnc}

\begin{majorILnc}{\LnpEC{TFTExampleB}}
	Prove that $\horseshoe{\Bl}{\parhorseshoe{\Cl}{\Bl}}$ is \CAPS{tft}.
	\begin{PROOF}
		Any model $\IntA$ will assign either $\TrueB$ or $\FalseB$ to $\Bl$. 
		If it assigns $\FalseB$ to $\Bl$, then $\horseshoe{\Bl}{\parhorseshoe{\Cl}{\Bl}}$ is true on $\IntA$, because, according to the def. of truth for $\HORSESHOE$, a conditional is true if the \CAPS{lhs} is false.
		If $\IntA$ assigns $\TrueB$ to $\Bl$, then $\horseshoe{\Cl}{\Bl}$ is true on $\IntA$, again because of the def. of truth for $\HORSESHOE$. 
		But if $\horseshoe{\Cl}{\Bl}$ is true on $\IntA$, it follows that $\horseshoe{\Bl}{\parhorseshoe{\Cl}{\Bl}}$ is true on $\IntA$. 
	\end{PROOF}
\end{majorILnc}%

\begin{majorILnc}{\LnpDC{GSL TFF}}
A sentence $\CAPPHI$ of \GSL{} is \nidf{truth functionally false}\index{falsehood!truth functional|textbf} (\CAPS{tff})\index{TFF|see{falsehood, truth functional}} \Iff it is $\False$ on all models for $\CAPPHI$.
\end{majorILnc}

\begin{majorILnc}{\LnpEC{TFTExampleC}}
	Prove that $\conjunction{\Al}{\negation{\Al}}$ is \CAPS{tff}. 
	\begin{PROOF}
		A model $\IntA$ for $\conjunction{\Al}{\negation{\Al}}$ has to assign either $\TrueB$ or $\FalseB$ to $\Al$. 
		If it assigns $\TrueB$ to $\Al$, then $\negation{\Al}$ is false on $\IntA$.
		So $\conjunction{\Al}{\negation{\Al}}$ is false on $\IntA$.
		But if $\IntA$ assigns $\FalseB$ to $\Al$, then $\conjunction{\Al}{\negation{\Al}}$ is false on $\IntA$. 
		Either way, the sentence is false on $\IntA$.
		This holds in all models $\IntA$, so $\conjunction{\Al}{\negation{\Al}}$ is \CAPS{tff}.
	\end{PROOF}
\end{majorILnc}

\begin{majorILnc}{\LnpEC{TFTExampleD}}
	Prove that $\conjunction{\negation{\Cl}}{\bparconjunction{\parhorseshoe{\Bl}{\Cl}}{\Bl}}$ is \CAPS{tff}.
	\begin{PROOF}
		Assume that the sentence \emph{isn't} \CAPS{tff}.
		Then there is some model $\IntA$ that makes it true.
		By def. of truth for $\WEDGE$, both $\negation{\Cl}$ and $\conjunction{\parhorseshoe{\Bl}{\Cl}}{\Bl}$ are true on $\IntA$.
		Since $\negation{\Cl}$ is true on $\IntA$, $\Cl$ is false on $\IntA$.
		Since $\conjunction{\parhorseshoe{\Bl}{\Cl}}{\Bl}$ is true on $\IntA$, then both $\horseshoe{\Bl}{\Cl}$ and $\Bl$ are true on $\IntA$.
		Thus $\Cl$ is true on $\IntA$ too.
		But $\IntA$ can't assign both $\FalseB$ and $\TrueB$ to $\Cl$.
		So there cannot be any model $\IntA$ that makes $\conjunction{\negation{\Cl}}{\bparconjunction{\parhorseshoe{\Bl}{\Cl}}{\Bl}}$ true.
		Therefore it's \CAPS{TFF}.
	\end{PROOF}

	\begin{commentary}
		This is an \emph{indirect proof}.
		To use an indirect proof, start by assuming the opposite of what you want to prove.
		Then show how that assumption leads to a contradiction.
		No contradiction can be true.
		So, if an assumption leads to a contradiction then it must be false.
		The opposite of the assumption must therefore be true.
		\commentaryspace
		Another name for indirect proof is \mention{\emph{reductio ad absurdum}} (sometimes just \mention{\emph{reductio}} or \mention{RAA}).  
		For many problems and theorems, RAA is the easiest method to use.
		It is called \mention{\emph{reductio ad absurdum}} because it `reduces' the initial assumption to a contradiction, an absurdity.
	\end{commentary}
\end{majorILnc}

\begin{majorILnc}{\LnpDC{GSL TFI}}
A sentence $\CAPPHI$ of \GSL{} is \nidf{truth functionally contingent}\index{indeterminate!truth functional|textbf} (\CAPS{tfc})\index{TFI|see{indeterminate, truth functional}} \Iff it is $\True$ on one model for $\CAPPHI$ and $\False$ on another. 
\end{majorILnc}

\begin{majorILnc}{\LnpEC{TFTExampleE}}
In example \mvref{GSLTVExampleA} we saw that the sentence $\conjunction{\cpardisjunction{\Al}{\parhorseshoe{\Dl}{\Bl}}}{\negation{\Gl}}$ is false on one model.
To see that it can also be true, and thus that the sentence is \CAPS{tfc}, consider the following model: $\IntA(\Al)=\FalseB$, $\IntA(\Dl)=\TrueB$, $\IntA(\Bl)=\TrueB$, and $\IntA(\Gl)=\TrueB$.
Convince yourself that $\IntA$ makes $\conjunction{\cpardisjunction{\Al}{\parhorseshoe{\Dl}{\Bl}}}{\negation{\Gl}}$ true. 
\end{majorILnc}

\noindent{}Every sentence of \GSL{} is either \CAPS{tft}, \CAPS{tff}, or \CAPS{tfc}.

Although the definitions of \CAPS{tft}, \CAPS{tff}, and \CAPS{tfc} are specific to \GSL{}---the models to which each definition refers are models of \GSL{}---we can use essentially the same definitions for \emph{any} formal language, as long as there is some notion of a model for that language. 
We can think of sentences which fit these definitions as being (respectively) \niidf{logically true}\index{logical!truth}\index{truth!logical}, \niidf{logically false}\index{logical!falsehood}\index{falsehood!logical}, and \niidf{logically contingent}\index{logical!indeterminate}\index{indeterminate!logical}. Hereafter we'll sometimes use the more general term \mention{logical truth} instead of \mention{truth functional truth}.\footnote{
	A logical truth is sometimes said to be \niidf{valid}, or said to be a \niidf{tautology}.\index{sentence!valid|see{truth, logical}}\index{tautology|see{truth, logical}} 
	We avoid using these terms for logical truths.
}

\subsection{Procedures for Testing TFT, TFF, \& TFC}\label{Proceduresfortesting}

In the above examples (\ref{TFTExampleA}--\ref{TFTExampleD}) we used the definition for truth (definition \mvref{True on a GSL interpretation}) to show whether a given \GSL{} sentence was \CAPS{tft}, \CAPS{tff}, or \CAPS{tfc}. 
But there are more systematic methods for classifying \GSL{} sentences. 

Perhaps the most well known method involves using truth tables.%
\footnote{%
	Peirce \citeyearpar{Peirce1902} was the first to use truth tables. See Hodges \citeyearpar[5]{Hodges2001}.
} 
Later in this chapter we prove theorem \mvref{thm:localityoftruth}, which says that the truth of a given \GSL{} sentence $\CAPPHI$ can be determined by a \emph{minimal model}.
Remember that a minimal model for $\CAPPHI$ assigns truth values only to the sentence letters appearing in $\CAPPHI$.
One way to test whether $\CAPPHI$ is \CAPS{tft}, \CAPS{tff}, or \CAPS{tfc} is to write down all the possible assignments of truth values to the sentence letters of $\CAPPHI$, then compute the truth value of $\CAPPHI$ for each assignment. 
The number of possible assignments (minimal models) is finite. 
For a sentence $\CAPPHI$ with $\integer{n}$ sentence letters (counted by type) there are $2^{\integer{n}}$ possible assignments. 
If $\CAPPHI$ is true in all assignments, then $\CAPPHI$ is \CAPS{tft}.  
If it's false in all of them, then $\CAPPHI$ is \CAPS{tff}. 
And if it is true in some assignments and false in others, then $\CAPPHI$ is \CAPS{tfc}. 

Consider the sentence $\conjunction{\cpardisjunction{\Al}{\parhorseshoe{\Dl}{\Bl}}}{\negation{\Gl}}$.
Because this sentence has 4 sentence letters its table has $2^4=16$ rows. 
First write the 4 sentence letters on a top row.
Then fill out the assignments by starting at the far right sentence letter ($\Gl$) and putting alternating $\TrueB$'s and $\FalseB$'s below it until all 16 rows are filled. (See table \ref{truthtableexample}.) 
\begin{table}[!ht]
\begin{center}
\begin{tabular}{ c c c c c}
$\Al$ & $\Bl$ & $\Dl$ & $\Gl$ & $\parconjunction{\cpardisjunction{\Al}{\parhorseshoe{\Dl}{\Bl}}}{\negation{\Gl}}$ \\
\hline
$ $ & $ $ & & & \\[-.25cm]
$\TrueB$ & $\TrueB$ & $\TrueB$ & $\TrueB$ & $\FalseB$ \\
$\TrueB$ & $\TrueB$ & $\TrueB$ & $\FalseB$& $\TrueB$ \\
$\TrueB$ & $\TrueB$ & $\FalseB$ & $\TrueB$ & $\FalseB$ \\
$\TrueB$ & $\TrueB$ & $\FalseB$ & $\FalseB$  & $\TrueB$ \\
$\TrueB$ &  $\FalseB$& $\TrueB$ & $\TrueB$	&$\FalseB$ \\
$\TrueB$ & $\FalseB$ & $\TrueB$ & $\FalseB$	& $\TrueB$  \\
$\TrueB$ &$\FalseB$  & $\FalseB$& $\TrueB$	&$\FalseB$ \\
$\TrueB$ & $\FalseB$ &$\FalseB$	& $\FalseB$	& $\TrueB$ \\
$\FalseB$	& $\TrueB$ & $\TrueB$ & $\TrueB$	& $\FalseB$ \\
$\FalseB$	& $\TrueB$ & $\TrueB$ & $\FalseB$	& $\TrueB$ \\
$\FalseB$	& $\TrueB$ & $\FalseB$&	$\TrueB$ &$\FalseB$ \\
$\FalseB$	& $\TrueB$ & $\FalseB$& $\FalseB$	& $\TrueB$ \\
$\FalseB$	& $\FalseB$	& $\TrueB$ & $\TrueB$	&$\FalseB$ \\
$\FalseB$	& $\FalseB$	& $\TrueB$ & $\FalseB$	& $\FalseB$ \\
$\FalseB$	& $\FalseB$	& $\FalseB$& $\TrueB$	& $\FalseB$ \\
$\FalseB$	& $\FalseB$& $\FalseB$& $\FalseB$	& $\TrueB$ \\
\end{tabular}
\end{center}
\caption{Sample Truth Table}
\label{truthtableexample}
\end{table}
Next, move to the second-to-the-right sentence letter ($\Dl$) and alternate by 2 $\TrueB$'s and 2 $\FalseB$'s for 16 rows. 
Move one more sentence letter to the left ($\Bl$) and alternate $\TrueB$'s and $\FalseB$'s 4 at a time. 
Finally, alternate by 8 at a time for the left-most sentence letter ($\Al$).
This completes the 16 rows so that each row is a unique assignment of truth values to the sentence letters.
All 16 possible assignments are guaranteed to to be present.
In general the pattern is to start at the far right column alternating $\TrueB$ and $\FalseB$, then move to the left doubling the number of $\TrueB$'s and $\FalseB$'s that appeared in the previous column. 

Then we write the sentence to the right of the sentence letters and under it put, in the respective rows, its truth value for each assignment. 
Once the truth value of the sentence is computed for all rows it's a trivial matter to determine from the table whether the sentence is \CAPS{tft}, \CAPS{tff}, or \CAPS{tfc}. 
If $\TrueB$ is under the sentence in every row, then it's \CAPS{tft}. 
If $\FalseB$ is in every row, then it's \CAPS{tff}.
And if each of $\TrueB$ and $\FalseB$ appear in at least one row, then it's \CAPS{tfc}. 
Table \ref{truthtableexample} shows that $\parconjunction{\cpardisjunction{\Al}{\parhorseshoe{\Dl}{\Bl}}}{\negation{\Gl}}$ is \CAPS{tfc}.

The process of filling out a truth table is entirely \mention{mechanical.}
We can carry out the process by following purely formal rules.
Once we have a truth table for some \GSL{} sentence, we can again use purely formal rules to determine whether it's \CAPS{tft}, \CAPS{tff}, or \CAPS{tfc}.
No creativity is necessary to determine whether any given sentence is, e.g., \CAPS{tft}.

\begin{majorILnc}{\LnpEC{TFTExampleA2}}
We saw in example \mvref{TFTExampleA} that $\conjunction{\Al}{\negation{\Al}}$ is \CAPS{tff}. 
The following truth table confirms this. 
\begin{center}
\begin{tabular}{ c c }
$\Al$ & $\conjunction{\Al}{\negation{\Al}}$ \\
\hline
$ $ & $ $ \\[-.25cm]
$\TrueB$ & $\FalseB$ \\
$\FalseB$ & $\FalseB$ \\
\end{tabular}
\end{center}
\end{majorILnc}
\begin{majorILnc}{\LnpEC{TFTExampleD2}}
In example \mvref{TFTExampleD} we saw that $\horseshoe{\Bl}{\parhorseshoe{\Cl}{\Bl}}$ is \CAPS{tft}.
This is confirmed by the following truth table. 
\begin{center}
\begin{tabular}{ c c c }
$\Bl$ & $\Cl$ & $\horseshoe{\Bl}{\parhorseshoe{\Cl}{\Bl}}$ \\
\hline
$ $ & $ $ & $ $ $ $ \\[-.25cm]
$\TrueB$ & $\TrueB$ & $\TrueB$ \\
$\TrueB$ & $\FalseB$& $\TrueB$ \\
$\FalseB$ & $\TrueB$ & $\TrueB$ \\
$\FalseB$ & $\FalseB$  & $\TrueB$ \\
\end{tabular}
\end{center}
\end{majorILnc}
\begin{majorILnc}{\LnpEC{TFTExampleF2}}
The sentence $\conjunction{\negation{\Cl}}{\bparconjunction{\parhorseshoe{\Bl}{\Cl}}{\Bl}}$ is \CAPS{tff}.
The following truth table proves it.  
\begin{center}
\begin{tabular}{ c c c }
$\Bl$ & $\Cl$ & $\conjunction{\negation{\Cl}}{\bparconjunction{\parhorseshoe{\Bl}{\Cl}}{\Bl}}$ \\
\hline
$ $ & $ $ & \\[-.25cm]
$\TrueB$ & $\TrueB$ & $\FalseB$ \\
$\TrueB$ & $\FalseB$& $\FalseB$ \\
$\FalseB$ & $\TrueB$ & $\FalseB$ \\
$\FalseB$ & $\FalseB$  & $\FalseB$ \\
\end{tabular}
\end{center}
\end{majorILnc} 

While truth tables are a convenient method for answering many questions about \GSL{} sentences, two warnings are in order.
First, for complex sentences the size of the truth table can be very large.
The number of rows needed for a truth table grows exponentially with the number of sentence letters.
Second, while truth tables are useful in \GSL{}, there is nothing comparable for the more sophisticated formal languages we cover in later chapters.
The sooner you learn to analyze sentences directly, rather than relying on a truth table, the better.
For example, consider the sentence $\disjunction{\parhorseshoe{\El}{\parconjunction{\Bl}{\negation{\pardisjunction{\Cl}{\Dl}}}}}{\parhorseshoe{\Al}{\Al}}$.
You \emph{could} evaluate this sentence with a 32 line truth table.
But it's much easier to prove that $\parhorseshoe{\Al}{\Al}$ is \CAPS{tft}, and then to show that it follows that the whole sentence is \CAPS{tft}.

The various formal derivation systems used in logic provide another class of procedures for testing \GSL{} sentences for \CAPS{tft} and \CAPS{tff}.
In chapter \ref{Derivations} we develop one such system.\footnote{
	We use a natural deduction system\index{natural deduction}.
	Natural deduction systems do not provide a procedure for testing whether sentences are \CAPS{tfc}.
	Some derivation systems---e.g. semantic tableaux systems---do. 
	A comprehensive introductory treatment of semantic tableaux (called truth trees by the author) is given in Nicholas J.J. Smith's \citeyearpar{Smith2012} textbook \emph{Logic: The Laws of Truth}. 
	Smith also gives a more detailed and thorough introduction to truth tables. 
	Other derivation systems, such as Hilbert-style axiomatic systems, and Gentzen-style sequent calculi, don't on their own provide direct means of testing for \CAPS{tfc} sentences. 
}
We also develop an algorithm in chapter \ref{completenesschapter} that can be used with our derivation system as a testing procedure for \CAPS{tfc} sentences.

%%%%%%%%%%%%%%%%%%%%%%%%%%%%%%%%%%%%%%%%%%%%%%%%%%
\section{Entailment and other Relations}
%%%%%%%%%%%%%%%%%%%%%%%%%%%%%%%%%%%%%%%%%%%%%%%%%%

\subsection{Entailment}\label{Entailment}
Now we turn to the notion of entailment. 
We write that a set of sentences $\Delta$ entails a single sentence $\CAPTHETA$ as follows: \mention{\:$\Delta\sdtstile{}{}\CAPTHETA\:$}.
The symbol \mention{$\:\sdtstile{}{}\:$}, called the double turnstile,\index{double turnstile}\index{$\sdtstile{}{}$} is \emph{not} a symbol of \GSL{}. 
Like the Greek letters it is a symbol of MathEnglish.

\begin{majorILnc}{\LnpDC{GSL Generalized Further Entailment}}
	If $\Delta$ is a set of \GSL{} sentences and $\CAPTHETA$ is an \GSL{} sentence, then the following are equivalent ways to define when $\Delta$ entails $\CAPTHETA$:
	\begin{cenumerate}
		\item $\Delta\sdtstile{}{}\CAPTHETA$ \Iff every model for $\Delta$ and $\CAPTHETA$ that makes all sentences in $\Delta$ $\True$ also makes $\CAPTHETA$ $\True$.
		\item $\Delta\sdtstile{}{}\CAPTHETA$ \Iff every model for $\Delta$ and $\CAPTHETA$ either makes at least one sentence in $\Delta$ $\False$ or makes $\CAPTHETA$ $\True$.
	\end{cenumerate}
\end{majorILnc}

\noindent{}Each of the following is a consequence of the definition of entailment for the case in which $\Delta$ is of finite size $n$:

\begin{cenumerate}
		\item $\CAPPHI_1,\CAPPHI_2,\CAPPHI_3,\ldots,\CAPPHI_{\integer{n}}\sdtstile{}{}\CAPTHETA$ \Iff every model for $\CAPPHI_1$, $\CAPPHI_2$, $\CAPPHI_3$, $\ldots$, $\CAPPHI_{\integer{n}}$, and $\CAPTHETA$ that makes all of $\CAPPHI_1$, $\CAPPHI_2$, $\CAPPHI_3$, $\ldots$ $\CAPPHI_{\integer{n}}$ $\True$ also makes $\CAPTHETA$ $\True$.
		\item $\CAPPHI_1,\CAPPHI_2,\CAPPHI_3,\ldots,\CAPPHI_{\integer{n}}\sdtstile{}{}\CAPTHETA$ \Iff every model for $\CAPPHI_1$, $\CAPPHI_2$, $\CAPPHI_3$, $\ldots$, $\CAPPHI_{\integer{n}}$, and $\CAPTHETA$ either makes at least one of $\CAPPHI_1$, $\CAPPHI_2$, $\CAPPHI_3$, $\ldots$ $\CAPPHI_{\integer{n}}$ $\False$ or makes $\CAPTHETA$ $\True$.
\end{cenumerate}

\noindent{}Each of the following is another consequence, for the case in which $\Delta$ contains just one sentence:

\begin{cenumerate}
\item A sentence $\CAPPHI$ {entails} another sentence $\CAPTHETA$ \Iff every model for $\CAPPHI$ and $\CAPTHETA$ that makes $\CAPPHI$ $\True$ also makes $\CAPTHETA$ $\True$.
\item A sentence $\CAPPHI$ {entails} another sentence $\CAPTHETA$ \Iff there is no model for $\CAPPHI$ and $\CAPTHETA$ that makes $\CAPPHI$ $\True$ and $\CAPTHETA$ $\False$.
\end{cenumerate}

\noindent{}Finally, $\Delta$ can be the empty set or an infinite set.
When $\Delta$ is empty and $\Delta\:\sdtstile{}{}\CAPTHETA$ we can write: $\:\sdtstile{}{}\CAPTHETA$.

\begin{THEOREM}{\LnpTC{entailmentTFT theorem}}
	For all \GSL{} sentences $\CAPTHETA$, $\:\sdtstile{}{}\CAPTHETA$ \Iff $\CAPTHETA$ is \CAPS{tft}.
\end{THEOREM} 
\begin{PROOF}
	($\Rightarrow$) Assume that $\:\sdtstile{}{}\CAPTHETA$.
	Then, by the definition of $\sdtstile{}{}$, every model for $\CAPTHETA$ either makes a sentence to the left of the turnstile $\False$ or makes $\CAPTHETA$ true. 
	But there are no sentences to the left of the turnstile.
	So, every model for $\CAPTHETA$ makes $\CAPTHETA$ true.
	Thus $\CAPTHETA$ is \CAPS{tft}.

	($\Leftarrow$) Assume that $\CAPTHETA$ is \CAPS{tft}.
	Then there is no model that makes $\CAPTHETA$ false.
	Let $\Delta$ be the empty set.
	Then there is no model that makes all sentences in $\Delta$ true and $\CAPTHETA$ false.
	So, by the definition of $\sdtstile{}{}$, $\Delta\:\sdtstile{}{}\CAPTHETA$.
	And since $\Delta$ is empty, $\:\sdtstile{}{}\CAPTHETA$.

	Therefore $\:\sdtstile{}{}\CAPTHETA$ \Iff $\CAPTHETA$ is \CAPS{tft}.
\end{PROOF}
\begin{commentary}
	The statement of this theorem is a biconditional, i.e. a sentence of the form $\Al$ iff $\Bl$.
	The typical method for proving biconditionals is to divide the proof into two parts.
	In the first part, indicated by \mention{($\Rightarrow$)}, assume $\Al$ and show that $\Bl$ follows.
	In the second part, indicated by \mention{($\Leftarrow$)}, assume $\Bl$ and show that $\Al$ follows.
	After both parts are demonstrated, then conclude that $\Al$ iff $\Bl$.
\end{commentary}

\begin{majorILnc}{\LnpEC{GSLEntailmentExA}}
Prove $\parconjunction{\Al}{\Bl}\sdtstile{}{}\Bl$. 
\end{majorILnc}
\begin{PROOF}
Let $\IntA$ be a model that makes $\parconjunction{\Al}{\Bl}$ true. 
By the definition of truth for $\WEDGE$---clause 3, \mvref{True on a GSL interpretation}---both $\Al$ and $\Bl$ are true in $\IntA$ as well. 
Thus any model that makes $\parconjunction{\Al}{\Bl}$ true also makes $\Bl$ true.
Therefore, by the definition of $\sdtstile{}{}$, $\parconjunction{\Al}{\Bl}\sdtstile{}{}\Bl$.
\end{PROOF}
\begin{commentary}
	One method for proving an entailment is to assume a model that makes all sentences to the left of the turnstile true, and then to show that the sentence on the right must therefore also be true.
\end{commentary}

\begin{majorILnc}{\LnpEC{GSLEntailmentExB}}
Show that $\Bl\sdtstile{}{}\pardisjunction{\Al}{\Bl}$. We leave this as an exercise the reader.
\end{majorILnc}
\begin{majorILnc}{\LnpEC{GSLEntailmentExC}}
Show that for all $\CAPPHI$ there is some $\CAPTHETA$ such that $\CAPPHI\sdtstile{}{}\CAPTHETA$. That is, that every \GSL{} sentence entails at least one \GSL{} sentence.
\end{majorILnc}
\begin{PROOF}
	Let $\CAPPHI$ be any \GSL{} sentence.
	Then let $\CAPTHETA=\CAPPHI$.
	Then there is no model that makes $\CAPPHI$ true and $\CAPTHETA$ false, since they are the same sentence.
	Therefore, by the definition of $\sdtstile{}{}$, $\CAPPHI\sdtstile{}{}\CAPTHETA$.
	Nothing particular was assumed about $\CAPPHI$, so the argument holds for all \GSL{} sentences.
\end{PROOF}
\begin{commentary}
	We must show that no matter what \GSL{} sentence $\CAPPHI$ you pick, there's always some \GSL{} sentence $\CAPTHETA$ entailed by it.
	But this is easy: every \GSL{} sentence entails itself.
	Thus, no matter what \GSL{} sentence you pick there's always at least one sentence entailed by it.
\end{commentary}

\begin{majorILnc}{\LnpEC{GSLEntailmentExD}}
Show that there is some $\CAPTHETA$ such that for all $\CAPPHI$, $\CAPPHI\sdtstile{}{}\CAPTHETA$. That is, show there's some \GSL{} sentence that's entailed by all \GSL{} sentences.
\end{majorILnc}
\begin{PROOF}
	Let $\CAPTHETA$ be $\pardisjunction{\Al}{\negation{\Al}}$. 
	It was shown previously that $\pardisjunction{\Al}{\negation{\Al}}$ is \CAPS{tft}.
	Then, by the definition of \CAPS{tft}, $\CAPTHETA$ is true on all models.
	So there is no model such that $\CAPPHI$ is true and $\CAPTHETA$ is false.
	Thus, by the definition of $\sdtstile{}{}$, $\CAPPHI\sdtstile{}{}\CAPTHETA$.
	This argument holds for all $\CAPPHI$.
\end{PROOF}

\begin{majorILnc}{\LnpEC{GSLEntailmentExE}}
Show that there is some $\CAPPHI$ such that, for all $\CAPTHETA$, $\CAPPHI\sdtstile{}{}\CAPTHETA$. We leave the proof as an exercise for the reader.
\end{majorILnc}

\subsection{Procedures for Testing Entailment}\label{TestingEntailment}

In the previous examples (\ref{GSLEntailmentExA}--\ref{GSLEntailmentExE}) we used the definition of entailment to prove entailment claims. 
But there are other methods for checking whether a given entailment holds.
Truth tables provide a simple, mechanical procedure.

To test whether a set of sentences $\Delta$ entails a sentence $\CAPTHETA$, we first write down all the sentence letters of $\CAPTHETA$ and $\Delta$. 
We put these on the left side of the truth table and under them (following the procedure outlined in section \pmvref{Proceduresfortesting}) we write all the possible assignments of truth values. 
Then, to the right of the sentence letters, we write each of the sentences from $\Delta$ (in separate columns), and to the right of those we write $\CAPTHETA$. 
Under $\CAPTHETA$ and each of the sentences from $\Delta$ we write the truth value of that sentence based on the assignment of truth values listed in that row. 
$\Delta$ entails $\CAPTHETA$ iff there is no row in the truth table in which all the sentences in $\Delta$ are true and $\CAPTHETA$ is false.
Any rows of the truth table that do not make all of the \CAPS{lhs} sentences true are irrelevant.
After marking one of the \CAPS{lhs} sentences as false we can omit the rest of the row.
\begin{majorILnc}{\LnpEC{GSLEntailmentExA2}}
In example \ref{GSLEntailmentExA} we saw that $\parconjunction{\Al}{\Bl}\sdtstile{}{}\Bl$. 
We show this with a truth table.  
\begin{center}
\begin{tabular}{ c c c c }
$\Al$ & $\Bl$ & $\parconjunction{\Al}{\Bl}$ & $\Bl$ \\
\hline
$ $ & $ $ & & \\[-.25cm]
$\TrueB$ & $\TrueB$ & $\TrueB$ & $\TrueB$ \\
$\TrueB$ & $\FalseB$& $\FalseB$ & $\FalseB$ \\
$\FalseB$ & $\TrueB$ & $\FalseB$ & $\TrueB$ \\
$\FalseB$ & $\FalseB$  & $\FalseB$ & $\FalseB$ \\
\end{tabular}
\end{center}
\end{majorILnc}
\begin{majorILnc}{\LnpEC{GSLEntailmentExB2}}
In example \ref{GSLEntailmentExB} we said that $\Bl\sdtstile{}{}\pardisjunction{\Al}{\Bl}$. 
The following truth table shows this. 
\begin{center}
\begin{tabular}{ c c c c }
$\Al$ & $\Bl$ & $\Bl$ & $\pardisjunction{\Al}{\Bl}$ \\
\hline
$ $ & $ $ & & \\[-.25cm]
$\TrueB$ & $\TrueB$ & $\TrueB$ & $\TrueB$ \\
$\TrueB$ & $\FalseB$& $\FalseB$ & $\TrueB$ \\
$\FalseB$ & $\TrueB$ & $\TrueB$ & $\TrueB$ \\
$\FalseB$ & $\FalseB$  & $\FalseB$ & $\FalseB$ \\
\end{tabular}
\end{center}
\end{majorILnc}
\begin{majorILnc}{\LnpEC{GSLEntailmentExAA}}
We conclude with a more complicated example. 
Here we show that $\disjunction{\Al}{\Bl},\horseshoe{\negation{\Cl}}{\negation{\Al}},\horseshoe{\Bl}{\Cl}\sdtstile{}{}\Cl$.
\begin{center}
\begin{tabular}{ c c c c c c c }
$\Al$ & $\Bl$ & $\Cl$ & $\disjunction{\Al}{\Bl}$ & $\horseshoe{\negation{\Cl}}{\negation{\Al}}$ & $\horseshoe{\Bl}{\Cl}$ & $\Cl$ \\
\hline
$ $ & $ $ & & & & & \\[-.25cm]
$\TrueB$ & $\TrueB$ & $\TrueB$ & $\TrueB$ & $\TrueB$ & $\TrueB$ & $\TrueB$\\
$\TrueB$ & $\TrueB$ & $\FalseB$& $\TrueB$ & $\FalseB$ & $\FalseB$ &  $\FalseB$\\
$\TrueB$ & $\FalseB$ & $\TrueB$ & $\TrueB$ & $\TrueB$ & $\TrueB$ &  $\TrueB$\\
$\TrueB$ & $\FalseB$ & $\FalseB$  & $\TrueB$ & $\FalseB$ & $\TrueB$ & $\FalseB$\\
$\FalseB$ & $\TrueB$ & $\TrueB$ & $\TrueB$ & $\TrueB$ & $\TrueB$ & $\TrueB$\\
$\FalseB$ & $\TrueB$ & $\FalseB$& $\TrueB$ & $\TrueB$ & $\FalseB$ & $\FalseB$\\
$\FalseB$ & $\FalseB$ & $\TrueB$ & $\FalseB$ & $\TrueB$ & $\TrueB$ & $\TrueB$\\
$\FalseB$ & $\FalseB$ & $\FalseB$  & $\FalseB$ & $\TrueB$ & $\TrueB$ & $\FalseB$\\
\end{tabular}
\end{center}
\end{majorILnc}

\subsection{Basic Results on Entailment}\label{Basic Results on Entailment} 
A simple, but important theorem involves moving sentences from one side of the turnstile to the other.
\begin{THEOREM}{\LnpTC{Exponentiation of Entailment} \GSL{} Exportation Theorem:} For all \GSL{} sentences $\CAPPHI$ and $\CAPTHETA$, $\CAPPHI\sdtstile{}{}\CAPTHETA$ \Iff $\:\sdtstile{}{}\horseshoe{\CAPPHI}{\CAPTHETA}$.
\end{THEOREM}
\begin{commentary}
	The statement of this theorem is a biconditional, so we prove it as we proved the last biconditional theorem.
	There are two parts.
	First, we assume that the left-hand side, $\CAPPHI\sdtstile{}{}\CAPTHETA$, is true, and show that $\:\sdtstile{}{}\horseshoe{\CAPPHI}{\CAPTHETA}$ follows from it. 
	Second, we assume that $\:\sdtstile{}{}\horseshoe{\CAPPHI}{\CAPTHETA}$ is true, and show that $\CAPPHI\sdtstile{}{}\CAPTHETA$ follows.
	That is sufficient to demonstrate that $\CAPPHI\sdtstile{}{}\CAPTHETA$ \Iff $\:\sdtstile{}{}\horseshoe{\CAPPHI}{\CAPTHETA}$.
\end{commentary}
\begin{PROOF}
$(\Rightarrow)$ Assume that $\CAPPHI\sdtstile{}{}\CAPTHETA$.
Then, by the definition of $\:\sdtstile{}{}$ it follows that on every model on which $\CAPPHI$ is true, $\CAPTHETA$ is also true.
According to the definition of truth for $\HORSESHOE$, $\IntA$ makes $\parhorseshoe{\CAPPHI}{\CAPTHETA}$ false only if it makes $\CAPPHI$ true and $\CAPTHETA$ false.
But we have already shown that there is no such model.
It follows that $\parhorseshoe{\CAPPHI}{\CAPTHETA}$ is \CAPS{tft} (def. of \CAPS{tft}, \pmvref{GSL TFT}).
And according to theorem \mvref{entailmentTFT theorem}, it follows that the entailment $\:\sdtstile{}{}\parhorseshoe{\CAPPHI}{\CAPTHETA}$ holds.

$(\Leftarrow)$ Assume that $\:\sdtstile{}{}\parhorseshoe{\CAPPHI}{\CAPTHETA}$.
By the definition of $\:\sdtstile{}{}$, since $\:\sdtstile{}{}\parhorseshoe{\CAPPHI}{\CAPTHETA}$ it follows that $\parhorseshoe{\CAPPHI}{\CAPTHETA}$ is \CAPS{tft}.  So, by the definition of \CAPS{tft}, $\parhorseshoe{\CAPPHI}{\CAPTHETA}$ is true on every model $\IntA$.
By the definition of truth of $\HORSESHOE$, it follows that there is no model that makes $\CAPPHI$ true and $\CAPTHETA$ false.  So, by the definition of $\:\sdtstile{}{}$, that means that $\CAPPHI\sdtstile{}{}\CAPTHETA$.
\end{PROOF}

There are a number of other theorems that are similar to, and expand upon, theorem \ref{Exponentiation of Entailment}. Here we state them and leave the proofs to the reader.
\begin{THEOREM}
{\LnpTC{expo generalizations}}
\begin{cenumerate}
\item If $\CAPPHI_1$, $\CAPPHI_2$, $\ldots$, $\CAPPHI_{\integer{n}}$ and $\CAPPSI$ are \GSL{} sentences, then
\begin{itemize}
\item[] $\CAPPHI_1,\CAPPHI_2,\ldots,\CAPPHI_{\integer{n}}\sdtstile{}{}\CAPPSI$ \Iff $\CAPPHI_2,\ldots,\CAPPHI_{\integer{n}}\sdtstile{}{}\parhorseshoe{\CAPPHI_1}{\CAPPSI}$
\item[] $\CAPPHI_1,\CAPPHI_2,\ldots,\CAPPHI_{\integer{n}}\sdtstile{}{}\CAPPSI$ \Iff $\CAPPHI_1,\CAPPHI_3,\ldots,\CAPPHI_{\integer{n}}\sdtstile{}{}\parhorseshoe{\CAPPHI_2}{\CAPPSI}$
\item[] \hspace{1in} $\vdots$
\item[] $\CAPPHI_1,\CAPPHI_2,\ldots,\CAPPHI_{n}\sdtstile{}{}\CAPPSI$ \Iff $\CAPPHI_1,\ldots,\CAPPHI_{n-1}\sdtstile{}{}\parhorseshoe{\CAPPHI_{\integer{n}}}{\CAPPSI}$
\end{itemize} 
\item If $\CAPPHI_1$, $\CAPPHI_2$, $\ldots$, $\CAPPHI_{\integer{n}}$ and $\CAPPSI$ are \GSL{} sentences, then
\begin{itemize}
\item[] $\CAPPHI_1,\CAPPHI_2,\ldots,\CAPPHI_{\integer{n}}\sdtstile{}{}\CAPPSI$ \Iff $\CAPPHI_1,\ldots,\CAPPHI_{n-1}\sdtstile{}{}\parhorseshoe{\CAPPHI_{\integer{n}}}{\CAPPSI}$
\item[] $\CAPPHI_1,\CAPPHI_2,\ldots,\CAPPHI_{\integer{n}}\sdtstile{}{}\CAPPSI$ \Iff $\CAPPHI_1,\ldots,\CAPPHI_{n-2}\sdtstile{}{}\parhorseshoe{\CAPPHI_{n-1}}{\parhorseshoe{\CAPPHI_{\integer{n}}}{\CAPPSI}}$
\item[] \hspace{1in} $\vdots$
\item[] $\CAPPHI_1,\CAPPHI_2,\ldots,\CAPPHI_{n}\sdtstile{}{}\CAPPSI$ \Iff $\sdtstile{}{}\parhorseshoe{\CAPPHI_1}{\parhorseshoe{\ldots}{\parhorseshoe{\CAPPHI_{n-1}}{\parhorseshoe{\CAPPHI_{\integer{n}}}{\CAPPSI}}}}$
\end{itemize}
\item If $\CAPPHI$ and $\CAPPSI$ are \GSL{} sentences and $\Delta$ is a set of \GSL{} sentences containing $\CAPPHI$ (i.e, $\CAPPHI\in\Delta$) and $\Delta^*$ is $\Delta$ with $\CAPPHI$ removed, then $\Delta\sdtstile{}{}\CAPPSI$ \Iff $\Delta^*\sdtstile{}{}\parhorseshoe{\CAPPHI}{\CAPPSI}$.
\item If $\CAPPHI$ and $\CAPPSI$ are \GSL{} sentences, then $\CAPPHI\sdtstile{}{}\CAPPSI$ \Iff $\CAPPHI,\negation{\CAPPSI}\sdtstile{}{}\parconjunction{\Al}{\negation{\Al}}$.
\item If $\CAPPHI_1,\ldots,\CAPPHI_{\integer{n}}$ and $\CAPPSI_1,\ldots,\CAPPSI_{\integer{n}}$ are all \GSL{} sentences, then
\begin{itemize}
\item[] $\CAPPHI_1,\ldots,\CAPPHI_{\integer{n}}\sdtstile{}{}\pardisjunction{\CAPPSI_1}{\disjunction{\ldots}{\CAPPSI_{\integer{n}}}}$ \Iff $\CAPPHI_1,\ldots,\CAPPHI_{\integer{n}},\negation{\CAPPSI_1}\sdtstile{}{}\pardisjunction{\CAPPSI_2}{\disjunction{\ldots}{\CAPPSI_{\integer{n}}}}$
\item[] $\CAPPHI_1,\ldots,\CAPPHI_{\integer{n}}\sdtstile{}{}\pardisjunction{\CAPPSI_1}{\disjunction{\ldots}{\CAPPSI_{\integer{n}}}}$ \Iff $\CAPPHI_1,\ldots,\CAPPHI_{\integer{n}},\negation{\CAPPSI_2}\sdtstile{}{}\pardisjunction{\CAPPSI_1}{\disjunction{\CAPPSI_3}{\disjunction{\ldots}{\CAPPSI_{\integer{n}}}}}$
\item[] \hspace{1in} $\vdots$
\item[] $\CAPPHI_1,\ldots,\CAPPHI_{\integer{n}}\sdtstile{}{}\pardisjunction{\CAPPSI_1}{\disjunction{\ldots}{\CAPPSI_{\integer{n}}}}$ \Iff $\CAPPHI_1,\ldots,\CAPPHI_{\integer{n}},\negation{\CAPPSI_{\integer{n}}}\sdtstile{}{}\pardisjunction{\CAPPSI_1}{\disjunction{\ldots}{\CAPPSI_{n-1}}}$
\end{itemize} 
\end{cenumerate}
\end{THEOREM}

\subsection{Other Relations}\label{Other Relations}

There are a number of other important relations between \GSL{} sentences:

\begin{majorILnc}{\LnpDC{GSL TFE}}
Two sentences $\CAPTHETA$ and $\CAPPHI$ are \index{equivalent sentences!truth functional|textbf} \nidf{truth functionally equivalent} (\CAPS{tfe}) \Iff for all models $\IntA$ for $\CAPTHETA$ and $\CAPPHI$, $\CAPTHETA$ and $\CAPPHI$ have the same truth value on $\IntA$.
\end{majorILnc}

\begin{THEOREM}{\LnpTC{tfe entailment}}
$\CAPTHETA$ and $\CAPPHI$ are \CAPS{tfe} \Iff $\CAPTHETA\sdtstile{}{}\CAPPHI$ and $\CAPPHI\sdtstile{}{}\CAPTHETA$.
\begin{PROOF}
	($\Rightarrow$) Assume that $\CAPTHETA$ and $\CAPPHI$ are \CAPS{tfe}.
	Then, by the definition of \CAPS{tfe}, there is no model on which $\CAPTHETA$ and $\CAPPHI$ have different truth values.
	So there is no model on which $\CAPTHETA$ is true and $\CAPPHI$ is false.
	Then, by the definition of $\entails$, $\CAPTHETA\entails\CAPPHI$.
	And there is no model on which $\CAPPHI$ is true and $\CAPTHETA$ is false.
	Then, by the definition of $\entails$, $\CAPPHI\entails\CAPTHETA$.

	($\Leftarrow$) Assume that $\CAPTHETA\sdtstile{}{}\CAPPHI$ and $\CAPPHI\sdtstile{}{}\CAPTHETA$.
	Since $\CAPTHETA\sdtstile{}{}\CAPPHI$, then by the definition of $\entails$ every model that makes $\CAPTHETA$ true also makes $\CAPPHI$ true.
	And since $\CAPPHI\sdtstile{}{}\CAPTHETA$, then by the definition of $\entails$ every model that makes $\CAPPHI$ true also makes $\CAPTHETA$ true.
	Since each sentence is true when the other is, and a sentence is false \Iff it's not true, $\CAPTHETA$ and $\CAPPHI$ must be the same value on all models.
	So they are \CAPS{tfe}
\end{PROOF}
\end{THEOREM}

\begin{majorILnc}{\LnpDC{GSL Contradictory}}
Two sentences $\CAPTHETA$ and $\CAPPHI$ are \nidf{truth functionally contradictory}\index{contradictory!truth functional|textbf} \Iff for all models $\IntA$ for $\CAPTHETA$ and $\CAPPHI$, $\CAPTHETA$ and $\CAPPHI$ have opposite truth values on $\IntA$.
\end{majorILnc}

\begin{THEOREM}{\LnpTC{tf contradictory entailment}}
	$\CAPTHETA$ and $\CAPPHI$ are truth functionally contradictory \Iff $\CAPTHETA$ is \CAPS{tfe} to $\negation{\CAPPHI}$.
	\begin{PROOF}
		($\Rightarrow$) Assume that $\CAPTHETA$ and $\CAPPHI$ are truth functionally contradictory.
		Then $\CAPTHETA$ and $\CAPPHI$ have opposite truth values on all models.
		By the definition of truth for $\negation{}$, $\CAPPHI$ and $\negation{\CAPPHI}$ have opposite truth values on all models.
		So on a model that makes $\CAPTHETA$ true, $\CAPPHI$ is false and $\negation{\CAPPHI}$ is true.
		And on a model that makes $\CAPTHETA$ false, $\CAPPHI$ is true and $\negation{\CAPPHI}$ is false.
		It follows that $\CAPTHETA$ and $\negation{\CAPPHI}$ have the same value on all models.
		Thus, by the definition of \CAPS{tfe}, $\CAPTHETA$ is \CAPS{tfe} to $\negation{\CAPPHI}$.
	
		($\Leftarrow$) Assume that $\CAPTHETA$ is \CAPS{tfe} to $\negation{\CAPPHI}$.
		Then by the definition of \CAPS{tfe} $\CAPTHETA$ and $\negation{\CAPPHI}$ have the same truth value on all models.
		By the definition of truth for $\negation{}$, $\CAPPHI$ and $\negation{\CAPPHI}$ have opposite truth values on all models.
		So on a model that makes $\CAPTHETA$ true, $\CAPPHI$ is true and $\negation{\CAPPHI}$ is false.
		And on a model that makes $\CAPTHETA$ false, $\CAPPHI$ is false and $\negation{\CAPPHI}$ is true.
		It follows that $\CAPTHETA$ and $\CAPPHI$ have opposite values on all models.
		Thus, $\CAPTHETA$ is truth functionally contradictory to $\CAPPHI$.

	\end{PROOF}
\end{THEOREM}

\begin{majorILnc}{\LnpDC{GSL Contrary}}
Two sentences $\CAPTHETA$ and $\CAPPHI$ are \nidf{truth functionally contrary}\index{contraries!truth functional|textbf} \Iff there is no model on which both are $\True$. 
\end{majorILnc}

\begin{majorILnc}{\LnpEC{tf contrary entailment}}
	$\CAPTHETA$ and $\CAPPHI$ are truth functionally contrary \Iff $\CAPTHETA,\CAPPHI\sdtstile{}{}\conjunction{\Al}{\negation{\Al}}$.
\end{majorILnc}
\begin{PROOF} ($\Rightarrow$) Assume that $\CAPTHETA$ and $\CAPPHI$ are truth functionally contrary.
	Then there is no model on which $\CAPTHETA$ and $\CAPPHI$ are both truth.
	It follows that there is no model that makes both $\CAPTHETA$ and $\CAPPHI$ true and $\conjunction{\Al}{\negation{\Al}}$ false.
	Thus, by the definition of $\entails$, $\CAPTHETA,\CAPPHI\sdtstile{}{}\conjunction{\Al}{\negation{\Al}}$.

	($\Leftarrow$) This direction is left as an exercise for the reader.
\end{PROOF}

\begin{majorILnc}{\LnpDC{GSL subcontrary}}
Two sentences $\CAPTHETA$ and $\CAPPHI$ are \nidf{truth functionally subcontrary}\index{subcontraries!truth functional|textbf} \Iff there is no model on which both are $\False$. 
\end{majorILnc}

\begin{THEOREM}{\LnpTC{tf subcontrary entailment}}
	$\CAPTHETA$ and $\CAPPHI$ are truth functionally subcontrary \Iff $\sdtstile{}{}\disjunction{\CAPTHETA}{\CAPPHI}$. This proof is left as an exercise for the reader.
\end{THEOREM}

\begin{majorILnc}{\LnpDC{GSL Independence}}
Two sentences $\CAPTHETA$ and $\CAPPHI$ are \nidf{truth functionally independent}\index{independent sentences!truth functional|textbf} \Iff there are four models:
\begin{cenumerate}
\item A model in which both $\CAPTHETA$ and $\CAPPHI$ are $\True$; 
\item A model in which both $\CAPTHETA$ and $\CAPPHI$ are $\False$;
\item A model in which $\CAPTHETA$ is $\True$ and $\CAPPHI$ is $\False$; and
\item A model in which $\CAPTHETA$ is $\False$ and $\CAPPHI$ is $\True$.
\end{cenumerate}
\end{majorILnc}

\noindent{}It follows truth functionally independent sentences are not truth functionally equivalent, contradictory, contrary, or subcontrary.

\begin{majorILnc}{\LnpEC{TFE Ex 4}}
Each pair of contradictory sentences is also contrary, but sentences can be contrary without being contradictory.
\end{majorILnc}
\begin{PROOF}
$\conjunction{\Cl}{\Dl}$ and $\conjunction{\Cl}{\negation{\Dl}}$ are contrary but not contradictory.

(Contrary:) If $\conjunction{\Cl}{\Dl}$ is true in a model $\IntA$, then $\Cl$ and $\Dl$ are each true on $\IntA$. 
So $\negation{\Dl}$ is false in $\IntA$, and $\conjunction{\Cl}{\negation{\Dl}}$ is false in $\IntA$.
If $\conjunction{\Cl}{\negation{\Dl}}$ is true in $\IntA$, then both $\Cl$ and $\negation{\Dl}$ are true on $\IntA$.
$\Dl$ is false in $\IntA$, and hence $\conjunction{\Cl}{\Dl}$ is false in $\IntA$.
Thus, by def. \ref{GSL Contrary}, the pair is contrary.

(Not Contradictory:) Any model $\IntA$ that assigns $\FalseB$ to $\Cl$ makes both $\conjunction{\Cl}{\Dl}$ and $\conjunction{\Cl}{\negation{\Dl}}$ false.
By def. \ref{GSL Contradictory}, the pair is not contradictory.
\end{PROOF}
\begin{majorILnc}{\LnpEC{TFE Ex 5}}
Contradictory sentences are also subcontrary, but sentences can be subcontrary without being contradictory. 
\end{majorILnc}
\begin{PROOF}
$\Dl$ and $\disjunction{\Cl}{\negation{\Dl}}$ are subcontrary but not contradictory.

(Subcontrary:) Assume that $\Dl$ is false on a model $\IntA$.
It follows that $\negation{\Dl}$ is true on $\IntA$ and so is $\disjunction{\Cl}{\negation{\Dl}}$. 
Alternatively, assume that $\disjunction{\Cl}{\negation{\Dl}}$ is false on $\IntA$.  Then both $\Cl$ and $\negation{\Dl}$ are false on $\IntA$.
So $\Dl$ is true on $\IntA$.
By def. \ref{GSL subcontrary}, the pair is subcontrary.

(Not Contradictory:) Any model that assigns $\TrueB$ to both $\Dl$ and $\Cl$ makes both $\Dl$ and $\disjunction{\Cl}{\negation{\Dl}}$ true. 
So by def. \ref{GSL Contradictory}, the pair isn't contradictory.
\end{PROOF}
\begin{majorILnc}{\LnpEC{TFE Ex 6}}
If two sentences are both contrary and subcontrary, they are contradictory. 
\end{majorILnc}
\begin{PROOF}
If two sentences are contrary, then by definition \ref{GSL Contrary} any model $\IntA$ that makes one true makes the other false. 
If two sentences are subcontrary, then by definition \ref{GSL subcontrary} any model $\IntA$ that makes one false makes the other true. 
Because every model either makes a sentence true or makes it false, no model assigns two sentences that are contrary and subcontrary the same truth value. 
So by definition \ref{GSL Contradictory}, two sentences that are contrary and subcontrary are also contradictory. 
\end{PROOF}
\begin{majorILnc}{\LnpEC{TFE Ex 1}}
$\disjunction{\Al}{\parconjunction{\Bl}{\Dl}}$ and $\conjunction{\pardisjunction{\Al}{\Bl}}{\pardisjunction{\Al}{\Dl}}$ are \CAPS{tfe}.
\end{majorILnc}
\begin{PROOF}
($\Rightarrow$) Assume that $\disjunction{\Al}{\parconjunction{\Bl}{\Dl}}$ is true on some $\IntA$. 
By the def. of truth of $\VEE$, either $\Al$ is true on $\IntA$ or $\conjunction{\Bl}{\Dl}$ is true on $\IntA$.
(Case 1) $\Al$ is true on $\IntA$.
Then both $\disjunction{\Al}{\Bl}$ and $\disjunction{\Al}{\Dl}$ are true on $\IntA$.
So, $\conjunction{\pardisjunction{\Al}{\Bl}}{\pardisjunction{\Al}{\Dl}}$ is true on $\IntA$.
(Case 2) $\conjunction{\Bl}{\Dl}$ is true on $\IntA$
Then then both $\Bl$ and $\Dl$ are true on $\IntA$, and so both $\disjunction{\Al}{\Bl}$ and $\disjunction{\Al}{\Dl}$ are true on $\IntA$.
Hence, $\conjunction{\pardisjunction{\Al}{\Bl}}{\pardisjunction{\Al}{\Dl}}$ is true on $\IntA$. 

In either case, $\conjunction{\pardisjunction{\Al}{\Bl}}{\pardisjunction{\Al}{\Dl}}$ is true on $\IntA$.
Thus, $\disjunction{\Al}{\parconjunction{\Bl}{\Dl}}\sdtstile{}{}\conjunction{\pardisjunction{\Al}{\Bl}}{\pardisjunction{\Al}{\Dl}}$.

($\Leftarrow$) We leave it to the reader to show that $\conjunction{\pardisjunction{\Al}{\Bl}}{\pardisjunction{\Al}{\Dl}}\sdtstile{}{}\disjunction{\Al}{\parconjunction{\Bl}{\Dl}}$. 
It then follows by theorem \ref{tfe entailment} that the two sentences are \CAPS{tfe}.
\end{PROOF}
\begin{majorILnc}{\LnpEC{TFE Ex 2}}
For any \GSL{} sentences $\CAPPHI$ and $\CAPTHETA$, $\horseshoe{\CAPPHI}{\CAPTHETA}$ and $\disjunction{\negation{\CAPPHI}}{\CAPTHETA}$ are \CAPS{tfe}.
\end{majorILnc}
\begin{PROOF}
($\Rightarrow$) Assume that $\horseshoe{\CAPPHI}{\CAPTHETA}$ is true on some model $\IntA$.
Then on $\IntA$ either $\CAPPHI$ is false or $\CAPTHETA$ is true. 
Hence either $\negation{\CAPPHI}$ or $\CAPTHETA$ is true on $\IntA$. 
It follows that $\disjunction{\negation{\CAPPHI}}{\CAPTHETA}$ is true on $\IntA$. 
Hence by the definition of $\:\sdtstile{}{}$, $\horseshoe{\CAPPHI}{\CAPTHETA}\sdtstile{}{}\disjunction{\negation{\CAPPHI}}{\CAPTHETA}$.

($\Leftarrow$) We leave it to the reader to show that $\disjunction{\negation{\CAPPHI}}{\CAPTHETA}\sdtstile{}{}\horseshoe{\CAPPHI}{\CAPTHETA}$.
It then follows by theorem \ref{tfe entailment} that the two sentences are \CAPS{tfe}.
\end{PROOF}
\begin{majorILnc}{\LnpEC{TFE Ex 3}}
For any \GSL{} sentence $\CAPPHI$, $\negation{\negation{\CAPPHI}}$ and $\CAPPHI$ are \CAPS{tfe}.
\end{majorILnc}
\begin{PROOF}
The proof is left to the reader.
\end{PROOF}

\subsection{Procedures for Testing Other Relations} 

As before we can use truth tables to test for the following relationships: truth-functional equivalence, truth-functional contradictory, truth-functional contrary, truth-functional subcontrary, and truth-functional independence. 
We can test for truth-functional equivalence by putting both $\CAPPHI$ and $\CAPTHETA$ in a truth table and checking whether they have the same truth value in every row. 
\begin{majorILnc}{\LnpEC{TFE Ex 1 2}}
In example \mvref{TFE Ex 1} we saw that $\disjunction{\Al}{\parconjunction{\Bl}{\Dl}}$ and $\conjunction{\pardisjunction{\Al}{\Bl}}{\pardisjunction{\Al}{\Dl}}$ are \CAPS{tfe}. 
We can also prove this with a truth table. 
\begin{center}
\begin{tabular}{ c c c c c }
$\Al$ & $\Bl$ & $\Dl$ & $\disjunction{\Al}{\parconjunction{\Bl}{\Dl}}$ & $\conjunction{\pardisjunction{\Al}{\Bl}}{\pardisjunction{\Al}{\Dl}}$ \\
\hline
$ $ & $ $ & & & \\[-.25cm]
$\TrueB$ & $\TrueB$ & $\TrueB$ & $\TrueB$ & $\TrueB$ \\
$\TrueB$ & $\TrueB$ & $\FalseB$& $\TrueB$ & $\TrueB$ \\
$\TrueB$ & $\FalseB$ & $\TrueB$ & $\TrueB$ & $\TrueB$\\
$\TrueB$ & $\FalseB$ & $\FalseB$  & $\TrueB$ & $\TrueB$\\
$\FalseB$ & $\TrueB$ & $\TrueB$ & $\TrueB$ & $\TrueB$\\
$\FalseB$ & $\TrueB$ & $\FalseB$& $\FalseB$ & $\FalseB$\\
$\FalseB$ & $\FalseB$ & $\TrueB$ & $\FalseB$ & $\FalseB$\\
$\FalseB$ & $\FalseB$ & $\FalseB$  & $\FalseB$ & $\FalseB$\\
\end{tabular}
\end{center}
\end{majorILnc}
\begin{majorILnc}{\LnpEC{TFE Ex 2 2}}
In example \mvref{TFE Ex 2} we saw that for any \GSL{} sentences $\CAPPHI$ and $\CAPTHETA$, $\horseshoe{\CAPPHI}{\CAPTHETA}$ and $\disjunction{\negation{\CAPPHI}}{\CAPTHETA}$ are \CAPS{tfe}. 
We confirm with a truth table:
\begin{center}
\begin{tabular}{ c c c c c }
$\CAPPHI$ & $\CAPTHETA$ & $\horseshoe{\CAPPHI}{\CAPTHETA}$ & $\disjunction{\negation{\CAPPHI}}{\CAPTHETA}$ & \\
\hline
$ $ & $ $ & & & \\[-.25cm]
$\TrueB$ & $\TrueB$ & $\TrueB$ & $\TrueB$ \\
$\TrueB$ & $\FalseB$& $\FalseB$ & $\FalseB$ \\
$\FalseB$ & $\TrueB$ & $\TrueB$ & $\TrueB$\\
$\FalseB$ & $\FalseB$  & $\TrueB$ & $\TrueB$\\
\end{tabular}
\end{center}
In this example we're not looking at \GSL{} sentences; instead we have sentence schemas. 
But for reasons to be discussed below this procedure still works: this truth table shows that for any two sentences $\CAPPHI$ and $\CAPTHETA$, $\horseshoe{\CAPPHI}{\CAPTHETA}$ and $\disjunction{\negation{\CAPPHI}}{\CAPTHETA}$ are \CAPS{tfe}.
\end{majorILnc}
While truth tables provide guaranteed answers to these questions, students should not become reliant on this method. 
For one thing, this method doesn't work for more complex languages, and it is important to practice with \GSL{} structures the reasoning skills and methods needed in later chapters. 
Secondly, truth tables can quickly become very large and tedious, and often simple reasoning suffices. 
More generally, truth tables give an answer but often don’t give insight.

%%%%%%%%%%%%%%%%%%%%%%%%%%%%%%%%%%%%%%%%%%%%%%%%%%
\section{Recursive Proofs}\label{Recursive Proofs}
%%%%%%%%%%%%%%%%%%%%%%%%%%%%%%%%%%%%%%%%%%%%%%%%%%

\subsection{The Method of Recursive Proof}
Many logic concepts are characterized by recursive definitions.  
To prove a theorem about a recursively defined concept, we usually want to employ a method called \df{recursive proof}. 
The structure of a recursive proof mirrors the structure of a recursive definition.

\begin{majorILnc}{\LnpDC{Definition of Recursive Proof}}
	Let $\Delta$ be some set whose members are defined recursively. A \df{recursive proof} that all members of $\Delta$ have some property $\CAPPHI$ proceeds as follows:
	\begin{description}
		\item[Base Step:] Show that everything identified by the base clause of the recursive definition has $\CAPPHI$.  
		\item[Inheritance Step:] Show that $\CAPPHI$ is inherited; i.e., show that if the previous elements from which new elements are generated (or found) by the generating clause have $\CAPPHI$ then the new ones have $\CAPPHI$ too.
		\item[Closure Step:] Finally, show that the base and inheritance steps are sufficient to show that all elements of $\Delta$ have $\CAPPHI$. 
		
	\end{description}
\end{majorILnc}
The inheritance step usually has two parts. The first part is a \emph{recursive assumption}.
We \emph{assume} that some previous elements---the elements from which new ones are generated by the generating clause---already have the property in question.
Typically we make this assumption by selecting metavariables to represent the previous elements.
We are entitled to this assumption because we know that there are some previous elements.
At the very least the elements identified in the base clause have the property.
Second, we prove that the new, generated elements must also have the property in question.

How much we need to write in the closure step varies from proof to proof. 
Sometimes we simply note that the closure condition (``that's all'') ensures nothing has been left out. 
Other times, if the proof is more complicated, we might also reiterate what we have just shown. 

\subsection{Recursive Proof and Mathematical Induction}
Many important mathematical concepts are defined recursively. 
We saw the recursive definition of natural numbers in the previous chapter. 
If we want to \emph{prove} something about all natural numbers, we can use mathematical induction\index{principle of mathematical induction}.
\emph{Mathematical induction} is a method of proof in which one shows (i) that some property holds of the first natural number, and (ii) that for any natural number $n$ having that property, the successor of $n$ also has that property.
Mathematical induction is one sort of recursive proof:
\begin{description}
	\item[Base Step:] $0$ has property $\CAPPHI$. 
	\item[Inheritance Step:] Whenever $\integer{n}$ has property $\CAPPHI$, its successor, $n+1$, also has $\CAPPHI$.
\end{description}
Therefore, we conclude that
\begin{description}
	\item[Closure Step:] All natural numbers have property $\CAPPHI$.
\end{description}

Let's go through an example of mathematical induction.
This first proof isn't exciting but it illustrates how recursive proofs work.

\begin{majorILnc}{\LnpEC{English Recursive Proof 1}} 
	There is no largest natural number $n$. 
	
	\begin{PROOF}	\begin{cenumerate}
			\item Base Step: The number $0$ isn't the largest; $1$ is larger.
			\item Inheritance Step: Assume that $n$ isn't the largest natural number. (This is the recursive assumption.)
			Is $n+1$ the largest natural number? No. $n+1$ can't be the largest because $n+1<n+2$.  
			\item Closure Step: Therefore no natural number $n$ is the largest.
	\end{cenumerate}
\end{PROOF}
\end{majorILnc}

\noindent{}Two things can give people trouble with recursive proofs.
One is that the base case is often easy or trivial. 
Make sure you have done it correctly, but don't worry if it seems too easy. 

Second, some students initially think that the recursive assumption in the inheritance step assumes what we are trying to prove.
But in a correct proof that isn't so.
Think of the recursive assumption this way: \emph{if} some elements have property $\CAPPHI$, then we can show that other elements also have $\CAPPHI$.
We know that some elements have the property in question: those identified in the base step.
The recursive assumption merely provides a label for these previously identified elements bearing the property in question.

%%%%%%%%%%%%%%%%%%%%%%%%%%%%%%%%%%%%%%%%%%%%%%%%%%
\section{Recursive Proofs in SL}\label{recursive proofs in SL}
%%%%%%%%%%%%%%%%%%%%%%%%%%%%%%%%%%%%%%%%%%%%%%%%%%

\subsection{Recursive Proof Examples in SL}

Let's use recursive proofs to establish two simple results about \GSL{} sentences.  

\begin{majorILnc}{\LnpEC{Recursive Proof Ex 1}} 
Prove that every (official) sentence has exactly as many left parentheses as right.
\begin{PROOF}
\begin{description}
	\item[Base Step:] All atomic sentences have zero left parentheses and zero right parentheses. 
	Clearly $0=0$.
	\item[Inheritance Step:] \hfill{}

	\begin{description}

		\item[Recursive Assumption:] Suppose $\CAPPHI$ and $\CAPTHETA,\CAPTHETA_1,\CAPTHETA_2,\ldots,\CAPTHETA_{\integer{n}}$ are sentences of order $k$ or less, each with exactly as many left parentheses as right.
		Sentences of order $k+1$ can be constructed as follows:

		\item[Negation:] Adding a negation sign to $\CAPPHI$ does not change the number of parentheses. Since, by recursive assumption (RA), $\CAPPHI$ has the same number of left and right parentheses, so does $\negation{\CAPPHI}$.

		\item[Conditional/Biconditional:] By RA, each of $\CAPPHI$ and $\CAPTHETA$ has matching numbers of left and right parentheses.
		It follows that the number of left parentheses in both sentences matches the number of right parentheses in both.
		The sentences $\parhorseshoe{\CAPPHI}{\CAPTHETA}$ and $\partriplebar{\CAPPHI}{\CAPTHETA}$ have an additional pair of parentheses, one left and one right, and this preserves the property of having the same number of left and right parentheses.

		\item[Disjunction/Conjunction:] By RA, each of $\CAPTHETA_1,\CAPTHETA_2,\ldots,\CAPTHETA_{\integer{n}}$ has a matching number of left and right parentheses.
		It follows that the number of left parentheses in all of them matches the number of right parentheses in all of them.
		The sentences $\parconjunction{\CAPTHETA_1}{\conjunction{\CAPTHETA_2}{\conjunction{\ldots}{\CAPTHETA_{\integer{n}}}}}$ and $\pardisjunction{\CAPTHETA_1}{\disjunction{\CAPTHETA_2}{\disjunction{\ldots}{\CAPTHETA_{\integer{n}}}}}$ have one additional pair of parentheses, one left and one right, the addition of which preserves the property that the number of left parentheses is the same as the number of right parentheses.
	\end{description}

	\item[Closure Step:] The above steps cover all the ways of generating an \GSL{} sentence. In each case the number of left parentheses equals the number of right parentheses.
\end{description}
\end{PROOF}
\end{majorILnc}

\begin{commentary}
	This recursive proof is structured to match the definition of an \GSL{} sentence, which is typical of proofs that all \GSL{} sentences have some property.
	The definition of a sentence has a base clause for sentence letters, and generating clauses for all the ways to construct a sentence of order $k+1$ from subsentences that have order $k$ or less.
	Accordingly, the recursive assumption is nearly always of the form ``Sentences of order $k$ or less have such and such property.''
	The goal in the rest of the inheritance step is to show that all the ways of constructing a sentence of order $k+1$ preserve that property.
\end{commentary}

\begin{majorILnc}{\LnpEC{Recursive Proof Ex 2}} 
Prove that in every official \GSL{} sentence the number of left parentheses is greater than or equal to the number of arrows.
\begin{PROOF}
Let $\LP\CAPPHI$ be the number of left parentheses in $\CAPPHI$ and $\HHH\CAPPHI$ be the number of arrows in $\CAPPHI$.
\begin{description}
\item[Base Step:] In the base case $\CAPPHI$ is atomic, so $\LP\CAPPHI=0$ and $\HHH\CAPPHI=0$.  Thus, $\LP\CAPPHI=\HHH\CAPPHI$ and so $\LP\CAPPHI\geq\HHH\CAPPHI$.
\item[Inheritance Step:] \hfill{}
\begin{description}
\item[Recursive Assumption:] Let $\CAPTHETA$, $\CAPTHETA_1$, $\CAPTHETA_2$, $\ldots$ $\CAPTHETA_{\integer{n}}$ be sentences of order $k$ or less. 
Assume that $\LP\CAPTHETA\geq\HHH\CAPTHETA$, $\LP\CAPTHETA_1\geq\HHH\CAPTHETA_1$, $\LP\CAPTHETA_2\geq\HHH\CAPTHETA_2$, $\ldots$ $\LP\CAPTHETA_{\integer{n}}\geq\HHH\CAPTHETA_{\integer{n}}$. Sentences of order $k+1$ can be constructed as follows: 
\item[Negation:] $\LP\negation{\CAPTHETA}=\LP\CAPTHETA$ and $\HHH\negation{\CAPTHETA}=\HHH\CAPTHETA$. By assumption $\LP\CAPTHETA\geq\HHH\CAPTHETA$, so $\LP\negation{\CAPTHETA}\geq\HHH\negation{\CAPTHETA}$ too.
\item[Conditional:] $\LP\parhorseshoe{\CAPTHETA_1}{\CAPTHETA_2}=\LP\CAPTHETA_1+\LP\CAPTHETA_2+1$ and $\HHH\parhorseshoe{\CAPTHETA_1}{\CAPTHETA_2}=\HHH\CAPTHETA_1+\HHH\CAPTHETA_2+1$. Because $\LP\CAPTHETA_1\geq\HHH\CAPTHETA_1$ and $\LP\CAPTHETA_2\geq\HHH\CAPTHETA_2$, clearly $\LP\CAPTHETA_1+\LP\CAPTHETA_2+1\geq\HHH\CAPTHETA_1+\HHH\CAPTHETA_2+1$ too. So $\LP\parhorseshoe{\CAPTHETA_1}{\CAPTHETA_2}\geq\HHH\parhorseshoe{\CAPTHETA_1}{\CAPTHETA_2}$.
\item[Biconditional:] $\LP\partriplebar{\CAPTHETA_1}{\CAPTHETA_2}=\LP\CAPTHETA_1+\LP\CAPTHETA_2+1$ and $\HHH\partriplebar{\CAPTHETA_1}{\CAPTHETA_2}=\HHH\CAPTHETA_1+\HHH\CAPTHETA_2$. 
Because $\LP\CAPTHETA_1\geq\HHH\CAPTHETA_1$ and $\LP\CAPTHETA_2\geq\HHH\CAPTHETA_2$, clearly $\LP\CAPTHETA_1+\LP\CAPTHETA_2+1\geq\HHH\CAPTHETA_1+\HHH\CAPTHETA_2$ too. 
So $\LP\partriplebar{\CAPTHETA_1}{\CAPTHETA_2}\geq\HHH\partriplebar{\CAPTHETA_1}{\CAPTHETA_2}$.
\item[Disjunction:] We have that $\LP\pardisjunction{\CAPTHETA_1}{\disjunction{\CAPTHETA_2}{\disjunction{\ldots}{\CAPTHETA_{\integer{n}}}}}=\LP\CAPTHETA_1+\LP\CAPTHETA_2+\ldots+\LP\CAPTHETA_{\integer{n}}+1$ and $\HHH\pardisjunction{\CAPTHETA_1}{\disjunction{\CAPTHETA_2}{\disjunction{\ldots}{\CAPTHETA_{\integer{n}}}}}=\HHH\CAPTHETA_1+\HHH\CAPTHETA_2+\ldots+\HHH\CAPTHETA_{\integer{n}}$. 
Because $\LP\CAPTHETA_1\geq\HHH\CAPTHETA_1$, $\LP\CAPTHETA_2\geq\HHH\CAPTHETA_2$, $\ldots$ $\LP\CAPTHETA_{\integer{n}}\geq\HHH\CAPTHETA_{\integer{n}}$, we have that $\LP\CAPTHETA_1+\LP\CAPTHETA_2+\ldots+\LP\CAPTHETA_{\integer{n}}+1\geq\HHH\CAPTHETA_1+\HHH\CAPTHETA_2+\ldots+\HHH\CAPTHETA_{\integer{n}}$. 
So, $\LP\pardisjunction{\CAPTHETA_1}{\disjunction{\CAPTHETA_2}{\disjunction{\ldots}{\CAPTHETA_{\integer{n}}}}}\geq\HHH\pardisjunction{\CAPTHETA_1}{\disjunction{\CAPTHETA_2}{\disjunction{\ldots}{\CAPTHETA_{\integer{n}}}}}$ too.
\item[Conjunction:] The argument is the same as that for disjunction, but with each \mention{$\VEE$} replaced with \mention{$\!\WEDGE\!$}.
\end{description}
\item[Closure Step:] These are all the ways of constructing \GSL{} sentences, and in every case the number of left parentheses is greater than or equal to the number of arrows.\end{description}
\end{PROOF}
\end{majorILnc}%

\subsection{Minimal Model Theorem}\label{minimal model theorem} 

Earlier we made the following claim: to determine whether some sentence $\CAPPHI$ is true on a model, you only need to check the assignments to the sentence letters in $\CAPPHI$.
Now we prove it.

\begin{THEOREM}{\LnpTC{thm:localityoftruth}}
	If two models for $\CAPPHI$, $\As{}{1}$ and $\As{}{2}$, make the same assignments to all the sentence letters contained in $\CAPPHI$, then $\CAPPHI$ is true on $\As{}{1}$ \Iff $\CAPPHI$ is true on $\As{}{2}$.
\end{THEOREM}
\begin{PROOF}
	\begin{description}
		\item[Base Step:]  Let $\CAPTHETA$ be a \GSL{} sentence of order 1.  It follows that $\CAPTHETA$ must be a lone sentence letter.  If $\As{}{1}$ and $\As{}{2}$ make the same assignments for all the sentence letters, then $\CAPTHETA$, which is just one sentence letter, is true on $\As{}{1}$ \Iff $\CAPTHETA$ is true on $\As{}{2}$.
		
		\item[Inheritance Step:] \hfill{}
		\begin{description}
			\item[Recursive Assumption:] Let $\CAPPHI$ and $\CAPTHETA_1,\CAPTHETA_2,\ldots,\CAPTHETA_{\integer{n}}$ be sentences of order $k$ or less. Assume that $\CAPPHI$ is true on $\As{}{1}$ \Iff $\CAPPHI$ is true on $\As{}{2}$, $\CAPTHETA_1$ is true on $\As{}{1}$ \Iff $\CAPTHETA_1$ is true on $\As{}{2}$, $\ldots$, $\CAPTHETA_n$ is true on $\As{}{1}$ \Iff $\CAPTHETA_n$ is true on $\As{}{2}$.

			\item[Negation:] ($\Rightarrow$) Assume that $\negation{\CAPPHI}$ is true on $\As{}{1}$.
			Then, by the definition of truth, $\NEGATION$, $\CAPPHI$ is false on $\As{}{1}$.
			So, by RA, $\CAPPHI$ is false on $\As{}{2}$.
			It follows that $\negation{\CAPPHI}$ is true on $\As{}{2}$.

			($\Leftarrow$) Assume that $\negation{\CAPPHI}$ is true on $\As{}{2}$.
			Then, by the definition of truth, $\NEGATION$, $\CAPPHI$ is false on $\As{}{2}$.
			So, by RA, $\CAPPHI$ is false on $\As{}{1}$.
			It follows that $\negation{\CAPPHI}$ is true on $\As{}{1}$.
			\begin{commentary}
				Note that the ($\Leftarrow$) part is the same as the ($\Rightarrow$) part, but with $\As{}{1}$ and $\As{}{2}$ switched.
				For the rest of this proof we omit such redundant details.
			\end{commentary}		
			
			\item[Conditional:] ($\Rightarrow$) Assume that $\horseshoe{\CAPTHETA_1}{\CAPTHETA_2}$ is true on $\As{}{1}$.
			Then, by the definition of truth, $\HORSESHOE$, $\As{}{1}$ makes either $\CAPTHETA_1$ false or $\CAPTHETA_2$ true.
			(Case 1) $\As{}{1}$ makes $\CAPTHETA_1$ false.
			So, by RA, $\As{}{2}$ makes $\CAPTHETA_1$ false.
			(Case 2) $\As{}{1}$ makes $\CAPTHETA_2$ true.
			So, by RA, $\As{}{2}$ makes $\CAPTHETA_2$ true.
			In both cases $\As{}{2}$ makes $\horseshoe{\CAPTHETA_1}{\CAPTHETA_2}$ true.

			($\Leftarrow$) This is the same as ($\Rightarrow$), but with $\As{}{1}$ and $\As{}{2}$ switched.
			
			\item[Biconditional:] ($\Rightarrow$) Assume that $\triplebar{\CAPTHETA_1}{\CAPTHETA_2}$ is true on $\As{}{1}$.
			Then, by the definition of truth, $\TRIPLEBAR$, $\As{}{1}$ either makes both $\CAPTHETA_1$ and $\CAPTHETA_2$ true or it makes both $\CAPTHETA_1$ and $\CAPTHETA_2$ false.
			(Case 1) $\As{}{1}$ makes $\CAPTHETA_1$ and $\CAPTHETA_2$ true.
			So, by RA, $\As{}{2}$ makes $\CAPTHETA_1$ and $\CAPTHETA_2$ true.
			(Case 2) $\As{}{1}$ makes $\CAPTHETA_1$ and $\CAPTHETA_2$ false.
			So, by RA, $\As{}{2}$ makes $\CAPTHETA_1$ and $\CAPTHETA_2$ false.
			In both cases $\As{}{2}$ makes $\triplebar{\CAPTHETA_1}{\CAPTHETA_2}$ true.

    		($\Leftarrow$) This is the same as ($\Rightarrow$), but with $\As{}{1}$ and $\As{}{2}$ switched.

			\item[Disjunction:] ($\Rightarrow$) Assume that $\pardisjunction{\CAPTHETA_1}{\disjunction{\CAPTHETA_2}{\disjunction{\ldots}{\CAPTHETA_{\integer{n}}}}}$ is true on $\As{}{1}$.
			Then, by the definition of truth, $\VEE$, there is at least one $\CAPTHETA_i$ that is true on $\As{}{1}$, where $1\leq i\leq n$.
			So, by RA, $\CAPTHETA_i$ is true on $\As{}{2}$.
			It follows that $\pardisjunction{\CAPTHETA_1}{\disjunction{\CAPTHETA_2}{\disjunction{\ldots}{\CAPTHETA_{\integer{n}}}}}$ is true on $\As{}{2}$.

			($\Leftarrow$) This is the same as ($\Rightarrow$), but with $\As{}{1}$ and $\As{}{2}$ switched.

			\item[Conjunction:] ($\Rightarrow$) Assume that $\parconjunction{\CAPTHETA_1}{\conjunction{\CAPTHETA_2}{\conjunction{\ldots}{\CAPTHETA_{\integer{n}}}}}$ is true on $\As{}{1}$.
			Then, by the definition of truth, $\WEDGE$, each $\CAPTHETA_i$ is true on $\As{}{1}$, where $1\leq i\leq n$.
			So, by RA, each $\CAPTHETA_i$ is true on $\As{}{2}$.
			It follows that $\parconjunction{\CAPTHETA_1}{\conjunction{\CAPTHETA_2}{\conjunction{\ldots}{\CAPTHETA_{\integer{n}}}}}$ is true on $\As{}{2}$.

			($\Leftarrow$) This is the same as ($\Rightarrow$), but with $\As{}{1}$ and $\As{}{2}$ switched.

		\end{description}
		\item[Closure Step:] These are all the ways of constructing \GSL{} sentences. Therefore, if $\As{}{1}$ and $\As{}{2}$ make the same assignments for all the sentence letters in some $\CAPPHI$, then $\CAPPHI$ is true on $\As{}{1}$ \Iff $\CAPPHI$ is true on $\As{}{2}$.
	\end{description}
\end{PROOF}


\subsection{Main Connective Theorem}\label{additional recur examples} 

We defined the main connective of a sentence $\CAPPHI$ as the connective that isn't in any proper subsentence of $\CAPPHI$ (\ref{GSL Main connective}).  Atomic sentences don't have any connectives, and so don't have main connectives.  Most non-atomic sentences have just one main connective, but there is an exception: conjunctions (or disjunctions) with more than two conjuncts (disjuncts). For example, all three of the \mention{$\WEDGE$} tokens in the following sentence are the main connectives: $\parconjunction{\conjunction{\Bl}{\Al}}{\conjunction{\Hl}{\Gl}}$.  Let's call sentences that have multiple tokens of \mention{$\WEDGE$} as main connectives \mention{extended conjunctions}, and those that have multiple tokens of \mention{$\VEE$} as main connectives \mention{extended disjunctions}.  With this terminology established, we turn to another recursive proof.

\begin{THEOREM}{\LnpTC{Recur Main Connective} Main Connective Theorem:} 
For\index{main connective!theorem} every \GSL{} sentence $\CAPPHI$, one of the following holds: (i) $\CAPPHI$ has no main connective; (ii) $\CAPPHI$ has exactly one main connective token; or (iii) $\CAPPHI$ is an extended conjunction (or disjunction) with $n-1$ main connective tokens, where $n$ is the number of conjuncts (disjuncts).
\end{THEOREM}
\begin{PROOFOF}{Thm. \ref{Recur Main Connective}, Main Connective Theorem}
\begin{description}
\item[Base Step:] 
Every \GSL{} sentence of order $1$ is just a sentence letter. Therefore it has no main connective.
 
\item[Inheritance Step:] \hfill{}

\begin{description}

\item[Recursive Assumption:] Let $\CAPPHI$ and $\CAPTHETA_1,\CAPTHETA_2,\ldots,\CAPTHETA_{\integer{n}}$ be sentences of order $k$ or less. Assume that one of (i)-(iii) holds of each of these sentences. Now consider the ways in which a sentence of order $k+1$ can be constructed from them:

\item[Negation:] Every connective in $\CAPPHI$ is in a proper subsentence of $\negation{\CAPPHI}$.
The only connective not in a proper subsentence of $\negation{\CAPPHI}$ is \mention{$\NEGATION$}.
So \mention{$\NEGATION$} is the main connective.
Thus $\negation{\CAPPHI}$ has precisely one main connective, thereby meeting condition (ii).

\item[Conditional/Biconditional:] All the connectives in each of $\CAPTHETA_1$ and $\CAPTHETA_2$ is in a proper subsentence of $\parhorseshoe{\CAPTHETA_1}{\CAPTHETA_2}$.
The only connective not in a proper subsentence of $\parhorseshoe{\CAPTHETA_1}{\CAPTHETA_2}$ is \mention{$\HORSESHOE$}.
So \mention{$\HORSESHOE$} is the main connective.
Thus $\parhorseshoe{\CAPTHETA_1}{\CAPTHETA_2}$ has precisely one main connective, thereby meeting condition (ii).
The same reasoning holds of $\partriplebar{\CAPTHETA_1}{\CAPTHETA_2}$, except that it has \mention{$\TRIPLEBAR$} as its main connective.

\item[Conjunction/Disjunction:] Consider a conjunction, $\parconjunction{\CAPTHETA_1}{\conjunction{\CAPTHETA_2}{\conjunction{\ldots}{\CAPTHETA_{\integer{n}}}}}$. If $n=2$ then the reasoning of the \mention{Conditional/Biconditional} clause holds, and there is only $1$ main connective. Otherwise $n>2$, in which case the conjunction is extended.
In that case, every connective in each of $\CAPTHETA_1,\CAPTHETA_2,\ldots,\CAPTHETA_n$ is in a proper subsentence of $\parconjunction{\CAPTHETA_1}{\conjunction{\CAPTHETA_2}{\conjunction{\ldots}{\CAPTHETA_{\integer{n}}}}}$.
The only connectives not in a proper subsentence are the \mention{$\WEDGE$} tokens between each pair of $\CAPTHETA_i$ and $\CAPTHETA_i+1$ subsentences, where $1\leq i\leq n-1$.
So the main connectives are those $n-1$ \mention{$\WEDGE$} tokens.
Thus $\parconjunction{\CAPTHETA_1}{\conjunction{\CAPTHETA_2}{\conjunction{\ldots}{\CAPTHETA_{\integer{n}}}}}$ has precisely $n-1$ main connectives, thereby meeting condition (iii).
The same reasoning holds of $\pardisjunction{\CAPTHETA_1}{\disjunction{\CAPTHETA_2}{\disjunction{\ldots}{\CAPTHETA_{\integer{n}}}}}$, except that it has $n-1$ \mention{$\VEE$} tokens as its main connectives.

\end{description}

\item[Closure Step:] These are all the ways of constructing \GSL{} sentences. Therefore, every \GSL{} sentences has either no main connective, one main connective, or in the case of extended conjunctions or disjunctions with $n$ conjuncts/disjuncts, $n-1$ main connective tokens.

\end{description}
\end{PROOFOF}

%%%%%%%%%%%%%%%%%%%%%%%%%%%%%%%%%%%%%%%%%%%%%%%%%%
\section[Disjunctive Normal Form]{Disjunctive Normal Form}\label{DNF and the TFE Replacement Theorem}
%%%%%%%%%%%%%%%%%%%%%%%%%%%%%%%%%%%%%%%%%%%%%%%%%%

Consider the sentences below.
Let's say we are given the truth values of their sentence letters.
Which sentences are easier to evaluate? 
Which are more difficult?
\begin{multicols}{2}
\begin{menumerate}
\item\label{dnf1} $\horseshoe{\negation{\bparhorseshoe{\Al}{\negation{\parconjunction{\Cl}{\Bl}}}}}{\parhorseshoe{\Al}{\Cl}}$
\item\label{dnf2} $\conjunction{\Al}{\conjunction{\Rl}{\conjunction{\Al}{\negation{\Rl}}}}$
\item\label{dnf3} $\negation{\cparhorseshoe{\negation{\bparhorseshoe{\Al}{\negation{\Rl}}}}{\parhorseshoe{\Al}{\Rl}}}$
\item\label{dnf4} $\disjunction{\bpardisjunction{\negation{\Al}}{\disjunction{\negation{\Cl}}{\negation{\Bl}}}}{\bpardisjunction{\negation{\Al}}{\Cl}}$
\end{menumerate}
\end{multicols}
\noindent{}Clearly \ref{dnf2} and \ref{dnf4} are simpler to figure out than \ref{dnf1} and \ref{dnf3}. 
In general, we find that the truth values of certain sentences are easier to calculate than others.
Say that we want to figure out the truth value of a difficult sentence and we know that it's \CAPS{tfe} to an easy sentence.
Then if we compute the truth value of the easy one we know the value of the difficult one.
This technique can be exploited to simplify the analysis of complicated \GSL{} sentences.
In the above list of sentences, \ref{dnf1} is TFE to \ref{dnf4} and \ref{dnf2} is TFE to \ref{dnf3}.

But how can we know whether some sentence is \CAPS{tfe} to a simpler, more transparent sentence?
It turns out that we can simplify a complicated sentence by a systematic series of steps, each of which replaces some subsentence with a simpler sentence that we know is \CAPS{tfe} to the subsentence being replaced.  

\subsection{The \CAPS{tfe} Replacement Theorem}\label{The TFE Replacement Theorem}

We want a method of transforming a complicated sentence to an equivalent but simpler one.
To that end we prove:

\begin{THEOREM}{\LnpTC{TFE Replacement} Truth Functional Equivalence Replacement:}
Let $\CAPPHI$ be a subsentence of $\CAPTHETA$, $\CAPPHI$ and $\CAPPHI^*$ be \CAPS{tfe}, and $\CAPTHETA^*$ be the result of replacing one occurrence of $\CAPPHI$ by $\CAPPHI^*$ in $\CAPTHETA$. Then $\CAPTHETA$ and $\CAPTHETA^*$ are \CAPS{tfe}.
\end{THEOREM}
% \noindent{}Before turning to the proof, recall from section \ref{Other Relations}, definition \mvref{GSL TFE} that two sentences $\CAPPHI$ and $\CAPPHI^*$ are truth functionally equivalent \Iff all models assign them the same truth value.  This is the same as saying they entail each other, i.e., that $\CAPPHI^*\sdtstile{}{}\CAPPHI$ and $\CAPPHI\sdtstile{}{}\CAPPHI^*$. 

\begin{PROOF}
\begin{description}
\item[Base Step:] Suppose that $\CAPTHETA$ is an atomic sentence. 
Then $\CAPTHETA=\CAPPHI$, since each atomic sentence has only itself as a subsentence.
It also follows that $\CAPTHETA^*=\CAPPHI^*$, since ``substitution'' is just replacement in this case.
Since $\CAPPHI$ and $\CAPPHI^*$ are \CAPS{tfe}, it follows that $\CAPTHETA^*$ and $\CAPTHETA$ are \CAPS{tfe}.

\item[Inheritance Step:] \hfill{}
\begin{description}

	\item[Recursive Assumption:] Let $\CAPPHI$ and $\CAPTHETA_1,\CAPTHETA_2,\ldots,\CAPTHETA_{\integer{n}}$ be \GSL{} sentences of order $k$ or less. And let $\CAPPHI^*$ and $\CAPTHETA_1^*,\CAPTHETA_2^*,\ldots,\CAPTHETA_{\integer{n}}^*$ be, respectively, \GSL{} sentences that are \CAPS{tfe} to them. Now consider the ways in which a substitution can be made into a sentence of order $k+1$.
	
	\item[Negation:] By RA, $\CAPPHI$ and $\CAPPHI^*$ are \CAPS{tfe}.
	Then a model $\IntA$ makes $\CAPPHI$ false \Iff it makes $\CAPPHI^*$ false.
	By the definition of truth, $\NEGATION$, $\IntA$ makes $\negation{\CAPPHI}$ true \Iff it makes $\CAPPHI$ false.
	Similarly, $\IntA$ makes $\negation{\CAPPHI^*}$ true \Iff it makes $\CAPPHI^*$ false.
	Then $\IntA$ makes $\negation{\CAPPHI}$ true \Iff it makes $\negation{\CAPPHI^*}$ true.
	Thus $\negation{\CAPPHI}$ and $\negation{\CAPPHI^*}$ are \CAPS{tfe}.

	\item[Conditional, \lhs Replacement:] ($\Rightarrow$) Assume that model $\IntA$ makes $\parhorseshoe{\CAPTHETA_1}{\CAPTHETA_2}$ true.
	Then, by the definition of truth, $\HORSESHOE$, $\m$ makes either $\CAPTHETA_1$ false or $\CAPTHETA_2$ true.
	$\CAPTHETA_1$ is to be replaced by $\CAPTHETA_1^*$.
	By RA, $\CAPTHETA_1$ is \tfe to $\CAPTHETA_1^*$.
	So $\m$ makes either $\CAPTHETA_1^*$ false or $\CAPTHETA_2$ true.
	Then $\m$ makes $\parhorseshoe{\CAPTHETA_1^*}{\CAPTHETA_2}$ true.
	By the definition of $\entails$, $\parhorseshoe{\CAPTHETA_1}{\CAPTHETA_2}\entails\parhorseshoe{\CAPTHETA_1^*}{\CAPTHETA_2}$.

	($\Leftarrow$) Assume that model $\m$ makes $\parhorseshoe{\CAPTHETA_1^*}{\CAPTHETA_2}$ true.
	Then, by the definition of truth, $\HORSESHOE$, $\m$ makes either $\CAPTHETA_1^*$ false or $\CAPTHETA_2$ true.
	By RA, $\CAPTHETA_1$ is \tfe to $\CAPTHETA_1^*$.
	So $\m$ makes either $\CAPTHETA_1$ false or $\CAPTHETA_2$ true.
	Then $\m$ makes $\parhorseshoe{\CAPTHETA_1}{\CAPTHETA_2}$ true.
	By the definition of $\entails$, $\parhorseshoe{\CAPTHETA_1^*}{\CAPTHETA_2}\entails\parhorseshoe{\CAPTHETA_1}{\CAPTHETA_2}$.

	Thus, by theorem \ref{tfe entailment}, $\parhorseshoe{\CAPTHETA_1}{\CAPTHETA_2}$ is \tfe to $\parhorseshoe{\CAPTHETA_1^*}{\CAPTHETA_2}$.
	\begin{commentary}
		Note that the ($\Leftarrow$) part is the same as the ($\Rightarrow$) part, but with $\CAPTHETA_1$ and $\CAPTHETA_1^*$ switched.
		For the rest of this proof we omit such redundant details.
	\end{commentary}

	\item[Conditional, \rhs Replacement:] ($\Rightarrow$) Assume that model $\IntA$ makes $\parhorseshoe{\CAPTHETA_1}{\CAPTHETA_2}$ true.
	Then, by the definition of truth, $\HORSESHOE$, $\m$ makes either $\CAPTHETA_1$ false or $\CAPTHETA_2$ true.
	$\CAPTHETA_2$ is to be replaced by $\CAPTHETA_2^*$.
	By RA, $\CAPTHETA_2$ is \tfe to $\CAPTHETA_2^*$.
	So $\m$ makes either $\CAPTHETA_1$ false or $\CAPTHETA_2^*$ true.
	Then $\m$ makes $\parhorseshoe{\CAPTHETA_1}{\CAPTHETA_2^*}$ true.
	By the definition of $\entails$, $\parhorseshoe{\CAPTHETA_1}{\CAPTHETA_2}\entails\parhorseshoe{\CAPTHETA_1}{\CAPTHETA_2^*}$.

	($\Leftarrow$) This is the same as ($\Rightarrow$), but with $\CAPTHETA_2$ and $\CAPTHETA_2^*$ switched.

	Thus, by theorem \ref{tfe entailment}, $\parhorseshoe{\CAPTHETA_1}{\CAPTHETA_2}$ is \tfe to $\parhorseshoe{\CAPTHETA_1}{\CAPTHETA_2^*}$.

	\item[Biconditional, \lhs Replacement:] ($\Rightarrow$) Assume that model $\IntA$ makes $\partriplebar{\CAPTHETA_1}{\CAPTHETA_2}$ true.
	Then, by the definition of truth, $\TRIPLEBAR$, $\m$ either makes $\CAPTHETA_1$ and $\CAPTHETA_2$ both true or makes $\CAPTHETA_1$ and $\CAPTHETA_2$ both false.
	$\CAPTHETA_1$ is to be replaced by $\CAPTHETA_1^*$.
	By RA, $\CAPTHETA_1$ is \tfe to $\CAPTHETA_1^*$.
	So $\m$ either makes $\CAPTHETA_1^*$ and $\CAPTHETA_2$ both true or makes $\CAPTHETA_1^*$ and $\CAPTHETA_2$ both false.
	Then $\m$ makes $\partriplebar{\CAPTHETA_1^*}{\CAPTHETA_2}$ true.
	By the definition of $\entails$, $\partriplebar{\CAPTHETA_1}{\CAPTHETA_2}\entails\partriplebar{\CAPTHETA_1^*}{\CAPTHETA_2}$.

	($\Leftarrow$) This is the same as ($\Rightarrow$), but with $\CAPTHETA_1$ and $\CAPTHETA_1^*$ switched.

	Thus, by theorem \ref{tfe entailment}, $\partriplebar{\CAPTHETA_1}{\CAPTHETA_2}$ is \tfe to $\partriplebar{\CAPTHETA_1^*}{\CAPTHETA_2}$.

	\item[Biconditional, \rhs Replacement:] The proof for this case is left for the reader.

	\item[Conjunction:] ($\Rightarrow$) Assume that model $\IntA$ makes $\parconjunction{\CAPTHETA_1}{\conjunction{\ldots}{\conjunction{\CAPTHETA_i}{\conjunction{\ldots}{\CAPTHETA_{\integer{n}}}}}}$ true, where $\CAPTHETA_i$ is the conjunct to be replaced.
	Keep in mind that $i$ could also be $1$ or $n$, so $1\leq i\leq n$.
	Then, by the defition of truth, $\WEDGE$, each of $\CAPTHETA_1,\ldots,\CAPTHETA_i,\ldots,\CAPTHETA_n$ is true on $\m$.
	By RA, $\CAPTHETA_i$ is \tfe to $\CAPTHETA_i^*$.
	So, $\CAPTHETA_1,\ldots,\CAPTHETA_i^*,\ldots,\CAPTHETA_n$ is true on $\m$.
	Then $\m$ makes $\parconjunction{\CAPTHETA_1}{\conjunction{\ldots}{\conjunction{\CAPTHETA_i^*}{\conjunction{\ldots}{\CAPTHETA_{\integer{n}}}}}}$ true.
	By the definition of $\entails$, $\parconjunction{\CAPTHETA_1}{\conjunction{\ldots}{\conjunction{\CAPTHETA_i}{\conjunction{\ldots}{\CAPTHETA_{\integer{n}}}}}}\entails\parconjunction{\CAPTHETA_1}{\conjunction{\ldots}{\conjunction{\CAPTHETA_i^*}{\conjunction{\ldots}{\CAPTHETA_{\integer{n}}}}}}$.

	($\Leftarrow$) This is the same as ($\Rightarrow$), but with $\CAPTHETA_i$ and $\CAPTHETA_i^*$ switched.

	Thus, by theorem \ref{tfe entailment}, $\parconjunction{\CAPTHETA_1}{\conjunction{\ldots}{\conjunction{\CAPTHETA_i}{\conjunction{\ldots}{\CAPTHETA_{\integer{n}}}}}}$ and $\parconjunction{\CAPTHETA_1}{\conjunction{\ldots}{\conjunction{\CAPTHETA_i^*}{\conjunction{\ldots}{\CAPTHETA_{\integer{n}}}}}}$ are \tfe.
	
	\item[Disjunction:] The proof for this case is left for the reader.
\end{description}
\item[Closure Step:] Those are the only ways \GSL{} sentences can be formed; hence the theorem is proved.
\end{description}
\end{PROOF}
\begin{majorILnc}{\LnpEC{TFE Replacement Example}}
We can use this theorem to show that \ref{dnf1} and \ref{dnf4} above are equivalent, as are \ref{dnf2} and \ref{dnf3}. 
Consider \ref{dnf1} and \ref{dnf4}. 
(We leave \ref{dnf2} and \ref{dnf3} to the reader.)
For any formulas $\CAPPHI$ and $\CAPTHETA$, 
\begin{menumerate}
\item\label{dnf5} $\parhorseshoe{\CAPPHI}{\CAPTHETA}$ and $\pardisjunction{\negation{\CAPPHI}}{\CAPTHETA}$ are \CAPS{tfe} (See ex. \pmvref{TFE Ex 2})
\item\label{dnf6} $\negation{\negation{\CAPPHI}}$ and $\CAPPHI$ are \CAPS{tfe} (See ex. \pmvref{TFE Ex 3})
\item\label{dnf7} $\negation{\parconjunction{\CAPTHETA}{\CAPPHI}}$ and $\pardisjunction{\negation{\CAPTHETA}}{\negation{\CAPPHI}}$ are \CAPS{tfe} (See \ref{HW Entailment 4} and \ref{HW Entailment 12}, section \ref{Entailment Problems for GSL})
\item\label{dnf8} $\pardisjunction{\CAPPHI}{\disjunction{\CAPTHETA}{\CAPPSI}}$ and $\pardisjunction{\CAPPHI}{\pardisjunction{\CAPTHETA}{\CAPPSI}}$ are \CAPS{tfe} (Obvious)
\end{menumerate}
By making substitutions starting with \ref{dnf1} that the theorem says result in successive truth functionally equivalent sentences, we can get from \ref{dnf1} to \ref{dnf4} and thereby have shown that \ref{dnf4} is truth functionally equivalent to \ref{dnf1}.
\begin{menumerate}
\item $\horseshoe{\negation{\bparhorseshoe{\Al}{\negation{\parconjunction{\Cl}{\Bl}}}}}{\parhorseshoe{\Al}{\Cl}}$ [\ref{dnf1}]
\item $\disjunction{\negation{\negation{\bparhorseshoe{\Al}{\negation{\parconjunction{\Cl}{\Bl}}}}}}{\parhorseshoe{\Al}{\Cl}}$ [\ref{dnf5}]
\item $\disjunction{\bparhorseshoe{\Al}{\negation{\parconjunction{\Cl}{\Bl}}}}{\parhorseshoe{\Al}{\Cl}}$ [\ref{dnf6}]
\item $\disjunction{\bpardisjunction{\negation{\Al}}{\negation{\parconjunction{\Cl}{\Bl}}}}{\pardisjunction{\negation{\Al}}{\Cl}}$ [\ref{dnf5}]
\item $\disjunction{\bpardisjunction{\negation{\Al}}{\pardisjunction{\negation{\Cl}}{\negation{\Bl}}}}{\pardisjunction{\negation{\Al}}{\Cl}}$ [\ref{dnf7}]
\item $\disjunction{\bpardisjunction{\negation{\Al}}{\disjunction{\negation{\Cl}}{\negation{\Bl}}}}{\pardisjunction{\negation{\Al}}{\Cl}}$ [\ref{dnf8}]
\end{menumerate}
\end{majorILnc}

\subsection{Disjunctive Normal Form}\label{Disjunctive Normal Form}

Sentences in disjunctive normal form are especially easy to evaluate.  
\begin{majorILnc}{\LnpDC{DNF Definition}}
A \GSL{} sentence is in \df{disjunctive normal form} (\CAPS{dnf})\index{DNF|see{disjunctive normal form}} \Iff
\begin{cenumerate}
\item it contains no conditional ($\HORSESHOE$) or biconditional ($\TRIPLEBAR$),
\item negations ($\NEGATION$) only govern sentence letters, and
\item no conjunction ($\WEDGE$) contains a disjunction ($\VEE$) as a subsentence.
\end{cenumerate}
\end{majorILnc}
\noindent{}A typical example of a sentence in \CAPS{dnf} is $\disjunction{\parconjunction{\Ql}{\negation{\Rl}}}{\parconjunction{\negation{\Pl}}{\Rl}}$.  The truth conditions for this sentence are transparent.  The sentence $\disjunction{\parconjunction{\Ql}{\negation{\Rl}}}{\parconjunction{\negation{\Pl}}{\Rl}}$ is true on a model $\IntA$ when either $\IntA(\Ql)=\TrueB$ and $\IntA(\Rl)=\FalseB$, or $\IntA(\Pl)=\FalseB$ and $\IntA(\Rl)=\TrueB$.  Some less typical examples are:
\begin{menumerate}
\item $\Ql$
\item $\negation{\Rl}$
\item $\conjunction{\Ql}{\negation{\Rl}}$
\item $\disjunction{\negation{\Ql}}{\Rl}$
\end{menumerate}
\CAPS{dnf} is important because we can prove the following theorem.
\begin{THEOREM}{\LnpTC{Disjunctive Normal Form Theorem} The Disjunctive Normal Form Theorem:}
Every sentence of \GSL{} is truth functionally equivalent to an \GSL{} sentence which is in \CAPS{dnf}.
\end{THEOREM}
\begin{PROOF}
The proof relies on three lemmas, each of which can be established rigorously by recursive proof.  We leave the details of these lemmas to the reader.

Here we provide a process showing how to turn any given sentence into one that's in \CAPS{dnf}. We proceed in three stages, corresponding to the three lemmas that are necessary for a proof.
\begin{description}
\item[Step A:] If a subsentence of $\CAPPHI$ has a conditional or biconditional as its main connective, i.e., is of the form $\parhorseshoe{\CAPPSI}{\CAPTHETA}$ or $\partriplebar{\CAPTHETA}{\CAPPSI}$, replace the subsentence by $\pardisjunction{\negation{\CAPPSI}}{\CAPTHETA}$ or $\disjunction{\parconjunction{\CAPPSI}{\CAPTHETA}}{\parconjunction{\negation{\CAPPSI}}{\negation{\CAPTHETA}}}$ respectively. 
Repeat as necessary to obtain a sentence $\CAPPHI'$ without conditionals or biconditionals.
\item[Step B:] \hfill{}
\begin{cenumerate}
\item Replace any subsentence of the form $\negation{\negation{\CAPPSI}}$ in $\CAPPHI'$ with $\CAPPSI$.
\item Replace any subsentence of the form $\negation{\parconjunction{\CAPPSI}{\CAPTHETA}}$ in $\CAPPHI'$ with $\pardisjunction{\negation{\CAPPSI}}{\negation{\CAPTHETA}}$. 
\item Replace $\negation{\pardisjunction{\CAPPSI}{\CAPTHETA}}$ in $\CAPPHI'$ with $\parconjunction{\negation{\CAPPSI}}{\negation{\CAPTHETA}}$. 
%(These last two are known as DeMorgan's laws after the logician who first explicitly formulated them.)
\end{cenumerate} 
Repeat as necessary to obtain $\CAPPHI''$ in which negations govern nothing but sentence letters.
\item[Step C:] The only thing that could prevent $\CAPPHI''$ from being in \CAPS{dnf} is that some conjunctions govern some disjunctions, i.e., there is a subsentence  $\conjunction{\CAPTHETA}{\pardisjunction{\CAPPSI_1}{\disjunction{\CAPPSI_2}{\disjunction{\ldots}{\CAPPSI_{\integer{n}}}}}}$, or the reverse $\conjunction{\pardisjunction{\CAPPSI_1}{\disjunction{\CAPPSI_2}{\disjunction{\ldots}{\CAPPSI_{\integer{n}}}}}}{\CAPTHETA}$. 
Those subsentences can be replaced by the equivalent $\disjunction{\parconjunction{\CAPPSI_1}{\CAPTHETA}}{\disjunction{\parconjunction{\CAPPSI_2}{\CAPTHETA}}{\disjunction{\ldots}{\parconjunction{\CAPPSI_{\integer{n}}}{\CAPTHETA}}}}$. 
Repeat as necessary.
\end{description}
\end{PROOF}

\noindent{}A recursive proof would be more rigorous---it would have a clause for each \GSL{} connective, and would explain in each clause how to construct from each subsentence another \CAPS{tfe} subsentence that is in \CAPS{dnf}. 

\CAPS{dnf} sentences allow us see some of the advantages of formal languages.  For instance, we can construct a simple, mechanical process that will tell us when a \CAPS{dnf} sentence is \CAPS{tff}.

Let $\CAPPHI$ be some \CAPS{dnf} sentence of the form $\disjunction{\CAPTHETA_1}{\disjunction{\CAPTHETA_2}{\disjunction{\ldots}{\CAPTHETA_n}}}$.
We can see that $\CAPPHI$ is \CAPS{tff} \Iff every disjunct $\CAPTHETA_i$ is \CAPS{tff}.
Because each $\CAPTHETA_i$ is a conjunction with negated and unnegated sentence letters as the conjuncts, there is only one way that it can be \CAPS{tff}.  A $\CAPTHETA_i$ is \CAPS{tff} \Iff it has some sentence letter $\CAPPSI$ as one conjunct and $\negation{\CAPPSI}$ as another.

It follows that we can use the following process to determine whether a \CAPS{dnf} sentence $\CAPPHI$ is \CAPS{tff}. Check every disjunct of $\CAPPHI$ to see if it has some $\CAPPSI$ and $\negation{\CAPPSI}$ as conjuncts.  If so, then $\CAPPHI$ is \CAPS{tff}; otherwise it isn't.  For example, consider the following \CAPS{dnf} sentence:

\begin{center}
\noindent{}$\disjunction{\parconjunction{\Ql}{\conjunction{\Rl}{\negation{\Ql}}}}{\disjunction{\parconjunction{\negation{\Ql}}{\conjunction{\Rl}{\Rl}}}{\parconjunction{\Ol}{\conjunction{\Rl}{\negation{\Ol}}}}}$
\end{center}

\noindent{}The first disjunct, $\parconjunction{\Ql}{\conjunction{\Rl}{\negation{\Ql}}}$, is \CAPS{tff} because it has $\Ql$ and $\negation{\Ql}$ as conjuncts; the third disjunct, $\parconjunction{\Ol}{\conjunction{\Rl}{\negation{\Ol}}}$, is also \CAPS{tff}.  But the second disjunct, $\parconjunction{\negation{\Ql}}{\conjunction{\Rl}{\Rl}}$, isn't \CAPS{tff}. So, the whole sentence isn't \CAPS{tff}.

If we were to replace the second disjunct with $\parconjunction{\negation{\Ql}}{\conjunction{\Rl}{\negation{\Rl}}}$, so that the new whole sentence is:

\begin{center}
	\noindent{}$\disjunction{\parconjunction{\Ql}{\conjunction{\Rl}{\negation{\Ql}}}}{\disjunction{\parconjunction{\negation{\Ql}}{\conjunction{\Rl}{\negation{\Rl}}}}{\parconjunction{\Ol}{\conjunction{\Rl}{\negation{\Ol}}}}}$
\end{center}

\noindent{}\ldots then the result \emph{is} \CAPS{tff}, because each disjunct has a sentence letter and its negation as conjuncts.

We now provide a mechanical method that determines whether \emph{any} \GSL{} sentence $\CAPPHI$ is \CAPS{tff}.  First, we use the process in \ref{Disjunctive Normal Form Theorem} to construct an equivalent \CAPS{dnf} sentence, $\CAPPHI^*$.  Then we use the method given above to determine whether $\CAPPHI^*$ is \CAPS{tff}.  Because they are equivalent, $\CAPPHI^*$ is \CAPS{tff} \Iff $\CAPPHI$ is \CAPS{tff}.  No creativity is needed to apply this method---each individual step requires nothing more than following simple instructions.

We extend this method to determine whether any \GSL{} sentence is \CAPS{tft}.
We take advantage of the fact that if you put a negation in front of a \CAPS{tft} sentence $\CAPPHI$, the resulting sentence, $\negation{\CAPPHI}$, is \CAPS{tff}.  So, all that is necessary to see whether $\CAPPHI$ is \CAPS{tft} is to negate $\CAPPHI$, put $\negation{\CAPPHI}$ into \CAPS{dnf}, and then to see whether the final result is \CAPS{tff}.
If so, then $\CAPPHI$ is \CAPS{tft}.
If not, then it isn't.
As before, this process is entirely mechanical or \emph{formal}.
Note that, along with the truth table method in section \ref{Proceduresfortesting}, we now have two different formal methods for determining logical truth in \GSL{}.
In later chapters our methods are closer to the \CAPS{dnf} approach.

Any given sentence of \GSL{} is truth functionally equivalent to more than one \CAPS{dnf} sentence. 
A sentence has \CAPS{dnf}s that differ slightly for at least two reasons.
First, for any sentence $\CAPPHI$ and sentence letter $\CAPPSI$, if $\CAPPHI$ is in \CAPS{dnf}, then $\disjunction{\CAPPHI}{\parconjunction{\CAPPSI}{\negation{\CAPPSI}}}$ is \CAPS{tfe} to $\CAPPHI$ and also in \CAPS{dnf}.
Second, sometimes a sentence in \CAPS{dnf} can be simplified. Thus 
\begin{menumerate}
\item $\disjunction{\parconjunction{\Ql}{\conjunction{\Rl}{\Ol}}}{\disjunction{\parconjunction{\Ql}{\conjunction{\Rl}{\Nl}}}{\parconjunction{\Ql}{\conjunction{\Rl}{\negation{\Ol}}}}}$
\end{menumerate} can be simplified to the \CAPS{dnf}
\begin{samepage}
\begin{menumerate}
\item $\disjunction{\parconjunction{\Ql}{\Rl}}{\parconjunction{\Ql}{\conjunction{\Rl}{\Nl}}}$
\end{menumerate} and further to 
\begin{menumerate}
\item $\parconjunction{\Ql}{\Rl}$.
\end{menumerate}
\end{samepage}

%\bigskip
%%%%%%%%%%%%%%%%%%%%%%%%%%%%%%%%%%%%%%%%%%%%%%%%%%
\section[Truth Functional Expressiveness]{Truth Functional Expressiveness}\label{Truth Functional Expressiveness} 
%%%%%%%%%%%%%%%%%%%%%%%%%%%%%%%%%%%%%%%%%%%%%%%%%%


The definition of truth in a model (def. \pmvref{True on a GSL interpretation}) associates each of the logical connectives of \GSL{} with a truth function.\footnote{See section \ref{Truth Functions Truth Tables and Boolean Operators} for more details.}
We can think of the logical connectives of \GSL{} as truth functions---i.e., as having a meaning that's exhausted by the definition of truth. 
Do the five logical connectives of \GSL{} exhaust the range of possible logical connectives?

In one sense it's obvious they do not. 
For example, we could introduce a new connective, $\%$, choose some number of places for it, and give some clause that describes how the truth value of a sentence with $\%$ as the main connective depends on the truth value of the component parts. 
As long as this clause differs from any of those in the definition of truth, $\%$ is distinct from the five in \GSL{}.

But though the five logical connectives of \GSL{} obviously do not exhaust all the possible connectives, there's still a sense in which they might indirectly cover them all. 
Even if $\%$ is distinct from all the connectives of \GSL{}, maybe there's still some sentence schema that is truth functionally equivalent to $\%$, and that uses only (but not necessarily all of) the five connectives of \GSL{}. 
For example, say $\%$ is a 3-place connective, such that a sentence $\CAPTHETA_1$ $\%$ $\CAPTHETA_2$ $\%$ $\CAPTHETA_3$ is true on a model $\IntA$ \Iff at least two of the $\CAPTHETA$ are true on $\IntA$.  So, the sentence is true if $\CAPTHETA_1$ and $\CAPTHETA_2$ are true, or if $\CAPTHETA_1$ and $\CAPTHETA_3$ are true, or if $\CAPTHETA_2$ and $\CAPTHETA_3$ are true.  We can express this without \mention{$\%$}, using $\WEDGE$ for \mention{and} and $\VEE$ for \mention{or}:  $\disjunction{\parconjunction{\CAPTHETA_1}{\CAPTHETA_2}}{\disjunction{\parconjunction{\CAPTHETA_1}{\CAPTHETA_3}}{\parconjunction{\CAPTHETA_2}{\CAPTHETA_3}}}$.  Even though $\%$ is a connective that isn't in our language, we can use the connectives of \GSL{} to construct a truth functionally equivalent sentence. 


A logical connective $\%$ is \niidf{definable}\index{definability} in terms of some set of other logical connectives $\Delta$ \Iff the connectives in $\Delta$ can be put together to define the same truth function as $\%$. 
With this in mind, we can ask whether every logical connective is definable in terms of the five connectives of \GSL{}.
We can also ask if any of the connectives of \GSL{} are definable in terms of the others.
We answer the first of these questions with theorem \pmvref{Truth-functional Expressive Completeness of GSL}.
The second we consider now in the following examples.\footnote{See \citetext{\citealt{Post1921}, \citealt[17]{Hodges2001}}.}

\begin{majorILnc}{\LnpEC{GSL Connective ID 1}}
	Define conjunction using negation and disjunction.
\end{majorILnc}
\begin{PROOF}
	Any sentence $\conjunction{\CAPPHI_1}{\conjunction{\ldots}{\CAPPHI_{\integer{n}}}}$ is \CAPS{tfe} to the sentence $\negation{\pardisjunction{\negation{\CAPPHI_1}}{\disjunction{\ldots}{\negation{\CAPPHI_{\integer{n}}}}}}$.
\end{PROOF}

\begin{majorILnc}{\LnpEC{GSL Connective ID 2}}
	Define disjunction using negation and conjunction.
\end{majorILnc}
\begin{PROOF}
	Any sentence $\disjunction{\CAPPHI_1}{\disjunction{\ldots}{\CAPPHI_{\integer{n}}}}$ is \CAPS{tfe} to the sentence $\negation{\parconjunction{\negation{\CAPPHI_1}}{\conjunction{\ldots}{\negation{\CAPPHI_{\integer{n}}}}}}$.
\end{PROOF}

\begin{majorILnc}{\LnpEC{GSL Connective ID two and half}}
	Define conditional using negation and disjunction.
\end{majorILnc}
\begin{PROOF}
	Any sentence $\horseshoe{\CAPPHI_1}{\CAPPHI_2}$ is \CAPS{tfe} to the sentence $\disjunction{\negation{\CAPPHI_1}}{\CAPPHI_2}$.
\end{PROOF}

\begin{majorILnc}{\LnpEC{GSL Connective ID 3}}
	The pairs $\NEGATION$ and $\WEDGE$, and $\NEGATION$ and $\VEE$ are each adequate to define the remaining connectives in \GSL{}.
\end{majorILnc}
\begin{PROOF}
	By example \ref{GSL Connective ID 2}, with $\NEGATION$ and $\WEDGE$ we can define $\VEE$. 
	By example \ref{TFE Ex 2}, $\horseshoe{\CAPPHI}{\CAPTHETA}$ and $\disjunction{\negation{\CAPPHI}}{\CAPTHETA}$ are \CAPS{tfe}.
	So we can define $\HORSESHOE$ with $\NEGATION$ and $\WEDGE$. 
	As the reader can check, $\triplebar{\CAPPHI}{\CAPPSI}$ is \CAPS{tfe} to $\conjunction{\parhorseshoe{\CAPPHI}{\CAPPSI}}{\parhorseshoe{\CAPPSI}{\CAPPHI}}$.
	So we can define $\TRIPLEBAR$ with $\NEGATION$ and $\WEDGE$.
	
	We leave it to the reader to show that $\NEGATION$ and $\VEE$ are adequate to define the remaining connectives in \GSL{}.
\end{PROOF}

\noindent{}We don't need all five connectives, but for the sake of convenience we keep them all.  We also asked the following: 
Are the five operations we have enough? 
That is, are there other logical operations we can't express and should add notation for? 
It turns out that our connectives are adequate. 
There are no other logical operations we can't express.
Although it does not strictly depend on the \CAPS{dnf} theorem, the idea behind that theorem lets us prove this.
\begin{THEOREM}{\LnpTC{Truth-functional Expressive Completeness of GSL} The Truth-functional Expressive Completeness Theorem:}
Any truth-functional connective of any fixed number of arguments (ternary, quadernary, etc.) is already expressible in \GSL{}.
\end{THEOREM}
\begin{PROOF}
Any truth functional connective of a fixed number of arguments assigns $\TrueB$ or $\FalseB$ depending only on the values of the components, so it can be exactly described by a truth table. 
For example, consider the 4-place operation \% given by truth \mbox{table \ref{DNFtruthtable}} below.
\begin{table}[!ht]
\begin{center}
\begin{tabular}{ c c c c c}
$\CAPPHI_1$ & $\CAPPHI_2$ & $\CAPPHI_3$ & $\CAPPHI_4$ & $\text{\%}(\CAPPHI_1,\CAPPHI_2,\CAPPHI_3,\CAPPHI_4)$ \\
\hline
$ $ $ $ \\[-.25cm]
$\TrueB$ & $\TrueB$ & $\TrueB$ & $\TrueB$ & $\TrueB$ \\
$\TrueB$ & $\TrueB$ & $\TrueB$ & $\FalseB$&$\FalseB$ \\
$\TrueB$ & $\TrueB$ & $\FalseB$ & $\TrueB$ & $\TrueB$ \\
$\TrueB$ & $\TrueB$ & $\FalseB$ & $\FalseB$  &$\FalseB$ \\
$\TrueB$ &  $\FalseB$& $\TrueB$ & $\TrueB$	&$\FalseB$ \\
$\TrueB$ & $\FalseB$ & $\TrueB$ & $\FalseB$	& $\TrueB$ \\
$\TrueB$ &$\FalseB$  & $\FalseB$& $\TrueB$	&$\FalseB$ \\
$\TrueB$ & $\FalseB$ &$\FalseB$	& $\FalseB$	&$\FalseB$ \\
$\FalseB$	& $\TrueB$ & $\TrueB$ & $\TrueB$	& $\TrueB$ \\
$\FalseB$	& $\TrueB$ & $\TrueB$ & $\FalseB$	&$\FalseB$ \\
$\FalseB$	& $\TrueB$ & $\FalseB$&	$\TrueB$ &$\FalseB$ \\
$\FalseB$	& $\TrueB$ & $\FalseB$& $\FalseB$	&$\FalseB$ \\
$\FalseB$	& $\FalseB$	& $\TrueB$ & $\TrueB$	&$\FalseB$ \\
$\FalseB$	& $\FalseB$	& $\TrueB$ & $\FalseB$	&$\FalseB$ \\
$\FalseB$	& $\FalseB$	& $\FalseB$& $\TrueB$	& $\TrueB$ \\
$\FalseB$	& $\FalseB$& $\FalseB$& $\FalseB$	&$\FalseB$ \\
\end{tabular}
\end{center}
\caption{Truth Table for \%}
\label{DNFtruthtable}
\end{table}
We could attempt to find a complicated sentence in terms of various
connectives that would express this, but it will be better for our purposes to construct
a \CAPS{dnf} equivalent systematically. We know from the first line that the expression $\text{\%}(\CAPPHI_1,\CAPPHI_2,\CAPPHI_3,\CAPPHI_4)$ is
true when all components are, that is, if $\parconjunction{\CAPPHI_1}{\conjunction{\CAPPHI_2}{\conjunction{\CAPPHI_3}{\CAPPHI_4}}}$ is true; we know from the third line it is true when the first two, $\CAPPHI_1$ and $\CAPPHI_2$, and fourth, $\CAPPHI_4$, are true and the third, $\CAPPHI_3$, false, i.e., $\parconjunction{\CAPPHI_1}{\conjunction{\CAPPHI_2}{\conjunction{\negation{\CAPPHI_3}}{\CAPPHI_4}}}$. We also know it is true when only the first, $\CAPPHI_1$, and third, $\CAPPHI_3$, are true, i.e., $\parconjunction{\CAPPHI_1}{\conjunction{\negation{\CAPPHI_2}}{\conjunction{\CAPPHI_3}{\negation{\CAPPHI_4}}}}$, when the first, $\CAPPHI_1$, is false and the other three, $\CAPPHI_2$, $\CAPPHI_3$, and $\CAPPHI_4$, are true i.e., $\parconjunction{\negation{\CAPPHI_1}}{\conjunction{\CAPPHI_2}{\conjunction{\CAPPHI_3}{\CAPPHI_4}}}$ and when all but the fourth, $\CAPPHI_4$, are false, i.e., $\parconjunction{\negation{\CAPPHI_1}}{\conjunction{\negation{\CAPPHI_2}}{\conjunction{\negation{\CAPPHI_3}}{\CAPPHI_4}}}$. Because each of these conjunctions is true exactly when the corresponding line is true the whole sentence will be true when any one of them is true, i.e., it is equivalent to the formula: 
\begin{menumerate} 
\item $\disjunction{\parconjunction{\CAPPHI_1}{\conjunction{\CAPPHI_2}{\conjunction{\CAPPHI_3}{\CAPPHI_4}}}}{\disjunction{\parconjunction{\CAPPHI_1}{\conjunction{\CAPPHI_2}{\conjunction{\negation{\CAPPHI_3}}{\CAPPHI_4}}}}{\parconjunction{\CAPPHI_1}{\conjunction{\negation{\CAPPHI_2}}{\conjunction{\CAPPHI_3}{\negation{\CAPPHI_4}}}}}}\:\VEE$\\ $\disjunction{\parconjunction{\negation{\CAPPHI_1}}{\conjunction{\CAPPHI_2}{\conjunction{\CAPPHI_3}{\CAPPHI_4}}}}{\parconjunction{\negation{\CAPPHI_1}}{\conjunction{\negation{\CAPPHI_2}}{\conjunction{\negation{\CAPPHI_3}}{\CAPPHI_4}}}}$
\end{menumerate}
We could simplify this sentence further, but that is not important for our purposes. 
We now can see how to read off from any truth table for any operation on any fixed number of sentences a \CAPS{dnf} representation that is equivalent. 
Our five connectives are enough. 
\end{PROOF}

%We know that we can define conjunction using negation and disjunction (ex. \pmvref{GSL Connective ID 1}), and we can define disjunction using conjunction and negation (ex. \pmvref{GSL Connective ID 2}).
Since either of the pairs \mention{$\NEGATION$} and \mention{$\!\WEDGE\!$}, or \mention{$\NEGATION$} and \mention{$\VEE$} are adequate to define the remaining connectives in \GSL{} (see ex. \pmvref{GSL Connective ID 1}) we can see that either of those pairs is adequate to define all truth-functional connectives.
We can even improve on that though, because there are two connectives either of which would be adequate all by itself. 
One is the Sheffer stroke\index{Sheffer stroke|see{NAND}}, named after the logician who first demonstrated its properties and the symbol he used, \mention{|}, but it is sometimes called NAND.\index{NAND} 
Its definition is that $(\CAPPHI_1|\CAPPHI_2|\ldots|\CAPPHI_{\integer{n}})$ is true \Iff at least one component is false. So, $(\CAPPHI_1|\CAPPHI_1)$ is equivalent to $\negation{\CAPPHI}$. 
And $(\negation{\CAPPHI_1}|\negation{\CAPPHI_2}|\ldots|\negation{\CAPPHI_{\integer{n}}})$ is true just in case at least one component is false. That means at least one $\CAPPHI_i$ is true, which is to say that the sentence is equivalent to a disjunction. 
The other connective that is adequate by itself is NOR,\index{NOR} which is defined to be true \Iff all the components are false. It is left to the reader as an optional exercise to show how to define the other connectives using NOR. 

Thus, to get a language just as expressive as \GSL{}, we only need one logical connective (either NAND or NOR), not five. 
If we had a taste for cutting down basic symbols, we could go further and generate an infinite set of sentence letters by using just one symbol, \mention{$\Al$}, and generating new sentence letters by concatenating prime marks \mention{$'$} to it. 
Then all we need are parentheses (though there are ways to do without these too). 
Such a language is sparse, but it is just as expressive as \GSL{}.  Most people would find such a language difficult to work with, but computers love them.



%%%%%%%%%%%%%%%%%%%%%%%%%%%%%%%%%%%%%%%%%%%%%%%%%%
\section{Exercises}
%%%%%%%%%%%%%%%%%%%%%%%%%%%%%%%%%%%%%%%%%%%%%%%%%%

\notocsubsection{Recursive Definition Problems}{ex:Recursive Definitions Problems}

\begin{enumerate}
\item Although it's not framed as one, definition \mvref{Order} of order is a recursive definition. Rewrite it so that the base, generating, and closure clauses are explicit. 
\item Although we don't give a recursive definition, a recursive definition can be given for the unofficial \GSL{} sentences. (Definition \pmvref{Unofficial Sentence of GSL} is the definition we give.) Write down a recursive definition for unofficial \GSL{} sentences.
\end{enumerate}

\notocsubsection{Construction Trees}{ex:Construction Trees}
Write the construction tree for each of the following \GSL{} sentences.
\begin{multicols}{2}
\begin{enumerate}
\item $\negation{\negation{\negation{\Bl}}}$
\item $\negation{\pardisjunction{\Bl}{\parhorseshoe{\Al}{\Al}}}$
\item $\pardisjunction{\negation{\Bl}}{\parhorseshoe{\Al}{\Al}}$
\item $\parhorseshoe{\partriplebar{\parconjunction{\Al}{\Bl}}{\Al}}{\negation{\parhorseshoe{\Bl}{\Cl}}}$
\item $\parhorseshoe{\parconjunction{\partriplebar{\Al}{\Bl}}{\Al}}{\negation{\parhorseshoe{\Bl}{\Cl}}}$
\item $\parconjunction{\Pl}{\parconjunction{\Ql}{\Rl}}$
\item $\parconjunction{\parconjunction{\Pl}{\Ql}}{\Rl}$
\item $\parhorseshoe{\parconjunction{\Pl}{\negation{\Rl}}}{\negation{\Ql}}$
\item $\parconjunction{\Pl}{\parhorseshoe{\negation{\Rl}}{\negation{\Ql}}}$
\item $\negation{\pardisjunction{\parhorseshoe{\Pl}{\Ql}}{\parhorseshoe{\Pl}{\Ql}}}$
\end{enumerate}
\end{multicols}

\notocsubsection{Official and Unofficial Sentences}{ex:Official and Unofficial Sentences} 
Which of these are official sentences? Which are unofficial? Which are neither official nor unofficial sentences? If
neither, how could you make it either an official or unofficial sentence? Note: there
might be multiple different ways to make it an official or unofficial sentence. Finally, if it's a sentence (official or unofficial), then give its order and the number of subsentences in it. 
\begin{multicols}{2}
\begin{enumerate}
\item {$\parhorseshoe{\Al}{\conjunction{\Bl}{\Cl}}$}
\item {$\parhorseshoe{\Al}{\bparhorseshoe{\Bl}{\Cl}}$}
\item {$\horseshoe{\Al}{\parconjunction{\Bl}{\conjunction{\Cl}{\Bl}}}$}
\item {$\parhorseshoe{\Al}{\parconjunction{\CAPTHETA}{\Cl}}$}
\item {$\parhorseshoe{\Al}{\parconjunction{\Bl}{\disjunction{\Cl}{\Dl}}}$}
\item {$\parhorseshoe{\Al}{\parconjunction{\Zl}{\Cl}}$}
\item {$\parhorseshoe{\negation{\Al}}{\parconjunction{\Bl_{374}}{\Cl}}$}
\item {$\parhorseshoe{\Al}{\parconjunction{\Bl}{\Cl}}$}
\item {$\parnegation{\parconjunction{\Bl}{\Cl}}$}
\item {$\bparconjunction{\Bl}{\conjunction{\negation{\negation{\Ml}}}{\Dl}}$}
\end{enumerate}
\end{multicols}

\notocsubsection{Truth in a Model}{ex:GSLTruth in an Interpretation}
Consider the model $\IntA_1$ such that $\IntA_1(\Al)=\TrueB$, $\IntA_1(\Bl)=\FalseB$, $\IntA_1(\Cl)=\TrueB$, $\IntA_1(\Dl)=\FalseB$, and $\IntA_1(\El)=\TrueB$; and the model $\IntA_2$ such that $\IntA_2(\Al)=\TrueB$, $\IntA_2(\Bl)=\TrueB$, $\IntA_2(\Cl)=\FalseB$, $\IntA_2(\Dl)=\FalseB$, and $\IntA_2(\El)=\FalseB$.
Give the truth values of each of the following \GSL{} sentences on each of these two models.
\begin{multicols}{2}
\begin{enumerate}
\item $\horseshoe{\pardisjunction{\Al}{\Bl}}{\parconjunction{\Cl}{\Dl}}$
\item $\horseshoe{\pardisjunction{\Al}{\Bl}}{\parconjunction{\Cl}{\negation{\Dl}}}$
\item $\horseshoe{\parconjunction{\Cl}{\Dl}}{\pardisjunction{\Al}{\Bl}}$
\item $\conjunction{\parhorseshoe{\Al}{\Cl}}{\El}$
\item $\negation{\conjunction{\parhorseshoe{\Bl}{\El}}{\Dl}}$
\item $\disjunction{\negation{\pardisjunction{\Al}{\Bl}}}{\parconjunction{\Al}{\conjunction{\Bl}{\parhorseshoe{\El}{\El}}}}$
\item $\disjunction{\Al}{\parhorseshoe{\Bl}{\parhorseshoe{\El}{\El}}}$
\item $\disjunction{\parconjunction{\Al}{\El}}{\disjunction{\parconjunction{\negation{\Dl}}{\Cl}}{\parconjunction{\Bl}{\negation{\Al}}}}$
\end{enumerate}
\end{multicols}

\notocsubsection{\CAPS{tft}, \CAPS{tff}, and \CAPS{tfc}}{ex:TFT, TFF, and TFI}
For each of the following say whether the sentence is \CAPS{tfc}, \CAPS{tff} or \CAPS{tft}. 
If it is \CAPS{tfc}, give a model which makes the sentence $\True$ and another model which makes it $\False$. 
If it is \CAPS{tff}, justify your answer without truth tables (i.e., explain why there is no model which makes the sentence $\True$). 
If it is \CAPS{tft}, again justify your answer without truth tables (i.e., explain why every model makes the sentence $\True$).

\begin{multicols}{2}
\begin{enumerate}
\item {$\horseshoe{\parhorseshoe{\Al}{\Bl}}{\pardisjunction{\negation{\Bl}}{\negation{\Al}}}$}
\item {$\horseshoe{\parconjunction{\Al}{\Bl}}{\partriplebar{\Al}{\Bl}}$}
\item {$\horseshoe{\pardisjunction{\negation{\Al}}{\negation{\Bl}}}{\negation{\parconjunction{\Al}{\Bl}}}$}
\item {$\disjunction{\Al}{\parhorseshoe{\Al}{\Bl}}$}
\item {$\horseshoe{\negation{\pardisjunction{\Al}{\Bl}}}{\parconjunction{\negation{\Al}}{\negation{\Bl}}}$}
\item {$\horseshoe{\negation{\partriplebar{\Al}{\Bl}}}{\partriplebar{\negation{\Al}}{\Bl}}$}
\item {$\horseshoe{\parconjunction{\Al}{\pardisjunction{\Bl}{\Cl}}}{\pardisjunction{\parconjunction{\Al}{\Bl}}{\Cl}}$}
\item {$\horseshoe{\negation{\Al}}{\parhorseshoe{\Al}{\Bl}}$}
\item {$\horseshoe{\parconjunction{\negation{\Al}}{\negation{\Bl}}}{\negation{\pardisjunction{\Al}{\Bl}}}$}
\item {$\horseshoe{\Al}{\parhorseshoe{\Al}{\Bl}}$}
\item {$\horseshoe{\partriplebar{\Al}{\Bl}}{\parconjunction{\Al}{\Bl}}$}
\item {$\horseshoe{\negation{\parhorseshoe{\Al}{\Bl}}}{\Al}$}
\item {$\horseshoe{\negation{\parconjunction{\Al}{\Bl}}}{\pardisjunction{\negation{\Al}}{\negation{\Bl}}}$}
\item {$\horseshoe{\Al}{\parhorseshoe{\Bl}{\Al}}$}
\item {$\horseshoe{\parhorseshoe{\Al}{\Bl}}{\parconjunction{\Al}{\negation{\Bl}}}$}
\item {$\negation{\pardisjunction{\Al}{\parhorseshoe{\Al}{\Bl}}}$}
\end{enumerate}
\end{multicols}


\notocsubsection{Entailment Problems for \GSL{}}{Entailment Problems for GSL}
For each of the following, without using truth tables show whether the entailment holds.

\begin{commentary}
There are a number of different methods for thinking through these problems.  
Remember that an entailment means that on all $\IntA$ if the \CAPS{lhs} is $\True$ then the \CAPS{rhs} is also $\True$.  One approach is to show that making the \CAPS{lhs} $\True$ forces the \CAPS{rhs} to be $\True$.  Another is to show that making the \CAPS{rhs} $\False$ forces the \CAPS{lhs} to be $\False$.  Both are examples of arguing that it is not possible for the \CAPS{lhs} to be $\True$ and the \CAPS{rhs} $\False$.  Another method is showing that all $\IntA$ either make the \CAPS{lhs} $\False$ or the \CAPS{rhs} $\True$.  If the entailment does not hold, you can show this by providing a counterexample.
\end{commentary}
\begin{multicols}{2}
\begin{enumerate}
\item {$\negation{\Al}\sdtstile{}{}\parhorseshoe{\Al}{\Bl}$}
\item {$\parhorseshoe{\Al}{\Bl}\sdtstile{}{}\pardisjunction{\negation{\Bl}}{\negation{\Al}}$}
\item {$\parconjunction{\negation{\Al}}{\negation{\Bl}}\sdtstile{}{}\:\negation{\pardisjunction{\Al}{\Bl}}$}
\item\label{HW Entailment 4} {$\pardisjunction{\negation{\Al}}{\negation{\Bl}}\sdtstile{}{}\:\negation{\parconjunction{\Al}{\Bl}}$}
\item {$\Al\sdtstile{}{}\parhorseshoe{\Al}{\Bl}$}
\item {$\negation{\partriplebar{\Al}{\Bl}}\sdtstile{}{}\partriplebar{\negation{\Al}}{\Bl}$} 
\vfill
\item {$\parconjunction{\Al}{\pardisjunction{\Bl}{\Cl}}\sdtstile{}{}\pardisjunction{\parconjunction{\Al}{\Bl}}{\Cl}$}
\item {$\negation{\pardisjunction{\Al}{\Bl}}\sdtstile{}{}\parconjunction{\negation{\Al}}{\negation{\Bl}}$}
\item {$\Al\sdtstile{}{}\parhorseshoe{\Bl}{\Al}$}
\item {$\parconjunction{\Al}{\Bl}\sdtstile{}{}\partriplebar{\Al}{\Bl}$}
\item {$\negation{\parhorseshoe{\Al}{\Bl}}\sdtstile{}{}\parhorseshoe{\Bl}{\Al}$}
\item\label{HW Entailment 12} {$\negation{\parconjunction{\Al}{\Bl}}\sdtstile{}{}\pardisjunction{\negation{\Al}}{\negation{\Bl}}$}
\item {$\negation{\parhorseshoe{\Al}{\Bl}}\sdtstile{}{}\Al$}
\item {$\partriplebar{\Al}{\Bl}\sdtstile{}{}\parconjunction{\Al}{\Bl}$}
\end{enumerate}
\end{multicols}

%\notocsubsection{Testing for Entailment}{ex:Testing for Entailment}
%For each problem in exercise \ref{Entailment Problems for GSL}, use truth tables to test whether the entailment holds. 

\notocsubsection{More Entailment Problems for \GSL{}}{ex:More Entailment Problems for GSL} 
Show whether, for any \GSL{} sentence $\CAPPHI$, $\CAPTHETA$, and $\CAPPSI$, each of the following statements is true.
%Show whether each of the following entailments hold, for any sentences got by substituting into the schemas.
\begin{enumerate}
\item {$\sdtstile{}{}\disjunction{\parhorseshoe{\CAPPHI}{\CAPTHETA}}{\parhorseshoe{\CAPTHETA}{\CAPPHI}}$}
\item {Either $\sdtstile{}{}\parhorseshoe{\CAPPHI}{\CAPTHETA}$, or $\sdtstile{}{}\parhorseshoe{\CAPTHETA}{\CAPPHI}$}
\item {If $\sdtstile{}{}\parhorseshoe{\CAPPHI}{\CAPTHETA}$ and $\sdtstile{}{}\CAPPHI$, then $\sdtstile{}{}\CAPTHETA$}
\item {If $\sdtstile{}{}\parhorseshoe{\CAPPHI}{\CAPTHETA}$ and $\sdtstile{}{}\:\negation{\CAPPHI}$, then $\sdtstile{}{}\:\negation{\CAPTHETA}$}
\item {$\sdtstile{}{}\disjunction{\parhorseshoe{\CAPPHI}{\CAPTHETA}}{\parhorseshoe{\CAPTHETA}{\CAPPSI}}$}
\item {If $\parconjunction{\CAPPHI}{\CAPTHETA}\sdtstile{}{}\CAPPSI$, then both $\CAPPHI\sdtstile{}{}\CAPPSI$ and $\CAPTHETA\sdtstile{}{}\CAPPSI$}
\item {If $\CAPPSI\sdtstile{}{}\parconjunction{\CAPPHI}{\CAPTHETA}$, then both $\CAPPSI\sdtstile{}{}\CAPTHETA$ and $\CAPPSI\sdtstile{}{}\CAPPHI$}
\end{enumerate}

\notocsubsection{Even More Entailment Problems: Truth-preservation Lemma}{exercises:truth-preservation lemma} 
Show that, for any \GSL{} sentence $\CAPPHI$, $\CAPTHETA$, and $\CAPPSI$, each of the following entailments holds.
Showing that these entailments hold will be helpful later, since they are needed in the proof of theorems \mvref{Soundess of Basic GSD Rules} and \mvref{Soundness of Std Shortcut Applications}.
\begin{multicols}{2}
\begin{enumerate}
\item $\CAPPHI\sdtstile{}{}\CAPPHI$.
\item $\horseshoe{\CAPTHETA}{\CAPPSI},\CAPTHETA\sdtstile{}{}\CAPPSI$
\item $\conjunction{\CAPTHETA}{\CAPPSI}\sdtstile{}{}\CAPPSI$
\item $\conjunction{\CAPTHETA}{\CAPPSI}\sdtstile{}{}\CAPTHETA$
\item $\CAPTHETA,\CAPPSI\sdtstile{}{}\conjunction{\CAPTHETA}{\CAPPSI}$
\item $\disjunction{\CAPTHETA}{\CAPPSI},\horseshoe{\CAPTHETA}{\CAPPHI},\horseshoe{\CAPPSI}{\CAPPHI}\sdtstile{}{}\CAPPHI$
\item $\CAPTHETA\sdtstile{}{}\disjunction{\CAPTHETA}{\CAPPSI}$
\item $\horseshoe{\CAPTHETA}{\parconjunction{\CAPPSI}{\negation{\CAPPSI}}}\sdtstile{}{}\negation{\CAPTHETA}$
\item $\horseshoe{\negation{\CAPTHETA}}{\parconjunction{\CAPPSI}{\negation{\CAPPSI}}}\sdtstile{}{}\CAPTHETA$
\item $\horseshoe{\CAPTHETA}{\CAPPSI},\horseshoe{\CAPPSI}{\CAPTHETA}\sdtstile{}{}\triplebar{\CAPTHETA}{\CAPPSI}$
\item $\triplebar{\CAPTHETA}{\CAPPSI},\CAPPSI\sdtstile{}{}\CAPTHETA$
\item $\triplebar{\CAPTHETA}{\CAPPSI},\CAPTHETA\sdtstile{}{}\CAPPSI$
\item $\horseshoe{\CAPPSI}{\CAPTHETA},\negation{\CAPTHETA}\sdtstile{}{}\negation{\CAPPSI}$
\item $\disjunction{\CAPPSI}{\CAPTHETA},\negation{\CAPTHETA}\sdtstile{}{}\CAPPSI$
\item $\disjunction{\CAPTHETA}{\CAPPSI},\negation{\CAPPSI}\sdtstile{}{}\CAPTHETA$
\item $\CAPTHETA,\negation{\CAPTHETA}\sdtstile{}{}\CAPPSI$
\item $\triplebar{\CAPPSI}{\CAPTHETA}\sdtstile{}{}\triplebar{\negation{\CAPPSI}}{\negation{\CAPTHETA}}$
\end{enumerate}
\end{multicols}

\notocsubsection{Truth Functional Equivalence}{exercises:GSDTFETheorem} 
Without using truth tables, show that each of the following pairs of sentences is \CAPS{tfe}, for any sentences got by substituting into the schemas. 
Showing that these pairs are \CAPS{tfe} will be helpful later, since it's both needed for the proof of theorem \mvref{Soundness of Std Shortcut Applications} and it amounts to proving theorem \mvref{ExchangeRuleGSDSoundnessLemma} (including for \Rule{$\TRIPLEBAR$-Exchange}), which is needed to prove theorem \mvref{ExchangeRuleGSDSoundness}.
\begin{enumerate}

\item $\negation{\parconjunction{\CAPPHI_1}{\conjunction{\ldots}{\CAPPHI_{\integer{n}}}}}$, $\disjunction{\negation{\CAPPHI_1}}{\disjunction{\ldots}{\negation{\CAPPHI_{\integer{n}}}}}$

%\item $\disjunction{\negation{\CAPPHI_1}}{\disjunction{\ldots}{\negation{\CAPPHI_{\integer{n}}}}}$, $\negation{\parconjunction{\CAPPHI_1}{\conjunction{\ldots}{\CAPPHI_{\integer{n}}}}}$
 
\item $\negation{\pardisjunction{\CAPPHI_1}{\disjunction{\ldots}{\CAPPHI_{\integer{n}}}}}$, $\conjunction{\negation{\CAPPHI_1}}{\conjunction{\ldots}{\negation{\CAPPHI_{\integer{n}}}}}$ 
 
%\item $\conjunction{\negation{\CAPPHI_1}}{\conjunction{\ldots}{\negation{\CAPPHI_{\integer{n}}}}}$, $\negation{\pardisjunction{\CAPPHI_1}{\disjunction{\ldots}{\CAPPHI_{\integer{n}}}}}$ 
 
\item $\negation{\negation{\CAPPHI}}$, $\CAPPHI$

%\item $\CAPPHI$, $\negation{\negation{\CAPPHI}}$ 

\item $\horseshoe{\CAPPHI}{\CAPTHETA}$, $\disjunction{\negation{\CAPPHI}}{\CAPTHETA}$ 

%\item $\disjunction{\negation{\CAPPHI}}{\CAPTHETA}$, $\horseshoe{\CAPPHI}{\CAPTHETA}$
 
\item $\horseshoe{\CAPPHI}{\CAPTHETA}$, $\horseshoe{\negation{\CAPTHETA}}{\negation{\CAPPHI}}$ 

%\item $\horseshoe{\negation{\CAPTHETA}}{\negation{\CAPPHI}}$, $\horseshoe{\CAPPHI}{\CAPTHETA}$ 
 
\item $\negation{\parhorseshoe{\CAPPHI}{\CAPTHETA}}$, $\conjunction{\CAPPHI}{\negation{\CAPTHETA}}$

%\item $\conjunction{\CAPPHI}{\negation{\CAPTHETA}}$, $\negation{\parhorseshoe{\CAPPHI}{\CAPTHETA}}$
 
\item $\conjunction{\CAPTHETA}{\pardisjunction{\CAPPHI_1}{\disjunction{\ldots}{\CAPPHI_{\integer{n}}}}}$, $\disjunction{\parconjunction{\CAPTHETA}{\CAPPHI_1}}{\disjunction{\ldots}{\parconjunction{\CAPTHETA}{\CAPPHI_{\integer{n}}}}}$

%\item $\disjunction{\parconjunction{\CAPTHETA}{\CAPPHI_1}}{\disjunction{\ldots}{\parconjunction{\CAPTHETA}{\CAPPHI_{\integer{n}}}}}$, $\conjunction{\CAPTHETA}{\pardisjunction{\CAPPHI_1}{\disjunction{\ldots}{\CAPPHI_{\integer{n}}}}}$
 

\item $\conjunction{\pardisjunction{\CAPPHI_1}{\disjunction{\ldots}{\CAPPHI_{\integer{n}}}}}{\CAPTHETA}$, $\disjunction{\parconjunction{\CAPPHI_1}{\CAPTHETA}}{\disjunction{\ldots}{\parconjunction{\CAPPHI_{\integer{n}}}{\CAPTHETA}}}$
 
%\item $\disjunction{\parconjunction{\CAPPHI_1}{\CAPTHETA}}{\disjunction{\ldots}{\parconjunction{\CAPPHI_{\integer{n}}}{\CAPTHETA}}}$, $\conjunction{\pardisjunction{\CAPPHI_1}{\disjunction{\ldots}{\CAPPHI_{\integer{n}}}}}{\CAPTHETA}$
 
 
\item $\disjunction{\CAPTHETA}{\parconjunction{\CAPPHI_1}{\conjunction{\ldots}{\CAPPHI_{\integer{n}}}}}$, $\conjunction{\pardisjunction{\CAPTHETA}{\CAPPHI_1}}{\conjunction{\ldots}{\pardisjunction{\CAPTHETA}{\CAPPHI_{\integer{n}}}}}$
 
%\item $\conjunction{\pardisjunction{\CAPTHETA}{\CAPPHI_1}}{\conjunction{\ldots}{\pardisjunction{\CAPTHETA}{\CAPPHI_{\integer{n}}}}}$, $\disjunction{\CAPTHETA}{\parconjunction{\CAPPHI_1}{\conjunction{\ldots}{\CAPPHI_{\integer{n}}}}}$

\item $\disjunction{\parconjunction{\CAPPHI_1}{\conjunction{\ldots}{\CAPPHI_{\integer{n}}}}}{\CAPTHETA}$, $\conjunction{\pardisjunction{\CAPPHI_1}{\CAPTHETA}}{\conjunction{\ldots}{\pardisjunction{\CAPPHI_{\integer{n}}}{\CAPTHETA}}}$

%\item $\conjunction{\pardisjunction{\CAPPHI_1}{\CAPTHETA}}{\conjunction{\ldots}{\pardisjunction{\CAPPHI_{\integer{n}}}{\CAPTHETA}}}$, $\disjunction{\parconjunction{\CAPPHI_1}{\conjunction{\ldots}{\CAPPHI_{\integer{n}}}}}{\CAPTHETA}$

\item $\triplebar{\CAPTHETA}{\CAPPSI}$, $\disjunction{\parconjunction{\CAPTHETA}{\CAPPSI}}{\parconjunction{\negation{\CAPTHETA}}{\negation{\CAPPSI}}}$

\end{enumerate}

%\notocsubsection{Testing for Truth Functional Equivalence}{exercises:TestingGSDTFETheorem} 
%Use truth tables to show that the sentences in each of the pairs given in exercise \ref{exercises:GSDTFETheorem} are \CAPS{tfe}. 

\notocsubsection{Relations Between \GSL{} Sentences}{ex:Relations Between GSL Sentences}
What relations hold among these sentences? Specifically, say whether they are contradictory, contrary, subcontrary, independent, or truth functionally equivalent. You need to supply 30 answers for each problem: for each sentence $\CAPPHI$, you must determine for each of the 5 relations whether it holds between $\CAPPHI$ and each of the 6 other sentences.
\begin{multicols}{2}
\begin{enumerate}
\item {$\Al$}
\item {$\conjunction{\Al}{\Bl}$}
\item {$\conjunction{\negation{\Al}}{\Bl}$}
\item {$\horseshoe{\Al}{\Cl}$}
\item {$\horseshoe{\Al}{\negation{\Cl}}$}
\item {$\disjunction{\parconjunction{\Al}{\Bl}}{\Cl}$}
\end{enumerate}
\end{multicols}
\begin{enumerate}[start=7]
\item {$\conjunction{\Dl}{\negation{\Dl}}$}
\end{enumerate}

%\notocsubsection{Testing for Relations Between \GSL{} Sentences}{ex:Testing Relations Between GSL Sentences}
%Write a single joint truth table for the 7 sentences listed in exercise \ref{ex:Relations Between GSL Sentences}. 
%Read off from that table, for each pair of sentences, whether they are contradictory, contrary, subcontrary, independent, or truth functionally equivalent. 

\notocsubsection{Recursive Proofs}{ex:Recursive Proofs}

\begin{enumerate}
	\item Prove that all positive multiples of 10 are also multiples of 5.
	\item For each even positive integer $n$, prove that dividing $n$ by 2 results in another positive integer.
\end{enumerate}


\notocsubsection{SL Recursive Proofs}{ex:SL Recursive Proofs} 
Prove each of the following claims using a recursive proof. 
\begin{enumerate}
\item In every \GSL{} sentence which is \CAPS{tff}, there is a subsentence which is \CAPS{tfc}.
\begin{commentary}
	\emph{Hint:} The idea behind this proof is easy. Don't over-think it. 	
\end{commentary}

\item Show that every \CAPS{tff} sentence of \GSL{} contains at least one \mention{$\NEGATION$}.
\begin{commentary}
	\emph{Hint:}
	To prove this, it is useful to prove a more specific statement, namely that if $\CAPPHI$ is an \GSL{} sentence that does not contain any negations then $\CAPPHI$ is $\True$ on the model that assigns $\True$ to all sentence letters.
\end{commentary}

\item For every sentence $\CAPPHI$ of \GSL{}, the number of left parentheses occurring in $\CAPPHI$ is less than the number of subsentences. In other words, if LP$\CAPPHI$ is the number of left parentheses in $\CAPPHI$ and SS$\CAPPHI$ is the number of subsentences, then LP$\CAPPHI$ $<$ SS$\CAPPHI$.
\item The number of subsentences in any official \GSL{} sentence $\CAPPHI$ is equal to: the number of tokens of sentence letters in $\CAPPHI$ plus the number of tokens of negation in $\CAPPHI$ plus the number of tokens of left parentheses in $\CAPPHI$.
\item For every sentence $\CAPPHI$ of \GSL{}, there is a \CAPS{tfe} sentence $\CAPPHI'$ without conditionals or biconditionals.
\end{enumerate}

\notocsubsection{DNF}{ex:DNF} 
Put the following into disjunctive normal form.
\begin{multicols}{2}
\begin{enumerate}
\item {$\disjunction{\negation{\parhorseshoe{\Ql}{\Rl}}}{\parhorseshoe{\Ql}{\negation{\Rl}}}$}
\item {$\conjunction{\Ol}{\parhorseshoe{\Ol}{\Ql}}$}
\item {$\horseshoe{\bparhorseshoe{\parhorseshoe{\Ql}{\Rl}}{\Ql}}{\Ql}$}
\item {$\conjunction{\negation{\parhorseshoe{\Ql}{\Rl}}}{\pardisjunction{\Ol}{\Pl}}$}
\end{enumerate}
\end{multicols}


%\theendnotes



%%%%%%%%%%%%%%%%%%%%%%%%%%%%%%%%%%%%%%%%%%%%%%%%%%
\chapter{Quantifier Language I}\label{quantifierlogic1}
%%%%%%%%%%%%%%%%%%%%%%%%%%%%%%%%%%%%%%%%%%%%%%%%%%
% \AddToShipoutPicture*{\BackgroundPicB}

%%%%%%%%%%%%%%%%%%%%%%%%%%%%%%%%%%%%%%%%%%%%%%%%%%
\section{The Language \GQL{}1}
%%%%%%%%%%%%%%%%%%%%%%%%%%%%%%%%%%%%%%%%%%%%%%%%%%

%\setcounter{DefThm}{0}

\subsection{Sentences of \GQL{}1}\label{Sec:GQLSymbols1}
\GSL{} allows us to investigate certain aspects of logical consequence, but leaves out a great deal of interest.  The \mention{atoms} of \GSL{} are sentence letters, which are each assigned either $\TrueB$ or $\FalseB$ by a model.  Sentence letters can only represent declarative sentences of the English language, which means that \GSL{} is too coarse-grained to capture the logical meaning of certain \emph{parts} of a sentence.  For example:

\begin{RESTARTmenumerate}
\item All women are mortal.
\item Ophelia is a woman.

Therefore,

\item Ophelia is mortal.
\end{RESTARTmenumerate}

\noindent{}The third sentence is a logical consequence of the first two, but we cannot use \GSL{} to represent this as an entailment.  Let sentence letter $\Al$ stand for \mention{All women are mortal,} let $\Bl$ stand for \mention{Ophelia is a woman,} and let $\Cl$ stand for \mention{Ophelia is mortal}.  
The entailment claim $\Al, \Bl \:\sdtstile{}{}\: \Cl$ does not hold, because there is a model $\IntA$ that assigns $\TrueB$ to $\Al$ and $\Bl$, but assigns $\FalseB$ to $\Cl$.  We need a formal language more expressive than \GSL{}.

This new language needs symbols that represent named objects, including people like Ophelia.  It also needs symbols that stand for the predicates of English.  Predicates correspond roughly to what you get if you take an English sentence and remove the subject, leaving a blank, e.g.: 

\begin{menumerate}
	\item \mention{John is tall} $\Rightarrow$ \mention{\_\_\_\_\_\_ is tall}
	\item \mention{Ophelia is a woman} $\Rightarrow$ \mention{\_\_\_\_\_\_ is a woman}
	\item \mention{Ophelia is mortal} $\Rightarrow$ \mention{\_\_\_\_\_\_ is mortal}	
\end{menumerate}

\noindent{}To apply a predicate without invoking a name we use a variable, which functions somewhat like a pronoun in English.  And to account for the word \mention{all} in our new language, we use a \emph{quantifier}.  (We discuss quantifiers a bit later.)  In this chapter we outline a formal language with these features, and thus which can capture the kind of logical consequence exhibited above.

Many logic texts call the following language \PL{}, for \idf{predicate language}. 
We use \mention{\QL{}} for \idf{quantifier language}, because the quantifiers are more important than the predicates. However, before turning to the full language of \GQL{} (in the next chapter), we first consider a simpler sublanguage which we'll call \mention{\GQL{}1}.  We call it \mention{\GQL{}1} because the predicates will all be 1-place.\footnote{In \GQL{} there will be many-place predicates.  The basics of a less formal version of \GQL{}1 were developed by Aristotle over 2,000 years before Gottlob Frege and others developed the full language we're calling \GQL{}.  \GQL{} was a big step for mankind.}
 
\GQL{}1 has all the basic symbols of \GSL{}, plus a few more. 
\begin{majorILnc}{\LnpDC{Symbols of GQL1}}
The \df{basic symbols} of \GQL{}1 are:
\begin{cenumerate}
\item Logical Connectives: those of \GSL{}, plus $\forall$ and $\exists$
\item Punctuation Symbols: those of \GSL{}
\item Sentence Letters: those of \GSL{}
\item Individual Constants: $\constant{a}$, $\constant{b}$, $\constant{c}$, $\constant{d}$, $\ldots$, $\constant{p}$, $\constant{a}_1$, $\constant{b}_1$, $\constant{c}_1$, $\ldots$, $\constant{p}_1$, $\constant{a}_2$, $\ldots$
\item Individual Variables:\index{variables!individual (GQL1)|textbf} $\variable{u}$, $\variable{v}$, $\variable{w}$, $\variable{x}$, $\variable{y}$, $\variable{z}$, $\variable{u}_1$, $\variable{v}_1$, $\ldots$, $\variable{z}_1$, $\variable{u}_2$, $\ldots$
\item 1-Place Predicates: $\Ap{'}$, $\Bp{'}$, $\ldots$, $\Tp{'}$, $\Ap{'}_1$, $\Bp{'}_1$, $\ldots$, $\Tp{'}_1$, $\Ap{'}_2$, $\Bp{'}_2$, $\ldots$
\end{cenumerate}
\end{majorILnc}
\noindent{}The most prominent additions are the new logical connectives, the two quantifiers.  The first, \mention{$\forall$}, is called the \idf{universal quantifier}. 
It corresponds to the words \mention{all} or \mention{every} in English. 
The second, \mention{$\exists$}, is called the \idf{existential quantifier}. 
It corresponds to \mention{there exists}, \mention{there is}, or \mention{some}, as in \mention{Some elephants live a long time.}\footnote{Although Frege, Peirce, and Mitchell first introduced quantifiers, the notation used here comes from Russell, who Church \citeyearpar[288]{Church1956} says modified Peano's notation.}

The individual constants of \GQL{} correspond roughly to names in English. 
They are lowercase Roman letters that start at \mention{$\constant{a}$} and stop at \mention{$\constant{p}$}, and then they start at \mention{$\constant{a}$} again with subscripted integers. So we have, for example, $\constant{a_1}$, $\constant{a_2}$, $\constant{a_3}$, and so on. 

Next, the individual variables correspond roughly to pronouns in English. 
They are Roman lowercase letters that go from \mention{$\variable{u}$} to \mention{$\variable{z}$} and then start at \mention{$\variable{u}$} again with subscripted positive integers, e.g. \mention{$\variable{u}_1$}. 

We also have 1-place predicates in \GQL{}1. The one-place predicates are capital Roman letters going from \mention{$\Ap{'}$} to \mention{$\Tp{'}$} and then starting from \mention{$\Ap{'}$} again with subscripted integers, e.g. \mention{$\Ap{'}_{1}$}.  There are an infinite number of each of the individual constants, variables, and 1-place predicates. 
This is so we never run out of \GQL{}1 symbols when analyzing some sentence or argument, no matter how complex or long it is.

\subsection{Formulas of \GQL{}1}\label{Formulas of GQL1}
Before we can define \emph{sentences} of \GQL{}1 we must first define a larger set of strings called \mention{formulas.}\index{formulas} 
But before that we must introduce another kind of metavariable to MathEnglish, one that ranges over for \GQL{}1 individual variables.\index{variables!MathEnglish}\footnote{MathEnglish variables are symbols of the metalanguage while individual variables of \GQL{}1 are symbols of the object language.} 
We use lowercase Greek letters to stand for variables of \GQL{}1, usually but not necessarily always from the beginning of the Greek alphabet (e.g., \mention{$\ALPHA$} and \mention{$\BETA$}). Although we also used lowercase Greek letters as variables for \GSL{} sentences and will continue to do so for \GQL{}1 sentences and formulas, confusion shouldn't arise as we typically use \mention{$\CAPPHI$}, \mention{$\CAPPSI$}, and \mention{$\CAPTHETA$} for \GSL{} and \GQL{}1 sentences and use \mention{$\ALPHA$} and \mention{$\BETA$} for \GQL{}1 variables. 
\begin{majorILnc}{\LnpDC{Definition of Formula of GQL1}} The \nidf{formulas} \underdf{of \GQL{}1}{formulas} are given by the following recursive definition:
\begin{description}
\item[Base Clauses:] \hfill{}
\begin{cenumerate}
\item A sentence letter is a formula.
\item\label{atomic pred} A 1-place predicate followed by one individual constant or variable is a formula.
\end{cenumerate}
\item[Generating Clauses:] \hfill{}
\begin{cenumerate}
\item If $\CAPPHI$ is a formula then so is $\negation{\CAPPHI}$.
\item If $\CAPPHI$ and $\CAPTHETA$ are formulas then so are $\parhorseshoe{\CAPPHI}{\CAPTHETA}$ and $\partriplebar{\CAPPHI}{\CAPTHETA}$.
\item\label{GQL conj disj} If all of $\CAPPHI_1,\CAPPHI_2,\CAPPHI_3,\CAPPHI_4,\ldots,\CAPPHI_{\integer{n}}$ are formulas (where $n$ is an integer $\geq 2$) then so are $\parconjunction{\CAPPHI_1}{\conjunction{\CAPPHI_2}{\conjunction{\CAPPHI_3}{\conjunction{\CAPPHI_4}{\conjunction{\ldots}{\CAPPHI_{\integer{n}}}}}}}$ and $\pardisjunction{\CAPPHI_1}{\disjunction{\CAPPHI_2}{\disjunction{\CAPPHI_3}{\disjunction{\CAPPHI_4}{\disjunction{\ldots}{\CAPPHI_{\integer{n}}}}}}}$.
\item\label{GQL quant} If $\ALPHA$ is a \GQL{}1 variable and $\CAPPHI$ is a formula that does not contain an expression of the form $\universal{\ALPHA}$ or $\existential{\ALPHA}$, then $\universal{\ALPHA}\CAPPHI$ and $\existential{\ALPHA}\CAPPHI$ are formulas.
\end{cenumerate}
\item[Closure Clause:] A string of symbols is a formula only if it can be generated by the clauses above.
\end{description}
\end{majorILnc}
\noindent{}For example, $\App{'}{\constant{b}}$ is a formula, and so is $\Dpp{'}{\variable{x}_4}$. Each of these formulas is atomic.  Formulas of the form $\universal{\ALPHA}\CAPPHI$ are called \underidf{universal}{formulas} formulas and formulas of the form $\existential{\ALPHA}\CAPPHI$ are called \underidf{existential}{formulas}. 
Here are some additional examples of \GQL{}1 formulas. 
\begin{multicols}{2}
\begin{enumerate}
\item $\universal{\variable{x}}\Jpp{'}{\variable{x}}$ 
\item $\negation{\existential{\variable{y}}\Kpp{'}{\variable{x}}}$ 
%\item $\universal{\variable{y}}\Gppp{''}{\variable{x}}{\variable{y}}$ 
%\item $\existential{\variable{y}}\Gppp{''}{\variable{x}}{\variable{y}}$ 
%\item $\universal{\variable{z}}\Gppp{''}{\variable{x}}{\variable{y}}$ 
\item $\existential{\variable{z}}\Lpp{'_{12}}{\constant{b}}$
\item $\existential{\variable{y}}\negation{\universal{\variable{x}}\Gpp{'}{\variable{y}}}$ 
%\item $\existential{\variable{x}}\universal{\variable{y}}\Gppp{''}{\variable{x}}{\variable{y}}$ 
%\item $\universal{\variable{x}}\universal{\variable{y}}\Gppp{''}{\variable{x}}{\variable{y}}$ 
\item $\universal{\variable{x}}\existential{\variable{y}}\Hpp{'}{\variable{z}}$ 
\item $\parhorseshoe{\universal{\variable{x}}\universal{\variable{z}}\Ppp{'}{\variable{z}_{176}}}{\universal{\variable{y}}\Gpp{'}{\variable{y}}}$ 
%\item $\universal{\variable{x}}\existential{\variable{z}}\Gppp{''}{\variable{x}}{\variable{y}}$
\end{enumerate}
\end{multicols}
\noindent{}Contrarily, $\universal{\variable{x}}\universal{\x}\Gpp{'}{\variable{x}}$ is \emph{not} a formula.  That's because it's of the form $\universal{\variable{x}}\CAPPHI$ where $\CAPPHI$ is a formula that contains the expression $\universal{\variable{x}}$.\footnote{Continuing the practice started in section \ref{use mention comment}, we do not always put symbols, expressions, and sentences of \GQL{}1 that are mentioned (instead of used) in single quotes. 
For example, the tokens of the universal and existential quantifiers in definition \mvref{Bound Variable} should, strictly speaking, be in quotes because they \emph{mention} the symbols. 
Being stringent in the use of single quotes---and overly sensitive to the \distinction{use}{mention} distinction in general---can cloud what are relatively clear and straightforward concepts.
Wherever it may be helpful, we provide footnotes with more rigorous and detailed explanation.} Neither is $\universal{\constant{a}}\Gp{'}\constant{a}$, because $\forall$ \emph{must} be paired with a variable, but \mention{$\constant{a}$} is a constant.

Finally, we have unofficial formulas, just as we had unofficial sentences in \GSL{} (compare with def. \pmvref{Unofficial Sentence of GSL}).
\begin{majorILnc}{\LnpDC{Unofficial Formula of GQL1}}
A string of symbols is an \nidf{unofficial} formula\index{formulas!unofficial|textbf} \Iff we can obtain it from an official formula by
\begin{cenumerate}
\item deleting outer parentheses,
\item replacing official parentheses ( ) with square brackets [ ] or curly brackets \{ \}, or
\item omitting primes $'$ on a predicate letter.
\end{cenumerate}
\end{majorILnc}
\noindent{}From an unofficial formula we can unambiguously reconstruct the corresponding official formula.
\subsection{Other Properties of Formulas}\label{Other Properties of Formulas1} 
As in section \ref{Other Properties of GSL Sentences} for sentences of \GSL{}, we define the concepts of subformula, order, main connective, and construction tree for formulas of \GQL{}1. 
\begin{majorILnc}{\LnpDC{GQL Subformulas}}
The following clauses define when one formula is a \df{subformula} of another:
\begin{cenumerate}
	\item Every formula is a subformula of itself.
	\item $\CAPPHI$ is a subformula of $\negation{\CAPPHI}$.
	\item $\CAPPHI$ and $\CAPTHETA$ are subformulas of $\parhorseshoe{\CAPPHI}{\CAPTHETA}$ and $\partriplebar{\CAPPHI}{\CAPTHETA}$.
	\item Each of $\CAPPHI_1,\CAPPHI_2,\ldots,\CAPPHI_{\integer{n}}$ is a subformula of $\parconjunction{\CAPPHI_1}{\conjunction{\CAPPHI_2}{\conjunction{\ldots}{\CAPPHI_{\integer{n}}}}}$\\ and $\pardisjunction{\CAPPHI_1}{\disjunction{\CAPPHI_2}{\disjunction{\ldots}{\CAPPHI_{\integer{n}}}}}$.
	\item $\CAPPHI$ is a subformula of $\universal{\ALPHA}\CAPPHI$ and $\existential{\ALPHA}\CAPPHI$.
	\item (Transitivity) If $\CAPPHI$ is a subformula of $\CAPTHETA$ and $\CAPTHETA$ is a subformula of $\CAPPSI$, then $\CAPPHI$ is a subformula of $\CAPPSI$.
	\item That's all. 
\end{cenumerate}

\end{majorILnc}
\noindent{}Importantly, the quantifier phrase is \emph{not} a subformula. 
Thus, $\universal{\variable{x}}\universal{\variable{z}}\Gpp{'}{\variable{x}}$ has three subformulas: $\universal{\variable{x}}\universal{\variable{z}}\Gpp{'}{\variable{x}}$, $\universal{\variable{z}}\Gpp{'}{\variable{x}}$, and $\Gpp{'}{\variable{x}}$. 
Neither $\universal{\variable{x}}$ nor $\universal{\variable{x}}\universal{\variable{z}}$ is a subformula.
Middle parts of the formula cannot be removed to form a subformula. So $\universal{\variable{x}}\Gpp{'}{\variable{x}}$ isn't one either.
\begin{majorILnc}{\LnpDC{GQL Order}}
The \df{order} of a formula is defined parallel to that of an \GSL{} sentence (see def. \pmvref{Order}) with extra clauses specifying that adding a quantifier increases the order by one.
The following clauses define the \df{order} of a formula. Let $\ORD{\CAPPHI}$ be the order of $\CAPPHI$. Then: 
\begin{cenumerate}
\item If $\CAPPHI$ is a sentence letter then $\ORD{\CAPPHI}=1$.
\item If $\Ppp{'}{}$ is a predicate letter and $\ALPHA$ is a constant or variable then $\ORD{\Ppp{'}{\ALPHA}}=1$.
\item For any formula $\CAPPHI$, $\ORD{\negation{\CAPPHI}}=\ORD{\CAPPHI}+1$.
\item For any formulas $\CAPPHI$ and $\CAPTHETA$, $\ORD{\parhorseshoe{\CAPPHI}{\CAPTHETA}}$ is one greater than the max of $\ORD{\CAPPHI}$ and $\ORD{\CAPTHETA}$. Likewise, $\ORD{\partriplebar{\CAPPHI}{\CAPTHETA}}$ is one greater than the max of $\ORD{\CAPPHI}$ and $\ORD{\CAPTHETA}$.
\item For any formulas $\CAPPHI_1,\ldots,\CAPPHI_\integer{n}$, $\ORD{\parconjunction{\CAPPHI_1}{\conjunction{\ldots}{\CAPPHI_\integer{n}}}}$ is one greater than the max of $\ORD{\CAPPHI_1}$, $\ldots$, $\ORD{\CAPPHI_\integer{n}}$.
\item For any formulas $\CAPPHI_1,\ldots,\CAPPHI_\integer{n}$, $\ORD{\pardisjunction{\CAPPHI_1}{\disjunction{\ldots}{\CAPPHI_\integer{n}}}}$ is one greater than the max of $\ORD{\CAPPHI_1}$, $\ldots$, $\ORD{\CAPPHI_\integer{n}}$. 
\item For formulas $\universal{\ALPHA}\CAPPHI$ and $\existential{\ALPHA}\CAPPHI$, $\ORD{\universal{\ALPHA}\CAPPHI}=\ORD{\existential{\ALPHA}\CAPPHI}=\ORD{\CAPPHI}+1$.
\item That's all.
\end{cenumerate}
\end{majorILnc}
\begin{majorILnc}{\LnpDC{GQL Main Connective}}
The \nidf{main connective}\index{main connective!of GQL|textbf} of a formula is the connective token (or tokens) that occur(s) in the formula but in no proper subformula.
\end{majorILnc}
\begin{majorILnc}{\LnpDC{GQL Construction Tree}}
The \df{construction tree} of a formula is defined as with \GSL{} (def. \pmvref{Construction Tree}) but with the obvious extension for quantifiers. 
The order of a formula is the height of the construction tree's longest branch, measured by counting nodes.  Each node in the tree (including the bottom node) is a subformula, and the main connective is the connective added at the very bottom of the tree. 
\end{majorILnc}
\begin{majorILnc}{\LnpEC{GQL1SubformulaPropertiesExampleA}}
Consider the (unofficial) formula $\horseshoe{\universal{\variable{x}}\Hp{'\variable{x}}}{\universal{\variable{x}}\Gp{'\variable{x}}}$. 
Its construction tree is:
\begin{center}
	\begin{tikzpicture}[grow=up]
	\tikzset{level distance=50pt}
	\tikzset{sibling distance=40pt}
	\tikzset{every tree node/.style={align=center,anchor=north}}
	\Tree%http://angasm.org/papers/qtree/    http://www.ling.upenn.edu/advice/latex/qtree/qtreenotes.pdf
	[.{$\horseshoe{\universal{\variable{x}}\Hp{'\variable{x}}}{\universal{\variable{x}}\Gp{'\variable{x}}}$}
	[.{$\universal{\variable{x}}\Gp{'\variable{x}}$} %!{\qsetw{3in}} 
	  [.{$\Gp{'\variable{x}}$}
	  ]
	]
	[.{$\universal{\variable{x}}\Hp{'\variable{x}}$}
	  [.{$\Hp{'\variable{x}}$}
	  ]
	]
	]%
	%\caption{Example formula tree}
	%\label{fig:ExampleFormulaTree}
	\end{tikzpicture}
\end{center}
The height of the construction tree is $3$, so that's the order of the formula.  It has five subformulas:
\begin{enumerate}[label=(\arabic*), leftmargin=1.85\parindent,
labelindent=.35\parindent, labelsep=*, itemsep=0pt]%,start=1
\item $\horseshoe{\universal{\variable{x}}\Hp{'\variable{x}}}{\universal{\variable{x}}\Gp{'\variable{x}}}$
\end{enumerate}
\vspace*{-.5cm}
\begin{multicols}{2}
\begin{enumerate}[label=(\arabic*), leftmargin=1.85\parindent,
labelindent=.35\parindent, labelsep=*, itemsep=0pt, start=2]
\item $\universal{\variable{x}}\Hp{'\variable{x}}$
\item $\universal{\variable{x}}\Gp{'\variable{x}}$
\item $\Hp{'\variable{x}}$
\item $\Gp{'\variable{x}}$
\end{enumerate}
\end{multicols}
\end{majorILnc}
\begin{majorILnc}{\LnpEC{GQL1SubformulaPropertiesExampleB}}
Next consider the formula $\universal{\variable{x}}\parhorseshoe{\Hp{'\variable{x}}}{\Gp{'\variable{x}}}$.
It is not to be confused with the formula from example \ref{GQL1SubformulaPropertiesExampleA}. 
Consider carefully the differences between the two.
The formula in \ref{GQL1SubformulaPropertiesExampleA} has \mention{$\HORSESHOE$} as its main connective, but the main connective of this formula is \mention{$\forall$}.
Its construction tree is:
\begin{center}
	\begin{tikzpicture}[grow=up]
	\tikzset{level distance=50pt}
	\tikzset{sibling distance=40pt}
	\tikzset{every tree node/.style={align=center,anchor=north}}
	\Tree%http://angasm.org/papers/qtree/    http://www.ling.upenn.edu/advice/latex/qtree/qtreenotes.pdf
	[.{$\universal{\variable{x}}\parhorseshoe{\Hp{'\variable{x}}}{\Gp{'\variable{x}}}$} %!{\qsetw{3in}}
	[.{$\horseshoe{\Hp{'\variable{x}}}{\Gp{'\variable{x}}}$}
	  [.{$\Gp{'\variable{x}}$}
	  ] 
	  [.{$\Hp{'\variable{x}}$}
	  ] 
	]
	]%
	%\caption{Example formula tree}
	%\label{fig:ExampleFormulaTree}
	\end{tikzpicture}
\end{center}
The height of the tree is $3$ so the order of the formula is $3$. It has four subformulas:
\begin{multicols}{2}
\begin{cenumerate}
\item $\universal{\variable{x}}\parhorseshoe{\Hp{'\variable{x}}}{\Gp{'\variable{x}}}$
\item $\parhorseshoe{\Hp{'\variable{x}}}{\Gp{'\variable{x}}}$
\item $\Hp{'\variable{x}}$
\item $\Gp{'\variable{x}}$
\end{cenumerate}
\end{multicols}
\end{majorILnc}
\begin{majorILnc}{\LnpEC{GQL1SubformulaPropertiesExampleC}}
Consider the formula $\disjunction{\existential{\variable{x}}\parconjunction{\universal{y}\Epp{'}{\variable{y}}}{\App{'}{\variable{x}}}}{\universal{\variable{z}}\parhorseshoe{\existential{\variable{y}}\Hp{'\constant{a}}}{\Gp{'\variable{x}}}}$.
Its construction tree is:
\begin{center}
	\begin{tikzpicture}[grow=up]
	\tikzset{level distance=50pt}
	\tikzset{level 1/.style={level distance=65pt}}
	\tikzset{sibling distance=40pt}
	\tikzset{every tree node/.style={align=center,anchor=north}}
	\Tree%http://angasm.org/papers/qtree/    http://www.ling.upenn.edu/advice/latex/qtree/qtreenotes.pdf
	[.{$\disjunction{\existential{\variable{x}}\parconjunction{\universal{y}\Epp{'}{\variable{y}}}{\Ap{'\variable{x}}}}{\universal{\variable{z}}\parhorseshoe{\existential{\variable{y}}\Hp{'\constant{a}}}{\Gp{'\variable{x}}}}$}
	[.{$\universal{\variable{z}}\parhorseshoe{\existential{\variable{y}}\Hp{'\constant{a}}}{\Gp{'\variable{x}}}$}
	[.{$\horseshoe{\existential{\variable{y}}\Hp{'\constant{a}}}{\Gp{'\variable{x}}}$}
	[.{$\Gp{'\variable{x}}$}
	]    
	[.{$\existential{\variable{y}}\Hp{'\constant{a}}$}
	  [.{$\Hp{'\constant{a}}$}		
	  ]
	]
	]
	]
	[.{$\existential{\variable{x}}\parconjunction{\universal{y}\Epp{'}{\variable{y}}}{\Ap{'\variable{x}}}$} %!{\qsetw{3in}}
	[.{$\conjunction{\universal{y}\Epp{'}{\variable{y}}}{\Ap{'\variable{x}}}$}
	[.{$\Ap{'\variable{x}}$}
	]    
	[.{$\universal{y}\Epp{'}{\variable{y}}$}
	  [.{$\Epp{'}{\variable{y}}$}		
	  ]
	] 
	]
	]
	]%
	%\caption{Example formula tree}
	%\label{fig:ExampleFormulaTree}
	\end{tikzpicture}
\end{center}
The height of the construction tree is $5$, so that's the order of the formula. It has eleven subformulas:
\begin{enumerate}[label=(\arabic*), leftmargin=1.85\parindent,
labelindent=.35\parindent, labelsep=*, itemsep=0pt]%,start=1
\item $\disjunction{\existential{\variable{x}}\parconjunction{\universal{y}\Epp{'}{\variable{y}}}{\Ap{'\variable{x}}}}{\universal{\variable{z}}\parhorseshoe{\existential{\variable{y}}\Hp{'\constant{a}}}{\Gp{'\variable{x}}}}$
\end{enumerate}
\vspace*{-.5cm}
\begin{multicols}{2}
\begin{enumerate}[label=(\arabic*), leftmargin=1.85\parindent,
labelindent=.35\parindent, labelsep=*, itemsep=0pt, start=2]%,start=1
\item $\existential{\variable{x}}\parconjunction{\universal{y}\Epp{'}{\variable{y}}}{\Ap{'\variable{x}}}$
\item $\conjunction{\universal{y}\Epp{'}{\variable{y}}}{\Ap{'\variable{x}}}$
\item $\universal{y}\Epp{'}{\variable{y}}$
\item $\Ap{'\variable{x}}$
\item $\Epp{'}{\variable{y}}$
\item $\universal{\variable{z}}\parhorseshoe{\existential{\variable{y}}\Hp{'\constant{a}}}{\Gp{'\variable{x}}}$
\item $\horseshoe{\existential{\variable{y}}\Hp{'\constant{a}}}{\Gp{'\variable{x}}}$
\item $\existential{\variable{y}}\Hp{'\constant{a}}$
\item $\Gp{'\variable{x}}$
\item $\Hp{'\constant{a}}$
\end{enumerate}
\end{multicols}
\end{majorILnc}

\subsection{Sentences of \GQL{}1}\label{Sentences of GQL1} 
\GQL{}1 sentences are defined in terms of \GQL{}1 formulas, but we need a few other definitions before we can proceed: \emph{scope}, and \emph{free} vs. \emph{bound} variables.

\begin{majorILnc}{\LnpDC{Scope Definition}}
	In a formula $\existential{\alpha}\CAPPHI$ or $\universal{\alpha}\CAPPHI$, we say that $\CAPPHI$ is the \df{scope} of the quantifier $\existential{\alpha}$ or $\universal{\alpha}$. 
\end{majorILnc}
\noindent{}For example, in the formula $\universal{y}\Epp{'}{\variable{y}}$, $\Epp{'}{\variable{y}}$ is the scope of $\universal{y}$.
Note that the scope of a quantifier may be only a small subformula of a much larger whole.
Consider $\existential{\variable{x}}\parconjunction{\universal{y}\Epp{'}{\variable{y}}}{\Ap{'\variable{x}}}$.
The scope of $\universal{y}$ is just $\Epp{'}{\variable{y}}$, but the scope of $\existential{\variable{x}}$ is $\parconjunction{\universal{y}\Epp{'}{\variable{y}}}{\Ap{'\variable{x}}}$.

\begin{majorILnc}{\LnpDC{Bound Variable}}
	A variable token $\alpha$ in a formula $\CAPPHI$ is \underdf{bound}{variables} \Iff either (i) it's part of a quantifier expression, $\existential{\alpha}$ or $\universal{\alpha}$; or (ii) it occurs within the scope of a quantifier expression with variable $\alpha$. 
\end{majorILnc}

\begin{majorILnc}{\LnpDC{Free Variable}}
	A variable token in a formula $\CAPPHI$ is \underdf{free}{variables} \Iff it is not bound.
\end{majorILnc}

\noindent{}In the formula $\existential{\variable{x}}\parconjunction{\Gp{'\variable{x}}}{\Hp{'\variable{x}}}$, the first token of $\variable{x}$ is bound because it's part of the quantifier expression \mention{$\existential{\variable{x}}$}.  The second and third tokens of $\variable{x}$ are bound because they are within the scope of \mention{$\existential{\variable{x}}$}.\footnote{
	We can also determine whether a variable is bound by thinking of a tree of the formula. A variable token is bound \Iff a quantifier with the same variable appears below or at the same level as (but still on the same branch as) that token. 
	The quantifier that binds a variable is the \emph{first} quantifier that appears below or at the same level as (but still on the same branch as) the variable. 
}

Now we are ready for the definition of a \GQL{}1 sentence:

\begin{majorILnc}{\LnpDC{GQL1 Sentence}}
A string of \GQL{}1 symbols is a \nidf{sentence}\index{sentence!of \GQL{}1|textbf} \Iff it is a formula that contains no free variables.
\end{majorILnc}

\begin{majorILnc}{\LnpDC{Atomic Sentence of GQL}}
An \underdf{atomic}{sentence} \nidf{sentence} of \GQL{}1 is an atomic formula of \GQL{}1 that has no free variable.
\end{majorILnc}
\noindent{}We have unofficial sentences too, just as we have unofficial formulas.
\begin{majorILnc}{\LnpDC{Unofficial Sentence of GQL}}
A string of symbols is an \nidf{unofficial} sentence\index{sentence!unofficial (of \GQL{})|textbf} \Iff it's an unofficial formula that contains no free variables. In other words, we can get an unofficial sentence from an official one by
\begin{cenumerate}
\item deleting outer parentheses,
\item replacing official parentheses ( ) with square brackets [ ] or curly brackets \{ \}, or
\item omitting primes $'$ on the predicate letters.
\end{cenumerate}
\end{majorILnc}
\begin{majorILnc}{\LnpEC{GQLSentenceFreeVariableExampleA}}
Both formulas $\horseshoe{\universal{\variable{x}}\Hp{'\variable{x}}}{\universal{\variable{x}}\Gp{'\variable{x}}}$ and $\universal{\variable{x}}\parhorseshoe{\Hp{'\variable{x}}}{\Gp{'\variable{x}}}$ from examples \ref{GQL1SubformulaPropertiesExampleA} and \ref{GQL1SubformulaPropertiesExampleB} are sentences, because in each all variables are bound.
Below we have arrows pointing from each variable token to the quantifier that binds it.  (There are no arrows for the variables in quantifier expressions since it's obvious which quantifiers they belong to.)
\begin{cenumerate}
\item $\horseshoe{\universal{\variable{x}}\Hp{'\variable{x}}\NextLineRef[black, distance=6, out=-70, in=-50]}{\universal{\variable{x}}\Gp{'\variable{x}}}\NextLineRef[black, distance=6, out=-70, in=-50]$
\item $\universal{\variable{x}}\parhorseshoe{\Hp{'\variable{x}}\NextLineRefJ[black, out=-50, in=-50]}{\Gp{'\variable{x}}\NextLineRefB[black, distance=9, out=205, in=315]}$\footnote{
	You can also use trees to show that these variables are all bound. Look at the construction trees of the formulas from examples \ref{GQL1SubformulaPropertiesExampleA} and \ref{GQL1SubformulaPropertiesExampleB}. You will see that in each case the indicated quantifier is the first that appears below, but still on the same branch as, the token variable.
}
\end{cenumerate}

\end{majorILnc}
\begin{majorILnc}{\LnpEC{GQLSentenceFreeVariableExampleB}}
The formula $\disjunction{\existential{\variable{x}}\parconjunction{\universal{y}\Epp{'}{\variable{x}}}{\Ap{'\variable{x}}}}{\universal{\variable{z}}\parhorseshoe{\existential{\variable{y}}\Hp{'\constant{a}}}{\Gp{'\variable{x}}}}$ is not a sentence. There is a free variable in it. 
The free variable is underlined, while arrows point from each bound variable to the quantifier that binds it (again, ignoring variables in quantifier expressions). 

\smallskip
\begin{cenumerate}
\item $\disjunction{\existential{\variable{x}}\parconjunction{\universal{y}\Epp{'}{\variable{x}\NextLineRefC[black, distance=15, out=135, in=45]}}{\Ap{'\variable{x}}\NextLineRefH[black, distance=13, out=295, in=305]}}{\universal{\variable{z}}\parhorseshoe{\existential{\variable{y}}\Hp{'\constant{a}}}{\Gp{'\underline{\variable{x}}}}}$
\end{cenumerate}

\medskip
\noindent{}The two quantifiers on the \CAPS{rhs} of the disjunction are not binding any token variables, besides the ones that appear in the quantifier expressions themselves.
\end{majorILnc}
\begin{majorILnc}{\LnpEC{GQLSentenceFreeVariableExampleC}}
In each of the following three formulas the free variable tokens are underlined, and arrows go from variable tokens to the quantifiers that bind them.

\smallskip
\begin{enumerate}[label=(\arabic*), leftmargin=1.85\parindent,
labelindent=.35\parindent, labelsep=*, itemsep=8pt]
\item $\universal{\variable{x}}\horseshoe{\parconjunction{\Bl}{\existential{\variable{z}}\Kpp{'}{\variable{x}\NextLineRefF[black, distance=17, out=135, in=145]}}}{\existential{\variable{x}}\Np{\variable{x}}\NextLineRefK[black, out=-50, in=-50]}$
\item $\conjunction{\Dl}{\pardisjunction{\existential{\variable{u}}\universal{\variable{w}}\Ppp{'}{\variable{u}\NextLineRefC[black, distance=12, out=135, in=45]}}{\Cpp{'}{\underline{\variable{y}}}}}$
\item $\existential{\variable{z}}\partriplebar{\parconjunction{\Hpp{'}{\underline{\variable{x}}}}{\Gl}}{\Dp{\variable{z}}\NextLineRefI[black, distance=25, out=205, in=315]}$
\end{enumerate}

\medskip
\noindent{}Formula (1) has no free variables but (2) and (3) do, so (1) is a sentence but (2) and (3) aren't. 
\end{majorILnc}

%%%%%%%%%%%%%%%%%%%%%%%%%%%%%%%%%%%%%%%%%%%%%%%%%%
\section{Models}\label{GQL1 Interpretations}
%%%%%%%%%%%%%%%%%%%%%%%%%%%%%%%%%%%%%%%%%%%%%%%%%%

As with \GSL{}, sentences of \GQL{}1 have no inherent meaning.  
But, also as with \GSL{}, we give \mention{models} for sentences of \GQL{}1.
These models allow us to define and investigate entailment for \GQL{}1.

Our goal is to define \mention{model} such that each \GQL{}1 sentence has a determinate truth value on each model.
Note that only sentences of \GQL{}1 have true values, and not formulas.
A formula that isn't a sentence has some free variable.
Accordingly, it corresponds to a grammatical sentence of English that doesn't have a determinate truth value, because it contains one or more pronouns.
For example, the sentence \mention{He is the author of Waverley,} may be either true or false, depending on who \mention{he} is.
In ordinary conversation we use context to determine the referent of a pronoun.
Someone discussing the English author Sir Walter Scott may read the above sentence and evaluate it as true.
However, in a conversation in which \mention{he} refers to Aristotle, the sentence would be evaluated as false.
The sentence cannot, however, be evaluated without a referent specified.
Analogously, the unbound variables of \GQL{}1 formulas don't have any context-independent \mention{referent} (i.e., assignment).
Therefore non-sentence formulas of \GQL{}1 are not to have determinate truth values on the models of \GQL{}1.  

\subsection{Models in \GQL{}1}\label{Interpretations in GQL1}
An \GSL{} model for $\CAPPHI$ assigns a truth value to each sentence letter in $\CAPPHI$, and that's it. 
\GQL{}1 has predicate letters and constants, which will require different kinds of assignments. 
Furthermore, \GQL{}1 has quantifiers.
The quantifiers roughly correspond to English words such as \emph{all} or \emph{some}, so each model must specify a set of objects that the quantifiers range over.
In other words, a \GQL{}1 model must fix a domain of objects over which to quantify. This is sometimes called a \mention{universe of discourse}.
\begin{majorILnc}{\LnpDC{GQL1 Interpretation}} 
A \df{model} for $\CAPPHI$, $\IntA$, consists of:
\begin{cenumerate}
\item an assignment of a truth value $\TrueB$ or $\FalseB$ to each sentence letter in $\CAPPHI$; 
\item a single, non-empty set $\integer{U}$, called the \df{universe} or \df{domain};
\item an assignment of a subset of $\integer{U}$ to each 1-place predicate in $\CAPPHI$;
\item an assignment of an element from $\integer{U}$ to each individual constant in $\CAPPHI$.
\end{cenumerate}
\end{majorILnc}
\noindent{}We use the following notational conventions: 

\begin{cenumerate}
	\item Given some sentence letter, like $\PP$, $\IntA(\PP)$ is the truth value $\IntA$ assigns to $\PP$.
	\item $\IntA(\integer{U})$ is the set $\m$ assigns to $\integer{U}$.
	\item Given a 1-place predicate, like $\Gp{'}$, $\IntA(\Gp{'})$ is the subset of $\integer{U}$ assigned to $\Gp{'}$ by $\IntA$.
	\item Given an individual constant, like $\constant{a}$, $\IntA(\constant{a})$ is the element from $\integer{U}$ assigned by $\IntA$ to $\constant{a}$.\footnote{%
		We pause here to make two points for those keeping careful score:
		\begin{enumerate*}[label=(\arabic*)]
		\item We can think of models as functions from the set of basic symbols of \GQL{}1 (less the logical operators, variables, and parentheses) to the kinds of objects mentioned in definition \mvref{GQL1 Interpretations} (objects or subsets of $\integer{U}$). 
		\item\label{pointtwo} Those trying to keep careful track of the \distinction{use}{mention} distinction\index{\distinction{use}{mention} distinction}\index{single quotes} should note that here we've been especially loose. 
		We justify our laxity on the grounds that strict adherence to the distinction would clutter up our notation with confusing layers of quotes. 
		\end{enumerate*}
		\label{Int Footnote}
	} 
\end{cenumerate}

\noindent{}Earlier we said that the individual constants are roughly similar to proper names in English.
One difference is that, in \GQL{}1, each individual constant in $\CAPPHI$ corresponds to exactly one object in the domain.
In English, on the other hand, some proper names---e.g., \mention{John Smith}---correspond to more than one person, and some---e.g., \mention{Mordecai Alonzo Frazzle III}---do not correspond to any person.
While each constant in $\CAPPHI$ is assigned an object from the domain, it is not required that different constants be assigned different objects. 

We distinguish different models by affixing integers as subscripts to the symbol \mention{$\As{}{}$}.  So, for example, $\As{}{1}$, $\As{}{2}$, $\As{}{3}$, \ldots, $\As{}{316}$, etc., are each different models.

As with \GSL{}, we have \GQL{}1 models for sets of sentences:

\begin{majorILnc}{\LnpDC{Definition of Model for QL1 Set}}
	$\IntA$ is a \df{model for a set of \GQL{}1 sentences $\Delta$} \Iff $\IntA$ is a model for each sentence in $\Delta$.
\end{majorILnc}

There are also models that make assignments to all the sentence letters, constants, and 1-place predicates of \GQL{}1.
Any such model is a model for every \GQL{}1 sentence.
Let's call these \emph{models for \GQL{}1}:

\begin{majorILnc}{\LnpDC{Definition of Model for QL1}}
	$\IntA$ is a \df{model for \GQL{}1} \Iff $\IntA$ is a model for every sentence of \GQL{}1.
\end{majorILnc}

We define truth in \GQL{}1 so that every model for $\CAPPHI$ fixes a unique truth value for $\CAPPHI$.
But there is a price we must pay for \GQL{}1's superior models.
\GQL{}1 is more complicated than \GSL{}, so its definition of truth requires some additional metalinguistic tools.
We must first define \mention{terms} and \mention{model variants}.

\begin{majorILnc}{\LnpDC{Terms}}
The \idf{individual terms} of \GQL{}1 are the constants and variables of \GQL{}1.
\end{majorILnc}
\noindent{}We typically use \mention{$\variable{q}$},\mention{$\variable{r}$}, \mention{$\variable{s}$}, and \mention{$\variable{t}$}  (along with subscripts) as MathEnglish metavariables for terms. This means that italic Roman \mention{$\variable{q}$}, \mention{$\variable{r}$}, \mention{$\variable{s}$}, \mention{$\variable{t}$}, and the Greek \mention{$\ALPHA$}, \mention{$\BETA$}, etc., can all be MathEnglish variables for \GQL{} variables.
However, while \mention{$\ALPHA$}, \mention{$\BETA$}, etc. stand \emph{only} for variables, the italic Roman letters can also range over \GQL{}1 constants.
The use of metavariables for individual terms in the object language simplifies our notation considerably.

At this point we could define truth for sentences without quantifiers, though for now we offer only a short sketch.
The sentence $\Gp{\constant{a}}$ is true on $\IntA$ \Iff the element $\IntA$ assigns to \mention{$\constant{a}$} is in the set $\IntA$ assigns to \mention{$\GG$}; i.e., \Iff $\IntA(\constant{a})\in\IntA(\GG)$.
If the element $\IntA$ assigns to \mention{$\constant{a}$} isn't in the set assigned to \mention{$\GG$}, $\Gp{\constant{a}}$ is false on $\IntA$.

Quantifiers require more complexity.
A sentence like $\universal{\variable{x}}\Ep{\variable{x}}$ is true \Iff every element in the domain, $\integer{U}$, is an element of $\IntA(\EE)$.
So if $\IntA$ assigns $\integer{U}$ the set of even integers and $\IntA(\EE)=\integer{U}$, $\universal{\variable{x}}\Ep{\variable{x}}$ is true.
But if instead $\IntA(\EE)=\{2, 4, 6\}$, then there are objects in the domain (e.g. $8$) not in $\IntA(\EE)$, and so $\universal{\variable{x}}\Ep{\variable{x}}$ is false.
For simple quantified sentences this quick definition is good enough; but it won't work for more complex sentences, such as $\universal{\variable{x}}\existential{\variable{y}}\parhorseshoe{\Hp{\variable{y}}}{\Gp{\variable{x}}}$.

To define truth for quantified sentences more precisely, we need a convenient notation for two models that make identical assignments everywhere except at one constant.

\begin{majorILnc}{\LnpDC{Variant}}
	Let $\m$ and $\m^{\variable{t}}$ be \GQL{}1 models and $\variable{t}$ be a term that is given some assignment by $\m^{\variable{t}}$. Then $\m^{\variable{t}}$ is a $\variable{t}$-\nidf{variant}\index{model!$\variable{t}$-variant} of $\m$ \Iff for every term $\variable{r}$ such that $\variable{r}\not=\variable{t}$, $\m(r)=\m^{\variable{t}}(r)$.
\end{majorILnc}
\noindent{}Put another way, a $\variable{t}$-variant of $\As{}{}$ is a model that makes all the same assignments as $\As{}{}$ except possibly at constant $\variable{t}$.
One result of this definition is that any model $\As{}{}$ that assigns something to $\variable{t}$ is a $\variable{t}$-variant of itself.
But what if $\As{}{}$ doesn't assign anything to $\variable{t}$? In that case, the $\variable{t}$-variants are all the models that are identical with $\As{}{}$ except with an additional assignment to $\variable{t}$.

We generally denote $\variable{t}$-variants of a model $\As{}{}$ by affixing $\variable{t}$ as a superscript to \mention{$\As{}{}$}. For example, $\As{\constant{c}}{}$ is a $\constant{c}$-variant of $\As{}{}$.
Then $\As{\constant{c}}{}$ and $\As{}{}$ make all the same assignments, except possibly to the constant $\constant{c}$.
We extend this notation when the symbol denoting the original model is itself complex.
So, given an $\constant{c}$-variant of model $\As{}{}$, i.e., $\As{\constant{c}}{}$, $\As{\constant{cd}}{}$ is a $\constant{d}$-variant of $\As{\constant{c}}{}$.  For another example, if $\As{\constant{e}}{4}$ is an $\constant{e}$-variant of $\As{}{4}$, then $\As{\constant{e}}{4}$ and $\As{}{4}$ make identical assignments to everything except maybe $\constant{e}$. 

We need one more piece of notation before we get to the definition of truth.
\begin{majorILnc}{\LnpDC{MathEnglishVariableSub1}}
	If $\CAPPHI$ is a \GQL{}1 formula and $\variable{t}$ and $\variable{s}$ are terms, then $\CAPPHI\variable{s}/\variable{t}$ is the formula you get by replacing each unbound token of $\variable{t}$ in $\CAPPHI$ with a token of $\variable{s}$.
\end{majorILnc}

\begin{majorILnc}{\LnpEC{MathEnglishVariableSubEx1}}
	\begin{cenumerate}
		\item If $\CAPPHI$ is $\Al$, then $\CAPPHI\variable{y}/\variable{x}$ is $\Al$.
		\item If $\CAPPHI$ is $\Bp{\variable{x}}$, then $\CAPPHI\variable{y}/\variable{x}$ is $\Bp{\variable{y}}$.
		\item If $\CAPPHI$ is $\Bp{\variable{y}}$, then $\CAPPHI\variable{y}/\variable{x}$ is $\Bp{\variable{y}}$.
		\item If $\CAPPHI$ is $\Bp{\variable{x}}$, then $\CAPPHI\variable{y}/\variable{w}$ is $\Bp{\variable{x}}$.
		\item If $\CAPPHI$ is $\universal{x}\Bp{\variable{x}}$, then $\CAPPHI\variable{y}/\variable{x}$ is $\universal{x}\Bp{\variable{x}}$.
		\item If $\CAPPHI$ is $\conjunction{\Cp{\variable{x}}}{\universal{x}\Bp{\variable{x}}}$, then $\CAPPHI\variable{y}/\variable{x}$ is $\conjunction{\Cp{\variable{y}}}{\universal{x}\Bp{\variable{x}}}$.
		\item If $\CAPPHI$ is $\existential{\variable{y}}\parconjunction{\Cp{\variable{x}}}{\universal{x}\Bp{\variable{x}}}$, then $\CAPPHI\variable{y}/\variable{x}$ is $\existential{\variable{y}}\parconjunction{\Cp{\variable{y}}}{\universal{x}\Bp{\variable{x}}}$.
		\item If $\CAPPHI$ is $\existential{\variable{y}}\parconjunction{\Cp{\variable{x}}}{\universal{x}\Bp{\variable{x}}}$, then $\CAPPHI\constant{a}/\variable{x}$ is $\existential{\variable{y}}\parconjunction{\Cp{\constant{a}}}{\universal{x}\Bp{\variable{x}}}$.
		\item If $\CAPPHI$ is $\existential{\variable{y}}\parconjunction{\Cp{\variable{x}}}{\Bp{\variable{x}}}$, then $\CAPPHI\constant{a}/\variable{x}$ is $\existential{\variable{y}}\parconjunction{\Cp{\constant{a}}}{\Bp{\constant{a}}}$.
	\end{cenumerate}
\end{majorILnc}

%\subsection{Truth in a Model: Preliminary Ideas}\label{GQL Truth in an Interpretation Prelims}

\subsection{Truth in a Model}\label{GQL1 Truth in an Interpretation}
We now define truth in a \GQL{}1 model.

\begin{majorILnc}{\LnpDC{Truth for GQL1 Sentence}}
The following clauses fix when a \GQL{}1 sentence $\CAPTHETA$ is \nidf{$\True$} (or \nidf{$\False$}) on a model for $\CAPTHETA$, $\IntA$:
\begin{cenumerate}
\item A sentence letter $\CAPPHI$ is $\True$ on $\IntA$ \Iff $\As{}{}(\CAPPHI)=\TrueB$.
\item An atomic sentence $\Pp{\variable{t}}$ with a 1-place predicate $\PP$ and an individual term $\variable{t}$ is $\True$ on $\IntA$ \Iff $\IntA(\variable{t})\in\IntA(\PP)$.
\item A negation $\negation{\CAPPHI}$ is $\True$ on $\IntA$ \Iff $\CAPPHI$ is $\False$ on $\IntA$.
\item A conjunction $\parconjunction{\CAPPHI_1}{\conjunction{\ldots}{\CAPPHI_{\integer{n}}}}$ is $\True$ on $\IntA$ \Iff all of $\CAPPHI_1,\ldots,\CAPPHI_{\integer{n}}$ are $\True$ on $\IntA$.
\item A disjunction $\pardisjunction{\CAPPHI_1}{\disjunction{\ldots}{\CAPPHI_{\integer{n}}}}$ is $\True$ on $\IntA$ \Iff at least one of $\CAPPHI_1,\ldots,\CAPPHI_{\integer{n}}$ is $\True$ on $\IntA$.
\item A conditional $\parhorseshoe{\CAPPSI}{\CAPPHI}$ is $\True$ on $\IntA$ \Iff the \CAPS{lhs} $\CAPPSI$ is $\False$ or the \CAPS{rhs} $\CAPPHI$ is $\True$ on $\IntA$, or both.
\item A biconditional $\partriplebar{\CAPPSI}{\CAPPHI}$ is $\True$ on $\IntA$ \Iff $\CAPPSI$ and $\CAPPHI$ have the same truth value on $\IntA$.
\item\label{GQL1TruthUnvQuant} A universal quantification $\universal{\ALPHA}\CAPPHI$ is $\True$ on $\IntA$ \Iff $\CAPPHI\variable{t}/\ALPHA$ is $\True$ on \emph{all} $\variable{t}$-variants of $\IntA$, where $\variable{t}$ is the first constant not in $\CAPPHI$.
\item An existential quantification $\existential{\ALPHA}\CAPPHI$ is $\True$ on $\IntA$ \Iff $\CAPPHI\variable{t}/\ALPHA$ is $\True$ on \emph{some} $\variable{t}$-variant of $\IntA$, where $\variable{t}$ is the first constant not in $\CAPPHI$.
\item A sentence $\CAPPHI$ is $\False$ on $\IntA$ \Iff $\CAPPHI$ is not $\True$ on $\IntA$.
\end{cenumerate}
\end{majorILnc}

\noindent{}For the following examples consult the models \emph{Pos Int} and \emph{States}, given in figure \mvref{table:Partial Models}.

\begin{figure}
\begin{longtable}[c]{ l l l l } %p{2.2in} p{2in}
	\toprule
	&\textbf{Symbol} & \multicolumn{2}{c}{\textbf{Model}} \\ \cmidrule(l){3-4}
	& & \textbf{Pos Int} & \textbf{States} \\
	\midrule 
	\endfirsthead
	\multicolumn{4}{c}{\emph{Continued from Previous Page}}\\
	\toprule
	&\textbf{Symbol} & \multicolumn{2}{c}{\textbf{Model}} \\ \cmidrule(l){3-4}
	& & \textbf{Pos Int} & \textbf{States} \\
	\midrule 
	\endhead
	\bottomrule
	\caption{Example Models}\\[-.15in]
	\multicolumn{4}{c}{\emph{Continued next Page}}\\
	\endfoot
	\bottomrule
	\caption{Example Models}\\%
	\endlastfoot%
	\label{table:Partial Models}%
	%\begin{tabular}{ l l l l } %p{2in} p{2in} %\begin{tabular}{ p{1in} l l } %p{2.2in} p{2in}
	%\toprule
	%&\textbf{Symbol} & \multicolumn{2}{c}{\textbf{Interpretation}} \\ \cmidrule(l){3-4}
	%& & \textbf{Pos Int} & \textbf{States} \\
	%\midrule 
	{Universe:} & & The set of positive integers & The set of US states (2024) \\ \addlinespace[.25cm]
	{Sent. Let.:}& A&$\True$&$\False$\\
	& B&$\True$&$\False$\\
	& C&$\False$&$\True$\\
	& D&$\True$&$\False$\\
	& E&$\True$&$\False$\\
	& G&$\False$&$\True$\\ \addlinespace[.25cm]
	{Constants:}&$\constant{a}$&1&Louisiana\\
	&$\constant{b}$&9&Maine\\
	&$\constant{c}$&72&Georgia\\
	&$\constant{d}$&3&Nebraska\\
	&$\constant{e}$&1&New Mexico\\
	&$\constant{f}$&2&Texas\\ \addlinespace[.25cm]
	{1-place:}&$\Ap{'}$&all pos int&Midwestern\\
	&$\Bp{'}$&empty set&name with $>5$ letters\\
	&$\Cp{'}$&even&Coastal\\
	&$\Dp{'}$&odd&on the Pacific coast\\
	&$\Ep{'}$&prime&\{Ohio\}\\
	&$\Gp{'}$&multiple of 7&\{Ohio,Alabama\}\\ \addlinespace[.25cm]
	%\bottomrule
\end{longtable}
\caption{Two \GQL{}1 models}
\end{figure}

\begin{majorILnc}{\LnpEC{GQL1TruthExamplePA}}
	The sentence $\negation{\Gp{\constant{b}}}$ is true on the model \emph{Pos Int}. 
\end{majorILnc}
\begin{PROOF}
	The model \emph{Pos Int} assigns $9$ to $\constant{b}$, and the set of multiples of $7$ to $\GG$.
	The number $9$ is not a multiple of $7$, so $\emph{Pos Int}(\constant{b})\notin\emph{Pos Int}(\GG)$.
	So, by the definition of truth, $\Gp{\constant{b}}$ is false on \emph{Pos Int}.
	Therefore, by the definition of truth for $\NEGATION$, \emph{Pos Int} makes $\negation{\Gp{\constant{b}}}$ true.

\end{PROOF}
\begin{commentary}
	$\emph{Pos Int}(\constant{b})\notin\emph{Pos Int}(\GG)$ asserts that what \emph{Pos Int} assigns to $\constant{b}$, $9$, is not an element of the set that \emph{Pos Int} assigns to $\GG$.
	To check this, we need only to look at the set assigned to $\GG$ and see whether $7$ is a member; it's not, since $9$ is not a multiple of $7$.
\end{commentary}

\begin{majorILnc}{\LnpEC{GQL1TruthExampleA}}
The sentence ${\parhorseshoe{\Gp{\constant{d}}}{\Dp{\constant{e}}}}$ is true on the model \emph{Pos Int} (\ref{table:Partial Models}). 
\end{majorILnc}
\begin{PROOF}
$\parhorseshoe{\Gp{\constant{d}}}{\Dp{\constant{e}}}$ is true on \emph{Pos Int} \Iff the \CAPS{LHS} is false or the \CAPS{RHS} is true.
\emph{Pos Int} assigns the number $3$ to $\constant{d}$ and the set of multiples of $7$ to $\GG$.
But $3$ isn't a multiple of $7$, so $\emph{Pos Int}(\constant{d})\notin\emph{Pos Int}(\GG)$.
It follows that $\Gp{\constant{d}}$ is false on \emph{Pos Int}.
Therefore $\parhorseshoe{\Gp{\constant{d}}}{\Dp{\constant{e}}}$ is true on \emph{Pos Int}.
\end{PROOF}

\begin{majorILnc}{\LnpEC{GQL1TruthExampleB}}
The sentence $\disjunction{\Ap{\constant{a}}}{\Gp{\constant{c}}}$ is false on the model \emph{States} (\ref{table:Partial Models}).
\end{majorILnc}
\begin{PROOF}
The model \emph{States} assigns Louisiana to $\constant{a}$, Georgia to $\constant{c}$, the set of Midwestern states to $\AA$, and the set $\{$Ohio, Alabama$\}$ to $\GG$.

$\Ap{\constant{a}}$ is true on \emph{States} \Iff $\emph{States}(\constant{a})\in\emph{States}(\AA)$.
But Louisiana isn't a Midwestern state.
So $\Ap{\constant{a}}$ is false on \emph{States}.
$\Gp{\constant{c}}$ is true on \emph{States} \Iff $\emph{States}(\constant{c})\in\emph{States}(\GG)$.
But \emph{States} assigns $\{$Ohio, Alabama$\}$ to $\GG$.
Georgia isn't in that set, so \emph{States} makes $\Gp{\constant{c}}$ false.

$\Ap{\constant{a}}$ and $\Gp{\constant{c}}$ are false on \emph{States}, so \emph{States} makes $\disjunction{\Ap{\constant{a}}}{\Gp{\constant{c}}}$ false.
\end{PROOF}
\noindent{}Pay attention to the full structure of a sentence when assessing its truth value.
The sentences $\horseshoe{\universal{\variable{x}}\Dp{\variable{x}}}{\universal{\variable{x}}\Gp{\variable{x}}}$ and $\universal{\variable{x}}\parhorseshoe{\Dp{\variable{x}}}{\Gp{\variable{x}}}$ may look similar, but they mean very different things.
Note that the first has \mention{$\HORSESHOE$} as its main connective, whereas the main connective of the second is \mention{$\forall$}.
We show that their truth values come apart on \emph{Pos Int} in the next two examples.

\begin{majorILnc}{\LnpEC{GQL1TruthExampleC}}
$\horseshoe{\universal{\variable{x}}\Dp{\variable{x}}}{\universal{\variable{x}}\Gp{\variable{x}}}$ is true on the model \emph{Pos Int} (\ref{table:Partial Models}).
\end{majorILnc}
\begin{PROOF}
	$\horseshoe{\universal{\variable{x}}\Dp{\variable{x}}}{\universal{\variable{x}}\Gp{\variable{x}}}$ is true on \emph{Pos Int} \Iff \emph{Pos Int} either makes the \CAPS{lhs} is false or the \CAPS{RHS} true.
	By the definition of truth for $\forall$, $\universal{\variable{x}}\Dp{\variable{x}}$ is true on \emph{Pos Int} \Iff $\Dp{\constant{a}}$ is true on all $\constant{a}$-variants of \emph{Pos Int}.
	\begin{commentary}
		The definition of truth for $\forall$ has us substitute the first constant not in $\Dp{\variable{x}}$.
		Since it has no constant we use the first one: $\constant{a}$.
		The result of the substitution is $\Dp{\constant{a}}$.
	\end{commentary}
	\noindent{}\emph{Pos Int} assigns the set of odd numbers to $\DD$.
	Consider the $\constant{a}$-variant of \emph{Pos Int}, \emph{Pos Int}\textsuperscript{a}, such that \emph{Pos Int}\textsuperscript{a}($\constant{a}$) is $2$.
	Clearly $2$ is not an odd number.
	It follows that $\Dp{\constant{a}}$ is false on \emph{Pos Int}\textsuperscript{a}, and so $\universal{\variable{x}}\Dp{\variable{x}}$ is false on \emph{Pos Int}.
	Therefore $\horseshoe{\universal{\variable{x}}\Dp{\variable{x}}}{\universal{\variable{x}}\Gp{\variable{x}}}$ is true on \emph{Pos Int}.
\end{PROOF}

\begin{majorILnc}{\LnpEC{GQL1TruthExampleC2}}
	$\universal{\variable{x}}\parhorseshoe{\Dp{\variable{x}}}{\Gp{\variable{x}}}$ is false on \emph{Pos Int} (\ref{table:Partial Models}).
\end{majorILnc}
\begin{PROOF}
	By the definition of truth for $\forall$, $\universal{\variable{x}}\parhorseshoe{\Dp{\variable{x}}}{\Gp{\variable{x}}}$ is true on \emph{Pos Int} \Iff every $\constant{a}$-variant of \emph{Pos Int} makes $\horseshoe{\Dp{\constant{a}}}{\Gp{\constant{a}}}$ true.
	\emph{Pos Int} assigns the set of odd numbers to $\DD$ and the set of multiples of $7$ to $\GG$.
	Let there be an $\constant{a}$-variant of \emph{Pos Int}, $\emph{Pos Int}^{\constant{a}}$, such that $\emph{Pos Int}^{\constant{a}}(\constant{a})=3$.
	The number $3$ is odd and not a multiple of $7$.
	So $\emph{Pos Int}^{\constant{a}}(\constant{a})\in\emph{Pos Int}^{\constant{a}}(\DD)$ and $\emph{Pos Int}^{\constant{a}}(\constant{a})\notin\emph{Pos Int}^{\constant{a}}(\GG)$.
	Thus, $\emph{Pos Int}^{\constant{a}}$ makes $\Dp{\constant{a}}$ true and $\Gp{\constant{a}}$ false, which in turn makes $\horseshoe{\Dp{\constant{a}}}{\Gp{\constant{a}}}$ false.
	Therefore $\universal{\variable{x}}\parhorseshoe{\Dp{\variable{x}}}{\Gp{\variable{x}}}$ is false on \emph{Pos Int}.
\end{PROOF}

\noindent{}Let's compare another pair of superficially similar sentences, this time with existential quantifiers: $\existential{\variable{x}}\parconjunction{\Cp{\variable{x}}}{\Dp{\variable{x}}}$ and $\parconjunction{\existential{\variable{x}}\Cp{\variable{x}}}{\existential{\variable{x}}\Dp{\variable{x}}}$.

\begin{majorILnc}{\LnpEC{GQLTruthExampleD}}
	$\existential{\variable{x}}\parconjunction{\Cp{\variable{x}}}{\Dp{\variable{x}}}$ is false on \emph{Pos Int} (\ref{table:Partial Models}).
\end{majorILnc}
\begin{PROOF}
By the definition of truth for $\exists$, $\existential{\variable{x}}\parconjunction{\Cp{\variable{x}}}{\Dp{\variable{x}}}$ is true on \emph{Pos Int} \Iff there is an $\constant{a}$-variant of \emph{Pos Int} that makes $\conjunction{\Cp{\constant{a}}}{\Dp{\constant{a}}}$ true.  
\emph{Pos Int} assigns the even numbers to $\CC$ and the odd numbers to $\DD$.
There is no element in the domain of \emph{Pos Int} that is both odd and even.
So there is no $\constant{a}$-variant of \emph{Pos Int} that makes both $\Cp{\constant{a}}$ and $\Dp{\constant{a}}$ true.
Hence $\conjunction{\Cp{\constant{a}}}{\Dp{\constant{a}}}$ is false on all $\constant{a}$-variants.
Therefore $\existential{\variable{x}}\parconjunction{\Cp{\variable{x}}}{\Dp{\variable{x}}}$ is false on \emph{Pos Int}.
\end{PROOF}	

\begin{majorILnc}{\LnpEC{GQLTruthExampleD2}}
	$\parconjunction{\existential{\variable{x}}\Cp{\variable{x}}}{\existential{\variable{x}}\Dp{\variable{x}}}$ is true on \emph{Pos Int} (\ref{table:Partial Models}).
\end{majorILnc}
\begin{PROOF}
$\parconjunction{\existential{\variable{x}}\Cp{\variable{x}}}{\existential{\variable{x}}\Dp{\variable{x}}}$ is true on \emph{Pos Int} \Iff $\existential{\variable{x}}\Cp{\variable{x}}$ and $\existential{\variable{x}}\Dp{\variable{x}}$ are true on \emph{Pos Int}.

By the definition of truth for $\exists$, $\existential{\variable{x}}\Cp{\variable{x}}$ is true on \emph{Pos Int} \Iff there is an $\constant{a}$-variant of \emph{Pos Int} that makes $\Cp{\constant{a}}$ true.
$\emph{Pos Int}$ assigns the even numbers to $\CC$.
Let $\emph{Pos Int}^{\constant{a}}$ be an $\constant{a}$-variant that assigns $4$ to $\constant{a}$.
The number $4$ is even, so $\emph{Pos Int}^{\constant{a}}(\constant{a})\in\emph{Pos Int}^{\constant{a}}(\CC)$.
Thus, $\emph{Pos Int}^{\constant{a}}$ makes $\Cp{\constant{a}}$ true, and so \emph{Pos Int} makes $\existential{\variable{x}}\Cp{\variable{x}}$ true.

By the definition of truth for $\exists$, $\existential{\variable{x}}\Dp{\variable{x}}$ is true on \emph{Pos Int} \Iff there is an $\constant{a}$-variant of \emph{Pos Int} that makes $\Dp{\constant{a}}$ true.
$\emph{Pos Int}$ assigns the odd numbers to $\DD$.
Let $\emph{Pos Int}^{\constant{a}}$ be an $\constant{a}$-variant that assigns $3$ to $\constant{a}$.
The number $3$ is odd, so $\emph{Pos Int}^{\constant{a}}(\constant{a})\in\emph{Pos Int}^{\constant{a}}(\DD)$.
Thus, $\emph{Pos Int}^{\constant{a}}$ makes $\Dp{\constant{a}}$ true.
It follows that \emph{Pos Int} makes $\existential{\variable{x}}\Dp{\variable{x}}$ true.
	
Therefore \emph{Pos Int} makes $\parconjunction{\existential{\variable{x}}\Cp{\variable{x}}}{\existential{\variable{x}}\Dp{\variable{x}}}$ true.
\end{PROOF}

\noindent{}On model \emph{Pos Int} we can interpret $\existential{\variable{x}}\parconjunction{\Cp{\variable{x}}}{\Dp{\variable{x}}}$ as saying, roughly that there is some positive integer that is both even and odd.
That's clearly false.
By contrast, $\parconjunction{\existential{\variable{x}}\Cp{\variable{x}}}{\existential{\variable{x}}\Dp{\variable{x}}}$ means, roughly, that there is some even positive integer and there is some odd positive integer, which is true.

We can calculate the truth value of a \GQL{}1 sentence $\CAPPHI$ on some $\IntA$ \Iff $\IntA$ is a model for $\CAPPHI$.  If $\IntA$ \emph{isn't} a model for $\CAPPHI$, then $\CAPPHI$ has no truth value on it.  Be careful: if $\As{}{1}$ is a model for $\CAPPHI$ and $\As{}{2}$ is a model for $\CAPPSI$, it doesn't follow that either $\As{}{1}$ or $\As{}{2}$ is a model for, say, $\horseshoe{\CAPPHI}{\CAPPSI}$.

\subsection{Minimal Models in \GQL{}1}\label{Minimal Models in GQL1}

For the examples above we used models with assignments that are not referenced in the sentences we evaluated.
The Minimal Model theorem below proves that assignments to irrelevant predicates, constants, etc. don't matter for the truth value of a sentence.

\begin{majorILnc}{\LnpDC{Definition of Minimal QL1 Model}}
	Model $\IntA$ is a \df{minimal model for $\CAPPHI$} \Iff $\IntA$ makes the minimum assignments necessary for $\IntA$ to be a model for $\CAPPHI$.
\end{majorILnc}

\noindent{}That is, a minimal model makes assignments to the universe $\integer{U}$, to each sentence letter, constant, and 1-place predicate in $\CAPPHI$, but to nothing else.
Consider the sentence $\parconjunction{\existential{\variable{x}}\Cp{\variable{x}}}{\existential{\variable{x}}\Dp{\variable{x}}}$.
This sentence has two 1-place predicates, $\CC$ and $\DD$, and no sentence letters or constants.
A minimal model for this sentence makes no assignment to any sentence letter or constant.
It assigns subsets of $\integer{U}$ to $\CC$ and $\DD$, but makes no assignment to any other 1-place predicate.
Most logic texts do not define minimal models but implicitly use them.
We prefer to define them explicitly.

When calculating the value of a sentence $\CAPPHI$ on a model, we only need to worry about the assignments to the symbols in $\CAPPHI$ and the universe.
We can ignore the other assignments.
The following theorem demonstrates this.
This is the \GQL{}1 version of theorem \ref{thm:localityoftruth} in chapter \ref{sententiallogic}.

\begin{THEOREM}{\LnpTC{Two Models}} 
	Let $\CAPPHI$ be any \GQL{}1 sentence.  If there are two models for $\CAPPHI$, $\As{}{1}$ and $\As{}{2}$, that have the same domain, $\integer{U}$, and make the same assignments for all the sentence letters, individual constants, and 1-place predicates contained in $\CAPPHI$, then $\CAPPHI$ is true on $\As{}{1}$ \Iff $\CAPPHI$ is true on $\As{}{2}$.
\end{THEOREM}	
\begin{PROOF}
	\begin{description}
		\item[Base Step:]  Let $\CAPTHETA$ be a sentence of order 1.  $\CAPTHETA$ is either (i) a sentence letter, or (ii), a 1-place predicate followed by a constant.
		
		(i) $\CAPTHETA$ is a sentence letter.
		We assume that $\As{}{1}$ and $\As{}{2}$ make the same assignments to all the sentence letters.
		Then $\CAPTHETA$ is true on $\As{}{1}$ \Iff $\CAPTHETA$ is true on $\As{}{2}$.
		
		(ii) $\CAPTHETA=\PP\variable{t}$, where $\PP$ is a 1-place predicate and $\variable{t}$ is a constant.
		We assume that $\As{}{1}$ and $\As{}{2}$ make the same assignments to all the constants and 1-place predicates.
		Then $\As{}{1}(\PP)=\As{}{2}(\PP)$ and $\As{}{1}(\variable{t})=\As{}{2}(\variable{t})$.
		It follows that $\As{}{1}(\variable{t})\in\As{}{1}(\PP)$ \Iff $\As{}{2}(\variable{t})\in\As{}{2}(\PP)$.
		Thus, $\CAPTHETA$ is true on $\As{}{1}$ \Iff $\CAPTHETA$ is true on $\As{}{2}$.
		
		\item[Inheritance Step:]  \hfill{}
		\begin{description}
			\item[Recursive Assumption:] Assume that $\CAPTHETA$ is a \GQL{}1 sentence of order $n$.
			Assume also that $\As{}{1}$ and $\As{}{2}$ are any two models for $\CAPTHETA$ that make the same assignments to all the sentence letters, individual constants, and 1-place predicates contained in $\CAPTHETA$.
			Then let $\CAPTHETA$ be true on $\As{}{1}$ \Iff $\CAPTHETA$ is true on $\As{}{2}$.
			
			\item[Negation, Conditional, Biconditional, Disjunction, Conjunction:]  The reasoning for these is the same as in the corresponding clauses of theorem \ref{thm:localityoftruth} in chapter \ref{sententiallogic}.
			
			\item[Quantifier Preface:] Let $\universal{\ALPHA}\CAPTHETA$ and $\existential{\ALPHA}\CAPTHETA$ be sentences of order $n+1$.
			By RA, $\model{}{1}$ and $\model{}{2}$ have the same universe, $\integer{U}$.
			It follows that for each element $u\in\integer{U}$, there is a $\variable{t}$-variant of $\model{}{1}$, $\model{\variable{t}}{1}$, and a $\variable{t}$-variant of $\model{}{2}$, $\model{\variable{t}}{2}$, such that $\model{\variable{t}}{1}(t)=\model{\variable{t}}{2}(t)=u$.
			Since $\As{}{1}$ is a model for $\universal{\ALPHA}\CAPTHETA$ and $\existential{\ALPHA}\CAPTHETA$, each $\model{\variable{t}}{1}$ is a model for $\CAPTHETA\variable{t}/\ALPHA$.
			By similar reasoning each $\model{\variable{t}}{2}$ is a model for $\CAPTHETA\variable{t}/\ALPHA$.
			$\CAPTHETA\variable{t}/\ALPHA$ is of order $n$.
			So by RA, for each pair of $\variable{t}$-variants such that $\model{\variable{t}}{1}(t)=\model{\variable{t}}{2}(t)$, $\CAPTHETA\variable{t}/\ALPHA$ is true on $\model{\variable{t}}{1}$ \Iff it's true on $\model{\variable{t}}{2}$.

			\begin{description}

				\item[Universal Quantification:] ($\Rightarrow$) Let $\universal{\ALPHA}\CAPTHETA$ be true on $\As{}{1}$.
				Then, by the definition of truth of $\forall$, all $\variable{t}$-variants of $\model{}{1}$ make $\CAPTHETA\variable{t}/\ALPHA$ true.
				Since by the quantifier preface there is a corresponding $\variable{t}$-variant of $\model{}{2}$ for each $\variable{t}$-variant of $\model{}{1}$, it follows that all $\variable{t}$-variants of $\model{}{2}$ make $\CAPTHETA\variable{t}/\ALPHA$ true.
				So, by the definition of truth for $\forall$, $\universal{\ALPHA}\CAPTHETA$ is true on $\As{}{2}$.

				($\Leftarrow$) Let $\universal{\ALPHA}\CAPTHETA$ be true on $\As{}{2}$.
				Then, by the definition of truth of $\forall$, all $\variable{t}$-variants of $\model{}{2}$ make $\CAPTHETA\variable{t}/\ALPHA$ true.
				Since by the quantifier preface there is a corresponding $\variable{t}$-variant of $\model{}{1}$ for each $\variable{t}$-variant of $\model{}{2}$, it follows that all $\variable{t}$-variants of $\model{}{1}$ make $\CAPTHETA\variable{t}/\ALPHA$ true.
				So, by the definition of truth for $\forall$, $\universal{\ALPHA}\CAPTHETA$ is true on $\As{}{1}$.

				Thus $\universal{\ALPHA}\CAPTHETA$ is true on $\As{}{1}$ \Iff $\universal{\ALPHA}\CAPTHETA$ is true on $\As{}{2}$.

				\item[Existential Quantification:] ($\Rightarrow$) Let $\existential{\ALPHA}\CAPTHETA$ be true on $\As{}{1}$.
				Then, by the definition of truth of $\exists$, there is some $\variable{t}$-variant of $\model{}{1}$ that makes $\CAPTHETA\variable{t}/\ALPHA$ true.
				Since by the quantifier preface there is a corresponding $\variable{t}$-variant of $\model{}{2}$ for each $\variable{t}$-variant of $\model{}{1}$, it follows that there is some $\variable{t}$-variant of $\model{}{2}$ that makes $\CAPTHETA\variable{t}/\ALPHA$ true.
				So, by the definition of truth for $\exists$, $\existential{\ALPHA}\CAPTHETA$ is true on $\As{}{2}$.

				($\Leftarrow$) Let $\existential{\ALPHA}\CAPTHETA$ be true on $\As{}{2}$.
				Then, by the definition of truth of $\exists$, there is some $\variable{t}$-variant of $\model{}{2}$ that makes $\CAPTHETA\variable{t}/\ALPHA$ true.
				Since by the quantifier preface there is a corresponding $\variable{t}$-variant of $\model{}{1}$ for each $\variable{t}$-variant of $\model{}{2}$, it follows that there is some $\variable{t}$-variant of $\model{}{1}$ that makes $\CAPTHETA\variable{t}/\ALPHA$ true.
				So, by the definition of truth for $\exists$, $\existential{\ALPHA}\CAPTHETA$ is true on $\As{}{1}$.

				Thus $\existential{\ALPHA}\CAPTHETA$ is true on $\As{}{1}$ \Iff $\existential{\ALPHA}\CAPTHETA$ is true on $\As{}{2}$.
			\end{description}
			
		\end{description}
		\item[Closure Step:] There is no other way to form a \GQL{}1 sentence $\CAPPHI$, so the above clauses are sufficient to show that if $\As{}{1}$ and $\As{}{2}$ have the same domain and make the same assignments, then $\CAPPHI$ is true on $\As{}{1}$ \Iff $\CAPPHI$ is true on $\As{}{2}$.
	\end{description}
\end{PROOF}

\subsection{Logical Truth: QT, QF, \& QC}\label{QT QT QI}
Just as we have the notions of \emph{logical truth}, \emph{falsity}, and \emph{contingency} for \GSL{} (see section \ref{TFT TFF TFI}), we have analogous notions for \GQL{}1.
\begin{majorILnc}{\LnpDC{QT}}
A sentence $\CAPPHI$ of \GQL{} is \nidf{quantificationally true}\index{truth!quantificational|textbf} (\CAPS{qt})\index{QT|see{truth, quantificational}} iff it is true on every model for $\CAPPHI$.
\end{majorILnc} 
\begin{majorILnc}{\LnpDC{QF}}
A sentence $\CAPPHI$ of \GQL{} is \nidf{quantificationally false}\index{falsehood!quantificational|textbf} (\CAPS{qf})\index{QF|see{falsehood, quantificational}} iff it is false on every model for $\CAPPHI$.
\end{majorILnc} 
\begin{majorILnc}{\LnpDC{QI}}
A sentence $\CAPPHI$ of \GQL{} is \nidf{quantificationally contingent}\index{indeterminate!quantificational|textbf} (\CAPS{qc})\index{QI|see{indeterminate, quantificational}} iff there's at least one model $\As{}{1}$ on which it's true and at least one model $\As{}{2}$ on which it's false.
\end{majorILnc} 
\begin{majorILnc}{\LnpEC{GQL1LogicallTruthExampleA}}
	Prove that $\universal{\variable{x}}\pardisjunction{\Bp{\variable{x}}}{\negation{\Bp{\variable{x}}}}$ is \CAPS{qt}.
\end{majorILnc}
\begin{PROOF}
	$\universal{\variable{x}}\pardisjunction{\Bp{\variable{x}}}{\negation{\Bp{\variable{x}}}}$ is true on $\IntA$ \Iff every $\constant{a}$-variant of $\IntA$, $\As{\constant{a}}{}$, makes $\disjunction{\Bp{\constant{a}}}{\negation{\Bp{\constant{a}}}}$ true.
	One of two things must be true of each $\As{\constant{a}}{}$.
	Either $\As{\constant{a}}{}(\constant{a})\in\As{\constant{a}}{}(\BB)$ or $\As{\constant{a}}{}(\constant{a})\notin\As{\constant{a}}{}(\BB)$.
	If $\As{\constant{a}}{}(\constant{a})\in\As{\constant{a}}{}(\BB)$, then $\Bp{\constant{a}}$ is true on $\As{\constant{a}}{}$, and so is $\disjunction{\Bp{\constant{a}}}{\negation{\Bp{\constant{a}}}}$.
	If $\As{\constant{a}}{}(\constant{a})\notin\As{\constant{a}}{}(\BB)$, then $\negation{\Bp{\constant{a}}}$ is true on $\As{\constant{a}}{}$, and so is $\disjunction{\Bp{\constant{a}}}{\negation{\Bp{\constant{a}}}}$.
	So, $\disjunction{\Bp{\constant{a}}}{\negation{\Bp{\constant{a}}}}$ is true on every $\As{\constant{a}}{}$.
	It follows that $\universal{\variable{x}}\pardisjunction{\Bp{\variable{x}}}{\negation{\Bp{\variable{x}}}}$ is true on $\IntA$.
	Nothing particular was assumed about $\IntA$, so $\universal{\variable{x}}\pardisjunction{\Bp{\variable{x}}}{\negation{\Bp{\variable{x}}}}$ is true on all models.
	Therefore, $\universal{\variable{x}}\pardisjunction{\Bp{\variable{x}}}{\negation{\Bp{\variable{x}}}}$ is \CAPS{qt}.
\end{PROOF}

\begin{majorILnc}{\LnpEC{GQL1LogicallTruthExampleA2}}
	Prove that $\universal{\variable{x}}\parconjunction{\Bp{\variable{x}}}{\negation{\Bp{\variable{x}}}}$ is \CAPS{qf}.
\end{majorILnc}
\begin{PROOF}
	$\universal{\variable{x}}\parconjunction{\Bp{\variable{x}}}{\negation{\Bp{\variable{x}}}}$ is true on $\IntA$ \Iff every $\constant{a}$-variant of $\IntA$, $\As{\constant{a}}{}$, makes $\conjunction{\Bp{\constant{a}}}{\negation{\Bp{\constant{a}}}}$ true.
	There is some $\constant{a}$-variant $\As{\constant{a}}{}$ such that either $\As{\constant{a}}{}(\constant{a})\in\As{\constant{a}}{}(\BB)$ or $\As{\constant{a}}{}(\constant{a})\notin\As{\constant{a}}{}(\BB)$.
	If $\As{\constant{a}}{}(\constant{a})\in\As{\constant{a}}{}(\BB)$ then $\negation{\Bp{\constant{a}}}$ is false on $\As{\constant{a}}{}$, and so is $\conjunction{\Bp{\constant{a}}}{\negation{\Bp{\constant{a}}}}$.
	Otherwise $\As{\constant{a}}{}(\constant{a})\notin\As{\constant{a}}{}(\BB)$, in which case $\Bp{\constant{a}}$ is false on $\As{\constant{a}}{}$, and so is $\conjunction{\Bp{\constant{a}}}{\negation{\Bp{\constant{a}}}}$.
	In both cases it follows that $\universal{\variable{x}}\parconjunction{\Bp{\variable{x}}}{\negation{\Bp{\variable{x}}}}$ is false on $\IntA$.
	Nothing particular was assumed about $\IntA$, so $\universal{\variable{x}}\parconjunction{\Bp{\variable{x}}}{\negation{\Bp{\variable{x}}}}$ is false on all models.
	Therefore, $\universal{\variable{x}}\parconjunction{\Bp{\variable{x}}}{\negation{\Bp{\variable{x}}}}$ is \CAPS{qf}.
\end{PROOF}

\noindent{}In the last two examples we give proofs that the sentences in question are either \CAPS{qt} or \CAPS{qf}.
To show that a \GQL{}1 sentence $\CAPPHI$ is \CAPS{qc}, we use a different strategy: provide one model on which $\CAPPHI$ is true and another on which $\CAPPHI$ is false.

For example, we determined earlier that $\horseshoe{\universal{\variable{x}}\Dp{\variable{x}}}{\universal{\variable{x}}\Gp{\variable{x}}}$ is true on the model \emph{Pos Int}.
To show that it's \CAPS{qc} we must provide another model in which it's false.
We can modify \emph{Pos Int} to make a new model, \emph{Pos Int$^*$}.  Let \emph{Pos Int$^*$} assign the entire domain to $\DD$ so that it makes $\universal{\variable{x}}\Dp{\variable{x}}$ true.  We can keep the assignment \emph{Pos Int} makes to $\GG$, the multiples of 7.  So, on \emph{Pos Int$^*$}, $\universal{\variable{x}}\Gp{\variable{x}}$ is false.  Thus, $\horseshoe{\universal{\variable{x}}\Dp{\variable{x}}}{\universal{\variable{x}}\Gp{\variable{x}}}$ is false on \emph{Pos Int$^*$}.

The definition of model only requires that predicates be assigned a subset of the domain.
The assigned subset can be the entire domain or the empty set.
You will find that using the empty set or the entire domain as assignments is often helpful in constructing models for a desired outcome.

\begin{majorILnc}{\LnpEC{GQL1LogicallTruthExampleB}}
	The sentence $\disjunction{\universal{\variable{x}}\Bp{\variable{x}}}{\universal{\variable{x}}\negation{\Bp{\variable{x}}}}$ is \CAPS{qc}.
\end{majorILnc}
\begin{PROOF}
Let $\IntA_1$ be a model such that the universe $\integer{U}=\{1\}$ and $\IntA_1(\BB)=\{1\}$. 
$\IntA_1$ makes $\universal{\variable{x}}\Bp{\variable{x}}$ true, which in turn makes $\disjunction{\universal{\variable{x}}\Bp{\variable{x}}}{\universal{\variable{x}}\negation{\Bp{\variable{x}}}}$ true.
\begin{commentary}
	When we remove the quantifier from $\universal{\variable{x}}\Bp{\variable{x}}$ and replace the variable with a constant, the result is $\Bp{\constant{a}}$.
	Everything in the universe is also in the set assigned to $\BB$.
	So, no matter what an $\constant{a}$-variant assigns to the constant, the result is true.
\end{commentary}
\noindent{}Let $\IntA_2$ be a model such that the universe $\integer{U}=\{1,2\}$ and $\IntA_2(\BB)=\{1\}$.
$\universal{\variable{x}}\Bp{\variable{x}}$ is false on $\IntA_2$ because not everything in the domain is in the set assigned to $\BB$.
And $\universal{\variable{x}}\negation{\Bp{\variable{x}}}$ is false on $\IntA_2$ because the model doesn't make $\BB$ the empty set.
Then $\disjunction{\universal{\variable{x}}\Bp{\variable{x}}}{\universal{\variable{x}}\negation{\Bp{\variable{x}}}}$ is false in $\IntA_2$, because both disjuncts are false.  
Therefore, the sentence $\disjunction{\universal{\variable{x}}\Bp{\variable{x}}}{\universal{\variable{x}}\negation{\Bp{\variable{x}}}}$ is \CAPS{qc}.
\end{PROOF}

\begin{majorILnc}{\LnpEC{GQL1LogicallTruthExampleC}}
Prove that $\horseshoe{\universal{\variable{x}}\Dp{\variable{x}}}{\negation{\existential{\variable{x}}\negation{\Dp{\variable{x}}}}}$ is \CAPS{qt}.
\end{majorILnc}
\begin{PROOF}
Assume for indirect proof there is some model $\IntA$ that makes $\horseshoe{\universal{\variable{x}}\Dp{\variable{x}}}{\negation{\existential{\variable{x}}\negation{\Dp{\variable{x}}}}}$ false.
Thus, $\universal{\variable{x}}\Dp{\variable{x}}$ is true on $\IntA$ and $\negation{\existential{\variable{x}}\negation{\Dp{\variable{x}}}}$ is false on $\IntA$. 
From the truth of $\universal{\variable{x}}\Dp{\variable{x}}$ it follows that every $\constant{a}$-variant of $\IntA$ makes $\Dp{\constant{a}}$ true.
Because $\IntA$ makes $\negation{\existential{\variable{x}}\negation{\Dp{\variable{x}}}}$ false, it follows that $\existential{\variable{x}}\negation{\Dp{\variable{x}}}$ is true.  Given that $\IntA$ makes $\existential{\variable{x}}\negation{\Dp{\variable{x}}}$ true, there must be some $\constant{a}$-variant, $\As{\constant{a}}{}$, that makes $\negation{\Dp{\constant{a}}}$ true.
On $\As{\constant{a}}{}$, $\Dp{\constant{a}}$ must be false.
But it was already shown that every $\constant{a}$-variant of $\IntA$ makes $\Dp{\constant{a}}$ true.
Thus there is no model such that $\horseshoe{\universal{\variable{x}}\Dp{\variable{x}}}{\negation{\existential{\variable{x}}\negation{\Dp{\variable{x}}}}}$ is false.
It is therefore true on all models, and so is \CAPS{qt}.
\end{PROOF}


%%%%%%%%%%%%%%%%%%%%%%%%%%%%%%%%%%%%%%%%%%%%%%%%%%
\section{Entailment and other Relations}\label{GQL1 Entailment and other Relations}
%%%%%%%%%%%%%%%%%%%%%%%%%%%%%%%%%%%%%%%%%%%%%%%%%%

We defined the following terms in \GSL{}: entailment, equivalence, contradictory, contrary, subcontrary, and logical independence. 
Now we define these for \GQL{}1 sentences. 
While the \GSL{} definitions refer to models of \GSL{} sentences, their \GQL{}1 counterparts refer to models of \GQL{}1 sentences. 
To mark this difference, we talk of \emph{truth functional} entailment, \emph{truth functional} equivalence, etc., for \GSL{} sentences, and \emph{quantificational} entailment, \emph{quantificational} equivalence, etc., for \GQL{}1. 
We don't want to overstate the difference; the concepts underlying the definitions are essentially the same.

\begin{majorILnc}{\LnpDC{GQL1 Definition of Entailment}}
 A set $\Delta$ of \GQL{}1 sentences \df{quantificationally entails} a \GQL{}1 sentence $\CAPTHETA$ \Iff every model for $\Delta$ and $\CAPTHETA$ either makes at least one sentence in $\Delta$ $\False$ or makes $\CAPTHETA$ $\True$.
\end{majorILnc}

\noindent{}Put another way, $\Delta$ entails $\CAPTHETA$ \Iff every model that makes all sentences in $\Delta$ $\True$ also makes $\CAPTHETA$ $\True$.
As we did with \GSL{} (section \ref{Entailment}), we give two narrower consequences of this definition. 

\begin{cenumerate}
\item A finite set of \GQL{}1 sentences $\CAPPHI_1,\ldots,\CAPPHI_{\integer{n}}$ quantificationally entails another \GQL{}1 sentence $\CAPTHETA$ \Iff every model for $\CAPPHI_1,\ldots,\CAPPHI_{\integer{n}}$, and $\CAPTHETA$ either makes at least one of $\CAPPHI_1,\ldots,\CAPPHI_{\integer{n}}$ $\False$ or makes $\CAPTHETA$ $\True$.
\item A sentence $\CAPPHI$ of \GQL{} quantificationally entails another sentence $\CAPTHETA$ of \GQL{}1 \Iff every model for $\CAPPHI$ and $\CAPTHETA$ either makes $\CAPPHI$ $\False$ or makes $\CAPTHETA$ $\True$.
\end{cenumerate}

\noindent{}We continue to use the double turnstile to represent the entailment relation. 
So if $\CAPPHI$ entails $\CAPTHETA$, we write \mention{$\CAPPHI\sdtstile{}{}\CAPTHETA$}. 
If the sentences $\CAPPHI_1,\ldots,\CAPPHI_{\integer{n}}$ entail $\CAPTHETA$, we write \mention{$\CAPPHI_1,\ldots,\CAPPHI_{\integer{n}}\sdtstile{}{}\CAPTHETA$}.
And if a set $\Delta$ of sentences entails $\CAPTHETA$ we write \mention{$\Delta\sdtstile{}{}\CAPTHETA$}.

\begin{majorILnc}{\LnpEC{GQL Entailment Example}}
	Show whether the following holds: $\universal{\variable{x}}\Gp{\variable{x}}\sdtstile{}{}\Gp{\constant{a}}$.
\end{majorILnc}
\begin{PROOF}
	Assume a model $\IntA$ such that $\universal{\variable{x}}\Gp{\variable{x}}$ is true.
	By the definition of truth for $\forall$, it follows that $\Gp{\constant{a}}$ is true on all $\constant{a}$-variants of $\IntA$.
	The model $\IntA$ is an $\constant{a}$-variant of itself, so $\Gp{\constant{a}}$ is true on $\IntA$.
	Nothing was assumed about the assignments $\IntA$ makes, so for each model on which the \CAPS{lhs} is true, the \CAPS{rhs} is also true.
	Therefore $\universal{\variable{x}}\Gp{\variable{x}}\sdtstile{}{}\Gp{\constant{a}}$.
\end{PROOF}
\begin{majorILnc}{\LnpEC{GQL Entailment Example 2}}
	Show that $\universal{\variable{x}}\Gp{\variable{x}}\sdtstile{}{}\Gp{\constant{b}}$.
	\begin{commentary}
		This entailment is slightly harder to prove than the last.
		It's a quirk of our definition of truth that makes the last so easy to establish.
		By changing the constant in the sentence on the \CAPS{rhs} from \mention{$\constant{a}$} to \mention{$\constant{b}$}, we add a few steps to our proof.
	\end{commentary}
\end{majorILnc}
\begin{PROOF}
	Assume a model $\IntA$ such that $\universal{\variable{x}}\Gp{\variable{x}}$ is true.
	Then $\Gp{\constant{a}}$ is true on all $\constant{a}$-variants of $\IntA$.
	All $\constant{a}$-variants of $\IntA$ have the same universe as $\IntA$.
	So there is some $\constant{a}$-variant, $\As{\constant{a}}{}$, such that $\As{\constant{a}}{}(a)=\IntA(\constant{b})$.
	It follows that $\As{\constant{a}}{}(\constant{a})\in\As{\constant{a}}{}(\GG)$.
	$\As{\constant{a}}{}$ and $\IntA$ assign the same set to $\GG$, so $\As{\constant{a}}{}(\constant{a})\in\As{}{}(\GG)$.
	And because $\As{\constant{a}}{}(\constant{a})$ is the same element as $\IntA(\constant{b})$, it follows that $\As{}{}(\constant{b})\in\As{}{}(\GG)$.
	And so $\Gp{\constant{b}}$ is true on $\IntA$.
	Nothing was assumed about the assignments $\IntA$ makes, so for each model on which the \CAPS{lhs} is true, the \CAPS{rhs} is also true.
	Therefore $\universal{\variable{x}}\Gp{\variable{x}}\sdtstile{}{}\Gp{\constant{b}}$.
\end{PROOF}

\begin{majorILnc}{\LnpEC{GQL1Entailment}}
	Show whether $\universal{\variable{x}}\parhorseshoe{\Cp{\variable{x}}}{\Dp{\variable{x}}}, \Cp{\constant{o}}\sdtstile{}{}\Dp{\constant{o}}$.
\end{majorILnc}
\begin{PROOF}
	The entailment holds.
	Assume some model $\IntA$ such that $\universal{\variable{x}}\parhorseshoe{\Cp{\variable{x}}}{\Dp{\variable{x}}}$ and $\Cp{\constant{o}}$ are true.
	By the definition of truth for $\forall$, it follows that $\parhorseshoe{\Cp{\constant{a}}}{\Dp{\constant{a}}}$ is true on all $\constant{a}$-variants of $\IntA$.
	All $\constant{a}$-variants of $\IntA$ have the same universe as $\IntA$.
	So there is an $\constant{a}$-variant such that $\As{\constant{a}}{}(\constant{a})=\IntA(\constant{o})$.
	
	The model $\IntA$ makes $\Cp{\constant{o}}$ true, so $\IntA(\constant{o})\in\IntA(\Cp{})$.
	Because $\IntA(\constant{o})=\As{\constant{a}}{}(\constant{a})$ and $\As{\constant{a}}{}(\Cp{})=\IntA(\Cp{})$, it follows by substitution that $\As{\constant{a}}{}(\constant{a})\in\As{\constant{a}}{}(\Cp{})$.
	Hence, $\As{\constant{a}}{}$ makes $\Cp{\constant{a}}$ true.
	And because $\parhorseshoe{\Cp{\constant{a}}}{\Dp{\constant{a}}}$ is true on $\As{\constant{a}}{}$, it follows that $\As{\constant{a}}{}$ makes $\Dp{\constant{a}}$ true.
	From this it follows that $\As{\constant{a}}{}(\constant{a})\in\As{\constant{a}}{}(\Dp{})$.  $\As{\constant{a}}{}(\constant{a})=\IntA(\constant{o})$ and $\As{\constant{a}}{}(\Dp{})=\IntA(\Dp{})$, so by substitution we get: $\IntA(\constant{o})\in\IntA(\Dp{})$.
	Thus, $\IntA$ makes $\Dp{\constant{o}}$ true.

	Any model that makes the LHS true also makes the RHS true.  Therefore the entailment holds.
\end{PROOF}	
\begin{commentary}
	This entailment resembles the argument discussed at the beginning of the chapter: (1) All women are mortal, (2) Ophelia is a woman, therefore (3) Ophelia is mortal.
	To see this, interpret $\constant{o}$ as Ophelia, $\CC$ as the set of women, and $\DD$ as the set of mortals.
\end{commentary}

Consider an entailment $\Delta\sdtstile{}{}\CAPPHI$ that holds when the set $\Delta$ is empty, i.e. such that $\sdtstile{}{}\CAPPHI$.
By the definition of $\sdtstile{}{}$ every model must either make a sentence on the left false or the sentence on the right true.
But in this case there are no sentences on the \CAPS{lhs}.
So every model must make the \CAPS{rhs}, $\CAPPHI$, true.
Therefore, as was the case with \GSL{} in chapter \ref{sententiallogic}, $\sdtstile{}{}\CAPPHI$ \Iff $\CAPPHI$ is \CAPS{qt}.

\begin{majorILnc}{\LnpDC{GQL1 TFE}}
Two \GQL{}1 sentences $\CAPTHETA$ and $\CAPPHI$ are \nidf{quantificationally equivalent}\index{equivalent sentences!quantificational|textbf} \Iff all models for $\CAPTHETA$ and $\CAPPHI$ assign them the same truth value.
\end{majorILnc}
\begin{majorILnc}{\LnpDC{GQL1 contradictory}}
Two \GQL{}1 sentences $\CAPTHETA$ and $\CAPPHI$ are \nidf{quantificationally contradictory}\index{contradictory!quantificational|textbf} \Iff all models for $\CAPTHETA$ and $\CAPPHI$ assign them opposite truth values.
\end{majorILnc}
\begin{majorILnc}{\LnpDC{GQL1 contrary}}
Two \GQL{}1 sentences $\CAPTHETA$ and $\CAPPHI$ are \nidf{quantificationally contrary}\index{contraries!quantificational|textbf} \Iff they cannot both be $\True$ in the same model $\IntA$.
\end{majorILnc}
\begin{majorILnc}{\LnpDC{GQL1 subcontrary}}
Two \GQL{}1 sentences $\CAPTHETA$ and $\CAPPHI$ are \nidf{quantificationally subcontrary}\index{subcontraries!quantificational|textbf} \Iff they cannot both be $\False$ in the same model $\IntA$.
\end{majorILnc}
\begin{majorILnc}{\LnpDC{GQL1 Independent}}
Two \GQL{}1 sentences $\CAPTHETA$ and $\CAPPHI$ are \nidf{quantificationally independent}\index{independent sentences!quantificational|textbf} \Iff there are four models:
\begin{cenumerate}
	\item A model in which both $\CAPTHETA$ and $\CAPPHI$ are $\True$; 
	\item A model in which both $\CAPTHETA$ and $\CAPPHI$ are $\False$;
	\item A model in which $\CAPTHETA$ is $\True$ and $\CAPPHI$ is $\False$; and
	\item A model in which $\CAPTHETA$ is $\False$ and $\CAPPHI$ is $\True$.
\end{cenumerate}
\end{majorILnc}

First, a minor note.
These definitions only make sense for \GQL{}1 sentences.
We do not assess formulas that \emph{aren't} sentences for truth value, so none of the definitions above apply to them. 

Except for the fact that these definitions apply to sentences of \GQL{}1 instead of \GSL{}, and models for \GQL{}1 sentences instead of models for \GSL{} sentences, they are the same as the corresponding ones for \GSL{}. 
We might say that these definitions have the same \sq{structure}. 
The \emph{ideas} of equivalence, being contradictory, etc., haven't changed, even though the details of the definitions are slightly different.

The following four facts from \GSL{} also hold in \GQL{}1 (compare with the examples in section \ref{Other Relations}):

\begin{cenumerate}
	\item Contradictory sentences are also contrary, but sentences can be contrary without being contradictory: e.g. $\conjunction{\Cl}{\Dl}$ and $\conjunction{\Cl}{\negation{\Dl}}$.
	\item Contradictory sentences are also subcontrary, but sentences can be subcontrary without being contradictory: e.g. $\Dl$ and $\disjunction{\Cl}{\negation{\Dl}}$.
	\item If two sentences are both contrary and subcontrary, they are contradictory.
	\item Any two atomic sentences are independent of each other.
\end{cenumerate}

\noindent{}Because every sentence of \GSL{} is also a sentence of \GQL{}1 we can reuse the examples from the previous chapter. 

Finally, recall that in \GSL{} we have theorem \pmvref{Exponentiation of Entailment}: for all \GSL{} sentences $\CAPPHI$ and $\CAPTHETA$, $\CAPPHI\sdtstile{}{}\CAPTHETA$ \Iff $\sdtstile{}{}\parhorseshoe{\CAPPHI}{\CAPTHETA}$. 
The same theorem holds for \GQL{}1 sentences and the proof is essentially the same. 
\begin{THEOREM}{\LnpTC{Exponentiation of Entailment GQL} \GQL{}1 Exportation Theorem:} For all \GQL{}1 sentences $\CAPPHI$ and $\CAPTHETA$, $\CAPPHI\sdtstile{}{}\CAPTHETA$ \Iff $\:\sdtstile{}{}\parhorseshoe{\CAPPHI}{\CAPTHETA}$.
\end{THEOREM}
\noindent{}In addition, all the generalizations of this theorem (\ref{expo generalizations}) also hold for \GQL{}1 sentences, and again the proofs are essentially same.


%%%%%%%%%%%%%%%%%%%%%%%%%%%%%%%%%%%%%%%%%%%%%%%%%%
\section{Exercises}
%%%%%%%%%%%%%%%%%%%%%%%%%%%%%%%%%%%%%%%%%%%%%%%%%%

\notocsubsection{Formulas, Order, and Subformulas}{ex:Formulas, Order, and Subformulas1} Which of the following are \GQL{}1 \emph{formulas}? 
For those that are formulas, what is their order? 
How many subformulas does each have?
\begin{multicols}{2}
\begin{enumerate}
\item {$\universal{\variable{x}}\parhorseshoe{\Hpp{'}{\variable{x}}}{\Gpp{'}{\variable{x}}}$}
\item {$\universal{\variable{x}}\parhorseshoe{\Hpp{'}{\variable{x}}}{\Gpp{''}{\variable{x}}}$}
\item {$\universal{\variable{x}}\parhorseshoe{\Hpp{'}{\variable{x}}}{\Gpp{'_7}{\variable{x}}}$}
\item {$\universal{\variable{x}}\universal{\variable{z}}\parhorseshoe{\Hpp{'}{\variable{x}}}{\Gppp{''}{\variable{x}}{\variable{y}}}$}
\item {$\existential{\variable{y}}\universal{\variable{x}}\parhorseshoe{\Hpp{'}{\variable{x}}}{\Gpp{'}{\variable{x}}}$}
\item {$\universal{\variable{t}}\parhorseshoe{\Hpp{'}{\variable{x}}}{\Gpp{'}{\variable{x}}}$}
\item {$\disjunction{\Hpp{'}{\variable{x}}}{\Gpp{'}{\variable{x}}}$}
\item {$\universal{\variable{x}}\parconjunction{\Hpp{'}{\variable{y}}}{\Gpp{'}{\variable{z}}}$}
\end{enumerate}
\end{multicols}


\begin{longtable}[c]{ l l l l } %p{2.2in} p{2in}
	\toprule
	&\textbf{Symbol} & \multicolumn{2}{c}{\textbf{Model}} \\ \cmidrule(l){3-4}
	& & \textbf{Pos Int} & \textbf{States} \\
	\midrule 
	\endfirsthead
	\multicolumn{4}{c}{\emph{Continued from Previous Page}}\\
	\toprule
	&\textbf{Symbol} & \multicolumn{2}{c}{\textbf{Model}} \\ \cmidrule(l){3-4}
	& & \textbf{Pos Int} & \textbf{States} \\
	\midrule 
	\endhead
	\bottomrule
	\caption{Example Models}\\[-.15in]
	\multicolumn{4}{c}{\emph{Continued next Page}}\\
	\endfoot
	\bottomrule
	\caption{Example Models}\\%
	\endlastfoot%
	\label{table:Partial Models Again}%
	%\begin{tabular}{ l l l l } %p{2in} p{2in} %\begin{tabular}{ p{1in} l l } %p{2.2in} p{2in}
	%\toprule
	%&\textbf{Symbol} & \multicolumn{2}{c}{\textbf{Interpretation}} \\ \cmidrule(l){3-4}
	%& & \textbf{Pos Int} & \textbf{States} \\
	%\midrule 
	{Universe:} & & The set of positive integers & The set of US states (2024) \\ \addlinespace[.25cm]
	{Sent. Let.:}& A&$\True$&$\False$\\
	& B&$\True$&$\False$\\
	& C&$\False$&$\True$\\
	& D&$\True$&$\False$\\
	& E&$\True$&$\False$\\
	& G&$\False$&$\True$\\ \addlinespace[.25cm]
	{Constants:}&$\constant{a}$&1&Louisiana\\
	&$\constant{b}$&9&Maine\\
	&$\constant{c}$&72&Georgia\\
	&$\constant{d}$&3&Nebraska\\
	&$\constant{e}$&1&New Mexico\\
	&$\constant{f}$&2&Texas\\ \addlinespace[.25cm]
	{1-place:}&$\Ap{'}$&all pos int&Midwestern\\
	&$\Bp{'}$&empty set&name with $>5$ letters\\
	&$\Cp{'}$&even&Coastal\\
	&$\Dp{'}$&odd&on the Pacific coast\\
	&$\Ep{'}$&prime&\{Ohio\}\\
	&$\Gp{'}$&multiple of 7&\{Ohio,Alabama\}\\ \addlinespace[.25cm]
	%\bottomrule
\end{longtable}

\notocsubsection{Truth in a Model}{ex:Truth in an Interpretation1} Give the truth value of each of the following sentences on both of the models found in table \mvref{table:Partial Models Again}. 
\begin{multicols}{2}
\begin{enumerate}
\item $\existential{\variable{x}}\Gp{\variable{x}}$
\item $\negation{\existential{\variable{x}}\Gp{\variable{x}}}$
\item $\existential{\variable{x}}\negation{\Gp{\variable{x}}}$
\item $\universal{\variable{x}}\Gp{\variable{x}}$
\item $\negation{\universal{\variable{x}}\Gp{\variable{x}}}$
\item $\universal{\variable{x}}\negation{\Gp{\variable{x}}}$
\item $\conjunction{\existential{\variable{x}}\Cp{\variable{x}}}{\existential{\variable{x}}\Dp{\variable{x}}}$
\item $\existential{\variable{x}}\parconjunction{\Cp{\variable{x}}}{\Dp{\variable{x}}}$
\item $\negation{\existential{\variable{x}}\parconjunction{\Cp{\variable{x}}}{\Dp{\variable{x}}}}$
\item $\universal{\variable{x}}\parconjunction{\Cp{\variable{x}}}{\Dp{\variable{x}}}$
\item $\universal{\variable{x}}\parhorseshoe{\Cp{\variable{x}}}{\Dp{\variable{x}}}$
\item $\horseshoe{\universal{\variable{x}}\Cp{\variable{x}}}{\universal{\variable{x}}\Dp{\variable{x}}}$
\item $\negation{\universal{\variable{x}}\parhorseshoe{\Cp{\variable{x}}}{\Dp{\variable{x}}}}$
\item $\existential{\variable{x}}\parhorseshoe{\Cp{\variable{x}}}{\Dp{\variable{x}}}$
\end{enumerate}
\end{multicols}

\notocsubsection{Quantificational Truth Problems}{ex:Quantificational Truth Problems} 
For each sentence below, say whether it's a quantificational truth. 
If so, prove it. 
If not, give a model $\IntA$ that makes it false.
\begin{multicols}{2}
\begin{enumerate}
\item {$\disjunction{\universal{\variable{y}}\bparhorseshoe{\Ap{\variable{y}}}{\Bp{\variable{y}}}}{\universal{\variable{y}}\bparhorseshoe{\Bp{\variable{y}}}{\Ap{\variable{y}}}}$}
\item {$\disjunction{\existential{\variable{y}}\bparhorseshoe{\Ap{\variable{y}}}{\Bp{\variable{y}}}}{\existential{\variable{y}}\bparhorseshoe{\Bp{\variable{y}}}{\Ap{\variable{y}}}}$}
\item {$\horseshoe{\universal{\variable{y}}\bparhorseshoe{\Ap{\variable{y}}}{\Bp{\variable{y}}}}{\bparhorseshoe{\existential{\variable{y}}\Ap{\variable{y}}}{\existential{\variable{y}}\Bp{\variable{y}}}}$}
\item {$\horseshoe{\existential{\variable{y}}\bparhorseshoe{\Ap{\variable{y}}}{\Bp{\variable{y}}}}{\bparhorseshoe{\existential{\variable{y}}\Ap{\variable{y}}}{\existential{\variable{y}}\Bp{\variable{y}}}}$}
\item {$\horseshoe{\existential{\variable{y}}\bparhorseshoe{\Ap{\variable{y}}}{\Bp{\variable{y}}}}{\bparhorseshoe{\universal{\variable{y}}\Ap{\variable{y}}}{\universal{\variable{y}}\Bp{\variable{y}}}}$}
\item {$\horseshoe{\universal{\variable{y}}\negation{\Ap{\variable{y}}}}{\negation{\existential{\variable{y}}\Ap{\variable{y}}}}$}
\item {$\horseshoe{\negation{\existential{\variable{y}}\Ap{\variable{y}}}}{\universal{\variable{y}}\negation{\Ap{\variable{y}}}}$}
\item {$\horseshoe{\negation{\universal{\variable{y}}\Ap{\variable{y}}}}{\existential{\variable{y}}\negation{\Ap{\variable{y}}}}$}
\end{enumerate}
\end{multicols}
\begin{enumerate}[start=9]
\item {$\horseshoe{\universal{\variable{y}}\bparhorseshoe{\Ap{\variable{y}}}{\Bp{\variable{y}}}}{\bparhorseshoe{\universal{\variable{y}}\Ap{\variable{y}}}{\universal{\variable{y}}\Bp{\variable{y}}}}$}
\item {$\horseshoe{\universal{\variable{z}}\bparhorseshoe{\Ap{\variable{z}}}{\pardisjunction{\Bp{\variable{z}}}{\Cp{\variable{z}}}}}{\cpardisjunction{\universal{\variable{z}}\bparhorseshoe{\Ap{\variable{z}}}{\Bp{\variable{z}}}}{\universal{\variable{z}}\bparhorseshoe{\Ap{\variable{z}}}{\Cp{\variable{z}}}}}$}
\item {$\horseshoe{\universal{\variable{y}}\bparhorseshoe{\Ap{\variable{y}}}{\Bp{\variable{y}}}}{\cparhorseshoe{\universal{\variable{y}}\bparhorseshoe{\Bp{\variable{y}}}{\Cp{\variable{y}}}}{\universal{\variable{y}}\bparhorseshoe{\Ap{\variable{y}}}{\Cp{\variable{y}}}}}$}
\item {$\horseshoe{\universal{\variable{y}}\bparhorseshoe{\Ap{\variable{y}}}{\Bp{\variable{y}}}}{\cparhorseshoe{\universal{\variable{y}}\bparhorseshoe{\Cp{\variable{y}}}{\Bp{\variable{y}}}}{\universal{\variable{y}}\bparhorseshoe{\Ap{\variable{y}}}{\Cp{\variable{y}}}}}$}
\item {$\horseshoe{\universal{\variable{y}}\bparhorseshoe{\Ap{\variable{y}}}{\Bp{\variable{y}}}}{\cparhorseshoe{\existential{\variable{y}}\bparhorseshoe{\Bp{\variable{y}}}{\Cp{\variable{y}}}}{\universal{\variable{y}}\bparhorseshoe{\Ap{\variable{y}}}{\Cp{\variable{y}}}}}$}
\item {$\horseshoe{\universal{\variable{y}}\bparhorseshoe{\Ap{\variable{y}}}{\Bp{\variable{y}}}}{\cparhorseshoe{\existential{\variable{y}}\bparhorseshoe{\Bp{\variable{y}}}{\Cp{\variable{y}}}}{\existential{\variable{y}}\bparhorseshoe{\Ap{\variable{y}}}{\Cp{\variable{y}}}}}$}
\end{enumerate}


\notocsubsection{Entailment Problems for \GQL{}1}{Entailment Problems for GQL1} For each entailment below, either prove that it holds or show that it doesn't hold by giving a model that make the sentences on the \CAPS{lhs} of the turnstile true and the sentence on the \CAPS{rhs} false.
\begin{multicols}{2}
\begin{enumerate}
\item {$\universal{\variable{y}}\parhorseshoe{\Ap{\variable{y}}}{\Bp{\variable{y}}}\text{, }\universal{\variable{y}}\Ap{\variable{y}}\sdtstile{}{}\universal{\variable{y}}\Bp{\variable{y}}$}
\item {$\universal{\variable{y}}\parhorseshoe{\Ap{\variable{y}}}{\Bp{\variable{y}}}\text{, }\existential{\variable{y}}\Ap{\variable{y}}\sdtstile{}{}\existential{\variable{y}}\Bp{\variable{y}}$}
\item {$\existential{\variable{y}}\parhorseshoe{\Ap{\variable{y}}}{\Bp{\variable{y}}}\text{, }\existential{\variable{y}}\Ap{\variable{y}}\sdtstile{}{}\existential{\variable{y}}\Bp{\variable{y}}$}
\item {$\horseshoe{\universal{\variable{y}}\Ap{\variable{y}}}{\universal{\variable{y}}\Bp{\variable{y}}}\sdtstile{}{}\universal{\variable{y}}\parhorseshoe{\Ap{\variable{y}}}{\Bp{\variable{y}}}$}
\item {$\existential{\variable{y}}\pardisjunction{\Ap{\variable{y}}}{\Bp{\variable{y}}}\sdtstile{}{}\disjunction{\existential{\variable{y}}\Ap{\variable{y}}}{\existential{\variable{y}}\Bp{\variable{y}}}$}
\item {$\existential{\variable{y}}\parhorseshoe{\Ap{\variable{y}}}{\Bp{\variable{y}}}\text{, }\universal{\variable{y}}\Ap{\variable{y}}\sdtstile{}{}\universal{\variable{y}}\Bp{\variable{y}}$}
\end{enumerate}
\end{multicols}
\begin{enumerate}[start=7]
\item {$\universal{\variable{z}}\bparhorseshoe{\Ap{\variable{z}}}{\pardisjunction{\Bp{\variable{z}}}{\Cp{\variable{z}}}}\sdtstile{}{}\cpardisjunction{\universal{\variable{z}}\bparhorseshoe{\Ap{\variable{z}}}{\Bp{\variable{z}}}}{\universal{\variable{z}}\bparhorseshoe{\Ap{\variable{z}}}{\Cp{\variable{z}}}}$}
\item {$\universal{\variable{y}}\parhorseshoe{\Ap{\variable{y}}}{\Bp{\variable{y}}}\text{, }\existential{\variable{y}}\parhorseshoe{\Bp{\variable{y}}}{\Cp{\variable{y}}}\sdtstile{}{}\existential{\variable{y}}\parhorseshoe{\Ap{\variable{y}}}{\Cp{\variable{y}}}$}
\end{enumerate}

\notocsubsection{Relations Between \GQL{}1 Sentences}{ex:Relations Between GQL1 Sentences} For each sentence below, say whether it entails, is entailed by, is equivalent to, contradicts, is contrary to, is subcontrary to, or is independent from each of the other sentences. 
\begin{multicols}{2}
\begin{enumerate}
\item {$\universal{\variable{z}}\parhorseshoe{\Gp{\variable{z}}}{\Dp{\variable{z}}}$}
\item {$\horseshoe{\universal{\variable{z}}\Gp{\variable{z}}}{\universal{\variable{z}}\Dp{\variable{z}}}$}
\item {$\existential{\variable{z}}\parconjunction{\Gp{\variable{z}}}{\negation{\Dp{\variable{z}}}}$}
\item {$\existential{\variable{z}}\parconjunction{\Gp{\variable{z}}}{\Dp{\variable{z}}}$}
\item {$\universal{\variable{z}}\parconjunction{\Gp{\variable{z}}}{\Dp{\variable{z}}}$}
\item {$\existential{\variable{z}}\parhorseshoe{\Gp{\variable{z}}}{\Dp{\variable{z}}}$}
\end{enumerate}
\end{multicols}
\begin{enumerate}[start=7] 
\item {$\universal{\variable{z}}\parhorseshoe{\Gp{\variable{z}}}{\negation{\Dp{\variable{z}}}}$}
\end{enumerate}



%\theendnotes



%%%%%%%%%%%%%%%%%%%%%%%%%%%%%%%%%%%%%%%%%%%%%%%%%%
\chapter{Quantifier Language II}\label{quantifierlogic}
%%%%%%%%%%%%%%%%%%%%%%%%%%%%%%%%%%%%%%%%%%%%%%%%%%
% \AddToShipoutPicture*{\BackgroundPicB}

%%%%%%%%%%%%%%%%%%%%%%%%%%%%%%%%%%%%%%%%%%%%%%%%%%
\section{The Language \GQL{}}
%%%%%%%%%%%%%%%%%%%%%%%%%%%%%%%%%%%%%%%%%%%%%%%%%%

%\setcounter{DefThm}{0}

\subsection{Symbols}\label{Sec:GQLSymbols}
In this chapter we extend our language to include many-place predicates.  The resulting language is \GQL{}, and its development was a significant event in the history of logic.\footnote{%
	The development of \GQL{} goes back to Gottlob Frege \citeyearpar{Frege1879,Frege1891,Frege1893}, O. H. Mitchell \citeyearpar{Mitchell1883} and Charles S. Peirce \citeyearpar{Peirce1883}, with Frege's work being independent of and unknown to the latter two. See \citealp[288]{Church1956} and \citealp[34]{Hodges2001}.  %
 Although it's probably safe to say that Frege and Peirce/Mitchell developed quantificational logic independently, the extent to which Peirce and his students (like Mitchell) knew of Frege's work is a matter of debate.
It's clear they at least knew of Frege.
E.g., Ladd-Franklin \citeyearpar{LaddFranklin1883} cites Frege's \citeyearpar{Frege1879} through a review of it by Ernst Schr\"oder. 
See \citep{Dipert1984} for a brief discussion on the situation.}
The 2-place predicates correspond roughly to what you get if you take an English sentence and remove two names, leaving blanks, e.g.:

\begin{RESTARTmenumerate}
	\item \mention{Goliath is taller than David} $\Rightarrow$ \mention{\_\_\_\_\_\_ is taller than \_\_\_\_\_\_}
\end{RESTARTmenumerate}

\noindent{}We may think of 2-place predicates is as expressing a 2-place \emph{relation}.  In the above example, we have the \mention{taller than} relation.  This relation holds between two objects when one is taller than the other.  Another 2-place relation is \mention{loves}:

\begin{menumerate}
		\item \mention{Juliet Capulet loves Romeo Montague} $\Rightarrow$ \mention{\_\_\_\_\_\_ loves \_\_\_\_\_\_}
\end{menumerate}

\noindent{}The \mention{loves} relation holds when one person---or object of whatever kind---loves another.  We may understand 3-place predicates in a similar way.  For example:

\begin{menumerate}
	\item \mention{Three is between two and four} $\Rightarrow$ \mention{\_\_\_\_\_\_ is between \_\_\_\_\_\_ and \_\_\_\_\_\_}
\end{menumerate}

For any $n\geq2$, an $n$-place predicate can mirror a corresponding $n$-place relation.  The introduction of many-place predicates significantly increases the power of our formal language.  Consider the following argument:

\begin{menumerate}
\item Socrates is older than Plato.
\item Plato is older than Aristotle.

Therefore,

\item Socrates is older than Aristotle.
\end{menumerate}

\noindent{}This argument is a good one, but we cannot represent it as an entailment in \GQL{}1.  The full language of \GQL{} can handle such arguments because it can represent the \mention{older than} relation with a 2-place predicate.  One caveat is necessary---the last sentence isn't a logical consequence of the first two.  The first two sentences only entail the third if we also assume that the \mention{older than} relation is transitive.  Roughly, the way to express this transitivity would be as follows:

\begin{center}
For any $x$, $y$, and $z$ such that $x$ is older than $y$ and $y$ is older than $z$, \\ $x$ is older than $z$.
\end{center}

\noindent{}Thankfully, we can express this in \GQL{}, using quantifiers and 2-place predicates.

\GQL{} has all the basic symbols of \GQL{}1, plus predicate letters for $n$-placed predicates, for every integer $n$ such that $n\geq2$. 
\begin{majorILnc}{\LnpDC{Symbols of GQL}}
The \df{basic symbols} of \GQL{} are:
\begin{cenumerate}
\item Logical Connectives, Punctuation Symbols, Sentence Letters, Individual Constants, Individual Variables: same as \GQL{}1
\item 1-Place Predicates: $\Ap{'}$, $\Bp{'}$, $\ldots$, $\Tp{'}$, $\Ap{'}_1$, $\Bp{'}_1$, $\ldots$, $\Tp{'}_1$, $\Ap{'}_2$, $\Bp{'}_2$, $\ldots$
\item 2-Place Predicates: $\Ap{''}$, $\Bp{''}$, $\ldots$, $\Tp{''}$, $\Ap{''}_1$, $\Bp{''}_1$, $\ldots$, $\Tp{''}_1$, $\Ap{''}_2$, $\Bp{''}_2$, $\ldots$
\item 3-Place Predicates: $\Ap{'''}$, $\Bp{'''}$, $\ldots$
\item[] \hspace{.5in} . . . and so on for all positive integers.
\end{cenumerate}
\end{majorILnc}

The only difference in appearance between 1-place and 2-place predicates is that the former have only one prime mark and the latter have two. 
Like their 1-place counterparts, 2-place predicates go from \mention{$\Ap{''}$} to \mention{$\Tp{''}$}.  And the same also goes for every $n$-place predicate.

The superscript prime marks play the significant logical role of marking the arity, or number of places, of the predicate, while the subscripts play the lesser role of making sure we have enough predicates.  For every integer $\integer{n}$, \GQL{} contains an infinite number of $\integer{n}$-place predicates. 

\subsection{Formulas of \GQL{}}\label{Formulas of GQL}
Our ultimate interest is in sentences of \GQL{}, but as with \GQL{}1 we must first define formulas.\index{formulas} The second base clause expands the definition from the \GQL{}1 definition of formula.
\begin{majorILnc}{\LnpDC{Definition of Formula of GQL}} The \nidf{formulas} \underdf{of \GQL{}}{formulas} are given by the following recursive definition:
\begin{description}
\item[Base Clauses:] \hfill{}
\begin{cenumerate}
\item A sentence letter (atomic sentence of \GSL{}) is a formula.
\item An $\integer{n}$-place predicate followed by $\integer{n}$ occurrences (tokens) of individual constants or variables is a formula.
\end{cenumerate}
\item[Generating Clauses:] \hfill{}
\begin{cenumerate}
\item If $\CAPPHI$ is a formula, then so is $\negation{\CAPPHI}$.
\item If $\CAPPHI$ and $\CAPTHETA$ are formulas, then so are $\parhorseshoe{\CAPPHI}{\CAPTHETA}$ and $\partriplebar{\CAPPHI}{\CAPTHETA}$.
\item If all of $\CAPPHI_1,\CAPPHI_2,\CAPPHI_3,\CAPPHI_4,\ldots,\CAPPHI_{\integer{n}}$ are formulas (the list must include at least two formulas and be finite), then so are $\parconjunction{\CAPPHI_1}{\conjunction{\CAPPHI_2}{\conjunction{\CAPPHI_3}{\conjunction{\CAPPHI_4}{\conjunction{\ldots}{\CAPPHI_{\integer{n}}}}}}}$ and $\pardisjunction{\CAPPHI_1}{\disjunction{\CAPPHI_2}{\disjunction{\CAPPHI_3}{\disjunction{\CAPPHI_4}{\disjunction{\ldots}{\CAPPHI_{\integer{n}}}}}}}$.
\item If $\CAPPHI$ is a formula and it does not contain an expression of the form $\universal{\ALPHA}$ or $\existential{\ALPHA}$ for some \GQL{} variable $\ALPHA$, then $\universal{\ALPHA}\CAPPHI$ and $\existential{\ALPHA}\CAPPHI$ are formulas.
\end{cenumerate}
\item[Closure Clause:] A string of symbols is a formula \Iff it can be generated by the clauses above.
\end{description}
\end{majorILnc}
\noindent{}$\App{'}{\constant{b}}$ is a formula (an atomic one, to be specific), as is $\Appp{''}{\variable{x}}{\constant{a}}$. 
But $\Bpp{''}{\variable{x}}$ is not a formula, because it has a 2-place predicate followed only by one individual variable. 
To determine whether some string is a formula, we must count \emph{tokens} of variables and constants. 
For example, $\Cppp{'''}{\variable{x}}{\variable{x}\variable{x}}$ is a formula because it has a 3-place predicate followed by three tokens of an individual variable.

From base clause 2 we know that $\Gppp{''}{\variable{x}}{\variable{y}}$ is a formula, and so from clause 4 we know that the following are also formulas. 
(The list is not exhaustive.) 
\begin{multicols}{2}
\begin{menumerate}
\item $\universal{\variable{x}}\Gppp{''}{\variable{x}}{\variable{y}}$ 
\item $\existential{\variable{x}}\Gppp{''}{\variable{x}}{\variable{y}}$ 
%\item $\universal{\variable{y}}\Gppp{''}{\variable{x}}{\variable{y}}$ 
%\item $\existential{\variable{y}}\Gppp{''}{\variable{x}}{\variable{y}}$ 
%\item $\universal{\variable{z}}\Gppp{''}{\variable{x}}{\variable{y}}$ 
\item $\existential{\variable{z}}\Gppp{''}{\variable{x}}{\variable{y}}$
\item $\existential{\variable{y}}\universal{\variable{x}}\Gppp{''}{\variable{x}}{\variable{y}}$ 
%\item $\existential{\variable{x}}\universal{\variable{y}}\Gppp{''}{\variable{x}}{\variable{y}}$ 
%\item $\universal{\variable{x}}\universal{\variable{y}}\Gppp{''}{\variable{x}}{\variable{y}}$ 
\item $\universal{\variable{x}}\existential{\variable{y}}\Gppp{''}{\variable{x}}{\variable{y}}$ 
\item $\universal{\variable{x}}\universal{\variable{z}}\Gppp{''}{\variable{x}}{\variable{y}}$ 
%\item $\universal{\variable{x}}\existential{\variable{z}}\Gppp{''}{\variable{x}}{\variable{y}}$
\end{menumerate}
\end{multicols}
\noindent{}But $\universal{\variable{x}}\universal{\x}\Gppp{''}{\variable{x}}{\variable{y}}$ is \emph{not} a formula, because it's of the form $\universal{\variable{x}}\CAPPHI$ where $\CAPPHI$ is a formula that contains the expression $\universal{\variable{x}}$.

As in \GSL{} and \GQL{}1, we have unofficial formulas to improve readability.
\begin{majorILnc}{\LnpDC{Unofficial Formula of GQL}}
A string of symbols is an \nidf{unofficial} formula\index{formulas!unofficial|textbf} \Iff we can obtain it from an official formula by
\begin{cenumerate}
\item deleting outer parentheses,
\item replacing official parentheses ( ) with square brackets [ ] or curly brackets \{ \}, or
\item omitting primes $'$ on a predicate letter.
\end{cenumerate}
\end{majorILnc}

A unique official formula can always be reconstructed from an unofficial formula.

\subsection{Other Properties of Formulas}\label{Other Properties of Formulas} 
The concepts of subformula, order, main connective, and construction tree for formulas of \GQL{} are the same as in \GQL{}1.\footnote{See section \ref{Other Properties of Formulas1} of the last chapter.} 
\begin{majorILnc}{\LnpEC{GQLSubformulaPropertiesExampleC}}
Consider the formula $\disjunction{\existential{\variable{x}}\parconjunction{\universal{y}\Eppp{''}{\variable{x}}{\variable{y}}}{\App{'}{\variable{x}}}}{\universal{\variable{z}}\parhorseshoe{\existential{\variable{y}}\Hp{'\constant{a}}}{\Gp{'\variable{x}}}}$.
This is a disjunction; its main connective is vee, $\VEE$.
It has eleven subformulas:
\begin{enumerate}[label=(\arabic*), leftmargin=1.85\parindent,
labelindent=.35\parindent, labelsep=*, itemsep=0pt]%,start=1
\item $\disjunction{\existential{\variable{x}}\parconjunction{\universal{y}\Eppp{''}{\variable{x}}{\variable{y}}}{\Ap{'\variable{x}}}}{\universal{\variable{z}}\parhorseshoe{\existential{\variable{y}}\Hp{'\constant{a}}}{\Gp{'\variable{x}}}}$
\end{enumerate}
\vspace*{-.5cm}
\begin{multicols}{2}
\begin{enumerate}[label=(\arabic*), leftmargin=1.85\parindent,
labelindent=.35\parindent, labelsep=*, itemsep=0pt, start=2]%,start=1
\item $\existential{\variable{x}}\parconjunction{\universal{y}\Eppp{''}{\variable{x}}{\variable{y}}}{\Ap{'\variable{x}}}$
\item $\conjunction{\universal{y}\Eppp{''}{\variable{x}}{\variable{y}}}{\Ap{'\variable{x}}}$
\item $\universal{y}\Eppp{''}{\variable{x}}{\variable{y}}$
\item $\Ap{'\variable{x}}$
\item $\Eppp{''}{\variable{x}}{\variable{y}}$
\item $\universal{\variable{z}}\parhorseshoe{\existential{\variable{y}}\Hp{'\constant{a}}}{\Gp{'\variable{x}}}$
\item $\horseshoe{\existential{\variable{y}}\Hp{'\constant{a}}}{\Gp{'\variable{x}}}$
\item $\existential{\variable{y}}\Hp{'\constant{a}}$
\item $\Gp{'\variable{x}}$
\item $\Hp{'\constant{a}}$
\end{enumerate}
\end{multicols}
The construction tree of the formula is:
\begin{center}
\begin{tikzpicture}[grow=up]
\tikzset{level distance=50pt}
\tikzset{level 1/.style={level distance=65pt}}
\tikzset{sibling distance=40pt}
\tikzset{every tree node/.style={align=center,anchor=north}}
	\Tree%http://angasm.org/papers/qtree/    http://www.ling.upenn.edu/advice/latex/qtree/qtreenotes.pdf
[.{$\disjunction{\existential{\variable{x}}\parconjunction{\universal{y}\Eppp{''}{\variable{x}}{\variable{y}}}{\Ap{'\variable{x}}}}{\universal{\variable{z}}\parhorseshoe{\existential{\variable{y}}\Hp{'\constant{a}}}{\Gp{'\variable{x}}}}$}
  [.{$\horseshoe{\existential{\variable{y}}\Hp{'\constant{a}}}{\Gp{'\variable{x}}}$\\ $\universal{\variable{z}}\parhorseshoe{\existential{\variable{y}}\Hp{'\constant{a}}}{\Gp{'\variable{x}}}$}
       [.{$\text{ }$\\ $\Gp{'\variable{x}}$}
       ]    
       [.{$\Hp{'\constant{a}}$\\ $\existential{\variable{y}}\Hp{'\constant{a}}$}
       ]
  ]
  [.{$\conjunction{\universal{y}\Eppp{''}{\variable{x}}{\variable{y}}}{\Ap{'\variable{x}}}$\\ $\existential{\variable{x}}\parconjunction{\universal{y}\Eppp{''}{\variable{x}}{\variable{y}}}{\Ap{'\variable{x}}}$} %!{\qsetw{3in}}
       [.{$\text{ }$\\ $\Ap{'\variable{x}}$}
       ]    
       [.{$\Eppp{''}{\variable{x}}{\variable{y}}$\\ $\universal{y}\Eppp{''}{\variable{x}}{\variable{y}}$}
       ] 
  ]
]%
	%\caption{Example formula tree}
	%\label{fig:ExampleFormulaTree}
\end{tikzpicture}
\end{center}
As you can see from the construction tree, the order of the formula is 5. 
\end{majorILnc}

\subsection{Sentences of \GQL{}}\label{Sentences of GQL} 
Sentences, atomic sentences, and unofficial sentences of \GQL{} are defined exactly as in \GQL{}1.\footnote{See section \ref{Sentences of GQL1} of the last chapter.} 


%%%%%%%%%%%%%%%%%%%%%%%%%%%%%%%%%%%%%%%%%%%%%%%%%%
\section{Models}\label{GQL Interpretations}
%%%%%%%%%%%%%%%%%%%%%%%%%%%%%%%%%%%%%%%%%%%%%%%%%%

\subsection{Models in \GQL{}}\label{Interpretations in GQL}
As you can imagine, models in \GQL{} are very similar to models of \GQL{}1 except that they accommodate many-place predicates.

\begin{majorILnc}{\LnpDC{GQL Interpretation}} 
	A \df{model} for $\CAPPHI$, $\IntA$, consists of:
	\begin{cenumerate}
		\item an assignment of a truth value $\TrueB$ or $\FalseB$ to each sentence letter in $\CAPPHI$; 
		\item a non-empty set $\integer{U}$, called the \df{universe} or \df{domain};
		\item an assignment of an object from $\integer{U}$ to each individual constant in $\CAPPHI$;
		\item an assignment of a subset of $\integer{U}$ to each 1-place predicate in $\CAPPHI$;
		\item an assignment of a set of ordered $\integer{n}$-tuples to each $\integer{n}$-place predicate in $\CAPPHI$.  The objects in each $\integer{n}$-tuple are members of $\integer{U}$.\footnote{For a reminder on what an $\integer{n}$-tuple is, see section \ref{orderedpairs} of Chapter \ref{introduction}.}
	\end{cenumerate}
\end{majorILnc}

\noindent{}The only new part of the definition of model for $\CAPPHI$ in \GQL{} is clause (5).  
Given an $\integer{n}$-place predicate, like $\Bp{'''}$, we let $\IntA(\Bp{'''})$ be the set of ordered $\integer{n}$-tuples (in this case, $3$-tuples) assigned to $\Bp{'''}$ by $\IntA$. 
For an illustration, imagine a model $\IntA$ on which $\Bp{'''}$ stands for the \mention{between} relation for positive integers from $1$ to $4$; i.e., such that the first member of each ordered $3$-tuple is a number between the second and third members.  The $3$-tuple $\langle 2, 1, 3\rangle$ is one example.  The model $\IntA$ would assign all such $3$-tuples to $\Bp{'''}$ as follows:

\bigskip
\noindent{}$\IntA(\Bp{'''})=\{\langle 2, 1, 3\rangle, \langle 2, 1, 4\rangle, \langle 3, 1, 4\rangle, \l\langle 3, 2, 4\rangle\}$ 
\bigskip

For a second illustration, consider these three people: Jack, Jill, and Bill.  Let's say that Jack is taller than Jill, and Jill is taller than Bill.  Let $\IntA$ be a model on which $\IntA(\Ap{''})$ is the \mention{taller than} relation.  The assignment to $\Ap{''}$ would be the following set of ordered pairs:

\bigskip
\noindent{}$\IntA(\Ap{''})=\{\langle$Jack, Jill$\rangle, \langle$Jill, Bill$\rangle, \langle$Jack, Bill$\rangle\}$ 
\bigskip

\subsection{Truth in a Model}\label{GQL Truth in an Interpretation}

The definition of truth in a model for \GQL{} is exactly the same as in \GQL{}1 except with an additional clause for many-place predicates.

\begin{majorILnc}{\LnpDC{Truth for GQL Sentence}}
The following clauses fix when a \GQL{} sentence $\CAPTHETA$ is \nidf{$\True$} (or \nidf{$\False$}) on a model for $\CAPTHETA$, $\IntA$:
\begin{cenumerate}
	\item A sentence letter $\CAPPHI$ is $\True$ on $\IntA$ \Iff $\As{}{}$ assigns $\True$ to it, i.e. \Iff $\As{}{}(\CAPPHI)=\TrueB$.
	\item An atomic sentence $\Pp{\variable{t}}$ with a 1-place predicate $\PP$ and an individual term $\variable{t}$ is $\True$ on $\IntA$ \Iff what $\IntA$ assigns to the individual term $\variable{t}$ is in the set $\IntA$ assigns to the predicate, i.e. \Iff $\IntA(\variable{t})\in\IntA(\PP)$.
	\item\label{formtruthatomicn} An atomic sentence $\Pp{\variable{t}_1\ldots\variable{t}_{\integer{n}}}$ with an $\integer{n}$-place predicate $\PP$ is $\True$ on $\IntA$ \Iff $\langle \As{}{}(\variable{t}_1),\As{}{}(\variable{t}_2),\ldots,\As{}{}(\variable{t}_{\integer{n}}) \rangle \in \As{}{}(\PP)$. 
	\item A negation $\negation{\CAPPHI}$ is $\True$ on $\IntA$ \Iff the unnegated formula $\CAPPHI$ is $\False$ on $\IntA$.
	\item A conjunction $\parconjunction{\CAPPHI_1}{\conjunction{\ldots}{\CAPPHI_{\integer{n}}}}$ is $\True$ on $\IntA$ \Iff all conjuncts $\CAPPHI_1,\ldots,\CAPPHI_{\integer{n}}$ are $\True$ on $\IntA$.
	\item A disjunction $\pardisjunction{\CAPPHI_1}{\disjunction{\ldots}{\CAPPHI_{\integer{n}}}}$ is $\True$ on $\IntA$ \Iff at least one disjunct $\CAPPHI_1,\ldots,\CAPPHI_{\integer{n}}$ is $\True$ on $\IntA$.
	\item A conditional $\parhorseshoe{\CAPPSI}{\CAPPHI}$ is $\True$ on $\IntA$ \Iff the \CAPS{lhs} $\CAPPSI$ is $\False$ or the \CAPS{rhs} $\CAPPHI$ is $\True$ on $\IntA$.
	\item A biconditional $\partriplebar{\CAPPSI}{\CAPPHI}$ is $\True$ on $\IntA$ \Iff both sides, $\CAPPSI$ and $\CAPPHI$, have the same truth value on $\IntA$.
	\item\label{GQLTruthUnvQuant} A universal quantification $\universal{\ALPHA}\CAPPHI$ is $\True$ on $\IntA$ \Iff $\CAPPHI\variable{t}/\ALPHA$ is $\True$ on \emph{all} $\variable{t}$-variants of $\IntA$ (where $\variable{t}$ is the first \emph{constant} not contained in $\CAPPHI$).
	\item An existential quantification $\existential{\ALPHA}\CAPPHI$ is $\True$ on $\IntA$ \Iff $\CAPPHI\variable{t}/\ALPHA$ is $\True$ on \emph{some} $\variable{t}$-variant of $\IntA$ (where $\variable{t}$ is the first \emph{constant} not contained in $\CAPPHI$).
	\item A sentence $\CAPPHI$ is $\False$ on $\IntA$ \Iff $\CAPPHI$ is not $\True$ on $\IntA$.
\end{cenumerate}
\end{majorILnc}

\noindent{}Let's look at a simple example to see how \GQL{} truth works for a two-place predicate.  Consider a model $\IntA$ with the following assignments:\\

\noindent{}$\IntA(\constant{j})=$ Jack\\
\noindent{}$\IntA(\constant{i})=$ Jill\\
\noindent{}$\IntA(\constant{b})=$ Bill\\
\noindent{}$\IntA(\Ap{''})=\{\langle$Jack, Jill$\rangle, \langle$Jill, Bill$\rangle, \langle$Jack, Bill$\rangle\}$\\

\noindent{}Is the \GQL{} sentence $\Ap{''\constant{j}\constant{b}}$ true on $\IntA$?  To find out, we must check the set $\IntA(\Ap{''})$ to see if $\langle\IntA(\constant{j}),\IntA(\constant{b})\rangle$ is a member.  We see that $\langle\IntA(\constant{j}),\IntA(\constant{b})\rangle=\langle$Jack, Bill$\rangle$.  We then find that $\langle$Jack, Bill$\rangle$ is a member of $\IntA(\Ap{''})$, so $\Ap{''\constant{j}\constant{b}}$ is true on $\IntA$.

Next, consider the sentence $\Ap{''\constant{b}\constant{i}}$ on the same model.  This sentence is true on $\IntA$ \Iff $\langle\IntA(\constant{b}),\IntA(\constant{i})\rangle$ is a member of the set $\IntA(\Ap{''})$.  We know that $\langle\IntA(\constant{b}),\IntA(\constant{i})\rangle=\langle$Bill, Jill$\rangle$.  But $\langle$Bill, Jill$\rangle$ is not a member of $\IntA(\Ap{''})$, so $\Ap{''\constant{b}\constant{i}}$ is false on $\IntA$.  (Remember that $\langle$Bill, Jill$\rangle$ is not the same as $\langle$Jill, Bill$\rangle$.  Order matters!)

Let's try more complicated examples, using the models provided on the table below:

\begin{longtable}[c]{ l l l l } %p{2.2in} p{2in}
	\toprule
	&\textbf{Symbol} & \multicolumn{2}{c}{\textbf{Model}} \\ \cmidrule(l){3-4}
	& & \textbf{Pos Int} & \textbf{States} \\
	\midrule 
	\endfirsthead
	\multicolumn{4}{c}{\emph{Continued from Previous Page}}\\
	\toprule
	&\textbf{Symbol} & \multicolumn{2}{c}{\textbf{Model}} \\ \cmidrule(l){3-4}
	& & \textbf{Pos Int} & \textbf{States} \\
	\midrule 
	\endhead
	\bottomrule
	\caption{Example Models}\\[-.15in]
	\multicolumn{4}{c}{\emph{Continued next Page}}\\
	\endfoot
	\bottomrule
	\caption{Example Models}\\%
	\endlastfoot%
	\label{table:Example Interpretations}%
	%\begin{tabular}{ l l l l } %p{2in} p{2in} %\begin{tabular}{ p{1in} l l } %p{2.2in} p{2in}
	%\toprule
	%&\textbf{Symbol} & \multicolumn{2}{c}{\textbf{Interpretation}} \\ \cmidrule(l){3-4}
	%& & \textbf{Pos Int} & \textbf{States} \\
	%\midrule 
	{Universe:} & & The set of positive integers & The set of US states (2015) \\ \addlinespace[.25cm]
	{Sent. Let.:}& A&$\True$&$\False$\\
	& B&$\True$&$\False$\\
	& C&$\False$&$\True$\\
	& D&$\True$&$\False$\\
	& E&$\True$&$\False$\\
	& G&$\False$&$\True$\\ \addlinespace[.25cm]
	{Constants:}&$\constant{a}$&1&Louisiana\\
	&$\constant{b}$&9&Maine\\
	&$\constant{c}$&72&Georgia\\
	&$\constant{d}$&3&Nebraska\\
	&$\constant{e}$&1&New Mexico\\
	&$\constant{f}$&2&Texas\\ \addlinespace[.25cm]
	{1-place:}&$\Ap{'}$&all pos int&Midwestern\\
	&$\Bp{'}$&empty set&name with $>5$ letters\\
	&$\Cp{'}$&even&Coastal\\
	&$\Dp{'}$&odd&on the Pacific Coast\\
	&$\Ep{'}$&prime&\{Ohio\}\\
	&$\Gp{'}$&multiple of 7&\{Ohio, Alabama\}\\ \addlinespace[.25cm]
	{2-place:}&$\Ap{''}$&first $>$ second&share a border\\
	&$\Bp{''}$&are equal&first is north of second\\
	&$\Cp{''}$&first = 2 times second&first $>$ second (area)\\
	&$\Dp{''}$&sum of them equals 7&first $>$ second (population)\\
	&$\Ep{''}$&first $<$ second&first is west of second\\
	&$\Gp{''}$&are relatively prime&both coastal, or neither\\ \addlinespace[.25cm]
	{3-place:}&$\Ap{'''}$&all equal&all same population\\
	&$\Bp{'''}$&first $<$ second $<$ third&first is north of others\\
	&$\Cp{'''}$&all odd or all even&first $>$ second $>$ third (area)\\
	&$\Dp{'''}$&first + second = third&first + second $>$ third (area)\\
	&$\Ep{'''}$&first $\times$ second = third&first is west of the others\\
	&$\Gp{'''}$&are all relatively prime& at least two coastal \\
	%\bottomrule
\end{longtable}

While looking over the many-place predicates in the table, you'll notice that we don't actually list sets of $n$-tuples.  Instead, for each many-place predicate we provide a brief description of some relation which could be used to identify such a set.  This will be our usual practice.  Writing out all the specific $n$-tuples in our chart would be excessively tedious for the model \emph{States}, and impossible for the model \emph{Pos Int}.\footnote{Finite beings often have difficulty performing tasks with an infinite number of steps.}

\begin{majorILnc}{\LnpEC{GQLTruthExampleA}}
Determine the truth values of the sentences $\universal\variable{y}\existential\variable{x}\App{''\variable{x}}{\variable{y}}$ and $\existential\variable{x}\universal\variable{y}\App{''\variable{x}}{\variable{y}}$ on the model \emph{Pos Int} in table \ref{table:Example Interpretations}.  The only difference between these two sentences is the order of the quantifiers.  If we can show that they differ in truth value on the same model, then we will have established that quantifier order matters for assessing truth in a model.
\end{majorILnc}
\begin{PROOF}
First, let's consider the sentence $\universal\variable{y}\existential\variable{x}\App{''\variable{x}}{\variable{y}}$.

(i)  $\universal\variable{y}\existential\variable{x}\App{''\variable{x}}{\variable{y}}$ is true on \emph{Pos Int} \Iff $\existential\variable{x}\App{''\variable{x}}{\constant{a}}$ is true on all $\constant{a}$-variants of \emph{Pos Int}.  The sentence $\existential\variable{x}\App{''\variable{x}}{\constant{a}}$ is true on an $\constant{a}$-variant of \emph{Pos Int}, $\emph{Pos Int}^{\constant{a}}$, \Iff $\App{''\constant{b}}{\constant{a}}$ is true on some $\constant{b}$-variant of $\emph{Pos Int}^{\constant{a}}$ (definition of truth, $\exists$).  \emph{Pos Int} assigns to $\App{''}{}$ the set of ordered pairs such that the first is greater than the second.  So, $\App{''\constant{b}}{\constant{a}}$ is true on some $\constant{b}$-variant of $\emph{Pos Int}^{\constant{a}}$ when it assigns a larger number to $\constant{b}$ than it assigns to $\constant{a}$.

It doesn't really matter what $\emph{Pos Int}^{\constant{a}}$ assigns to $\constant{a}$.  There is some larger number that a $\constant{b}$-variant of $\emph{Pos Int}^{\constant{a}}$ can assign to $\constant{b}$.  Hence, regardless of what number $\emph{Pos Int}^{\constant{a}}(\constant{a})$ is, $\existential\variable{x}\App{''\variable{x}}{\constant{a}}$ is true on $\emph{Pos Int}^{\constant{a}}$.

For every assignment to $\constant{a}$, there is some assignment to $\constant{b}$ that is larger.  So $\existential\variable{x}\App{''\variable{x}}{\constant{a}}$ is true on all $\constant{a}$-variants of \emph{Pos Int}.  Thus, the sentence $\universal\variable{y}\existential\variable{x}\App{''\variable{x}}{\variable{y}}$ is true on \emph{Pos Int}.

(ii) $\existential\variable{x}\universal\variable{y}\App{''\variable{x}}{\variable{y}}$ is true on \emph{Pos Int} \Iff $\universal\variable{y}\App{''\constant{a}}{\variable{y}}$ is true on some $\constant{a}$-variant of \emph{Pos Int}.  The sentence $\universal\variable{y}\App{''\constant{a}}{\variable{y}}$ is true on some $\constant{a}$-variant of \emph{Pos Int}, $\emph{Pos Int}^{\constant{a}}$, \Iff $\App{''\constant{a}}{\constant{b}}$ is true on all $\constant{b}$-variants of $\emph{Pos Int}^{\constant{a}}$.

But no matter how large of a number that a $\constant{b}$-variant of $\emph{Pos Int}^{\constant{a}}$ assigns to $\constant{a}$, there will always be some larger number that it could assign to $\constant{b}$.  So, $\universal\variable{y}\App{''\constant{a}}{\variable{y}}$ is false on all $\constant{a}$-variants of \emph{Pos Int}.  Thus, $\existential\variable{x}\universal\variable{y}\App{''\variable{x}}{\variable{y}}$ is false on \emph{Pos Int}.
\end{PROOF}
The different quantifier order affects the meanings of the sentences.  The sentence $\universal\variable{y}\existential\variable{x}\App{''\variable{x}}{\variable{y}}$ relative to \emph{Pos Int} means, roughly, that \mention{for every positive integer there is a larger positive integer.}  The sentence $\existential\variable{x}\universal\variable{y}\App{''\variable{x}}{\variable{y}}$ relative to \emph{Pos Int} means, roughly, that \mention{there is some positive integer that is larger than every positive integer.}  Clearly the former is true and the latter isn't.\footnote{We will further discuss translations of sentences in formal languages in Chapter \ref{Translations}.}









\begin{majorILnc}{\LnpEC{GQLTruthExampleB}}
The sentence $\universal{\variable{x}}\universal{\variable{y}}\universal{\variable{z}}\parhorseshoe{\parconjunction{\Cpp{\variable{x}}{\variable{y}}}{\Dppp{\variable{x}}{\variable{y}}{\variable{z}}}}{\Bppp{\variable{y}}{\variable{x}}{\variable{z}}}$ is (i) true in the model \emph{Pos Int} given in table \mvref{table:Example Interpretations}, but (ii) is false in the model \emph{States}. 
\end{majorILnc}
\begin{PROOF}
(i) The model \emph{Pos Int} assigns to $\CC$ the set of positive integer pairs $\langle \variable{u},\variable{v}\rangle$ such that $\variable{u}=2\variable{v}$, to $\DD$ the set of positive integer triples $\langle \variable{u},\variable{v},\variable{w}\rangle$ such that $\variable{u}+\variable{v}=\variable{w}$, and to $\BB$ the set of positive integer triples $\langle \variable{v},\variable{u},\variable{w}\rangle$ such that $\variable{v}<\variable{u}<\variable{w}$.  

Consider an instantiation of the sentence $\universal{\variable{x}}\universal{\variable{y}}\universal{\variable{z}}\parhorseshoe{\parconjunction{\Cpp{\variable{x}}{\variable{y}}}{\Dppp{\variable{x}}{\variable{y}}{\variable{z}}}}{\Bppp{\variable{y}}{\variable{x}}{\variable{z}}}$; let's say $\parhorseshoe{\parconjunction{\Cpp{\constant{a}}{\constant{b}}}{\Dppp{\constant{a}}{\constant{b}}{\constant{c}}}}{\Bppp{\constant{b}}{\constant{a}}{\constant{c}}}$.\footnote{An \mention{instantiation} of a quantified sentence is a second sentence in which the quantifier is first dropped, and then the newly unbound variables are replaced with constants.  In this case we drop all three quantifiers and substitute three different constants for the three variables in the original sentence.}  Now consider an $\constant{a},\constant{b},\constant{c}$-variant of \emph{Pos Int} such that $\emph{Pos Int}^{\constant{a}\constant{b}\constant{c}}(\constant{a})=2\emph{Pos Int}^{\constant{a}\constant{b}\constant{c}}(\constant{b})$ and $\emph{Pos Int}^{\constant{a}\constant{b}\constant{c}}(\constant{a})+\emph{Pos Int}^{\constant{a}\constant{b}\constant{c}}(\constant{b})=\emph{Pos Int}^{\constant{a}\constant{b}\constant{c}}(\constant{c})$.

Both $\Cpp{\constant{a}}{\constant{b}}$ and $\Dppp{\constant{a}}{\constant{b}}{\constant{c}}$ are true on any such variant. The \CAPS{rhs} of the conditional, $\Bppp{\constant{b}}{\constant{a}}{\constant{c}}$, is also true on that variant.  Because $\emph{Pos Int}^{\constant{a}\constant{b}\constant{c}}(\constant{a})=2\emph{Pos Int}^{\constant{a}\constant{b}\constant{c}}(\constant{b})$, it's clear that $\emph{Pos Int}^{\constant{a}\constant{b}\constant{c}}(\constant{b})<\emph{Pos Int}^{\constant{a}\constant{b}\constant{c}}(\constant{a})$ (0 is not a positive integer). 
And because $\emph{Pos Int}^{\constant{a}\constant{b}\constant{c}}(\constant{b})+\emph{Pos Int}^{\constant{a}\constant{b}\constant{c}}(\constant{a})=\emph{Pos Int}^{\constant{a}\constant{b}\constant{c}}(\constant{c})$, it's clear that $\emph{Pos Int}^{\constant{a}\constant{b}\constant{c}}(\constant{a})<\emph{Pos Int}^{\constant{a}\constant{b}\constant{c}}(\constant{c})$.
So, $\Bppp{\constant{b}}{\constant{a}}{\constant{c}}$ is true on any such variant of \emph{Pos Int}.
Thus, every variant of \emph{Pos Int} such that $\emph{Pos Int}^{\constant{a}\constant{b}\constant{c}}(\constant{a})=2\emph{Pos Int}^{\constant{a}\constant{b}\constant{c}}(\constant{b})$ and $\emph{Pos Int}^{\constant{a}\constant{b}\constant{c}}(\constant{a})+\emph{Pos Int}^{\constant{a}\constant{b}\constant{c}}(\constant{b})=\emph{Pos Int}^{\constant{a}\constant{b}\constant{c}}(\constant{c})$ makes the conditional $\parhorseshoe{\parconjunction{\Cpp{\constant{a}}{\constant{b}}}{\Dppp{\constant{a}}{\constant{b}}{\constant{c}}}}{\Bppp{\constant{b}}{\constant{a}}{\constant{c}}}$ true. 

Any variant of \emph{Pos Int} such that either $\emph{Pos Int}^{\constant{a}\constant{b}\constant{c}}(\constant{a})\neq{}2\emph{Pos Int}^{\constant{a}\constant{b}\constant{c}}(\constant{b})$ or $\emph{Pos Int}^{\constant{a}\constant{b}\constant{c}}(\constant{a})+\emph{Pos Int}^{\constant{a}\constant{b}\constant{c}}(\constant{b})\neq{}\emph{Pos Int}^{\constant{a}\constant{b}\constant{c}}(\constant{c})$ makes either  $\Cpp{\constant{a}}{\constant{b}}$ or $\Dppp{\constant{a}}{\constant{b}}{\constant{c}}$ false (def. of truth, $\WEDGE$), and hence makes the conditional $\parhorseshoe{\parconjunction{\Cpp{\constant{a}}{\constant{b}}}{\Dppp{\constant{a}}{\constant{b}}{\constant{c}}}}{\Bppp{\constant{b}}{\constant{a}}{\constant{c}}}$ true (def. of truth, $\HORSESHOE$).

Therefore, every $\constant{a},\constant{b},\constant{c}$-variant of \emph{Pos Int} makes the conditional $\parhorseshoe{\parconjunction{\Cpp{\constant{a}}{\constant{b}}}{\Dppp{\constant{a}}{\constant{b}}{\constant{c}}}}{\Bppp{\constant{b}}{\constant{a}}{\constant{c}}}$ true. Put in a slightly different way, every $\constant{c}$-variant of $\emph{Pos Int}^{\constant{a}\constant{b}}$ makes $\parhorseshoe{\parconjunction{\Cpp{\constant{a}}{\constant{b}}}{\Dppp{\constant{a}}{\constant{b}}{\constant{c}}}}{\Bppp{\constant{b}}{\constant{a}}{\constant{c}}}$ true.  Hence $\universal{\variable{z}}\parhorseshoe{\parconjunction{\Cpp{\constant{a}}{\constant{b}}}{\Dppp{\constant{a}}{\constant{b}}{\variable{z}}}}{\Bppp{\constant{b}}{\constant{a}}{\variable{z}}}$ is true on every $\constant{a},\constant{b}$-variant of \emph{Pos Int}. 

We may repeat this kind of reasoning to add quantifiers and replace the two remaining constants with variables.   Every $\constant{b}$-variant of $\emph{Pos Int}^{\constant{a}}$ makes $\universal{\variable{z}}\parhorseshoe{\parconjunction{\Cpp{\constant{a}}{\constant{b}}}{\Dppp{\constant{a}}{\constant{b}}{\variable{z}}}}{\Bppp{\constant{b}}{\constant{a}}{\variable{z}}}$ true.  So, by the definition of truth for $\forall$, $\universal{\variable{y}}\universal{\variable{z}}\parhorseshoe{\parconjunction{\Cpp{\constant{a}}{\variable{y}}}{\Dppp{\constant{a}}{\variable{y}}{\variable{z}}}}{\Bppp{\variable{y}}{\constant{a}}{\variable{z}}}$ is true on all $\constant{a}$-variants of \emph{Pos Int}.  It follows that $\universal{\variable{x}}\universal{\variable{y}}\universal{\variable{z}}\parhorseshoe{\parconjunction{\Cpp{\variable{x}}{\variable{y}}}{\Dppp{\variable{x}}{\variable{y}}{\variable{z}}}}{\Bppp{\variable{y}}{\variable{x}}{\variable{z}}}$ is true on \emph{Pos Int}.

\bigskip

(ii) The model \emph{States} assigns to $\CC$ pairs of states $\langle \variable{u},\variable{v}\rangle$ where $\variable{u}>\variable{v}$ (area), to $\DD$ triples of states $\langle \variable{u},\variable{v},\variable{w}\rangle$ where $\variable{u}+\variable{v}>\variable{w}$ (area), and to $\BB$ triples of states $\langle \variable{v},\variable{u},\variable{w}\rangle$ where $\variable{v}$ is north of $\variable{u}$ and $\variable{w}$.  As in the last section, let's reason using an instance of $\universal{\variable{x}}\universal{\variable{y}}\universal{\variable{z}}\parhorseshoe{\parconjunction{\Cpp{\variable{x}}{\variable{y}}}{\Dppp{\variable{x}}{\variable{y}}{\variable{z}}}}{\Bppp{\variable{y}}{\variable{x}}{\variable{z}}}$; again, let's use $\parhorseshoe{\parconjunction{\Cpp{\constant{a}}{\constant{b}}}{\Dppp{\constant{a}}{\constant{b}}{\constant{c}}}}{\Bppp{\constant{b}}{\constant{a}}{\constant{c}}}$.

Now consider an $\constant{a},\constant{b},\constant{c}$-variant of \emph{States} that assigns $\constant{a}$ to Alaska, $\constant{b}$ to Delaware, and $\constant{c}$ to Rhode Island.  
Alaska has an area of approximately $1.7\times{}10^6\text{ km}^2$, Delaware an area of approximately $2.5\times{}10^3\text{ km}^2$, and Rhode Island an area of approximately $1.5\times{}10^3\text{ km}^2$.
Hence $\emph{States}^{\constant{a}\constant{b}\constant{c}}(\constant{a})>\emph{States}^{\constant{a}\constant{b}\constant{c}}(\constant{b})$ (area), $\emph{States}^{\constant{a}\constant{b}\constant{c}}(\constant{a})+\emph{States}^{\constant{a}\constant{b}\constant{c}}(\constant{b})>\emph{States}^{\constant{a}\constant{b}\constant{c}}(\constant{c})$ (area), but $\emph{States}^{\constant{a}\constant{b}\constant{c}}(\constant{b})$ is not north of both $\emph{States}^{\constant{a}\constant{b}\constant{c}}(\constant{a})$ and $\emph{States}^{\constant{a}\constant{b}\constant{c}}(\constant{c})$; that is, Delaware is not north of Alaska. 
So, by clause (3) of the definition of truth, \ref{Truth for GQL Sentence}, $\Cpp{\constant{a}}{\constant{b}}$ and $\Dppp{\constant{a}}{\constant{b}}{\constant{c}}$ are true on $\emph{States}^{\constant{a}\constant{b}\constant{c}}$, while $\Bppp{\constant{b}}{\constant{a}}{\constant{c}}$ is false on $\emph{States}^{\constant{a}\constant{b}\constant{c}}$. 
So by the definition of truth ($\WEDGE$ and $\HORSESHOE$), $\parhorseshoe{\parconjunction{\Cpp{\constant{a}}{\constant{b}}}{\Dppp{\constant{a}}{\constant{b}}{\constant{c}}}}{\Bppp{\constant{b}}{\constant{a}}{\constant{c}}}$ is false on $\emph{States}^{\constant{a}\constant{b}\constant{c}}$. 

We may word this slightly differently: there is a $\constant{c}$-variant of $\emph{States}^{\constant{a}\constant{b}}$ on which $\parhorseshoe{\parconjunction{\Cpp{\constant{a}}{\constant{b}}}{\Dppp{\constant{a}}{\constant{b}}{\constant{c}}}}{\Bppp{\constant{b}}{\constant{a}}{\constant{c}}}$ is false.  It follows (by the definition of truth for $\forall$) that $\universal{\variable{z}}\parhorseshoe{\parconjunction{\Cpp{\constant{a}}{\constant{b}}}{\Dppp{\constant{a}}{\constant{b}}{\variable{z}}}}{\Bppp{\constant{b}}{\constant{a}}{\variable{z}}}$ is false on $\emph{States}^{\constant{a}\constant{b}}$.  So, there is, in turn, a $\constant{b}$-variant of $\emph{States}^{\constant{a}}$ on which $\universal{\variable{z}}\parhorseshoe{\parconjunction{\Cpp{\constant{a}}{\constant{b}}}{\Dppp{\constant{a}}{\constant{b}}{\variable{z}}}}{\Bppp{\constant{b}}{\constant{a}}{\variable{z}}}$ is false.  So, again, by the definition of truth for $\forall$, that $\universal{\variable{y}}\universal{\variable{z}}\parhorseshoe{\parconjunction{\Cpp{\constant{a}}{\variable{y}}}{\Dppp{\constant{a}}{\variable{y}}{\variable{z}}}}{\Bppp{\variable{y}}{\constant{a}}{\variable{z}}}$ is false on $\emph{States}^{\constant{a}}$.

Finally, because there is an $\constant{a}$-variant of \emph{States} on which $\universal{\variable{y}}\universal{\variable{z}}\parhorseshoe{\parconjunction{\Cpp{\constant{a}}{\variable{y}}}{\Dppp{\constant{a}}{\variable{y}}{\variable{z}}}}{\Bppp{\variable{y}}{\constant{a}}{\variable{z}}}$ is false, $\universal{\variable{x}}\universal{\variable{y}}\universal{\variable{z}}\parhorseshoe{\parconjunction{\Cpp{\variable{x}}{\variable{y}}}{\Dppp{\variable{x}}{\variable{y}}{\variable{z}}}}{\Bppp{\variable{y}}{\variable{x}}{\variable{z}}}$ is false on \emph{States} (definition of truth, $\forall$).
\end{PROOF}

\subsection{Logical Truth: QT, QF, \& QC}\label{QT QF QI GQL}
The concepts of quantificational truth (\CAPS{qt}), quantificational falsehood (\CAPS{qf}), and quantificational contingency (\CAPS{qc}) are defined for \GQL{} exactly as in \GQL{}1.\footnote{See section \vref{QT QT QI} of the last chapter.} 

%%%%%%%%%%%%%%%%%%%%%%%%%%%%%%%%%%%%%%%%%%%%%%%%%%
\subsection{Entailment and other Relations}\label{GQL Entailment and other Relations}
%%%%%%%%%%%%%%%%%%%%%%%%%%%%%%%%%%%%%%%%%%%%%%%%%%

The concepts for entailment and the other logical relations are also defined for \GQL{} exactly as in \GQL{}1.\footnote{See section \vref{GQL1 Entailment and other Relations} of the last chapter.} 


%%%%%%%%%%%%%%%%%%%%%%%%%%%%%%%%%%%%%%%%%%%%%%%%%%
\section{The Dragnet Theorem}\label{Dragnet Theorem}
%%%%%%%%%%%%%%%%%%%%%%%%%%%%%%%%%%%%%%%%%%%%%%%%%%

In example \ref{GQL Entailment Example 2} we were able to establish that the entailment  \mention{$\universal{\variable{x}}\Gp{\variable{x}}\sdtstile{}{}\Gp{\constant{b}}$} holds by reasoning as follows:

\begin{enumerate}[label=(\roman*)]
	\item Any model $\IntA$ that makes $\universal{\variable{x}}\Gp{\variable{x}}$ true makes $\Gp{\constant{a}}$ true on all $\constant{a}$-variants of $\IntA$.
	\item All $\constant{a}$-variants of $\IntA$ share the same domain and make the same set assignment to $\GG$.
	\item There is some $\constant{a}$-variant of $\IntA$, $\As{\constant{a}}{}$, such that $\As{\constant{a}}{}(\constant{a})=\IntA(\constant{b})$.
	\item Because $\As{\constant{a}}{}(\constant{a})\in\As{\constant{a}}{}(\GG)$ and $\As{\constant{a}}{}(\GG)=\IntA(\GG)$, $\As{\constant{a}}{}(\constant{a})\in\As{}{}(\GG)$.
	\item Because $\As{\constant{a}}{}(\constant{a})\in\As{}{}(\GG)$ and $\As{\constant{a}}{}(\constant{a})=\IntA(\constant{b})$, $\As{}{}(\constant{b})\in\As{}{}(\GG)$.
	Thus, $\Gp{\constant{b}}$ is true on $\IntA$.
\end{enumerate}

\noindent{}By analogous reasoning, we can show that $\universal{\variable{x}}\Gp{\variable{x}}$ entails $\Gp{\constant{c}}$, $\Gp{\constant{d}}$, $\Gp{\constant{e}}$, and so on.  But what if we want to prove that the entailment holds, regardless of what constant we pick?  Proving that the entailment always holds, regardless of the constant, requires the use of metatheory.  For such proofs we'll need to use a metavariable, in this case $\variable{t}$: $\universal{\variable{x}}\Gp{\variable{x}}\sdtstile{}{}\Gp{\variable{x}}\variable{t}/\variable{x}$, where $\variable{t}$ is any constant.

And we will also want to prove rather stronger entailments, such as the following: $\universal{\variable{x}}\CAPPHI\sdtstile{}{}\CAPPHI\constant{b}/\variable{x}$, where $\CAPPHI$ is some \GQL{} sentence with only $\variable{x}$ free.

Such proofs are much easier to complete with the following theorem, which we call the \mention{Dragnet Theorem}.  Let's say that $\CAPPHI$ and $\CAPPHI^*$ are sentences of \GQL{} such that $\CAPPHI^*=\CAPPHI\variable{s}/\variable{t}$, where $\variable{s}$ and $\variable{t}$ are constants.\footnote{Recall from section \pmvref{MathEnglishVariableSubEx1} in Chapter \ref{quantifierlogic1} that $\CAPPHI\variable{s}/\variable{t}$ is the sentence you get by replacing each unbound token of $\variable{t}$ in $\CAPPHI$ with a token of $\variable{s}$.  With Dragnet, we are concerned solely with cases in which $\variable{s}$ and $\variable{t}$ are constants.}  Let's also say that we have two models, $\As{}{1}$ and $\As{}{2}$, that make all the same assignments except that what $\As{}{1}$ assigns to $\variable{t}$, $\As{}{2}$ assigns to $\variable{s}$. I.e., $\As{}{1}(\variable{t})=\As{}{2}(\variable{s}).$  It would seem that $\CAPPHI$ is true on $\As{}{1}$ \Iff $\CAPPHI^*$ is true on $\As{}{2}$.  This intuitively plausible claim is true, and the Dragnet theorem proves it.  There are many cases in which Dragnet is crucial for proving properties or relations of sentences (e.g., in theorem \pmvref{Soundness of Quantifier Logic}, and \pmvref{MethodLemmaC}).\footnote{Although the basic claim behind Dragnet seems obviously true, it turns out that stating the theorem precisely and proving is difficult. 
There are different ways to state the theorem, not all exactly equivalent. See \citealt[66]{Mates1972} and \citealt[577]{Bergmann2003} for two alternative examples.}

We need not restrict the claim to pairs of sentences that vary on only one constant.  A pair of otherwise identical sentences $\CAPPHI$ and $\CAPPHI^*$ may differ on as many constants as you like, and as long as two models, $\IntA_1$ and $\IntA_2$, make the same assignments for the replacement constants in $\CAPPHI^*$, $\CAPPHI$ is true on $\IntA_1$ \Iff $\CAPPHI^*$ is true on $\IntA_2$. 
That is, if there are two sentences $\CAPPHI$ and $\CAPPHI^*$ such that  $\CAPPHI^*=\CAPPHI\variable{s}_1/\variable{t}_1,\variable{s}_2/\variable{t}_2,\ldots,\variable{s}_{\integer{i}}/\variable{t}_{\integer{i}}$, and two models $\IntA_1$ and $\IntA_2$ such that $\IntA_1(\variable{t}_1)=\IntA_2(\variable{s}_1)$, $\IntA_1(\variable{t}_2)=\IntA_2(\variable{s}_2)$, $\ldots$, and $\IntA_1(\variable{t}_{\integer{i}})=\IntA_2(\variable{s}_{\integer{i}})$, then $\CAPPHI$ is true on $\IntA_1$ \Iff $\CAPPHI^*$ is true on $\IntA_2$.  We prove this theorem below.

\begin{THEOREM}{\LnpTC{The Dragnet Theorem} The Dragnet Theorem:}
If 
\begin{cenumerate}
\item a \GQL{} sentence $\CAPPHI$ contains one or more of each of the constant(s) $\variable{t}_1$, $\variable{t}_2$, $\ldots$, $\variable{t}_{\variable{i}}$, and another \GQL{} sentence $\CAPPHI^*=\CAPPHI\variable{s}_1/\variable{t}_1,\variable{s}_2/\variable{t}_2,\ldots,\variable{s}_{\integer{i}}/\variable{t}_{\integer{i}}$; and
\item The models $\As{}{1}$ and $\As{}{2}$ differ only in that what $\As{}{1}$ assigns to $\variable{t}_1$, $\As{}{2}$ assigns to $\variable{s}_1$, $\As{}{1}(\variable{t}_2)=\As{}{2}(\variable{s}_2)$, $\ldots$, $\As{}{1}(\variable{t}_{\integer{i}})=\As{}{2}(\variable{s}_{\integer{i}})$,\footnote{In 
other words, $\As{}{1}$ and $\As{}{2}$ make the same assignments to $\integer{U}$, the predicates, the sentence letters, and the constants, except that what $\As{}{1}$ assigns to $\variable{t}_1$, $\As{}{2}$ assigns to $\variable{s}_1$; what $\As{}{1}$ assigns to $\variable{t}_2$, $\As{}{2}$ assigns to $\variable{s}_2$; and so on.}
\end{cenumerate} 
then: $\CAPPHI$ is true on $\As{}{1}$ iff $\CAPPHI^*$ is true on $\As{}{2}$.

\end{THEOREM}
\noindent{}We proceed by recursive proof.
\begin{PROOF} 
Throughout the proof we treat $^*$ as a function that takes a \GQL{} sentence and returns the sentence you get by replacing all occurrences of $\variable{t}_1$ with $\variable{s}_1$, $\variable{t}_2$ with $\variable{s}_2$, and so on. 
Without further stipulation, $\CAPPHI^*=\CAPPHI\variable{s}_1/\variable{t}_1,\variable{s}_2/\variable{t}_2,\ldots,\variable{s}_{\integer{i}}/\variable{t}_{\integer{i}}$, $\CAPPSI^*=\CAPPSI\variable{s}_1/\variable{t}_1,\variable{s}_2/\variable{t}_2,\ldots,\variable{s}_{\integer{i}}/\variable{t}_{\integer{i}}$, $\CAPTHETA^*=\CAPTHETA\variable{s}_1/\variable{t}_1,\variable{s}_2/\variable{t}_2,\ldots,\variable{s}_{\integer{i}}/\variable{t}_{\integer{i}}$, and so on for all metavariables.  Accordingly, $\CAPPHI$ and $\CAPPHI^*$ satisfy Dragnet condition (1) above; and so do $\CAPPSI$ and $\CAPPSI^*$, as well as $\CAPTHETA$ and $\CAPTHETA^*$, etc. 

Additionally, throughout the proof we assume that $\As{}{1}$ and $\As{}{2}$ are two arbitrary models that satisfy Dragnet condition (2) above.
\begin{description}
\item[Base Step:] $\CAPPHI$ is atomic.
\begin{cenumerate}
\item If $\CAPPHI$ is a sentence letter, then there are no constants and $\CAPPHI$ and $\CAPPHI^*$ must be identical. So, clearly, $\CAPPHI$ is true on $\As{}{1}$ iff $\CAPPHI^*$ is true on $\As{}{2}$.
\item Say that $\CAPPHI$ is a predicate letter $\PP$ followed by one constant: $\Pp{\variable{t}}$.  Then $\CAPPHI^*$ is the same, but with a different constant: $\Pp{\variable{s}}$.  In accordance with Dragnet condition (2), $\As{}{1}$ and $\As{}{2}$ make all the same assignments except that what $\As{}{1}$ assigns to $\variable{t}$, $\As{}{2}$ assigns to $\variable{s}$.  So, $\As{}{1}(\PP)=\As{}{2}(\PP)$ and $\As{}{1}(\variable{t})=\As{}{2}(\variable{s})$.

By the definition of truth, 

\begin{center}
$\Pp{\variable{t}}$ is true on $\As{}{1}$ \Iff $\As{}{1}(\variable{t})\in\As{}{1}(\PP)$.
\end{center}

But because $\As{}{1}(\PP)=\As{}{2}(\PP)$, we can substitute on the RHS to get:

\begin{center}
	$\Pp{\variable{t}}$ is true on $\As{}{1}$ \Iff $\As{}{1}(\variable{t})\in\As{}{2}(\PP)$.
\end{center}

We also know that $\As{}{1}(\variable{t})=\As{}{2}(\variable{s})$, so we can make another substitution to get:

\begin{center}
	$\Pp{\variable{t}}$ is true on $\As{}{1}$ \Iff $\As{}{2}(\variable{s})\in\As{}{2}(\PP)$.
\end{center}

Finally, by the definition of truth, we can replace the RHS to get:

\begin{center}
	$\Pp{\variable{t}}$ is true on $\As{}{1}$ \Iff $\Pp{\variable{s}}$ is true on $\As{}{2}$.
\end{center}

And that's what we wanted to prove for this base clause.

\item Say that $\CAPPHI$ is a predicate letter $\PP$ followed by $n$ constants,  $\variable{q}_{\integer{1}}$, $\variable{q}_{\integer{2}}$, $\ldots$, $\variable{q}_{\integer{n}}$.  As we stipulated earlier, $\CAPPHI^*$ is exactly the same, except that some or all of the constants of $\CAPPHI$ have been replaced with other constants.  Let's assume, without loss of generality, that $\CAPPHI=\Pp{\variable{q}_{\integer{1}}\ldots\variable{t}_1\ldots\variable{t}_2\ldots\variable{t}_{\integer{i}}\ldots\variable{q}_{\integer{n}}}$.  That is, let $\variable{t}_1$, $\variable{t}_2$, $\ldots $, and $\variable{t}_\integer{i}$ be the constants of $\CAPPHI$ that will be replaced in $\CAPPHI^*$.  So, $\CAPPHI^*$ is $\Pp{\variable{q}_{\integer{1}}\ldots\variable{s}_1\ldots\variable{s}_2\ldots\variable{s}_{\integer{i}}\ldots\variable{q}_{\integer{n}}}$. By the definition of truth, 
\begin{center}
$\CAPPHI$ is true on $\As{}{1}$ iff $\langle\As{}{1}(\variable{q}_{\integer{1}}),\ldots,\As{}{1}(\variable{t}_1),\ldots,\As{}{1}(\variable{t}_{\integer{i}}),\ldots,\As{}{1}(\variable{q}_{\integer{n}})\rangle\in\As{}{1}(\PP)$.
\end{center}
We know that $\As{}{1}(\PP)=\As{}{2}(\PP)$, so we substitute on the RHS to get:
\begin{center}
	$\CAPPHI$ is true on $\As{}{1}$ iff $\langle\As{}{1}(\variable{q}_{\integer{1}}),\ldots,\As{}{1}(\variable{t}_1),\ldots,\As{}{1}(\variable{t}_{\integer{i}}),\ldots,\As{}{1}(\variable{q}_{\integer{n}})\rangle\in\As{}{2}(\PP)$.
\end{center}
Because models $\As{}{1}$ and $\As{}{2}$ meet Dragnet condition (2),  $\As{}{1}(\variable{t}_1)=\As{}{2}(\variable{s}_1)$, $\As{}{1}(\variable{t}_2)=\As{}{2}(\variable{s}_2)$, $\ldots$, $\As{}{1}(\variable{t}_{\integer{i}})=\As{}{2}(\variable{s}_{\integer{i}})$. Hence, by more substitutions on the RHS we get:
\begin{center}
$\CAPPHI$ is true on $\As{}{1}$ iff $\langle\As{}{1}(\variable{q}_{\integer{1}}),\ldots,\As{}{2}(\variable{s}_1),\ldots,\As{}{2}(\variable{s}_{\integer{i}}),\ldots,\As{}{1}(\variable{q}_{\integer{n}})\rangle\in\As{}{2}(\PP)$.
\end{center}
The models $\As{}{1}$ and $\As{}{2}$ otherwise make all the same assignments; so
for the constants that aren't changed from $\CAPPHI$ to $\CAPPHI^*$, they are each assigned the same object on both models. I.e., for the unchanged constants of $\variable{q}_{\integer{1}}$ through $\variable{q}_{\integer{n}}$ we know that $\As{}{1}(\variable{q}_{\integer{1}})=\As{}{2}(\variable{q}_{\integer{1}})$, $\As{}{1}(\variable{q}_{\integer{2}})=\As{}{2}(\variable{q}_{\integer{2}})$, $\ldots$,  $\As{}{1}(\variable{q}_{\integer{n}})=\As{}{2}(\variable{q}_{\integer{n}})$.  Thus, we can carry out even more substitutions to get: 
\begin{center}
$\CAPPHI$ is true on $\As{}{1}$ iff $\langle\As{}{2}(\variable{q}_{\integer{1}}),\ldots,\As{}{2}(\variable{s}_1),\ldots,\As{}{2}(\variable{s}_{\integer{i}}),\ldots,\As{}{2}(\variable{q}_{\integer{n}})\rangle\in\As{}{2}(\PP)$.
\end{center}
And we also know, by the definition of truth, that:
\begin{center} $\langle\As{}{2}(\variable{q}_{\integer{1}}),\ldots,\As{}{2}(\variable{s}_1),\ldots,\As{}{2}(\variable{s}_{\integer{i}}),\ldots,\As{}{2}(\variable{q}_{\integer{n}})\rangle\in\As{}{2}(\PP)$ iff $\CAPPHI^*$ is true on $\As{}{2}$.
\end{center}
So it follows that $\CAPPHI$ is true on $\As{}{1}$ iff $\CAPPHI^*$ is true on $\As{}{2}$.
\end{cenumerate}
\item[Inheritance Step:] \hfill 
\begin{description}
\item[Recursive Assumption] Assume that the Dragnet theorem holds for all \GQL{} sentences of order $k$ or less, and that $\CAPPHI$ is an \GQL{} sentence of order $k+1$.  Consider the following ways in which $\CAPPHI$ might be of order $k+1$.

\item[Negation:] $\CAPPHI$ is a negation; i.e., is of the form $\negation{\CAPPSI}$. 
 $\CAPPHI^*$ is the result of substituting $\variable{s}_1$ for $\variable{t}_1$, $\variable{s}_2$ for $\variable{t}_2$, etc., in $\negation{\CAPPSI}$, which is the same as making the substitutions in $\CAPPSI$ and putting a \mention{$\NEGATION$} in front. 
That is, $\negation{(\CAPPSI)^*}$ is the same formula as $(\negation{\CAPPSI})^*$. 
Because $\CAPPSI$ is of order $k$, by the recursive assumption (RA):
\begin{center}
$\CAPPSI$ is true on $\As{}{1}$ iff $\CAPPSI^*$ is true on $\As{}{2}$,
\end{center}
and it follows from this that:
\begin{center}
$\CAPPSI$ is false on $\As{}{1}$ iff $\CAPPSI^*$ is false on $\As{}{2}$.
\end{center}
The sentence $\CAPPSI$ is false on $\As{}{1}$ \Iff $\negation{\CAPPSI}$ is true on $\As{}{1}$; and the same holds for $\CAPPSI^*$.  So,
\begin{center}
$\negation{\CAPPSI}$ is true on $\As{}{1}$ iff $\negation{\CAPPSI^*}$ is true on $\As{}{2}$,
\end{center}
which is what we want to show.

\item[Conjunction:] $\CAPPHI$ is a conjunction; i.e., is of the form $\parconjunction{\CAPPSI_1}{\conjunction{\CAPPSI_2}{\conjunction{\ldots}{\CAPPSI_{\integer{n}}}}}$. 
The sentence $\CAPPHI^*$ is the result of substituting $\variable{s}_1$ for $\variable{t}_1$, $\variable{s}_2$ for $\variable{t}_2$, etc., in $\CAPPHI$, which is the same as if we make the substitutions in each conjunct and then put \mention{$\WEDGE$}(s) between them, i.e. $\parconjunction{\CAPPSI_1^*}{\conjunction{\CAPPSI_2^*}{\conjunction{\ldots}{\CAPPSI_{\integer{n}}^*}}}$. 

Each conjunct $\CAPPSI_{\integer{j}}$ is of order $k$ or lower.  (Let $\CAPPSI_{\integer{j}}$ be the $\integer{j}^{th}$ conjunct of $\CAPPHI$, where $\integer{j}$ is some arbitrary integer from $1$ to $\integer{n}$.)  So, by RA,
\begin{center}
$\CAPPSI_j$ is true on $\As{}{1}$ iff $\CAPPSI_j^*$ is true on $\As{}{2}$.
\end{center}
There is one such biconditional for each conjunct of $\CAPPHI$.  So, conjoining all the left-hand sides and all the right-hand sides of these $n$ biconditionals we get:
\begin{center}
All of $\CAPPSI_1$, $\CAPPSI_2$, $\ldots$, $\CAPPSI_{\integer{n}}$ are true on $\As{}{1}$ iff all of $\CAPPSI_1^*$, $\CAPPSI_2^*$, $\ldots$, $\CAPPSI_{\integer{n}}^*$ are true on $\As{}{2}$.
\end{center}
So, by the definition of truth for $\WEDGE$,
\begin{center}
$\parconjunction{\CAPPSI_1}{\conjunction{\CAPPSI_2}{\conjunction{\ldots}{\CAPPSI_{\integer{n}}}}}$ is true on $\As{}{1}$ iff  $\parconjunction{\CAPPSI_1^*}{\conjunction{\CAPPSI_2^*}{\conjunction{\ldots}{\CAPPSI_{\integer{n}}^*}}}$ is true on $\As{}{2}$.
\end{center}
And that's what we want to prove.

\item[Disjunction:] We leave this case for the reader to do as an exercise.

\item[Conditional:] $\CAPPHI$ is of the form $\parhorseshoe{\CAPPSI}{\CAPTHETA}$. By the definition of truth for $\HORSESHOE$ we know that
\begin{center}
$\parhorseshoe{\CAPPSI}{\CAPTHETA}$ is true on $\As{}{1}$
  iff  either//
   (i) $\CAPPSI$ is false on $\As{}{1}$, or (ii)
   $\CAPTHETA$ is true on $\As{}{1}$.
\end{center}
$\CAPPSI$ is of order $k$ or lower; so by the RA, we know that:
\begin{center}
$\CAPPSI$ is true on $\As{}{1}$ iff $\CAPPSI^*$ is true on $\As{}{2}$.
\end{center}
From which it follows that:
\begin{center}
$\CAPPSI$ is false on $\As{}{1}$ iff $\CAPPSI^*$ is false on $\As{}{2}$.
\end{center}
Using this and the earlier result in this clause, we substitute to get that
\begin{center}
$\parhorseshoe{\CAPPSI}{\CAPTHETA}$ is true on $\As{}{1}$
  iff  either  
	  (i) $\CAPPSI^*$ is false on $\As{}{2}$,
        or (ii) $\CAPTHETA$ is true on $\As{}{1}$.
\end{center}
$\CAPTHETA$ is of order $k$ or lower; so by the RA we also know that:
\begin{center}
$\CAPTHETA$ is true on $\As{}{1}$ iff $\CAPTHETA^*$ is true on $\As{}{2}$.
\end{center}
We substitute again to get:
\begin{center}
$\parhorseshoe{\CAPPSI}{\CAPTHETA}$ is true on $\As{}{1}$
 iff either 
  (i) $\CAPPSI^*$ is false on $\As{}{2}$,
 or (ii) $\CAPTHETA^*$ is true on $\As{}{2}$.
\end{center}
Finally, by the definition of truth ($\HORSESHOE$), we can substitute to get:
\begin{center}
$\parhorseshoe{\CAPPSI}{\CAPTHETA}$ is true on $\As{}{1}$
 iff $\parhorseshoe{\CAPPSI^*}{\CAPTHETA^*}$ is true on $\As{}{2}$.
\end{center}
From which we get what we wanted to show:
\begin{center}
$\parhorseshoe{\CAPPSI}{\CAPTHETA}$ is true on $\As{}{1}$ iff  $\parhorseshoe{\CAPPSI}{\CAPTHETA}^*$ is true on $\As{}{2}$.
\end{center}

\item[Biconditional:] We leave this case for the reader to do as an exercise.

\item[Universal Quantification:] $\CAPPHI$ is a universal quantification; i.e., is of the form $\universal{\BETA}\CAPPSI$, where $\CAPPSI$ is a formula that has exactly one free variable, $\BETA$.  Because $\BETA$ is a variable, and thus is different from all constants, $(\universal{\BETA}\CAPPSI)^*=\universal{\BETA}\CAPPSI^*$.

According to the definition of truth for $\forall$,
\begin{center}
$\universal{\BETA}\CAPPSI$ is true on $\As{}{1}$
  iff $\CAPPSI\variable{q}/\BETA$ is true on every $\variable{q}$-variant of $\As{}{1}$, \\where $\variable{q}$ is the first constant not in $\CAPPSI$,
\end{center}
 and...
\begin{center}
	$\universal{\BETA}\CAPPSI^*$ is true on $\As{}{2}$
	iff $\CAPPSI^*\variable{r}/\BETA$ is true on every $\variable{r}$-variant of $\As{}{2}$, \\where $\variable{r}$ is the first constant not in $\CAPPSI^*$.\footnote{The definition of truth for $\forall$ uses \mention{$\variable{t}$} to stand for the first constant not in the formula in question.  Here we instead use \mention{$\variable{q}$} and \mention{$\variable{r}$} so that we don't confuse these with the list of constants $\variable{t}_1$, $\variable{t}_2$, $\ldots$, $\variable{t}_{\integer{i}}$. We also take care to distinguish \mention{$\variable{q}$} and \mention{$\variable{r}$}.  We cannot assume that $\variable{q}=\variable{r}$, because there are certain cases in which they're different. For example, consider the case such that $\universal{\BETA}\CAPPSI=\universal{\variable{x}}\parhorseshoe{\Dp{\variable{x}}}{\Gpp{\variable{x}}{\constant{b}}}$ and $\universal{\BETA}\CAPPSI^*=\universal{\BETA}\CAPPSI\constant{a}/\constant{b}=\universal{\variable{x}}\parhorseshoe{\Dp{\variable{x}}}{\Gpp{\variable{x}}{\constant{a}}}$.}
\end{center}

There are two differences between the sentences $\CAPPSI\variable{q}/\BETA$ and $\CAPPSI^*\variable{r}/\BETA$.  First, where $\CAPPSI\variable{q}/\BETA$ contains the constant $\variable{q}$, the sentence $\CAPPSI^*\variable{r}/\BETA$ contains the constant $\variable{r}$.  Second, whereas constants $\variable{t}_1$, $\ldots$, $\variable{t}_\integer{n}$ are in $\CAPPSI\variable{q}/\BETA$, they are replaced by the constants $\variable{s}_1$, $\ldots$, $\variable{s}_\integer{n}$ in $\CAPPSI^*\variable{r}/\BETA$.  Apart from these differences $\CAPPSI\variable{q}/\BETA$ and $\CAPPSI^*\variable{r}/\BETA$ are identical, and so they satisfy Dragnet condition (1).

We must prove that $\universal{\BETA}\CAPPSI$ is true on $\As{}{1}$ \Iff $\universal{\BETA}\CAPPSI^*$ is true on $\As{}{2}$.  To show this, we can take the RHS from each of the two biconditionals above, and prove the following biconditional:

\begin{center}
($\CAPPSI\variable{q}/\BETA$ is true on every $\variable{q}$-variant of $\As{}{1}$) iff\\
($\CAPPSI^*\variable{r}/\BETA$ is true on every $\variable{r}$-variant of $\As{}{2}$).
\end{center}
Because this will take some work, we separate it off as a subproof.
\begin{SUBPROOF} 
$(\Rightarrow)$ Assume that the \CAPS{lhs} of the biconditional is true: $\CAPPSI\variable{q}/\BETA$ is true on every $\variable{q}$-variant of $\As{}{1}$.  We want to show that $\CAPPSI^*\variable{r}/\BETA$ is true on every $\variable{r}$-variant of $\As{}{2}$.

Assume some arbitrary $\variable{q}$-variant of $\As{}{1}$; call it $\As{\variable{q}}{1}$.  Let's give an arbitrary name to the object that $\As{\variable{q}}{1}$ assigns to $\variable{q}$: \mention{Kate}.  The models $\As{}{1}$ and $\As{}{2}$ share the same domain, and thus so do all variants of those models.  Therefore, we know that there is an $\variable{r}$-variant of $\As{}{2}$ that assigns Kate to $\variable{r}$.  Let's call this $\variable{r}$-variant $\As{\variable{r}}{2}$.  To put it another way, let $\As{\variable{r}}{2}$ be such that
\begin{center}
	$\As{\variable{q}}{1}(\variable{q})=\As{\variable{r}}{2}(\variable{r})$.
\end{center}

We know that $\As{}{1}(\variable{t}_1)=\As{}{2}(\variable{s}_1)$, $\As{}{1}(\variable{t}_2)=\As{}{2}(\variable{s}_2)$, $\ldots$, $\As{}{1}(\variable{t}_{\integer{i}})=\As{}{2}(\variable{s}_{\integer{i}})$.  We need to show that the same equivalences hold for the model variants, $\As{q}{1}$ and $\As{r}{2}$.   The model $\As{}{1}$ differs from $\As{\variable{q}}{1}$ on the assignment to $\variable{q}$, but $\variable{q}$ is the first constant not in $\CAPPSI$.  Hence, $\variable{q}$ is not in the list of constants $\variable{t}_1$, $\ldots$, $\variable{t}_\integer{i}$.  $\As{}{1}$ and $\As{\variable{q}}{1}$ differ only on $\variable{q}$, so it follows that they make the same assignments to each of $\variable{t}_1$, $\ldots$, $\variable{t}_\integer{i}$.  Analogous reasoning shows that $\As{}{2}$ and $\As{r}{2}$ make the same assignments to each of $\variable{s}_1$, $\ldots$, $\variable{s}_\integer{i}$.  Therefore, $\As{q}{1}(\variable{t}_1)=\As{r}{2}(\variable{s}_1)$, $\As{q}{1}(\variable{t}_2)=\As{r}{2}(\variable{s}_2)$, $\ldots$, $\As{q}{1}(\variable{t}_{\integer{i}})=\As{r}{2}(\variable{s}_{\integer{i}})$.

And because $\As{\variable{q}}{1}(\variable{q})=\As{\variable{r}}{2}(\variable{r})$, it follows that what $\As{q}{1}$ assigns to each constant of $\CAPPSI\variable{q}/\BETA$ is the same as what $\As{r}{2}$ assigns to the analogous constant in the corresponding location of $\CAPPSI^*\variable{r}/\BETA$.  Thus, $\As{q}{1}$ and $\As{r}{2}$ satisfy Dragnet condition (2).

The sentences $\CAPPSI\variable{q}/\BETA$ and $\CAPPSI^*\variable{r}/\BETA$ are each of order $k$, so, by RA,
 
\begin{center}
	$\CAPPSI\variable{q}/\BETA$ is true on $\As{q}{1}$ \Iff $\CAPPSI^*\variable{r}/\BETA$ is true on $\As{r}{2}$.
\end{center}

We initially assumed that $\CAPPSI\variable{q}/\BETA$ is true on every $\variable{q}$-variant of $\As{}{1}$.  It follows that $\CAPPSI\variable{q}/\BETA$ is true on $\As{q}{1}$.  Therefore, given the above, $\CAPPSI^*\variable{r}/\BETA$ is true on $\As{\variable{r}}{2}$.

We also assumed a single $\variable{q}$-variant of $\As{}{1}$, $\As{q}{1}$; but nothing in our argument depends on anything specific about the object $\As{q}{1}$ assigns to $\variable{q}$.  We did nothing more than give it an arbitrary name.  So the argument above is perfectly general, and it applies equally for any other $\variable{q}$-variant as well.  That is, for each $\variable{q}$-variant of $\As{}{1}$, there is a corresponding $\variable{r}$-variant of $\As{}{2}$, such that, 
\begin{center}
	$\As{\variable{q}}{1}(\variable{q})=\As{\variable{r}}{2}(\variable{r})$.
\end{center}
Therefore, because $\CAPPSI\variable{q}/\BETA$ is true on every $\variable{q}$-variant of $\As{}{1}$, it follows that $\CAPPSI^*\variable{r}/\BETA$ is true on every $\variable{r}$-variant of $\As{}{2}$.\\

$(\Leftarrow)$ The argument in this section of the subproof perfectly mirrors that of the last. 

First assume the \CAPS{rhs} is true. That is, assume that $\CAPPSI^*\variable{r}/\BETA$ is true on every $\variable{r}$-variant of $\As{}{2}$.  We want to show that the \CAPS{lhs} follows, i.e., that $\CAPPSI\variable{q}/\BETA$ is true on every $\variable{q}$-variant of $\As{}{1}$.

Assume some arbitrary $\variable{r}$-variant of $\As{}{2}$, called $\As{\variable{r}}{2}$.  Let's name the object that $\As{\variable{r}}{2}$ assigns to $\variable{r}$ \mention{Irene}.  The variants of $\As{}{1}$ and $\As{}{2}$ share the same domain, so there's a $\variable{q}$-variant of $\As{}{1}$ that assigns Irene to $\variable{q}$.  Name that $\variable{q}$-variant $\As{\variable{q}}{1}$.  So,
\begin{center}
	$\As{\variable{r}}{2}(\variable{r})=\As{\variable{q}}{1}(\variable{q})$.
\end{center}

As before, $\As{}{1}(\variable{t}_1)=\As{}{2}(\variable{s}_1)$, $\As{}{1}(\variable{t}_2)=\As{}{2}(\variable{s}_2)$, $\ldots$, $\As{}{1}(\variable{t}_{\integer{i}})=\As{}{2}(\variable{s}_{\integer{i}})$.  The model $\As{}{2}$ differs from $\As{\variable{r}}{2}$ on $\variable{r}$, but $\variable{r}$ is the first constant not in $\CAPPSI^*$.  Hence, $\variable{r}$ is not any of the following constants: $\variable{s}_1$, $\ldots$, $\variable{s}_\integer{i}$.  $\As{}{2}$ and $\As{\variable{r}}{2}$ differ only on $\variable{r}$, so they make the same assignments to each of $\variable{s}_1$, $\ldots$, $\variable{s}_\integer{i}$.  Likewise, $\As{}{1}$ and $\As{q}{1}$ make the same assignments to each of $\variable{t}_1$, $\ldots$, $\variable{t}_\integer{i}$.  Therefore, $\As{q}{1}(\variable{t}_1)=\As{r}{2}(\variable{s}_1)$, $\As{q}{1}(\variable{t}_2)=\As{r}{2}(\variable{s}_2)$, $\ldots$, $\As{q}{1}(\variable{t}_{\integer{i}})=\As{r}{2}(\variable{s}_{\integer{i}})$.

Because $\As{\variable{r}}{2}(\variable{r})=\As{\variable{q}}{1}(\variable{q})$, it follows that what $\As{r}{2}$ assigns to each constant of $\CAPPSI^*\variable{r}/\BETA$ is the same as what $\As{q}{1}$ assigns to the analogous constant in the corresponding location of $\CAPPSI\variable{q}/\BETA$.  Thus, $\As{r}{2}$ and $\As{q}{1}$ satisfy Dragnet condition (2).

The sentences $\CAPPSI^*\variable{r}/\BETA$ and $\CAPPSI\variable{q}/\BETA$ are each of order $k$, so, by RA,

\begin{center}
	$\CAPPSI^*\variable{r}/\BETA$ is true on $\As{r}{2}$ \Iff $\CAPPSI\variable{q}/\BETA$ is true on $\As{q}{1}$.
\end{center}

We assumed that $\CAPPSI^*\variable{r}/\BETA$ is true on every $\variable{r}$-variant of $\As{}{2}$.  So $\CAPPSI^*\variable{r}/\BETA$ is true on $\As{r}{2}$.  Therefore, given the above, $\CAPPSI\variable{q}/\BETA$ is true on $\As{\variable{q}}{1}$.

As before, we used a single $\variable{r}$-variant of $\As{}{2}$, $\As{r}{2}$, without our argument depends on anything specific about the object it assigns to $\variable{r}$.  So the argument above is generalizable, and it applies equally to all other $\variable{r}$-variants.  That is, for each $\variable{r}$-variant of $\As{}{2}$, there is a corresponding $\variable{q}$-variant of $\As{}{1}$, such that, 
\begin{center}
	$\As{\variable{r}}{2}(\variable{r})=\As{\variable{q}}{1}(\variable{q})$.
\end{center}
So, because $\CAPPSI^*\variable{r}/\BETA$ is true on every $\variable{r}$-variant of $\As{}{2}$, it follows that $\CAPPSI\variable{q}/\BETA$ is true on every $\variable{q}$-variant of $\As{}{1}$.\\

\end{SUBPROOF}

We have established that
\begin{center}
	($\CAPPSI\variable{q}/\BETA$ is true on every $\variable{q}$-variant of $\As{}{1}$) iff\\
	($\CAPPSI^*\variable{r}/\BETA$ is true on every $\variable{r}$-variant of $\As{}{2}$),
\end{center}
...so, by the definition of truth for $\forall$, it follows that
\begin{center}
	$\universal{\BETA}\CAPPSI$ is true on $\As{}{1}$ \Iff $\universal{\BETA}\CAPPSI^*$ is true on $\As{}{2}$.
\end{center}

\item[Existential Quantification:] We leave this case for the reader to do as an exercise.
\end{description}
\item[Closure Step:] We have now exhausted the ways to construct a sentence of order $k+1$.  There is no other way to construct a sentence of \GQL{}, and so we have shown that the Dragnet theorem holds for all sentences $\CAPPHI$.
\end{description}
\end{PROOF}

Dragnet is particularly useful when making arguments about sentences with unknown structures.  To illustrate, let's look at the following example.

\begin{majorILnc}{\LnpEC{DragnetExampleTwo}}
 Let $\CAPPHI$ be some formula whose only free variable is $\variable{x}$.  Demonstrate the following for all such substitutions for $\CAPPHI$: $\universal{\variable{x}}\CAPPHI\sdtstile{}{}\CAPPHI\constant{b}/\variable{x}$.  
\end{majorILnc} 
 
\begin{PROOF}
Suppose $\universal{\variable{x}}\CAPPHI$ is true on $\IntA$. 
Thus, every $\variable{t}$-variant of $\IntA$ makes $\CAPPHI\variable{t}/\variable{x}$ true, where $\variable{t}$ is the first constant not in $\CAPPHI$. 
Consider the $\variable{t}$-variant, $\As{\variable{t}}{}$, that makes the same assignment to $\variable{t}$ that $\IntA$ assigns to $\constant{b}$. Because all $\variable{t}$-variants make $\CAPPHI\variable{t}/\variable{x}$ true, it follows that $\As{\variable{t}}{}$ does too.

The sentences $\CAPPHI\constant{b}/\variable{x}$ and $\CAPPHI\variable{t}/\variable{x}$ are the same, except that the latter substitutes the constant $\variable{t}$ for $\constant{b}$.  So Dragnet condition (1) is met.  The models $\IntA$ and $\As{\variable{t}}{}$ make all the same assignments, except that $\As{\variable{t}}{}$ assigns to $\variable{t}$ what $\IntA$ assigns to $\constant{b}$.  So Dragnet condition (2) is met.

Thus, by Dragnet:
\begin{center}
	$\CAPPHI\constant{b}/\variable{x}$ is true on $\IntA$ iff $\CAPPHI\variable{t}/\variable{x}$ is true on $\As{\variable{t}}{}$.
\end{center}
And because $\CAPPHI\variable{t}/\variable{x}$ is true on $\As{\variable{t}}{}$, it follows that $\IntA$ makes $\CAPPHI\constant{b}/\variable{x}$ true.  Therefore, the entailment holds, regardless of the internal structure of $\CAPPHI$.
\end{PROOF} 

How do you know when the Dragnet Theorem might be useful? 
The general answer is that when you are trying to prove something about truth of a sentence given some information about a closely related sentence, that is, one that differs by some constant substitution(s), then you should think about whether the Dragnet Theorem will help. 
In some cases you will need to select an appropriate variant of a model. 
To apply the theorem you need two sentences and two models that meet the two Dragnet restrictions. Practice will help.

While the Dragnet theorem is helpful, it can be a bit unwieldy.  So, we prove another theorem, the \mention{Free Choice} theorem, which builds on Dragnet and saves us a good deal of trouble.

\begin{THEOREM}{\LnpTC{The Free Choice Theorem} The Free Choice Theorem:}
(i) A \GQL{} sentence of the form $\universal\ALPHA\CAPPHI$ is true on some model $\IntA$ \Iff all $\variable{s}$-variants of $\IntA$ make $\CAPPHI\variable{s}/\ALPHA$ true, where $\variable{s}$ is any constant not in $\CAPPHI$; and (ii) a \GQL{} sentence of the form $\existential\ALPHA\CAPPHI$ is true on some model $\IntA$ \Iff some $\variable{s}$-variant of $\IntA$ makes $\CAPPHI\variable{s}/\ALPHA$ true, where $\variable{s}$ is any constant not in $\CAPPHI$.
\end{THEOREM}

\begin{PROOF}
(i) $(\Rightarrow)$: Assume that a \GQL{} sentence of the form $\universal\ALPHA\CAPPHI$ is true on some model $\IntA$.  By the definition of truth for $\forall$, $\CAPPHI\variable{t}/\ALPHA$ is true on all $\variable{t}$-variants of $\IntA$.  Let $\variable{s}$ be some arbitrary constant not in $\CAPPHI$.  We want to show that $\CAPPHI\variable{s}/\ALPHA$ is true on all $\variable{s}$-variants of $\IntA$.

The sentences $\CAPPHI\variable{t}/\ALPHA$ and $\CAPPHI\variable{s}/\ALPHA$ are exactly the same, except the latter substitutes the constant $\variable{s}$ where the former has $\variable{t}$.  So this pair of sentences satisfies condition (1) of Dragnet.

For any $\variable{t}$-variant of $\IntA$, there is a corresponding $\variable{s}$-variant of $\IntA$ that assigns to $\variable{s}$ what the $\variable{t}$ assigns to $\variable{t}$.  Let's take one such pair of variants, $\As{\variable{t}}{}$ and $\As{\variable{s}}{}$, such that $\As{\variable{t}}{}(\variable{t})=\As{\variable{s}}{}(\variable{s})$.  Does this pair, $\As{\variable{t}}{}$ and $\As{\variable{s}}{}$, satisfy condition (2) of Dragnet?  Sadly, no; they vary on assignments to two constants, $\variable{s}$ and $\variable{t}$.  We'll get around this problem by using an intermediate model.

Let $\IntA^*$ be a \mention{buffer} model, which make all the same assignments as $\As{\variable{t}}{}$, except that what $\As{\variable{t}}{}$ assigns to $\variable{t}$, $\IntA^*$ assigns to $\variable{s}$.  Thus, $\IntA^*$ and $\As{\variable{t}}{}$ satisfy condition (2) of Dragnet.  By Dragnet,

\begin{center}
$\CAPPHI\variable{t}/\ALPHA$ is true on $\As{\variable{t}}{}$ \Iff $\CAPPHI\variable{t}/\ALPHA$ is true on $\IntA^*$.
\end{center}

So $\IntA^*$ makes $\CAPPHI\variable{t}/\ALPHA$ true.  $\IntA^*$ differs from $\As{\variable{s}}{}$ on only one constant assignment: what $\As{\variable{s}}{}$ assigns to $\variable{s}$, $\IntA^*$ assigns to $\variable{t}$.  Thus, $\IntA^*$ and $\As{\variable{s}}{}$ satisfy condition (2) of Dragnet.  By Dragnet,

\begin{center}
	$\CAPPHI\variable{t}/\ALPHA$ is true on $\IntA^*$ \Iff $\CAPPHI\variable{s}/\ALPHA$ is true on $\As{\variable{s}}{}$.
\end{center}

So $\As{\variable{s}}{}$ makes $\CAPPHI\variable{s}/\ALPHA$ true.

This argument holds not just for this pair of variants, $\As{\variable{t}}{}$ and $\As{\variable{s}}{}$.  All $\variable{t}$-variants of $\IntA$ make $\CAPPHI\variable{t}/\ALPHA$ true, and using analogous reasoning, we see that all $\variable{s}$-variants of $\IntA$ make $\CAPPHI\variable{s}/\ALPHA$ true.

We assumed nothing special about $\variable{s}$, except that it's a constant not in $\CAPPHI$.  Therefore, all $\variable{s}$-variants of $\IntA$ make $\CAPPHI\variable{s}/\ALPHA$ true, where $\variable{s}$ is any constant not in $\CAPPHI$.

$(\Leftarrow)$:  Assume that all $\variable{s}$-variants of some model $\IntA$ make $\CAPPHI\variable{s}/\ALPHA$ true, where $\variable{s}$ is any constant not in $\CAPPHI$.  Let constant $\variable{t}$ be the first constant not in $\CAPPHI$.  It follows that all $\variable{t}$-variants of $\IntA$ make $\CAPPHI\variable{t}/\ALPHA$ true.  Therefore, by the definition of truth for $\forall$, $\universal\ALPHA\CAPPHI$ is true on $\IntA$.

(ii) We leave the case with the existential quantifier as an exercise for the reader.
\end{PROOF}

Let's do an example proof that uses the Free Choice theorem.


\begin{majorILnc}{\LnpEC{FreeChoiceExampleOne}}
	Let $\CAPPHI$ and $\CAPPSI$ be formulas whose only free variable is $\variable{x}$.  Prove that $\universal{\variable{x}}\parhorseshoe{\CAPPHI}{\CAPPSI}\sdtstile{}{}\horseshoe{\universal{\variable{x}}\CAPPHI}{\universal{\variable{x}}\CAPPSI}$.  
\end{majorILnc} 
 
\begin{PROOF}
	Let's assume, for reductio, that $\IntA$ is a counterexample to the entailment.  So, $\IntA$ makes $\universal{\variable{x}}\parhorseshoe{\CAPPHI}{\CAPPSI}$ true and $\horseshoe{\universal{\variable{x}}\CAPPHI}{\universal{\variable{x}}\CAPPSI}$ false.
	
	Because $\IntA$ makes $\horseshoe{\universal{\variable{x}}\CAPPHI}{\universal{\variable{x}}\CAPPSI}$ false, it makes $\universal{\variable{x}}\CAPPHI$ true and $\universal{\variable{x}}\CAPPSI$ false (definition of truth, $\HORSESHOE$). 
	It follows that all $\variable{t}$-variants of $\IntA$ make $\CAPPHI\variable{t}/\variable{x}$ true, where $\variable{t}$ is the \emph{some} constant not in $\CAPPHI$ (Free Choice theorem).  In fact, let's stipulate that $\variable{t}$ occurs nowhere in the entire entailment claim, in neither $\CAPPHI$ nor $\CAPPSI$.\footnote{The Free Choice theorem allows us to make this stipulation, and it makes the proof considerably easier.} 
	 And because $\IntA$ makes $\universal{\variable{x}}\CAPPSI$ false, there is some $\variable{t}$-variant of $\IntA$ makes $\CAPPSI\variable{t}/\variable{x}$ false (Free Choice).%\footnote{Free Choice allows us to use the same constant throughout.  You might wonder why we distinguish $\variable{t}_1$ from $\variable{t}_2$.  	It's because we can't assume that they are the same constant.  	For example, what if $\CAPPHI=\Lppp{'''\variable{x}}{\constant{a}}{\constant{b}}$, so that the first constant not in the formula is \mention{$\constant{c}$}; and what if $\CAPPSI=\Dppp{''}{\variable{x}}{\constant{c}}$, so that the first constant not in that formula is \mention{$\constant{a}$}? 	On these substitutions for $\CAPPHI$ and $\CAPPSI$, $\variable{t}_1=\constant{c}$ and $\variable{t}_2=\constant{a}$.  	Cases like this are why we distinguish the term metavariables $\variable{t}_1$ from $\variable{t}_2$ in our argument.} 

	The model $\IntA$ makes the LHS, $\universal{\variable{x}}\parhorseshoe{\CAPPHI}{\CAPPSI}$, true.  
	Hence, all $\variable{t}$-variants of $\IntA$ make $\parhorseshoe{\CAPPHI\variable{t}/\variable{x}}{\CAPPSI\variable{t}/\variable{x}}$ true (Free Choice). %\footnote{We distinguish $\variable{t}_3$ from $\variable{t}_1$ and $\variable{t}_2$ because there are cases in which it must be a different constant.  	Consider the formula substitutions from the last footnote, which are such that $\parhorseshoe{\CAPPHI}{\CAPPSI}=\parhorseshoe{\Lppp{'''\variable{x}}{\constant{a}}{\constant{b}}}{\Dppp{''}{\variable{x}}{\constant{c}}}$. 	The first constant not in this formula is \mention{$\constant{d}$}, so on this substitution $\variable{t}_3=\constant{d}$.  So we should distinguish $\variable{t}_3$ from the other term metavariables.} 
	 Because $\CAPPHI\variable{t}/\variable{x}$ is true on all $\variable{t}$-variants of $\IntA$, it follows that $\CAPPSI\variable{t}/\variable{x}$ is true on all $\variable{t}$-variants of $\IntA$ (definition of truth, $\HORSESHOE$).  But we'd concluded that some $\variable{t}$-variant of $\IntA$ makes $\CAPPSI\variable{t}/\variable{x}$ false.  This is contradictory.
	 
	 Therefore, our initial assumption---that some model is a counterexample to the entailment---is false.  No such model exists.  The entailment holds.	
	  
\end{PROOF}






%%%%%%%%%%%%%%%%%%%%%%%%%%%%%%%%%%%%%%%%%%%%%%%%%%
\section{Exercises}
%%%%%%%%%%%%%%%%%%%%%%%%%%%%%%%%%%%%%%%%%%%%%%%%%%

\notocsubsection{Formulas, Order, and Subformulas}{ex:Formulas, Order, and Subformulas} Which of the following are \emph{formulas}? 
For those which are formulas, what is their order? 
How many subformulas does each have?
\begin{multicols}{2}
\begin{enumerate}
\item {$\universal{\variable{x}}\parhorseshoe{\Hpp{'}{\variable{x}}}{\Gpp{'}{\variable{x}}}$}
\item {$\universal{\variable{x}}\parhorseshoe{\Hpp{'}{\variable{x}}}{\Gpp{''}{\variable{x}}}$}
\item {$\universal{\variable{x}}\parhorseshoe{\Hpp{'}{\variable{x}}}{\Gpp{'_7}{\variable{x}}}$}
\item {$\universal{\variable{x}}\universal{\variable{z}}\parhorseshoe{\Hpp{'}{\variable{x}}}{\Gppp{''}{\variable{x}}{\variable{y}}}$}
\item {$\universal{\variable{x}}\universal{\variable{y}}\parhorseshoe{\Hpp{'}{\variable{x}}}{\Gppp{''}{\variable{x}}{\variable{y}}}$}
\item {$\universal{\variable{x}}\universal{\variable{z}}\parhorseshoe{\Hpp{'}{\constant{a}}}{\Gppp{''}{\variable{x}}{\variable{y}}}$}
\item {$\universal{\variable{x}}\universal{\variable{z}}\parhorseshoe{\Hp{\variable{x}}}{\Gppp{''}{\variable{x}}{\variable{y}}}$}
\item {$\universal{\variable{x}}\universal{\variable{z}}\parhorseshoe{\Wpp{'}{\variable{x}}}{\Gppp{''}{\variable{x}}{\variable{y}}}$}
\item {$\universal{\variable{x}}\universal{\variable{z}}\bparhorseshoe{\Hpp{'}{\variable{x}}}{\Gppp{''}{\variable{x}}{\variable{y}}}$}
\item {$\universal{\variable{x}_9}\universal{\variable{z}}\parhorseshoe{\Hpp{'}{\variable{x}_9}}{\Gppp{''}{\variable{x}}{\variable{y}}}$}
\item {$\negation{\universal{\variable{x}}\universal{\variable{y}}\parhorseshoe{\Hpp{'}{\variable{x}}}{\Gppp{''}{\variable{x}}{\variable{y}}}}$}
\item {$\universal{\constant{b}}\universal{\variable{y}}\parhorseshoe{\Hpp{'}{\constant{b}}}{\Gppp{''}{\variable{x}}{\variable{y}}}$}
\item {$\universal{\variable{x}}\parhorseshoe{\universal{\variable{x}}\Hpp{'}{x}}{\Gppp{''}{\variable{x}}{\variable{y}}}$}
\item {$\universal{\variable{x}}\parhorseshoe{\universal{\variable{z}}\Hpp{'}{x}}{\Gppp{''}{\variable{x}}{\variable{y}}}$}
\item {$\universal{\variable{y}}\parhorseshoe{\universal{\variable{x}}\Hpp{'}{x}}{\universal{\variable{x}}\Gppp{''}{\variable{x}}{\variable{y}}}$}
\end{enumerate}
\end{multicols}

\notocsubsection{Sentences and Order}{ex:Sentences and Order} For each of the following, say whether it is an official sentence, an unofficial sentence, an official formula but not a sentence, an unofficial formula but not a sentence or none of the above. 
If it is a formula or sentence (official or unofficial) say what its order is and how many sub\emph{formulas} it has.
\begin{multicols}{2}
\begin{enumerate}
\item {$\universal{\variable{x}}\universal{\variable{y}}\parhorseshoe{\Hpp{'}{\variable{x}}}{\Gppp{''}{\variable{x}}{\variable{y}}}$}
\item {$\universal{\variable{x}}\existential{\variable{z}}\parhorseshoe{\Hpp{'}{\variable{x}}}{\Gppp{''}{\variable{x}}{\variable{y}}}$}
\item {$\universal{\variable{x}}\universal{\variable{y}}\parconjunction{\Hp{\variable{x}}}{\conjunction{\Gpp{\variable{x}}{\variable{y}}}{\Kp{\variable{y}}}}$}
\item {$\universal{\variable{x}}\universal{\variable{y}}\parhorseshoe{\Hp{\variable{x}}}{\horseshoe{\Gpp{\variable{x}}{\variable{y}}}{\Kp{\variable{y}}}}$}
\item {$\universal{\variable{x}}\universal{\variable{y}}\parconjunction{\negation{\Hp{\variable{x}}}}{\conjunction{\Gpp{\variable{z}}{\variable{y}}}{\Kp{\variable{y}}}}$}
\item {$\universal{\variable{x}}\universal{\variable{y}}\parconjunction{\Hp{\variable{x}}}{\parconjunction{\Gpp{\variable{z}}{\variable{y}}}{\Kp{\variable{y}}}}$}
\item {$\universal{\variable{x}}\universal{\variable{y}}\parconjunction{\Hpp{'}{\variable{x}}}{\parconjunction{\Gppp{''}{\variable{z}}{\variable{y}}}{\Kpp{''}{\variable{y}}}}$}
\item {$\universal{\variable{x}}\universal{\variable{y}}\conjunction{\Hp{\variable{x}}}{\parconjunction{\Gpp{\variable{z}}{\variable{y}}}{\Kp{\variable{y}}}}$}
\item {$\universal{\variable{x}}\existential{\variable{z}}\universal{\variable{y}}\parconjunction{\Hp{\variable{x}}}{\conjunction{\Gpp{\variable{x}}{\variable{y}}}{\Kp{\variable{y}}}}$}
\item {$\universal{\variable{x}}\existential{\variable{y}}\universal{\variable{z}}\parconjunction{\Hp{\variable{x}}}{\conjunction{\universal{\variable{y}}\Gpp{\variable{x}}{\variable{y}}}{\Kp{\variable{y}}}}$}
\end{enumerate}
\end{multicols}

\notocsubsection{Truth in a Model}{ex:Truth in an Interpretation} Give the truth value of each of the following sentences on both of the models found in table \mvref{table:Example Interpretations Exercise}. 
\begin{multicols}{2}
\begin{enumerate}
\item {$\universal{\variable{x}}\universal{\variable{y}}\cparhorseshoe{\parconjunction{\Ap{\variable{x}}}{\Bp{\variable{y}}}}{\App{\variable{x}}{\variable{y}}}$}
\item {$\universal{\variable{x}}\universal{\variable{y}}\cparhorseshoe{\parconjunction{\Cp{\variable{x}}}{\Dp{\variable{y}}}}{\Dpp{\variable{x}}{\variable{y}}}$}
\item {$\universal{\variable{x}}\cparhorseshoe{\parconjunction{\Cp{\variable{x}}}{\Ep{\variable{x}}}}{\Cpp{\variable{x}}{\constant{a}}}$}
\item {$\universal{\variable{x}}\universal{\variable{y}}\cparhorseshoe{\Cpp{\variable{x}}{\variable{y}}}{\App{\variable{x}}{\variable{y}}}$}
\item {$\horseshoe{\universal{\variable{x}}\Cp{\variable{x}}}{\universal{\variable{y}}\Dp{\variable{y}}}$}
\item {$\universal{\variable{z}}\universal{\variable{w}}\cparhorseshoe{\parconjunction{\Gp{\variable{z}}}{\Gp{\variable{w}}}}{\negation{\Gpp{\variable{z}}{\variable{w}}}}$}
\item {$\universal{\variable{z}}\universal{\variable{w}}\universal{\variable{x}}\cparhorseshoe{\Cppp{\variable{x}}{\variable{z}}{\variable{w}}}{\bpartriplebar{\Cp{\variable{x}}}{\Cp{\variable{w}}}}$}
\item {$\universal{\variable{x}}\cparhorseshoe{\Ap{\variable{x}}}{\existential{\variable{y}}\parconjunction{\Cp{\variable{y}}}{\Bpp{\variable{y}}{\variable{x}}}}$}
\item {$\existential{\variable{y}}\existential{\variable{x}}\parconjunction{\Cpp{\variable{x}}{\variable{y}}}{\Dpp{\variable{x}}{\variable{y}}}$}
\item {$\existential{\variable{y}}\existential{\variable{x}}\parconjunction{\Epp{\variable{x}}{\variable{y}}}{\Gpp{\variable{x}}{\variable{y}}}$}
\item {$\universal{\variable{x}}\cparhorseshoe{\Gp{\variable{x}}}{\universal{\variable{w}}\bparhorseshoe{\App{\variable{x}}{\variable{w}}}{\pardisjunction{\Ap{\variable{w}}}{\Cp{\variable{w}}}}}$}
\item {$\existential{\variable{x}}\cparhorseshoe{\Cp{\variable{x}}}{\universal{\variable{y}}\Cp{\variable{y}}}$}
\item {$\universal{\variable{z}}\universal{\variable{w}}\universal{\variable{x}}\cparhorseshoe{\Dppp{\variable{x}}{\variable{z}}{\variable{w}}}{\App{\variable{x}}{\variable{w}}}$}
\item {$\universal{\variable{y}}\bparhorseshoe{\Ap{\variable{y}}}{\existential{\variable{x}}\parconjunction{\Ep{\variable{x}}}{\App{\variable{x}}{\variable{y}}}}$}
\item {$\universal{\variable{x}}\universal{\variable{y}}\parhorseshoe{\Gpp{\variable{x}}{\variable{y}}}{\Gpp{\variable{y}}{\variable{x}}}$}
\end{enumerate}
\end{multicols}

\begin{longtable}[c]{ l l l l } %p{2.2in} p{2in}
	\toprule
	&\textbf{Symbol} & \multicolumn{2}{c}{\textbf{Model}} \\ \cmidrule(l){3-4}
	& & \textbf{Pos Int} & \textbf{States} \\
	\midrule 
	\endfirsthead
	\multicolumn{4}{c}{\emph{Continued from Previous Page}}\\
	\toprule
	&\textbf{Symbol} & \multicolumn{2}{c}{\textbf{Model}} \\ \cmidrule(l){3-4}
	& & \textbf{Pos Int} & \textbf{States} \\
	\midrule 
	\endhead
	\bottomrule
	\caption{Example Models}\\[-.15in]
	\multicolumn{4}{c}{\emph{Continued next Page}}\\
	\endfoot
	\bottomrule
	\caption{Example Models}\\%
	\endlastfoot%
	\label{table:Example Interpretations Exercise}%
	%\begin{tabular}{ l l l l } %p{2in} p{2in} %\begin{tabular}{ p{1in} l l } %p{2.2in} p{2in}
	%\toprule
	%&\textbf{Symbol} & \multicolumn{2}{c}{\textbf{Interpretation}} \\ \cmidrule(l){3-4}
	%& & \textbf{Pos Int} & \textbf{States} \\
	%\midrule 
	{Universe:} & & The set of positive integers & The set of US states (2015) \\ \addlinespace[.25cm]
	{Sent. Let.:}& A&$\True$&$\False$\\
	& B&$\True$&$\False$\\
	& C&$\False$&$\True$\\
	& D&$\True$&$\False$\\
	& E&$\True$&$\False$\\
	& G&$\False$&$\True$\\ \addlinespace[.25cm]
	{Constants:}&$\constant{a}$&1&Louisiana\\
	&$\constant{b}$&9&Maine\\
	&$\constant{c}$&72&Georgia\\
	&$\constant{d}$&3&Nebraska\\
	&$\constant{e}$&1&New Mexico\\
	&$\constant{f}$&2&Texas\\ \addlinespace[.25cm]
	{1-place:}&$\Ap{'}$&all pos int&Midwestern\\
	&$\Bp{'}$&empty set&name with $>5$ letters\\
	&$\Cp{'}$&even&Coastal\\
	&$\Dp{'}$&odd&one of original 13\\
	&$\Ep{'}$&prime&\{Ohio\}\\
	&$\Gp{'}$&multiple of 7&\{Ohio, Alabama\}\\ \addlinespace[.25cm]
	{2-place:}&$\Ap{''}$&first $>$ second&share a border\\
	&$\Bp{''}$&are equal&first is north of second\\
	&$\Cp{''}$&first = 2 times second&first $>$ second (area)\\
	&$\Dp{''}$&sum of them equals 7&first $>$ second (population)\\
	&$\Ep{''}$&first $<$ second&first is west of second\\
	&$\Gp{''}$&are relatively prime&both coastal, or neither\\ \addlinespace[.25cm]
	{3-place:}&$\Ap{'''}$&all equal&all same population\\
	&$\Bp{'''}$&first $<$ second $<$ third&first is north of others\\
	&$\Cp{'''}$&all odd or all even&first $>$ second $>$ third (area)\\
	&$\Dp{'''}$&first + second = third&first + second $>$ third (area)\\
	&$\Ep{'''}$&first $\times$ second = third&first is west of the others\\
	&$\Gp{'''}$&are all relatively prime& at least two coastal \\
	%\bottomrule
\end{longtable}

\notocsubsection{Quantificational Truth Problems}{ex:More Quantificational Truth Problems} For each sentence below, say whether or not it's a quantificational truth. 
If so, prove it. 
If not, show it by giving a model $\IntA$ that makes it false.
\begin{multicols}{2} 
\begin{enumerate}
\item {$\horseshoe{\universal{\variable{x}}\universal{\variable{y}}\Hpp{\variable{x}}{\variable{y}}}{\universal{\variable{y}}\universal{\variable{x}}\Hpp{\variable{x}}{\variable{y}}}$}
\item {$\horseshoe{\existential{\variable{x}}\existential{\variable{y}}\Hpp{\variable{x}}{\variable{y}}}{\existential{\variable{y}}\existential{\variable{x}}\Hpp{\variable{x}}{\variable{y}}}$}
\item {$\horseshoe{\universal{\variable{x}}\existential{\variable{y}}\Hpp{\variable{x}}{\variable{y}}}{\existential{\variable{y}}\universal{\variable{x}}\Hpp{\variable{x}}{\variable{y}}}$}
\item {$\horseshoe{\existential{\variable{y}}\universal{\variable{x}}\Hpp{\variable{x}}{\variable{y}}}{\universal{\variable{x}}\existential{\variable{y}}\Hpp{\variable{x}}{\variable{y}}}$}
\item {$\universal{\variable{x}}\bparhorseshoe{\Ap{\variable{x}}}{\existential{\variable{y}}\parconjunction{\Hpp{\variable{x}}{\variable{y}}}{\Bp{\variable{y}}}}$}
\item {$\existential{\variable{y}}\bparconjunction{\Ap{\variable{y}}}{\universal{\variable{z}}\parhorseshoe{\Bp{\variable{z}}}{\Hpp{\variable{y}}{\variable{z}}}}$}
\end{enumerate}
\end{multicols}
\begin{enumerate}[start=7]
\item {$\universal{\variable{x}}\bparconjunction{\existential{\variable{y}}\Hpp{\variable{x}}{\variable{y}}}{\conjunction{\negation{\Hpp{\variable{x}}{\variable{x}}}}{\universal{\variable{y}}\universal{z}\cparhorseshoe{\parconjunction{\Hpp{\variable{x}}{\variable{y}}}{\Hpp{\variable{y}}{\variable{z}}}}{\Hpp{\variable{x}}{\variable{z}}}}}$}
\end{enumerate}

\notocsubsection{Preliminary Dragnet Practice Problems}{ex:Preliminary Dragnet Practice Problems} 
\begin{enumerate}
\item If $\CAPPHI$ is $\horseshoe{\universal{\variable{z_{\integer{3}}}}\parconjunction{\Gpp{\variable{z_{\integer{1}}}}{\variable{z_{\integer{3}}}}}{\existential{\variable{x}}\Bppp{\variable{x}}{\variable{z_{\integer{3}}}}{\variable{y_{\integer{2}}}}}}{\parhorseshoe{\Al}{\universal{\variable{y_{\integer{2}}}}\Dpp{\variable{z_{\integer{3}}}}{\variable{y_{\integer{2}}}}}}$, what is
\begin{enumerate}
\item $\CAPPHI\constant{a}/\variable{z_{\integer{3}}}$
\item $\CAPPHI\constant{a}/\variable{y_{\integer{2}}}$
\item $\CAPPHI\variable{z_{\integer{3}}}/\variable{x}$
\item $\CAPPHI\constant{a}/\variable{x}$
\end{enumerate} 
\item If $\CAPPHI\variable{x}/\variable{y}$ is $\Dpp{\variable{x}}{\variable{x}}$, can you determine what $\CAPPHI$ is? If so, what is it? If not, why not?
\end{enumerate}

\notocsubsection{Dragnet Practice}{ex:Dragnet Practice} (i) For each of the following statements, work out whether it is true or false. 
(ii) For each statement, determine whether Dragnet would be required to prove it if it is true. 
(iii) For which ones is Dragnet required to disprove it if it is false? Note that each of $\CAPTHETA$, $\CAPPSI$, and $\CAPPHI$ should be understood as standing for complex formulas of \GQL{} whose only free variables are $\variable{x}$, and which do not contain the variables $\variable{y}$, $\variable{z}$ or $\variable{w}$. 
%Recall, for example, that $\CAPPSI{\variable{w}/\variable{x}}$ is the formula $\CAPPSI$ with the variable $\variable{w}$ replacing all occurrences of $\variable{x}$ (see def. \pmvref{MathEnglishVariableSub}).

%For all $\CAPTHETA$, $\CAPPSI$ and $\CAPPHI$:
\begin{enumerate}
\item {$\universal{\variable{w}}\parconjunction{\CAPPHI{\variable{w}/\variable{x}}}{\CAPPSI{\variable{w}/\variable{x}}}\sdtstile{}{}\horseshoe{\universal{\variable{x}}\CAPPHI}{\universal{\variable{x}}\CAPPSI}$}
\item {$\horseshoe{\universal{\variable{x}}\CAPPHI}{\universal{\variable{x}}\CAPPSI}\sdtstile{}{}\horseshoe{\universal{\variable{y}}\CAPPHI{\variable{y}/\variable{x}}}{\universal{\variable{z}}\CAPPSI{\variable{z}/\variable{x}}}$}
\item {$\horseshoe{\universal{\variable{x}}\CAPPHI}{\universal{\variable{x}}\CAPPSI}\sdtstile{}{}\existential{\variable{x}}\parconjunction{\CAPPHI}{\CAPPSI}$}
\item {$\universal{\variable{x}}\parhorseshoe{\CAPPHI}{\CAPPSI}\sdtstile{}{}\horseshoe{\universal{\variable{y}}\CAPPHI{\variable{y}/\variable{x}}}{\universal{\variable{z}}\CAPPSI{\variable{z}/\variable{x}}}$}
\item {$\horseshoe{\universal{\variable{y}}\CAPPHI{\variable{y}/\variable{x}}}{\universal{\variable{z}}\CAPPSI{\variable{z}/\variable{x}}}\sdtstile{}{}\universal{\variable{x}}\parhorseshoe{\CAPPHI}{\CAPPSI}$}
\item {$\horseshoe{\universal{\variable{y}}\CAPPHI{\variable{y}/\variable{x}}}{\universal{\variable{z}}\CAPPSI{\variable{z}/\variable{x}}}\text{, }\existential{\variable{x}}\parconjunction{\CAPPHI}{\CAPPSI}\sdtstile{}{}\universal{\variable{x}}\parhorseshoe{\CAPPHI}{\CAPPSI}$}
\item {$\text{If }\sdtstile{}{}\universal{\variable{x}}\parhorseshoe{\CAPPHI}{\CAPPSI}\text{, then }\sdtstile{}{}\universal{\variable{x}}\cparhorseshoe{\parconjunction{\CAPPHI}{\CAPTHETA}}{\parconjunction{\CAPPSI}{\CAPTHETA}}$}
\item {$\existential{\variable{y}}\parconjunction{\CAPPHI{\variable{y}/\variable{x}}}{\negation{\CAPPSI{\variable{y}/\variable{x}}}}\sdtstile{}{}\universal{\variable{y}}\parhorseshoe{\CAPPHI{\variable{y}/\variable{x}}}{\negation{\CAPPSI{\variable{y}/\variable{x}}}}$}
\item {$\negation{\universal{\variable{x}}\CAPPHI}\sdtstile{}{}\existential{\variable{x}}\negation{\CAPPHI}$}
\item {$\sdtstile{}{}\disjunction{\universal{\variable{x}}\parhorseshoe{\CAPPHI}{\CAPPSI}}{\universal{\variable{y}}\parhorseshoe{\CAPPHI{\variable{y}/\variable{x}}}{\negation{\CAPPSI{\variable{y}/\variable{x}}}}}$}
\end{enumerate}

%\theendnotes


%%%%%%%%%%%%%%%%%%%%%%%%%%%%%%%%%%%%%%%%%%%%%%%%%%
\chapter{Translations}\label{Translations}
%%%%%%%%%%%%%%%%%%%%%%%%%%%%%%%%%%%%%%%%%%%%%%%%%%
% \AddToShipoutPicture*{\BackgroundPicC}

%%%%%%%%%%%%%%%%%%%%%%%%%%%%%%%%%%%%%%%%%%%%%%%%%%
\section{SL Applications}\label{SLApplications}
%%%%%%%%%%%%%%%%%%%%%%%%%%%%%%%%%%%%%%%%%%%%%%%%%%


Sentences in English have meanings.  Declarative sentences say something about the world and in virtue of that they have truth conditions: ways the world must be for them to be true.  By contrast, the sentences of \GSL{} (and \GQL{}) are, considered in isolation, meaningless strings of symbols. To the extent that we can ascribe meaning to \GSL{} (and \GQL{}) sentences at all, that meaning comes from a model.  The way that we specify models in Chapter \ref{sententiallogic} does not distinguish between different sentences with the same truth value, because for the purposes of assessing logical truth in \GSL{}, truth value is all that matters.  But if we want to translate English into \GSL{} we must specify the models less directly: by associating English sentences with the sentence letters.  If we are careful in our selection of English sentences we may assign each of them unambiguous truth values.  We then take these values as the assignments that some model $\IntA$ makes to the corresponding sentence letters.  We assign truth values to the sentences according to what we know about reality, in which case $\IntA$ would be a \mention{model} of reality, or at least that part of reality described by those English sentences.  Alternatively, we may assign truth values to English sentences so that they do not match what we know about reality, in which case $\IntA$ would be a model of a non-actual state of affairs.  Finally, we could consider sentences whose values we don't know and assign them values, in which case $\IntA$ would be a model of a hypothetical state of affairs.

Ideally, the English sentences in question should be unambiguous, not vague, and will not have a truth value that changes over time. At the least, we should use sentences that are unambiguous in the appropriate context, specific enough to have determinate truth values, and whose truth values will not change during the discussion.  One source of sentences with these characteristics is mathematics.  For other sentences, we can usually add enough detail about context to meet our criteria.  For example, we might casually say that \mention{Texas is a US state} is a true sentence of English.  The context would normally indicate that we meant Texas is a US state at the time the sentence is uttered.  If we consider the sentence apart from some context of utterance matters aren't so clear.  The sentence was false until 1845 and has been true since 1865, so its truth value has varied over time.\footnote{The Texas government declared the state in secession from the US in 1861, and did not relinquish this claim until the Union defeated the Confederacy in 1865.}  Officially we prefer a less ambiguous sentence, such as \mention{Texas was a US state at noon CDT on May 1, 1983.}  Unofficially, we often make background assumptions about what is relevant, who is being discussed, the time of assessment, and so on.  For similar reasons, \mention{Trump is elected President} has an obvious and natural interpretation, one that's distinct from \mention{Joan Trump is elected President of the Montrose Dog Walkers Association.}

There is a controversy in philosophy of logic and philosophy of language over which objects are the most fundamental bearers of truth value: sentence types, sentence tokens, propositions, judgments, or statements.  We regard sentences, whether types or tokens, as better objects to work with than the alternatives.  We have a better understanding of what a sentence is than what propositions, judgments, or statements are.  That is why in Chapter \ref{sententiallogic} we use the name \mention{Sentential Logic} rather than \mention{Propositional Logic}, which is used by some logic texts.  For our purposes we can be agnostic as to whether it's sentence types or tokens because, given our restrictions, all tokens of a type will have the same truth value in our specific translation context. 

We mentioned associating English sentences with \GSL{} sentence letters.  To do this, we give a \idf{translation key} that maps sentence letters from \GSL{} to sentences of English.  A translation key is different from a model, but a good translation key will help fix a specific model for the relevant sentences of \GSL{}.  We just need to supply the truth values of the English sentences of the key.  Once we've mapped sentences of English to sentence letters we can use the key to translate more complex English sentences into \GSL{}.  We may think of complex sentences of English as being composed of (i) simpler English sentences and (ii) logical connectives of English.

But how do we know when a sentence of English is sufficiently simple to be assigned to a sentence letter?  The answer is that in putting together a key we should capture as much of the relevant logical structure of the English as is practical and useful.  All else being equal, translation keys that hide important logical structure in a sentence letter are deficient.  For example, the following sentence is not ordinarily a good candidate for sentence letter assignment:

\begin{smenumerate}
	\item Mares eat oats and does eat oats and little lambs eat ivy.
\end{smenumerate}

\noindent{}The word \mention{and} plays a truth-functional role in this sentence, analogous to the role that \mention{$\WEDGE$} plays in \GSL{}.  A better translation key would assign each of \mention{Mares eat oats}, \mention{Does eat oats}, and \mention{Little lambs eat ivy} its own sentence letter, so that the original sentence of English could be expressed as a conjunction in \GSL{}.

More generally, we want to translate the truth-functional connectives of English sentences with appropriate logical connectives from \GSL{}. 
But before considering other examples, we need to say a little more about connectives.\index{logical connectives} 
In Chapter \ref{sententiallogic} (in definition \pmvref{Basic Symbols of GSL}) we simply listed what we called the \sq{logical connectives} of \GSL{}.  We've seen the role they play in building the sentences of \GSL{} and in fixing the truth value of those sentences in a model. 
But we haven't said anything explicit about what connectives are. 
The basic idea of a sentential \idf{connective}---whether a \sq{logical} one in some formal language or an expression in a natural language like English---is a bit of a language that can be used to combine, or connect, one or more sentences of the language into a new sentence.\footnote{Other connectives play a subtler role, as with the quantifiers of QL.  In QL quantifiers can be used to combine predicates to make a more complex predicate.} 
We say that one connective is within the \idf{scope} of a second connective iff the first connective is within a sentence directly connected with the second connective.  The \idf{main connective} is the one connective of the sentence that's not within the scope of any other connectives. 

A connective is \niidf{truth-functional}\index{connective!truth-functional} iff the truth value of every new sentence formed by that connective depends solely on the truth value of the constituent sentences. 
(If we're talking about a connective in a formal language like \GSL{}, we mean the truth value of a sentence \emph{in a model}.)
Not all connectives of English are truth-functional. In particular, modal connectives---such as \mention{necessarily} and \mention{possibly}, among others---defy truth-functional characterization.  For example, the sentence

\begin{menumerate}
	\item Necessarily, $2+2=4$.
\end{menumerate}

\noindent{}seems plausibly true.  If we replace \mention{$2+2=4$} with another true claim, however, the result may not be so plausible:

\begin{menumerate}
	\item Necessarily, the current year is 2016.
\end{menumerate}

\noindent{}Both \mention{$2+2=4$} and \mention{The current year is 2016} are true.  The above examples make clear that the truth of a sentence with a modal main connective depends on more than the truth value of the part of the sentence governed by the modal connective.  

On the other hand, our definition of truth in a model (def. \pmvref{True on a GSL interpretation}) makes it apparent that the five logical connectives of \GSL{} are all truth-functional.  This disparity means that \GSL{} is not capable of capturing all of the logical structure of English; the best it can do is to approximate certain features of English.  Some formal languages have the resources to capture the logical structure of modal claims.  We consider one such language in Chapter \ref{furtherdirections}.  We hasten to add, however, that \GSL{} has sufficient resources to render many English sentences without significantly distorting their basic meaning.

\subsection{Connectives and Translations}\label{GSLConnectives and Trans}

A logical connective of \GSL{} is a suitable translation of a (truth-functional) connective of English iff they express the same truth function (see section \pmvref{Truth Functions Truth Tables and Boolean Operators})---that is, iff the new sentences formed by those connectives share the same truth value whenever the constituent sentences that made them up share the same truth values.  We've already mentioned one such pair of corresponding connectives: \mention{and} and \mention{$\WEDGE$}.  We know that the English sentence

\begin{menumerate}
	\item\label{GSLTransConjunction1} The sun is shining \emph{and} the birds are chirping.
\end{menumerate}

\noindent{}is true \Iff the following two sentences are also true:

\begin{menumerate}
	\item The sun is shining.
	\item The birds are chirping.
\end{menumerate}

\noindent{}The same holds of other conjunctions in English.  The \GSL{} connective \mention{$\WEDGE$} is a good translation of \mention{and} most of the time.  Sometimes the structure of English conjunctions is more hidden, as in the following example:

\begin{menumerate}
	\item\label{GSLTransConjunction2} Nathanael Green \emph{and} `Light Horse Harry' Lee were officers in the Continental Army.
\end{menumerate}

\noindent{}We see here the word \mention{and}, but a translation key should not break the sentence at that word and then give the \CAPS{lhs}---\mention{Nathanael Green}---its own sentence letter.  The \CAPS{lhs} is not a declarative sentence.  It's instead clear that (\ref{GSLTransConjunction2}) expresses the conjunction of the following two sentences:

\begin{menumerate}
	\item Nathanael Green was an officer in the Continental Army.
	\item `Light Horse Harry' Lee was an officer in the Continental Army.
\end{menumerate}

\noindent{}It is thus appropriate to map each of these latter sentences to a sentence letter---say, $\Nl$ and $\Ll$, respectively---and express \mention{Nathanael Green and `Light Horse Harry' Lee were officers in the Continental Army} as a conjunction of these sentence letters---e.g., $\parconjunction{\Nl}{\Ll}$.  Other words of English also play the role of conjunction.  Consider the sentence:

\begin{menumerate}
	\item\label{GSLTransConjunction3} Mary wants to drive her mother's car, \emph{but} she is \emph{not} old enough.
\end{menumerate}

\noindent{}Here the word \mention{but} connects two independent claims.  The sentence as a whole is true \Iff the \CAPS{lhs} and the \CAPS{rhs} are each true.  So \mention{but} also plays the same truth-functional role as \mention{$\WEDGE$} in \GSL{}.   The word \mention{not} is another connective.  As you can probably guess, it corresponds to the \mention{$\NEGATION$} of \GSL{}.  Notice that to make sense of the claim on the \CAPS{rhs}, context makes clear that \mention{she} is Mary, and that mention of her being not \mention{old enough} is about the fact that she is not old enough \emph{to drive her mother's car}. English is, in this respect, more efficient than \GSL{}. So, to translate this sentence into \GSL{} we'll need to use something like the following:

	\begin{description}[itemsep=0em]
		\item[Translation Key:] \hfill{} 
		\begin{description}[itemsep=0em]
			\item $\Al$: Mary wants to drive her mother's car.
			\item $\Ol$: Mary is old enough to drive her mother's car.
		\end{description} 
	\end{description}

\noindent{}We took out the \mention{not} because we can represent that with the \mention{$\NEGATION$} So now we may render the original sentence in \GSL{} as $\parconjunction{\Al}{\negation{\Ol}}$.  This is indistinguishable from the result that we would give if we were instead translating

\begin{menumerate}
	\item Mary wants to drive her mother's car \emph{and} she is not old enough to drive her mother's car.
\end{menumerate}

\noindent{}We typically say \mention{$\PHI$ but $\negation{\THETA}$} when we think \mention{$\PHI$ and $\negation{\THETA}$} is true and we think that our listeners will expect \mention{$\THETA$} when they hear \mention{$\PHI$}.  Often the latter conjunct in a \mention{but} sentence has an implicit or explicit negation, as in sentence (\ref{GSLTransConjunction3}); though occasionally the negation may be on the left, as in \mention{John didn't study, but he passed the exam}.  In other cases, the latter conjunct has no negation but draws a contrast or expresses something that the speaker thinks will defy the expectations of the listeners, given the earlier conjunct(s).  This difference from the word \mention{and} cannot be captured in \GSL{}---it will be lost in translation.  Remember that these \GSL{} translations are approximations of English, and as such they can't always preserve all of the original meaning.

The word \mention{and} can do more than serve as a conjunction.  Sometimes the order of the conjuncts makes a difference in how we interpret meaning. Compare the following two sentences:

\begin{menumerate}
	\item\label{GSLTransConjunction4} John took off his clothes and went to bed.
\end{menumerate}
\begin{menumerate}
	\item\label{GSLTransConjunction5} John went to bed and took off his clothes.
\end{menumerate}

The order of the conjuncts suggests that John performed these actions in a different order.  Yet if we were to translate (\ref{GSLTransConjunction4}) and (\ref{GSLTransConjunction5}) into \GSL{}, the resulting sentences will be truth-functionally equivalent.  We think this equivalence is defensible, because (\ref{GSLTransConjunction4}) and (\ref{GSLTransConjunction5}) don't literally describe the order of events; we tend to read order into them because it's usual in conversation to recount events in chronological order.  Notice that there is nothing contradictory about the following sentence:

\begin{menumerate}
	\item John went to bed and took off his clothes, but I don't know in which order.
\end{menumerate}

\begin{table}
	\renewcommand{\arraystretch}{1.5}%
	\begin{center}
		\begin{tabular}{ l l } %p{2.2in} p{2in}
			\toprule
			\textbf{English} & \textbf{\GSL{}} \\ 
			\midrule
			$\PHI$ and $\THETA$ & $\conjunction{\CAPPHI}{\CAPTHETA}$ \\
			both $\PHI$ and $\THETA$ & $\conjunction{\CAPPHI}{\CAPTHETA}$ \\
			$\PHI$, but $\THETA$ & $\conjunction{\CAPPHI}{\CAPTHETA}$ \\
			$\PHI$, but not $\THETA$ & $\conjunction{\CAPPHI}{\negation{\CAPTHETA}}$ \\
			not $\PHI$, but $\THETA$ & $\conjunction{\negation{\CAPPHI}}{\CAPTHETA}$ \\
			\bottomrule
		\end{tabular} 
		\caption{Translations for Common English Conjunctions}
		\label{TransTableD} 
	\end{center}
\end{table}

Another connective of English is the word \mention{or}:

\begin{menumerate}
	\item \emph{Either} the tigers will get us \emph{or} the lions will.
\end{menumerate}

\noindent{}Here the word \mention{or} connects two claims.  The sentence as a whole is true \Iff a claim on either side of the \mention{or} is true.  This is analogous to the truth-function of the \mention{$\VEE$} connective in \GSL{}, and so that's how we'll translate it.

	\begin{description}[itemsep=0em]
		\item[Translation Key:] \hfill{} 
		\begin{description}[itemsep=0em]
			\item $\Il$: The tigers will get us.
			\item $\Ll$: The lions will get us.
		\end{description} 
	\end{description}

\noindent{}The resulting translation: $\pardisjunction{\Il}{\Ll}$.  We again use context to fill out the sentence \mention{the lions will [get us]}.  

It might not seem quite right to translate \mention{or} as \mention{$\VEE$}. 
 The potential problem is that in \GSL{} $\pardisjunction{\CAPPHI}{\CAPTHETA}$ is true (on a model) when both $\CAPPHI$ and $\CAPTHETA$ are true (on that model). 
 But sometimes in English when we say \mention{$\PHI$ or $\THETA$} we mean that one or the other of $\PHI$ and $\THETA$ is true---but not both. 
 We say that a disjunction which allows for both disjuncts to be true is \niidf{inclusive}\index{inclusive disjunction}, while one that doesn't allow for both disjuncts to be true is \niidf{exclusive}.\index{exclusive disjunction}
 Does this mean that \mention{$\PHI$ or $\THETA$} is exclusive, and hence shouldn't be translated by the inclusive $\pardisjunction{\CAPPHI}{\CAPTHETA}$? 
 
This is a complicated question that we cannot fully answer here. 
We believe, however, that \mention{or} is inclusive in English, and that it is best translated as \mention{$\VEE$}.  Only in certain conversational contexts is the exclusive disjunction clearly intended.  We will motivate our stance with a few brief considerations.\footnote{This discussion borrows from Smith \citeyear[117--22]{Smith2012}.}
 	
One case in which \mention{or} sounds exclusive involves disjuncts that cannot possibly both be true. 
 	For example, I might say \q{Mia took the highway or a plane}. 
 	If I'm referring to two possible routes for the same journey, clearly it's just not physically possible for Mia to have taken both the highway and a plane. 
 	But it doesn't follow from the fact that it's impossible for both disjuncts to be true that the utterance of the sentence itself says, or has the meaning that, both disjuncts cannot be true. 
 	For cases in which it's not possible for both disjuncts to be true it's tempting to assimilate that fact into the meaning itself. 
 	But that leap is unjustified. 
 	Someone could, after all, understand \q{Mia took the highway or a plane} and agree that it's true, but also think that both disjuncts are true.  Perhaps Mia's journey was more disjointed and involved both.  To infer that this disjunction is exclusive requires more information than the sentence itself conveys.
 	
 	\begin{table}
 		\renewcommand{\arraystretch}{1.5}%
 		\begin{center}
 			\begin{tabular}{ l l } %p{2.2in} p{2in}
 				\toprule
 				\textbf{English} & \textbf{\GSL{}} \\ 
 				\midrule
 				$\PHI$ or $\THETA$ & $\disjunction{\CAPPHI}{\CAPTHETA}$ \\
 				either $\PHI$ or $\THETA$ & $\disjunction{\CAPPHI}{\CAPTHETA}$ \\
 				$\PHI$ or $\THETA$, but not both & $\conjunction{\pardisjunction{\CAPPHI}{\CAPTHETA}}{\negation{\parconjunction{\CAPPHI}{\CAPTHETA}}}$ \\
 				\bottomrule
 			\end{tabular} 
 			\caption{Translations for Common English Disjunctions}
 			\label{TransTableC} 
 		\end{center}
 	\end{table}
 	
  	There are two kinds of cases in which it's more plausible that \mention{or} is exclusive.  
 	The first case involves commands or rules. 
 	If you are at a fancy restaurant and a waiter says, \q{You may have the soup or a salad}, it's typically understood that you may have one or the other, but not both. The second case involves elliptical clauses. 
 	A disjunctive phrase \mention{$\PHI$ or $\THETA$} can be elliptical for the longer \mention{$\PHI$ or $\THETA$, but not both}. 
 	If so, then the intended message is exclusive. 
 	But this doesn't count as a case in which \mention{or} by itself expresses exclusive disjunction. 
 	It's a case in which \mention{or} plus the modifying phrase \mention{but not both} together express exclusive disjunction---it's just that the latter phrase \mention{but not both} is tacit. 
 	Special cases like these, which are rare but salient, might lead you to think that disjunctions in English are sometimes exclusive.  However, in these cases contextual clues give such disjunctions their exclusive character, independently of the literal meaning of the relevant sentences.  We know as a matter of cultural familiarity that restaurants don't offer \emph{both} soup \emph{and} salad unless we pay extra.  When intergalactic visitors discover Earth and visit our restaurants, we recommend that waiters make the exclusivity of the disjunction explicit.  (Otherwise there could be an interstellar incident!)  They can even use \GSL{} to do so:
 	
 \begin{menumerate}
 	\item The visiting intergalactic visitor may have the soup \emph{or} a salad, \emph{but} \emph{not} both.
 \end{menumerate}
 	
\noindent{}This sentence has several truth-functional connectives, so we must be careful to understand which connectives govern which.  We recognize \mention{or}, \mention{but}, and \mention{not}.  In this case the \mention{but} governs everything else, so the whole sentence is a conjunction.  The visitor may have the soup or a salad, \emph{and} he may not have both the soup and a salad.  (As we fill out the rest of the compressed meaning of the sentence we find another connective.)  On to the translation:

	\begin{description}[itemsep=0em]
		\item[Translation Key:] \hfill{} 
		\begin{description}[itemsep=0em]
			\item $\Pl$: The visiting intergalactic visitor may have the soup.
			\item $\Dl$: The visiting intergalactic visitor may have a salad.
		\end{description} 
	\end{description}

\noindent{}First we translate the \CAPS{lhs} of the \mention{but}: $\pardisjunction{\Pl}{\Dl}$.  If we translate the \CAPS{rhs} without the \mention{not}---i.e., \mention{the intergalactic visitor may have both soup and a salad}---we get: $\parconjunction{\Pl}{\Dl}$.  We negate the result to account for the \mention{not}: $\negation{\parconjunction{\Pl}{\Dl}}$.  Putting it all together, the resulting translation is: $\parconjunction{\pardisjunction{\Pl}{\Dl}}{\negation{\parconjunction{\Pl}{\Dl}}}$.

Another connective of English is the \emph{conditional}.  Conditionals are expressible in many ways, but perhaps the most distinctive way is \mention{if $\ldots$ then $\ldots$.}  For example,

\begin{menumerate}
	\item If George Washington crosses the Delaware River then the Hessians will be defeated.
\end{menumerate}

\noindent{}Is the conditional of English a truth-functional connective?  This is a hotly debated topic that has yet to be resolved.  SL has only truth-functional connectives, so if the conditional is not truth-functional then we cannot fully capture the meaning in SL.  For our purposes, we can show hat $\HORSESHOE$ is the best model of the conditional and as we discuss translations we can see how well the model fits English.

\begin{table}
	\renewcommand{\arraystretch}{1.5}%
	\begin{center}
		\begin{tabular}{ l l } %p{2.2in} p{2in}
			\toprule
			\textbf{English} & \textbf{\GSL{}} \\ 
			\midrule
			if $\PHI$, then $\THETA$ & $\horseshoe{\CAPPHI}{\CAPTHETA}$ \\
			$\PHI$ only if $\THETA$ & $\horseshoe{\CAPPHI}{\CAPTHETA}$ \\
			$\PHI$ if $\THETA$ & $\horseshoe{\CAPTHETA}{\CAPPHI}$ \\
			$\PHI$ provided that $\THETA$ & $\horseshoe{\CAPTHETA}{\CAPPHI}$ \\
			provided $\PHI$, $\THETA$ & $\horseshoe{\CAPPHI}{\CAPTHETA}$ \\
			$\PHI$ assuming that $\THETA$ & $\horseshoe{\CAPTHETA}{\CAPPHI}$ \\
			assuming $\PHI$, $\THETA$ & $\horseshoe{\CAPPHI}{\CAPTHETA}$ \\
			for $\PHI$, it's necessary that $\THETA$ & $\horseshoe{\CAPPHI}{\CAPTHETA}$ \\
			for $\PHI$, it's sufficient that $\THETA$ & $\horseshoe{\CAPTHETA}{\CAPPHI}$ \\
			\bottomrule
		\end{tabular}% 
		\caption{Translations for Common English Conditionals}
		\label{TransTableA}
	\end{center}
\end{table}

How well does the \mention{$\HORSESHOE$} capture the meaning of the English language conditional?  The only way for an \GSL{} sentence of the form $\parhorseshoe{\CAPPHI}{\CAPTHETA}$ to be false on a model $\IntA$ is for $\IntA$ to make $\CAPPHI$ true and $\CAPTHETA$ false.  Clearly such an assignment of truth values makes corresponding English language conditionals false: \q{If $2+2=4$ then the US State Department is secretly controlled by super-intelligent hamsters.}  So far, so good.  On all other combinations of truth values for $\CAPPHI$ and $\CAPTHETA$, \GSL{} sentences of the form $\parhorseshoe{\CAPPHI}{\CAPTHETA}$ are true.  Consider cases in which both the \CAPS{lhs} and the \CAPS{rhs} of a conditional are false.  Some such conditionals are true, for example: \q{If George Washington landed on the moon then George Washington landed on the moon.}  Indeed, this sentence seems to be a logical truth.  Others are more dubious: \q{If Soviet cosmonauts landed on the moon in 1968, then a cabal of super-intelligent hamsters will seize control of the US federal government later this year.}  We cannot distinguish these connectives purely by reference to the truth values of each side; in both cases, the \CAPS{lhs} and the \CAPS{rhs} are each false.  In order to admit the former as true, we must also admit the latter.  This is one price of understanding conditionals as truth-functional.

Next we address the case in which an \GSL{} model makes $\CAPPHI$ false and $\CAPTHETA$ true, and hence also makes $\parhorseshoe{\CAPPHI}{\CAPTHETA}$ true.  But first consider the following, in which each side is true:

\begin{menumerate}
	\item If George Washington crossed the Delaware River then George Washington crossed the Delaware River.
\end{menumerate}

\noindent{}This sentence is a logical truth of English.  But if we were to turn the \CAPS{lhs} into a conjunction and insert a false conjunct, we'll still think the result is true.  E.g.,

\begin{menumerate}
	\item If George Washington crossed the Delaware River and Thomas Jefferson invented bifocals, then George Washington crossed the Delaware River.
\end{menumerate}

\noindent{}(The true inventor of bifocals was probably Benjamin Franklin.)  Even though the conjunction on the \CAPS{lhs} is false, the whole conditional is still a logical truth.  And, indeed, the \GSL{} translation is \CAPS{tft}: $\parhorseshoe{\parconjunction{\Gl}{\Bl}}{\Gl}$.  So, while the \mention{$\HORSESHOE$} is not a perfect analogue of the English language conditional, it seems to be the best truth-functional translation.

Biconditionals in English are closely related to conditionals, and so they're subject to some of the same worries.  Nevertheless, we can treat them as truth-functional connectives for the purpose of translating them into \GSL{}:

\begin{menumerate}
	\item Ruth may play outside if and only if she cleans her room.
\end{menumerate}

	\begin{description}[itemsep=0em]
		\item[Translation Key:] \hfill{} 
		\begin{description}[itemsep=0em]
			\item $\Hl$: Ruth may play outside.
			\item $\Cl$: Ruth cleans her room.
		\end{description} 
	\end{description}

\noindent{}With this key we may translate the sentence as: $\partriplebar{\Hl}{\Cl}$.  We use the \mention{$\TRIPLEBAR$} to translate English biconditionals.

\begin{table}
	\renewcommand{\arraystretch}{1.5}%
	\begin{center}
		\begin{tabular}{ l l } %p{2.2in} p{2in}
			\toprule
			\textbf{English} & \textbf{\GSL{}} \\ 
			\midrule
			$\PHI$ if and only if $\THETA$ & $\triplebar{\CAPPHI}{\CAPTHETA}$ \\
			for $\PHI$ it's necessary and sufficient that $\THETA$ & $\triplebar{\CAPPHI}{\CAPTHETA}$ \\
			$\PHI$ just when $\THETA$ & $\triplebar{\CAPPHI}{\CAPTHETA}$ \\
			$\PHI$ just in case $\THETA$ & $\triplebar{\CAPPHI}{\CAPTHETA}$ \\
			\bottomrule
		\end{tabular} 
		\caption{Translations for Common English Biconditionals}
		\label{TransTableB}
	\end{center}
\end{table}

We saw above that the restaurant context may add exclusivity to an \mention{or}, which by itself is inclusive.  Similarly, an \mention{if} in a particular context may be interpreted as an \mention{iff} because of background context.  As in other cases, this context adds more information but is not itself part of the sentence.  A parent who says, \mention{You can have ice cream if you finish your homework,} is relying on context to add \mention{but not otherwise}.  Otherwise, it might turn out that the child can have ice cream either way.  We know that parents often use punishments and rewards to shape child behavior, so it is appropriate to interpret them accordingly.  Nevertheless, our policy is to translate what the sentence says literally. If it is important for an argument to articulate what is being added by context, that should be specified explicitly.

Let's look at a few more example translations.

\begin{majorILnc}{\LnpEC{GSLTranslationExampleB}} %implicature
	\begin{menumerate}
		\item\label{GSLTransSentenceE} Either Mia flew or both Jackson and Harper took off early.
	\end{menumerate} 
	The first step is to identify the main connective of the sentence. 
	In this case it's \mention{either $\ldots$ or}. When we translate \mention{either $\ldots$ or} as $\VEE$, we get: 
	\begin{menumerate}
		\item\label{GSLTransSentenceF} (Mia flew) $\VEE$ (both Jackson and Harper took off early).
	\end{menumerate}
	We then identify and translate the main connectives in the connecting clauses. 
	There are no connectives in \mention{Mia flew}, so there is nothing to do with it. 
	There is one connective in \mention{both Jackson and Harper took off early}, \mention{both $\ldots$ and}. 
	So we finish by translating this connective as a conjunction ($\WEDGE$):  
	\begin{menumerate}
		\item\label{GSLTransSentenceG} (Mia flew) $\VEE$ ((Jackson took off early) $\WEDGE$ (Harper took off early)).
	\end{menumerate}
	\begin{description}[itemsep=0em]
		\item[Translation Key:] \hfill{} 
		\begin{description}[itemsep=0em]
			\item[] $\Nl$: Mia flew.
			\item[] $\Jl$: Jackson took off early. 
			\item[] $\Hl$: Harper took off early. 
		\end{description}
	\end{description}
	\begin{menumerate}
		\item\label{GSLTransSentenceH} $\disjunction{\Nl}{\parconjunction{\Jl}{\Hl}}$
	\end{menumerate}
\end{majorILnc}

\begin{majorILnc}{\LnpEC{GSLTranslationExampleC}}
	\begin{menumerate}
		\item\label{GSLTransSentenceI} If Mia got the job and Jackson didn't, then Mia will take off tomorrow and Harper will have to come in.
	\end{menumerate}
	As before we identify the main connective, which is \mention{if $\ldots$ then}. 
	This is translated as a conditional ($\rightarrow$), yielding: 
	\begin{menumerate}
		\item\label{GSLTransSentenceJ} (Mia got the job and Jackson didn't) $\HORSESHOE$ (Mia will take off tomorrow and Harper will have to come in).
	\end{menumerate}
	Next we look at the \CAPS{LHS} of the conditional, \mention{Mia got the job and Jackson didn't}. 
	The main (and only) connective of this sentence is \mention{and}, which we translate as a conjunction ($\WEDGE$), yielding: 
	\begin{menumerate}
		\item\label{GSLTransSentenceK} ((Mia got the job) $\WEDGE$ (Jackson didn't get the job)) $\HORSESHOE$ (Mia will take off tomorrow and Harper will have to come in).
	\end{menumerate}

	Note that the right-hand conjunct \mention{Jackson didn't} is an elliptical clause. 
	The sentence is saying that Mia got the job and Jackson didn't \emph{get the job}.  The word \mention{did} can function sort of like a pronoun, except instead of referring to some person or object it alludes to another clause.
	We've made this explicit in by putting the part of the clause that was left tacit in brackets. 
The clause \mention{Jackson didn't} has a negation in it. It combines \mention{not} with \mention{Jackson got the job}. 
	
	We finish the translation by working on the \CAPS{RHS} of the sentence, which is a conjunction: 
	\begin{menumerate}
		\item\label{GSLTransSentenceL} ((Mia got the job) $\WEDGE$ ( $\NEGATION$ (Jackson got the job))) $\HORSESHOE$ ((Mia will take off tomorrow) $\WEDGE$ (Harper will have to come in)).
	\end{menumerate}
	There are no more connectives, so from here we just need a translation key. 
	\begin{description}[itemsep=0em]
		\item[Translation Key:] \hfill{} 
		\begin{description}[itemsep=0em]
			\item[] $\Nl$: Mia got the job.
			\item[] $\Jl$: Jackson got the job. 
			\item[] $\Il$: Mia will take off tomorrow.
			\item[] $\Hl$: Harper will have to come in late.  
		\end{description}
	\end{description}
	We can then finish the translation: 
	\begin{menumerate}
		\item\label{GSLTransSentenceM} $\horseshoe{\parconjunction{\Nl}{\negation{\Jl}}}{\parconjunction{\Il}{\Hl}}$
	\end{menumerate}
\end{majorILnc}


\begin{majorILnc}{\LnpEC{GSLTranslationExampleE}}
	Some connectives in English are not complex, but their translations into \GSL{} are complex because we did not introduce individual
	symbols for all possible truth-functions. Nevertheless, we can express any truth-function using some combination of the \GSL{} connectives that we \emph{do} have.
	For example: 
	\begin{menumerate}
		\item\label{GSLTransSentenceU} Neither Mia nor Harper were late.
	\end{menumerate} 
	Here the main connective is \mention{neither $\ldots$ nor}, and it joins the two sentences \mention{Mia [was late]} and \mention{Harper [was] late}. 
	(Neither of these two sentences is made up of any other connectives.) 
	Although \mention{neither $\ldots$ nor} is not complex, it is complex relative to \GSL{}. 
	\GSL{} has no single connective which can correctly translate \mention{neither $\ldots$ nor}.%
	\footnote{%
		But recall the logical connective \CAPS{NOR}, discussed at the very end of section \mvref{Disjunctive Normal Form}. 
		This connective would correctly translate \mention{neither $\ldots$ nor}, but isn't part of \GSL{}. 
	} 
	Recall theorem \mvref{Truth-functional Expressive Completeness of GSL}, which says that any truth-functional connective can be expressed in \GSL{}. 
	This means there must be some way to capture \mention{neither $\ldots$ nor} using a complex of connectives in \GSL{}. 
	There are two ways to do this.  We may translate \mention{neither $\PHI$ nor $\THETA$} as either $\negation{\pardisjunction{\CAPPHI}{\CAPTHETA}}$ or $\conjunction{\negation{\CAPPHI}}{\negation{\CAPTHETA}}$.
	\begin{table}
		\renewcommand{\arraystretch}{1.5}%
		\begin{center}
			\begin{tabular}{ l l } %p{2.2in} p{2in}
				\toprule
				\textbf{English} & \textbf{\GSL{}} \\ 
				\midrule
				not $\PHI$ & $\negation{\CAPPHI}$ \\
				it's not the case that $\PHI$ & $\negation{\CAPPHI}$ \\
				$\PHI$ unless $\THETA$ & $\horseshoe{\negation{\CAPTHETA}}{\CAPPHI}$ \\
				$\PHI$ unless $\THETA$ & $\disjunction{\CAPTHETA}{\CAPPHI}$ \\
				unless $\PHI$, $\THETA$ & $\horseshoe{\negation{\CAPPHI}}{\CAPTHETA}$ \\
				$\PHI$ if not $\THETA$ & $\horseshoe{\negation{\CAPTHETA}}{\CAPPHI}$ \\
				neither $\PHI$ nor $\THETA$ & $\negation{\pardisjunction{\CAPPHI}{\CAPTHETA}}$ \\
				& $\conjunction{\negation{\CAPPHI}}{\negation{\CAPTHETA}}$ \\
				not both $\PHI$ and $\THETA$ & $\negation{\parconjunction{\CAPPHI}{\CAPTHETA}}$ \\
				& $\disjunction{\negation{\CAPPHI}}{\negation{\CAPTHETA}}$ \\
				\bottomrule
			\end{tabular}
			\caption{Translations for Common English Negations and Complex Connectives}
			\label{TransTableF} 
		\end{center}
	\end{table}  
	So we have: 
	\begin{menumerate}
		\item\label{GSLTransSentenceV} $\NEGATION$ ((Mia was late) $\VEE$ (Harper was late)).
	\end{menumerate} 
	With the key: 
	\begin{description}[itemsep=0em]
		\item[Translation Scheme:] \hfill{} 
		\begin{description}[itemsep=0em]
			\item $\Nl$: Mia was late.
			\item $\Hl$: Harper was late. 
		\end{description} 
	\end{description}
	The final translation of sentence (\ref{GSLTransSentenceU}) is: 
	\begin{menumerate}
		\item\label{GSLTransSentenceW} $\negation{\pardisjunction{\Nl}{\Hl}}$
	\end{menumerate} 
\end{majorILnc}

\noindent{}We conclude this section by emphasizing two points.
First, the tables in this chapter should be thought of as rough-and-ready guides. 
Although many particular uses of \mention{and} express conjunction, not all do. 
Sometimes \mention{and} doesn't function as a connective at all, e.g. as in \mention{it will be years and years before the trees bear fruit} \citep[107]{Smith2012}.
Other times \mention{and} functions as a connective, but expresses a conditional instead of a conjunction, e.g. \mention{study hard, and you will pass the exam} \citep[107]{Smith2012}.

Whenever deciding on how to translate a connective from English you must carefully determine what is actually being expressed. Many English sentences are ambiguous.  We usually don’t notice because the context directs our attention to one meaning rather than the other.  For example, ``Maria and Paul are married'' can convey the conjunction ``Maria is married and Paul is married'', or it can express ``Maria and Paul are married to each other'' in which case the `and' is not a conjunction but is serving a different logical purpose to be discussed in the next section.

The second point concerns how to decide what \GSL{} connective (or complex of connectives) translates a given English connective. 
The idea, again, is to make sure that the chosen \GSL{} connective expresses the same truth function as the English connective---that is, that the truth of the sentences formed by these connectives depends in the same way on the truth of the (sub)sentences that were joined. 
  

%%%%%%%%%%%%%%%%%%%%%%%%%%%%%%%%%%%%%%%%%%%%%%%%%%
\section{QL Applications}
%%%%%%%%%%%%%%%%%%%%%%%%%%%%%%%%%%%%%%%%%%%%%%%%%%

\subsection{Constants and Predicates}
Let's say we want to translate the following sentence into \emph{\GSL{}}:

\begin{smenumerate}
	\item\label{GQLTrans1} Mary is happy, smart, adorable, and a child.
\end{smenumerate}

\noindent{}The only connective we can translate is the \mention{and}.  The translation key will look something like the following:

\begin{description}[itemsep=0em]
	\item[Translation Key:] \hfill{} 
	\begin{description}[itemsep=0em]
		\item[] $\Hl$: Mary is happy.
		\item[] $\Rl$: Mary is smart. 
		\item[] $\Al$: Mary is adorable.
		\item[] $\Cl$: Mary is a child.  
	\end{description}
\end{description}

\noindent{}So the \GSL{} result is: $\parconjunction{\conjunction{\Hl}{\Rl}}{\conjunction{\Al}{\Cl}}$.  This translation might work for certain purposes, but the various conjuncts have nothing in common, as a matter of logical structure.  There is nothing to indicate explicitly that each conjunct is about the same person, unless we consult the translation key.

\GQL{} gives us more precision by providing individual constants and predicates.  Instead of picking out simple sentences of English, we may instead pick out subjects and predicates of simple sentences.  Let's use the following \GQL{} key to translate (\ref{GQLTrans1}):

\begin{description}[itemsep=0em]
	\item[Translation Key:] \hfill{} 
	\begin{description}[itemsep=0em]
		\item[] $\constant{m}$: Mary
		\item[] $\Hl\variable{t}$: $\variable{t}$ a child. 
		\item[] $\Rl\variable{t}$: $\variable{t}$ is smart. 
		\item[] $\Al\variable{t}$: $\variable{t}$ is adorable.
		\item[] $\Cl\variable{t}$: $\variable{t}$ is a child.  
	\end{description}
\end{description}

\noindent{}Now the result is: $\parconjunction{\conjunction{\Hp{\constant{m}}}{\Rp{\constant{m}}}}{\conjunction{\Ap{\constant{m}}}{\Cp{\constant{m}}}}$.  This translation is more complex than the \GSL{} translation, but we can see that we are predicating several things about the same person.  This gives \GQL{} more power and precision than \GSL{}.  Consider the following sentence:

\begin{menumerate}
	\item\label{GQLTrans2} Ronnie and Demaryius are athletic, but Peyton isn't.
\end{menumerate}

\noindent{}We \emph{could} translate this as a conjunction with three conjuncts in \GSL{}.  However, this would not make clear that the same predicate either applies, or doesn't, to each of the three.  Instead, let's use the following \GQL{} key:

\begin{description}[itemsep=0em]
	\item[Translation Key:] \hfill{} 
	\begin{description}[itemsep=0em]
		\item[] $\constant{r}$: Ronnie
		\item[] $\constant{d}$: Demaryius
		\item[] $\constant{p}$: Peyton
		\item[] $\Al\variable{t}$: $\variable{t}$ is athletic.
	\end{description}
\end{description}

\noindent{}The result: $\parconjunction{\conjunction{\Ap{\constant{r}}}{\Ap{\constant{d}}}}{\negation{\Ap{\constant{p}}}}$.  Now we can clearly see the common predicate in each conjunct.

The translation keys here resemble the \GQL{} models we provide; for example, see table \ref{table:Example Interpretations} in Chapter \ref{quantifierlogic}.  One difference is that we don't have a domain assigned in either of the \GQL{} keys above.  However, we could add a domain to each of the above, and understand the constant and predicate lines of the key as making assignments from the domain.  Hence, we can effectively treat model assignments as also playing a translation key role.

\subsection{Quantifiers}

We mentioned in Chapter \ref{quantifierlogic1} that the quantifier \mention{$\forall$} corresponds to the English phrases \mention{all} or \mention{every}.  Let's say we want to claim that every member of some class is also a member of some other class.  For example:

\begin{menumerate}
	\item\label{GQLTrans3} All dogs are furry.
\end{menumerate}

\noindent{}We can imagine that we have two sets: the set of all dogs and the set of all furry things.  Sentence (\ref{GQLTrans3}) effectively claims that the set of dogs is a subset of the set of furry things.  How do we translate this into \GQL{}?  We see the word \mention{all}, so we'll want to use a universal quantifier.  This isn't obvious from the sentence itself, but whenever we want to make claims such that one set is a subset of another, we'll nearly always want to translate it into the following form: $\universal{\ALPHA}\parhorseshoe{\CAPPHI}{\CAPTHETA}$; i.e., with a \mention{$\forall$} governing an \mention{$\HORSESHOE$}.  This makes more sense if we paraphrase (\ref{GQLTrans3}) in MathEnglish as: \mention{For all $\variable{x}$, if $\variable{x}$ is a dog then $\variable{x}$ is furry.}  For the rest of the translations in this section, let's use the following model as our translation key:

\begin{description}[itemsep=0em]
	\item[Animals model:] \hfill{} 
	\begin{description}[itemsep=0em]
		\item[] $\emph{Animals}(\variable{U})$: All animals.
		\item[] $\emph{Animals}(\Ap{'})$: is a mammal.
		\item[] $\emph{Animals}(\Cp{'})$: is a cat.
		\item[] $\emph{Animals}(\Dp{'})$: is a dog.
		\item[] $\emph{Animals}(\Ep{'})$: is energetic.
		\item[] $\emph{Animals}(\Hp{'})$: is a happy.
		\item[] $\emph{Animals}(\Rp{'})$: is furry.
		\item[] $\emph{Animals}(\Ap{''})$: is smarter than.
	\end{description}
\end{description}

\noindent{}So we may translate (\ref{GQLTrans3}) as: $\universal{\variable{x}}\parhorseshoe{\Dp{\variable{x}}}{\Rp{\variable{x}}}$.  We may translate other English words using the universal quantifier as well.  The following uses the word \mention{no} to make a universal claim.

\begin{menumerate}
	\item\label{GQLTrans4} No dogs are furry.
\end{menumerate}

\noindent{}Again, we can think of this as a claim about two sets.  This sentence basically claims that the set of dogs and the set of furry things are disjoint, i.e., that nothing is a member of both sets.  We can usually translate such claims into the following form in \GQL{}: $\universal{\ALPHA}\parhorseshoe{\CAPPHI}{\negation{\CAPTHETA}}$.  To see this, consider the following MathEnglish paraphrase: \mention{For all $\variable{x}$, if $\variable{x}$ is a dog then $\variable{x}$ is not furry.}  The resulting translation is thus:  $\universal{\variable{x}}\parhorseshoe{\Dp{\variable{x}}}{\negation{\Rp{\variable{x}}}}$.  The word \mention{only} can also be used for universal claims:

\begin{menumerate}
	\item\label{GQLTrans5} Only dogs are furry.
\end{menumerate}

\noindent{}This is translated in exactly the same way as sentence (\ref{GQLTrans3}), except that we reverse the order of the LHS and the RHS of the conditional governed by the \mention{$\forall$}.  It's roughly equivalent to the claim that \mention{All furry things are dogs.}  So: $\universal{\variable{x}}\parhorseshoe{\Rp{\variable{x}}}{\Dp{\variable{x}}}$.

We said in Chapter \ref{quantifierlogic} that \mention{$\exists$} corresponds to the English phrases \mention{there exists}, \mention{there is}, or \mention{some}.  Consider the following existential sentences.

\begin{menumerate}
	\item\label{GQLTrans6} Some dogs are furry.
	\item\label{GQLTrans7} Some dogs are not furry.
\end{menumerate}

\noindent{}If we again think of the set of dogs and the set of furry things, (\ref{GQLTrans6}) is a claim that there is at least one thing that is a member of each set.  Notice that we are interpreting (\ref{GQLTrans6}) as a claim about at least one object.  In English we typically think that (\ref{GQLTrans6}) is a claim about at least two dogs.  For now we will ignore certain plural/singular distinctions in the way we interpret existential claims.  For our purposes, \mention{some} will mean \mention{at least one}.  We will be able to handle \mention{some} in a more satisfactory way when we add to \GQL{} in Chapter \ref{furtherdirections}.

For now, we will translate sentences like (\ref{GQLTrans6}) into the form: $\existential{\ALPHA}\parconjunction{\CAPPHI}{\CAPTHETA}$.  So, for (\ref{GQLTrans6}) itself: $\existential{\variable{x}}\parconjunction{\Dp{\variable{x}}}{\Rp{\variable{x}}}$.  And we can account for the \mention{not} in sentence (\ref{GQLTrans7}) as follows: $\existential{\variable{x}}\parconjunction{\Dp{\variable{x}}}{\negation{\Rp{\variable{x}}}}$.

We can also translate sentence (\ref{GQLTrans4}), \mention{No dogs are furry}, as an existential governed by a negation.  We could translate \mention{There is a furry dog} as: $\existential{\variable{x}}\parconjunction{\Rp{\variable{x}}}{\Dp{\variable{x}}}$.  Now we may negate the result to capture the meaning of (\ref{GQLTrans4}): $\negation{\existential{\variable{x}}\parconjunction{\Rp{\variable{x}}}{\Dp{\variable{x}}}}$.  In fact, this latter translation is logically equivalent to the one we gave earlier: $\universal{\variable{x}}\parhorseshoe{\Dp{\variable{x}}}{\negation{\Rp{\variable{x}}}}$.  To see this, join these two sentences together with a biconditional, \mention{$\TRIPLEBAR$}, and prove that the result is \CAPS{qt}.

Let's translate some more complicated sentences into \GQL{}:

\begin{menumerate}
	\item\label{GQLTrans8} All happy dogs are furry and energetic.
\end{menumerate}

\noindent{}We'll want to translate this as a universal quantifier governing a conditional, but we must be careful to translate each side of the conditional correctly.  We can paraphrase (\ref{GQLTrans8}) in MathEnglish as: For all $\variable{x}$, if ($\variable{x}$ is happy and $\variable{x}$ is a dog) then ($\variable{x}$ is furry and $\variable{x}$ is energetic).  So we can consider each side of the conditional as a conjunction: $\universal{\variable{x}}\parhorseshoe{\parconjunction{\Hp{\variable{x}}}{\Dp{\variable{x}}}}{\parconjunction{\Rp{\variable{x}}}{\Ep{\variable{x}}}}$.

\begin{menumerate}
	\item\label{GQLTrans9} All cats and dogs are mammals.
\end{menumerate}

\noindent{}This sentence is also going to be translated as a universal quantifier governing a conditional, but the word \mention{and} can be tricky.  Here `and' conjoins predicates, not sentences.  We may be tempted to translate (\ref{GQLTrans9}) as: $\universal{\variable{x}}\parhorseshoe{\parconjunction{\Cp{\variable{x}}}{\Dp{\variable{x}}}}{\Ap{\variable{x}}}$.  But this \GQL{} sentence can be translated into MathEnglish as: \mention{For every $\variable{x}$, if ($\variable{x}$ is a cat and $\variable{x}$ is a dog) then $\variable{x}$ is a mammal.}  But that's silly.  Unless mad scientists are involved, nothing is both a cat and a dog.  Instead, we should translate the \mention{and} in (\ref{GQLTrans9}) as a \mention{$\VEE$}:  $\universal{\variable{x}}\parhorseshoe{\pardisjunction{\Cp{\variable{x}}}{\Dp{\variable{x}}}}{\Ap{\variable{x}}}$.  Let's translate this \GQL{} sentence into MathEnglish: \mention{For all $\variable{x}$, if ($\variable{x}$ is a cat or $\variable{x}$ is a dog), then $\variable{x}$ is a mammal.}  Take a moment to see how this better expresses the meaning of (\ref{GQLTrans9}).

Now let's look at a sentence with multiple quantifiers:

\begin{menumerate}
	\item\label{GQLTrans10} All dogs are smarter than all cats.
\end{menumerate}

\noindent{}We see the words \mention{all} twice in this sentence, so we'll want to include two universal quantifiers in the translation.  Now consider a paraphrase into MathEnglish: \mention{For every $\variable{x}$, if $\variable{x}$ is a dog then (for all $\variable{y}$, if $\variable{y}$ is a cat then $\variable{x}$ is smarter than $\variable{y}$).}  So we may translate (\ref{GQLTrans10}) as: $\universal{\variable{x}}\parhorseshoe{\Dp{\variable{x}}}{\universal{\variable{y}}\parhorseshoe{\Cp{\variable{y}}}{\App{\variable{x}}{\variable{y}}}}$.

We have not said much about the role of domains in translations.  Because the main point of translations, other than the sheer joy of doing it, is to evaluate arguments, it is appropriate to choose a domain suitable for the arguments in question.  Often a suitable choice of domain simplifies the translations.  If we are translating arguments that involve essential mention of dogs and natural numbers, we need to include both in the domain and to have a predicate for each.  If the arguments deal only with dogs, then we can take the domain to be dogs and we don’t need a predicate for `dog'.

As we observed earlier many English sentences are ambiguous.  One systematic ambiguity is in sentences of the form ``All As are not Bs'', which can either mean that it is not true that All As are Bs, or that all As are not-Bs.  Context or content (inclusively) usually indicate what is meant:  ``All sheep are not good pets'' would usually have the first meaning: $\negation{\universal{\variable{x}}\parhorseshoe{\Sl\variable{x}}{\parconjunction{\Gp{\variable{x}}}{\Pp{\variable{x}}}}}$; while ``All sharks are not good pets'' would have the second: $\universal{\variable{x}}\parhorseshoe{\Sl\variable{x}}{\negation{\parconjunction{\Gp{\variable{x}}}{\Pp{\variable{x}}}}}$.  One of the values of formalization is that we can clearly and unambiguously express the structure of both.

It is important to distinguish cases of ambiguity---English sentences that  have more than one meaning---from sentences for which there is more than one good translation, but which are equivalent.  We see this with ``No sharks are good pets'': $\universal{\variable{x}}\parhorseshoe{\Sl\variable{x}}{\negation{\parconjunction{\Gp{\variable{x}}}{\Pp{\variable{x}}}}}$ and $\negation{\existential{\variable{x}}\parconjunction{\Sl\variable{x}}{\parconjunction{\Gp{\variable{x}}}{\Pp{\variable{x}}}}}$ are equally good (and equivalent) translations.

%%%%%%%%%%%%%%%%%%%%%%%%%%%%%%%%%%%%%%%%%%%%%%%%%%
\section{Exercises}
%%%%%%%%%%%%%%%%%%%%%%%%%%%%%%%%%%%%%%%%%%%%%%%%%%

\notocsubsection{\GSL{} to English Translations}{ex:GSL to English Translations}
Given the following glossary, translate the following \GSL{} sentences into English. 
\begin{description}[itemsep=0em]
	\item[Glossary:] \hfill{} 
	\begin{description}[itemsep=0em]
		\item $\Cl$: Cindy the Capybara is a picky eater.
		\item $\Ol$: Oscar the Ocelot sleeps all day.
		\item $\Rl$: Ralph the Rhinoceros goes for a swim.
		\item $\Al$: France is east of Spain. 
	\end{description} 
\end{description}
\begin{multicols}{2}
	\begin{enumerate}
		\item $\horseshoe{\Cl}{\Ol}$
		\item $\horseshoe{\Ol}{\Cl}$
		\item $\negation{\parhorseshoe{\Rl}{\Al}}$
		\item $\horseshoe{\negation{\Rl}}{\Al}$
		\item $\disjunction{\parconjunction{\Cl}{\Ol}}{\negation{\partriplebar{\Cl}{\Al}}}$
		\item $\conjunction{\Cl}{\pardisjunction{\Ol}{\negation{\partriplebar{\Cl}{\Al}}}}$
		\item $\horseshoe{\Al}{\negation{\parconjunction{\Cl}{\Rl}}}$
		\item $\horseshoe{\parconjunction{\Cl}{\Rl}}{\negation{\Al}}$
	\end{enumerate}
\end{multicols}

\notocsubsection{English to \GSL{} Translations \#1}{ex:English to GSL Translations 1}
Using some sensible translation key translate the following English sentences into \GSL{}. 
\begin{enumerate}
	\item If the sprockets come in on time, then we can fill the order.
	\item Only if the sprockets come in on time can we fill the order. 
	\item Either the order gets filled, or the cogs come in late and the sprockets never show up. 
	\item It's not the case that the sprockets need to come in for the order to be filled. 
	\item While filling the order is important, getting the sprockets in is more so. 
	\item The sprockets and cogs are late, but it's still not the case that we can't fill the order on time. 
	\item Assuming the order gets out on time, the sprockets will fail to arrive only if the cogs are either late or defective. 
	\item The spork is the least appreciated utensil. 
	\item They dined on mince, and slices of quince, [which] they ate with a runcible spoon. (1871, Edward Lear, “Owl \& Pussy-Cat” in \emph{Nonsense Songs})
	\item You eat with a spork if and only if you eat with a foon. 
	\item Although Jan will be amused, if you eat with a spork Jill will leave or at least not laugh.
	\item If you have a runcible spoon, then you don't need a fork, knife, or spoon. 
\end{enumerate}

\notocsubsection{English to \GSL{} Translations \#2}{ex:English to GSL Translations 2}
Using some sensible translation key translate the following English sentences into \GSL{}. 
\begin{enumerate}
	\item If 14-year-olds had the vote, I'd be president. (Evel Knievel)
	\item If Miami beats Cornell today and Penn State defeats Michigan State Miami will win the tournament.
	\item Should senator Ervin run again, he would be a formidable opponent. 
	\item If Congress does not find a way to force the banking industry to lower interest rates, and if the Securities and Exchange Commission does not stop authorizing unjustified new financing by corporations, we will face an unbearable depression. 
	\item Provided, but only provided, that the French Fleet is sailed forthwith for British harbors, His Majesty's Government give their full consent to an armistice for France. (Churchill, June 1940)
	\item For the tenability of the thesis that mathematics is logic it is not only sufficient but also necessary that all mathematical expressions be capable of definition on the basis solely of logical ones. (W.V.O. Quine)
\end{enumerate}


\notocsubsection{Translations}{Translation Problems} Translate each of the following English sentences into \GQL{} sentences about the model $\IntA$ given in table \mvref{Trans Int Table}.
\begin{multicols}{2}
	\begin{enumerate}
		\item {All Pacific states that border a mountainous state are coastal.}
		\item {Some Atlantic state and some mountainous state both share a border with a state that is neither.}
		\item {All states are coastal and mountainous if and only if they are Pacific.}
		\item {All Atlantic states smaller than Montana share a border with Rhode Island.}
		\item {Only mountainous Pacific states are coastal.}
		\item {A Pacific state is mountainous.}
		\item {No state is larger than itself.}
		\item {Every non-mountainous state borders a state that is larger.}
		\item {Some Pacific states are mountainous.}
		\item {All Pacific states are mountainous.}
		\item {All Pacific states are larger than all Atlantic states.}
		\item {No Gulf state is mountainous.}
		\item {All Atlantic states are not mountainous.}
		\item {Some Gulf state is larger than all states that border it.}
		\item {Some Gulf state is an Atlantic state.}
		\item {All states that border a Pacific state are mountainous.}
		\item {Any state that is mountainous is larger than Rhode Island.}
		\item {Any state that is mountainous is larger than all Atlantic
			states.}
		\item {If any state is mountainous, California is.}
		\item {If any state is mountainous, it is larger than Rhode Island.}
		\item {Any state that has no bordering states is mountainous.}
		\item {All states that are bigger than all mountainous states are
			coastal.}
		\item {No state is bigger than Montana unless it is coastal.}
	\end{enumerate}
\end{multicols}
%\begin{table}[!ht]
%\renewcommand{\arraystretch}{1.5}
%\begin{center}
\begin{longtable}[c]{ l l l } %p{2.2in} p{2in}
	\toprule
	&\textbf{Symbol} & \textbf{Model Assignment} \\
	\midrule 
	\endfirsthead
	\multicolumn{3}{c}{\emph{Continued from Previous Page}}\\
	\toprule
	&\textbf{Symbol} & \textbf{Model Assignment} \\
	\midrule 
	\endhead
	\bottomrule
	\caption{Model for Translations in Section \ref{Translation Problems}}\\ %[-.15in]
	\multicolumn{3}{c}{\emph{Continued next Page}}\\
	\endfoot
	\bottomrule
	\caption{Model for Translations in Section \ref{Translation Problems}}\\%
	\endlastfoot%
	\label{Trans Int Table}% 
	Universe:& & The set of states \\ \addlinespace[.25cm]
	Constants:& $\constant{c}$& CA\\
	& $\constant{m}$& MT\\
	& $\constant{h}$& RI\\
	& $\constant{e}$& TX\\ \addlinespace[.25cm]
	1 place predicates: &$\Pp{'}$& Pacific states\\
	&$\Ap{'}$& Atlantic states\\
	&$\Gp{'}$& Gulf states\\
	&$\Mp{'}$& Mountainous states\\
	&$\Cp{'}$& Coastal states\\ \addlinespace[.25cm]
	2 place predicates:&$\Lp{''}$& is larger than (area)\\
	&$\Bp{''}$& borders\\
\end{longtable}

\notocsubsection{More Translations}{ex:More Translations}
Translate each of the following English sentences into \GQL{} sentences.
\begin{multicols}{2}
	\begin{enumerate}
		\item {All beavers avoid some kangaroo.}
		\item {All beavers avoid all kangaroos.}
		\item {Some beaver avoids all kangaroos.}
		\item {Every kangaroo is avoided by some beaver.}
		\item {All beavers avoid any kangaroo that frightens them.}
		\item {Some beavers avoid any kangaroo that frightens them.}
		\item {No kangaroo frightens any beaver.}
		\item {No beaver is frightened by any kangaroo.}
		\item {No beaver avoids a kangaroo unless the kangaroo frightens it.}
		\item {Some kangaroo frightens itself.}
		\item {No beaver avoids a kangaroo unless the beaver frightens the kangaroo.}
		\item {Any kangaroo that is frightened of itself is frightened by any beaver.}
		\item {Beavers avoid kangaroos only if they frighten them.}
		\item {Kangaroos that frighten beavers frighten themselves.}
		\item {All kangaroos avoid any kangaroo that avoids them.}
		\item {When a kangaroo frightens a beaver, the beaver avoids it.}
		\item {Beavers only avoid kangaroos.}
		\item {Beavers are frightened of all kangaroos unless they avoid them.}
		\item {Some beavers avoid only kangaroos that frighten them.}
		\item {No beaver that avoids all kangaroos frightens itself.}
	\end{enumerate}
\end{multicols}



%\theendnotes

%%%%%%%%%%%%%%%%%%%%%%%%%%%%%%%%%%%%%%%%%%%%%%%%%%
\chapter{Derivations}\label{Derivations}
%%%%%%%%%%%%%%%%%%%%%%%%%%%%%%%%%%%%%%%%%%%%%%%%%%
%\AddToShipoutPicture*{\BackgroundPicC}

%%%%%%%%%%%%%%%%%%%%%%%%%%%%%%%%%%%%%%%%%%%%%%%%%%
\section{Introduction}\label{Derivation Preliminaries}
%%%%%%%%%%%%%%%%%%%%%%%%%%%%%%%%%%%%%%%%%%%%%%%%%%


We were able to make sense of semantic notions like truth, logical truth, and entailment by introducing models for \GSL{} and \GQL{}.
But proving that a sentence of \GSL{} or \GQL{} is a logical truth, or proving that an entailment holds, is difficult and involves informal reasoning in MathEnglish. 
We'd like a way to prove such things that's easier and which relies less on intuitive judgment. 
Recall from section \ref{Formal Languages} that \GSL{} and \GQL{} are formal in the sense that which strings of basic symbols are sentences of each is given by explicit definitions that make reference only to the shapes and arrangement of those basic symbols. 
Can we find a \emph{formal} method for establishing logical truth and entailments?

For this purpose we'll introduce formal rules for manipulating \GSL{} and \GQL{} sentences which will allow us to write formal derivations. 
Roughly, a \emph{derivation} is a finite sequence of sentences, with rules governing which sentences may follow which. 
For every sentence in the derivation there is a rule which sanctions (or permits) us to write it down next, usually depending on the previous sentences in the derivation.%
\footnote{%
Often times the terms \mention{derivation} and \mention{proof} (or \mention{formal proof}) are used interchangeably. 
The subfield of logic that studies derivations is even called proof theory.
Here, for clarity, we'll always use \mention{derivation} to refer to the formal proofs we'll work with in this chapter and \mention{proof} to refer to the more informal mathematical proofs we write in MathEnglish.
} 
Our derivations are \emph{formal} because the rules they use are formal.
That is, the rules are specified entirely in terms of the shape, or form, of the sentences.

This raises an important point: since the rules are formal, there is no prima facie connection between them (or derivations written using them) and our semantic notions of truth, logical truth, and entailment. 
For example, the rules will not say things like: 
\begin{RESTARTmenumerate}
\item If $\CAPPHI\sdtstile{}{}\CAPPSI$ and you have $\CAPPHI$ on a previous line, then you can write down $\CAPPSI$. 
\end{RESTARTmenumerate}
Instead, they'll say things like: 
\begin{menumerate}
\item If you have $\conjunction{\CAPPHI}{\CAPPSI}$ on a previous line, then you can write down $\CAPPSI$.
\end{menumerate} 
Despite this lack of a transparent connection, we want our rules to match up with the semantics we defined in previous chapters. 
For example, an important constraint is that the rules are truth-preserving. 
\begin{description}
\item[Truth-preservation:] If a rule allows us to write down $\CAPPHI$ when applied to sentences $\CAPPSI_1,\ldots,\CAPPSI_{\integer{n}}$, then $\CAPPSI_1,\ldots,\CAPPSI_{\integer{n}}\sdtstile{}{}\CAPPHI$.\index{truth-preserving}\index{derivation!rule!truth-preserving} 
\end{description} 
More generally, because we want to use the derivations as a means of showing that a sentence is a logical truth or of showing that some set of sentences entails some other sentence, we want at least:
\begin{description}
\item[Soundness:] If $\CAPPHI$ can be derived from $\CAPPSI_1,\ldots,\CAPPSI_{\integer{n}}$, then $\CAPPSI_1,\ldots,\CAPPSI_{\integer{n}}\sdtstile{}{}\CAPPHI$.\index{soundness}
\end{description} 
\noindent{}And ideally we'd like to be able to get:
\begin{description}
\item[Completeness:] If $\CAPPSI_1,\ldots,\CAPPSI_{\integer{n}}\sdtstile{}{}\CAPPHI$, then $\CAPPHI$ can be derived from $\CAPPSI_1,\ldots,\CAPPSI_{\integer{n}}$.\index{completeness}
\end{description}

We will discuss and prove these two results (and variations of them) for both \GSL{} and \GQL{} in the next chapter.

While our derivation systems (our sets of rules with which we write derivations) are definitionally independent of our semantics (our models and definitions for truth, logical truth, and entailment), there are in fact deep and important connections between them. 
We'll prove the most important of these connections---including soundness and completeness---in the next chapter. 
For now though it will probably be beneficial not to worry about them and to approach derivations on their own terms. 
It can be useful to think of writing a derivation as solving a puzzle, or trying to win a game. 
Like any game there are rules for the moves you can make, or like a puzzle there's a way the inferences made in the derivation can fit together. 
(Although, unlike a puzzle there are always many ways they can fit together.) 

%%%%%%%%%%%%%%%%%%%%%%%%%%%%%%%%%%%%%%%%%%%%%%%%%%
\section{The Basic System \GSD{}}
%%%%%%%%%%%%%%%%%%%%%%%%%%%%%%%%%%%%%%%%%%%%%%%%%%

\subsection{Introduction and Elimination Rules} 
We will work with two kinds of rules: basic rules and shortcut rules. 
The shortcut rules are, as the name suggests, not necessary.\index{derivation!rule!shortcut} 
Their purpose will be to make derivations easier and shorter. 
Anything we can derive with the shortcut rules can be derived from the basic rules alone (see thm. \pmvref{GSD Shortcut Theorem3}). 
The basic rules\index{derivation!rule!basic}\index{basic rule} make up what we call \idf{Sentential Derivation System}, or \GSD{}. 
\GSD{} consists of two special rules, \Rule{Assumption} and \Rule{Repetition}, and a pair of basic rules for each of the logical connectives in \GSL{}. 
When we move on to \GQL{} we'll add more basic rules for the logical connectives unique to \GQL{}, i.e. the quantifiers. 
One of the pair will be an \emph{introduction} rule,\index{derivation!rule!introduction}\index{introduction rule} the other will be an \emph{elimination} rule.\index{derivation!rule!elimination}\index{elimination rule} 

For example, consider the conjunction, $\WEDGE$. If we earlier derived the sentences $\CAPPHI$ and $\CAPPSI$, we may introduce their conjunction $\conjunction{\CAPPHI}{\CAPPSI}$ with the $\WEDGE$ introduction rule. 
Alternatively, if we have derived $\conjunction{\CAPPHI}{\CAPPSI}$, the $\WEDGE$ elimination rule lets us take out a part of the conjunction and write either $\CAPPHI$ or $\CAPPSI$.
Elimination rules typically give us a choice for the next step, and learning to use the system effectively partly involves learning strategies for choosing the next step.
Table \mvref{GSD} gives all the basic SD rules.\footnote{As mentioned in the preface, this kind of introduction and elimination rule-based derivation system is called a \idf{natural deduction} system.
Natural deduction systems were first invented by Stanislaw Jaskowski in 1926, though he did not publish until 1934.  Other important developments are due to Gerhard Gentzen \citeyearpar{Gentzen1934}; other influential natural deduction systems are given by J. M. Anderson and H. W. Johnstone \citeyearpar{Anderson1962}, Frederic Fitch \citeyearpar{Fitch1952}, Donald Kalish and Richard Montague \citeyearpar{Kalish1964}, Lemmon E. J. \citeyearpar{Lemmon1965}, Dag Prawitz \citeyearpar{Prawitz1965}, Willard Quine \citeyearpar{Quine1950}, Patrick Suppes \citeyearpar{Suppes1957}, Neil Tennant \citeyearpar{Tennant1978}, R. H. Thomason \citeyearpar{Thomason1970}, and Dirk van Dalen \citeyearpar{Dalen1980} \citep[28]{Hodges2001}.  Unlike many systems, ours uses boxes to discharge assumptions; Jaskowski mentions in the 1934 article that he used boxes in 1926, but they aren't part of his official system in 1934.}
%\begin{table}[!ht]
\renewcommand{\arraystretch}{1.5}
%\begin{center}
\begin{longtable}[c]{ p{1in} l l } %p{2.2in} p{2in}
\toprule
\textbf{Name} & \textbf{Given} & \textbf{May Add} \\ 
\midrule
\endfirsthead
\multicolumn{3}{c}{\emph{Continued from Previous Page}}\\
\toprule
\textbf{Name} & \textbf{Given} & \textbf{May Add} \\ 
\midrule
\endhead
\bottomrule
\caption{Basic Rules of \GSD{}}\\[-.15in]
\multicolumn{3}{c}{\emph{Continued next Page}}\\
\endfoot
\bottomrule
\caption{Basic Rules of \GSD{}}\\%
\endlastfoot%
\label{GSD}%
\Rule{Assume} & & | $\CAPPHI$ \\
\Rule{Rep.} & $\CAPPHI$ & $\CAPPHI$ \\
\Rule{$\HORSESHOE$-Elim} & $\horseshoe{\CAPTHETA}{\CAPPSI}$, $\CAPTHETA$ & $\CAPPSI$ \\
\Rule{$\HORSESHOE$-Intro} &  | $\CAPTHETA$ &  \\
 &  | $\vdots$ &  \\
 &  | $\CAPPSI$ & $\horseshoe{\CAPTHETA}{\CAPPSI}$, Draw box\footnote{When using the rule $\HORSESHOE$-Intro, you must draw a box around all lines from the $\CAPTHETA$ to the $\CAPPSI$.  The line with $\CAPTHETA$ must be sanctioned by the rule \mention{Assume}.} \\
\Rule{$\!\WEDGE\!$-Elim} &{}$\conjunction{\CAPTHETA_1}{\conjunction{\CAPTHETA_2}{\conjunction{\ldots}{\CAPTHETA_{\integer{n}}}}}$&{}Any one of the conjuncts\\[-.25cm]
 & &{}i.e., $\CAPTHETA_{\integer{i}}$\\
\Rule{$\!\WEDGE\!$-Intro} & $\CAPTHETA_1$, $\CAPTHETA_2$, $\ldots$ $\CAPTHETA_{\integer{n}}$ & $\conjunction{\CAPTHETA_1}{\conjunction{\CAPTHETA_2}{\conjunction{\ldots}{\CAPTHETA_{\integer{n}}}}}$ \\
\Rule{$\VEE$-Elim} & $\disjunction{\CAPTHETA_1}{\disjunction{\CAPTHETA_2}{\disjunction{\ldots}{\CAPTHETA_{\integer{n}}}}}$, &  \\
 &  $\horseshoe{\CAPTHETA_1}{\CAPPSI}$,  &  \\
 &  $\horseshoe{\CAPTHETA_2}{\CAPPSI}$,  &  \\
 &  $\vdots$  &  \\
 &  $\horseshoe{\CAPTHETA_{\integer{n}}}{\CAPPSI}$ & $\CAPPSI$ \\
\Rule{$\VEE$-Intro} & $\CAPTHETA$ & $\disjunction{\CAPPSI_1}{\disjunction{\CAPPSI_2}{\disjunction{\ldots}{\CAPPSI_{\integer{n}}}}}$, \\[-.25cm]
 \nopagebreak
 &  & where $\CAPTHETA$ is $\CAPPSI_i$ for some $i$. \\
\Rule{$\NEGATION$-Intro} & $\horseshoe{\CAPTHETA}{\parconjunction{\CAPPSI}{\negation{\CAPPSI}}}$ & $\negation{\CAPTHETA}$ \\
\Rule{$\NEGATION$-Elim} & $\horseshoe{\negation{\CAPTHETA}}{\parconjunction{\CAPPSI}{\negation{\CAPPSI}}}$ & $\CAPTHETA$ \\
\Rule{$\TRIPLEBAR$-Intro} & $\horseshoe{\CAPTHETA}{\CAPPSI}$, $\horseshoe{\CAPPSI}{\CAPTHETA}$ & $\triplebar{\CAPTHETA}{\CAPPSI}$ \\
\Rule{$\TRIPLEBAR$-Elim} & $\triplebar{\CAPTHETA}{\CAPPSI}$, $\CAPPSI$ & $\CAPTHETA$ \\
\nopagebreak
\Rule{$\TRIPLEBAR$-Elim} & $\triplebar{\CAPTHETA}{\CAPPSI}$, $\CAPTHETA$ & $\CAPPSI$ \\
%\bottomrule
\end{longtable}
\index{derivation!rule!basic}\index{basic rule}
\index{derivation!rule!introduction}\index{introduction rule}
\index{derivation!rule!elimination}\index{elimination rule}
%\end{center}
%\caption{Basic Rules of \GSD{}}
%\label{GSD}
%\end{table}

Before doing some examples, two notes about the rules are in order. 
First, to use the \Rule{Assumption} rule, you must draw a new vertical line to the left of your assumption (and to the right of any other vertical lines from previous steps). 
The vertical lines are assumption lines; assumption lines are an accounting device to keep track of what you have assumed as opposed to what you have derived (see example \pmvref{secondexample}). You may assume any sentence you like, but generally these assumptions (usually all of them) will need to be discharged.  Vertical assumption lines are like your credit card balance.  As with a credit card, managing assumptions wisely is a skill to be acquired.

How do you get rid of an assumption? And how do you keep track of them when you are adding some and eliminating others? We have a second accounting device: boxes. When an assumption is discharged, the relevant assumption line is turned into a box containing everything derived from the originating assumption. Writing boxes can sometimes seem tedious, but we have found that using them significantly reduces student error.

Assumptions are discharged using the \Rule{$\HORSESHOE$-Intro} rule.  For \Rule{$\HORSESHOE$-Intro} you need a (part of a) derivation that begins with an unboxed assumption \mbox{| $\CAPTHETA$} and ends with \mbox{| $\CAPPSI$}. 
After you've added $\horseshoe{\CAPTHETA}{\CAPPSI}$ to your derivation, you \emph{must} box off the part of the derivation that began with assumption | $\CAPTHETA$ and ended with | $\CAPPSI$.  That is, if we have the following:

\begin{gproof}[\label{conditionalintro}]
	\galineNC{1}{$\CAPTHETA$}{\Rule{Assume}}
	\galineNC{2}{}{}
	\galineNC{3}{$\qquad\vdots$}{}
	\galineNC{4}{}{}
	\galineNC{5}{$\CAPPSI$}{}
\end{gproof}

\noindent{}we may then use the rule \Rule{$\HORSESHOE$-Intro} to get:

\begin{gproof}[\label{conditionalintroclosed}]
	\gaproof{
		\galine{1}{$\CAPTHETA$}{\Rule{Assume}}
		\galine{2}{}{}
		\galine{3}{$\qquad\vdots$}{}
		\galine{4}{}{}
		\galine{5}{$\CAPPSI$}{}
	}
	\gline{6}{$\horseshoe{\CAPTHETA}{\CAPPSI}$}{\Rule{$\HORSESHOE$-Intro}, 1--5}
\end{gproof}

\noindent{}The point in boxing off that part of the proof is to indicate visually that those sentences in the box can't be used anymore. 

\subsection{Writing Derivations}

We've said, in rough terms, that a derivation is a finite series of sentences each of which is either an assumption, or there is a rule which, given the previous sentences in the derivation, sanctions (or permits) us to write down that sentence. 
We'll add more structure to derivations for the sake of visual clarity and organization. 
First, we'll write the sentences vertically with earlier sentences above later ones so that each sentence sits on its own row in the derivation. 
Each row is to be numbered with consecutive positive integers, starting with $1$.
In a column to the right on each row we'll put either \Rule{Assume}, if the sentence is an assumption, or the name of the rule plus the numbers of the previous rows to which the rule was applied in order to get the sentence on the current row. 
Lastly, in a column between the row numbers and the sentences we'll keep track of the assumptions by running vertical lines in the column.

As an example, consider how we'd derive $\conjunction{\Cl}{\Bl}$ from $\conjunction{\Bl}{\Cl}$. This derivation would begin:
\begin{gproof}[\label{onelinederivation}]
\galineNC{1}{$\conjunction{\Bl}{\Cl}$}{\Rule{Assume}}
\end{gproof}
\noindent{}Note that \Rule{$\WEDGE$-Elim} allows us to write down one of the conjuncts of a conjunction we already have. 
The conjuncts of $\conjunction{\Bl}{\Cl}$ are $\Bl$ and $\Cl$. 
Accordingly, we may write down $\Bl$ and $\Cl$ on new lines. 
We continue:
\begin{gproof}[\label{threelinederivation}]
\galineNC{1}{$\conjunction{\Bl}{\Cl}$}{\Rule{Assume}}
\galineNC{2}{$\Bl$}{\Rule{$\WEDGE$-Elim}, 1}
\galineNC{3}{$\Cl$}{\Rule{$\WEDGE$-Elim}, 1}
\end{gproof}
\noindent{}Finally, using \Rule{$\WEDGE\!$-Intro} we put the sentences from lines 2 and 3 together in a conjunction:
\begin{gproof}[\label{simpleconjunction}]
\galineNC{1}{$\conjunction{\Bl}{\Cl}$}{\Rule{Assume}}
\galineNC{2}{$\Bl$}{\Rule{$\WEDGE$-Elim}, 1}
\galineNC{3}{$\Cl$}{\Rule{$\WEDGE$-Elim}, 1}
\galineNC{4}{$\conjunction{\Cl}{\Bl}$}{\Rule{$\WEDGE\!$-Intro}, 2,3}
\end{gproof}
\noindent{}This is a four-line derivation of $\conjunction{\Cl}{\Bl}$ from $\conjunction{\Bl}{\Cl}$. Notice that steps 2 and 3 could have been done in opposite order. In many cases the order of sentences in a derivation doesn’t matter, but in other cases it is crucial. Learning the difference is an important skill.

Just as we used the double turnstile to represent when one sentence (or a set of sentences) entailed another, we use what's called the \emph{single turnstile}, \mention{\:$\sststile{}{}\:$}, to represent when one sentence is derivable from another, or from another (finite) set of sentences. 
The four-line derivation shows that $\conjunction{\Bl}{\Cl}\sststile{}{}\conjunction{\Cl}{\Bl}$. 
It's also worth noting that each of the two partially completed pieces of the derivation just given are derivations themselves. 
The first (\ref{onelinederivation}) is a one-line derivation that shows that $\conjunction{\Bl}{\Cl}\sststile{}{}\conjunction{\Bl}{\Cl}$, while the second (\ref{onelinederivation}) is a three-line derivation that shows that $\conjunction{\Bl}{\Cl}\sststile{}{}\Cl$.

Next, as an example of how assumption lines stack up in a derivation, consider the following derivation of $\Dl$ from $\conjunction{\Al}{\Bl}$ and $\horseshoe{\Bl}{\parconjunction{\Cl}{\Dl}}$.
\begin{gproof}[\label{secondexample}]
\galineNC{1}{$\conjunction{\Al}{\Bl}$}{\Rule{Assume}}
\gaalineNC{2}{$\horseshoe{\Bl}{\parconjunction{\Cl}{\Dl}}$}{\Rule{Assume}}
\gaalineNC{3}{$\Bl$}{\Rule{$\WEDGE$-Elim}, 1}
\gaalineNC{4}{$\conjunction{\Cl}{\Dl}$}{\Rule{$\HORSESHOE$-Elim}, 2,3}
\gaalineNC{5}{$\Dl$}{\Rule{$\WEDGE$-Elim}, 4}
\end{gproof}
\noindent{}As is shown, any time a new assumption line is added it must be placed to the right of the previous (unboxed) assumption lines. (We discuss this example a little more in the next section, while discussing the basic top-down and bottom-up strategies for $\HORSESHOE$.)

The last major point about the mechanics of writing a derivation is how the assumptions are closed, or discharged, when \Rule{$\HORSESHOE$-Intro} is used. This rule says that if you have a sentence $\CAPPSI$ on a line as an assumption, and you've worked your way down to a line with sentence $\CAPTHETA$, then you can ``close off'' the assumption by drawing a box around that part of the proof and writing $\horseshoe{\CAPPSI}{\CAPTHETA}$ on the next line. We can demonstrate this by applying the rule to close off the assumptions in the previous two proofs:
\begin{gproof}[\label{simpleconjunctionclosed}]
\gaproof{
\galine{1}{$\conjunction{\Bl}{\Cl}$}{\Rule{Assume}}
\galine{2}{$\Bl$}{\Rule{$\WEDGE$-Elim}, 1}
\galine{3}{$\Cl$}{\Rule{$\WEDGE$-Elim}, 1}
\galine{4}{$\conjunction{\Cl}{\Bl}$}{\Rule{$\WEDGE\!$-Intro}, 2,3}
}
\gline{5}{$\horseshoe{\parconjunction{\Bl}{\Cl}}{\parconjunction{\Cl}{\Bl}}$}{\Rule{$\HORSESHOE$-Intro}, 1--4}
\end{gproof}
\begin{gproof}[\label{secondexamplefinished}]
\gaproof{
\galine{1}{$\conjunction{\Al}{\Bl}$}{\Rule{Assume}}
\gaaproof{
\gaaline{2}{$\horseshoe{\Bl}{\parconjunction{\Cl}{\Dl}}$}{\Rule{Assume}}
\gaaline{3}{$\Bl$}{\Rule{$\WEDGE$-Elim}, 1}
\gaaline{4}{$\conjunction{\Cl}{\Dl}$}{\Rule{$\HORSESHOE$-Elim}, 2,3}
\gaaline{5}{$\Dl$}{\Rule{$\WEDGE$-Elim}, 4}
}
\galine{6}{$\horseshoe{\parhorseshoe{\Bl}{\bparconjunction{\Cl}{\Dl}}}{\Dl}$}{\Rule{$\HORSESHOE$-Intro}, 2--5}
}
\gline{7}{$\horseshoe{\parconjunction{\Al}{\Bl}}{\cparhorseshoe{\parhorseshoe{\Bl}{\bparconjunction{\Cl}{\Dl}}}{\Dl}}$}{\Rule{$\HORSESHOE$-Intro}, 1--6}
\end{gproof}
\noindent{}In these two cases, since we've boxed off the assumptions they are no longer derivations of some sentence from another (or others). The first is a derivation of  $\horseshoe{\parconjunction{\Bl}{\Cl}}{\parconjunction{\Cl}{\Bl}}$, while the second is a derivation of $\horseshoe{\parconjunction{\Al}{\Bl}}{\cparhorseshoe{\parhorseshoe{\Bl}{\bparconjunction{\Cl}{\Dl}}}{\Dl}}$. When we have a derivation from no assumptions (or, rather, with all assumptions boxed), we represent that with a single turnstile with no formulas on the left. So these two derivations show, respectively, that $\sststile{}{}\horseshoe{\parconjunction{\Bl}{\Cl}}{\parconjunction{\Cl}{\Bl}}$ and $\sststile{}{}\horseshoe{\parconjunction{\Al}{\Bl}}{\cparhorseshoe{\parhorseshoe{\Bl}{\bparconjunction{\Cl}{\Dl}}}{\Dl}}$.\footnote{If  $\sststile{}{}\CAPPHI$, it's often said that $\CAPPHI$ is a theorem of the derivation system. ``Theorem'' is used ambiguously between the derivation of an \GSL  (or \GQL) sentence from the empty set of assumptions and things we prove in MathEnglish.  We will follow that standard practice.}

It is important to note that once we have closed off an assumption we can no longer use the sentences that we've boxed off. 
We can't apply rules to them anymore; they are no longer \mention{Given}, to use the term in the Basic Rules chart.
The reason we can't use sentences inside boxes is because if we could, then \GSD{} would not be sound.
That is, we would be able to derive a sentence $\CAPPHI$ from others $\CAPPSI_1,\ldots,\CAPPSI_{\integer{n}}$ but $\CAPPSI_1,\ldots,\CAPPSI_{\integer{n}}\sdtstile{}{}\CAPPHI$ would not hold.
To see this, consider the following example.
\begin{gproof}
\galineNC{1}{$\horseshoe{\Al}{\Bl}$}{\Rule{Assume}}
\gaaproof{
\gaalineNCS{2}{$\Al$}{\Rule{Assume}}
\gaalineNCS{3}{$\Bl$}{\Rule{$\HORSESHOE$-Elim}, 1,2}
}
\galineNC{4}{$\horseshoe{\Al}{\Bl}$}{\Rule{$\HORSESHOE$-Intro}, 2--3}
\galineNC{5}{$\Bl$}{\Rule{$\HORSESHOE$-Elim}, 2,4}
\end{gproof}
\noindent{}In line 5 we go back into the box to use $\Al$ from line 2 to get $\Bl$ from line 4 using \Rule{$\HORSESHOE$-Intro}. 
But it should be easy to see that $\horseshoe{\Al}{\Bl}\sdtstile{}{}\Bl$ does not hold. 
(Consider a model $\IntA$ such that $\IntA(\Al)=\FalseB$ and $\IntA(\Bl)=\FalseB$.) 
%Sometimes there are sentences inside a closed box that we could use without running into this problem, but that doesn't mean we can use that sentence. 
%The rule is simply that if a sentence appears inside a closed box, then you cannot use it anymore in the proof. 

\subsection{*The Recursive Definition of a Derivation}\label{RecDefOfDerv}
Now that we've done some examples, we give an explicit definition of a derivation. 
The definition of a derivation is recursive. 
The explicit definition we give here is mainly for the sake of proving soundness in the next chapter. 
Although the actual definition is complicated, the basic idea is straightforward: 
A single \GSL{} sentence that's an assumption is a derivation (e.g., derivation \ref{onelinederivation}), and any finite sequence of \GSL{} sentences every one of which (after the first) is either an assumption or sanctioned by some rule of \GSD{} is a derivation. 
We have to complicate this to handle the rule \Rule{$\HORSESHOE$-Intro} and \Rule{Assumption}, neither of which quite work like the other rules.
To understand these rules we must first define what an \emph{open assumption} is. An assumption is open \Iff it appears on an assumption line and is not in a box.
% the extra features of derivations: 
%the numbers we write on the left to keep track of the sentences, the vertical lines we use to track the scope of the assumptions, and the rules we note on the right along with the numbers of the lines they were applied to. 
%\begin{majorILnc}{\LnpDC{Derivation Line}}
%A \nidf{derivation} \underdf{line}{derivation} consists of:
%\begin{cenumerate}
%\item An integer $\integer{n}$, called the line number.
%\item Some number of vertical lines \mention{|} indicating the number of assumptions the line falls in the scope of.
%\item A sentence of \GSL{}.
%\item The name of a rule and line numbers to which the rule was applied.
%\end{cenumerate}
%\end{majorILnc}
\begin{majorILnc}{\LnpDC{Recursive definition of Derivation}} The following recursive clauses fix which finite sequences of derivation lines are \nidf{derivations}\index{derivation|textbf} in \GSD{}:
\begin{description}
\item[Base Clause:] For any \GSL{} sentence $\CAPPHI$, the following single derivation line (i.e., sequence of derivation lines of length 1) is a derivation: 
\begin{gproofnn}
\galineNC{1}{$\CAPPHI$\qquad}{\Rule{Assume}}
\end{gproofnn}
\item[Generating Clause:] \hfill{}
\begin{description}
\item[Case 1:] If you have some $\integer{n}$-line derivation with $\integer{k}$ assumptions still open at line $\integer{n}$, and in which the sentence $\CAPPSI$ on $\integer{n}$ is sanctioned by rule $\Rule{R}$ when applied to lines $\integer{j}_1,\ldots,\integer{j}_k$, 
\begin{gproofnn}
\gline{1}{}{}
\glinend{}{}{}
\glinend{}{$\qquad\vdots$}{}
\glinend{}{}{}
\gline{$\integer{n}$}{$\CAPPSI$}{\Rule{R}, $\integer{j}_1,\ldots,\integer{j}_k$}
\end{gproofnn}
then the sequence of derivation lines you get by adding a new line $\integer{n}+1$ with $\integer{k}$ open assumptions and the sentence $\CAPTHETA$ is a derivation,
\begin{gproofnn}
\gline{1}{}{}
\glinend{}{}{}
\glinend{}{$\qquad\vdots$}{}
\glinend{}{}{}
\gline{$\integer{n}$}{$\CAPPSI$}{\Rule{R}, $\integer{j}_1,\ldots,\integer{j}_k$}
\gline{$\integer{n}+1$}{$\CAPTHETA$}{\Rule{R$'$}, $\integer{h}_1,\ldots,\integer{h}_l$}
\end{gproofnn}
so long as $\CAPTHETA$ is sanctioned by some rule \Rule{R$'$} (other than \Rule{$\HORSESHOE$-Intro}) when applied to previous lines $\integer{h}_1,\ldots,\integer{h}_l$ already in the derivation. 

\item[Case 2:] If you have some $\integer{n}$-line derivation with $\integer{k}$ assumptions still open at line $\integer{n}$, and in which the sentence $\CAPPSI$ on $\integer{n}$ is sanctioned by rule $\Rule{R}$ when applied to lines $\integer{j}_1,\ldots,\integer{j}_k$, 
\begin{gproofnn}
\gline{1}{}{}
\glinend{}{}{}
\glinend{}{$\qquad\vdots$}{}
\glinend{}{}{}
\galineNC{$\integer{n}$}{$\CAPPSI$\hfill{}}{\Rule{R}, $\integer{j}_1,\ldots,\integer{j}_k$}
\end{gproofnn}
then, for any sentence $\CAPTHETA$, the sequence of derivation lines you get by adding a new line $\integer{n}+1$ with $\integer{k}+1$ open assumptions and $\CAPTHETA$ sanctioned by \Rule{Assumption} is a derivation,
\begin{gproofnn}
\gline{1}{}{}
\glinend{}{}{}
\glinend{}{$\qquad\vdots$}{}
\glinend{}{}{}
\galineNC{$\integer{n}$}{$\CAPPSI$}{\Rule{R}, $\integer{j}_1,\ldots,\integer{j}_k$}
\gaalineNC{$\integer{n}+1$}{$\CAPTHETA$\qquad}{\Rule{Assume}}
\end{gproofnn}

\item[Case 3:] If you have some $\integer{n}$-line derivation with $\integer{k}$ assumptions still open at line $\integer{n}$, and in which the sentence $\CAPPSI$ on $\integer{n}$ is sanctioned by rule $\Rule{R}$ when applied to lines $\integer{j}_1,\ldots,\integer{j}_k$, and in which the $\integer{k}$th assumption was opened on line $\integer{m}$, 
\begin{gproofnn}
\gline{1}{}{}
\glinend{}{}{}
\glinend{}{$\qquad\vdots$}{}
\glinend{}{}{}
\galineNC{$\integer{m}$}{$\CAPPHI$}{\Rule{Assume}}
\galineNCnd{}{}{}
\galineNCnd{}{$\qquad\vdots$}{}
\galineNCnd{}{}{}
\galineNC{$\integer{n}$}{$\CAPPSI$}{\Rule{R}, $\integer{j}_1,\ldots,\integer{j}_k$}
\end{gproofnn}
then the sequence of derivation lines you get by adding a new line $\integer{n}+1$ with $\integer{k}-1$ open assumptions and the sentence $\horseshoe{\CAPPHI}{\CAPPSI}$ is a derivation,
\begin{gproofnn}
\gline{1}{}{}
\glinend{}{}{}
\glinend{}{$\qquad\vdots$}{}
\glinend{}{}{}
\gaproof{
\galine{$\integer{m}$}{$\CAPPHI$}{\Rule{Assume}}
\galinend{}{}{}
\galinend{}{$\quad\vdots$}{}
\galinend{}{}{}
\galine{$\integer{n}$}{$\CAPPSI$}{\Rule{R}, $\integer{j}_1,\ldots,\integer{j}_k$}
}
\gline{$\integer{n}+1$}{$\horseshoe{\CAPPHI}{\CAPPSI}$}{\Rule{$\HORSESHOE$-Intro}, $\integer{m}$--$\integer{n}$}
\end{gproofnn}
so long as you close the assumption opened on line $\integer{m}$ by drawing a box around lines $\integer{m}$--$\integer{n}$ and write down \Rule{$\HORSESHOE$-Intro}, $\integer{m}$--$\integer{n}$ as the rule which sanctions line $\integer{n}+1$. 
\end{description}

\item[Closure Clause:] Nothing else is a derivation of \GSL{}.
\end{description}
\end{majorILnc}
\noindent{}Note that this definition of a derivation makes use of something, sanctioning, which we haven't yet defined explicitly. 
Definition \ref{RuleSanctioning} will make this explicit too. 
Also note that in the generating clauses we specify that we start with a derivation with $\integer{k}$ open assumptions on the last line. 
In cases 1 and 2 the schematic drawings given don't explicitly depict the $\integer{k}$ vertical lines that should be running between the line numbers and the sentences, but the drawings are not intended to suggest that you can write derivations without the running vertical lines which track open assumptions. 
The schematic drawing in case 3 similarly only depicts the last vertical assumption line (the one for the assumption opened on line $\integer{m}$), but that's not to suggest the others aren't there or can be left off.   

\subsection{Restrictions on Applying Rules}\label{Restrictions on Applying Rules}
Above we noted that every line of a derivation must either be an assumption, or there must be a rule which, applied to some previous unboxed sentences in the derivation, \emph{sanctions} us to write down that sentence.
(Extending a derivation with the introduction rule for $\HORSESHOE$, \Rule{$\HORSESHOE$-Intro}, as in case 3 of definition \ref{Recursive definition of Derivation} is a special case, but still we say that the rule sanctions writing down the new sentence.) 
It's very important to be clear on just when a rule sanctions\index{derivation!rule}\index{sanctions} (or we might say \niidf{licenses}\index{licenses|see{sanctions}}) writing down a sentence. 
The restriction, which we tacitly followed in the examples given above, is that a rule can be applied only if the connectives mentioned in the rule are the main connectives of the sentences to which that rule is being applied. That is, a rule can only be applied to whole sentences on a line---it can't be applied to subsentences on a line.

For example, we can only apply \Rule{$\HORSESHOE$-Elim} on two lines of a derivation if the sentence on one of the lines is a conditional $\horseshoe{\CAPPHI}{\CAPTHETA}$ and the sentence on the other line is $\CAPPHI$. 
If $\horseshoe{\CAPPHI}{\CAPTHETA}$ is the sentence on one line and $\CAPPHI$ is merely contained as a subsentence in the sentence on the other (say the sentence has the form $\conjunction{\CAPPHI}{\CAPPSI}$), then we cannot apply \Rule{$\HORSESHOE$-Elim} to those lines. 
Likewise, if a line has a sentence $\CAPPHI$ and the conditional $\horseshoe{\CAPPHI}{\CAPTHETA}$ is merely contained as a subsentence in the sentence on another line (say the sentence has the form $\disjunction{\bparhorseshoe{\CAPPHI}{\CAPTHETA}}{\CAPPSI}$), then we cannot apply \Rule{$\HORSESHOE$-Elim} to those lines.

For a more concrete example, consider derivation \pmvref{secondexamplefinished}. 
Even though $\Bl$ appears on line 1 (as the conjunct of $\conjunction{\Al}{\Bl}$), we cannot apply \Rule{$\HORSESHOE$-Elim} to lines 1 and 2 to get $\conjunction{\Cl}{\Dl}$. 
The fact that we can easily get $\Bl$ on its own line through \Rule{$\WEDGE$-Elim} doesn't matter. 
The rule \Rule{$\HORSESHOE$-Elim} will only sanction writing $\conjunction{\Cl}{\Dl}$ on a line, given $\horseshoe{\Bl}{\parconjunction{\Cl}{\Dl}}$ on line 2, if we have another line with the \CAPS{lhs} of the conditional by itself. 
In just the same way we cannot apply \Rule{$\WEDGE$-Elim} to line 2 of derivation \ref{secondexamplefinished} to get $\Cl$ or $\Dl$, since the conjunction in line 2 is not the main connective. 
Instead, it's the main connective of the \CAPS{rhs} subsentence of $\horseshoe{\Bl}{\parconjunction{\Cl}{\Dl}}$. 
Again it doesn't matter that we can get the conjunct by itself using \Rule{$\HORSESHOE$-Elim} (after using \Rule{$\WEDGE$-Elim} on line 1), the rule \Rule{$\WEDGE$-Elim} cannot be applied to a line unless the sentence on that line is a conjunction. 

One reason for this restriction is because without it many applications of the rules would not be \emph{truth-preserving}.  One of the goals of \GSD{} is that our derivation system ``mirrors'' the results we could prove in \GSL{}.  In other words, we want something to be derivable in \GSD{} \Iff we can prove the same result semantically in \GSL{}.

However, we are not yet ready to show that our derivation system has this property.  We will not establish this result until next chapter.  For now, we note merely that a new line is derivable solely in virtue of the \emph{form} of previous lines, or in virtue of the rules for introducing a new line.  Our derivation system is nothing more than a formal model, with \emph{stipulated} restrictions regarding how to apply the rules we define.  It does not, in and of itself, say anything about what is true or false in \GSL{}.

\subsection{*Decidability}\label{Section:Intro to Decidability}
There are multiple algorithms to follow for applying the rules which, if there does exist a derivation, will halt when the last line written down is the sentence to be derived. 
For \GSD{} the algorithms are intuitive and straightforward (see Sec. \pmvref{Section:Completeness for GSD}), while for \GQD{} (the basic derivation system for \GQL{} we'll introduce in Sec. \pmvref{Section GQD}) they need to become much more complicated. 
In the next chapter (Sec. \pmvref{The Method Section}) we will give one such algorithm for \GQD{}, which we call The Method.\index{method, the} 
It will be crucial for showing some of the important connections between derivations and entailment mentioned in section \ref{Derivation Preliminaries}, including completeness. 

But although these algorithms are guaranteed to end in a derivation if the sentence can be derived, there are at least two reasons why you don't want to do most of your derivations using them. 
First, the derivations produced by them tend to be much longer and more complicated than is necessary. 
You will almost always be able to come up with a much shorter and more direct proof on your own. 
Second, at least for \GQD{} (but not \GSD{}), although \emph{if} there is a derivation the algorithms will ``find it'', if there is \emph{not} a derivation then the algorithms may never ``find out''. 
That is, if there is not a derivation then the algorithms (any one you pick) do essentially one of two things: either they halt in a way that indicates there is no derivation, or they never halt. 
If you happen to be working on a problem in the latter case and you're only following the algorithm, then you'll never find out whether there is a derivation. 
(If a derivation system, like \GQD{}, has this feature, then it's said to be \idf{undecidable}. 
If there is an algorithm that always halts either in a derivation or with an indication that there's no derivation (as in the case of \GSD{}), then the system is said to be \idf{decidable} and the algorithm is said to be a \niidf{decision procedure}\index{decision procedure}.
%So, \GQD{} is undecidable, while \GSD{} is decidable.
Note that if we restrict \GQL{} to just 1-place predicates, then \GQD{} is decidable.
See section \pmvref{Decidability and Churchs Theorem} for more details and discussion.)
At any point in applying the algorithm you'll have neither produced a derivation or reached any indication that there is no derivation.

Enderton \citeyearpar{Enderton2010} provides a general, contemporary introduction to computability and decision procedures. Kleene \citeyearpar[ch.~5]{Kleene1967} provides a lucid and concise discussion within roughly the framework devolved here.

\subsection{Some Strategies}\label{Sec:Some Strategies}
We don't want to use these sorts of algorithms to find derivations, if we can avoid it. 
But there are general strategies we will use. 
For each logical connective there are two types of strategies: those for what to do if you already have sentences with that as their main connective, and those for what to do if you want to get a sentence with that as its main connective. 
We'll call the first top-down strategies and the second bottom-up strategies.
In many cases, doing a derivation is like planning a plane trip. The top-down method is like figuring out the nearest convenient airport from your current location, and the bottom-up method is like figuring out which airport is convenient for getting to your destination. In derivations, sometimes you have to go through intermediate sentences (as with intermediate cities in travel).

We will begin with some basic top-down and bottom-up strategies for each connective, adding more later in section \ref{Sec:Shortcut Rule Strategies} when we add shortcut rules.

\subsubsection*{Conjunction} 
We start with the basic top-down and bottom-up strategies for conjunction. They are the most straightforward.
\begin{description}
\item[$\WEDGE\!$ Top-down:] If you have a sentence of the form $\conjunction{\conjunction{\conjunction{\CAPPHI_1}{\CAPPHI_2}}{\ldots}}{\CAPPHI_n}$, then break it apart using \Rule{$\WEDGE$-Elim} to get each of the conjuncts $\CAPPHI_1$ through $\CAPPSI_n$, each on a new line.
\item[$\WEDGE\!$ Bottom-up:] If you want to get a sentence of the form $\conjunction{\conjunction{\conjunction{\CAPPHI_1}{\CAPPHI_2}}{\ldots}}{\CAPPHI_n}$, then derive each of $\CAPPHI_1$ through $\CAPPHI_n$ individually and use \Rule{$\WEDGE$-Intro} to derive it from them. 
\end{description} 
Both strategies are exemplified in example derivation \pmvref{simpleconjunction}. 
There we wanted to derive the sentence $\conjunction{\Cl}{\Bl}$, so in line with the bottom-up strategy for $\!\WEDGE\!$ we first derived both $\Cl$ and $\Bl$ and then used \Rule{$\WEDGE$-Intro} to derive $\conjunction{\Cl}{\Bl}$. 
In line with the top-down strategy, we took our assumption $\conjunction{\Bl}{\Cl}$ and broke it apart using \Rule{$\WEDGE$-Elim} (which got us the sentences, $\Cl$ and $\Bl$, we were looking to derive). 

\subsubsection*{Conditionals}
The basic strategies for conditionals are also straightforward and have already been exemplified. 
\begin{description}
\item[$\HORSESHOE$ Top-down:] If you have a sentence of the form $\horseshoe{\CAPPHI}{\CAPPSI}$, then first derive the \CAPS{lhs} $\CAPPHI$ and then break it apart using \Rule{$\HORSESHOE$-Elim} to get the \CAPS{rhs} $\CAPPSI$ on a new line.
\item[$\HORSESHOE$ Bottom-up:] If you want to get a sentence of the form $\horseshoe{\CAPPHI}{\CAPPSI}$, then assume the \CAPS{lhs} $\CAPPHI$, derive the \CAPS{rhs} $\CAPPSI$, and then use \Rule{$\HORSESHOE$-Intro} to write the conditional on the next line.
\end{description} 
In derivation \pmvref{secondexample}, we wanted to derive $\Dl$. 
We saw that we had a conditional, $\horseshoe{\Bl}{\parconjunction{\Cl}{\Dl}}$. 
In line with the top-down strategy, we derived its \CAPS{lhs} $\Bl$ (in the process using the top-down strategy for $\!\WEDGE\!$), then used \Rule{$\HORSESHOE$-Elim} to get the \CAPS{rhs} $\parconjunction{\Cl}{\Dl}$. 
We wanted $\parconjunction{\Cl}{\Dl}$, of course, because from it we could use \Rule{$\WEDGE$-Elim} to get $\Dl$. 
In derivation \pmvref{secondexamplefinished}, we wanted to derive $\horseshoe{\parconjunction{\Al}{\Bl}}{\cparhorseshoe{\parhorseshoe{\Bl}{\bparconjunction{\Cl}{\Dl}}}{\Dl}}$. 
In line with the bottom-up strategy, we assumed the \CAPS{lhs} $\parconjunction{\Al}{\Bl}$, derived the \CAPS{rhs} $\cparhorseshoe{\parhorseshoe{\Bl}{\bparconjunction{\Cl}{\Dl}}}{\Dl}$, and then used \Rule{$\HORSESHOE$-Intro} to write the conditional on the next line.

\subsubsection*{Biconditionals}
The basic strategies for biconditionals are similar to those for conditionals, as one might expect.
\begin{description}
\item[$\TRIPLEBAR$ Top-down:] If you have a sentence of the form $\triplebar{\CAPPHI}{\CAPPSI}$, then either 
\begin{enumerate}
\item first derive the \CAPS{lhs} $\CAPPHI$ and then break it apart using \Rule{$\TRIPLEBAR$-Elim} to get the \CAPS{rhs} $\CAPPSI$ on a new line,
\item first derive the \CAPS{rhs} $\CAPPSI$ and then break it apart using \Rule{$\TRIPLEBAR$-Elim} to get the \CAPS{lhs} $\CAPPHI$ on a new line,
\item or do both.
\end{enumerate}
\item[$\TRIPLEBAR$ Bottom-up:] If you want to get a sentence of the form $\triplebar{\CAPPHI}{\CAPPSI}$, then first derive both $\horseshoe{\CAPPHI}{\CAPPSI}$ and $\horseshoe{\CAPPSI}{\CAPPHI}$ and then use \Rule{$\TRIPLEBAR$-Intro} to write the biconditional on the next line.
\end{description}

\subsubsection*{Negations}
In the case of negations, we don't have a basic top-down strategy, only a basic bottom-up.  Later in this chapter we will develop shortcut rules which allow us to provide top-down strategies for negation.
\begin{description}
\item[$\NEGATION$ Bottom-up:] If you want to get a sentence of the form $\negation{\CAPPHI}$, then first assume $\CAPPHI$, derive a contradiction $\conjunction{\CAPPSI}{\negation{\CAPPSI}}$, and then in two separate steps use \Rule{$\HORSESHOE$-Intro} and \Rule{$\NEGATION$-Intro} to write the negation on the next line.
\end{description}
For example, say we want to derive the sentence $\negation{\parconjunction{\bparhorseshoe{\Al}{\negation{\Bl}}}{\bparconjunction{\Al}{\Bl}}}$. In line with the basic bottom-up strategy for $\NEGATION$, we first assume $\parconjunction{\bparhorseshoe{\Al}{\negation{\Bl}}}{\bparconjunction{\Al}{\Bl}}$ and try to derive a contradiction:
\begin{gproof}
\galineNC{1}{$\parconjunction{\bparhorseshoe{\Al}{\negation{\Bl}}}{\bparconjunction{\Al}{\Bl}}$}{\Rule{Assume}}
\galineNCnd{}{}{}
\galineNCnd{}{$\qquad\vdots$}{}
\galineNCnd{}{}{}
\galineNC{$\integer{n}$}{$\conjunction{\CAPPSI}{\negation{\CAPPSI}}$}{}
\end{gproof}
\noindent{}It should be clear that we can derive $\conjunction{\Bl}{\negation{\Bl}}$, so that will be our goal:
\begin{gproof}
\galineNC{1}{$\parconjunction{\bparhorseshoe{\Al}{\negation{\Bl}}}{\bparconjunction{\Al}{\Bl}}$}{\Rule{Assume}}
\galineNCnd{}{}{}
\galineNCnd{}{$\qquad\vdots$}{}
\galineNCnd{}{}{}
\galineNC{$\integer{n}$}{$\conjunction{\Bl}{\negation{\Bl}}$}{}
\end{gproof}
The bottom-up strategy for $\!\WEDGE\!$ says to get this we should derive both $\Bl$ and $\negation{\Bl}$.
\begin{gproof}
\galineNC{1}{$\parconjunction{\bparhorseshoe{\Al}{\negation{\Bl}}}{\bparconjunction{\Al}{\Bl}}$}{\Rule{Assume}}
\galineNCnd{}{}{}
\galineNCnd{}{$\qquad\vdots$}{}
\galineNCnd{}{}{}
\galineNC{$\integer{n}-2$}{$\Bl$}{}
\galineNC{$\integer{n}-1$}{$\negation{\Bl}$}{}
\galineNC{$\integer{n}$}{$\conjunction{\Bl}{\negation{\Bl}}$}{\Rule{$\WEDGE$-Intro}, $\integer{n}-$1,$\integer{n}-$2}
\end{gproof}
\noindent{}The bottom-down strategy for $\!\WEDGE\!$ is our only option at this point, so we break line 1 apart using \Rule{$\WEDGE$-Elim}.
\begin{gproof}
\galineNC{1}{$\parconjunction{\bparhorseshoe{\Al}{\negation{\Bl}}}{\bparconjunction{\Al}{\Bl}}$}{\Rule{Assume}}
\galineNC{2}{$\bparhorseshoe{\Al}{\negation{\Bl}}$}{\Rule{$\WEDGE$-Elim}, 1}
\galineNC{3}{$\bparconjunction{\Al}{\Bl}$}{\Rule{$\WEDGE$-Elim}, 1}
\galineNCnd{}{}{}
\galineNCnd{}{$\qquad\vdots$}{}
\galineNCnd{}{}{}
\galineNC{$\integer{n}-2$}{$\Bl$}{}
\galineNC{$\integer{n}-1$}{$\negation{\Bl}$}{}
\galineNC{$\integer{n}$}{$\conjunction{\Bl}{\negation{\Bl}}$}{\Rule{$\WEDGE$-Intro}, $\integer{n}-$1,$\integer{n}-$2}
\end{gproof}
\noindent{}Now we continue to work bottom-down, using \Rule{$\WEDGE$-Elim} to break apart line 3. Note that in this step we've partway joined up the top and bottom of the proof, since breaking apart line 3 gets us what we were calling line $\integer{n}-$2.
\begin{gproof}
\galineNC{1}{$\parconjunction{\bparhorseshoe{\Al}{\negation{\Bl}}}{\bparconjunction{\Al}{\Bl}}$}{\Rule{Assume}}
\galineNC{2}{$\bparhorseshoe{\Al}{\negation{\Bl}}$}{\Rule{$\WEDGE$-Elim}, 1}
\galineNC{3}{$\bparconjunction{\Al}{\Bl}$}{\Rule{$\WEDGE$-Elim}, 1}
\galineNC{4}{$\Al$}{\Rule{$\WEDGE$-Elim}, 3}
\galineNC{5}{$\Bl$}{\Rule{$\WEDGE$-Elim}, 3}
\galineNCnd{}{}{}
\galineNCnd{}{$\qquad\vdots$}{}
\galineNCnd{}{}{}
\galineNC{$\integer{n}-1$}{$\negation{\Bl}$}{}
\galineNC{$\integer{n}$}{$\conjunction{\Bl}{\negation{\Bl}}$}{\Rule{$\WEDGE$-Intro}, $\integer{n}-$1,$\integer{n}-$2}
\end{gproof}
\noindent{}Next we see that we can work bottom-down from lines 2 and 4, breaking the conditional on line 2 apart. In doing so we finish the proof, since the result of doing this is $\negation{\Bl}$, which is all that was left to get the contradiction. 
\begin{gproof}[\label{helpful1}]
\galineNC{1}{$\parconjunction{\bparhorseshoe{\Al}{\negation{\Bl}}}{\bparconjunction{\Al}{\Bl}}$}{\Rule{Assume}}
\galineNC{2}{$\bparhorseshoe{\Al}{\negation{\Bl}}$}{\Rule{$\WEDGE$-Elim}, 1}
\galineNC{3}{$\bparconjunction{\Al}{\Bl}$}{\Rule{$\WEDGE$-Elim}, 1}
\galineNC{4}{$\Al$}{\Rule{$\WEDGE$-Elim}, 3}
\galineNC{5}{$\Bl$}{\Rule{$\WEDGE$-Elim}, 3}
\galineNC{6}{$\negation{\Bl}$}{\Rule{$\HORSESHOE$-Elim}, 2,4}
\galineNC{7}{$\conjunction{\Bl}{\negation{\Bl}}$}{\Rule{$\WEDGE$-Intro}, 5,6}
\end{gproof}
Of course, we haven't yet derived $\negation{\!\parconjunction{\bparhorseshoe{\Al}{\negation{\Bl}}}{\bparconjunction{\Al}{\Bl}}}$, but we can now do so by discharging the assumption through \Rule{$\HORSESHOE$-Intro} and then applying \Rule{$\NEGATION$-Intro}.
\begin{gproof}
\gaproof{
\galine{1}{$\parconjunction{\bparhorseshoe{\Al}{\negation{\Bl}}}{\bparconjunction{\Al}{\Bl}}$}{\Rule{Assume}}
\galine{2}{$\bparhorseshoe{\Al}{\negation{\Bl}}$}{\Rule{$\WEDGE$-Elim}, 1}
\galine{3}{$\bparconjunction{\Al}{\Bl}$}{\Rule{$\WEDGE$-Elim}, 1}
\galine{4}{$\Al$}{\Rule{$\WEDGE$-Elim}, 3}
\galine{5}{$\Bl$}{\Rule{$\WEDGE$-Elim}, 3}
\galine{6}{$\negation{\Bl}$}{\Rule{$\HORSESHOE$-Elim}, 2,4}
\galine{7}{$\conjunction{\Bl}{\negation{\Bl}}$}{\Rule{$\WEDGE$-Intro}, 5,6}
}
\gline{8}{$\horseshoe{\parconjunction{\bparhorseshoe{\Al}{\negation{\Bl}}}{\bparconjunction{\Al}{\Bl}}}{\cparconjunction{\Bl}{\negation{\Bl}}}$}{$\HORSESHOE$-Intro, 1--7}
\gline{9}{$\negation{\parconjunction{\bparhorseshoe{\Al}{\negation{\Bl}}}{\bparconjunction{\Al}{\Bl}}}$}{\Rule{$\NEGATION$-Intro}, 8}
\end{gproof}

\subsubsection*{Disjunctions}
Our last pair of strategies is for disjunctions. As with conjunctions, we give the strategies for the case where there are only two disjuncts. Generalizing the strategies for disjunctions with more than two disjuncts is left to the reader.
\begin{description}
\item[$\VEE$ Top-down:] If you have a sentence of the form $\disjunction{\CAPPHI}{\CAPPSI}$ and you want to derive a sentence $\CAPTHETA$, first derive the conditionals $\horseshoe{\CAPPHI}{\CAPTHETA}$ and $\horseshoe{\CAPPSI}{\CAPTHETA}$, and then use \Rule{$\VEE$-Elim} to write down $\CAPTHETA$ on the next line. The order in which you derive the intermediate conditionals doesn't matter.
\item[$\VEE$ Bottom-up:] If you want a sentence of the form $\disjunction{\CAPPHI}{\CAPPSI}$, then first derive either $\CAPPHI$ or derive $\CAPPSI$, and then use \Rule{$\VEE$-Intro} to write it down.
\end{description}
The basic bottom-up strategy isn't always the right tool for deriving disjunctions, because usually you can't derive one of the disjuncts. 
This is an important point: it might be that a disjunction is derivable even if neither disjunct is. 
It's important that we can derive disjunctions without first deriving one or the other disjunct, since $\disjunction{\Bl}{\negation{\Bl}}$ is a logical truth. We would like to be able to derive it, though neither $\Bl$ nor $\negation{\Bl}$ is a logical truth.

Later on we'll get more useful bottom-up strategies for disjunctions. For now we'll focus on the top-down strategy. 

As an example, say we want to derive $\negation{\parconjunction{\bpardisjunction{\negation{\Al}}{\negation{\Bl}}}{\bparconjunction{\Al}{\Bl}}}$. 
The whole sentence itself is a negation, so we start just as in the last example. 
So we need some contradiction to aim for. 
We'll try to derive $\conjunction{\Bl}{\negation{\Bl}}$.
\begin{gproof}
\galineNC{1}{$\parconjunction{\bpardisjunction{\negation{\Al}}{\negation{\Bl}}}{\bparconjunction{\Al}{\Bl}}$}{\Rule{Assume}}
\galineNCnd{}{}{}
\galineNCnd{}{$\qquad\vdots$}{}
\galineNCnd{}{}{}
\galineNC{$\integer{n}$}{$\conjunction{\Bl}{\negation{\Bl}}$}{}
\end{gproof}
As in the last example the bottom-up strategy for $\!\WEDGE\!$ has us try to derive both $\Bl$ and $\negation{\Bl}$.
\begin{gproof}
\galineNC{1}{$\parconjunction{\bpardisjunction{\negation{\Al}}{\negation{\Bl}}}{\bparconjunction{\Al}{\Bl}}$}{\Rule{Assume}}
\galineNCnd{}{}{}
\galineNCnd{}{$\qquad\vdots$}{}
\galineNCnd{}{}{}
\galineNC{$\integer{n}-2$}{$\Bl$}{}
\galineNC{$\integer{n}-1$}{$\negation{\Bl}$}{}
\galineNC{$\integer{n}$}{$\conjunction{\Bl}{\negation{\Bl}}$}{\Rule{$\WEDGE$-Intro}, $\integer{n}-2$,$\integer{n}-1$}
\end{gproof}
And, the top-down strategy leads us to break apart the conjunction on line 1, which leads to another conjunction to break apart as well. This gets us one of the conjuncts of line $\integer{n}$ in the process.
\begin{gproof}
\galineNC{1}{$\parconjunction{\bpardisjunction{\negation{\Al}}{\negation{\Bl}}}{\bparconjunction{\Al}{\Bl}}$}{\Rule{Assume}}
\galineNC{2}{$\bpardisjunction{\negation{\Al}}{\negation{\Bl}}$}{\Rule{$\WEDGE$-Elim}, 1}
\galineNC{3}{$\bparconjunction{\Al}{\Bl}$}{\Rule{$\WEDGE$-Elim}, 1}
\galineNC{4}{$\Bl$}{\Rule{$\WEDGE$-Elim}, 3}
\galineNCnd{}{}{}
\galineNCnd{}{$\qquad\vdots$}{}
\galineNCnd{}{}{}
\galineNC{$\integer{n}-1$}{$\negation{\Bl}$}{\Rule{$\NEGATION$-Intro}, $\integer{n}-2$}
\galineNC{$\integer{n}$}{$\conjunction{\Bl}{\negation{\Bl}}$}{\Rule{$\WEDGE$-Intro}, $\integer{n}-2$,$\integer{n}-1$}
\end{gproof}
Now we only need to get $\negation{\Bl}$. 
To do this, we follow the bottom-up strategy for negation. 
We assume $\Bl$ and try to get a contradiction.
\begin{gproof}
\galineNC{1}{$\parconjunction{\bpardisjunction{\negation{\Al}}{\negation{\Bl}}}{\bparconjunction{\Al}{\Bl}}$}{\Rule{Assume}}
\galineNC{2}{$\bpardisjunction{\negation{\Al}}{\negation{\Bl}}$}{\Rule{$\WEDGE$-Elim}, 1}
\galineNC{3}{$\bparconjunction{\Al}{\Bl}$}{\Rule{$\WEDGE$-Elim}, 1}
\galineNC{4}{$\Bl$}{\Rule{$\WEDGE$-Elim}, 3}
\gaaproof{
\gaalineNCS{5}{$\Bl$}{\Rule{Assume}}
\gaalineNCndS{}{}{}
\gaalineNCndS{}{$\qquad\vdots$}{}
\gaalineNCndS{}{}{}
\gaalineNCS{$\integer{n}-3$}{$\conjunction{\CAPPSI}{\negation{\CAPPSI}}$}{}
}
\galineNC{$\integer{n}-2$}{$\horseshoe{\Bl}{\parconjunction{\CAPPSI}{\negation{\CAPPSI}}}$}{\Rule{$\HORSESHOE$-Intro}, 5--$\integer{n}-3$}
\galineNC{$\integer{n}-1$}{$\negation{\Bl}$}{\Rule{$\NEGATION$-Intro}, $\integer{n}-2$}
\galineNC{$\integer{n}$}{$\conjunction{\Bl}{\negation{\Bl}}$}{\Rule{$\WEDGE$-Intro}, $\integer{n}-2$,$\integer{n}-1$}
\end{gproof}
Note that although it may look wrong to have an assumption line for $\Bl$ when we have already derived it, there's nothing wrong with this derivation. 
You can always assume whatever you want, even if you already have it. 
Now we have two questions to think about: what contradiction could we get on line $\integer{n}-3$? 
And, how do we get it? 
The second question is straightforward. 
We want to get a contradiction, and at this point it's going to have to come from the disjunction on line 2. 
So we follow the top-down strategy for $\VEE$ which tells us how to get a sentence from a disjunction.
\begin{gproof}
\galineNC{1}{$\parconjunction{\bpardisjunction{\negation{\Al}}{\negation{\Bl}}}{\bparconjunction{\Al}{\Bl}}$}{\Rule{Assume}}
\galineNC{2}{$\bpardisjunction{\negation{\Al}}{\negation{\Bl}}$}{\Rule{$\WEDGE$-Elim}, 1}
\galineNC{3}{$\bparconjunction{\Al}{\Bl}$}{\Rule{$\WEDGE$-Elim}, 1}
\galineNC{4}{$\Bl$}{\Rule{$\WEDGE$-Elim}, 3}
\gaaproof{
\gaalineNCS{5}{$\Bl$}{\Rule{Assume}}
\gaalineNCndS{}{}{}
\gaalineNCndS{}{$\qquad\vdots$}{}
\gaalineNCndS{}{}{}
\gaalineNCS{$\integer{n}-5$}{$\horseshoe{\negation{\Bl}}{\parconjunction{\CAPPSI}{\negation{\CAPPSI}}}$}{}
\gaalineNCS{$\integer{n}-4$}{$\horseshoe{\negation{\Al}}{\parconjunction{\CAPPSI}{\negation{\CAPPSI}}}$}{}
\gaalineNCS{$\integer{n}-3$}{$\conjunction{\CAPPSI}{\negation{\CAPPSI}}$}{\Rule{$\VEE$-Elim}, 2,$\integer{n}-5$,$\integer{n}-4$}
}
\galineNC{$\integer{n}-2$}{$\horseshoe{\Bl}{\parconjunction{\CAPPSI}{\negation{\CAPPSI}}}$}{\Rule{$\HORSESHOE$-Intro}, 5--$\integer{n}-3$}
\galineNC{$\integer{n}-1$}{$\negation{\Bl}$}{\Rule{$\NEGATION$-Intro}, $\integer{n}-2$}
\galineNC{$\integer{n}$}{$\conjunction{\Bl}{\negation{\Bl}}$}{\Rule{$\WEDGE$-Intro}, $\integer{n}-2$,$\integer{n}-1$}
\end{gproof}
To get the conditionals on lines $\integer{n}-5$ and $\integer{n}-4$, we'll have to use \Rule{$\HORSESHOE$-Into}. So the setup is:
\begin{gproof}
\galineNC{1}{$\parconjunction{\bpardisjunction{\negation{\Al}}{\negation{\Bl}}}{\bparconjunction{\Al}{\Bl}}$}{\Rule{Assume}}
\galineNC{2}{$\bpardisjunction{\negation{\Al}}{\negation{\Bl}}$}{\Rule{$\WEDGE$-Elim}, 1}
\galineNC{3}{$\bparconjunction{\Al}{\Bl}$}{\Rule{$\WEDGE$-Elim}, 1}
\galineNC{4}{$\Bl$}{\Rule{$\WEDGE$-Elim}, 3}
\gaaproof{
\gaalineNCS{5}{$\Bl$}{\Rule{Assume}}
\gaaaproof{
\gaaalineS{6}{$\negation{\Bl}$}{\Rule{Assume}}
\gaaalinendS{}{}{}
\gaaalinendS{}{$\qquad\vdots$}{}
\gaaalinendS{}{}{}
\gaaalineS{$\integer{m}$}{$\parconjunction{\CAPPSI}{\negation{\CAPPSI}}$}{}
}
\gaalineNCS{$\integer{m}+1$}{$\horseshoe{\negation{\Bl}}{\parconjunction{\CAPPSI}{\negation{\CAPPSI}}}$}{\Rule{$\HORSESHOE$-Intro}, 6--$\integer{m}$}
\gaaaproof{
\gaaalineS{$\integer{m}+2$}{$\negation{\Al}$}{\Rule{Assume}}
\gaaalinendS{}{}{}
\gaaalinendS{}{$\qquad\vdots$}{}
\gaaalinendS{}{}{}
\gaaalineS{$\integer{n}-5$}{$\parconjunction{\CAPPSI}{\negation{\CAPPSI}}$}{}
}
\gaalineNCS{$\integer{n}-4$}{$\horseshoe{\negation{\Al}}{\parconjunction{\CAPPSI}{\negation{\CAPPSI}}}$}{\Rule{$\HORSESHOE$-Intro}, $\integer{m}+2$--$\integer{n}-5$}
\gaalineNCS{$\integer{n}-3$}{$\conjunction{\CAPPSI}{\negation{\CAPPSI}}$}{\Rule{$\VEE$-Elim}, 2,$\integer{m}+1$,$\integer{n}-4$}
}
\galineNC{$\integer{n}-2$}{$\horseshoe{\Bl}{\parconjunction{\CAPPSI}{\negation{\CAPPSI}}}$}{\Rule{$\HORSESHOE$-Intro}, 5--$\integer{n}-3$}
\galineNC{$\integer{n}-1$}{$\negation{\Bl}$}{\Rule{$\NEGATION$-Intro}, $\integer{n}-2$}
\galineNC{$\integer{n}$}{$\conjunction{\Bl}{\negation{\Bl}}$}{\Rule{$\WEDGE$-Intro}, $\integer{n}-2$,$\integer{n}-1$}
\end{gproof}
Now we can decide what contradiction $\parconjunction{\CAPPSI}{\negation{\CAPPSI}}$ to aim for. If we choose $\parconjunction{\Bl}{\negation{\Bl}}$ again, then the first conditional will be easy. We will be able to get it by using \Rule{$\WEDGE$-Intro} on lines 5 and 6.
\begin{gproof}
\galineNC{1}{$\parconjunction{\bpardisjunction{\negation{\Al}}{\negation{\Bl}}}{\bparconjunction{\Al}{\Bl}}$}{\Rule{Assume}}
\galineNC{2}{$\bpardisjunction{\negation{\Al}}{\negation{\Bl}}$}{\Rule{$\WEDGE$-Elim}, 1}
\galineNC{3}{$\bparconjunction{\Al}{\Bl}$}{\Rule{$\WEDGE$-Elim}, 1}
\galineNC{4}{$\Bl$}{\Rule{$\WEDGE$-Elim}, 3}
\gaaproof{
\gaalineNCS{5}{$\Bl$}{\Rule{Assume}}
\gaaaproof{
\gaaalineS{6}{$\negation{\Bl}$}{\Rule{Assume}}
\gaaalineS{7}{$\parconjunction{\Bl}{\negation{\Bl}}$}{\Rule{$\WEDGE$-Intro}, 5,6}
}
\gaalineNCS{8}{$\horseshoe{\negation{\Bl}}{\parconjunction{\Bl}{\negation{\Bl}}}$}{\Rule{$\HORSESHOE$-Intro}, 6--7}
\gaaaproof{
\gaaalineS{9}{$\negation{\Al}$}{\Rule{Assume}}
\gaaalinendS{}{}{}
\gaaalinendS{}{$\qquad\vdots$}{}
\gaaalinendS{}{}{}
\gaaalineS{$\integer{n}-5$}{$\parconjunction{\Bl}{\negation{\Bl}}$}{}
}
\gaalineNCS{$\integer{n}-4$}{$\horseshoe{\negation{\Al}}{\parconjunction{\Bl}{\negation{\Bl}}}$}{\Rule{$\HORSESHOE$-Intro}, 9--$\integer{n}-5$}
\gaalineNCS{$\integer{n}-3$}{$\conjunction{\Bl}{\negation{\Bl}}$}{\Rule{$\VEE$-Elim}, 2,8,$\integer{n}-4$}
}
\galineNC{$\integer{n}-2$}{$\horseshoe{\Bl}{\parconjunction{\Bl}{\negation{\Bl}}}$}{\Rule{$\HORSESHOE$-Intro}, 5--$\integer{n}-3$}
\galineNC{$\integer{n}-1$}{$\negation{\Bl}$}{\Rule{$\NEGATION$-Intro}, $\integer{n}-2$}
\galineNC{$\integer{n}$}{$\conjunction{\Bl}{\negation{\Bl}}$}{\Rule{$\WEDGE$-Intro}, $\integer{n}-2$,$\integer{n}-1$}
\end{gproof}
This leaves us with just the second conditional, deriving $\parconjunction{\Bl}{\negation{\Bl}}$ from $\negation{\Al}$. At first glance this may seem impossible. 
We were able to derive $\parconjunction{\Bl}{\negation{\Bl}}$ from $\negation{\Bl}$ because, given that we already had $\Bl$, assuming $\negation{\Bl}$ allowed us to use \Rule{$\WEDGE$-Intro}. 
But $\negation{\Al}$ obviously isn't sufficient to allow us to use \Rule{$\WEDGE$-Intro}, and there's no clear way to get what we need for \Rule{$\WEDGE$-Intro}, $\negation{\Bl}$, from $\negation{\Al}$. 
So we won't be able to get $\parconjunction{\Bl}{\negation{\Bl}}$ by using \Rule{$\WEDGE$-Intro}. 

Now so far we only have one strategy for getting a conjunction and that's to derive the conjuncts and use \Rule{$\WEDGE$-Intro}. 
But there's another strategy, not tied to any particular connective, which we can use here. 
This strategy is to use \Rule{$\NEGATION$-Elim}. 
We will assume $\negation{\parconjunction{\Bl}{\negation{\Bl}}}$, derive a contradiction (any will do), and then use \Rule{$\HORSESHOE$-Intro} to get what we need. 
It should be reasonably clear that this strategy will work, since we obviously can derive the contradiction $\parconjunction{\Al}{\negation{\Al}}$.
\begin{gproof}[\label{bycontradiction}]
\galineNC{1}{$\parconjunction{\bpardisjunction{\negation{\Al}}{\negation{\Bl}}}{\bparconjunction{\Al}{\Bl}}$}{\Rule{Assume}}
\galineNC{2}{$\bpardisjunction{\negation{\Al}}{\negation{\Bl}}$}{\Rule{$\WEDGE$-Elim}, 1}
\galineNC{3}{$\bparconjunction{\Al}{\Bl}$}{\Rule{$\WEDGE$-Elim}, 1}
\galineNC{4}{$\Bl$}{\Rule{$\WEDGE$-Elim}, 3}
\gaaproof{
\gaalineNCS{5}{$\Bl$}{\Rule{Assume}}
\gaaaproof{
\gaaalineS{6}{$\negation{\Bl}$}{\Rule{Assume}}
\gaaalineS{7}{$\parconjunction{\Bl}{\negation{\Bl}}$}{\Rule{$\WEDGE$-Intro}, 5,6}
}
\gaalineNCS{8}{$\horseshoe{\negation{\Bl}}{\parconjunction{\Bl}{\negation{\Bl}}}$}{\Rule{$\HORSESHOE$-Intro}, 6--7}
\gaaaproof{
\gaaalineS{9}{$\negation{\Al}$}{\Rule{Assume}}
\gaaaaproof{
\gaaaalineS{10}{$\negation{\parconjunction{\Bl}{\negation{\Bl}}}$}{\Rule{Assume}}
\gaaaalineS{11}{$\Al$}{\Rule{$\WEDGE$-Elim}, 3}
\gaaaalineS{12}{$\conjunction{\Al}{\negation{\Al}}$}{\Rule{$\WEDGE$-Intro}, 9,11}
}
\gaaalineS{13}{$\horseshoe{\negation{\parconjunction{\Bl}{\negation{\Bl}}}}{\parconjunction{\Al}{\negation{\Al}}}$}{\Rule{$\HORSESHOE$-Intro}, 10--12}
\gaaalineS{14}{$\parconjunction{\Bl}{\negation{\Bl}}$}{\Rule{$\NEGATION$-Elim}, 13}
}
\gaalineNCS{15}{$\horseshoe{\negation{\Al}}{\parconjunction{\Bl}{\negation{\Bl}}}$}{\Rule{$\HORSESHOE$-Intro}, 9--14}
\gaalineNCS{16}{$\conjunction{\Bl}{\negation{\Bl}}$}{\Rule{$\VEE$-Elim}, 2,8,15}
}
\galineNC{17}{$\horseshoe{\Bl}{\parconjunction{\Bl}{\negation{\Bl}}}$}{\Rule{$\HORSESHOE$-Intro}, 5--16}
\galineNC{18}{$\negation{\Bl}$}{\Rule{$\NEGATION$-Intro}, 17}
\galineNC{19}{$\conjunction{\Bl}{\negation{\Bl}}$}{\Rule{$\WEDGE$-Intro}, 4,18}
\end{gproof}
Note that we didn't actually use the assumption $\negation{\parconjunction{\Bl}{\negation{\Bl}}}$ from line 10 in deriving the contradiction we end with on line 12, $\conjunction{\Al}{\negation{\Al}}$. There is nothing wrong with this, since \Rule{$\HORSESHOE$-Intro} doesn't require that the sentence $\CAPTHETA$ with which you started is actually used in deriving the sentence $\CAPPSI$ with which you finished. 

Now that we've derived a contradiction from $\parconjunction{\bpardisjunction{\negation{\Al}}{\negation{\Bl}}}{\bparconjunction{\Al}{\Bl}}$ we can finish the derivation by discharging the assumption with \Rule{$\HORSESHOE$-Intro} and then use \Rule{$\NEGATION$-Intro}.
\begin{gproof}[\label{cangetlong}]
\gaproof{
\galine{1}{$\parconjunction{\bpardisjunction{\negation{\Al}}{\negation{\Bl}}}{\bparconjunction{\Al}{\Bl}}$}{\Rule{Assume}}
\galine{2}{$\bpardisjunction{\negation{\Al}}{\negation{\Bl}}$}{\Rule{$\WEDGE$-Elim}, 1}
\galine{3}{$\bparconjunction{\Al}{\Bl}$}{\Rule{$\WEDGE$-Elim}, 1}
\galine{4}{$\Bl$}{\Rule{$\WEDGE$-Elim}, 3}
\gaaproof{
\gaaline{5}{$\Bl$}{\Rule{Assume}}
\gaaaproof{
\gaaaline{6}{$\negation{\Bl}$}{\Rule{Assume}}
\gaaaline{7}{$\parconjunction{\Bl}{\negation{\Bl}}$}{\Rule{$\WEDGE$-Intro}, 5,6}
}
\gaaline{8}{$\horseshoe{\negation{\Bl}}{\parconjunction{\Bl}{\negation{\Bl}}}$}{\Rule{$\HORSESHOE$-Intro}, 6--7}
\gaaaproof{
\gaaaline{9}{$\negation{\Al}$}{\Rule{Assume}}
\gaaaaproof{
\gaaaaline{10}{$\negation{\parconjunction{\Bl}{\negation{\Bl}}}$}{\Rule{Assume}}
\gaaaaline{11}{$\Al$}{\Rule{$\WEDGE$-Elim}, 3}
\gaaaaline{12}{$\conjunction{\Al}{\negation{\Al}}$}{\Rule{$\WEDGE$-Intro}, 9,11}
}
\gaaaline{13}{$\horseshoe{\negation{\parconjunction{\Bl}{\negation{\Bl}}}}{\parconjunction{\Al}{\negation{\Al}}}$}{\Rule{$\HORSESHOE$-Intro}, 10--12}
\gaaaline{14}{$\parconjunction{\Bl}{\negation{\Bl}}$}{\Rule{$\NEGATION$-Elim}, 13}
}
\gaaline{15}{$\horseshoe{\negation{\Al}}{\parconjunction{\Bl}{\negation{\Bl}}}$}{\Rule{$\HORSESHOE$-Intro}, 9--14}
\gaaline{16}{$\conjunction{\Bl}{\negation{\Bl}}$}{\Rule{$\VEE$-Elim}, 2,8,15}
}
\galine{17}{$\horseshoe{\Bl}{\parconjunction{\Bl}{\negation{\Bl}}}$}{\Rule{$\HORSESHOE$-Intro}, 5--16}
\galine{18}{$\negation{\Bl}$}{\Rule{$\NEGATION$-Intro}, 17}
\galine{19}{$\conjunction{\Bl}{\negation{\Bl}}$}{\Rule{$\WEDGE$-Intro}, 4,18}
}
\gline{20}{$\horseshoe{\parconjunction{\bpardisjunction{\negation{\Al}}{\negation{\Bl}}}{\bparconjunction{\Al}{\Bl}}}{\parconjunction{\Bl}{\negation{\Bl}}}$}{\Rule{$\HORSESHOE$-Intro}, 1--19}
\gline{21}{$\negation{\parconjunction{\bpardisjunction{\negation{\Al}}{\negation{\Bl}}}{\bparconjunction{\Al}{\Bl}}}$}{\Rule{$\NEGATION$-Intro}, 20}
\end{gproof}

\subsubsection*{Proof by Contradiction}
A general strategy involving negation was used in derivation \pmvref{cangetlong}. 
This strategy formalizes an informal proof method often called proof by contradiction. 
In proof by contradiction, one proves that some sentence $\CAPPHI$ is true by assuming it's false and showing that a contradiction follows. 
The corresponding strategy in \GSD{} is:
\begin{description}
\item[Proof by Contradiction:] If you want to get some sentence $\CAPPHI$, then first derive $\horseshoe{\negation{\CAPPHI}}{\parconjunction{\CAPPSI}{\negation{\CAPPSI}}}$ and then use \Rule{$\NEGATION$-Elim} to get $\CAPPHI$ from this conditional.  
\end{description}
This strategy has the virtue that if you can derive $\CAPPHI$ (without any assumptions), then you will be able to derive a contradiction $\parconjunction{\CAPPSI}{\negation{\CAPPSI}}$ from $\negation{\CAPPHI}$ as an assumption. In other words, the strategy will always work. But despite this, there are usually much faster and better ways to derive a sentence. 
For example, write a derivation of $\conjunction{\Cl}{\Bl}$ from $\conjunction{\Bl}{\Cl}$ using proof by contradiction and then compare it to derivation \pmvref{simpleconjunction}. 
There are some cases where proof by contradiction is the \emph{only} strategy that will work. 
So there are two cases where it is a good idea to use the strategy: cases where all other available strategies haven't worked (or where there aren't any other available strategies), and cases where you can see a straightforward way to use it. 

\subsubsection*{Top-down and Bottom-up} Finally, the reader should reflect on the general method we have been using. In slogan form, the method goes: work top-down \emph{and} bottom-up. When writing derivation it's important to work not only from the assumptions, seeing how you can move down from them to the conclusion using the rules, but also to work up from the sentence you're trying to derive, looking for the different ways you can get there. Hence the grouping of strategies into top-down and bottom-up. The top-down strategies help guide what moves you can make as you work from the assumptions to the conclusion, while the bottom-up strategies help see what different paths there are to get to that conclusion. 

As an aside, the top-down and bottom-up method is something useful outside of writing formal derivations. While most arguments, including those found in philosophy, mathematical proofs, and formal derivations, are presented as a series of steps from premises to conclusion, they are seldom devised that way. Usually a mathematician starts with a conjecture and tries to work ``up'' from it to results that have already been proven. Rarely does one from the start know what premises will be needed to prove a conjecture. Something similar goes for philosophy and other disciplines that rely on argument. So, the reader can think of the top-down, bottom-up method used in derivations as a reflection of how informal arguments and proofs are constructed in philosophy, mathematics, and other disciplines.

%%%%%%%%%%%%%%%%%%%%%%%%%%%%%%%%%%%%%%%%%%%%%%%%%%
\section{Shortcut Rules for \GSD{}}
%%%%%%%%%%%%%%%%%%%%%%%%%%%%%%%%%%%%%%%%%%%%%%%%%%

\subsection{Standard Shortcut Rules}\label{Standard Shortcut Rules GSD}
From derivation \pmvref{cangetlong}, we can see that derivations in \GSD{} of even simple looking sentences can be long and involve roundabout strategies. 
This is the main reason why we want to introduce shortcut rules. 
Shortcut rules can be thought of as ways of cutting out parts of derivations that we find ourselves doing repeatedly. 
The basic rules of \GSD{} plus the shortcut rules (both those in table \pncmvref{GSDplus1} and table \pncmvref{GSDplus2}) make up the derivation system which we call \GSDP{}.

For example, consider lines 10--14 of proof \pmvref{cangetlong}. Here we have two sentences, $\Al$ and $\negation{\Al}$, and we used them to derive the sentence $\conjunction{\Bl}{\negation{\Bl}}$. 
We can rewrite the relevant parts of the proof here, putting them in a less idiosyncratic order:
\begin{gproof}
\galineNC{1}{$\Al$}{}
\galineNC{2}{$\negation{\Al}$}{}
\gaaproof{
\gaalineNCS{3}{$\negation{\parconjunction{\Bl}{\negation{\Bl}}}$}{\Rule{Assume}}
\gaalineNCS{4}{$\conjunction{\Al}{\negation{\Al}}$}{\Rule{$\WEDGE\!$-Intro, 1,2}}
}
\galineNC{5}{$\horseshoe{\negation{\parconjunction{\Bl}{\negation{\Bl}}}}{\parconjunction{\Al}{\negation{\Al}}}$}{\Rule{$\HORSESHOE$-Intro}, 3--4}
\galineNC{6}{$\conjunction{\Bl}{\negation{\Bl}}$}{\Rule{$\NEGATION$-Elim}}
\end{gproof}
It should be clear from looking at this derivation that we can replace $\conjunction{\Bl}{\negation{\Bl}}$ with any sentence $\CAPPSI$ and the string of sentences that results from this replacement will also be a derivation. 
It should also be clear that we can replace $\Al$ and $\negation{\Al}$ with any pair $\CAPPHI$ and $\negation{\CAPPHI}$ and the resulting string of sentences will be a derivation. 
That is,
\begin{gproof}[\label{anycontradictionSC}]
\galineNC{1}{$\CAPPHI$}{}
\galineNC{2}{$\negation{\CAPPHI}$}{}
\gaaproof{
\gaalineNCS{3}{$\negation{\CAPPSI}$}{\Rule{Assume}}
\gaalineNCS{4}{$\conjunction{\CAPPHI}{\negation{\CAPPHI}}$}{\Rule{$\WEDGE\!$-Intro, 1,2}}
}
\galineNC{5}{$\horseshoe{\negation{\CAPPSI}}{\parconjunction{\CAPPHI}{\negation{\CAPPHI}}}$}{\Rule{$\HORSESHOE$-Intro}, 3--4}
\galineNC{6}{$\CAPPSI$}{\Rule{$\NEGATION$-Elim}}
\end{gproof}
will be a derivation of $\CAPPSI$ for any sentences $\CAPPHI$ and $\CAPPSI$. 
More importantly, any time we're doing a derivation and we have two lines, one with a sentence $\CAPPHI$ and the other with its negation $\negation{\CAPPHI}$, we could insert a derivation of just this form to get a sentence $\CAPPSI$. 
So we can introduce a new rule which says that given two sentences $\CAPPHI$ and $\negation{\CAPPHI}$, we can add any sentence $\CAPPSI$. 
We call this new rule \Rule{Any Contradiction}, or \Rule{A.C.} for short. 
The key feature of this rule is that in any derivation where we use the rule, we could have gotten the same results without it. 
All we'd have to do is insert the appropriate instance of \ref{anycontradictionSC} into the derivation. (We leave it to the reader to rewrite derivation \pncmvref{cangetlong} using \Rule{A.C.})

%\begin{table}[!ht]
%\renewcommand{\arraystretch}{1.5}
%\begin{center}
%\begin{tabular}{ p{1in} l l } %p{2.2in} p{2in}
%\toprule
%\textbf{Name} & \textbf{Given} & \textbf{May Add} \\ 
%\midrule
%\begin{table}[!ht]
\renewcommand{\arraystretch}{1.5}
\begin{longtable}[c]{ p{1in} l l } %p{2.2in} p{2in}
	\toprule
	\textbf{Name} & \textbf{Given} & \textbf{May Add} \\ 
	\midrule
	\endfirsthead
	\multicolumn{3}{c}{\emph{Continued from Previous Page}}\\
	\toprule
	\textbf{Name} & \textbf{Given} & \textbf{May Add} \\ 
	\midrule
	\endhead
	\bottomrule
	\caption{Standard Shortcut Rules for \GSD{}}\\[-.15in]
	\multicolumn{3}{c}{\emph{Continued next Page}}\\
	\endfoot
	\bottomrule
	\caption{Standard Shortcut Rules for \GSD{}}\\
	\endlastfoot
	\label{GSDplus1}\Rule{M.T.} & $\horseshoe{\CAPPHI}{\CAPTHETA}$, $\negation{\CAPTHETA}$ & $\negation{\CAPPHI}$ \\
	\Rule{D.S.} & $\disjunction{\CAPPHI_1}{\disjunction{\ldots}{\disjunction{\CAPPHI_i}{\disjunction{\ldots}{\CAPPHI_{\integer{n}}}}}}$, $\negation{\CAPPHI_i}$ & $\disjunction{\CAPPHI_1}{\disjunction{\ldots}{\disjunction{\CAPPHI_{i-1}}{\disjunction{\CAPPHI_{i+1}}{\disjunction{\ldots}{\CAPPHI_{\integer{n}}}}}}}$ \\
	\nopagebreak
	& $\disjunction{\CAPPHI_1}{\disjunction{\ldots}{\disjunction{\negation{\CAPPHI_i}}{\disjunction{\ldots}{\CAPPHI_{\integer{n}}}}}}$, ${\CAPPHI_i}$ & $\disjunction{\CAPPHI_1}{\disjunction{\ldots}{\disjunction{\CAPPHI_{i-1}}{\disjunction{\CAPPHI_{i+1}}{\disjunction{\ldots}{\CAPPHI_{\integer{n}}}}}}}$ \\
	\Rule{A.C.} & ${\CAPPHI},{\negation{\CAPPHI}}$ & $\CAPPSI$ \\
	\Rule{$\NEGATION$/$\TRIPLEBAR$-Intro} & $\triplebar{\CAPPHI}{\CAPPSI}$ & $\triplebar{\negation{\CAPPHI}}{\negation{\CAPPSI}}$ \\
	\Rule{Ext. $\WEDGE$-Elim} &{}$\conjunction{\CAPTHETA_1}{\conjunction{\CAPTHETA_2}{\conjunction{\ldots}{\CAPTHETA_{\integer{n}}}}}$&{}Conjunction of any\\[-.25cm]
	\nopagebreak
	& &{}subset of the conjuncts\\
\end{longtable}
%\bottomrule
%\end{tabular}
%\end{center}
%\caption{Standard Short-Cut Rules for \GSD{} (\GSD{})}
% \label{GSDplus1}
%\end{table}
 
The idea will be the same for all the shortcut rules we'll introduce here. These come in two types, standard and exchange, and are listed in table \pmvref{GSDplus1} and \pmvref{GSDplus2}. The idea is, (i) for each shortcut rule \Rule{R}, for given any application of \Rule{R} we can derive, using only the basic rules, the sentence $\CAPPHI$ (which \Rule{R} allows us to write down) from the sentences $\CAPPSI_1,\ldots,\CAPPSI_{\integer{n}}$ to which we applied \Rule{R}. 
And, (ii) in general, anything we can derive using a rule, any application of which can be derived using only basic rules, can itself be derived using only basic rules.  
So, (iii) anything we can derive using the basic rules of \GSD{} and any of the shortcut rules can be derived from just the basic rules alone.
We have restated (ii) more precisely below as theorem \mvref{GSD Shortcut Theorem}, (i) as theorem \mvref{GSD Shortcut Theorem2}, and (iii) as theorem \mvref{GSD Shortcut Theorem3}. 
Exchange rules are short cut rules that work in both directions.

%\begin{table}[!ht]
%\renewcommand{\arraystretch}{1.5}
%\begin{center}
%\begin{tabular}{ p{1in} l l } %p{2.2in} p{2in}
%\toprule
%\textbf{Name} & \textbf{Given} & \textbf{May Add} \\ 
%\midrule
\renewcommand{\arraystretch}{1.5}
\begin{longtable}[c]{ p{1in} l l } %p{2.2in} p{2in}
\toprule
\textbf{Name} & \textbf{Given} & \textbf{May Add} \\ 
\midrule
\endfirsthead
\multicolumn{3}{c}{\emph{Continued from Previous Page}}\\
\toprule
\textbf{Name} & \textbf{Given} & \textbf{May Add} \\ 
\midrule
\endhead
\bottomrule
\caption{Exchange Short-Cut Rules for \GSD{}}\\[-.15in]
\multicolumn{3}{c}{\emph{Continued next Page}}\\
\endfoot
\bottomrule
\caption{Exchange Short-Cut Rules for \GSD{}}\\
\endlastfoot
\label{GSDplus2}\Rule{DeM} & $\negation{\parconjunction{\CAPPHI_1}{\conjunction{\ldots}{\CAPPHI_{\integer{n}}}}}$ & $\disjunction{\negation{\CAPPHI_1}}{\disjunction{\ldots}{\negation{\CAPPHI_{\integer{n}}}}}$\\
 & $\disjunction{\negation{\CAPPHI_1}}{\disjunction{\ldots}{\negation{\CAPPHI_{\integer{n}}}}}$ & $\negation{\parconjunction{\CAPPHI_1}{\conjunction{\ldots}{\CAPPHI_{\integer{n}}}}}$\\
 & $\negation{\pardisjunction{\CAPPHI_1}{\disjunction{\ldots}{\CAPPHI_{\integer{n}}}}}$ & $\conjunction{\negation{\CAPPHI_1}}{\conjunction{\ldots}{\negation{\CAPPHI_{\integer{n}}}}}$ \\
 & $\conjunction{\negation{\CAPPHI_1}}{\conjunction{\ldots}{\negation{\CAPPHI_{\integer{n}}}}}$ & $\negation{\pardisjunction{\CAPPHI_1}{\disjunction{\ldots}{\CAPPHI_{\integer{n}}}}}$ \\
\Rule{$\NEGATION\NEGATION$-Elim} & $\negation{\negation{\CAPPHI}}$ & $\CAPPHI$ \\
\Rule{$\NEGATION\NEGATION$-Intro} & $\CAPPHI$ & $\negation{\negation{\CAPPHI}}$ \\
\Rule{$\HORSESHOE$/$\VEE$-Exch.} & $\horseshoe{\CAPPHI}{\CAPTHETA}$ & $\disjunction{\negation{\CAPPHI}}{\CAPTHETA}$ \\
\nopagebreak
 & $\disjunction{\negation{\CAPPHI}}{\CAPTHETA}$ & $\horseshoe{\CAPPHI}{\CAPTHETA}$  \\
\Rule{Contraposition} & $\horseshoe{\CAPPHI}{\CAPTHETA}$ & $\horseshoe{\negation{\CAPTHETA}}{\negation{\CAPPHI}}$ \\
 & $\horseshoe{\negation{\CAPTHETA}}{\negation{\CAPPHI}}$ & $\horseshoe{\CAPPHI}{\CAPTHETA}$ \\
\Rule{$\NEGATION$/$\HORSESHOE$-Exch.} & $\negation{\parhorseshoe{\CAPPHI}{\CAPTHETA}}$ & $\conjunction{\CAPPHI}{\negation{\CAPTHETA}}$ \\
\nopagebreak
 & $\conjunction{\CAPPHI}{\negation{\CAPTHETA}}$ & $\negation{\parhorseshoe{\CAPPHI}{\CAPTHETA}}$ \\
\Rule{Distribution} & $\conjunction{\CAPTHETA}{\pardisjunction{\CAPPHI_1}{\disjunction{\ldots}{\CAPPHI_{\integer{n}}}}}$ & $\disjunction{\parconjunction{\CAPTHETA}{\CAPPHI_1}}{\disjunction{\ldots}{\parconjunction{\CAPTHETA}{\CAPPHI_{\integer{n}}}}}$\\
\nopagebreak
 & $\disjunction{\parconjunction{\CAPTHETA}{\CAPPHI_1}}{\disjunction{\ldots}{\parconjunction{\CAPTHETA}{\CAPPHI_{\integer{n}}}}}$ & $\conjunction{\CAPTHETA}{\pardisjunction{\CAPPHI_1}{\disjunction{\ldots}{\CAPPHI_{\integer{n}}}}}$\\
\nopagebreak 
 & $\conjunction{\pardisjunction{\CAPPHI_1}{\disjunction{\ldots}{\CAPPHI_{\integer{n}}}}}{\CAPTHETA}$ & $\disjunction{\parconjunction{\CAPPHI_1}{\CAPTHETA}}{\disjunction{\ldots}{\parconjunction{\CAPPHI_{\integer{n}}}{\CAPTHETA}}}$\\
\nopagebreak 
 & $\disjunction{\parconjunction{\CAPPHI_1}{\CAPTHETA}}{\disjunction{\ldots}{\parconjunction{\CAPPHI_{\integer{n}}}{\CAPTHETA}}}$  & $\conjunction{\pardisjunction{\CAPPHI_1}{\disjunction{\ldots}{\CAPPHI_{\integer{n}}}}}{\CAPTHETA}$\\
\nopagebreak 
 & $\disjunction{\CAPTHETA}{\parconjunction{\CAPPHI_1}{\conjunction{\ldots}{\CAPPHI_{\integer{n}}}}}$ & $\conjunction{\pardisjunction{\CAPTHETA}{\CAPPHI_1}}{\conjunction{\ldots}{\pardisjunction{\CAPTHETA}{\CAPPHI_{\integer{n}}}}}$\\
\nopagebreak 
 & $\conjunction{\pardisjunction{\CAPTHETA}{\CAPPHI_1}}{\conjunction{\ldots}{\pardisjunction{\CAPTHETA}{\CAPPHI_{\integer{n}}}}}$ & $\disjunction{\CAPTHETA}{\parconjunction{\CAPPHI_1}{\conjunction{\ldots}{\CAPPHI_{\integer{n}}}}}$\\
\nopagebreak 
 & $\disjunction{\parconjunction{\CAPPHI_1}{\conjunction{\ldots}{\CAPPHI_{\integer{n}}}}}{\CAPTHETA}$ & $\conjunction{\pardisjunction{\CAPPHI_1}{\CAPTHETA}}{\conjunction{\ldots}{\pardisjunction{\CAPPHI_{\integer{n}}}{\CAPTHETA}}}$\\
\nopagebreak
 & $\conjunction{\pardisjunction{\CAPPHI_1}{\CAPTHETA}}{\conjunction{\ldots}{\pardisjunction{\CAPPHI_{\integer{n}}}{\CAPTHETA}}}$ & $\disjunction{\parconjunction{\CAPPHI_1}{\conjunction{\ldots}{\CAPPHI_{\integer{n}}}}}{\CAPTHETA}$\\
\end{longtable}
%\bottomrule
%\end{tabular}
%\end{center}
%\caption{Exchange Short-Cut Rules for \GSD{} (\GSD{})}
%\label{GSDplus2}
%\end{table}

Theorem \mvref{GSD Shortcut Theorem} is a general claim that doesn't require a different proof for each new rule we introduce. 
The proof for it fills out the quick argument given when \Rule{A.C.} was introduced. There we argued that since any application of the rule can be derived using just the basic rules, we can eliminate that application of the rule from the proof by cutting and pasting the derivation of that application into the original proof. 
\begin{majorILnc}{\LnpDC{RuleInstanceDerivability}}
We say that every application of a rule \Rule{R$_1$} is derivable using the rules \Rule{R$_2$}, $\ldots$, \Rule{R$_\integer{p}$} and the basic rules of \GSD{} \Iff for all \GSL{} sentences $\CAPPHI_1,\ldots,\CAPPHI_{\integer{m}}$ and $\CAPPSI$, if \Rule{R$_1$} sanctions writing down $\CAPPSI$ when applied to $\CAPPHI_1,\ldots,\CAPPHI_{\integer{m}}$ on previous unboxed lines, then $\CAPPSI$ can be derived from $\CAPPHI_1,\ldots,\CAPPHI_{\integer{m}}$ using only rules \Rule{R$_2$}$,\ldots,$\Rule{R$_\integer{p}$} and the basic rules of \GSD{}.
\end{majorILnc}
\begin{THEOREM}{\LnpTC{GSD Shortcut Theorem}}
For all \GSL{} sentences $\CAPTHETA_1,\ldots,\CAPTHETA_{\integer{n}},\DELTA$ and rules \Rule{R$_1$}$,\ldots,$\Rule{R$_\integer{p}$}, if
\begin{cenumerate}
\item $\DELTA$ can be derived from $\CAPTHETA_1,\ldots,\CAPTHETA_{\integer{n}}$ using rules \Rule{R$_1$}$,\ldots,$\Rule{R$_\integer{p}$} and the basic rules of \GSD{}, and
\item every application of a rule \Rule{R$_1$} is derivable using the rules \Rule{R$_2$}, $\ldots$, \Rule{R$_\integer{p}$} and the basic rules of \GSD{},
\end{cenumerate}
then $\DELTA$ can be derived from $\CAPTHETA_1,\ldots,\CAPTHETA_{\integer{n}}$ using only rules \Rule{R$_2$}$,\ldots,$\Rule{R$_\integer{p}$} and the basic rules of \GSD{}.
%Any sentence $\CAPPHI$ that can be derived from the sentences $\CAPPSI_1,\ldots,\CAPPSI_{\integer{n}}$ using the basic rules plus some of the shortcut rules in tables \ref{GSDplus1} and \ref{GSDplus2} can be derived from $\CAPPSI_1,\ldots,\CAPPSI_{\integer{n}}$ using the basic rules alone. 
\end{THEOREM}
\begin{PROOF}
Call the original derivation of $\DELTA$ from $\CAPTHETA_1,\ldots,\CAPTHETA_{\integer{n}}$ using rules \Rule{R$_1$}$,\ldots,$\Rule{R$_\integer{p}$} derivation $\Derivation{D}_1$. 
Say the first use of \Rule{R$_1$} happens in $\Derivation{D}_1$ on line $\integer{q}$. 
Say in that case sentence $\CAPPSI$ was written down (on line $\integer{q}$) and the application of the rule used previous lines $\integer{r}_1,\ldots,\integer{r}_{\integer{m}}$ with sentences $\CAPPHI_1,\ldots,\CAPPHI_{\integer{m}}$, respectively, on them. 
By assumption, $\CAPPSI$ can be derived from sentences $\CAPPHI_1,\ldots,\CAPPHI_{\integer{m}}$ using only \Rule{R$_2$}$,\ldots,$\Rule{R$_\integer{p}$} and the basic rules of \GSD{}. Call this derivation $\Derivation{D}^*$ and assume $\CAPPHI_1,\ldots,\CAPPHI_{\integer{m}}$ are, respectively, on lines 1 through $\integer{m}$ in $\Derivation{D}^*$. Assume there are $\integer{s}$ more lines (call these the middle lines of $\Derivation{D}^*$), and finally on line $\integer{m}+\integer{s}+1$ of $\Derivation{D}^*$ is $\CAPPSI$.

Now in between lines $\integer{q}-1$ and $\integer{q}$ of $\Derivation{D}_1$ insert $\integer{s}$ new lines. 
On these lines put the appropriate sentence from the $\integer{s}$ middle line of $\Derivation{D}^*$. 
(So, put the sentence on the first middle line of $\Derivation{D}^*$ on the first new line inserted into $\Derivation{D}_1$, the second on the second, etc.) 
Write the same rules as justifications on these new lines as were on the middle lines of $\Derivation{D}^*$ and for each justification if $\integer{t}$ was the number of a line cited by that justification in $\Derivation{D}^*$, cite line number $\integer{r}_\integer{t}$ if $\integer{t}\leq\integer{m}$ and cite line number $(\integer{t}-\integer{m})+(\integer{q}-1)$ if $\integer{t}>\integer{m}$. 
Next, on the line originally numbered $\integer{q}$ (now numbered $\integer{q}+\integer{s}$) erase rule \Rule{R$_1$} as the justification and whatever line numbers $\integer{t}$ were cited and in place of that put whatever rule was used to justify the last line of $\Derivation{D}^*$, citing instead lines $\integer{r}_\integer{t}$ if $\integer{t}\leq\integer{m}$ and lines $(\integer{t}-\integer{m})+(\integer{q}-1)$ if $\integer{t}>\integer{m}$.
Finally, on all the lines of $\Derivation{D}_1$ that were originally numbered $\integer{q}$ or higher (and now are numbered $\integer{q}+\integer{s}$ and higher), if the justification cites line $\integer{t}$ for $\integer{t}<\integer{q}$, do nothing. 
If it cites line $\integer{t}$ for $\integer{t}\geq\integer{q}$, replace $\integer{t}$ with $\integer{t}+\integer{s}$. 

Note that we've now produced a derivation $\Derivation{D}_2$ of $\DELTA$ from $\CAPTHETA_1,\ldots,\CAPTHETA_{\integer{n}}$ that has one less application of \Rule{R$_1$} than $\Derivation{D}_1$ had. 
Now if there is an application of rule \Rule{R$_1$} in $\Derivation{D}_2$, repeat exactly this procedure for $\Derivation{D}_2$. 
Repeating the procedure will lead to a derivation $\Derivation{D}_3$ with one less application of \Rule{R$_1$} than $\Derivation{D}_2$ had. 
We can continue this, producing some series $\Derivation{D}_1,\Derivation{D}_2,\ldots,\Derivation{D}_{\integer{l}}$ of derivations which will eventually end in a derivation $\Derivation{D}_{\integer{l}}$ that has no applications of \Rule{R$_1$}. 
(This procedure must end, since there could only have been a finite number of applications of \Rule{R$_1$} in $\Derivation{D}_1$.) 
Thus, $\Derivation{D}_{\integer{l}}$ will be a derivation of $\DELTA$ from $\CAPTHETA_1,\ldots,\CAPTHETA_{\integer{n}}$ that uses only rules \Rule{R$_2$}$,\ldots,$\Rule{R$_\integer{p}$} and the basic rules of \GSD{}.
\end{PROOF}
\begin{THEOREM}{\LnpTC{GSD Shortcut Theorem2}}
For all standard and exchange shortcut rules \Rule{R} (see tables \ref{GSDplus1} and \ref{GSDplus2}), every application of \Rule{R} is derivable using the basic rules of \GSD{}.
\end{THEOREM}
\begin{PROOF}
See the discussion immediately following the proof of theorem \ref{GSD Shortcut Theorem3}.
\end{PROOF}
\begin{THEOREM}{\LnpTC{GSD Shortcut Theorem3} Shortcut Rule Elimination Theorem:}
For all \GSL{} sentences $\CAPPHI_1,\ldots,\CAPPHI_{\integer{m}}$ and $\CAPPSI$, if $\CAPPSI$ can be derived from $\CAPPHI_1,\ldots,\CAPPHI_{\integer{m}}$ in \GSDP{} (that is, using the basic rules of \GSD{} and any of the standard and exchange shortcut rules), then $\CAPPSI$ can be derived from $\CAPPHI_1,\ldots,\CAPPHI_{\integer{m}}$ in \GSD{} (that is, using only the basic rules).
\end{THEOREM}
\begin{PROOF}
Assume that $\CAPPSI$ can be derived from $\CAPPHI_1,\ldots,\CAPPHI_{\integer{m}}$ using the basic rules of \GSD{} and the standard and exchange shortcut rules. 
Consider any of the shortcut rules, say \Rule{M.T.} 
Let \Rule{R$_1$} be \Rule{M.T.} and rules \Rule{R$_2$} through \Rule{R$_{25}$} be the other standard and exchange rules.
By assumption, condition (1) in theorem \mvref{GSD Shortcut Theorem} holds for these sentences, while by theorem \mvref{GSD Shortcut Theorem2} condition (2) in theorem \ref{GSD Shortcut Theorem} holds for \Rule{M.T.} 
So, it follows from theorem \ref{GSD Shortcut Theorem} that $\CAPPSI$ can be derived from $\CAPPHI_1,\ldots,\CAPPHI_{\integer{m}}$ using the basic rules of \GSD{} and all the standard and exchange shortcut rules besides \Rule{M.T.} By reapplying theorem \ref{GSD Shortcut Theorem} in just the same way to all the standard and exchange shortcut rules, we get that $\CAPPSI$ can be derived from $\CAPPHI_1,\ldots,\CAPPHI_{\integer{m}}$ using only the basic rules of \GSD{}. 
(That is, we reapply theorem \ref{GSD Shortcut Theorem} twenty four more times, each time showing that another shortcut rule wasn't needed.)
\end{PROOF}

\bigskip
\noindent{}Unlike theorem \ref{GSD Shortcut Theorem}, for theorem \ref{GSD Shortcut Theorem2} we need a separate argument for each shortcut rule (both standard and exchange). 
We've already done one rule, \Rule{Any Contradiction}. 
Our argument in this case was that any application of the rule will involve writing some sentence $\CAPPSI$ on a new line from sentences $\CAPPHI$ and $\negation{\CAPPHI}$. 
But whatever sentences $\CAPPSI$ and $\CAPPHI$ we pick, if we substitute them into \ref{anycontradictionSC}, the result will be a derivation of $\CAPPSI$ from $\CAPPHI$ and $\negation{\CAPPHI}$. 

Before continuing, it's important to note that, strictly speaking, \ref{anycontradictionSC} is \emph{not} a derivation. 
This is because a derivation, as we've defined it, is a series of \GSL{} sentences. 
The strings of symbols on each line of \ref{anycontradictionSC} are not \GSL{} sentences because they contain MathEnglish variables for \GSL{} sentences. 
Instead, they are sentence schemas.\index{sentence schema}
But, as we've done in \ref{anycontradictionSC}, nothing stops us from treating sentence schemas like the ones in \ref{anycontradictionSC} as \GSL{} sentences and applying rules to them. 
The key fact---why this is useful---is that what we get when we do this becomes a derivation whenever we substitute \GSL{} sentences in for the MathEnglish variables. 
For this reason we'll call these \niidf{derivation} \underidf{schemas}{derivation} instead of derivations. 

Returning to theorem \mvref{GSD Shortcut Theorem2}, we can handle the other rules in just the same way we handled \Rule{Any Contradiction}. 
For example, if we write a derivation schema of $\negation{\CAPPHI}$ from $\horseshoe{\CAPPHI}{\CAPTHETA}$ and $\negation{\CAPTHETA}$ using only basic rules of \GSD{}, then this will be sufficient to show that any application of the rule \Rule{M.T.} (\Rule{Modus Tollens}) can be derived using only basic rules of \GSD{}.\footnote{For
those keeping track of the \distinction{use}{mention} distinction, here we have mentioned the MathEnglish variables \mention{$\CAPPHI$} and \mention{$\CAPTHETA$} as well as the strings of symbols \mention{$\negation{\CAPPHI}$}, \mention{$\horseshoe{\CAPPHI}{\CAPTHETA}$}, and \mention{$\negation{\CAPTHETA}$}. 
So, strictly speaking, we should have put them all in quotes.
This is different how we normally use these symbols, since we're normally actually using them (as variables) instead of mentioning them (as the objects of derivation schemas).}
(Recall def. \pmvref{RuleInstanceDerivability}: Saying that an application of a rule can be derived using only basic rules is a shorthand way of saying that the sentence written down on a new line, in some application of the rule, can be derived using only basic rules from the sentences to which the rule was applied.)
So, in order to complete the proof for theorem \ref{GSD Shortcut Theorem2}, we need to write derivation schemas for all the standard and exchange rules (for all the rules in tables \ref{GSDplus1} and \ref{GSDplus2}).
That is, for each rule we need to write a derivation schema that has the given schemas of the rule as premises and the may-add schema of the rule as conclusion. 

Note that once we have shown that a shortcut rule can be eliminated from a proof (i.e., once we've shown that theorem \ref{GSD Shortcut Theorem2}, holds at least for that rule), then we can use that shortcut rule in derivation schemas for new shortcut rules we haven't yet shown can be eliminated. 
If we write a derivation schema for some new shortcut rule we're trying to show can be eliminated and that schema uses a previous shortcut rule, then any derivation got from the schema will contain an application of the previous rule.
But we already know that these applications of the previous rule can be eliminated, so that's not a problem.  
%But, we can use rules, for which we've already completed a derivation schema, in our derivation schemas for rules we haven't yet completed. 

Most of the derivation schemas for the shortcut rules (both standard and exchange) are left to the reader as exercises. 
(See section \ref{exercisesGSDshortcutrules}.) 
There is one complication though. 
We cannot actually write a single derivation schema for \Rule{D.S.} (\Rule{Disjunctive Syllogism}), \Rule{DeM} (\Rule{DeMorgans}), or \Rule{Distribution}. 
This is because these rules mention arbitrarily long conjunctions and disjunctions and we can only write a derivation schema involving conjunctions and disjunctions of definite, fixed (and finite) length. 
A less rigorous option is to write a few derivation schemas for \Rule{D.S.}, \Rule{DeM}, and \Rule{Distribution} for the cases when the conjunctions and disjunctions are small (say 2- or 3-place) and convince ourselves that we can keep writing similar schemas no matter how large the conjunctions and disjunctions get.
A more rigorous option is to write the derivation schema for the 2-place case and then use mathematical induction on the length of the conjunctions and disjunctions to show derivation schemas can be written for all lengths. 
In the exercises, in section \ref{exercisesGSDshortcutrules}, we only ask the reader to write the derivation schemas for these three rules for the cases where the conjunctions and disjunctions are 2-place.

Here we write the derivation schema needed for the last of the four \Rule{DeMorgans} rules in table \mvref{GSDplus2}, assuming that the conjunction and disjunction are only 2-place.
So we need to derive $\negation{\pardisjunction{\CAPPHI}{\CAPTHETA}}$ from $\conjunction{\negation{\CAPPHI}}{\negation{\CAPTHETA}}$. 
The sentence we want is a negation, so we use our basic bottom-up strategy for negation:
\begin{gproof}
\galineNC{1}{$\conjunction{\negation{\CAPPHI}}{\negation{\CAPTHETA}}$}{\Rule{Assume}}
\gaaproof{
\gaalineNCS{2}{$\pardisjunction{\CAPPHI}{\CAPTHETA}$}{\Rule{Assume}}
\gaalineNCndS{}{}{}
\gaalineNCndS{}{$\qquad\vdots$}{}
\gaalineNCndS{}{}{}
\gaalineNCS{$\integer{n}-2$}{$\conjunction{\CAPPSI}{\negation{\CAPPSI}}$}{ }
}
\galineNC{$\integer{n}-1$}{$\horseshoe{\pardisjunction{\CAPPHI}{\CAPTHETA}}{\parconjunction{\CAPPSI}{\negation{\CAPPSI}}}$}{\Rule{$\HORSESHOE$-Intro}, 2--$\integer{n}-2$}
\galineNC{$\integer{n}$}{$\negation{\pardisjunction{\CAPPHI}{\CAPTHETA}}$}{\Rule{$\NEGATION$-Intro, $\integer{n}-1$}}
\end{gproof}
\noindent{}As always with \Rule{$\NEGATION$-Intro}, we need some contradiction $\conjunction{\CAPPSI}{\negation{\CAPPSI}}$. 
Before we had to be careful about setting up the right contradiction (recall derivation \ref{cangetlong}). %\pncmvref{cangetlong}). 
But now that we have rule \Rule{A.C.}, we don't need to be so careful. 
So long as we can get a contradiction, we can always use \Rule{A.C.} to get whatever other contradiction we need to make the derivation work.
 
Returning to the proof, we need to work top-down from the disjunction on line 2 to get to a contradiction. So we use the basic strategy:
\begin{gproof}
\galineNC{1}{$\conjunction{\negation{\CAPPHI}}{\negation{\CAPTHETA}}$}{\Rule{Assume}}
\gaaproof{
\gaalineNCS{2}{$\pardisjunction{\CAPPHI}{\CAPTHETA}$}{\Rule{Assume}}
\gaalineNCndS{}{}{}
\gaalineNCndS{}{$\qquad\vdots$}{}
\gaalineNCndS{}{}{}
\gaalineNCS{$\integer{n}-4$}{$\horseshoe{\CAPPHI}{\parconjunction{\CAPPSI}{\negation{\CAPPSI}}}$}{ }
\gaalineNCS{$\integer{n}-3$}{$\horseshoe{\CAPTHETA}{\parconjunction{\CAPPSI}{\negation{\CAPPSI}}}$}{ }
\gaalineNCS{$\integer{n}-2$}{$\conjunction{\CAPPSI}{\negation{\CAPPSI}}$}{\Rule{$\VEE$-Elim}, 2,$\integer{n}-4$,$\integer{n}-3$}
}
\galineNC{$\integer{n}-1$}{$\horseshoe{\pardisjunction{\CAPPHI}{\CAPTHETA}}{\parconjunction{\CAPPSI}{\negation{\CAPPSI}}}$}{\Rule{$\HORSESHOE$-Intro}, 2--$\integer{n}-2$}
\galineNC{$\integer{n}$}{$\negation{\pardisjunction{\CAPPHI}{\CAPTHETA}}$}{\Rule{$\NEGATION$-Intro, $\integer{n}-1$}}
\end{gproof}
\noindent{}And from here we need to work bottom-up from the conditionals on lines $\integer{n}-4$ and $\integer{n}-3$. 
\begin{gproof}
\galineNC{1}{$\conjunction{\negation{\CAPPHI}}{\negation{\CAPTHETA}}$}{\Rule{Assume}}
\gaaproof{
\gaalineNCS{2}{$\pardisjunction{\CAPPHI}{\CAPTHETA}$}{\Rule{Assume}}

\gaaaproof{
\gaaalineS{3}{$\CAPPHI$}{\Rule{Assume}}
\gaaalinendS{}{}{}
\gaaalinendS{}{$\qquad\vdots$}{}
\gaaalinendS{}{}{}
\gaaalineS{$\integer{m}$}{$\parconjunction{\CAPPSI}{\negation{\CAPPSI}}$}{}
}
\gaalineNCS{$\integer{m}+1$}{$\horseshoe{\CAPPHI}{\parconjunction{\CAPPSI}{\negation{\CAPPSI}}}$}{\Rule{$\HORSESHOE$-Intro}, 3-$\integer{m}$}

\gaaaproof{
\gaaalineS{$\integer{m}+2$}{$\CAPTHETA$}{\Rule{Assume}}
\gaaalinendS{}{}{}
\gaaalinendS{}{$\qquad\vdots$}{}
\gaaalinendS{}{}{}
\gaaalineS{$\integer{n}-4$}{$\parconjunction{\CAPPSI}{\negation{\CAPPSI}}$}{}
}
\gaalineNCS{$\integer{n}-3$}{$\horseshoe{\CAPTHETA}{\parconjunction{\CAPPSI}{\negation{\CAPPSI}}}$}{\Rule{$\HORSESHOE$-Intro}, $\integer{m}+2$--$\integer{n}-4$}

\gaalineNCS{$\integer{n}-2$}{$\conjunction{\CAPPSI}{\negation{\CAPPSI}}$}{\Rule{$\VEE$-Elim}, 2,$\integer{m}+1$,$\integer{n}-3$}
}
\galineNC{$\integer{n}-1$}{$\horseshoe{\pardisjunction{\CAPPHI}{\CAPTHETA}}{\parconjunction{\CAPPSI}{\negation{\CAPPSI}}}$}{\Rule{$\HORSESHOE$-Intro}, 2--$\integer{n}-2$}
\galineNC{$\integer{n}$}{$\negation{\pardisjunction{\CAPPHI}{\CAPTHETA}}$}{\Rule{$\NEGATION$-Intro, $\integer{n}-1$}}
\end{gproof}
And now we can finish the proof by working top-down, breaking apart the conjunction on line 1 and using \Rule{A.C.}
\begin{gproof}[\label{DeMDerivationSchema}]
\galineNC{1}{$\conjunction{\negation{\CAPPHI}}{\negation{\CAPTHETA}}$}{\Rule{Assume}}
\gaaproof{
\gaalineNCS{2}{$\pardisjunction{\CAPPHI}{\CAPTHETA}$}{\Rule{Assume}}

\gaaaproof{
\gaaalineS{3}{$\CAPPHI$}{\Rule{Assume}}
\gaaalinendS{4}{$\negation{\CAPPHI}$}{\Rule{$\WEDGE$-Elim}, 1}
\gaaalinendS{5}{$\conjunction{\negation{\CAPPHI}}{\CAPPHI}$}{\Rule{$\WEDGE$-Intro}, 3,4}
\gaaalineS{6}{$\parconjunction{\CAPPSI}{\negation{\CAPPSI}}$}{\Rule{A.C.}, 5}
}
\gaalineNCS{7}{$\horseshoe{\CAPPHI}{\parconjunction{\CAPPSI}{\negation{\CAPPSI}}}$}{\Rule{$\HORSESHOE$-Intro}, 3--6}

\gaaaproof{
\gaaalineS{8}{$\CAPTHETA$}{\Rule{Assume}}
\gaaalinendS{9}{$\negation{\CAPTHETA}$}{\Rule{$\WEDGE$-Elim}, 1}
\gaaalinendS{10}{$\conjunction{\negation{\CAPTHETA}}{\CAPTHETA}$}{\Rule{$\WEDGE$-Intro}, 8,9}
\gaaalineS{11}{$\parconjunction{\CAPPSI}{\negation{\CAPPSI}}$}{\Rule{A.C.}, 10}
}
\gaalineNCS{12}{$\horseshoe{\CAPTHETA}{\parconjunction{\CAPPSI}{\negation{\CAPPSI}}}$}{\Rule{$\HORSESHOE$-Intro}, 8--11}

\gaalineNCS{13}{$\conjunction{\CAPPSI}{\negation{\CAPPSI}}$}{\Rule{$\VEE$-Elim}, 2,7,12}
}
\galineNC{14}{$\horseshoe{\pardisjunction{\CAPPHI}{\CAPTHETA}}{\parconjunction{\CAPPSI}{\negation{\CAPPSI}}}$}{\Rule{$\HORSESHOE$-Intro}, 2--13}
\galineNC{15}{$\negation{\pardisjunction{\CAPPHI}{\CAPTHETA}}$}{\Rule{$\NEGATION$-Intro}, 14}
\end{gproof}
And so we now have a derivation schema showing how to derive a sentence of the form $\negation{\pardisjunction{\CAPPHI}{\CAPTHETA}}$ from one of the form $\conjunction{\negation{\CAPPHI}}{\negation{\CAPTHETA}}$. Note that we could have slightly shortened the derivation schema by doing \Rule{$\HORSESHOE$-Intro} on lines 3--5 instead of 3--6, using $\conjunction{\negation{\CAPPHI}}{\CAPPHI}$ as our contradiction instead of $\horseshoe{\CAPPHI}{\parconjunction{\CAPPSI}{\negation{\CAPPSI}}}$. In this case would have then used \Rule{A.C.} on line 11 to get $\conjunction{\negation{\CAPPHI}}{\CAPPHI}$. Similarly, we could have also used $\conjunction{\negation{\CAPTHETA}}{\CAPTHETA}$ as our contradiction. We leave it to the reader to show what slight modifications would be needed to the derivation schema in this case. 
%$\sststile{}{}\horseshoe{\negation{\pardisjunction{\CAPPHI}{\CAPTHETA}}}{\parconjunction{\negation{\CAPPHI}}{\negation{\CAPTHETA}}}$

\bigskip
\noindent{}Finally, we should note that every application of every shortcut rule (whether a standard or exchange shortcut rule) is truth-preserving. 
(Recall def. \pmvref{Derivation Rule Soundness}.) 
Although you might be tempted to try to prove this as a corollary to theorem \pmvref{Soundess of Basic GSD Rules}, and theorem \pmvref{GSD Shortcut Theorem2}, there's no easy way to get from (i) the claim that every application of every basic rule of \GSD{} is truth-preserving and (ii) the claim that every application of every rule of \GSDP{} is derivable in \GSD{}, to the further claim that every application of every rule of \GSDP{} is truth-preserving. 
Instead, the easiest way to prove this result more or less follows the lines of the proof for theorem \mvref{Soundess of Basic GSD Rules}.
\begin{THEOREM}{\LnpTC{Soundness of Std Shortcut Applications}}
Every application of every rule of \GSDP{}, including both standard and exchange shortcut rules, is truth-preserving.
\end{THEOREM}
\begin{PROOF}
Adjusting the proof of theorem \ref{Soundess of Basic GSD Rules} to work for the rules of \GSDP{} is left for the reader. 
The key is extending the truth-preservation lemma mentioned there to the rules of \GSDP{}.
We asked the reader to show this in exercises \mvref{exercises:truth-preservation lemma} and \mvref{exercises:GSDTFETheorem}.
\end{PROOF}

\subsection{Exchange Shortcut Rules}\label{Exchange Shortcut Rules GSD}
Recall our restriction in section \ref{Restrictions on Applying Rules} on the basic rules of \GSD{}, which said that a rule only sanctions writing down a sentence if the connectives it mentions are the main connectives of sentences on the lines to which it's applied. 
We put this restriction in place because not every application of the basic rules is truth-preserving without it. 
But with the restriction in place, every application of the basic rules becomes truth-preserving (theorem \pmvref{Soundess of Basic GSD Rules}). 
When we introduced the shortcut rules we carried over the restriction because, again, without it there are some shortcut rules with nontruth-preserving applications. 
But for some of our shortcut rules, the exchange shortcut rules (table \pmvref{GSDplus2}), every application is truth-preserving even without the restriction. 
%(The shortcut rules we've called standard, in table \pmvref{GSDplus1}, still need the restriction.)
To be explicit, for the exchange shortcut rules we can use the following definition of sanctioning (while still using def. \pncmvref{RuleSanctioning} for the basic rules and the standard exchange rules):
\begin{majorILnc}{\LnpDC{ExchangeRuleSanctioning}}
An exchange shortcut rule \Rule{R} from \GSDP{}, applied to a line with sentence $\CAPPSI$, \emph{sanctions} writing down sentence $\CAPPSI^*$ \Iff
\begin{cenumerate}
\item there is some substitution of \GSL{} sentences that, for the given schema of \Rule{R}, results in a sentence $\CAPPHI$ and, for the may-add schema, results in a sentence $\CAPPHI^*$,
\item $\CAPPHI$ is a subsentence of $\CAPPSI$, and
\item $\CAPPSI^*$ is the \GSL{} sentence you get when you replace one instance (token) of $\CAPPHI$ with an instance (token) of $\CAPPHI^*$ in $\CAPPSI$. 
\end{cenumerate}
\end{majorILnc}
\noindent{}Note that right below derivation \mvref{cangetlong3} is a detailed example of how the definition works in practice. 
With a moments thought it should be clear that if $\CAPPHI$ actually is $\CAPPSI$, then (recalling that all the exchange rules only have one given schema) this definition lines up with the original definition \mvref{RuleSanctioning}. This is just what we would expect.
\begin{THEOREM}{\LnpTC{ExchangeRuleGSDSoundness}}
Every application of every exchange shortcut rule from \GSDP{} is truth-preserving, even if we extend the notion of sanctioning for them with definition \ref{ExchangeRuleSanctioning}. 
\end{THEOREM} 
\noindent{}Note that this result was not already stated in theorem \mvref{Soundness of Std Shortcut Applications}. 
Although theorem \ref{Soundness of Std Shortcut Applications} says that every application of every rule of \GSDP{} is truth-preserving, truth-preservation is defined in terms of sanctioning (def. \ref{Derivation Rule Soundness}).
And, in theorem \ref{Soundness of Std Shortcut Applications} it was assumed that the exchange shortcut rules only sanctioned writing down a sentence if they were applied to whole sentences on lines. 
But now we're extending the notion of sanctioning for the exchange shortcut rules with definition \ref{ExchangeRuleSanctioning}. 
Theorem \ref{ExchangeRuleGSDSoundness} notes that even if we do this, the exchange shortcut rules are still truth-preserving.

We will use theorem \mvref{TFE Replacement} and the following theorem to prove theorem \ref{ExchangeRuleGSDSoundness}. This theorem will play the same rule as the truth-preservation lemma from the proof of theorem \mvref{Soundess of Basic GSD Rules}. Theorem \ref{TFE Replacement} is actually stronger than what we need, but since we already have it we will use it.
\begin{THEOREM}{\LnpTC{ExchangeRuleGSDSoundnessLemma}}
For all exchange shortcut rules \Rule{R} from \GSDP{}, if $\CAPPHI$ and $\CAPPHI^*$ are the sentences you get after substituting \GSL{} sentences into the given and may-add schemas of \Rule{R}, respectively, then $\CAPPHI$ and $\CAPPHI^*$ are truth functionally equivalent. 
\end{THEOREM}
\begin{PROOF}
This theorem follows immediately from what the reader showed in exercises \mvref{exercises:GSDTFETheorem}. 
\end{PROOF}
\begin{PROOFOF}{Thm. \ref{ExchangeRuleGSDSoundness}}
Consider some arbitrary application of some exchange shortcut rule \Rule{R} from \GSDP{}. 
Say that in this application \Rule{R} is applied to sentence $\CAPPSI$ and permits, or sanctions, you to write down $\CAPPSI^*$. 
By definition \mvref{ExchangeRuleSanctioning}, (i) there is some substitution of \GSL{} sentences that, for the given schema of \Rule{R}, results in a sentence $\CAPPHI$ and, for the may-add schema, results in a sentence $\CAPPHI^*$, (ii) $\CAPPHI$ is a subsentence of $\CAPPSI$, and (iii) $\CAPPSI^*$ is the \GSL{} sentence you get when you replace one instance (token) of $\CAPPHI$ with an instance (token) of $\CAPPHI^*$ in $\CAPPSI$. 
From (i) and theorem \ref{ExchangeRuleGSDSoundnessLemma}, $\CAPPHI$ and $\CAPPHI^*$ are truth functionally equivalent.
So, from theorem \mvref{TFE Replacement} it follows that $\CAPPSI$ and $\CAPPSI^*$ are also truth functionally equivalent.
From the definition \mvref{GSL TFE} of truth functional equivalence it follows that $\CAPPSI\sdtstile{}{}\CAPPSI^*$.
So, by definition \mvref{Derivation Rule Soundness}, this application is truth-preserving. 
\end{PROOFOF}

\bigskip
\noindent{}The intuitive idea behind definition \mvref{ExchangeRuleSanctioning} is that the exchange shortcut rules can be applied to subsentences on lines (while the basic rules and standard shortcut rules can only be applied to whole sentences on lines). 
Some examples will hopefully make things more clear. 
Recall derivation \pmvref{cangetlong}, which showed that \mbox{$\sststile{}{}\;\negation{\parconjunction{\bpardisjunction{\negation{\Al}}{\negation{\Bl}}}{\bparconjunction{\Al}{\Bl}}}$}. 
First, consider how we might rewrite this proof using \Rule{DeMorgans}, in addition to the basic rules of \GSD{}, but only applying it to the whole sentence on a line. 
\begin{gproof}[\label{cangetlong2}]
\gaproof{
\galine{1}{$\conjunction{\bpardisjunction{\negation{\Al}}{\negation{\Bl}}}{\bparconjunction{\Al}{\Bl}}$}{\Rule{Assume}}
\galine{2}{$\bpardisjunction{\negation{\Al}}{\negation{\Bl}}$}{\Rule{$\WEDGE$-Elim}, 1}
\galine{3}{$\bparconjunction{\Al}{\Bl}$}{\Rule{$\WEDGE$-Elim}, 1}
\galine{4}{$\negation{\bparconjunction{\Al}{\Bl}}$}{\Rule{DeM}, 2}
\galine{5}{$\conjunction{\bparconjunction{\Al}{\Bl}}{\negation{\bparconjunction{\Al}{\Bl}}}$}{\Rule{$\WEDGE$-Intro}, 3,4}
}
\gline{6}{$\horseshoe{\parconjunction{\bpardisjunction{\negation{\Al}}{\negation{\Bl}}}{\bparconjunction{\Al}{\Bl}}}{\cparconjunction{\bparconjunction{\Al}{\Bl}}{\negation{\bparconjunction{\Al}{\Bl}}}}$}{\Rule{$\HORSESHOE$-Intro}, 1--5}
\gline{7}{$\negation{\parconjunction{\bpardisjunction{\negation{\Al}}{\negation{\Bl}}}{\bparconjunction{\Al}{\Bl}}}$}{\Rule{$\NEGATION$-Intro}, 6}
\end{gproof}
Obviously this is much shorter than \ref{cangetlong}, and much shorter than \ref{cangetlong} would be even if we rewrote it using \Rule{A.C.} (as we suggested the reader do above).
Here we have applied \Rule{DeMorgans} to line 2. 
But if we allow ourselves to apply \Rule{DeMorgans} to subsentences on lines in addition to whole sentences, then the proof can be made even shorter. 
\begin{gproof}[\label{cangetlong3}]
\gaproof{
\galine{1}{$\conjunction{\bpardisjunction{\negation{\Al}}{\negation{\Bl}}}{\bparconjunction{\Al}{\Bl}}$}{\Rule{Assume}}
\galine{2}{$\conjunction{\negation{\bparconjunction{\Al}{\Bl}}}{\bparconjunction{\Al}{\Bl}}$}{\Rule{DeM}, 1}
}
\gline{3}{$\horseshoe{\parconjunction{\bpardisjunction{\negation{\Al}}{\negation{\Bl}}}{\bparconjunction{\Al}{\Bl}}}{\parconjunction{\negation{\bparconjunction{\Al}{\Bl}}}{\bparconjunction{\Al}{\Bl}}}$}{\Rule{$\HORSESHOE$-Intro}, 1--2}
\gline{4}{$\negation{\parconjunction{\bpardisjunction{\negation{\Al}}{\negation{\Bl}}}{\bparconjunction{\Al}{\Bl}}}$}{\Rule{$\NEGATION$-Intro}, 3}
\end{gproof}
Unlike in \ref{cangetlong2}, here we did not need to break line 1 apart into its conjuncts. We simply applied \Rule{DeMorgans} to the first conjunct of line 1 and wrote the result down as line 2. To see how definition \mvref{ExchangeRuleSanctioning} captures this intuitive idea, consider how we would show, directly from the definition, that \Rule{DeMorgans} sanctions writing $\CAPPSI^*=\conjunction{\negation{\bparconjunction{\Al}{\Bl}}}{\bparconjunction{\Al}{\Bl}}$ on line 2 when applied to $\CAPPSI=\conjunction{\bpardisjunction{\negation{\Al}}{\negation{\Bl}}}{\bparconjunction{\Al}{\Bl}}$ on line 1. The given schema for \Rule{DeMorgans} relevant to line 1 is $\disjunction{\negation{\CAPPHI_1}}{\negation{\CAPPHI_2}}$, while the relevant may-add schema is $\negation{\parconjunction{\CAPPHI_1}{\CAPPHI_2}}$. The substitution we want for clause (1) of definition \ref{ExchangeRuleSanctioning} is $\CAPPHI_1=\Al$ and $\CAPPHI_2=\Bl$. This substitution results in $\CAPPHI=\bpardisjunction{\negation{\Al}}{\negation{\Bl}}$ and $\CAPPHI^*=\;\negation{\bparconjunction{\Al}{\Bl}}$. In line with clause (2) of the definition, $\CAPPHI=\bpardisjunction{\negation{\Al}}{\negation{\Bl}}$ is a subsentence of $\CAPPSI=\conjunction{\bpardisjunction{\negation{\Al}}{\negation{\Bl}}}{\bparconjunction{\Al}{\Bl}}$. And, in line with clause (3), $\CAPPSI^*=\conjunction{\negation{\bparconjunction{\Al}{\Bl}}}{\bparconjunction{\Al}{\Bl}}$ is the \GSL{} sentence you get when you replace one instance (token) of $\CAPPHI=\bpardisjunction{\negation{\Al}}{\negation{\Bl}}$ with an instance (token) of $\CAPPHI^*=\;\negation{\bparconjunction{\Al}{\Bl}}$ in $\CAPPSI=\conjunction{\bpardisjunction{\negation{\Al}}{\negation{\Bl}}}{\bparconjunction{\Al}{\Bl}}$.

\bigskip
\noindent{}We have shown that the exchange shortcut rules, more liberally allowed to be applied to subsentences on a line, still have truth-preserving applications (Thm. \ref{ExchangeRuleGSDSoundness}).
In the last section, \mvref{Standard Shortcut Rules GSD}, we showed that anything we can derive using the standard and exchange shortcut rules from \GSDP{} can be derived using only the basic rules of \GSD{} (Thm. \pmvref{GSD Shortcut Theorem3}). 
But to prove this we used theorem \mvref{GSD Shortcut Theorem} and theorem \ref{GSD Shortcut Theorem2} and we need to think about how more liberally allowing the exchange shortcut rules to be applied to subsentences on a line affects the proofs of these two theorems. 
That is, if we want to show that theorem \ref{GSD Shortcut Theorem3} still holds, we need to show that the theorems we used to prove it still hold.

It should not be hard to see that the change from definition \mvref{RuleSanctioning} to \mvref{ExchangeRuleSanctioning}, i.e. the move to more liberal applications of exchange shortcut rules, does not affect the proof of \mvref{GSD Shortcut Theorem}. 
But it does affect the proof of \ref{GSD Shortcut Theorem2}. 
We proved this theorem (or, rather, asked the reader to prove most of the cases) by giving, for each rule \Rule{R}, a derivation schema using only the basic rules of \GSD{} that has the given schemas of \Rule{R} as the premises and the may-add schema of \Rule{R} as the conclusion (as the last line). 
This was sufficient to show that every application of the rules of \GSDP{}, if restricted to whole sentences on lines, is derivable from the basic rules of \GSD{} because if the rules are only being applied to whole sentences, then these derivation schema will yield derivations that use only basic rules for every application. 
But this will not work if we're more liberally allowing the exchange rules to be applied to subsentences on a line.

For example, consider derivation schema \mvref{DeMDerivationSchema} for \Rule{DeMorgans}. 
Say that we have the sentence $\disjunction{\Al}{\parconjunction{\negation{\Bl}}{\negation{\Cl}}}$ on some line of a derivation on which we're working and we want to apply \Rule{DeMorgans} to the right disjunct. 
Under the more liberal definition \mvref{ExchangeRuleSanctioning} we can do this, and \Rule{DeMorgans} will permit, or sanction, us to write down $\disjunction{\Al}{\negation{\pardisjunction{\Bl}{\Cl}}}$. 
But it should be clear that substituting $\CAPPHI=\Bl$ and $\CAPTHETA=\Cl$ into derivation schema \ref{DeMDerivationSchema} will not result in a derivation of $\disjunction{\Al}{\negation{\pardisjunction{\Bl}{\Cl}}}$ from $\disjunction{\Al}{\parconjunction{\negation{\Bl}}{\negation{\Cl}}}$. 

To show that theorem \ref{GSD Shortcut Theorem2} still holds for our more liberal use of exchange shortcut rules we will rely on two facts. 
First, definition \ref{ExchangeRuleSanctioning} ensures that if an exchange shortcut rule, applied to a sentence $\CAPPSI$, sanctions writing down $\CAPPSI^*$, then there's a specific relationship between $\CAPPSI$ and $\CAPPSI^*$. 
Specifically, there are two sentences $\CAPPHI$ and $\CAPPHI^*$, got by substituting sentences into the given and may-add schemas of the rules, and $\CAPPSI^*$ is $\CAPPSI$ with $\CAPPHI$ replaced with $\CAPPHI^*$. 
The second fact is that these sentences $\CAPPHI$ and $\CAPPHI^*$ are always \niidf{provably equivalent}, or as we might say \niidf{derivationally equivalent}\index{derivationally equivalent|see{provabbly equivalent}}.\footnote{Compare 
this definition with definition \mvref{GQL Provably Equivalent}, which generalizes it for formulas of \GQL{}.} 
(This is proved in exercise \pmvref{exercisesGSDshortcutrules}.)
\begin{majorILnc}{\LnpDC{GSDprovablyequivalent}}
Two sentences of \GSL{} are \nidf{provably equivalent}\index{provably equivalent!sentences of \GSL{}|textbf} \Iff one of the following two equivalent conditions holds:
\begin{cenumerate}
\item both $\CAPPHI\sststile{}{}\CAPPSI$ and $\CAPPSI\sststile{}{}\CAPPHI$, or
\item $\sststile{}{}\triplebar{\CAPPHI}{\CAPPSI}$.
\end{cenumerate}
\end{majorILnc}

These two facts, along with the following theorem,\footnote{Compare 
this theorem to theorem \mvref{GQD Replacement Theorem}, which is a stronger version of it generalized to \GQD{}. 
Note that the Restricted Replacement Theorem for \GSD{} follows immediately from the generalized \GQD{} version (but we provide a separate proof here). 
Also note that the proof of theorem \ref{GQD Replacement Theorem} uses the One-step Replacement Lemmas (Thm. \pmvref{OneStepReplacementLemmas}), and the proofs for these involve constructing derivation schemas. The proof given here for the Restricted Replacement Theorem for \GSD{} also involves the construction of derivation schemas, but does so by giving instructions in the inheritance step for how to write the relevant derivations instead of explicitly writing them out in a separate lemma.} 
are enough to show that theorem \ref{GSD Shortcut Theorem2} still holds for our more liberal use of exchange shortcut rules. (The reader should make sure they are convinced of that.)
\begin{THEOREM}{\LnpTC{ExchangeRuleTheorem} Restricted Replacement Theorem for \GSD{}:}
For all sentences $\CAPPSI$ of \GSL{}: if
\begin{cenumerate}
\item $\CAPPHI$ and $\CAPPHI^*$ are \GSL{} sentences such that $\CAPPHI\sststile{}{}\CAPPHI^*$ and $\CAPPHI^*\sststile{}{}\CAPPHI$, and
\item if $\CAPPHI$ is a subsentence of $\CAPPSI$, then $\CAPPSI^*$ is the \GSL{} sentence you get when you replace one instance (token) of $\CAPPHI$ with an instance (token) of $\CAPPHI^*$, and $\CAPPSI^*$ is $\CAPPSI$ if not, 
\end{cenumerate}
then $\CAPPSI^*$ can be derived from $\CAPPSI$ using only the basic rules of \GSD{}, i.e. $\CAPPSI\sststile{}{}\CAPPSI^*$.
\end{THEOREM}
\begin{PROOF}
Assume that $\CAPPHI$ and $\CAPPHI^*$ are two \GSL{} sentences such that $\CAPPHI\sststile{}{}\CAPPHI^*$ and $\CAPPHI^*\sststile{}{}\CAPPHI$.
\begin{description}
\item[Base Step:] If $\CAPPSI$ is atomic, then it's just a sentence letter. 
So, if $\CAPPHI$ is a subsentence of $\CAPPSI$, it itself must just be that same sentence letter. 
So, $\CAPPHI$ is the same sentence as $\CAPPSI$, and $\CAPPSI^*$ is the same as $\CAPPHI^*$. 
Since $\CAPPHI\sststile{}{}\CAPPHI^*$, it follows immediately that $\CAPPSI\sststile{}{}\CAPPSI^*$.
\item[Inheritance Step:] For the recursive hypothesis, assume that the theorem holds for the sentences $\CAPTHETA,\CAPTHETA_1,\ldots,\CAPTHETA_n,\DELTA$.
\begin{description}
\item[Conjunction:] Assume that $\CAPPSI$ is the conjunction $\conjunction{\CAPTHETA_1}{\conjunction{\ldots}{\CAPTHETA_{\integer{n}}}}$. 
Either $\CAPPHI$ is not a subsentence of $\CAPPSI$, it's a subsentence of $\CAPPSI$ but not the same as $\CAPPSI$, or is the same as $\CAPPSI$. 
If it's not a subsentence of $\CAPPSI$ or it's the same as $\CAPPSI$, then just as in the base step it follows immediately that $\CAPPSI\sststile{}{}\CAPPSI^*$.

So assume that $\CAPPHI$ is a subsentence of $\CAPPSI$ but not the same as it. 
Then $\CAPPHI$ is a subsentence of one of the conjuncts $\CAPTHETA_{\integer{i}}$ of $\CAPPSI$ and $\CAPPSI^*$ is the conjunction $\conjunction{\CAPTHETA_1}{\conjunction{\ldots}{\conjunction{\CAPTHETA_{\integer{i}}^*}{\conjunction{\ldots}{\CAPTHETA_{\integer{n}}}}}}$. 
By the recursive hypothesis, $\CAPTHETA_{\integer{i}}\sststile{}{}\CAPTHETA_{\integer{i}}^*$. 
It should be clear that by using \Rule{$\WEDGE$-Elim} we have that $\CAPPSI\sststile{}{}\CAPTHETA_1$, $\ldots$, $\CAPPSI\sststile{}{}\CAPTHETA_{\integer{i}}$, $\ldots$, $\CAPPSI\sststile{}{}\CAPTHETA_{\integer{n}}$. 
By transitivity, $\CAPPSI\sststile{}{}\CAPTHETA_{\integer{i}}^*$.
And, it should be clear that if each of $\CAPTHETA_1,\ldots,\CAPTHETA_{\integer{i}}^*,\ldots,\CAPTHETA_{\integer{n}}$ can be derived from $\CAPPSI$, then by using \Rule{$\WEDGE$-Intro} we can derive their conjunction $\conjunction{\CAPTHETA_1}{\conjunction{\ldots}{\conjunction{\CAPTHETA_{\integer{i}}^*}{\conjunction{\ldots}{\CAPTHETA_{\integer{n}}}}}}$ from $\CAPPSI$. 
But this conjunction just is $\CAPPSI^*$, so $\CAPPSI\sststile{}{}\CAPPSI^*$.

\item[Disjunction:] Assume that $\CAPPSI$ is the disjunction $\disjunction{\CAPTHETA_1}{\disjunction{\ldots}{\CAPTHETA_{\integer{n}}}}$. 
Either $\CAPPHI$ is not a subsentence of $\CAPPSI$, it's a subsentence of $\CAPPSI$ but not the same as $\CAPPSI$, or is the same as $\CAPPSI$. 
If it's not a subsentence of $\CAPPSI$ or it's the same as $\CAPPSI$, then just as in he base step it follows immediately that $\CAPPSI\sststile{}{}\CAPPSI^*$.

So assume that $\CAPPHI$ is a subsentence of $\CAPPSI$ but not the same as it. 
Then $\CAPPHI$ is a subsentence of one of the disjuncts $\CAPTHETA_{\integer{i}}$ of $\CAPPSI$ and $\CAPPSI^*$ is the disjunction $\disjunction{\CAPTHETA_1}{\disjunction{\ldots}{\disjunction{\CAPTHETA_{\integer{i}}^*}{\disjunction{\ldots}{\CAPTHETA_{\integer{n}}}}}}$.

We want to show that $\CAPPSI\sststile{}{}\CAPPSI^*$, i.e. that $\disjunction{\CAPTHETA_1}{\disjunction{\ldots}{\CAPTHETA_{\integer{n}}}}\sststile{}{}\disjunction{\CAPTHETA_1}{\disjunction{\ldots}{\disjunction{\CAPTHETA_{\integer{i}}^*}{\disjunction{\ldots}{\CAPTHETA_{\integer{n}}}}}}$. 
Using the proof by contradiction strategy, we write $\CAPPSI$ on the first line and write $\negation{\CAPPSI^*}$ on line 2 as an assumption with the goal of deriving a contradiction. 
Now we apply \Rule{DeMorgans} to line 2, getting $\conjunction{\negation{\CAPTHETA_1}}{\conjunction{\ldots}{\conjunction{\negation{\CAPTHETA_{\integer{i}}^*}}{\conjunction{\ldots}{\negation{\CAPTHETA_{\integer{n}}}}}}}$. 
Now using \Rule{$\WEDGE$-Elim} we break apart this conjunction, getting each conjunct $\negation{\CAPTHETA_1},\ldots,\negation{\CAPTHETA_{\integer{i}}^*},\ldots,\negation{\CAPTHETA_{\integer{n}}}$ on a separate line. 
Then using these conjuncts and \Rule{Disjunctive Syllogism} on line 1 to get $\CAPTHETA_{\integer{i}}$ on its own line. 
By the recursive hypothesis, $\CAPTHETA_{\integer{i}}\sststile{}{}\CAPTHETA_{\integer{i}}^*$, so from the line with $\CAPTHETA_{\integer{i}}$ we can derive $\CAPTHETA_{\integer{i}}^*$. 
Since we already have $\negation{\CAPTHETA_{\integer{i}}^*}$ on its own line, we can use \Rule{$\WEDGE$-Intro} to get $\conjunction{\CAPTHETA_{\integer{i}}^*}{\negation{\CAPTHETA_{\integer{i}}^*}}$. 
We then close the assumption on line 2 by using \Rule{$\HORSESHOE$-Intro} to get $\horseshoe{\negation{\CAPPSI^*}}{\parconjunction{\CAPTHETA_{\integer{i}}^*}{\negation{\CAPTHETA_{\integer{i}}^*}}}$. 
We finally use \Rule{$\NEGATION$-Elim} to get $\CAPPSI^*$.

\item[Negation:] Assume that $\CAPPSI$ is the negation $\negation{\CAPTHETA}$. 
Either $\CAPPHI$ is not a subsentence of $\CAPPSI$, it's a subsentence of $\CAPPSI$ but not the same as $\CAPPSI$, or is the same as $\CAPPSI$. 
If it's not a subsentence of $\CAPPSI$ or it's the same as $\CAPPSI$, then just as in he base step it follows immediately that $\CAPPSI\sststile{}{}\CAPPSI^*$.

So assume that $\CAPPHI$ is a subsentence of $\CAPPSI$ but not the same as it. 
Then $\CAPPHI$ is a subsentence of $\CAPTHETA$ and $\CAPPSI^*$ is the negation $\negation{\CAPTHETA^*}$.

We want to show that $\CAPPSI\sststile{}{}\CAPPSI^*$, i.e. that $\negation{\CAPTHETA}\sststile{}{}\;\negation{\CAPTHETA^*}$. Using our usual basic bottom-up strategy for negation, we set up this proof by putting $\negation{\CAPTHETA}$ on the first line and assuming $\CAPTHETA^*$ with the goal of deriving a contradiction. But, by the recursive hypothesis, $\CAPTHETA^*\sststile{}{}\CAPTHETA$. So we know that from assumption $\CAPTHETA^*$ we can derive $\CAPTHETA$, and at that point we'll have $\negation{\CAPTHETA}$ on one line and $\CAPTHETA$ on another. Using \Rule{$\WEDGE$-Intro} we can get $\conjunction{\CAPTHETA}{\negation{\CAPTHETA}}$. Then with \Rule{$\HORSESHOE$-Intro} we'll get $\horseshoe{\CAPTHETA^*}{\parconjunction{\CAPTHETA}{\negation{\CAPTHETA}}}$, and applying \Rule{$\NEGATION$-Intro} to this sentence will get us $\negation{\CAPTHETA^*}$. 

\item[Conditional:] Left to the reader as an exercise.

\item[Biconditional:] Also left to the reader as an exercise.

\end{description}
\item[Closure Step:] Since the inheritance step covers all the ways to generate \GSL{} sentences, we've shown that the theorem holds for all \GSL{} sentences $\CAPPSI$.  
\end{description}
\end{PROOF}

\noindent{}Thus we have shown that anything we can derive in \GSDP{} can be derived in \GSD{}.

\subsection{Shortcut Rule Strategies}\label{Sec:Shortcut Rule Strategies}
As we discussed in section \ref{Sec:Some Strategies}, for each logical connective there are two types of strategies: those for what to do if you already have sentences with that as their main connective (top-down strategies), and those for what to do if you want to get a sentence with that as its main connective (bottom-up strategies). 
In section \ref{Sec:Some Strategies} we covered basic top-down and bottom-up strategies for each connective.
We now add new strategies based on shortcut rules to this basic stock.
(Since shortcut rules don't divide nicely by connective, and often involve schemas with multiple connectives, some of the groupings here a bit arbitrary; but this isn't a substantial issue.)

\subsubsection*{Conjunction} 
\begin{description}
\item[\Rule{DeM} Top-down:] If you have a sentence of the form $\conjunction{\negation{\CAPPHI_1}}{\conjunction{\ldots}{\negation{\CAPPHI_{\integer{n}}}}}$, then convert it using \Rule{DeM} to get $\negation{\pardisjunction{\CAPPHI_1}{\disjunction{\ldots}{\CAPPHI_{\integer{n}}}}}$ on a new line.
\item[\Rule{$\NEGATION/\HORSESHOE$-Exchange} Top-down:] If you have a sentence of the form $\conjunction{\CAPPHI}{\negation{\CAPPSI}}$, then convert it using \Rule{$\NEGATION/\HORSESHOE$-Exchange} to get $\negation{\horseshoe{\CAPPHI}{\CAPPSI}}$ on a new line. 
\end{description} 
\subsubsection*{Disjunction} 
\begin{description}
\item[\Rule{D.S.} Top-down:] If you have a sentence of the form $\disjunction{\CAPPHI_1}{\disjunction{\ldots}{\disjunction{\CAPPHI_i}{\disjunction{\ldots}{\CAPPHI_{\integer{n}}}}}}$, and another sentence of the form $\negation{\CAPPHI_i}$, then eliminate one of the disjuncts using \Rule{D.S.} to get $\disjunction{\CAPPHI_1}{\disjunction{\ldots}{\disjunction{\CAPPHI_{i-1}}{\disjunction{\CAPPHI_{i+1}}{\disjunction{\ldots}{\CAPPHI_{\integer{n}}}}}}}$ on a new line.
\item[\Rule{DeM} Top-down:] If you have a sentence of the form $\disjunction{\negation{\CAPPHI_1}}{\disjunction{\ldots}{\negation{\CAPPHI_{\integer{n}}}}}$, then convert it using \Rule{DeM} to get $\negation{\parconjunction{\CAPPHI_1}{\conjunction{\ldots}{\CAPPHI_{\integer{n}}}}}$ on a new line.
\item[\Rule{$\HORSESHOE/\NEGATION$-Exchange} Top-down:] If you have a sentence of the form $\disjunction{\negation{\CAPPHI}}{\CAPPSI}$, then convert it using \Rule{$\HORSESHOE/\NEGATION$-Exchange} to get $\horseshoe{\CAPPHI}{\CAPPSI}$ on a new line.
\end{description} 
\subsubsection*{Negation} 
\begin{description}
\item[\Rule{DeM} Top-down:] If you have a sentence of the form $\negation{\parconjunction{\CAPPHI_1}{\conjunction{\ldots}{\CAPPHI_{\integer{n}}}}}$, then convert it using \Rule{DeM} to get $\disjunction{\negation{\CAPPHI_1}}{\disjunction{\ldots}{\negation{\CAPPHI_{\integer{n}}}}}$ on a new line.
\item[\Rule{DeM} Top-down:] If you have a sentence of the form $\negation{\pardisjunction{\CAPPHI_1}{\disjunction{\ldots}{\CAPPHI_{\integer{n}}}}}$, then convert it using \Rule{DeM} to get $\conjunction{\negation{\CAPPHI_1}}{\conjunction{\ldots}{\negation{\CAPPHI_{\integer{n}}}}}$ on a new line.
\item[\Rule{$\NEGATION\NEGATION$-Elim} Top-down:] If you have a sentence of the form $\negation{\negation{\CAPPHI}}$, then convert it using \Rule{$\NEGATION\NEGATION$-Elim} to get $\CAPPHI$ on a new line.
\item[\Rule{$\NEGATION/\HORSESHOE$-Exchange} Top-down:] If you have a sentence of the form $\negation{\parhorseshoe{\CAPPHI}{\CAPPSI}}$, then convert it using \Rule{$\NEGATION/\HORSESHOE$-Exchange} to get $\conjunction{\CAPPHI}{\negation{\CAPPSI}}$ on a new line.
\item[\Rule{$\NEGATION\NEGATION$-Intro} Bottom-up:] If you want to get a sentence of the form $\negation{\negation{\CAPPHI}}$, then first derive $\CAPPHI$ and then use \Rule{$\NEGATION\NEGATION$-Intro} to derive it. 
\end{description} 
\subsubsection*{Conditionals} 
\begin{description}
\item[\Rule{M.T.} Top-down:] If you have a sentence of the form $\horseshoe{\CAPPHI}{\CAPPSI}$, and another $\negation{\CAPPSI}$, then break the conditional apart using \Rule{M.T.} to get $\negation{\CAPPHI}$ on a new line.
\item[\Rule{$\HORSESHOE/\NEGATION$-Exchange} Top-down:] If you have a sentence of the form $\horseshoe{\CAPPHI}{\CAPPSI}$, then convert it using \Rule{$\HORSESHOE/\NEGATION$-Exchange} to get $\disjunction{\negation{\CAPPHI}}{\CAPPSI}$ on a new line.
\item[\Rule{Contraposition} Top-down:] If you have a sentence of the form $\horseshoe{\CAPPHI}{\CAPPSI}$, then convert it using \Rule{Contraposition} to get $\horseshoe{\negation{\CAPPSI}}{\negation{\CAPPHI}}$ on a new line.
\end{description} 
\subsubsection*{Biconditionals} 
\begin{description}
\item[\Rule{$\NEGATION/\TRIPLEBAR$-Intro} Top-down:] If you have a sentence of the form $\triplebar{\CAPPHI}{\CAPPSI}$, then convert it using \Rule{$\NEGATION/\TRIPLEBAR$-Intro} to get $\triplebar{\negation{\CAPPHI}}{\negation{\CAPPSI}}$ on a new line.
\end{description} 
\subsubsection*{Misc} 
\begin{description}
\item[\Rule{A.C.} Bottom-up:] If you want to get a sentence of the form $\CAPPSI$, then first derive both $\CAPPHI$ and $\negation{\CAPPHI}$ and then use \Rule{A.C.} to derive it from them. 
\end{description} 
\noindent{}Note that we've left strategies derived from \Rule{Distribution}. 
This isn't because they're not useful (quite the contrary). 
Instead, we leave it to the reader to place the appropriate Top-down strategies from \Rule{Distribution} under conjunction and disjunction. 

Also note that we've presented most of the strategies as top-down. 
For any of the strategies based on an exchange shortcut rule, it should be clear that you can ``reverse'' the strategy and read it as bottom-up. 

%%%%%%%%%%%%%%%%%%%%%%%%%%%%%%%%%%%%%%%%%%%%%%%%%%
\section{\GQD{}}\label{Section GQD}
%%%%%%%%%%%%%%%%%%%%%%%%%%%%%%%%%%%%%%%%%%%%%%%%%%

\subsection{Introduction and Elimination Rules}

Now our goal is to extend \GSD{} to  natural deduction system that will allow us to write derivations for \GQL{}. 
This system, which we call \idf{Quantificational Derivation System}, or \GQD{}, consists of all the rules of \GSD{}, plus an introduction and elimination rule for each quantifier (given in table \ref{GQD}). 
Like \GSD{}, we will extend \GQD{} by adding shortcut rules. 
First, we can extend \GQD{} by adding all the shortcut rules we gave for \GSD{}. 
Second, we can extend \GQD{} by adding shortcut rules specifically for the quantifiers (see table \ref{GQDplus}).
We'll call the system we get by adding all the shortcut rules from \GSD{} (tables \ref{GSDplus1} and \ref{GSDplus1}) and the new shortcut rules for the quantifiers (table \ref{GQDplus}) \GQDP{}.
Note that some of the rules for \GQD{} have special restrictions.
We explain these in the examples below. 
%Since there are only two new connectives, introduce rules, strategies and do examples all at once.

%\begin{table}[!ht]
%\renewcommand{\arraystretch}{1.5}
%\begin{center}
%\begin{tabular}{ p{1in} l l } %p{2.2in} p{2in}
%\toprule
%\textbf{Name} & \textbf{Given} & \textbf{May Add} \\ 
%\midrule 
\renewcommand{\arraystretch}{1.5}
\begin{longtable}[c]{ p{1in} l l } %p{2.2in} p{2in}
\toprule
\textbf{Name} & \textbf{Given} & \textbf{May Add} \\ 
\midrule
\endfirsthead
\multicolumn{3}{c}{\emph{Continued from Previous Page}}\\
\toprule
\textbf{Name} & \textbf{Given} & \textbf{May Add} \\ 
\midrule
\endhead
\bottomrule
\caption{(New) Basic Rules for \GQD{}}\\[-.15in]
\multicolumn{3}{c}{\emph{Continued next Page}}\\
\endfoot
\bottomrule
\caption{(New) Basic Rules for \GQD{}}\\
\endlastfoot
\label{GQD}\Rule{$\forall$-Elim} & $\universal{\BETA}\CAPPHI$ & $\CAPPHI\constant{a}/\BETA$, for \mention{a} any  \\[-.25cm]
\nopagebreak
 &   &   individual constant \\
\Rule{$\forall$-Intro} & $\CAPPHI\constant{a}/\BETA$ & $\universal{\BETA}\CAPPHI$, iff \mention{a} does  \\[-.25cm]
\nopagebreak
 &  &  not occur in $\CAPPHI$  \\[-.25cm]
 \nopagebreak
 &  & nor in any unboxed assumption \\
\Rule{$\exists$-Intro} & $\CAPPHI\constant{a}/\BETA$ & $\existential{\BETA}\CAPPHI$ \\
\Rule{$\exists$-Elim} & $\existential{\BETA}\CAPPHI$, $\horseshoe{\CAPPHI{\constant{a}/\BETA}}{\CAPTHETA}$ & $\CAPTHETA$, \Iff \mention{a} does \\[-.25cm]
\nopagebreak
 &  &  not occur in $\CAPPHI$ or $\CAPTHETA$, \\[-.25cm]
\nopagebreak
 & &  nor in any unboxed assumption\\
\end{longtable}
%\bottomrule
%\end{tabular}
%\end{center}
%\caption{(New) Basic Rules for \GQD{}}
%\label{GQD}
%\end{table}

The mechanics of writing proofs in \GQD{} are no different than the mechanics of writing proofs in \GSD{}, but we do have to slightly adapt the definition for sanctioning (Def. \pmvref{RuleSanctioning}).
\begin{majorILnc}{\LnpDC{GQDRuleSanctioning}}
A rule \Rule{R}, applied to unboxed lines $\integer{m}_1,\ldots,\integer{m}_{\integer{j}}$ with, respectively, sentences $\CAPPSI_1,\ldots,\CAPPSI_{\integer{j}}$, \df{sanctions} writing the sentence $\CAPPHI$ \Iff there's some substitution of \GQL{} \emph{formulas} that, for the given schemas of \Rule{R}, results in $\CAPPSI_1,\ldots,\CAPPSI_{\integer{j}}$ and, for the may-add schema, results in $\CAPPHI$. 
\end{majorILnc}
Another difference is that there are four new rules available. 
We now look at four examples which highlight each rule.

Our first example demonstrates \Rule{$\forall$-Elim}. 
Say we want to show that $\universal{\variable{x}}\parhorseshoe{\Ap{\variable{x}}}{\Bp{\variable{x}}},\Ap{\constant{d}}\sststile{}{}\Bp{\constant{d}}$.
We start by setting the two sentences on the \CAPS{lhs} of the turnstile as assumptions:
\begin{gproof}[\label{GQDExampleA}]
\galineNC{1}{$\universal{\variable{x}}\parhorseshoe{\Ap{\variable{x}}}{\Bp{\variable{x}}}$}{\Rule{Assume}}
\gaalineNC{2}{$\Ap{\constant{d}}$}{\Rule{Assume}}
\end{gproof}
Next we use \Rule{$\forall$-Elim} on line 1. 
Note that there are no restrictions on the use of \Rule{$\forall$-Elim}.
\begin{gproof}[\label{GQDExampleB}]
\galineNC{1}{$\universal{\variable{x}}\parhorseshoe{\Ap{\variable{x}}}{\Bp{\variable{x}}}$}{\Rule{Assume}}
\gaalineNC{2}{$\Ap{\constant{d}}$}{\Rule{Assume}}
\gaalineNC{3}{$\horseshoe{\Ap{\constant{d}}}{\Bp{\constant{d}}}$}{\Rule{$\forall$-Elim}, 1}
\end{gproof}
Although we could have substituted any constant for $\variable{x}$ in line 3, we chose $\constant{d}$ so that we can next apply \Rule{$\HORSESHOE$-Elim}. 
\begin{gproof}[\label{GQDExampleC}]
\galineNC{1}{$\universal{\variable{x}}\parhorseshoe{\Ap{\variable{x}}}{\Bp{\variable{x}}}$}{\Rule{Assume}}
\gaalineNC{2}{$\Ap{\constant{d}}$}{\Rule{Assume}}
\gaalineNC{3}{$\horseshoe{\Ap{\constant{d}}}{\Bp{\constant{d}}}$}{\Rule{$\forall$-Elim}, 1}
\gaalineNC{4}{$\Bp{\constant{d}}$}{\Rule{$\HORSESHOE$-Elim}, 2,3}
\end{gproof}
And this completes the derivation. 

The next example demonstrates \Rule{$\exists$-Intro}. Here we will show that $\universal{\variable{y}}\bparhorseshoe{\existential{\variable{x}}\Dpp{\variable{x}}{\variable{y}}}{\Bp{\variable{y}}},\Dpp{\constant{a}}{\constant{b}}\sststile{}{}\Bp{\constant{b}}$.
As before, we start with the assumptions.
\begin{gproof}[\label{GQDExampleD}]
\galineNC{1}{$\universal{\variable{y}}\bparhorseshoe{\existential{\variable{x}}\Dpp{\variable{x}}{\variable{y}}}{\Bp{\variable{y}}}$}{\Rule{Assume}}
\gaalineNC{2}{$\Dpp{\constant{a}}{\constant{b}}$}{\Rule{Assume}}
\end{gproof}
Again as before, we need to use \Rule{$\forall$-Elim} so we can eventually use \Rule{$\HORSESHOE$-Elim} to finish. 
\begin{gproof}[\label{GQDExampleE}]
\galineNC{1}{$\universal{\variable{y}}\bparhorseshoe{\existential{\variable{x}}\Dpp{\variable{x}}{\variable{y}}}{\Bp{\variable{y}}}$}{\Rule{Assume}}
\gaalineNC{2}{$\Dpp{\constant{a}}{\constant{b}}$}{\Rule{Assume}}
\gaalineNC{3}{$\horseshoe{\existential{\variable{x}}\Dpp{\variable{x}}{\constant{b}}}{\Bp{\variable{\constant{b}}}}$}{\Rule{$\forall$-Elim}, 1}
\end{gproof}
Again we strategically chose the constant we did, $\constant{b}$, so that using \Rule{$\HORSESHOE$-Elim} will get us the right result. 
But we can't apply \Rule{$\HORSESHOE$-Elim} yet, since the \CAPS{lhs} of the horseshoe in line 3, $\existential{\variable{x}}\Dpp{\variable{x}}{\constant{b}}$, is an existential sentence, while what we have on line 2, $\Dpp{\constant{a}}{\constant{b}}$, is not. 
But, by using \Rule{$\exists$-Intro} we can easily get what we need:
\begin{gproof}[\label{GQDExampleF}]
\galineNC{1}{$\universal{\variable{y}}\bparhorseshoe{\existential{\variable{x}}\Dpp{\variable{x}}{\variable{y}}}{\Bp{\variable{y}}}$}{\Rule{Assume}}
\gaalineNC{2}{$\Dpp{\constant{a}}{\constant{b}}$}{\Rule{Assume}}
\gaalineNC{3}{$\horseshoe{\existential{\variable{x}}\Dpp{\variable{x}}{\constant{b}}}{\Bp{\variable{\constant{b}}}}$}{\Rule{$\forall$-Elim}, 1}
\gaalineNC{4}{$\existential{\variable{x}}\Dpp{\variable{x}}{\constant{b}}$}{\Rule{$\exists$-Intro}, 2}
\end{gproof}
Note that just as with \Rule{$\forall$-Elim} there are no restrictions on \Rule{$\HORSESHOE$-Elim}.
Also note that we could have used \Rule{$\HORSESHOE$-Elim} to generalize the constant $\constant{b}$, getting $\existential{\variable{x}}\Dpp{\constant{a}}{\variable{x}}$, but that wouldn't have helped us.
Also, we could have generalized using a different variable, getting, say $\existential{\variable{z}}\Dpp{\variable{z}}{\constant{b}}$.
But again that wouldn't have helped us. 
Any constant is legal to instantiate with \Rule{$\forall$-Elim}, but often only one is a wise choice.

We now have the right setup for \Rule{$\HORSESHOE$-Elim}:
\begin{gproof}[\label{GQDExampleG}]
\galineNC{1}{$\universal{\variable{y}}\bparhorseshoe{\existential{\variable{x}}\Dpp{\variable{x}}{\variable{y}}}{\Bp{\variable{y}}}$}{\Rule{Assume}}
\gaalineNC{2}{$\Dpp{\constant{a}}{\constant{b}}$}{\Rule{Assume}}
\gaalineNC{3}{$\horseshoe{\existential{\variable{x}}\Dpp{\variable{x}}{\constant{b}}}{\Bp{\variable{\constant{b}}}}$}{\Rule{$\forall$-Elim}, 1}
\gaalineNC{4}{$\existential{\variable{x}}\Dpp{\variable{x}}{\constant{b}}$}{\Rule{$\exists$-Intro}, 2}
\gaalineNC{5}{$\Bp{\constant{b}}$}{\Rule{$\HORSESHOE$-Elim}, 3,4}
\end{gproof}
And this completes the derivation. 

The next example demonstrates \Rule{$\exists$-Elim}. 
This is our first rule with restrictions.
We will show that $\universal{\variable{x}}\negation{\Qpp{\variable{x}}{\constant{b}}},\existential{\variable{z}}\pardisjunction{\Qpp{\variable{z}}{\constant{b}}}{\Gp{\variable{z}}}\sststile{}{}\existential{\variable{x}}\Gp{\variable{x}}$.
As before, we start by setting out the assumptions. 
\begin{gproof}[\label{GQDExampleH}]
\galineNC{1}{$\universal{\variable{x}}\negation{\Qpp{\variable{x}}{\constant{b}}}$}{\Rule{Assume}}
\gaalineNC{2}{$\existential{\variable{z}}\pardisjunction{\Qpp{\variable{z}}{\constant{b}}}{\Gp{\variable{z}}}$}{\Rule{Assume}}
\end{gproof}
When using both \Rule{$\exists$-Elim} and \Rule{$\forall$-Elim}, it's generally best to do what needs to be done for \Rule{$\exists$-Elim} first and use \Rule{$\forall$-Elim} only when necessary.
The reason for this has to do with the restriction on \Rule{$\exists$-Elim} and will become apparent with some practice. 
Now, according to \Rule{$\exists$-Elim}, we can get a sentence $\CAPTHETA$ in this case by showing that $\horseshoe{\pardisjunction{\Qpp{\variable{t}}{\constant{b}}}{\Gp{\variable{t}}}}{\CAPTHETA}$, for some constant $\variable{t}$ that fits the restriction on \Rule{$\exists$-Elim}.
(Just what constants will work and what sentence $\CAPTHETA$ we want will take some thought.)
So, we need to use \Rule{$\HORSESHOE$-Intro}. 
\begin{gproof}[\label{GQDExampleI}]
\galineNC{1}{$\universal{\variable{x}}\negation{\Qpp{\variable{x}}{\constant{b}}}$}{\Rule{Assume}}
\gaalineNC{2}{$\existential{\variable{z}}\pardisjunction{\Qpp{\variable{z}}{\constant{b}}}{\Gp{\variable{z}}}$}{\Rule{Assume}}
\gaaaproof{
\gaaalineSS{3}{$\disjunction{\Qpp{\constant{a}}{\constant{b}}}{\Gp{\constant{a}}}$}{\Rule{Assume}}
\gaaalinendSS{}{}{}
\gaaalinendSS{}{$\qquad\vdots$}{}
\gaaalinendSS{}{}{}
\gaaalineSS{$\integer{n}$}{$\CAPTHETA$}{}
}
\gaalineNC{$\integer{n}+1$}{$\horseshoe{\pardisjunction{\Qpp{\constant{a}}{\constant{b}}}{\Gp{\constant{a}}}}{\CAPTHETA}$}{\Rule{$\HORSESHOE$-Intro}, 3--$\integer{n}$}
\gaalineNC{$\integer{n}+2$}{$\CAPTHETA$}{\Rule{$\exists$-Elim}, 2,$\integer{n}+1$}
\end{gproof}
Here we have chosen $\constant{a}$ to substitute for $\variable{z}$ in line 3 because (so long as we pick $\CAPTHETA$ correctly) it will meet our restriction. 
According to the restriction on \Rule{$\exists$-Elim}, the constant we pick can't appear in any open assumptions (the assumption used to derive the needed conditional, in this case line 3, is not open by the time we apply the rule). 
It also can't appear in the scope of the existential quantifier, which in this case is $\pardisjunction{\Qpp{\variable{z}}{\constant{b}}}{\Gp{\variable{z}}}$. 
Since $\constant{a}$ does not appear in any open assumptions and does not appear in $\pardisjunction{\Qpp{\variable{z}}{\constant{b}}}{\Gp{\variable{z}}}$, it will meet our restriction.
(Clearly other constants would have also met the restriction.)
Now we have to decide what sentence $\CAPTHETA$ will enable us to finish the derivation. 
Note that if we derive $\Gp{\variable{t}}$ for any constant $\variable{t}\neq\constant{a}$, then we can use \Rule{$\exists$-Intro} to get the needed sentence $\existential{\variable{x}}\Gp{\variable{x}}$. 
(We will set this up, but we will see in a moment that this first guess won't work.)
We can't pick $\variable{t}=\constant{a}$ because then we couldn't apply \Rule{$\exists$-Elim} on line $\integer{n}+2$.
\begin{gproof}[\label{GQDExampleJ}]
\galineNC{1}{$\universal{\variable{x}}\negation{\Qpp{\variable{x}}{\constant{b}}}$}{\Rule{Assume}}
\gaalineNC{2}{$\existential{\variable{z}}\pardisjunction{\Qpp{\variable{z}}{\constant{b}}}{\Gp{\variable{z}}}$}{\Rule{Assume}}
\gaaaproof{
\gaaalineSS{3}{$\disjunction{\Qpp{\constant{a}}{\constant{b}}}{\Gp{\constant{a}}}$}{\Rule{Assume}}
\gaaalinendSS{}{}{}
\gaaalinendSS{}{$\qquad\vdots$}{}
\gaaalinendSS{}{}{}
\gaaalineSS{$\integer{n}$}{$\Gp{\constant{c}}$}{}
}
\gaalineNC{$\integer{n}+1$}{$\horseshoe{\pardisjunction{\Qpp{\constant{a}}{\constant{b}}}{\Gp{\constant{a}}}}{\Gp{\constant{c}}}$}{\Rule{$\HORSESHOE$-Intro}, 3--$\integer{n}$}
\gaalineNC{$\integer{n}+2$}{$\Gp{\constant{c}}$}{\Rule{$\exists$-Elim}, 2,$\integer{n}+1$}
\gaalineNC{$\integer{n}+3$}{$\existential{\variable{x}}\Gp{\variable{x}}$}{\Rule{$\exists$-Intro}, $\integer{n}+2$}
\end{gproof}
Now we only need to complete the derivation by deriving $\Gp{\constant{c}}$ from $\pardisjunction{\Qpp{\constant{a}}{\constant{b}}}{\Gp{\constant{a}}}$. 
We can \emph{try} do this by using \Rule{$\forall$-Elim} and \Rule{D.S.}, but it becomes clear at once that this won't work.
\begin{gproof}[\label{GQDExampleK}]
\galineNC{1}{$\universal{\variable{x}}\negation{\Qpp{\variable{x}}{\constant{b}}}$}{\Rule{Assume}}
\gaalineNC{2}{$\existential{\variable{z}}\pardisjunction{\Qpp{\variable{z}}{\constant{b}}}{\Gp{\variable{z}}}$}{\Rule{Assume}}
\gaaaproof{
\gaaalineSS{3}{$\disjunction{\Qpp{\constant{a}}{\constant{b}}}{\Gp{\constant{a}}}$}{\Rule{Assume}}
\gaaalineSS{4}{$\negation{\Qpp{\constant{a}}{\constant{b}}}$}{\Rule{$\forall$-Elim}, 1}
\gaaalinendSS{5}{$\Gp{\constant{a}}$}{\Rule{D.S.}, 3,4}
\gaaalinendSS{}{}{}
\gaaalinendSS{}{$\qquad\vdots$}{}
\gaaalinendSS{}{}{}
\gaaalineSS{$\integer{n}$}{$\Gp{\constant{c}}$}{}
}
\gaalineNC{$\integer{n}+1$}{$\horseshoe{\pardisjunction{\Qpp{\constant{a}}{\constant{b}}}{\Gp{\constant{a}}}}{\Gp{\constant{c}}}$}{\Rule{$\HORSESHOE$-Intro}, 3--$\integer{n}$}
\gaalineNC{$\integer{n}+2$}{$\Gp{\constant{c}}$}{\Rule{$\exists$-Elim}, 2,$\integer{n}+1$}
\gaalineNC{$\integer{n}+3$}{$\existential{\variable{x}}\Gp{\variable{x}}$}{\Rule{$\exists$-Intro}, $\integer{n}+2$}
\end{gproof}
It should be clear that this \Rule{$\forall$-Elim}/\Rule{D.S.} strategy won't work, since for it to work we'd need the token of $\GG$ that appears in line 3 to be followed by the \emph{same} constant that follows the token of $\GG$ that appears in line $\integer{n}$. 
But that constant is the constant we substitute in for \Rule{$\exists$-Elim}, and we would then violate the restriction for the rule. 

So we need to go back to $\CAPTHETA$ and consider another strategy. 
This time we'll try letting $\CAPTHETA=\existential{\variable{x}}\Gp{\variable{x}}$.
\begin{gproof}[\label{GQDExampleL}]
\galineNC{1}{$\universal{\variable{x}}\negation{\Qpp{\variable{x}}{\constant{b}}}$}{\Rule{Assume}}
\gaalineNC{2}{$\existential{\variable{z}}\pardisjunction{\Qpp{\variable{z}}{\constant{b}}}{\Gp{\variable{z}}}$}{\Rule{Assume}}
\gaaaproof{
\gaaalineSS{3}{$\disjunction{\Qpp{\constant{a}}{\constant{b}}}{\Gp{\constant{a}}}$}{\Rule{Assume}}
\gaaalinendSS{}{}{}
\gaaalinendSS{}{$\qquad\vdots$}{}
\gaaalinendSS{}{}{}
\gaaalineSS{$\integer{n}$}{$\existential{\variable{x}}\Gp{\variable{x}}$}{}
}
\gaalineNC{$\integer{n}+1$}{$\horseshoe{\pardisjunction{\Qpp{\constant{a}}{\constant{b}}}{\Gp{\constant{a}}}}{\existential{\variable{x}}\Gp{\variable{x}}}$}{\Rule{$\HORSESHOE$-Intro}, 3--$\integer{n}$}
\gaalineNC{$\integer{n}+2$}{$\existential{\variable{x}}\Gp{\variable{x}}$}{\Rule{$\exists$-Elim}, 2,$\integer{n}+1$}
\end{gproof}
Note that we still keep our choice of $\constant{a}$ on line 3 within the restrictions of \Rule{$\exists$-Elim}. 
Now we can try again at finishing the proof. 
This time we can:
\begin{gproof}[\label{GQDExampleM}]
\galineNC{1}{$\universal{\variable{x}}\negation{\Qpp{\variable{x}}{\constant{b}}}$}{\Rule{Assume}}
\gaalineNC{2}{$\existential{\variable{z}}\pardisjunction{\Qpp{\variable{z}}{\constant{b}}}{\Gp{\variable{z}}}$}{\Rule{Assume}}
\gaaaproof{
\gaaalineSS{3}{$\disjunction{\Qpp{\constant{a}}{\constant{b}}}{\Gp{\constant{a}}}$}{\Rule{Assume}}
\gaaalineSS{4}{$\negation{\Qpp{\constant{a}}{\constant{b}}}$}{\Rule{$\forall$-Elim}, 1}
\gaaalineSS{5}{$\Gp{\constant{a}}$}{\Rule{D.S.}, 3,4}
\gaaalineSS{6}{$\existential{\variable{x}}\Gp{\variable{x}}$}{\Rule{$\exists$-Intro}, 5}
}
\gaalineNC{7}{$\horseshoe{\pardisjunction{\Qpp{\constant{a}}{\constant{b}}}{\Gp{\constant{a}}}}{\existential{\variable{x}}\Gp{\variable{x}}}$}{\Rule{$\HORSESHOE$-Intro}, 3--6}
\gaalineNC{8}{$\existential{\variable{x}}\Gp{\variable{x}}$}{\Rule{$\exists$-Elim}, 2,7}
\end{gproof}
And this completes the derivation. 

The final example demonstrates \Rule{$\forall$-Intro}. 
This rule also has restrictions. 
We will show that $\universal{\variable{x}}\parhorseshoe{\Ap{\variable{x}}}{\Bp{\variable{x}}},\universal{\variable{x}}\parhorseshoe{\Bp{\variable{x}}}{\Hp{\variable{x}}}\sststile{}{}\universal{\variable{x}}\parhorseshoe{\Ap{\variable{x}}}{\Hp{\variable{x}}}$.
It's worth mentioning that the fact that all the examples so far have two sentences on the \CAPS{lhs} of the turnstile is just a coincidence.
As before, we start by setting out the assumptions.
\begin{gproof}[\label{GQDExampleN}]
\galineNC{1}{$\universal{\variable{x}}\parhorseshoe{\Ap{\variable{x}}}{\Bp{\variable{x}}}$}{\Rule{Assume}}
\gaalineNC{2}{$\universal{\variable{x}}\parhorseshoe{\Bp{\variable{x}}}{\Hp{\variable{x}}}$}{\Rule{Assume}}
\end{gproof}
Before moving forward, we need to think about how \Rule{$\forall$-Intro} works. 
According to the rule, if we want to get $\universal{\variable{x}}\parhorseshoe{\Ap{\variable{x}}}{\Hp{\variable{x}}}$ by using \Rule{$\forall$-Intro}, we need to first get $\horseshoe{\Ap{\variable{t}}}{\Hp{\variable{t}}}$ on some line, where $\variable{t}$ is some constant that does not appear in any unboxed assumptions, or in $\horseshoe{\Ap{\variable{x}}}{\Hp{\variable{x}}}$.
Since no constants appear in $\horseshoe{\Ap{\variable{x}}}{\Hp{\variable{x}}}$, it provides no constraints. 
Further, there are no constants in either unboxed assumption in derivation \ref{GQDExampleN}, so we are free to chose any constant we like. 
Pick $\constant{a}$. 
So we want to derive $\horseshoe{\Ap{\constant{a}}}{\Hp{\constant{a}}}$.
\begin{gproof}[\label{GQDExampleO}]
\galineNC{1}{$\universal{\variable{x}}\parhorseshoe{\Ap{\variable{x}}}{\Bp{\variable{x}}}$}{\Rule{Assume}}
\gaalineNC{2}{$\universal{\variable{x}}\parhorseshoe{\Bp{\variable{x}}}{\Hp{\variable{x}}}$}{\Rule{Assume}}
\gaaaproof{
\gaaalineSS{3}{$\Ap{\constant{a}}$}{\Rule{Assume}}
\gaaalinendSS{}{}{}
\gaaalinendSS{}{$\qquad\vdots$}{}
\gaaalinendSS{}{}{}
\gaaalineSS{$\integer{n}$}{$\Hp{\constant{a}}$}{}
}
\gaalineNC{$\integer{n}+1$}{$\horseshoe{\Ap{\constant{a}}}{\Hp{\constant{a}}}$}{\Rule{$\HORSESHOE$-Intro}, 3--$\integer{n}$}
\gaalineNC{$\integer{n}+2$}{$\universal{\variable{x}}\parhorseshoe{\Ap{\variable{x}}}{\Hp{\variable{x}}}$}{\Rule{$\forall$-Intro}, $\integer{n}+1$}
\end{gproof}
Now we just have to finish the derivation of $\Hp{\constant{a}}$ from $\Ap{\constant{a}}$ without introducing any new unboxed assumptions with constant $\constant{a}$. 
(Of course, if we did the bottom half of this proof wouldn't work, since we coudn't close the assumption on line 3 with \Rule{$\HORSESHOE$-Intro} if unboxed assumptions appeared after line 3.)
A natural next step is to use \Rule{$\forall$-Elim} on lines 1 and 2.
\begin{gproof}[\label{GQDExampleP}]
\galineNC{1}{$\universal{\variable{x}}\parhorseshoe{\Ap{\variable{x}}}{\Bp{\variable{x}}}$}{\Rule{Assume}}
\gaalineNC{2}{$\universal{\variable{x}}\parhorseshoe{\Bp{\variable{x}}}{\Hp{\variable{x}}}$}{\Rule{Assume}}
\gaaaproof{
\gaaalineSS{3}{$\Ap{\constant{a}}$}{\Rule{Assume}}
\gaaalineSS{4}{$\horseshoe{\Ap{\constant{a}}}{\Bp{\constant{a}}}$}{\Rule{$\forall$-Elim}, 1}
\gaaalineSS{5}{$\horseshoe{\Bp{\constant{a}}}{\Hp{\constant{a}}}$}{\Rule{$\forall$-Elim}, 2}
\gaaalinendSS{}{}{}
\gaaalinendSS{}{$\qquad\vdots$}{}
\gaaalinendSS{}{}{}
\gaaalineSS{$\integer{n}$}{$\Hp{\constant{a}}$}{}
}
\gaalineNC{$\integer{n}+1$}{$\horseshoe{\Ap{\constant{a}}}{\Hp{\constant{a}}}$}{\Rule{$\HORSESHOE$-Intro}, 3--$\integer{n}$}
\gaalineNC{$\integer{n}+2$}{$\universal{\variable{x}}\parhorseshoe{\Ap{\variable{x}}}{\Hp{\variable{x}}}$}{\Rule{$\forall$-Intro}, $\integer{n}+1$}
\end{gproof}
Now getting to $\Hp{\constant{a}}$ is just a matter of using \Rule{$\HORSESHOE$-Elim}:
\begin{gproof}[\label{GQDExampleQ}]
\galineNC{1}{$\universal{\variable{x}}\parhorseshoe{\Ap{\variable{x}}}{\Bp{\variable{x}}}$}{\Rule{Assume}}
\gaalineNC{2}{$\universal{\variable{x}}\parhorseshoe{\Bp{\variable{x}}}{\Hp{\variable{x}}}$}{\Rule{Assume}}
\gaaaproof{
\gaaalineSS{3}{$\Ap{\constant{a}}$}{\Rule{Assume}}
\gaaalineSS{4}{$\horseshoe{\Ap{\constant{a}}}{\Bp{\constant{a}}}$}{\Rule{$\forall$-Elim}, 1}
\gaaalineSS{5}{$\horseshoe{\Bp{\constant{a}}}{\Hp{\constant{a}}}$}{\Rule{$\forall$-Elim}, 2}
\gaaalineSS{6}{$\Bp{\constant{a}}$}{\Rule{$\HORSESHOE$-Elim}, 3,4}
\gaaalineSS{7}{$\Hp{\constant{a}}$}{\Rule{$\HORSESHOE$-Elim}, 5,6}
}
\gaalineNC{8}{$\horseshoe{\Ap{\constant{a}}}{\Hp{\constant{a}}}$}{\Rule{$\HORSESHOE$-Intro}, 3--7}
\gaalineNC{9}{$\universal{\variable{x}}\parhorseshoe{\Ap{\variable{x}}}{\Hp{\variable{x}}}$}{\Rule{$\forall$-Intro}, 8}
\end{gproof}
And this completes the derivation. Notice that since there are infinitely many constants in \GQL{}, and any of them would have worked in the derivation, there are infinitely many derivations of this
sentence.

\subsection{Shortcut Rules for \GQD{}}

All of the shortcut rules for \GSD{}, both the standard and exchange rules, can be carried over as shortcut rules for \GQD{}. 
In addition, there are four new shortcut rules for \GQD{}, the quantifier negation rules. 
These are found in table \ref{GQDplus}.
%\begin{table}[!ht]
%\renewcommand{\arraystretch}{1.5}
%\begin{center}
%\begin{tabular}{ p{1in} l l } %p{2.2in} p{2in}
%\toprule
%\textbf{Name} & \textbf{Given} & \textbf{May Add} \\ 
%\midrule
\renewcommand{\arraystretch}{1.5}
\begin{longtable}[c]{ p{1in} l l } %p{2.2in} p{2in}
\toprule
\textbf{Name} & \textbf{Given} & \textbf{May Add} \\ 
\midrule
\endfirsthead
\multicolumn{3}{c}{\emph{Continued from Previous Page}}\\
\toprule
\textbf{Name} & \textbf{Given} & \textbf{May Add} \\ 
\midrule
\endhead
\bottomrule
\caption{Exchange Short-Cut Rules for \GQD{}}\\[-.15in]
\multicolumn{3}{c}{\emph{Continued next Page}}\\
\endfoot
\bottomrule
\caption{Exchange Short-Cut Rules for \GQD{}}\\
\endlastfoot
\label{GQDplus}\Rule{QN} & $\negation{\universal{\BETA}{\CAPPHI}}$ & $\existential{\BETA}\negation{{\CAPPHI}}$ \\
 & $\existential{\BETA}\negation{{\CAPPHI}}$ & $\negation{\universal{\BETA}{\CAPPHI}}$  \\
 & $\negation{\existential{\BETA}{\CAPPHI}}$ & $\universal{\BETA}\negation{{\CAPPHI}}$ \\
 &  $\universal{\BETA}\negation{{\CAPPHI}}$ & $\negation{\existential{\BETA}{\CAPPHI}}$ \\
\end{longtable}
%\bottomrule
%\end{tabular}
%\end{center}
%\caption{Exchange Short-Cut Rules for \GQD{} (\GQDP{})}
%\label{GQDplus}
%\end{table}

Just as we had to slightly modify the definition of sanctioning used in \GSD{} for the basic (intro and elimination) rules and standard shortcut rules for \GQD{}, we also have to slightly modify the definition of sanctioning used in \GSDP{} (Def. \pmvref{ExchangeRuleSanctioning}) for the exchange shortcut rules for \GQD{} that make up \GQDP{}.
\begin{majorILnc}{\LnpDC{GQDExchangeRuleSanctioning}}
An exchange shortcut rule \Rule{R} of \GQDP{} (a rule from table \pmvref{GSDplus2}, or table \pmvref{GQDplus}), applied to a line with a \GQL{} sentence $\CAPPSI$, \emph{sanctions} writing down sentence $\CAPPSI^*$ \Iff
\begin{cenumerate}
\item there is some substitution of \GQL{} formulas that, for the given schema of \Rule{R}, results in a formula $\CAPPHI$ and, for the may-add schema, results in a formula $\CAPPHI^*$,
\item $\CAPPHI$ is a subformula of $\CAPPSI$, and
\item $\CAPPSI^*$ is the \GSL{} sentence you get when you replace one instance (token) of $\CAPPHI$ with an instance (token) of $\CAPPHI^*$ in $\CAPPSI$. 
\end{cenumerate}
\end{majorILnc}

\subsection{Shortcut Rule Elimination Theorem for \GQD{}}\label{Shortcut Rule Elimination Theorem Section}

% we want to show that theorem \ref{GSD Shortcut Theorem3} still holds; or rather, that: if you can derive it in GQD+, then you can derive it in GSD. I think the proof of \ref{GSD Shortcut Theorem3} will carry over, so long as we make sure \ref{GSD Shortcut Theorem2} and \ref{GSD Shortcut Theorem} still hold. The proof for \ref{GSD Shortcut Theorem3} does need to be addressed again, since the derivation schemas for the shortcut rules for GQD+ need to be careful about variable substitutions. For \ref{GSD Shortcut Theorem2}, we need to show that the shortcut rules for GQD+ are provably equivalent (make hw?) and we need to show that theorem \ref{ExchangeRuleTheorem} holds for all GQL sentences too. To do this we just need to extend the recursive proof with two new clauses for univ and ext quantifiers. I think that's all we have to do to extend \ref{GSD Shortcut Theorem3}.

In this section we want to extend the Shortcut Rule Elimination Theorem (Thm. \pmvref{GSD Shortcut Theorem3}) to \GQD{}.
\begin{THEOREM}{\LnpTC{GQD Shortcut Theorem3} Shortcut Rule Elimination Theorem for \GQDP{}:}
For all \GQL{} sentences $\CAPPHI_1,\ldots,\CAPPHI_{\integer{m}}$ and $\CAPPSI$, if $\CAPPSI$ can be derived from $\CAPPHI_1,\ldots,\CAPPHI_{\integer{m}}$ in \GQDP{}, then $\CAPPSI$ can be derived from $\CAPPHI_1,\ldots,\CAPPHI_{\integer{m}}$ in \GQD{}.
\end{THEOREM}
\noindent{}The same proof we used for Shortcut Rule Elimination Theorem for \GSD{} (Thm. \ref{GSD Shortcut Theorem3}) will work for this version, so long as appropriate versions of theorems \mvref{GSD Shortcut Theorem} and \mvref{GSD Shortcut Theorem2} hold for the shortcut rules of \GQD{}.
Specifically:
\begin{THEOREM}{\LnpTC{GQD Shortcut Theorem}}
For all \GQL{} sentences $\CAPTHETA_1,\ldots,\CAPTHETA_{\integer{n}},\DELTA$ and rules \Rule{R$_1$}$,\ldots,$\Rule{R$_\integer{p}$}, if
\begin{cenumerate}
\item $\DELTA$ can be derived from $\CAPTHETA_1,\ldots,\CAPTHETA_{\integer{n}}$ using rules \Rule{R$_1$}$,\ldots,$\Rule{R$_\integer{p}$} and the basic rules of \GSD{}, and
\item every application of a rule \Rule{R$_1$} is derivable using the rules \Rule{R$_2$}, $\ldots$, \Rule{R$_\integer{p}$} and the basic rules of \GQD{} (recall Def. \pmvref{RuleInstanceDerivability}),
\end{cenumerate}
then $\DELTA$ can be derived from $\CAPTHETA_1,\ldots,\CAPTHETA_{\integer{n}}$ using only rules \Rule{R$_2$}$,\ldots,$\Rule{R$_\integer{p}$} and the basic rules of \GQD{}.
\end{THEOREM}
\begin{THEOREM}{\LnpTC{GQD Shortcut Theorem2}}
For all standard and exchange shortcut rules \Rule{R} (see tables \ref{GSDplus1}, \ref{GSDplus2}, and \ref{GQDplus}), every application of \Rule{R} is derivable using the basic rules of \GQD{} (see tables \ref{GSD} and \ref{GQD}).
\end{THEOREM}
\noindent{}We leave it to the reader to prove the Shortcut Rule Elimination Theorem (Thm. \ref{GQD Shortcut Theorem3}) using theorems \ref{GQD Shortcut Theorem} and \ref{GQD Shortcut Theorem2}.

Turning to the proofs for theorems \ref{GQD Shortcut Theorem} and \ref{GQD Shortcut Theorem2}, note that nothing in the proof of \ref{GSD Shortcut Theorem} depended on any special features of \GSD{}. 
Thus, the proof of theorem \ref{GSD Shortcut Theorem} can be adapted to \ref{GQD Shortcut Theorem} just by changing all the references to \GSD{} to references to \GQD{}. 
But unfortunately theorem \ref{GQD Shortcut Theorem2} is not nearly as straightforward.  

It is helpful to break theorem \ref{GQD Shortcut Theorem2} into two parts: (i) the claim that, for all standard shortcut rules (table \ref{GSDplus1}), every application is derivable using the basic rules of \GQD{}, and (ii) the claim that, for all the exchange shortcut rules (tables \ref{GSDplus2} and \ref{GQDplus}), every application is derivable using the basic rules of \GQD{}. 
Proving part (i) of theorem \ref{GQD Shortcut Theorem2} is no different from proving it for theorem \ref{GSD Shortcut Theorem2}; the same arguments using the derivation schemas written for theorem \ref{GSD Shortcut Theorem2} will work.
But nothing done so far will help with part (ii).
This is because applications of exchange shortcut rules in \GQDP{}, even those shared with \GSDP{} (see table \ref{GSDplus2}), use a different definition of sanctioning than was used in \GSDP{} (compare Def. \ref{ExchangeRuleSanctioning} and \ref{GQDExchangeRuleSanctioning}). 
According to definition \mvref{GQDExchangeRuleSanctioning}, in \GQDP{} exchange rules can be applied not only to subsentences, but also to subformulas. 
We have to show that allowing exchange rules to be applied not only to subsentences, but also to subformulas, doesn't prevent us from deriving their applications using only the basic rules.

To do this we will use (1) an extended (and generalized) version of the Restricted Replacement Theorem for \GSD{} (Thm. \pmvref{ExchangeRuleTheorem}) and (2) the fact that any two formulas got by substituting \GQL{} formulas into the may-add and given schemas of the exchange shortcut rules for \GQD{} are provably equivalent. 
(We ask the reader to prove this second fact in exercise \pmvref{exer:GQDSCprovablyequiv}.)
\begin{THEOREM}{\LnpTC{GQD Replacement Theorem} The Replacement Theorem for \GQD{}:}
If $\CAPPHI$ and $\CAPPHI^*$ are provably equivalent formulas of \GQL{}, and $\CAPTHETA$ and $\CAPTHETA^*$ differ only in that $\CAPTHETA$ contains the subformula $\CAPPHI$ in one place where $\CAPTHETA^*$ contains the subformula $\CAPPHI^*$, then $\CAPTHETA$ and $\CAPTHETA^*$ are provably equivalent.
\end{THEOREM}
\noindent{}Before proving this theorem, we need to define when two \emph{formulas} of \GQL{} are provably equivalent. 
The definition given for \GSL{} sentences (def. \pmvref{GSDprovablyequivalent}) will not carry over to \GQL{} formulas, since formulas just aren't the sort of thing which can be derived. 
For example, we would like to be able to say that $\parhorseshoe{\Qp{\variable{x}}}{\Gp{\variable{y}}}$ and $\pardisjunction{\negation{\Qp{\variable{x}}}}{\Gp{\variable{y}}}$ are provably equivalent even though we cannot derive the formula $\triplebar{\parhorseshoe{\Qp{\variable{x}}}{\Gp{\variable{y}}}}{\pardisjunction{\negation{\Qp{\variable{x}}}}{\Gp{\variable{y}}}}$ (because it's not a sentence). 

The way we extend the notion of provable equivalence is through the universal closure of a formula. 
\begin{majorILnc}{\LnpDC{Universal Closure}}
The \df{universal closure} of a formula $\CAPTHETA$, written $\forall\CAPTHETA$, is the sentence that results by prefixing universal quantifiers in alphabetical order for all free variables of $\CAPTHETA$. 
\end{majorILnc}
\noindent{}E.g., $\forall\bpartriplebar{\parhorseshoe{\Qp{\variable{x}}}{\Gp{\variable{y}}}}{\pardisjunction{\negation{\Qp{\variable{x}}}}{\Gp{\variable{y}}}}$, the universal closure of $\bpartriplebar{\parhorseshoe{\Qp{\variable{x}}}{\Gp{\variable{y}}}}{\pardisjunction{\negation{\Qp{\variable{x}}}}{\Gp{\variable{y}}}}$, is $\universal{\variable{x}}\universal{\variable{y}}\bpartriplebar{\parhorseshoe{\Qp{\variable{x}}}{\Gp{\variable{y}}}}{\pardisjunction{\negation{\Qp{\variable{x}}}}{\Gp{\variable{y}}}}$.

We can now define provable equivalence for \GQL{} formulas using the universal closure:
\begin{majorILnc}{\LnpDC{GQL Provably Equivalent}}
Two \GQL{} formulas $\CAPTHETA$ and $\CAPPHI$ are \nidf{provably equivalent}\index{provably equivalent!formulas of \GQL{}|textbf} \Iff the universal closure of the formula that has a biconditional as main connective and $\CAPTHETA$ and $\CAPPHI$ as immediate constituents is derivable in \GQD{}; in other words, \Iff $\sststile{}{}\forall\bpartriplebar{\CAPTHETA}{\CAPPHI}$ in \GQD{}. 
\end{majorILnc}
\noindent{}It's important to note that this definition really is a generalization of definition \mvref{GSDprovablyequivalent}.
If $\CAPTHETA$ and $\CAPPHI$ are sentences of \GSL{} (recall that every sentence of \GSL{} is also a formula of \GQL{}), then they are provably equivalent on definition \ref{GSDprovablyequivalent} \Iff they are provably equivalent on definition \ref{GQL Provably Equivalent}.

Before moving to the proof of the Replacement Theorem for \GQD{} (Thm. \pncmvref{GQD Replacement Theorem}), it will be convenient to prove the following one-step Replacement Lemmas:
\begin{THEOREM}{\LnpTC{OneStepReplacementLemmas} One-step Replacement Lemmas:}
If $\sststile{}{}\forall\partriplebar{\CAPPHI}{\CAPPHI^*}$, then:
\begin{cenumerate}
\item $\sststile{}{}\forall\partriplebar{\negation{\CAPPHI}}{\negation{\CAPPHI^*}}$
\item\label{exampleonesteplemma}
$\sststile{}{}\forall\partriplebar{\parconjunction{\CAPPHI}{\conjunction{\CAPPHI_1}{\conjunction{\ldots}{\CAPPHI_{\integer{p}}}}}}{\parconjunction{\CAPPHI^*}{\conjunction{\CAPPHI_1}{\conjunction{\ldots}{\CAPPHI_{\integer{p}}}}}}$
\item[] \hspace{1in} $\vdots$
\item $\sststile{}{}\forall\partriplebar{\parconjunction{\CAPPHI_1}{\conjunction{\ldots}{\conjunction{\CAPPHI_{\integer{p}}}{\CAPPHI}}}}{\parconjunction{\CAPPHI_1}{\conjunction{\ldots}{\conjunction{\CAPPHI_{\integer{p}}}{\CAPPHI^*}}}}$
\item 
$\sststile{}{}\forall\partriplebar{\pardisjunction{\CAPPHI}{\disjunction{\CAPPHI_1}{\disjunction{\ldots}{\CAPPHI_{\integer{p}}}}}}{\pardisjunction{\CAPPHI^*}{\disjunction{\CAPPHI_1}{\disjunction{\ldots}{\CAPPHI_{\integer{p}}}}}}$
\item[] \hspace{1in} $\vdots$
\item $\sststile{}{}\forall\partriplebar{\pardisjunction{\CAPPHI_1}{\disjunction{\ldots}{\disjunction{\CAPPHI_{\integer{p}}}{\CAPPHI}}}}{\pardisjunction{\CAPPHI_1}{\disjunction{\ldots}{\disjunction{\CAPPHI_{\integer{p}}}{\CAPPHI^*}}}}$
\item $\sststile{}{}\forall\bpartriplebar{\parhorseshoe{\CAPPHI}{\CAPPSI}}{\parhorseshoe{\CAPPHI^*}{\CAPPSI}}$
\item $\sststile{}{}\forall\bpartriplebar{\parhorseshoe{\CAPPSI}{\CAPPHI}}{\parhorseshoe{\CAPPSI}{\CAPPHI^*}}$
\item $\sststile{}{}\forall\bpartriplebar{\partriplebar{\CAPPHI}{\CAPPSI}}{\partriplebar{\CAPPHI^*}{\CAPPSI}}$
\item $\sststile{}{}\forall\bpartriplebar{\partriplebar{\CAPPSI}{\CAPPHI}}{\partriplebar{\CAPPSI}{\CAPPHI^*}}$
\end{cenumerate}
And, if $\sststile{}{}\forall\universal{\BETA}\bpartriplebar{\CAPPHI}{\CAPPHI^*}$, then:
\begin{enumerate}[label=(\arabic*), leftmargin=1.85\parindent,
labelindent=.35\parindent, labelsep=*, itemsep=0pt, start=10]%
\item $\sststile{}{}\forall\bpartriplebar{\universal{\BETA}\CAPPHI}{\universal{\BETA}\CAPPHI^*}$
\item $\sststile{}{}\forall\bpartriplebar{\existential{\BETA}\CAPPHI}{\existential{\BETA}\CAPPHI^*}$
\end{enumerate}
\end{THEOREM}
\noindent{}We will prove \ref{exampleonesteplemma} for the case of a 2-place conjunction and leave the rest to the reader to prove in a similar way.
We will use the following notation. 
If $\CAPPHI$ is a \GQL{} formula, then let $\variable{x}_1,\ldots,\variable{x}_{\integer{m}}$ be the complete list of free variables in $\CAPPHI$. 
Further, let $\CAPPHI\constant{c_{\integer{1}}}\ldots\constant{c_{\integer{\integer{m}}}}/\variable{x}_1\ldots\variable{x}_{\integer{m}}$ be the formula you get by substituting $\constant{c_1}$ for $\variable{x}_1$, $\ldots$, and $\constant{c_{\integer{m}}}$ for $\variable{x}_{\integer{m}}$.
\begin{PROOFOF}{Thm. \ref{OneStepReplacementLemmas}, \ref{exampleonesteplemma}, for 2-place Conjunctions}
Assume that $\sststile{}{}\forall\partriplebar{\CAPPHI}{\CAPPHI^*}$. 
Then consider some derivation $\Derivation{D}$ in \GQD{} of $\forall\partriplebar{\CAPPHI}{\CAPPHI^*}$.
The basic idea will be to extend this derivation to a derivation of $\forall\partriplebar{\parconjunction{\CAPPHI}{\CAPPSI}}{\parconjunction{\CAPPHI^*}{\CAPPSI}}$ by first stripping away the initial quantifiers, then manipulating the truth functional connectives, and, finally, restoring the quantifiers. In detail, the new extended derivation should go (to save space when numbering lines, let $\integer{q}=\integer{n}+\integer{m}$):
\begin{gproofnn}
\glinend{ }{$\qquad\vdots$}{ }
\gline{$\integer{n}$}{$\forall\partriplebar{\CAPPHI}{\CAPPHI^*}$}{last line of $\Derivation{D}$}
\gline{$\integer{n}+1$}{$\forall[\partriplebar{\CAPPHI}{\CAPPHI^*}\constant{c}_1/\variable{x}_1]$}{\Rule{$\forall$-Elim}, $\integer{n}$}
\glinend{ }{$\qquad\vdots$}{ }
\gline{$\integer{n}+\integer{m}$}{$\partriplebar{\CAPPHI}{\CAPPHI^*}\constant{c_{\integer{1}}}\ldots\constant{c_{\integer{\integer{m}}}}/\variable{x}_1\ldots\variable{x}_{\integer{m}}$}{\Rule{$\forall$-Elim}, $\integer{n}+\integer{m}-1$}
\gaproof{
\galine{$\integer{q}+1$}{$\parconjunction{\CAPPHI}{\CAPPSI}\constant{c_{\integer{1}}}\ldots\constant{c_{\integer{\integer{m}}}}/\variable{x}_1\ldots\variable{x}_{\integer{m}}$}{\Rule{Assume}}
\galine{$\integer{q}+2$}{$\CAPPHI\constant{c_{\integer{1}}}\ldots\constant{c_{\integer{\integer{m}}}}/\variable{x}_1\ldots\variable{x}_{\integer{m}}$}{\Rule{$\WEDGE$-Elim}, $\integer{q}+1$}
\galine{$\integer{q}+3$}{$\CAPPHI^*\constant{c_{\integer{1}}}\ldots\constant{c_{\integer{\integer{m}}}}/\variable{x}_1\ldots\variable{x}_{\integer{m}}$}{\Rule{$\TRIPLEBAR$-Elim}, $\integer{q}$, $\integer{q}+2$}
\galine{$\integer{q}+4$}{$\CAPPSI\constant{c_{\integer{1}}}\ldots\constant{c_{\integer{\integer{m}}}}/\variable{x}_1\ldots\variable{x}_{\integer{m}}$}{\Rule{$\WEDGE$-Elim}, $\integer{q}+1$}
\galine{$\integer{q}+5$}{$\parconjunction{\CAPPHI^*}{\CAPPSI}\constant{c_{\integer{1}}}\ldots\constant{c_{\integer{\integer{m}}}}/\variable{x}_1\ldots\variable{x}_{\integer{m}}$}{\Rule{$\WEDGE$-Intro}, $\integer{q}+3$, $\integer{q}+4$}
}
\gline{$\integer{q}+6$}{${\parconjunction{\CAPPHI}{\CAPPSI}\constant{c_{\integer{1}}}\ldots\constant{c_{\integer{\integer{m}}}}/\variable{x}_1\ldots\variable{x}_{\integer{m}}}\HORSESHOE$}{ }
\glinend{}{$\qquad{\parconjunction{\CAPPHI^*}{\CAPPSI}\constant{c_{\integer{1}}}\ldots\constant{c_{\integer{\integer{m}}}}/\variable{x}_1\ldots\variable{x}_{\integer{m}}}$}{\Rule{$\HORSESHOE$-Intro}, $\integer{q}+1$--$\integer{q}+5$}

\gaproof{
\galine{$\integer{q}+7$}{$\parconjunction{\CAPPHI^*}{\CAPPSI}\constant{c_{\integer{1}}}\ldots\constant{c_{\integer{\integer{m}}}}/\variable{x}_1\ldots\variable{x}_{\integer{m}}$}{\Rule{Assume}}
\galine{$\integer{q}+8$}{$\CAPPHI^*\constant{c_{\integer{1}}}\ldots\constant{c_{\integer{\integer{m}}}}/\variable{x}_1\ldots\variable{x}_{\integer{m}}$}{\Rule{$\WEDGE$-Elim}, $\integer{q}+7$}
\galine{$\integer{q}+9$}{$\CAPPSI\constant{c_{\integer{1}}}\ldots\constant{c_{\integer{\integer{m}}}}/\variable{x}_1\ldots\variable{x}_{\integer{m}}$}{\Rule{$\WEDGE$-Elim}, $\integer{q}+7$}
\galine{$\integer{q}+10$}{$\CAPPHI\constant{c_{\integer{1}}}\ldots\constant{c_{\integer{\integer{m}}}}/\variable{x}_1\ldots\variable{x}_{\integer{m}}$}{\Rule{$\TRIPLEBAR$-Elim}, $\integer{q}$, $\integer{q}+8$}
\galine{$\integer{q}+11$}{$\parconjunction{\CAPPHI}{\CAPPSI}\constant{c_{\integer{1}}}\ldots\constant{c_{\integer{\integer{m}}}}/\variable{x}_1\ldots\variable{x}_{\integer{m}}$}{\Rule{$\WEDGE$-Intro}, $\integer{q}+9$, $\integer{q}+10$}
}

\gline{$\integer{q}+12$}{${\parconjunction{\CAPPHI^*}{\CAPPSI}\constant{c_{\integer{1}}}\ldots\constant{c_{\integer{\integer{m}}}}/\variable{x}_1\ldots\variable{x}_{\integer{m}}}\HORSESHOE$}{ }
\glinend{}{$\qquad{\parconjunction{\CAPPHI}{\CAPPSI}\constant{c_{\integer{1}}}\ldots\constant{c_{\integer{\integer{m}}}}/\variable{x}_1\ldots\variable{x}_{\integer{m}}}$}{\Rule{$\HORSESHOE$-Intro}, $\integer{q}+7$--$\integer{q}+11$}

\gline{$\integer{q}+13$}{$[{\parconjunction{\CAPPHI}{\CAPPSI}}\TRIPLEBAR$}{ }
\glinend{}{$\qquad{\parconjunction{\CAPPHI^*}{\CAPPSI}]\constant{c_{\integer{1}}}\ldots\constant{c_{\integer{\integer{m}}}}/\variable{x}_1\ldots\variable{x}_{\integer{m}}}$}{\Rule{$\TRIPLEBAR$-Intro}, $\integer{q}+6$, $\integer{q}+12$}

\gline{$\integer{q}+14$}{$\forall[{\parconjunction{\CAPPHI}{\CAPPSI}}\TRIPLEBAR$}{ }
\glinend{}{$\qquad{\parconjunction{\CAPPHI^*}{\CAPPSI}]\constant{c_{\integer{1}}}\ldots\constant{c_{\integer{\integer{m}-1}}}/\variable{x}_1\ldots\variable{x}_{\integer{m}-1}}$}{\Rule{$\forall$-Intro}, $\integer{q}+13$}

\glinend{ }{$\qquad\vdots$}{ }

\gline{$\integer{n}+2\integer{m}$}{$\forall\bpartriplebar{\parconjunction{\CAPPHI}{\CAPPSI}}{\parconjunction{\CAPPHI^*}{\CAPPSI}}$}{\Rule{$\forall$-Intro}, $\integer{n}+2\integer{m}-13$}

\end{gproofnn}
\noindent{}It is important to note that all the constants introduced on lines $\integer{n}+1$ through $\integer{n}+\integer{m}$ need to be new constants that do not appear in any previous lines. 
If not, then there's no guarantee that we'll be able to do \Rule{$\forall$-Intro} on the end lines. 
\end{PROOFOF}

\begin{PROOFOF}{Thm. \ref{GQD Replacement Theorem}}
Just as with the proof of the Restricted Replacement Theorem for \GSD{} (Thm. \pmvref{ExchangeRuleTheorem}), the proof for the Replacement Theorem for \GQD{} is a recursive proof. But, since the definition of provably equivalent is different (we've extended it to formulas of \GQL{}) we can't simply extend the proof of theorem \ref{ExchangeRuleTheorem} by adding new cases to the inheritance step for the quantifiers. 

Assume that $\CAPPHI$ and $\CAPPHI^*$ are provably equivalent formulas of \GQL{} (assume that $\sststile{}{}\forall\partriplebar{\CAPPHI}{\CAPPHI^*}$), that $\CAPPHI$ is a subformula of $\CAPTHETA$, and that $\CAPTHETA^*$ is the result of replacing $\CAPPHI$ with $\CAPPHI^*$ in $\CAPTHETA$.
\begin{description}
\item[Base Step:]
Similar to the base step in the proof of theorem \ref{ExchangeRuleTheorem}, in the base case $\CAPTHETA$ is atomic and so has no subformula other than itself.
So, if $\CAPPHI$ is a subformula of $\CAPTHETA$, then $\CAPPHI=\CAPTHETA$. Hence $\CAPTHETA^*=\CAPPHI^*$. Since $\CAPPHI$ and $\CAPPHI^*$ are provably equivalent, it follows immediately that $\CAPTHETA$ and $\CAPTHETA^*$ are provably equivalent. 

\item[Inheritance Step:] \hfill 

\begin{description}
\item[Recursive Assumption:] 
Assume that the theorem holds for formulas $\CAPPSI$, $\CAPPSI_1$, $\ldots$, $\CAPPSI_{\integer{k}}$; that is, assume that if $\CAPPSI^*$ is the result of replacing $\CAPPHI$ with $\CAPPHI^*$, then $\sststile{}{}\forall\partriplebar{\CAPPSI}{\CAPPSI^*}$, and similarly for the others.

\item[Negation:]
Assume that $\CAPTHETA=\;\negation{\CAPPSI}$.
Either $\CAPPHI=\CAPTHETA$, in which case it trivially follows that $\sststile{}{}\forall\partriplebar{\CAPTHETA}{\CAPTHETA^*}$, or $\CAPPHI$ is a subformula of $\CAPPSI$ (and hence $\CAPTHETA^*=\;\negation{\CAPPSI^*}$).
By the recursive assumption, $\sststile{}{}\forall\partriplebar{\CAPPSI}{\CAPPSI^*}$.
It follows by the One-step Replacement Lemma (Thm. \ref{OneStepReplacementLemmas}) that $\sststile{}{}\forall\partriplebar{\negation{\CAPPSI}}{\negation{\CAPPSI^*}}$. 

\item[Conjunction:]
Assume that $\CAPTHETA=\conjunction{\CAPPSI_1}{\conjunction{\ldots}{\CAPPSI_{\integer{k}}}}$.
Either $\CAPPHI=\CAPTHETA$, in which case it trivially follows that $\sststile{}{}\forall\partriplebar{\CAPTHETA}{\CAPTHETA^*}$, or $\CAPPHI$ is a subformula of one of the conjuncts $\CAPPSI_{\integer{i}}$ (and hence $\CAPTHETA^*=\conjunction{\CAPPSI_1}{\conjunction{\ldots}{\conjunction{\CAPPSI_{\integer{i}}^*}{\conjunction{\ldots}{\CAPPSI_{\integer{k}}}}}}$).
By the recursive assumption, $\sststile{}{}\forall\partriplebar{\CAPPSI_{\integer{i}}}{\CAPPSI_{\integer{i}}^*}$.
It follows by the One-step Replacement Lemma (Thm. \ref{OneStepReplacementLemmas}) that $\sststile{}{}\partriplebar{\parconjunction{\CAPPSI_1}{\conjunction{\ldots}{\conjunction{\CAPPSI_{\integer{i}}}{\conjunction{\ldots}{\CAPPSI_{\integer{k}}}}}}}{\parconjunction{\CAPPSI_1}{\conjunction{\ldots}{\conjunction{\CAPPSI_{\integer{i}}^*}{\conjunction{\ldots}{\CAPPSI_{\integer{k}}}}}}}$.

\item[Disjunction:]
This case is left to the reader.

\item[Conditional]
This case is also left to the reader.

\item[Biconditional:]
This case is also left to the reader.

\item[Universal:]
Assume that $\CAPTHETA=\universal{\BETA}\CAPPSI$. 
Either $\CAPPHI=\CAPTHETA$, in which case it trivially follows that $\sststile{}{}\forall\partriplebar{\CAPTHETA}{\CAPTHETA^*}$, or $\CAPPHI$ is a subformula of $\CAPPSI$ (and hence $\CAPTHETA^*=\universal{\BETA}\CAPPSI^*$).
By the recursive assumption, $\sststile{}{}\forall\partriplebar{\CAPPSI}{\CAPPSI^*}$.
To derive $\forall\partriplebar{\universal{\BETA}\CAPPSI}{\universal{\BETA}\CAPPSI^*}$, start with the derivation of $\forall\partriplebar{\CAPPSI}{\CAPPSI^*}$. 
Extend it by adding as many steps of \Rule{$\forall$-Elim} as needed to get to the sentence $\partriplebar{\CAPPSI}{\CAPPSI^*}\constant{c_{\integer{1}}}\ldots\constant{c_{\integer{\integer{m}}}}/\variable{x}_1\ldots\variable{x}_{\integer{m}}$.
Then using \Rule{$\forall$-Intro} first on $\BETA$, then on the others we can get $\forall\universal{\BETA}\partriplebar{\CAPPSI}{\CAPPSI^*}$ on the line. 
Hence $\sststile{}{}\forall\universal{\BETA}\partriplebar{\CAPPSI}{\CAPPSI^*}$,
so by the One-step Replacement Lemma (Thm. \ref{OneStepReplacementLemmas}) we get that $\sststile{}{}\forall\bpartriplebar{\universal{\BETA}\CAPPSI}{\universal{\BETA}\CAPPSI^*}$.

\item[Existential:] This case is exactly the same, except that a different result from the One-step Replacement Lemma is used.

\end{description}

\item[Closure Step:]
Since the inheritance step covers all the ways to generate \GQL{} formulas, we've shown that the theorem holds for all \GQL{} formulas $\CAPTHETA$.
\end{description}
\end{PROOFOF}
\begin{PROOFOF}{Thm. \ref{GQD Shortcut Theorem2}, Part (ii)}
Since any two formulas $\CAPPHI$ and $\CAPPHI^*$ got by substituting \GQL{} formulas into the may-add and given schemas of the exchange shortcut rules from \GQDP{} (tables \ref{GSDplus2} and \ref{GQDplus}) are provably equivalent, it follows from the Replacement Theorem for \GQD{} (Thm. \pncmvref{GQD Replacement Theorem}) that if $\CAPTHETA^*$ is a sentence sanctioned by an exchange rule applied to some sentence $\CAPTHETA$, then $\CAPTHETA$ and $\CAPTHETA^*$ are provably equivalent. That is, $\sststile{}{}\forall\partriplebar{\CAPTHETA}{\CAPTHETA^*}$. Since $\CAPTHETA$ and $\CAPTHETA^*$ are sentences, the universal closure of their biconditional $\triplebar{\CAPTHETA}{\CAPTHETA^*}$ is just the biconditional itself, so  $\sststile{}{}\partriplebar{\CAPTHETA}{\CAPTHETA^*}$.
But since  $\sststile{}{}\partriplebar{\CAPTHETA}{\CAPTHETA^*}$, it should be clear that $\CAPTHETA\sststile{}{}\CAPTHETA^*$. 
Thus, any application of an exchange rule from \GQDP{} is derivable using the basic rules of \GQD{} alone. 
\end{PROOFOF}


%%%%%%%%%%%%%%%%%%%%%%%%%%%%%%%%%%%%%%%%%%%%%%%%%%
\section{Exercises}
%%%%%%%%%%%%%%%%%%%%%%%%%%%%%%%%%%%%%%%%%%%%%%%%%%

%\notocsubsection{Section Review Exercises}{Section Review Exercises}
%\begin{enumerate}
%\item Rewrite derivation \ref{cangetlong} using \Rule{A.C.}
%\item Definition \mvref{GSDprovablyequivalent} assumes/claims that: both $\CAPPHI\sststile{}{}\CAPPSI$ and $\CAPPSI\sststile{}{}\CAPPHI$ \Iff $\sststile{}{}\triplebar{\CAPPHI}{\CAPPSI}$. Prove that this is true.
%\item Give the needed arguments for conditionals and biconditionals in the inheritance step of the proof for theorem \mvref{ExchangeRuleTheorem}. 
%\end{enumerate}

\notocsubsection{\GSD{} Practice Problems}{GSD Practice Problems} 
Write derivations for each of the following using only the rules specified by the instructor. 
It is probably a good idea to do the problems in order, as the earlier ones tend to be easier than the later ones. 
\begin{multicols}{2}
\begin{enumerate}
\item $\sststile{}{}\horseshoe{\Al}{\parhorseshoe{\Bl}{\Al}}$
\item $\sststile{}{}\horseshoe{\parhorseshoe{\Al}{\Bl}}{\bparhorseshoe{\parhorseshoe{\Bl}{\Cl}}{\parhorseshoe{\Al}{\Cl}}}$
\item $\sststile{}{}\horseshoe{\parhorseshoe{\Al}{\Bl}}{\bparhorseshoe{\pardisjunction{\Al}{\Cl}}{\pardisjunction{\Bl}{\disjunction{\Cl}{\Dl}}}}$
\item $\sststile{}{}\horseshoe{\parconjunction{\conjunction{\bparhorseshoe{\Al}{\Bl}}{\Cl}}{\bpartriplebar{\Cl}{\Al}}}{\Bl}$
\item $\sststile{}{}\disjunction{\negation{\Al}}{\parhorseshoe{\Bl}{\Al}}$
\item $\sststile{}{}\disjunction{\parhorseshoe{\Al}{\Bl}}{\parhorseshoe{\Bl}{\Cl}}$
\end{enumerate}
\end{multicols}
\begin{enumerate}[start=7]
\item $\sststile{}{}\horseshoe{\cparconjunction{\negation{\Cl}}{\bpardisjunction{\parhorseshoe{\Al}{\Cl}}{\parhorseshoe{\Bl}{\Cl}}}}{\negation{\parconjunction{\Al}{\Bl}}}$
\end{enumerate}

\notocsubsection{\GSD{} Shortcut Rules}{exercisesGSDshortcutrules} 
Finish the proof of theorem \mvref{GSD Shortcut Theorem2}. 
To do this, write derivation schemas for each of the following using only the basic rules of \GSD{} or shortcut rules for which you've already done a derivation schema. 
Note that this is also sufficient to show that sentences got by substituting into the given and may-add schemas of the \GSD{} exchange shortcut rules are provably equivalent (Def. \pmvref{GSDprovablyequivalent}).
Note that a star $^*$ has been placed next to the ones that have already been done in the text above. 
(They are left in the list for completeness.) 
Hint: derivations \ref{helpful1} and \ref{bycontradiction} should be helpful in doing two of the problems below.
\begin{multicols}{2}
\begin{description}
\item[M.T.]\hfill{}
\begin{enumerate}
\item $\horseshoe{\CAPPHI}{\CAPTHETA},\negation{\CAPTHETA}\sststile{}{}\;\negation{\CAPPHI}$
\end{enumerate}
\item[D.S.]\hfill{}
\begin{enumerate}[start=2]
\item $\disjunction{\CAPPHI}{\CAPTHETA}, \negation{\CAPPHI}\sststile{}{}\CAPTHETA$
\item $\disjunction{\negation{\CAPPHI}}{\CAPTHETA},\CAPPHI\sststile{}{}\CAPTHETA$
\end{enumerate}
\item[A.C.]\hfill{}
\begin{enumerate}[start=4]
\item $\CAPPHI,\negation{\CAPPHI}\sststile{}{}\CAPPSI$ $^*$ (derivation \ref{anycontradictionSC})
\end{enumerate}
\item[$\NEGATION/\TRIPLEBAR$-Intro]\hfill{}
\begin{enumerate}[start=5]
\item $\triplebar{\CAPPHI}{\CAPPSI}\sststile{}{}\triplebar{\negation{\CAPPHI}}{\negation{\CAPPSI}}$
\end{enumerate}
\item[DeM]\hfill{}
\begin{enumerate}[start=6]
\item ${\negation{\parconjunction{\CAPPHI}{\CAPTHETA}}}\sststile{}{}{\pardisjunction{\negation{\CAPPHI}}{\negation{\CAPTHETA}}}$
\item ${\disjunction{\negation{\CAPPHI}}{\negation{\CAPTHETA}}}\sststile{}{}{\negation{\parconjunction{\CAPPHI}{\CAPTHETA}}}$
\item ${\negation{\pardisjunction{\CAPPHI}{\CAPTHETA}}}\sststile{}{}{\parconjunction{\negation{\CAPPHI}}{\negation{\CAPTHETA}}}$ 
\item ${\conjunction{\negation{\CAPPHI}}{\negation{\CAPTHETA}}}\sststile{}{}{\negation{\pardisjunction{\CAPPHI}{\CAPTHETA}}}$ $^*$ (derivation \ref{DeMDerivationSchema})
\end{enumerate}
\item[$\NEGATION\NEGATION$-Elim]\hfill{}
\begin{enumerate}[start=10]
\item $\negation{\negation{\CAPPHI}}\sststile{}{}\CAPPHI$
\end{enumerate}
\item[$\NEGATION\NEGATION$-Intro]\hfill{}
\begin{enumerate}[start=11]
\item $\CAPPHI\sststile{}{}\;\negation{\negation{\CAPPHI}}$
\end{enumerate}
\item[$\HORSESHOE/\:\VEE$-Exchange]\hfill{}
\begin{enumerate}[start=12]
\item ${\horseshoe{\CAPPHI}{\CAPTHETA}}\sststile{}{}{\disjunction{\negation{\CAPPHI}}{\CAPTHETA}}$
\item ${\disjunction{\negation{\CAPPHI}}{\CAPTHETA}}\sststile{}{}{\horseshoe{\CAPPHI}{\CAPTHETA}}$
\end{enumerate}
\item[Contraposition]\hfill{}
\begin{enumerate}[start=14]
\item $\horseshoe{\CAPPHI}{\CAPTHETA}\sststile{}{}\horseshoe{\negation{\CAPTHETA}}{\negation{\CAPPHI}}$
\item $\horseshoe{\negation{\CAPTHETA}}{\negation{\CAPPHI}}\sststile{}{}\horseshoe{\CAPPHI}{\CAPTHETA}$
\end{enumerate}
\item[$\NEGATION/\HORSESHOE$-Exchange]\hfill{}
\begin{enumerate}[start=16]
\item ${\negation{\parhorseshoe{\CAPPHI}{\CAPTHETA}}}\sststile{}{}{\conjunction{\CAPPHI}{\negation{\CAPTHETA}}}$
\item ${\conjunction{\CAPPHI}{\negation{\CAPTHETA}}}\sststile{}{}{\negation{\parhorseshoe{\CAPPHI}{\CAPTHETA}}}$
\end{enumerate}
\end{description}
\end{multicols}
\begin{description} 
\item[Distribution]\hfill{}
\begin{enumerate}[start=18]
\item $\conjunction{\CAPTHETA}{\pardisjunction{\CAPPHI_1}{\CAPPHI_2}}\sststile{}{}\disjunction{\parconjunction{\CAPTHETA}{\CAPPHI_1}}{\parconjunction{\CAPTHETA}{\CAPPHI_2}}$
\item $\disjunction{\parconjunction{\CAPTHETA}{\CAPPHI_1}}{\parconjunction{\CAPTHETA}{\CAPPHI_2}}\sststile{}{}\conjunction{\CAPTHETA}{\pardisjunction{\CAPPHI_1}{\CAPPHI_2}}$

\item $\conjunction{\pardisjunction{\CAPPHI_1}{\CAPPHI_2}}{\CAPTHETA}\sststile{}{}\disjunction{\parconjunction{\CAPPHI_1}{\CAPTHETA}}{\parconjunction{\CAPPHI_2}{\CAPTHETA}}$
\item $\disjunction{\parconjunction{\CAPPHI_1}{\CAPTHETA}}{\parconjunction{\CAPPHI_2}{\CAPTHETA}}\sststile{}{}\conjunction{\pardisjunction{\CAPPHI_1}{\CAPPHI_2}}{\CAPTHETA}$

\item $\disjunction{\CAPTHETA}{\parconjunction{\CAPPHI_1}{\CAPPHI_2}}\sststile{}{}\conjunction{\pardisjunction{\CAPTHETA}{\CAPPHI_1}}{\pardisjunction{\CAPTHETA}{\CAPPHI_2}}$
\item $\conjunction{\pardisjunction{\CAPTHETA}{\CAPPHI_1}}{\pardisjunction{\CAPTHETA}{\CAPPHI_2}}\sststile{}{}\disjunction{\CAPTHETA}{\parconjunction{\CAPPHI_1}{\CAPPHI_2}}$

\item $\disjunction{\parconjunction{\CAPPHI_1}{\CAPPHI_2}}{\CAPTHETA}\sststile{}{}\conjunction{\pardisjunction{\CAPPHI_1}{\CAPTHETA}}{\pardisjunction{\CAPPHI_2}{\CAPTHETA}}$
\item $\conjunction{\pardisjunction{\CAPPHI_1}{\CAPTHETA}}{\pardisjunction{\CAPPHI_2}{\CAPTHETA}}\sststile{}{}\disjunction{\parconjunction{\CAPPHI_1}{\CAPPHI_2}}{\CAPTHETA}$

\item $\triplebar{\CAPTHETA}{\CAPPSI}\sststile{}{}\disjunction{\parconjunction{\CAPTHETA}{\CAPPSI}}{\parconjunction{\negation{\CAPTHETA}}{\negation{\CAPPSI}}}$

\item $\disjunction{\parconjunction{\CAPTHETA}{\CAPPSI}}{\parconjunction{\negation{\CAPTHETA}}{\negation{\CAPPSI}}}\sststile{}{}\triplebar{\CAPTHETA}{\CAPPSI}$
\end{enumerate}
\end{description} 

%\notocsubsection{Provably Equivalence of GSD Exchange Shortcut Rules}{exercisesGSDPEshortcutrules} 
%Show that any sentences got by substituting into the given and may-add schemas of the \GSD{} exchange shortcut rules are provably equivalent (Def. \pmvref{GSDprovablyequivalent}).
%That is, find derivation schemas showing that each sentence schema in the following pairs is derivable from the other.
%\begin{enumerate}

%\item $\negation{\parconjunction{\CAPPHI_1}{\CAPPHI_2}}$, $\disjunction{\negation{\CAPPHI_1}}{\negation{\CAPPHI_2}}$

% %%\item $\disjunction{\negation{\CAPPHI_1}}{\disjunction{\ldots}{\negation{\CAPPHI_{\integer{n}}}}}$, $\negation{\parconjunction{\CAPPHI_1}{\conjunction{\ldots}{\CAPPHI_{\integer{n}}}}}$
 
%\item $\negation{\pardisjunction{\CAPPHI_1}{\CAPPHI_2}}$, $\conjunction{\negation{\CAPPHI_1}}{\negation{\CAPPHI_2}}$ 
 
% %%\item $\conjunction{\negation{\CAPPHI_1}}{\conjunction{\ldots}{\negation{\CAPPHI_{\integer{n}}}}}$, $\negation{\pardisjunction{\CAPPHI_1}{\disjunction{\ldots}{\CAPPHI_{\integer{n}}}}}$ 
 
%\item $\negation{\negation{\CAPPHI}}$, $\CAPPHI$

% %%\item $\CAPPHI$, $\negation{\negation{\CAPPHI}}$ 

%\item $\horseshoe{\CAPPHI}{\CAPTHETA}$, $\disjunction{\negation{\CAPPHI}}{\CAPTHETA}$ 

% %%\item $\disjunction{\negation{\CAPPHI}}{\CAPTHETA}$, $\horseshoe{\CAPPHI}{\CAPTHETA}$
 
%\item $\horseshoe{\CAPPHI}{\CAPTHETA}$, $\horseshoe{\negation{\CAPTHETA}}{\negation{\CAPPHI}}$ 

% %%\item $\horseshoe{\negation{\CAPTHETA}}{\negation{\CAPPHI}}$, $\horseshoe{\CAPPHI}{\CAPTHETA}$ 
 
%\item $\negation{\parhorseshoe{\CAPPHI}{\CAPTHETA}}$, $\conjunction{\CAPPHI}{\negation{\CAPTHETA}}$

% %%\item $\conjunction{\CAPPHI}{\negation{\CAPTHETA}}$, $\negation{\parhorseshoe{\CAPPHI}{\CAPTHETA}}$
 
%\item $\conjunction{\CAPTHETA}{\pardisjunction{\CAPPHI_1}{\CAPPHI_2}}$, $\disjunction{\parconjunction{\CAPTHETA}{\CAPPHI_1}}{\parconjunction{\CAPTHETA}{\CAPPHI_2}}$

% %%\item $\disjunction{\parconjunction{\CAPTHETA}{\CAPPHI_1}}{\disjunction{\ldots}{\parconjunction{\CAPTHETA}{\CAPPHI_{\integer{n}}}}}$, $\conjunction{\CAPTHETA}{\pardisjunction{\CAPPHI_1}{\disjunction{\ldots}{\CAPPHI_{\integer{n}}}}}$
 

%\item $\conjunction{\pardisjunction{\CAPPHI_1}{\CAPPHI_2}}{\CAPTHETA}$, $\disjunction{\parconjunction{\CAPPHI_1}{\CAPTHETA}}{\parconjunction{\CAPPHI_2}{\CAPTHETA}}$
 
% %%\item $\disjunction{\parconjunction{\CAPPHI_1}{\CAPTHETA}}{\disjunction{\ldots}{\parconjunction{\CAPPHI_{\integer{n}}}{\CAPTHETA}}}$, $\conjunction{\pardisjunction{\CAPPHI_1}{\disjunction{\ldots}{\CAPPHI_{\integer{n}}}}}{\CAPTHETA}$
 
 
%\item $\disjunction{\CAPTHETA}{\parconjunction{\CAPPHI_1}{\CAPPHI_2}}$, $\conjunction{\pardisjunction{\CAPTHETA}{\CAPPHI_1}}{\pardisjunction{\CAPTHETA}{\CAPPHI_2}}$
 
% %%\item $\conjunction{\pardisjunction{\CAPTHETA}{\CAPPHI_1}}{\conjunction{\ldots}{\pardisjunction{\CAPTHETA}{\CAPPHI_{\integer{n}}}}}$, $\disjunction{\CAPTHETA}{\parconjunction{\CAPPHI_1}{\conjunction{\ldots}{\CAPPHI_{\integer{n}}}}}$

%\item $\disjunction{\parconjunction{\CAPPHI_1}{\CAPPHI_2}}{\CAPTHETA}$, $\conjunction{\pardisjunction{\CAPPHI_1}{\CAPTHETA}}{\pardisjunction{\CAPPHI_2}{\CAPTHETA}}$

% %%\item $\conjunction{\pardisjunction{\CAPPHI_1}{\CAPTHETA}}{\conjunction{\ldots}{\pardisjunction{\CAPPHI_{\integer{n}}}{\CAPTHETA}}}$, $\disjunction{\parconjunction{\CAPPHI_1}{\conjunction{\ldots}{\CAPPHI_{\integer{n}}}}}{\CAPTHETA}$

%\item $\triplebar{\CAPTHETA}{\CAPPSI}$, $\disjunction{\parconjunction{\CAPTHETA}{\CAPPSI}}{\parconjunction{\negation{\CAPTHETA}}{\negation{\CAPPSI}}}$

%\end{enumerate}

\notocsubsection{\GQD{} Shortcut Rules}{exer:GQDSCprovablyequiv}
Prove that any two formulas got by substituting \GQL{} formulas into the may-add and given schemas of the exchange shortcut rules for \GQD{} are provably equivalent (Def. \pmvref{GQL Provably Equivalent}).
That is, show that the following hold for all \GQL{} formulas $\CAPPHI,\CAPPHI_1,\CAPPHI_2,\CAPTHETA,\CAPPSI$ by writing the appropriate derivation schemas.
Note that all but (7) and (8) deal with exchange shortcut rules from \GSDP{}.
For these virtually all the work has been done in exercise \ref{exercisesGSDshortcutrules}; all you need to do is show how to put the derivation schemas done there together and how to remove and put back on the quantifiers needed to make the universal closure.
\begin{multicols}{2}
\begin{enumerate}
\item $\sststile{}{}\forall[\negation{\parconjunction{\CAPPHI_1}{\CAPPHI_2}}\TRIPLEBAR\pardisjunction{\negation{\CAPPHI_1}}{\negation{\CAPPHI_2}}]$

% %%\item $\disjunction{\negation{\CAPPHI_1}}{\disjunction{\ldots}{\negation{\CAPPHI_{\integer{n}}}}}$, $\negation{\parconjunction{\CAPPHI_1}{\conjunction{\ldots}{\CAPPHI_{\integer{n}}}}}$
 
\item $\sststile{}{}\forall[\negation{\pardisjunction{\CAPPHI_1}{\CAPPHI_2}}\TRIPLEBAR\parconjunction{\negation{\CAPPHI_1}}{\negation{\CAPPHI_2}}]$ 
 
% %%\item $\conjunction{\negation{\CAPPHI_1}}{\conjunction{\ldots}{\negation{\CAPPHI_{\integer{n}}}}}$, $\negation{\pardisjunction{\CAPPHI_1}{\disjunction{\ldots}{\CAPPHI_{\integer{n}}}}}$ 
 
\item $\sststile{}{}\forall[\negation{\negation{\CAPPHI}}\TRIPLEBAR\CAPPHI]$

% %%\item $\CAPPHI$, $\negation{\negation{\CAPPHI}}$ 

\item $\sststile{}{}\forall[\parhorseshoe{\CAPPHI}{\CAPTHETA}\TRIPLEBAR\pardisjunction{\negation{\CAPPHI}}{\CAPTHETA}]$ 

% %%\item $\disjunction{\negation{\CAPPHI}}{\CAPTHETA}$, $\horseshoe{\CAPPHI}{\CAPTHETA}$
 
\item $\sststile{}{}\forall[\parhorseshoe{\CAPPHI}{\CAPTHETA}\TRIPLEBAR\parhorseshoe{\negation{\CAPTHETA}}{\negation{\CAPPHI}}]$ 

% %%\item $\horseshoe{\negation{\CAPTHETA}}{\negation{\CAPPHI}}$, $\horseshoe{\CAPPHI}{\CAPTHETA}$ 
 
\item $\sststile{}{}\forall[\negation{\parhorseshoe{\CAPPHI}{\CAPTHETA}}\TRIPLEBAR\parconjunction{\CAPPHI}{\negation{\CAPTHETA}}]$

% %%\item $\conjunction{\CAPPHI}{\negation{\CAPTHETA}}$, $\negation{\parhorseshoe{\CAPPHI}{\CAPTHETA}}$

\item $\sststile{}{}\forall[\negation{\universal{\BETA}{\CAPPHI}}\TRIPLEBAR\existential{\BETA}\negation{{\CAPPHI}}]$

\item $\sststile{}{}\forall[\negation{\existential{\BETA}{\CAPPHI}}\TRIPLEBAR\universal{\BETA}\negation{{\CAPPHI}}]$

\end{enumerate}
\end{multicols}
\begin{enumerate}[start=9]
\item $\sststile{}{}\forall[\parconjunction{\CAPTHETA}{\pardisjunction{\CAPPHI_1}{\CAPPHI_2}}\TRIPLEBAR\pardisjunction{\parconjunction{\CAPTHETA}{\CAPPHI_1}}{\parconjunction{\CAPTHETA}{\CAPPHI_2}}]$

% %%\item $\disjunction{\parconjunction{\CAPTHETA}{\CAPPHI_1}}{\disjunction{\ldots}{\parconjunction{\CAPTHETA}{\CAPPHI_{\integer{n}}}}}$, $\conjunction{\CAPTHETA}{\pardisjunction{\CAPPHI_1}{\disjunction{\ldots}{\CAPPHI_{\integer{n}}}}}$

\item $\sststile{}{}\forall[\parconjunction{\pardisjunction{\CAPPHI_1}{\CAPPHI_2}}{\CAPTHETA}\TRIPLEBAR\pardisjunction{\parconjunction{\CAPPHI_1}{\CAPTHETA}}{\parconjunction{\CAPPHI_2}{\CAPTHETA}}]$
 
% %%\item $\disjunction{\parconjunction{\CAPPHI_1}{\CAPTHETA}}{\disjunction{\ldots}{\parconjunction{\CAPPHI_{\integer{n}}}{\CAPTHETA}}}$, $\conjunction{\pardisjunction{\CAPPHI_1}{\disjunction{\ldots}{\CAPPHI_{\integer{n}}}}}{\CAPTHETA}$
 
\item $\sststile{}{}\forall[\pardisjunction{\CAPTHETA}{\parconjunction{\CAPPHI_1}{\CAPPHI_2}}\TRIPLEBAR\parconjunction{\pardisjunction{\CAPTHETA}{\CAPPHI_1}}{\pardisjunction{\CAPTHETA}{\CAPPHI_2}}]$
 
% %%\item $\conjunction{\pardisjunction{\CAPTHETA}{\CAPPHI_1}}{\conjunction{\ldots}{\pardisjunction{\CAPTHETA}{\CAPPHI_{\integer{n}}}}}$, $\disjunction{\CAPTHETA}{\parconjunction{\CAPPHI_1}{\conjunction{\ldots}{\CAPPHI_{\integer{n}}}}}$

\item $\sststile{}{}\forall[\pardisjunction{\parconjunction{\CAPPHI_1}{\CAPPHI_2}}{\CAPTHETA}\TRIPLEBAR\parconjunction{\pardisjunction{\CAPPHI_1}{\CAPTHETA}}{\pardisjunction{\CAPPHI_2}{\CAPTHETA}}]$

% %%\item $\conjunction{\pardisjunction{\CAPPHI_1}{\CAPTHETA}}{\conjunction{\ldots}{\pardisjunction{\CAPPHI_{\integer{n}}}{\CAPTHETA}}}$, $\disjunction{\parconjunction{\CAPPHI_1}{\conjunction{\ldots}{\CAPPHI_{\integer{n}}}}}{\CAPTHETA}$

\item $\sststile{}{}\forall[\partriplebar{\CAPTHETA}{\CAPPSI}\TRIPLEBAR\pardisjunction{\parconjunction{\CAPTHETA}{\CAPPSI}}{\parconjunction{\negation{\CAPTHETA}}{\negation{\CAPPSI}}}]$
\end{enumerate}

\notocsubsection{\GQD{} Practice Problems}{GQD Practice Problems} 
Write derivations for each of the following using only the rules specified by the instructor. 
It is probably a good idea to do the problems in order, as the earlier ones tend to be easier than the later ones. 
\begin{multicols}{2}
\begin{enumerate}
\item $\sststile{}{}\horseshoe{\universal{\variable{x}}\universal{\variable{y}}\Qpp{\variable{x}}{\variable{y}}}{\universal{\variable{z}}\Qpp{\variable{z}}{\variable{z}}}$
\item $\sststile{}{}\horseshoe{\universal{\variable{x}}\universal{\variable{y}}\Qpp{\variable{x}}{\variable{y}}}{\universal{\variable{x}}\universal{\variable{y}}\Qpp{\variable{y}}{\variable{x}}}$
\item $\sststile{}{}\horseshoe{\universal{\variable{x}}\parconjunction{\Qp{\variable{x}}}{\Gp{\variable{x}}}}{\bparconjunction{\universal{\variable{x}}\Qp{\variable{x}}}{\universal{\variable{x}}\Gp{\variable{x}}}}$
\item $\sststile{}{}\horseshoe{\bparconjunction{\universal{\variable{x}}\Qp{\variable{x}}}{\universal{\variable{x}}\Gp{\variable{x}}}}{\universal{\variable{x}}\parconjunction{\Qp{\variable{x}}}{\Gp{\variable{x}}}}$
\item $\sststile{}{}\horseshoe{\bpardisjunction{\universal{\variable{x}}\Qp{\variable{x}}}{\universal{\variable{x}}\Gp{\variable{x}}}}{\universal{\variable{x}}\pardisjunction{\Qp{\variable{x}}}{\Gp{\variable{x}}}}$
\item $\sststile{}{}\horseshoe{\universal{\variable{x}}\parhorseshoe{\Qp{\variable{x}}}{\Gp{\variable{x}}}}{\bparhorseshoe{\universal{\variable{x}}\Qp{\variable{x}}}{\universal{\variable{x}}\Gp{\variable{x}}}}$
\item $\sststile{}{}\horseshoe{\universal{\variable{x}}\parconjunction{\Pl}{\Qp{\variable{x}}}}{\parconjunction{\Pl}{\universal{\variable{x}}\Qp{\variable{x}}}}$
\item $\sststile{}{}\horseshoe{\parconjunction{\Pl}{\universal{\variable{x}}\Qp{\variable{x}}}}{\universal{\variable{x}}\parconjunction{\Pl}{\Qp{\variable{x}}}}$

\item $\sststile{}{}\horseshoe{\universal{\variable{x}}\pardisjunction{\Pl}{\Qp{\variable{x}}}}{\pardisjunction{\Pl}{\universal{\variable{x}}\Qp{\variable{x}}}}$
\item $\sststile{}{}\horseshoe{\pardisjunction{\Pl}{\universal{\variable{x}}\Qp{\variable{x}}}}{\universal{\variable{x}}\pardisjunction{\Pl}{\Qp{\variable{x}}}}$

\item $\sststile{}{}\horseshoe{\universal{\variable{x}}\parhorseshoe{\Pl}{\Qp{\variable{x}}}}{\parhorseshoe{\Pl}{\universal{\variable{x}}\Qp{\variable{x}}}}$
\item $\sststile{}{}\horseshoe{\parhorseshoe{\Pl}{\universal{\variable{x}}\Qp{\variable{x}}}}{\universal{\variable{x}}\parhorseshoe{\Pl}{\Qp{\variable{x}}}}$

\item $\sststile{}{}\horseshoe{\existential{\variable{x}}\universal{\variable{y}}\Qpp{\variable{x}}{\variable{y}}}{\universal{\variable{y}}\existential{\variable{x}}\Qpp{\variable{x}}{\variable{y}}}$

\item $\sststile{}{}\horseshoe{\universal{\variable{x}}\parhorseshoe{\Qp{\variable{x}}}{\Gp{\variable{x}}}}{\bparhorseshoe{\existential{\variable{x}}\Qp{\variable{x}}}{\existential{\variable{x}}\Gp{\variable{x}}}}$

\item $\sststile{}{}\horseshoe{\existential{\variable{x}}\parconjunction{\Qp{\variable{x}}}{\Gp{\variable{x}}}}{\bparconjunction{\existential{\variable{x}}\Qp{\variable{x}}}{\existential{\variable{x}}\Gp{\variable{x}}}}$

\item $\sststile{}{}\horseshoe{\bpardisjunction{\existential{\variable{x}}\Qp{\variable{x}}}{\existential{\variable{x}}\Gp{\variable{x}}}}{\existential{\variable{x}}\pardisjunction{\Qp{\variable{x}}}{\Gp{\variable{x}}}}$

\item $\sststile{}{}\horseshoe{\existential{\variable{x}}\parconjunction{\Pl}{\Qp{\variable{x}}}}{\parconjunction{\Pl}{\existential{\variable{x}}\Qp{\variable{x}}}}$
\item $\sststile{}{}\horseshoe{\parconjunction{\Pl}{\existential{\variable{x}}\Qp{\variable{x}}}}{\existential{\variable{x}}\parconjunction{\Pl}{\Qp{\variable{x}}}}$

\item $\sststile{}{}\horseshoe{\existential{\variable{x}}\pardisjunction{\Pl}{\Qp{\variable{x}}}}{\pardisjunction{\Pl}{\existential{\variable{x}}\Qp{\variable{x}}}}$
\item $\sststile{}{}\horseshoe{\pardisjunction{\Pl}{\existential{\variable{x}}\Qp{\variable{x}}}}{\existential{\variable{x}}\pardisjunction{\Pl}{\Qp{\variable{x}}}}$

\item $\sststile{}{}\horseshoe{\existential{\variable{x}}\parhorseshoe{\Pl}{\Qp{\variable{x}}}}{\parhorseshoe{\Pl}{\existential{\variable{x}}\Qp{\variable{x}}}}$
\item $\sststile{}{}\horseshoe{\parhorseshoe{\Pl}{\existential{\variable{x}}\Qp{\variable{x}}}}{\existential{\variable{x}}\parhorseshoe{\Pl}{\Qp{\variable{x}}}}$

\item $\sststile{}{}\horseshoe{\universal{\variable{x}}\parhorseshoe{\Qp{\variable{x}}}{\Pl}}{\bparhorseshoe{\existential{\variable{x}}\Qp{\variable{x}}}{\Pl}}$
\item $\sststile{}{}\horseshoe{\bparhorseshoe{\existential{\variable{x}}\Qp{\variable{x}}}{\Pl}}{\universal{\variable{x}}\parhorseshoe{\Qp{\variable{x}}}{\Pl}}$
\end{enumerate}
\end{multicols}
\begin{enumerate}[start=25]
\item $\sststile{}{}\horseshoe{\universal{\variable{x}}\existential{\variable{y}}\parconjunction{\Qp{\variable{x}}}{\Gp{\variable{y}}}}{\bparconjunction{\universal{\variable{x}}\Qp{\variable{x}}}{\existential{\variable{y}}\Gp{\variable{y}}}}$

\item $\sststile{}{}\horseshoe{\bparconjunction{\universal{\variable{x}}\Qp{\variable{x}}}{\existential{\variable{y}}\Gp{\variable{y}}}}{\universal{\variable{x}}\existential{\variable{y}}\parconjunction{\Qp{\variable{x}}}{\Gp{\variable{y}}}}$

\item $\sststile{}{}\horseshoe{\universal{\variable{x}}\existential{\variable{y}}\parconjunction{\Qp{\variable{x}}}{\Gp{\variable{y}}}}{\existential{\variable{y}}\universal{\variable{x}}\parconjunction{\Qp{\variable{x}}}{\Gp{\variable{y}}}}$

\item $\sststile{}{}\horseshoe{\universal{\variable{x}}\existential{\variable{y}}\pardisjunction{\Qp{\variable{x}}}{\Gp{\variable{y}}}}{\bpardisjunction{\universal{\variable{x}}\Qp{\variable{x}}}{\existential{\variable{y}}\Gp{\variable{y}}}}$

\item $\sststile{}{}\horseshoe{\bpardisjunction{\existential{\variable{y}}\Gp{\variable{y}}}{\universal{\variable{x}}\Qp{\variable{x}}}}{\universal{\variable{x}}\existential{\variable{y}}\pardisjunction{\Qp{\variable{x}}}{\Gp{\variable{y}}}}$

\item $\sststile{}{}\horseshoe{\existential{\variable{y}}\universal{\variable{x}}\pardisjunction{\Qp{\variable{x}}}{\Gp{\variable{y}}}}{\universal{\variable{x}}\existential{\variable{y}}\pardisjunction{\Qp{\variable{x}}}{\Gp{\variable{y}}}}$

\item $\sststile{}{}\horseshoe{\existential{\variable{y}}\universal{\variable{x}}\parhorseshoe{\Qp{\variable{x}}}{\Gp{\variable{y}}}}{\universal{\variable{x}}\existential{\variable{y}}\parhorseshoe{\Qp{\variable{x}}}{\Gp{\variable{y}}}}$

\item $\sststile{}{}\horseshoe{\universal{\variable{x}}\existential{\variable{y}}\pardisjunction{\Qp{\variable{x}}}{\Gp{\variable{y}}}}{\existential{\variable{y}}\universal{\variable{x}}\pardisjunction{\Qp{\variable{x}}}{\Gp{\variable{y}}}}$

\item $\sststile{}{}\horseshoe{\bparhorseshoe{\existential{\variable{y}}\Qp{\variable{y}}}{\existential{\variable{x}}\Gp{\variable{x}}}}{\existential{\variable{y}}\universal{\variable{x}}\parhorseshoe{\Qp{\variable{x}}}{\Gp{\variable{y}}}}$

\item $\sststile{}{}\horseshoe{\existential{\variable{y}}\universal{\variable{x}}\parhorseshoe{\Qp{\variable{x}}}{\Gp{\variable{y}}}}{\bparhorseshoe{\existential{\variable{x}}\Qp{\variable{x}}}{\existential{\variable{x}}\Gp{\variable{x}}}}$

\item 
$\sststile{}{}\horseshoe{\universal{\variable{x}}\existential{\variable{y}}\parhorseshoe{\Qp{\variable{x}}}{\Gp{\variable{y}}}}{\existential{\variable{y}}\universal{\variable{x}}\parhorseshoe{\Qp{\variable{x}}}{\Gp{\variable{y}}}}$

\item $\sststile{}{}\horseshoe{\existential{\variable{x}}\existential{\variable{y}}\parconjunction{\Qp{\variable{x}}}{\negation{\Qp{\variable{y}}}}}{\bparconjunction{\existential{\variable{x}}\Qp{\variable{x}}}{\existential{\variable{x}}\negation{\Qp{\variable{x}}}}}$

\item $\sststile{}{}\horseshoe{\existential{\variable{x}}\universal{\variable{y}}\parhorseshoe{\Qp{\variable{x}}}{\Gp{\variable{y}}}}{\bparhorseshoe{\universal{\variable{x}}\Qp{\variable{x}}}{\universal{\variable{x}}\Gp{\variable{x}}}}$

\item $\sststile{}{}\horseshoe{\bparconjunction{\existential{\variable{x}}\Qp{\variable{x}}}{\existential{\variable{x}}\negation{\Qp{\variable{x}}}}}{\existential{\variable{x}}\existential{\variable{y}}\parconjunction{\Qp{\variable{x}}}{\negation{\Qp{\variable{y}}}}}$

\item $\sststile{}{}\horseshoe{\bparhorseshoe{\universal{\variable{x}}\Qp{\variable{x}}}{\universal{\variable{x}}\Gp{\variable{x}}}}{\existential{\variable{x}}\universal{\variable{y}}\parhorseshoe{\Qp{\variable{x}}}{\Gp{\variable{y}}}}$

\item $\sststile{}{}\horseshoe{\bpardisjunction{\negation{\existential{\variable{x}}\Qp{\variable{x}}}}{\universal{\variable{x}}\Qp{\variable{x}}}}{\universal{\variable{x}}\universal{\variable{y}}\parhorseshoe{\Qp{\variable{x}}}{\Qp{\variable{y}}}}$
\end{enumerate}

%\theendnotes


%%%%%%%%%%%%%%%%%%%%%%%%%%%%%%%%%%%%%%%%%%%%%%%%%%
\chapter{Soundness and Completeness}\label{completenesschapter}
%%%%%%%%%%%%%%%%%%%%%%%%%%%%%%%%%%%%%%%%%%%%%%%%%%

%%%%%%%%%%%%%%%%%%%%%%%%%%%%%%%%%%%%%%%%%%%%%%%%%%
\section{Introduction}
%%%%%%%%%%%%%%%%%%%%%%%%%%%%%%%%%%%%%%%%%%%%%%%%%%

In the previous chapter, we stipulated restrictions for the rule applications of \GSD{} (and \GQD{}) so that the rules would be \emph{truth-preserving}.

\begin{majorILnc}{\LnpDC{Derivation Rule Soundness}}
	A rule is \df{truth-preserving}\index{derivation!rule!truth-preserving}\index{truth-preserving} \Iff the sentence or sentences to which the rule is applied entail any sentence which the rule  sanctions you to write as the next step. 
\end{majorILnc}


\begin{majorILnc}{\LnpDC{RuleSchemas}}
	A \nidf{formal derivation rule}\index{derivation!rule|textbf} is a sequence of sentence schemas, the first through the second last of which is called the \df{given schemas} and the last is called the \df{may-add schema}. 
\end{majorILnc}
\noindent{}As the names suggest, table \mvref{GSD} lists rules by putting the first through second to last schemas in the left column, labeled ``Given'', and the last schema in the right column, labeled ``May Add''. 
We only bring out that we can think of rules as sequences of sentence schemas, and call them the given schemas and the may-add schema so the next definition is easier to state.
\begin{majorILnc}{\LnpDC{RuleSanctioning}}
	A rule \Rule{R}, applied to unboxed lines $\integer{m}_1,\ldots,\integer{m}_{\integer{j}}$ with, respectively, sentences $\CAPPSI_1,\ldots,\CAPPSI_{\integer{j}}$, \df{sanctions} writing the sentence $\CAPPHI$ \Iff there's some substitution of \GSL{} sentences that, for the given schemas of \Rule{R}, results in $\CAPPSI_1,\ldots,\CAPPSI_{\integer{j}}$ and, for the may-add schema, results in $\CAPPHI$. 
\end{majorILnc}
\noindent{}As an example, consider again derivation \pmvref{secondexamplefinished}. 
The rule \Rule{$\HORSESHOE$-Elim} was applied to line 2, which had sentence $\horseshoe{\Bl}{\parconjunction{\Cl}{\Dl}}$, and line 3, which had sentence $\Bl$, to get line 4, which had sentence $\conjunction{\Cl}{\Dl}$. 
Using definition \ref{RuleSanctioning} we can show that this move is sanctioned by \Rule{$\HORSESHOE$-Elim} by noting, from table \mvref{GSD}, that \Rule{$\HORSESHOE$-Elim} has two given schemas, $\horseshoe{\CAPTHETA}{\CAPPSI}$ and $\CAPTHETA$, and the may-add scheme $\CAPPSI$. 
Substituting $\CAPTHETA=\Bl$ and $\CAPPSI=\conjunction{\Cl}{\Dl}$ in the given schemas gets us lines 2 and 3, while making this same substitution in the may-add schema gets us line 4. 

Lastly, we end with the following theorem:
\begin{THEOREM}{\LnpTC{Soundess of Basic GSD Rules}}
	Every application of every basic rule of \GSD{} is truth-preserving.
\end{THEOREM}
\begin{PROOF}
	It can be shown that: for any basic rule \Rule{R} of \GSD{}, if some substitution of \GSL{} sentences into the given schema of \Rule{R} results in \GSL{} sentences $\CAPPSI_1,\ldots,\CAPPSI_{\integer{n}}$ and that same substitution into the may-add schema of \Rule{R} results in the \GSL{} sentence $\CAPPHI$, then $\CAPPSI_1,\ldots,\CAPPSI_{\integer{n}}\sdtstile{}{}\CAPPHI$.
	Call this the truth-preservation lemma.
	(We asked the reader to show that the lemma is true in exercises \pmvref{exercises:truth-preservation lemma}.)
	Now consider some arbitrary application of some basic rule \Rule{R} of \GSD{}. 
	Say that in this application \Rule{R} is applied to sentences $\CAPTHETA_1,\ldots,\CAPTHETA_{\integer{m}}$ and permits, or sanctions, you to write down $\DELTA$. 
	By definition \mvref{RuleSanctioning}, there's some substitution of \GSL{} sentences that, for the given schemas of \Rule{R}, results in $\CAPTHETA_1,\ldots,\CAPTHETA_{\integer{m}}$ and, for the may-add schema, results in $\DELTA$. 
	By the truth preservation lemma, $\CAPTHETA_1,\ldots,\CAPTHETA_{\integer{m}}\sdtstile{}{}\DELTA$.
	By definition \mvref{Derivation Rule Soundness}, this application is truth-preserving. 
	
	This proof doesn't cover \Rule{$\HORSESHOE$-Intro} or \Rule{Assume}.  We will have to give these rules special treatment.  The former is the only rule that eliminates an assumption, and the latter is the only rule that adds an assumption, so they each have a special role.
\end{PROOF}

Recall from section \mvref{Derivation Preliminaries} that we want to use derivations as a way of showing that a sentence is a logical truth, or of showing that some set of sentences entails some other sentence.
Specifically, we want to use derivations in \GSD{} and \GQD{} to show that sentences of \GSL{} and \GQL{} are \CAPS{tft} and \CAPS{qt}, or to show entailments between sentences of \GSL{} or between sentences of \GQL{}.  
But derivations can only fill this role if our derivation systems are sound.  Generally speaking, a derivation system is only totally satisfactory if it is also complete.
Let \Language{L} be some formal language for which we have defined some kind of models.
\begin{majorILnc}{\LnpDC{LSoundness}}
A derivation system \DerivationSystem{D} for \Language{L} is \nidf{sound}\index{soundness|textbf} \Iff for every set $\Delta$ of sentences of \Language{L} and every sentence $\CAPPHI$ of \Language{L}, if $\Delta\sststile{}{}\CAPPHI$, then $\Delta\sdtstile{}{}\CAPPHI$.
\end{majorILnc} 


%%%%%%%%%%%%%%%%%%%%%%%%%%%%%%%%%%%%%%%%%%%%%%%%%%
\section{Soundness}
%%%%%%%%%%%%%%%%%%%%%%%%%%%%%%%%%%%%%%%%%%%%%%%%%%

\subsection{Soundness of \GSD{}}
We begin by proving the soundness of \GSD{}.\index{soundness!of \GSD{}}
\begin{THEOREM}{\LnpTC{Soundness of Sentential Logic} \GSD{} Soundness Theorem:}
\GSD{} is sound; i.e., for every set $\Delta$ of sentences of \GSL{} and every sentence $\CAPPHI$ of \GSL{}, if $\Delta\sststile{}{}\CAPPHI$ in \GSD{}, then $\Delta\sdtstile{}{}\CAPPHI$.
\end{THEOREM}
\noindent{}We will first prove the following result about derivations:
\begin{THEOREM}{\LnpTC{Main GSL Soundness Lemma} Soundness Lemma:}
For any sequence of derivation lines that is a derivation (see definition \pmvref{Recursive definition of Derivation}), the sentence $\CAPPHI$ on the last line is entailed by the set $\Delta$ of sentences that are on unboxed lines and are sanctioned by \Rule{Assumption}. 
\end{THEOREM}
\noindent{}Since the definition of a derivation (def. \pmvref{Recursive definition of Derivation}) is a recursive definition, the most natural way to prove theorem \ref{Main GSL Soundness Lemma} is through a recursive proof. 
The recursive proof we give will use three easily proved lemmas (proofs are left to the reader; the first lemma, on monotonicity, is also used to prove thm. \ref{Soundness of Sentential Logic}). 
Two are facts about entailment and one is about derivation.
\begin{THEOREM}{\LnpTC{Monotonicity of Entailment} Monotonicity of Entailment:}
For all \GSL{} sentences $\CAPPHI_1,\ldots,\CAPPHI_{\integer{n}},\CAPTHETA,\CAPPSI$:
\begin{center}
If $\CAPPHI_1,\CAPPHI_2,\ldots,\CAPPHI_{\integer{n}}\sdtstile{}{}\CAPPSI$, then $\CAPPHI_1,\CAPPHI_2,\ldots,\CAPPHI_{\integer{n}},\CAPTHETA\sdtstile{}{}\CAPPSI$
\end{center}
\end{THEOREM}
\begin{THEOREM}{\LnpTC{Transitivity of Entailment} Transitivity of Entailment:}
For all \GSL{} sentences $\CAPPHI_1,\ldots,\CAPPHI_{\integer{n}}$, $\CAPTHETA$, and $\CAPPSI_1,\ldots,\CAPPSI_{\integer{k}}$:
\begin{center}
\begin{tabular}{ l@{\hspace{.25em}}l@{\hspace{.25em}}l }
If & $\CAPPHI_1,\CAPPHI_2,\ldots,\CAPPHI_{\integer{n}}\sdtstile{}{}\CAPPSI_1$ & and \\
   & $\CAPPHI_1,\CAPPHI_2,\ldots,\CAPPHI_{\integer{n}}\sdtstile{}{}\CAPPSI_2$ & and \\
   & \hspace{.5in} $\vdots$ &  \\
   & $\CAPPHI_1,\CAPPHI_2,\ldots,\CAPPHI_{\integer{n}}\sdtstile{}{}\CAPPSI_{\integer{k}}$ & and \\
   & $\CAPPSI_1,\CAPPSI_2,\ldots,\CAPPSI_{\integer{k}}\sdtstile{}{}\CAPTHETA$ & then: \\
   & & $\CAPPHI_1,\CAPPHI_2,\ldots,\CAPPHI_{\integer{n}}\sdtstile{}{}\CAPTHETA$   \\
\end{tabular}
\end{center}
\end{THEOREM}
\begin{THEOREM}{\LnpTC{Non-decreasing Assumption Principle} Non-decreasing Assumption Principle (NDAP):}
If $\Delta_1$ is the set of assumptions of an unboxed line and $\Delta_2$ is the set of assumptions of a later unboxed line, then $\Delta_1$ is a subset of $\Delta_2$, i.e., $\Delta_1\subseteq\Delta_2$.
\end{THEOREM}
\begin{PROOFOF}{Thm. \ref{Main GSL Soundness Lemma}, Soundness Lemma}
\begin{description}

\item[Base Step:] 
The base case is a single-line derivation $\Derivation{D}$ sanctioned by the rule \Rule{Assumption}. 
Say the sentence on that line is $\CAPPHI$.
We have to show that the sentence on the last line is entailed by all the sentences, on unboxed lines, that are sanctioned by \Rule{Assumption}. 
But in this case the sentence on the last line is $\CAPPHI$, and the set of unboxed sentences sanctioned by \Rule{Assumption} only contains $\CAPPHI$. 
Obviously $\CAPPHI\sdtstile{}{}\CAPPHI$, so the theorem holds in the base case. 

\item[Inheritance Step:] 
In the inheritance step we start with a derivation $\Derivation{D}$.
Say $\Delta$ is the set of unboxed assumptions occurring in $\Derivation{D}$, and $\Delta_\integer{i}$ is the set of unboxed assumptions occurring in $\Derivation{D}$ up to (and including) line number $\integer{i}$. 
%We then want to show that if some rule \Rule{R} of \GSD{} applied to unboxed lines of $\Derivation{D}$ sanctions writing down sentence $\CAPPHI$, then $Delta^*\sdtstile{}{}\CAPPHI$, where $\Delta^*$ is the set of unboxed assumptions for the new line with $\CAPPHI$.\footnote{Note 
We then want to show that if we add another line to $\Derivation{D}$ with sentence $\CAPPHI$ sanctioned by rule \Rule{R}, then $\Delta^*\sdtstile{}{}\CAPPHI$, where $\Delta^*$ is the set of unboxed assumptions for the new line. 
(Notice that this requires more than the fact that the rules are truth preserving; we also have to attend to how we define a derivation. The fact that the rules are truth preserving is essential of course.) 
We need to consider each rule \Rule{R} of \GSD{} as its own case.

\begin{description}

\item[Recursive Assumption:]  
The recursive assumption is that for all lines $\Derivation{L}_\integer{i}$ in the derivation $\Derivation{D}$, if $\CAPPHI$ is the sentence on the line, then $\Delta_{\integer{i}}\sdtstile{}{}\CAPPHI$. 

\item[\Rule{Assumption}:] 
Say we add another line to $\Derivation{D}$ with sentence $\CAPPHI$ sanctioned by \Rule{Assumption}. 
Note that the set $\Delta^*$ of unboxed assumptions for this new line are those in $\Delta$ plus $\CAPPHI$. 
We know $\CAPPHI\sdtstile{}{}\CAPPHI$, and $\Delta,\CAPPHI\sdtstile{}{}\CAPPHI$ follows from this by monotonicity.

\item[\Rule{Repetition}:] 
Say $\CAPPHI$ already occurs somewhere in $\Derivation{D}$, say on line number $\integer{i}$. 
Then $\Delta_{\integer{i}}\sdtstile{}{}\CAPPHI$ by the recursive assumption. 
Suppose we add another line to $\Derivation{D}$ with $\CAPPHI$ sanctioned by \Rule{Repetition}. 
By NDAP, $\Delta_{\integer{i}}\subseteq\Delta^*$. 
So by monotonicity, $\Delta^*\sdtstile{}{}\CAPPHI$.

\item[\Rule{$\VEE$-Intro}:]
Assume we add another line to $\Derivation{D}$ with sentence $\CAPPHI$ sanctioned by \Rule{$\VEE$-Intro}. 
Then there's some earlier line $\integer{i}$ with the sentence $\CAPTHETA$ and $\CAPPHI$ is a disjunction with $\CAPTHETA$ as one disjunct. 
We have that $\Delta_{\integer{i}}$ is the set of unboxed assumptions of line $\integer{i}$, and by NDAP $\Delta_{\integer{i}}\subseteq\Delta^*$.
By the recursive assumption, $\Delta_{\integer{i}}\sdtstile{}{}\CAPTHETA$.
So by monotonicity, $\Delta^*\sdtstile{}{}\CAPTHETA$.
Because the rule is truth preserving we know that $\CAPPHI$ is a disjunction with $\CAPTHETA$ as one disjunct, $\CAPTHETA\sdtstile{}{}\CAPPHI$. 
So by transitivity, $\Delta^*\sdtstile{}{}\CAPPHI$. 

\item[\Rule{$\WEDGE\!$-Elim}:]
Say we add another line to $\Derivation{D}$ with sentence $\CAPPHI$ sanctioned by \Rule{$\WEDGE\!$-Elim}.
Then there's some earlier line $\integer{i}$ with the sentence $\conjunction{\CAPTHETA_\integer{1}}{\conjunction{\ldots}{\CAPTHETA_{\integer{m}}}}$ and $\CAPPHI$ is one of the conjuncts. 
As before, by NDAP $\Delta_{\integer{i}}\subseteq\Delta^*$.
By the recursive assumption, $\Delta_{\integer{i}}\sdtstile{}{}\conjunction{\CAPTHETA_\integer{1}}{\conjunction{\ldots}{\CAPTHETA_{\integer{m}}}}$. 
So by monotonicity, $\Delta^*\sdtstile{}{}\conjunction{\CAPTHETA_\integer{1}}{\conjunction{\ldots}{\CAPTHETA_{\integer{m}}}}$.
And $\CAPPHI$ is one of the conjuncts of $\conjunction{\CAPTHETA_\integer{1}}{\conjunction{\ldots}{\CAPTHETA_{\integer{m}}}}$, so it follows that $\conjunction{\CAPTHETA_\integer{1}}{\conjunction{\ldots}{\CAPTHETA_{\integer{m}}}}\sdtstile{}{}\CAPPHI$. 
So by transitivity, $\Delta^*\sdtstile{}{}\CAPPHI$.

\item[\Rule{$\NEGATION$-Elim}:] 
Suppose we add another line to $\Derivation{D}$ with sentence $\CAPPHI$ sanctioned by \Rule{$\NEGATION$-Elim}.
Then there's some earlier line $\integer{i}$ with the sentence $\horseshoe{\negation{\CAPPHI}}{\parconjunction{\CAPPSI}{\negation{\CAPPSI}}}$.
As before, by NDAP $\Delta_{\integer{i}}\subseteq\Delta^*$.
By the recursive assumption, $\Delta_{\integer{i}}\sdtstile{}{}\horseshoe{\negation{\CAPPHI}}{\parconjunction{\CAPPSI}{\negation{\CAPPSI}}}$.
So by monotonicity, $\Delta^*\sdtstile{}{}\horseshoe{\negation{\CAPPHI}}{\parconjunction{\CAPPSI}{\negation{\CAPPSI}}}$.
Since the \CAPS{rhs} of $\horseshoe{\negation{\CAPPHI}}{\parconjunction{\CAPPSI}{\negation{\CAPPSI}}}$ is false in all models (it's \CAPS{tff}), the conditional is true in a model $\IntA{}$ only if the \CAPS{lhs} is false in $\IntA{}$.
So if the conditional is true in a model $\IntA{}$, $\CAPPHI$ is true in $\IntA{}$.
In other words, $\horseshoe{\negation{\CAPPHI}}{\parconjunction{\CAPPSI}{\negation{\CAPPSI}}}\sdtstile{}{}\CAPPHI$. 
So by transitivity, $\Delta^*\sdtstile{}{}\CAPPHI$.

\item[\Rule{$\NEGATION$-Intro}:] 
This case is very similar to the last and is left to the reader. 

\item[\Rule{$\HORSESHOE$-Elim}:]
Assume we add another line to $\Derivation{D}$ with sentence $\CAPPHI$ sanctioned by \Rule{$\HORSESHOE$-Elim}.
Then there's two earlier lines $\integer{i}$ and $\integer{j}$, and (say) line $\integer{i}$ has a sentence $\horseshoe{\CAPTHETA}{\CAPPHI}$ and line $\integer{j}$ has sentence $\CAPTHETA$. 
By NDAP we have that $\Delta_{\integer{i}}\subseteq\Delta^*$ and $\Delta_{\integer{j}}\subseteq\Delta^*$.
By the recursive assumption, $\Delta_{\integer{i}}\sdtstile{}{}\horseshoe{\CAPTHETA}{\CAPPHI}$ and $\Delta_{\integer{j}}\sdtstile{}{}\CAPTHETA$.
By monotonicity, 
$\Delta^*\sdtstile{}{}\horseshoe{\CAPTHETA}{\CAPPHI}$ and $\Delta^*\sdtstile{}{}\CAPTHETA$.
Because the rule is truth preserving we know that $\CAPTHETA,\horseshoe{\CAPTHETA}{\CAPPHI}\sdtstile{}{}\CAPPHI$.
So by transitivity, $\Delta^*\sdtstile{}{}\CAPPHI$.

\item[\Rule{$\TRIPLEBAR$-Elim}:] The argument for each of the two versions of \Rule{$\TRIPLEBAR$-Elim} is the same as that for \Rule{$\HORSESHOE$-Elim}.

\item[\Rule{$\TRIPLEBAR$-Intro}:]
Say we add another line to $\Derivation{D}$ with sentence $\CAPPHI=\triplebar{\CAPPHI}{\CAPTHETA}$ sanctioned by \Rule{$\TRIPLEBAR$-Intro}.
Then there's two earlier lines $\integer{i}$ and $\integer{j}$, and (say) line $\integer{i}$ has a sentence $\horseshoe{\CAPTHETA}{\CAPPHI}$ and line $\integer{j}$ has sentence $\horseshoe{\CAPPHI}{\CAPTHETA}$. 
By NDAP we have that $\Delta_{\integer{i}}\subseteq\Delta^*$ and $\Delta_{\integer{j}}\subseteq\Delta^*$.
By the recursive assumption, $\Delta_{\integer{i}}\sdtstile{}{}\horseshoe{\CAPTHETA}{\CAPPHI}$ and $\Delta_{\integer{j}}\sdtstile{}{}\horseshoe{\CAPPHI}{\CAPTHETA}$.
By monotonicity, $\Delta^*\sdtstile{}{}\horseshoe{\CAPTHETA}{\CAPPHI}$ and $\Delta^*\sdtstile{}{}\horseshoe{\CAPPHI}{\CAPTHETA}$.
Because the rule is truth preserving we know that $\horseshoe{\CAPPHI}{\CAPTHETA},\horseshoe{\CAPTHETA}{\CAPPHI}\sdtstile{}{}\triplebar{\CAPPHI}{\CAPTHETA}$.
So by transitivity, $\Delta^*\sdtstile{}{}\CAPPHI$.

\item[\Rule{$\WEDGE\!$-Intro}:]
Suppose we add another line to $\Derivation{D}$ with sentence $\CAPPHI=\conjunction{\CAPPHI_1}{\conjunction{\ldots}{\CAPPHI_{\integer{m}}}}$ sanctioned by \Rule{$\WEDGE\!$-Intro}.
Then there are $\integer{m}$ earlier lines numbered $\integer{i}_{1},\ldots,\integer{i}_{\integer{m}}$ with, respectively, sentences $\CAPPHI_1,\ldots,\CAPPHI_{\integer{m}}$. 
By NDAP we have that $\Delta_{\integer{i}_1}\subseteq\Delta^*,\ldots,\Delta_{\integer{i}_\integer{m}}\subseteq\Delta^*$.
By the recursive assumption, $\Delta_{\integer{i}_1}\sdtstile{}{}\CAPPHI_1,\ldots,\Delta_{\integer{i}_\integer{m}}\sdtstile{}{}\CAPPHI_{\integer{m}}$.
So by monotonicity, $\Delta^*\sdtstile{}{}\CAPPHI_1,\ldots,\Delta^*\sdtstile{}{}\CAPPHI_{\integer{m}}$.
We now observe that $\CAPPHI_1,\ldots,\CAPPHI_{\integer{m}}\sdtstile{}{}\conjunction{\CAPPHI}{\conjunction{\ldots}{\CAPPHI_{\integer{m}}}}$.
So by transitivity, $\Delta^*\sdtstile{}{}\CAPPHI$.

\item[\Rule{$\VEE$-Elim}:]
Assume we add another line to $\Derivation{D}$ with sentence $\CAPPHI$ sanctioned by \Rule{$\VEE$-Elim}.
Then there are $\integer{m}+1$ earlier lines numbered $\integer{i}_{1},\ldots,\integer{i}_{\integer{m}},\integer{i}_{\integer{m}+1}$ with, respectively, sentences $\horseshoe{\CAPTHETA_1}{\CAPPHI}$, $\ldots$, $\horseshoe{\CAPTHETA_{\integer{m}}}{\CAPPHI}$, and  $\disjunction{\CAPTHETA_1}{\disjunction{\ldots}{\CAPTHETA_{\integer{m}}}}$.
By NDAP we have that $\Delta_{\integer{i}_1}\subseteq\Delta^*,\ldots,\Delta_{\integer{i}_\integer{m}}\subseteq\Delta^*$ and $\Delta_{\integer{i}_{\integer{m}+1}}\subseteq\Delta^*$.
By the recursive assumption, $\Delta_{\integer{i}_1}\sdtstile{}{}\horseshoe{\CAPTHETA_1}{\CAPPHI}$, $\ldots$, $\Delta_{\integer{i}_\integer{m}}\sdtstile{}{}\horseshoe{\CAPTHETA_{\integer{m}}}{\CAPPHI}$ and $\Delta_{\integer{i}_{\integer{m}+1}}\sdtstile{}{}\disjunction{\CAPTHETA_1}{\disjunction{\ldots}{\CAPTHETA_{\integer{m}}}}$.
By monotonicity, $\Delta^*\sdtstile{}{}\horseshoe{\CAPTHETA_1}{\CAPPHI}$, $\ldots$, $\Delta^*\sdtstile{}{}\horseshoe{\CAPTHETA_{\integer{m}}}{\CAPPHI}$ and $\Delta^*\sdtstile{}{}\disjunction{\CAPTHETA_1}{\disjunction{\ldots}{\CAPTHETA_{\integer{m}}}}$.
We now observe that $\horseshoe{\CAPTHETA_1}{\CAPPHI},\ldots,\horseshoe{\CAPTHETA_{\integer{m}}}{\CAPPHI},\disjunction{\CAPTHETA_1}{\disjunction{\ldots}{\CAPTHETA_{\integer{m}}}}\sdtstile{}{}\CAPPHI$.
So by transitivity, $\Delta^*\sdtstile{}{}\CAPPHI$.

\item[\Rule{$\HORSESHOE$-Intro}:]
Like \Rule{Assumption}, the assumptions change in \Rule{$\HORSESHOE$-Intro}. 
If the new line sanctioned by \Rule{$\HORSESHOE$-Intro} has sentence $\horseshoe{\CAPPHI}{\CAPTHETA}$ and unboxed assumptions $\Delta^*$, then earlier we have an assumption line (now in a box) that starts with $\CAPPHI$ and $\Delta^*$ as its other assumptions, and we have a line (now at the bottom of the box) with $\CAPTHETA$ on it with assumptions $\CAPPHI,\Delta^*$. 
By the recursive assumption we have that $\CAPPHI,\Delta^*\sdtstile{}{}\CAPTHETA$. 
Consider any model $\IntA{}$ that makes $\Delta^*$ true;
if it also makes $\CAPPHI$ true, then $\CAPTHETA$ is true in $\IntA{}$ as well and so is $\horseshoe{\CAPPHI}{\CAPTHETA}$. 
If $\IntA{}$ makes $\CAPPHI$ false, then $\horseshoe{\CAPPHI}{\CAPTHETA}$ is true. 
(Notice that this step only works because we defined the conditional to be true when the \CAPS{lhs} is false.)
So if $\IntA{}$ makes $\Delta^*$ true, it makes $\horseshoe{\CAPPHI}{\CAPTHETA}$ true too. 
So, $\Delta^*\sdtstile{}{}\horseshoe{\CAPPHI}{\CAPTHETA}$.  

\end{description}
\item[Closure Step:] We have now covered all the generating cases for derivations. By the closure clause of the definition, we have proved soundness for all derivations. 
\end{description}
\end{PROOFOF} 

\begin{PROOFOF}{Thm. \ref{Soundness of Sentential Logic}, SL Soundness Theorem}
Assume that $\Delta$ is a set of \GSL{} sentences. 
Assume $\Delta\sststile{}{}\CAPPHI$ and consider some derivation $\Derivation{D}$ of $\CAPPHI$ from $\Delta$. 
Let $\Delta'$ be the set of sentences in $\Delta$ that appear as unboxed assumptions in $\Derivation{D}$. 
By the soundness lemma (Thm. \pmvref{Main GSL Soundness Lemma}), $\Delta'\sdtstile{}{}\CAPPHI$. 
It follows immediately by monotonicity that $\Delta\sdtstile{}{}\CAPPHI$.  
\end{PROOFOF} 

\subsection{Soundness of \GQD{}}
In this section we prove that \GQD{} is also sound.\index{soundness!of \GQD{}}
\begin{THEOREM}{\LnpTC{Soundness of Quantifier Logic} \GQD{} Soundness Theorem:}
\GQD{} is sound; i.e., for every set $\Delta$ of sentences of \GQL{} and every sentence $\CAPPHI$ of \GQL{}, if $\Delta\sststile{}{}\CAPPHI$ in \GSD{}, then $\Delta\sdtstile{}{}\CAPPHI$.
\end{THEOREM}
\noindent{}The proof given in the last section of the \GSL{} Soundness Theorem (Thm. \pmvref{Soundness of Sentential Logic}) can be carried over to the \GQL{} Soundness Theorem. 
That proof relied on the monotonicity of entailment and the soundness lemma (Them. \pmvref{Main GSL Soundness Lemma}). 
It should be clear that entailment is also monotonic in the case of \GQL{}. 
Since \GQD{} is just an extension of \GSD{} (it's just \GSD{} plus the rules for the quantifiers in table \pncmvref{GQD}), all we need to do to show that the soundness lemma holds for \GQD{} is add a case, for each new rule of \GQD{}, to the inheritance step of the proof of the soundness lemma for \GSD{}.
\begin{PROOFOF}{Thm. \ref{Main GSL Soundness Lemma} for GQD}
\begin{description}

\item[Base Step:] 
The base case has been covered in the proof for \GSD{}. 

\item[Inheritance Step:] 
Just as in the proof for \GSD{}, in the inheritance step we start with a derivation $\Derivation{D}$.
Say $\Delta$ is the set of unboxed assumptions occurring in $\Derivation{D}$, and $\Delta_\integer{i}$ is the set of unboxed assumptions occurring in $\Derivation{D}$ up to (and including) line number $\integer{i}$. 
%We then want to show that if some rule \Rule{R} of \GSD{} applied to unboxed lines of $\Derivation{D}$ sanctions writing down sentence $\CAPPHI$, then $Delta^*\sdtstile{}{}\CAPPHI$, where $\Delta^*$ is the set of unboxed assumptions for the new line with $\CAPPHI$.\footnote{Note 
We then want to show that if we add another line to $\Derivation{D}$ with sentence $\CAPPHI$ sanctioned by rule \Rule{R}, then $\Delta^*\sdtstile{}{}\CAPPHI$, where $\Delta^*$ is the set of unboxed assumptions for the new line. %\footnote{Note 
%that this is not the same as showing that the rule \Rule{R} is truth-preserving (see def. \pncmvref{Derivation Rule Soundness}).
%} 
Again we need to consider each rule \Rule{R} of \GQD{} as its own case.
Most of the rules have already been covered in the proof of \GSD{}, so we only need to cover the introduction and elimination rules for the quantifiers. 

\begin{description}

\item[Recursive Assumption:]  
The recursive assumption, as in the proof for \GSD{}, is that for all lines $\Derivation{L}_\integer{i}$ in the derivation $\Derivation{D}$, if $\CAPPHI$ is the sentence on the line, then $\Delta_{\integer{i}}\sdtstile{}{}\CAPPHI$. 

\item[\Rule{$\forall$-Elim}:]
Say we add another line to $\Derivation{D}$ with sentence $\CAPPHI\variable{s}/\BETA$ sanctioned by \Rule{$\forall$-Elim}.
Then there's some earlier line $\integer{i}$ with the sentence $\universal{\BETA}\CAPPHI$. 
We have that $\Delta_{\integer{i}}$ is the set of unboxed assumptions of line $\integer{i}$, and by NDAP $\Delta_{\integer{i}}\subseteq\Delta^*$.
By the recursive assumption, $\Delta_{\integer{i}}\sdtstile{}{}\universal{\BETA}\CAPPHI$.
So by monotonicity, $\Delta^*\sdtstile{}{}\universal{\BETA}\CAPPHI$.

Assume some model $\IntA$ such that $\universal{\BETA}\CAPPHI$ is true.  By the def. of truth for $\forall$, $\CAPPHI{\variable{t}/\BETA}$ is true on all $\variable{t}$-variants of $\IntA$.  Notice that $\CAPPHI{\variable{t}/\BETA}$ and $\CAPPHI{\variable{s}/\BETA}$ are exactly the same, except that the latter has $\variable{s}$ substituted for $\variable{t}$.  These sentences satisfy condition (1) of Dragnet.

Now let's consider, in particular, the $\variable{t}$-variant that assigns to $\variable{t}$ what $\IntA$ assigns to $\variable{s}$.  Name that $\variable{t}$-variant $\As{\variable{t}}{}$.  The models $\IntA$ and $\As{\variable{t}}{}$ meet Dragnet condition (2).  Thus, by Dragnet, $\CAPPHI{\variable{t}/\BETA}$ is true on $\As{\variable{t}}{}$ iff $\CAPPHI{\variable{s}/\BETA}$ is true on $\IntA$.  Therefore, $\CAPPHI{\variable{s}/\BETA}$ is true on $\IntA$.

Any model such that $\universal{\BETA}\CAPPHI$ is true also makes $\CAPPHI{\variable{s}/\BETA}$ true.  Thus, $\universal{\BETA}\CAPPHI\sdtstile{}{}\CAPPHI\variable{s}/\BETA$.  So by transitivity, $\Delta^*\sdtstile{}{}\CAPPHI\variable{s}/\BETA$.   

\item[\Rule{$\exists$-Intro}:]
Say we add another line to $\Derivation{D}$ with sentence $\existential{\BETA}\CAPPHI$ sanctioned by \Rule{$\exists$-Intro}.
Then there's some earlier line $\integer{i}$ with the sentence $\CAPPHI\variable{s}/\BETA$.
Again we have that $\Delta_{\integer{i}}$ is the set of unboxed assumptions of line $\integer{i}$, and by NDAP $\Delta_{\integer{i}}\subseteq\Delta^*$.
By the recursive assumption, $\Delta_{\integer{i}}\sdtstile{}{}\CAPPHI\variable{s}/\BETA$.
By monotonicity, $\Delta^*\sdtstile{}{}\CAPPHI\variable{s}/\BETA$.

Assume some model $\IntA$ such that $\existential{\BETA}\CAPPHI$ is false.  By the def. of truth for $\exists$, there is no $\variable{t}$-variant of $\IntA$ that makes $\CAPPHI{\variable{t}/\BETA}$ true.  Notice that $\CAPPHI{\variable{t}/\BETA}$ and $\CAPPHI{\variable{s}/\BETA}$ are exactly the same, except that the latter has $\variable{s}$ substituted for $\variable{t}$.  These sentences satisfy condition (1) of Dragnet.

Now let's consider, in particular, the $\variable{t}$-variant that assigns to $\variable{t}$ what $\IntA$ assigns to $\variable{s}$.  Name that $\variable{t}$-variant $\As{\variable{t}}{}$.  The models $\IntA$ and $\As{\variable{t}}{}$ meet Dragnet condition (2).  Thus, by Dragnet, $\CAPPHI{\variable{t}/\BETA}$ is true on $\As{\variable{t}}{}$ iff $\CAPPHI{\variable{s}/\BETA}$ is true on $\IntA$.  Therefore, $\CAPPHI{\variable{s}/\BETA}$ is false on $\IntA$.

Any model that makes $\existential{\BETA}\CAPPHI$ false also makes $\CAPPHI{\variable{s}/\BETA}$ false.  Thus,  $\CAPPHI\variable{s}/\BETA\sdtstile{}{}\existential{\BETA}\CAPPHI$. So by transitivity, $\Delta^*\sdtstile{}{}\existential{\BETA}\CAPPHI$.

\item[\Rule{$\forall$-Intro}:]
Say we add another line to $\Derivation{D}$ with sentence $\universal{\BETA}\CAPPHI$ sanctioned by \Rule{$\forall$-Intro}.
Then there's some earlier line $\integer{i}$ with the sentence $\CAPPHI\variable{s}/\BETA$. 
Again we have that $\Delta_{\integer{i}}$ is the set of unboxed assumptions of line $\integer{i}$, and by NDAP $\Delta_{\integer{i}}\subseteq\Delta^*$.
By the recursive assumption, $\Delta_{\integer{i}}\sdtstile{}{}\CAPPHI\variable{s}/\BETA$. 
By monotonicity, $\Delta^*\sdtstile{}{}\CAPPHI\variable{s}/\BETA$.
However we know that $\CAPPHI\variable{s}/\BETA$ does not entail $\universal{\BETA}\CAPPHI$, so we have to do some extra work and make use of the restrictions on the rule \Rule{$\forall$-Intro}. 

Let $\IntA$ be some model that makes all of $\Delta^*$ true; and let's assume for \emph{reductio} that $\IntA$ that makes $\universal{\BETA}\CAPPHI$ false.  One of the restrictions for \Rule{$\forall$-Intro} is that $\variable{s}$ must not occur in $\universal{\BETA}\CAPPHI$.  Hence, by the Free Choice Theorem, $\CAPPHI\variable{s}/\BETA$ is false on some $\variable{s}$-variant of $\IntA$.  Let's name that $\variable{s}$-variant $\As{\variable{s}}{}$.

The other rule restriction for \Rule{$\forall$-Intro} is that $\variable{s}$ must not occur in $\Delta^*$.  The variant $\As{\variable{s}}{}$ differs from $\IntA$ only on the assignment to $\variable{s}$; otherwise, they make all the same assignments.  Because $\variable{s}$ doesn't occur in $\Delta^*$ and $\IntA$ makes all of $\Delta^*$ true, $\As{\variable{s}}{}$ also makes all of $\Delta^*$ true.  The assignment $\As{\variable{s}}{}$ makes to $\variable{s}$ doesn't matter for this result.

Because $\Delta^*\sdtstile{}{}\CAPPHI\variable{s}/\BETA$, $\As{\variable{s}}{}$ makes $\CAPPHI\variable{s}/\BETA$ true.  But we had concluded that $\CAPPHI\variable{s}/\BETA$ is false on $\As{\variable{s}}{}$.  We've inferred a contradiction.  Our assumption that $\IntA$ makes $\universal{\BETA}\CAPPHI$ false must be wrong.

Therefore, if $\IntA$ is a model that makes all of $\Delta^*$ true, then $\IntA$ makes $\universal{\BETA}\CAPPHI$ true as well. 
So, $\Delta^*\sdtstile{}{}\universal{\BETA}\CAPPHI$.

\item[\Rule{$\exists$-Elim}:]
Say we add another line to $\Derivation{D}$ with sentence $\CAPTHETA$ sanctioned by \Rule{$\exists$-Elim}.
Then there's some earlier line $\integer{i}$ with the sentence $\horseshoe{\CAPPHI\variable{s}/\BETA}{\CAPTHETA}$ and an earlier line $\integer{j}$ with the sentence $\existential{\BETA}\CAPPHI$. 
Again we have that $\Delta_{\integer{i}}$ is the set of unboxed assumptions of line $\integer{i}$ and $\Delta_{\integer{j}}$ the unboxed assumptions of line $\integer{j}$.
By NDAP $\Delta_{\integer{i}}\subseteq\Delta^*$ and $\Delta_{\integer{j}}\subseteq\Delta^*$.
By the recursive assumption, $\Delta_{\integer{i}}\sdtstile{}{}\horseshoe{\CAPPHI\variable{s}/\BETA}{\CAPTHETA}$ and $\Delta_{\integer{j}}\sdtstile{}{}\existential{\BETA}\CAPPHI$.
By monotonicity, $\Delta^*\sdtstile{}{}\horseshoe{\CAPPHI\variable{s}/\BETA}{\CAPTHETA}$ and $\Delta^*\sdtstile{}{}\existential{\BETA}\CAPPHI$.
Again we have to do some extra work and make use of the restrictions on the rule \Rule{$\exists$-Elim} to show that $\Delta^*\sdtstile{}{}\CAPTHETA$. 

Let $\IntA$ be some model that makes all of $\Delta^*$ true. 
Because $\Delta^*\sdtstile{}{}\existential{\BETA}\CAPPHI$, $\IntA$ also makes $\existential{\BETA}\CAPPHI$ true.  One of the rule restrictions for \Rule{$\exists$-Elim} is that $\variable{s}$ must not occur in $\existential{\BETA}\CAPPHI$.  Hence, by the Free Choice theorem, $\CAPPHI{\variable{s}/\BETA}$ is true on some $\variable{s}$-variant of $\IntA$.  Name that $\variable{s}$-variant $\As{\variable{s}}{}$.  

Another of the rule restrictions for \Rule{$\exists$-Elim} is that $\variable{s}$ must not occur in $\Delta^*$.  The variant $\As{\variable{s}}{}$ makes all the same assignments as $\IntA$ except in what it assigns to $\variable{s}$.  Because $\IntA$ makes $\Delta^*$ true and $\Delta^*$ doesn't contain $\variable{s}$, $\As{\variable{s}}{}$ also makes $\Delta^*$ true.  The assignment $\As{\variable{s}}{}$ makes to $\variable{s}$ doesn't make any difference.

Thus, because $\Delta^*\sdtstile{}{}\horseshoe{\CAPPHI\variable{s}/\BETA}{\CAPTHETA}$, $\As{\variable{s}}{}$ makes $\horseshoe{\CAPPHI\variable{s}/\BETA}{\CAPTHETA}$ true.  We saw earlier that $\As{\variable{s}}{}$ makes $\CAPPHI\variable{s}/\BETA$ true, so $\As{\variable{s}}{}$ makes $\CAPTHETA$ true as well (def. of truth, $\HORSESHOE$).  

According to the third rule restriction for \Rule{$\exists$-Elim}, $\variable{s}$ must not occur in $\CAPTHETA$.  Because $\As{\variable{s}}{}$ makes $\CAPTHETA$ true and $\variable{s}$ isn't in $\CAPTHETA$, $\IntA$ also makes $\CAPTHETA$ true.  The assignment that $\As{\variable{s}}{}$ makes to $\variable{s}$ is irrelevant.

So, we have shown that $\Delta^*\sdtstile{}{}\CAPTHETA$.

\end{description}

\item[Closure Step:] We have now covered all the generating cases for derivations. By the closure clause of the definition, we have proved soundness for all derivations in \GQD{}. 

\end{description}
\end{PROOFOF} 

%%%%%%%%%%%%%%%%%%%%%%%%%%%%%%%%%%%%%%%%%%%%%%%%%%
\section{Completeness}\label{Section:Completeness for GSD}
%%%%%%%%%%%%%%%%%%%%%%%%%%%%%%%%%%%%%%%%%%%%%%%%%%

In this section we first prove the completeness of \GSD{}.

\begin{majorILnc}{\LnpDC{LRCompleteness}}
	A derivation system \DerivationSystem{D} for \Language{L} is \nidf{complete}\index{completeness|textbf} \Iff for every finite set $\Delta$ of sentences of \Language{L} and every sentence $\CAPPHI$ of \Language{L}, if $\Delta\sdtstile{}{}\CAPPHI$, then $\Delta\sststile{}{}\CAPPHI$.
\end{majorILnc} 
\noindent{}If we limit $\Delta$ to just the empty set, we get weak completeness:
\begin{majorILnc}{\LnpDC{LWCompleteness}}
	A derivation system \DerivationSystem{D} for \Language{L} is \nidf{weakly complete}\index{completeness!weak|textbf} \Iff for every sentence $\CAPPHI$ of \Language{L}, if $\sdtstile{}{}\CAPPHI$, then $\sststile{}{}\CAPPHI$.
\end{majorILnc} 
\noindent{}The following theorem can be proved using basic results we already have.  For other systems of logic, what we are calling completeness and weak completeness are not equivalent.  Because they are equivalent in our systems, we will not always distinguish them in what follows.  Strong completeness also holds for our systems, but is not trivially equivalent to completeness.  As in the first cases, there are systems that are complete but not strongly complete.
\begin{THEOREM}{\LnpTC{RegWeakCompletenessEquiv}}
	\GSD{} is weakly complete \Iff it's complete; and likewise for \GQD{}.
\end{THEOREM}
\begin{PROOF}
	$(\Leftarrow)$ This direction of the biconditional is trivial. 
	Assume that \GSD{}/\GQD{} is complete. 
	Then for any finite set $\Delta$, if $\Delta\sdtstile{}{}\CAPPHI$, then $\Delta\sststile{}{}\CAPPHI$. 
	By definition, this includes the case when $\Delta$ is the empty set. 
	Hence \GSD{}/\GQD{} is weakly complete. 
	
	$(\Rightarrow)$ Assume that \GSD{}/\GQD{} is weakly complete: for any sentence $\CAPPHI$, if $\sdtstile{}{}\CAPPHI$, then $\sststile{}{}\CAPPHI$. 
	Now assume that, for some finite set $\Delta$ of sentences and sentence $\CAPPHI$, $\Delta\sdtstile{}{}\CAPPHI$.
	Since $\Delta$ is finite, we can consider the conjunction of all the sentences in $\Delta$.
	Let $\DELTA$ be this conjunction. 
	%From the \GQL{} Entailment-Exponentiation Theorem (Thm. \pmvref{Exponentiation of Entailment GQL}) we know that $\DELTA\sdtstile{}{}\CAPPHI$ \Iff $\sdtstile{}{}\horseshoe{\DELTA}{\CAPPHI}$.
	We want to show that $\sdtstile{}{}\horseshoe{\DELTA}{\CAPPHI}$;
	to do so, assume that there's some model $\IntA$ that makes $\horseshoe{\DELTA}{\CAPPHI}$ false.
	By the definition of truth for $\HORSESHOE$ and $\WEDGE$, it follows that $\IntA$ makes all the conjuncts of $\DELTA$ true and $\CAPPHI$ false. 
	But that would mean that $\IntA$ makes all the sentences in $\Delta$ true and $\CAPPHI$ false.
	But we assumed that $\Delta\sdtstile{}{}\CAPPHI$, so there's no model $\IntA$ that makes $\horseshoe{\DELTA}{\CAPPHI}$ false.
	Hence $\sdtstile{}{}\horseshoe{\DELTA}{\CAPPHI}$, and so by weak completeness, $\sststile{}{}\horseshoe{\DELTA}{\CAPPHI}$.
	It should be clear to the reader that if $\sststile{}{}\horseshoe{\DELTA}{\CAPPHI}$, then $\DELTA\sststile{}{}\CAPPHI$.
	Hence, $\DELTA\sststile{}{}\CAPPHI$.
	Finally, since $\Delta\sststile{}{}\DELTA$ and $\sststile{}{}$ is transitive, $\Delta\sststile{}{}\CAPPHI$.
	%To show that $\Delta\sststile{}{}\CAPPHI$, first write each sentence in $\Delta$ (which are the conjuncts of $\DELTA$) as an assumption on a line in a derivation.
	%Then derive $\horseshoe{\DELTA}{\CAPPHI}$, which we know can be done without any assumptions. 
	%Next, use \Rule{$\WEDGE\!$-Intro} to conjoin all the assumptions from $\Delta$; this will result in $\DELTA$ on a line in the derivation. 
	%Then use \Rule{$\HORSESHOE$-Elim} on $\horseshoe{\DELTA}{\CAPPHI}$ and $\DELTA$. 
	%This will get us $\CAPPHI$ on a line with all and only the sentences of $\Delta$ as assumptions, thus showing that $\Delta\sststile{}{}\CAPPHI$. 
\end{PROOF}
\begin{majorILnc}{\LnpDC{LCompleteness}}
	A derivation system \DerivationSystem{D} for \Language{L} is \nidf{strongly complete}\index{completeness!strong|textbf} \Iff for every set $\Delta$ of sentences of \Language{L} and every sentence $\CAPPHI$ of \Language{L}, if $\Delta\sdtstile{}{}\CAPPHI$, then $\Delta\sststile{}{}\CAPPHI$.
\end{majorILnc} 
\noindent{}Note that in the definitions the \CAPS{rhs} of the biconditionals must hold even in the special case when $\Delta$ is the empty set and the special case when $\Delta$ is infinite. 
If we limit $\Delta$ so that it must be finite (but still allow it to be empty), we get (regular) completeness.
\noindent{}Note that both \GSD{} and \GQD{} are strongly complete, but there is no simple theorem that uses results we already have which extends weak completeness to strong completeness in the way this theorem (Thm. \ref{RegWeakCompletenessEquiv}) extends weak completeness to (regular) completeness.

Returning to strong completeness, letting $\Delta$ be infinite may seem problematic, since as we've defined them (def. \pmvref{Recursive definition of Derivation}) derivations can only have finitely many lines. 
Hence, a derivation can only have finitely many assumptions. 
And, as we've defined the single turnstile, $\Delta\sststile{}{}\CAPPHI$ iff there's a derivation of $\CAPPHI$ from the sentences in $\Delta$. 
But there's nothing problematic about letting $\Delta$ be infinite, because showing that there's a derivation of $\CAPPHI$ from the sentences in $\Delta$ doesn't require that the derivation use \emph{all} the sentences in $\Delta$ as assumptions. 
In general, even when $\Delta$ is finite, any derivation of $\CAPPHI$ from some subset of sentences in $\Delta$ will show that $\Delta\sststile{}{}\CAPPHI$. 
So, if $\Delta$ is infinite and $\Delta\sdtstile{}{}\CAPPHI$, if the derivation system \DerivationSystem{D} is complete we'll know that $\CAPPHI$ can be derived from some finite subset of sentences of $\Delta$.\footnote{Before turning to the proofs of these theorems, some historical background might be of interest. 
	As mentioned above (Sec. \ref{Sec:GQLSymbols}), quantificational languages were first developed by Frege, Peirce and Mitchell in the 1870's and 1880's. 
	But it wasn't until David Hilbert and Wilhelm Ackermann published their hugely influential text \emph{Grundz\"uge der theoretischen Logik} (Principles of Mathematical Logic) in \citeyear{Hilbert1928} that the question of completeness was clearly formulated. 
	While Kurt G\"odel, in his \citeyear{Godel1929} doctorial dissertation (republished in \citeyear{Godel1930}), is widely accepted as the first person to prove that quantificational logic is strongly complete, Church \citeyearpar[291,~fn.464]{Church1956} reports that the Jacques Herbrand's dissertation in 1930 had the essential material for the same proof.  
	Further, completeness follows from results of Skolem \citeyearpar{Skolem1928}, but since the question of completeness hadn't been clearly raised yet no one seems to have noticed. 
	Leon Henkin \citeyearpar{Henkin1949} later developed a method of proving completeness different from G\"odels. 
	Henkin's approach is probably the most common one used today in logic textbooks, but the proof we give here is a constructive proof closer to G\"odel's original.
	(Ours owes much to Willard Quine's completeness proof \citeyearpar{Quine1982}.)}



We'll prove that \GSD{} is weakly complete and then show that, for \GSD{}, completeness and weak completeness are equivalent. 
By theorem \mvref{RegWeakCompletenessEquiv}, this will be sufficient to show that \GSD{} is complete.
The equivalent statement is:
\begin{THEOREM}{\LnpTC{GSDCompletenessLemma} The \GSD{} Weak Completeness Lemma:}
For\index{completeness!weak \GSD{}} any sentence $\CAPPHI$ of \GSD{}, either $\CAPPHI\sststile{}{}\conjunction{\Al}{\negation{\Al}}$, or $\CAPPHI$ is true in some model $\IntA$.
\end{THEOREM}

%\noindent{}We need a systematic way of looking for derivations.  Our method will be to assume the opposite of the sentence of interest and then derive a sentence in DNF that is provably equivalent.  The advantage of DNF is that it's simple and transparent.

%Once we have a DNF sentence, it is easy to either extract a contradiction or to read off a model that makes the original sentence true.  This means that we can show either $\negation{\CAPPHI}\sdtstile{}{}\parconjunction{\Al}{\negation{\Al}}$ or define a model that makes $\negation{\CAPPHI}$ true.
%So we either have a derivation of $\CAPPHI$ in a few more steps, or a model that shows $\CAPPHI$ is not a logical truth.

Before proving the theorem, it will be useful to introduce a new exchange rule for \GQD{} 
and then show that anything we can derive using \GQD{} and this rule can be derived using \GQD{} alone. We call the rule \Rule{$\TRIPLEBAR$-Exchange}.
(Note that every application of \Rule{$\TRIPLEBAR$-Exchange} is truth preserving, as the last problem in exercise \pmvref{exercises:GSDTFETheorem}, extends theorem \pmvref{ExchangeRuleGSDSoundnessLemma}, to it.)
It's given in table \ref{GSDplusDNF}.
\begin{table}[!ht]
\renewcommand{\arraystretch}{1.5}
\begin{center}
\begin{tabular}{ p{1in} l l } %p{2.2in} p{2in}
\toprule
\textbf{Name} & \textbf{Given} & \textbf{May Add} \\ 
\midrule
\Rule{$\TRIPLEBAR$-Exchange} &  $\triplebar{\CAPTHETA}{\CAPPSI}$ & $\disjunction{\parconjunction{\CAPTHETA}{\CAPPSI}}{\parconjunction{\negation{\CAPTHETA}}{\negation{\CAPPSI}}}$ \\
\nopagebreak
 & $\disjunction{\parconjunction{\CAPTHETA}{\CAPPSI}}{\parconjunction{\negation{\CAPTHETA}}{\negation{\CAPPSI}}}$ &  $\triplebar{\CAPTHETA}{\CAPPSI}$ \\
\bottomrule
\end{tabular}
\end{center}
\caption{\Rule{$\TRIPLEBAR$-Exchange}}
\label{GSDplusDNF}%
\end{table}
\index{derivation!rule!for DNF}\index{DNF}
\noindent{}Recall from section \ref{Shortcut Rule Elimination Theorem Section} that all we need to do to show that anything that can be derived using \GQD{} and this rule can be derived using just \GQD{} is to prove the following:
\begin{THEOREM}{\LnpTC{GQD NDF Rule}}
Any two \GQL{} formulas got by substituting other \GQL{} formulas into the may-add and given schemas of \Rule{$\TRIPLEBAR$-Exchange} are provably equivalent; that is, $\sststile{}{}\forall\bpartriplebar{\partriplebar{\CAPTHETA}{\CAPPSI}}{\pardisjunction{\parconjunction{\CAPTHETA}{\CAPPSI}}{\parconjunction{\negation{\CAPTHETA}}{\negation{\CAPPSI}}}}$.
\end{THEOREM}
\begin{PROOF}
We show that $\sststile{}{}\forall\bpartriplebar{\partriplebar{\CAPTHETA}{\CAPPSI}}{\pardisjunction{\parconjunction{\CAPTHETA}{\CAPPSI}}{\parconjunction{\negation{\CAPTHETA}}{\negation{\CAPPSI}}}}$ by giving a derivation schema, which for any two formulas $\CAPTHETA$ and $\CAPPSI$ will result in the needed derivation. 
(Note that to save space $\integer{q}=\integer{n}+\integer{m}$.)
\begin{gproofnn}
\gaproof{
\galine{1}{$\partriplebar{\CAPTHETA}{\CAPPSI}\constant{c_{\integer{1}}}\ldots\constant{c_{\integer{\integer{m}}}}/\variable{x}_1\ldots\variable{x}_{\integer{m}}$}{\Rule{Assume}}
\galine{2}{$\partriplebar{\negation{\CAPTHETA}}{\negation{\CAPPSI}}\constant{c_{\integer{1}}}\ldots\constant{c_{\integer{\integer{m}}}}/\variable{x}_1\ldots\variable{x}_{\integer{m}}$}{\Rule{$\NEGATION$/$\TRIPLEBAR$-Intro}, 1}
\gaaproof{
\gaaline{3}{$\negation{\parconjunction{\CAPTHETA}{\CAPPSI}}\constant{c_{\integer{1}}}\ldots\constant{c_{\integer{\integer{m}}}}/\variable{x}_1\ldots\variable{x}_{\integer{m}}$}{\Rule{Assume}}
\gaaline{4}{$\pardisjunction{\negation{\CAPTHETA}}{\negation{\CAPPSI}}\constant{c_{\integer{1}}}\ldots\constant{c_{\integer{\integer{m}}}}/\variable{x}_1\ldots\variable{x}_{\integer{m}}$}{\Rule{DeM}, 3}
\gaaaproof{
\gaaaline{5}{$\negation{\CAPTHETA}\constant{c_{\integer{1}}}\ldots\constant{c_{\integer{\integer{m}}}}/\variable{x}_1\ldots\variable{x}_{\integer{m}}$}{\Rule{Assume}}
\gaaaline{6}{$\negation{\CAPPSI}\constant{c_{\integer{1}}}\ldots\constant{c_{\integer{\integer{m}}}}/\variable{x}_1\ldots\variable{x}_{\integer{m}}$}{\Rule{$\TRIPLEBAR$-Elim}, 2, 5}
\gaaaline{7}{$\parconjunction{\negation{\CAPTHETA}}{\negation{\CAPPSI}}\constant{c_{\integer{1}}}\ldots\constant{c_{\integer{\integer{m}}}}/\variable{x}_1\ldots\variable{x}_{\integer{m}}$}{\Rule{$\WEDGE\!$-Intro}, 5, 6}
}
\gaaline{8}{$\parhorseshoe{\negation{\CAPTHETA}}{\parconjunction{\negation{\CAPTHETA}}{\negation{\CAPPSI}}}\constant{c_{\integer{1}}}\ldots\constant{c_{\integer{\integer{m}}}}/\variable{x}_1\ldots\variable{x}_{\integer{m}}$}{\Rule{$\HORSESHOE$-Intro}, 5--7}

\gaaaproof{
\gaaaline{9}{$\negation{\CAPPSI}\constant{c_{\integer{1}}}\ldots\constant{c_{\integer{\integer{m}}}}/\variable{x}_1\ldots\variable{x}_{\integer{m}}$}{\Rule{Assume}}
\gaaaline{10}{$\negation{\CAPTHETA}\constant{c_{\integer{1}}}\ldots\constant{c_{\integer{\integer{m}}}}/\variable{x}_1\ldots\variable{x}_{\integer{m}}$}{\Rule{$\TRIPLEBAR$-Elim}, 2, 9}
\gaaaline{11}{$\parconjunction{\negation{\CAPTHETA}}{\negation{\CAPPSI}}\constant{c_{\integer{1}}}\ldots\constant{c_{\integer{\integer{m}}}}/\variable{x}_1\ldots\variable{x}_{\integer{m}}$}{\Rule{$\WEDGE\!$-Intro}, 9, 10}
}
\gaaline{12}{$\parhorseshoe{\negation{\CAPPSI}}{\parconjunction{\negation{\CAPTHETA}}{\negation{\CAPPSI}}}\constant{c_{\integer{1}}}\ldots\constant{c_{\integer{\integer{m}}}}/\variable{x}_1\ldots\variable{x}_{\integer{m}}$}{\Rule{$\HORSESHOE$-Intro}, 9--11}
\gaaline{13}{$\parconjunction{\negation{\CAPTHETA}}{\negation{\CAPPSI}}\constant{c_{\integer{1}}}\ldots\constant{c_{\integer{\integer{m}}}}/\variable{x}_1\ldots\variable{x}_{\integer{m}}$}{\Rule{$\VEE$-Intro}, 4, 8, 12}
}
\galine{14}{$\parhorseshoe{\negation{\parconjunction{\CAPTHETA}{\CAPPSI}}}{\parconjunction{\negation{\CAPTHETA}}{\negation{\CAPPSI}}}\constant{c_{\integer{1}}}\ldots\constant{c_{\integer{\integer{m}}}}/\variable{x}_1\ldots\variable{x}_{\integer{m}}$}{\Rule{$\HORSESHOE$-Intro}, 3--13}
\galine{15}{$\pardisjunction{\negation{\negation{\parconjunction{\CAPTHETA}{\CAPPSI}}}}{\parconjunction{\negation{\CAPTHETA}}{\negation{\CAPPSI}}}\constant{c_{\integer{1}}}\ldots\constant{c_{\integer{\integer{m}}}}/\variable{x}_1\ldots\variable{x}_{\integer{m}}$}{\Rule{$\HORSESHOE$/$\VEE$-Exch.}, 14}
\galine{16}{$\pardisjunction{\parconjunction{\CAPTHETA}{\CAPPSI}}{\parconjunction{\negation{\CAPTHETA}}{\negation{\CAPPSI}}}\constant{c_{\integer{1}}}\ldots\constant{c_{\integer{\integer{m}}}}/\variable{x}_1\ldots\variable{x}_{\integer{m}}$}{\Rule{$\NEGATION\NEGATION$-Elim}, 15}
}
\gline{17}{$[\partriplebar{\CAPTHETA}{\CAPPSI}\HORSESHOE$}{ }
\nopagebreak
\glinend{ }{$\qquad\pardisjunction{\parconjunction{\CAPTHETA}{\CAPPSI}}{\parconjunction{\negation{\CAPTHETA}}{\negation{\CAPPSI}}}]\constant{c_{\integer{1}}}\ldots\constant{c_{\integer{\integer{m}}}}/\variable{x}_1\ldots\variable{x}_{\integer{m}}$}{\Rule{$\HORSESHOE$-Intro}, 1--16}
\gaproof{
\galine{18}{$\pardisjunction{\parconjunction{\CAPTHETA}{\CAPPSI}}{\parconjunction{\negation{\CAPTHETA}}{\negation{\CAPPSI}}}\constant{c_{\integer{1}}}\ldots\constant{c_{\integer{\integer{m}}}}/\variable{x}_1\ldots\variable{x}_{\integer{m}}$}{\Rule{Assume}}
\galinend{ }{ }{ }
\galinend{ }{$\qquad\vdots$}{ }
\galinend{ }{ }{ }
\galine{$\integer{n}$}{$\partriplebar{\CAPTHETA}{\CAPPSI}\constant{c_{\integer{1}}}\ldots\constant{c_{\integer{\integer{m}}}}/\variable{x}_1\ldots\variable{x}_{\integer{m}}$}{ }
}
\gline{$\integer{n}+1$}{$[\pardisjunction{\parconjunction{\CAPTHETA}{\CAPPSI}}{\parconjunction{\negation{\CAPTHETA}}{\negation{\CAPPSI}}}\HORSESHOE$}{ }
\glinend{ }{$\qquad\partriplebar{\CAPTHETA}{\CAPPSI}]\constant{c_{\integer{1}}}\ldots\constant{c_{\integer{\integer{m}}}}/\variable{x}_1\ldots\variable{x}_{\integer{m}}$}{\Rule{$\HORSESHOE$-Intro}, 18--$\integer{n}$}
\gline{$\integer{n}+2$}{$[\partriplebar{\CAPTHETA}{\CAPPSI}\TRIPLEBAR$}{\Rule{$\TRIPLEBAR$-Intro}, 17,}
\glinend{ }{$\qquad\pardisjunction{\parconjunction{\CAPTHETA}{\CAPPSI}}{\parconjunction{\negation{\CAPTHETA}}{\negation{\CAPPSI}}}]\constant{c_{\integer{1}}}\ldots\constant{c_{\integer{\integer{m}}}}/\variable{x}_1\ldots\variable{x}_{\integer{m}}$}{$\integer{n}+1$}
\glinend{ }{ }{ }
\glinend{ }{$\qquad\vdots$}{ }
\glinend{ }{ }{ }
\gline{$\integer{q}+2$}{$\forall\bpartriplebar{\partriplebar{\CAPTHETA}{\CAPPSI}}{\pardisjunction{\parconjunction{\CAPTHETA}{\CAPPSI}}{\parconjunction{\negation{\CAPTHETA}}{\negation{\CAPPSI}}}}$}{\Rule{$\forall$-Intro}, $\integer{q}+1$}
\end{gproofnn}
\noindent{}Note that we have left steps $18$--$\integer{n}$ for the reader; 
this is just the derivation of the other conditional needed for \Rule{$\TRIPLEBAR$-Intro} on line $\integer{n}+2$. 
Also note that the last steps, lines $\integer{n}+3$ to the end, are all \Rule{$\forall$-Intro} meant to eliminate the constants $\constant{c_{\integer{1}}},\ldots,\constant{c_{\integer{\integer{m}}}}$.
\end{PROOF}
%Now we turn to the proof of the \GSD{} Completeness Lemma.
\begin{PROOFOF}{Thm. \ref{GSDCompletenessLemma}}
To prove the theorem, we shall describe an algorithm for applying the rules of \GSDP{} and \Rule{$\TRIPLEBAR$-Exchange} that takes a \GSL{} sentence $\CAPPHI$ and either halts in a derivation of $\conjunction{\Al}{\negation{\Al}}$, or halts with a sentence in \CAPS{dnf} for which there is some model $\IntA$ that makes $\CAPPHI$ true.
Since a sentence can be derived using the rules of \GSDP{} and \Rule{$\TRIPLEBAR$-Exchange} \Iff it can be derived using the basic rules of \GSD{}, this will be sufficient to prove the theorem. 

The algorithm begins with $\CAPPHI$ as an assumption on line 1. 
The algorithm then applies the method studied earlier in section \mvref{Disjunctive Normal Form} to produce a sentence $\CAPPHI'$ in \CAPS{dnf} that's \CAPS{tfe} to $\CAPPHI$.
We have to show that each step of the earlier method can be carried out in steps using the rules of \GSDP{} and \Rule{$\TRIPLEBAR$-Exchange}.
The earlier method proceeded in three stages. 
\begin{description}
\item[Step A:] \hfill
\begin{cenumerate}
\item If a subsentence of $\CAPPHI$ has $\HORSESHOE$ as its main connective, i.e. if $\CAPPHI=\horseshoe{\CAPTHETA}{\CAPPSI}$, replace the subsentence by $\disjunction{\negation{\CAPTHETA}}{\CAPPSI}$.
Repeat as necessary to obtain a sentence $\CAPPHI^*$ without conditionals. 
Each of these steps are sanctioned by \Rule{$\HORSESHOE$/$\VEE$-Exchange}.

\item If a subsentence of $\CAPPHI$ has $\TRIPLEBAR$ as its main connective, i.e. if $\CAPPHI=\triplebar{\CAPTHETA}{\CAPPSI}$, it is replaced with the subsentence $\disjunction{\parconjunction{\CAPTHETA}{\CAPPSI}}{\parconjunction{\negation{\CAPTHETA}}{\negation{\CAPPSI}}}$.
Repeat as necessary to obtain a sentence $\CAPPHI^{**}$ without biconditionals.
Each of these steps are sanctioned by \Rule{$\TRIPLEBAR$-Exchange}.
\end{cenumerate}

\item[Step B:]
In the case where $\CAPPHI^{**}$ contains a subsentence whose main connective is negation and which contains other connectives, we replace that subsentence by the following steps:
\begin{cenumerate}
\item Replace $\negation{\negation{\CAPTHETA}}$ by $\CAPTHETA$; this step is sanctioned by \Rule{$\NEGATION\NEGATION$-Elim}.
\item Replace $\negation{\parconjunction{\CAPTHETA}{\CAPPSI}}$ by $\disjunction{\negation{\CAPTHETA}}{\negation{\CAPPSI}}$; this step is sanctioned by \Rule{DeM}.
\item Replace $\negation{\pardisjunction{\CAPTHETA}{\CAPPSI}}$ by $\conjunction{\negation{\CAPTHETA}}{\negation{\CAPPSI}}$; this step is sanctioned by \Rule{DeM}.
\end{cenumerate}
Repeat as necessary to obtain a sentence $\CAPPHI^{***}$ in which negations govern nothing but sentence letters. 

\item[Step C:]
The only thing that could prevent $\CAPPHI^{***}$ from being in \CAPS{dnf} is that some conjunctions govern some disjunctions, i.e., there is a subsentence of the form $\conjunction{\CAPTHETA}{\pardisjunction{\CAPPSI_1}{\disjunction{\ldots}{\CAPPSI_{\integer{n}}}}}$, or the reverse $\conjunction{\pardisjunction{\CAPPSI_1}{\disjunction{\ldots}{\CAPPSI_{\integer{n}}}}}{\CAPTHETA}$.
Those subsentences can each be replaced by the equivalent sentence $\disjunction{\parconjunction{\CAPTHETA}{\CAPPSI_1}}{\disjunction{\ldots}{\parconjunction{\CAPTHETA}{\CAPPSI_{\integer{n}}}}}$ or $\disjunction{\parconjunction{\CAPPSI_1}{\CAPTHETA}}{\disjunction{\ldots}{\parconjunction{\CAPPSI_{\integer{n}}}{\CAPTHETA}}}$.
These steps are sanctioned by \Rule{Distribution}.
\end{description}
\noindent{}Applying the above steps A, B, and C will provide us a derivation starting with $\CAPPHI$ as an assumption (and no other assumptions) and ending with a \CAPS{dnf} sentence that's \CAPS{tfe} to $\CAPPHI$. 
We now have two possibilities:
\begin{description}
\item[Case 1:] 
Every disjunct contains a sentence letter and the negation of that sentence letter. 
That is, each disjunction has the form $\parconjunction{\CAPPSI_1}{\conjunction{\ldots}{\conjunction{\CAPPSI_{\integer{i}}}{\conjunction{\ldots}{\conjunction{\negation{\CAPPSI_{\integer{i}}}}{\conjunction{\ldots}{\CAPPSI_{\integer{n}}}}}}}}$; for example: $\parconjunction{\Al}{\conjunction{B}{\conjunction{\Cl}{\conjunction{\negation{\El}}{\conjunction{\negation{\Bl}}{\negation{\Kl}}}}}}$.

\item[Case 2:]
At least one disjunct contains no sentence letter such that the negation of the sentence letter is also in the disjunct. 
\end{description}
\noindent{}We can show in case $1$ that the original sentence leads to a contradiction.
First, we observe that any conjunction that contains a sentence letter and its negation leads to a contradiction by repeated steps of \Rule{$\WEDGE\!$-Elim}. 
Thus we can derive the negation of any such conjunction using \Rule{$\NEGATION$-Intro}.
So, if the last line of our derivation so far is of the form $\pardisjunction{\CAPPSI_1}{\disjunction{\ldots}{\CAPPSI_{\integer{n}}}}$ and each $\CAPPSI_{\integer{i}}$ contains a sentence letter and the negation of that sentence letter, then we can add to the derivation lines that establish the negation of each $\CAPPSI_{\integer{i}}$. 
Thus by $\integer{n}-1$ steps of \Rule{D.S.} we get a single $\CAPPSI_{\integer{i}}$ by itself with only the first line as an assumption.
Since by hypothesis in this case $\CAPPSI_{\integer{i}}$ leads to a contradiction, we can show the initial assumption leads to a contradiction.  %Here is an example of case $1$, where $\CAPPHI=$


%\begin{gproofnn}
%	\gaproof{%
%		\galine{1}{$\conjunction{\Dp{\text{a}}}{\universal{\variable{z}}\Hp{\variable{z}}}$}{A}%
%		\galine{2}{$\Dp{\text{a}}$}{$\WEDGE$-Elim 1}%
%		\galine{3}{$\universal{\variable{z}}\Hp{\variable{z}}$}{$\WEDGE$-Elim 1}%
%		\galine{4}{$\Hp{\text{a}}$}{$\forall$-Elim 3}%
%		\galine{5}{$\conjunction{\Dp{\text{a}}}{\Hp{\text{a}}}$}{$\WEDGE$-Intro 2,4}%
%		\galine{6}{$\existential{\variable{y}}\parconjunction{\Dp{\variable{y}}}{\Hp{\variable{y}}}$}{$\exists$-Intro 5}%
%	}%
%	\gline{7}{$\horseshoe{\parconjunction{\Dp{\text{a}}}{\universal{\variable{z}}\Hp{\variable{z}}}}{\existential{\variable{y}}\parconjunction{\Dp{\variable{y}}}{\Hp{\variable{y}}}}$}{$\HORSESHOE$-Intro 1--6}%
%	\gline{8}{$\universal{\variable{x}}\bparhorseshoe{\parconjunction{\Dp{\variable{x}}}{\universal{\variable{z}}\Hp{\variable{z}}}}{\existential{\variable{y}}\parconjunction{\Dp{\variable{y}}}{\Hp{\variable{y}}}}$}{$\forall$-Intro 7 (a)}%
%\end{gproofnn}















Generally, the procedure will look like this:

\begin{gproofnn}
\glinend{ }{$\CAPPHI$}{\Rule{Assume}} %\marginnote{\scriptsize{}The original sentence}[0cm]
\glinend{ }{ }{ }
\glinend{ }{$\qquad\vdots$}{ }
\glinend{ }{ }{ }
\glinend{ }{$\disjunction{\CAPPSI_1}{\disjunction{\ldots}{\CAPPSI_{\integer{n}}}}$}{ } %\marginnote{\scriptsize{}The \CAPS{dnf} sentence after steps A, B, and C}[0cm]
\gaproof{
\galinend{ }{$\CAPPSI_1$}{\Rule{Assume}}
\galinend{ }{ }{ }
\galinend{ }{$\qquad\vdots$}{ }
\galinend{ }{ }{ }
\galinend{ }{$\conjunction{\CAPTHETA_1}{\negation{\CAPTHETA_1}}$}{ }
}
\glinend{ }{$\horseshoe{\CAPPSI_1}{\parconjunction{\CAPTHETA_1}{\negation{\CAPTHETA_1}}}$}{\Rule{$\HORSESHOE$-Intro}} %\marginnote{\scriptsize{}We start deriving the negation of each disjunct}[0cm]
\glinend{ }{$\negation{\CAPPSI_1}$}{\Rule{$\NEGATION$-Intro}}
\glinend{ }{ }{ }
\glinend{ }{$\qquad\vdots$}{ }
\glinend{ }{ }{ }
\gaproof{
\galinend{ }{$\CAPPSI_{\integer{n}}$}{\Rule{Assume}}
\galinend{ }{ }{ }
\galinend{ }{$\qquad\vdots$}{ }
\galinend{ }{ }{ }
\galinend{ }{$\conjunction{\CAPTHETA_{\integer{n}}}{\negation{\CAPTHETA_{\integer{n}}}}$}{ }
}
\glinend{ }{$\horseshoe{\CAPPSI_{\integer{n}}}{\parconjunction{\CAPTHETA_{\integer{n}}}{\negation{\CAPTHETA_{\integer{n}}}}}$}{\Rule{$\HORSESHOE$-Intro}}
\glinend{ }{$\negation{\CAPPSI_{\integer{n}}}$}{\Rule{$\NEGATION$-Intro}}
\glinend{ }{$\disjunction{\CAPPSI_1}{\disjunction{\ldots}{\CAPPSI_{\integer{n}-1}}}$}{\Rule{D.S.}} %\marginnote{\scriptsize{}Start applying \Rule{D.S.}}[0cm]
\glinend{ }{$\disjunction{\CAPPSI_1}{\disjunction{\ldots}{\CAPPSI_{\integer{n}-2}}}$}{\Rule{D.S.}}
\glinend{ }{$\disjunction{\CAPPSI_1}{\disjunction{\ldots}{\CAPPSI_{\integer{n}-3}}}$}{\Rule{D.S.}}
\glinend{ }{ }{ }
\glinend{ }{$\qquad\vdots$}{ }
\glinend{ }{ }{ }
\glinend{ }{$\disjunction{\CAPPSI_1}{\CAPPSI_2}$}{\Rule{D.S.}}
\glinend{ }{$\CAPPSI_1$}{\Rule{D.S.}}
\glinend{ }{$\negation{\CAPPSI_1}$}{\Rule{Rep.}}
\glinend{ }{$\conjunction{\Al}{\negation{\Al}}$}{\Rule{A.C.}} %\marginnote{\scriptsize{}Finally we reach a contradiction}[0cm]
\end{gproofnn}
We can show in case $2$ that we can find a model that makes all sentences in the derivation true, starting with the last. 
We choose the disjunction that does not contain a sentence letter and its negation (if there is more than one, it doesn't matter which we choose), and we construct a model $\IntA$ by assigning $\TrueB$ to each sentence letter that occurs positively (without a negation in front) in the conjunction and $\FalseB$ to each sentence letter that occurs negatively (with a negation in front).
We can do this since none occur in both modes. 

This model makes each element of the conjunction true and thus makes the entire conjunction true. 
Since the sentence containing it is a disjunction, this is sufficient to make the entire sentence true.
Thus we can make the last line of the derivation true.
Observe now that all of the steps we used in the derivation were replacement of provably equivalence sentences;
that is, they used exchange shortcut rules.
Thus, we know that we could also construct a derivation by \mention{turning this proof upside down}, so to speak.  In other words, we could construct a new derivation, with the last step of the original derivation as the initial assumption step.  Then we could use the exchange shortcut rules to work back to $\CAPPHI$ of the original derivation.

Thus, by soundness, we know that if the first sentence of the \mention{upside-down derivation} (the sentence in \CAPS{dnf} that was at the bottom) is true in a model, then so is everything that can be derived from it, including our original sentence $\CAPPHI$ that is now at the end of the inverted derivation. 
Therefore, $\CAPPHI$ is true in some model. 
\end{PROOFOF}
\begin{THEOREM}{\LnpTC{GSDWCompleteness} Weak \GSD{} Completeness Theorem:}
For all \GSL{} sentences $\CAPPHI$: if $\sdtstile{}{}\CAPPHI$, then $\sststile{}{}\CAPPHI$ in \GSD{}.
\end{THEOREM}
\begin{PROOF}
We apply the method above from the \GSD{} Completeness Lemma to the negation of $\CAPPHI$. 
This either produces a derivation of a contradiction from $\negation{\CAPPHI}$, in which case we can prove $\CAPPHI$ by adding two more steps justified by \Rule{$\HORSESHOE$-Intro} and \Rule{$\NEGATION$-Elim}, or it produces a model that makes $\negation{\CAPPHI}$ true and that therefore makes $\CAPPHI$ false. So, $\CAPPHI$ is either false in some model or is derivable in \GSD{}. 
\end{PROOF}
\noindent{}Finally, as a corollary we get:
\begin{THEOREM}{\LnpTC{GSDCompleteness} \GSD{} Completeness Theorem:}
For every finite set $\Delta$ of sentences of \GSL{} and every sentence $\CAPPHI$ of \GSL{}, if $\Delta\sdtstile{}{}\CAPPHI$, then $\Delta\sststile{}{}\CAPPHI$ in \GSD{}.
\end{THEOREM}
\begin{PROOF}
This follows immediately from the Weak \GSD{} Completeness Theorem and theorem \mvref{RegWeakCompletenessEquiv}.
\end{PROOF}

%%%%%%%%%%%%%%%%%%%%%%%%%%%%%%%%%%%%%%%%%%%%%%%%%%
\section{Completeness of \GQD{}}\label{Sec:Completeness of GQD}
%%%%%%%%%%%%%%%%%%%%%%%%%%%%%%%%%%%%%%%%%%%%%%%%%%

In this section we shall prove that \GQD{} is complete.  We would like to use the same kind of strategy for \GQD{} we did for \GSD{}, so we have deal with the quantifiers.
Unfortunately, the quantifiers prevent us from proving the analogue of DNF for \GQL{}. For example:  $\universal{\variable{x}}\pardisjunction{\Hp{\variable{x}}}{\Gp{\variable{x}}}$ is not equivalent to $\disjunction{\universal{\variable{x}}\Hp{\variable{x}}}{\universal{\variable{x}}\Gp{\variable{x}}}$.

To get around problems like this, we must show we can move all of the quantifiers to the front of the sentence.  This has the advantage of separating the quantifier parts of the logical structure from the SL parts.


\subsection{Prenex Definition and Steps}\label{Prenex Definition and Steps}
\begin{majorILnc}{\LnpDC{PrenexNF}}
A sentence $\CAPPHI$ of \GQL{} is in \df{prenex normal form} \Iff every quantifier is in initial position, or, in other words, the scope of all quantifiers is greater than that of any non-quantifier connective.
\end{majorILnc}
\begin{THEOREM}{\LnpTC{PrenexNFTheorem} Prenex Normal Form Theorem:}
For all sentence $\CAPTHETA$ of \GQL{}, there is a provably equivalent sentence $\CAPTHETA^*$ in prenex normal form; that is, $\CAPTHETA^*$ is in prenex normal form and $\sststile{}{}\triplebar{\CAPTHETA}{\CAPTHETA^*}$ in \GQD{}.
\end{THEOREM}
\begin{PROOF}
As with \CAPS{dnf} we have a set of steps for turning sentence $\CAPTHETA$ into a sentence $\CAPTHETA^*$ in prenex normal form. 
First we give the steps, and then show that each step can be sanctioned either by \Rule{QN}, \Rule{$\TRIPLEBAR$-Exchange}, or an exchange rule that can be introduced.
(We'll call these new exchange rules the \niidf{Prenex Exchange Rules}.\index{Exchange Rules!Prenex}) 
Because all the steps in the process are justified by exchange rules, we can either read the resulting series of steps top-down as a derivation of $\CAPTHETA^*$ from $\CAPTHETA$, or bottom-up as a derivation of $\CAPTHETA$ from $\CAPTHETA^*$. 
So, we'll have shown that $\sststile{}{}\triplebar{\CAPTHETA}{\CAPTHETA^*}$ in the derivation system consisting of \Rule{$\TRIPLEBAR$-Exchange} and the Prenex Exchange Rules.
But, as with all the other exchange rules anything that can be derived using the Prenex Exchange Rules can be derived in \GQD{} alone;
so, this will be sufficient to show that $\sststile{}{}\triplebar{\CAPTHETA}{\CAPTHETA^*}$ in \GQD{}. First, the steps are:
\begin{cenumerate}
\item Replace biconditionals with disjunctions of conjunctions; i.e. replace $\triplebar{\CAPPHI}{\CAPPSI}$ with $\disjunction{\parconjunction{\CAPPHI}{\CAPPSI}}{\parconjunction{\negation{\CAPPHI}}{\negation{\CAPPSI}}}$.
\item Rewrite any variables that occur bound by more than one quantifier.
\item Move the first quantifier not in prenex position one step towards the front by the following principles. Repeat this step as often as necessary.  Keep in mind that you can't move forward a quantifier that binds the variable \mention{$\variable{x}$} if it will now have within its scope a new subformula that has a free \mention{$\variable{x}$}.  But we have prevented that problem by eliminating potentially clashing variables in Step 2.
\begin{longtable}[c]{ l l }
\toprule
\textbf{Replace} & \textbf{by} \\
\midrule
$\parconjunction{(\#\variable{x})\CAPTHETA}{\CAPPSI}$ & $(\#\variable{x})\parconjunction{\CAPTHETA}{\CAPPSI}$ \\
$\parconjunction{\CAPTHETA}{(\#\variable{x})\CAPPSI}$ & $(\#\variable{x})\parconjunction{\CAPTHETA}{\CAPPSI}$ \\

$\pardisjunction{(\#\variable{x})\CAPTHETA}{\CAPPSI}$ & $(\#\variable{x})\pardisjunction{\CAPTHETA}{\CAPPSI}$ \\
$\pardisjunction{\CAPTHETA}{(\#\variable{x})\CAPPSI}$ & $(\#\variable{x})\pardisjunction{\CAPTHETA}{\CAPPSI}$ \\

$\parhorseshoe{\CAPTHETA}{(\#\variable{x})\CAPPSI}$ & $(\#\variable{x})\parhorseshoe{\CAPTHETA}{\CAPPSI}$ \\

$\parhorseshoe{\existential{\variable{x}}\CAPTHETA}{\CAPPSI}$ & $\universal{\variable{x}}\parhorseshoe{\CAPTHETA}{\CAPPSI}$ \\
$\parhorseshoe{\universal{\variable{x}}\CAPTHETA}{\CAPPSI}$ & $\existential{\variable{x}}\parhorseshoe{\CAPTHETA}{\CAPPSI}$ \\

$\negation{\existential{\variable{x}}\CAPTHETA}$ & $\universal{\variable{x}}\negation{\CAPTHETA}$ \\
$\negation{\universal{\variable{x}}\CAPTHETA}$ & $\existential{\variable{x}}\negation{\CAPTHETA}$ \\
\bottomrule
\end{longtable}
Note that $(\#\variable{x})$ is just a dummy quantifier standing for either; 
replacement is the same for both quantifiers.  Also, we use \mention{$\variable{x}$} in the chart above, but the same principles hold for quantifiers with any other variable.
\end{cenumerate}
After applying these steps to a sentence $\CAPTHETA$ we will get a sentence $\CAPTHETA^*$ that is in prenex normal form.\footnote{For more discussion of Prenex Form, see \citealt[132]{Kleene1967}, \citealt[54]{Hodges2001}, \citeyear[30]{Hodges2001b}.} 
We now have to show that each step can be sanctioned by an exchange rule.
Step (1) is straightforward, since obviously it will be sanctioned by \Rule{$\TRIPLEBAR$-Exchange}.
But steps (2) and (3) we need new rules (although the replacements involving negations in (3) can be handled with \Rule{QN}).
The most straightforward strategy is to read the needed exchange rules right off the steps. 
Thus, the Prenex Exchange Rules are given in the following chart.
%\begin{table}[!ht]
%\renewcommand{\arraystretch}{1.5}
%\begin{center}
%\begin{tabular}{ p{1in} l l } %p{2.2in} p{2in}
%\toprule
%\textbf{Name} & \textbf{Given} & \textbf{May Add} \\ 
%\midrule
\renewcommand{\arraystretch}{1.5}
\begin{longtable}[c]{ p{1in} l l } %p{2.2in} p{2in}
\toprule
\textbf{Name} & \textbf{Given} & \textbf{May Add} \\ 
\midrule
\endfirsthead
\multicolumn{3}{c}{\emph{Continued from Previous Page}}\\
\toprule
\textbf{Name} & \textbf{Given} & \textbf{May Add} \\ 
\midrule
\endhead
\bottomrule
\caption{Prenex Exchange Short-Cut Rules for \GQD{}}\\[-.15in]
\multicolumn{3}{c}{\emph{Continued next Page}}\\
\endfoot
\bottomrule
\caption{Prenex Exchange Shortcut Rules for \GQD{}}\\
\endlastfoot
\label{GSDplusPrenex}\Rule{$\ALPHA$/$\BETA$-Exch} & $(\#\ALPHA)\CAPPHI$ & $(\#\BETA)\CAPPHI\BETA/\ALPHA$ \\
\Rule{Q Shuffling} & $\parconjunction{(\#\variable{x})\CAPTHETA}{\CAPPSI}$ & $(\#\variable{x})\parconjunction{\CAPTHETA}{\CAPPSI}$ \\
& $\parconjunction{\CAPTHETA}{(\#\variable{x})\CAPPSI}$ & $(\#\variable{x})\parconjunction{\CAPTHETA}{\CAPPSI}$ \\

& $\pardisjunction{(\#\variable{x})\CAPTHETA}{\CAPPSI}$ & $(\#\variable{x})\pardisjunction{\CAPTHETA}{\CAPPSI}$ \\
& $\pardisjunction{\CAPTHETA}{(\#\variable{x})\CAPPSI}$ & $(\#\variable{x})\pardisjunction{\CAPTHETA}{\CAPPSI}$ \\

& $\parhorseshoe{\CAPTHETA}{(\#\variable{x})\CAPPSI}$ & $(\#\variable{x})\parhorseshoe{\CAPTHETA}{\CAPPSI}$ \\

& $\parhorseshoe{\existential{\variable{x}}\CAPTHETA}{\CAPPSI}$ & $\universal{\variable{x}}\parhorseshoe{\CAPTHETA}{\CAPPSI}$ \\
& $\parhorseshoe{\universal{\variable{x}}\CAPTHETA}{\CAPPSI}$ & $\existential{\variable{x}}\parhorseshoe{\CAPTHETA}{\CAPPSI}$ \\
\end{longtable}
%\bottomrule
%\end{tabular}
%\end{center}
%\caption{Exchange Short-Cut Rules for \GSD{} (\GSD{})}
%\label{GSDplus2}
%\end{table}
\noindent{}Now all that's left to show is that anything that can be derived using the Prenex Exchange Rules can be derived using the basic rules of \GQD{} alone.
Recall from section \ref{Shortcut Rule Elimination Theorem Section} that all we need to do to show this is to prove the following:
\begin{THEOREM}{\LnpTC{GQD NDF Rule2}}
For all Prenex Exchange Rules \Rule{R}, any two \GQL{} formulas got by substituting other \GQL{} formulas into the may-add and given schemas of \Rule{R} are provably equivalent.
\end{THEOREM}
\noindent{}We leave the proof of this theorem to the reader, since as with the other exchange rules it just involves writing down the appropriate derivation schemas. 
\end{PROOF}

\subsection{The Strategy for Proving \GQD{} Completeness}
Our goal is to prove the strong completeness of \GQD{}; 
that is, we want to prove that for any set $\Delta$ of \GQL{} sentences and \GQL{} sentence $\CAPPHI$, if $\Delta\sdtstile{}{}\CAPPHI$, then $\Delta\sststile{}{}\CAPPHI$. 
Our strategy will be to first prove the completeness of \GQD{} and then show how to modify the method to prove the strong completeness.
Our strategy for proving  completeness will be to show that for any sentence we can either (1) find a derivation of it or (2) prove that there is a model that makes it false. 
(This part of the strategy is more or less the same as what we did to show that \GSD{} is  complete.)
In other words, $\CAPPHI$ is either derivable or not quantificationally true, from which it immediately follows that if $\CAPPHI$ is quantificationally true, then it is derivable. 

The method,\index{method, the} in brief, is to negate the sentence $\CAPPHI$ and begin a derivation.
Then we transform the negation of the sentence into prenex normal form, using the steps outlined in section \ref{Prenex Definition and Steps}. 
Next we transform the inner part of the sentence (remember the quantifiers are all up front) into \CAPS{dnf} form, using our standard method for that (see Sec. \pmvref{Disjunctive Normal Form}).
This will not introduce any new assumptions. 
We then systematically take instances of the bound variables and try to derive a contradiction.
If we can derive a contradiction we can then (assuming all goes well) use \Rule{$\NEGATION$-Elim} to obtain a derivation of $\CAPPHI$.

We must be very systematic since we have to be sure that if we get a contradiction we can derive it from the initial sentence $\CAPPHI$, and that if we do not get a contradiction we have not overlooked anything and that we can show the existence of a model making the sentence on the first line, $\CAPPHI$, true. 

It is important that we know the form of the sentence we have reached and are able to prescribe a uniform systematic method.
The sentence has been highly standardized; 
there are no biconditionals or conditionals (these have been eliminated in early steps of the transformation), negations govern only atomic sentences, and conjunctions govern only atomic sentences or their negations. 
These last, atomic sentence and their negations, are called \idf{ions}. 
We say an ion occurs \niidf{positively} \Iff it's an atomic sentence without a negation, and it occurs \niidf{negatively} \Iff it's a negated atomic sentence. 
We will call the quantifier free part of the original sentence the \idf{matrix}. 
It is usually not a sentence since it may have free variables.
We will call the sentences that are obtained from the matrix by substitution in the process of constructing the derivation \niidf{matrix instances}\index{matrix!instances|textbf}. 
To put some of our jargon together, the matrix of the sentence will consist of disjunctions of conjunctions of ions. 

\subsection{The Method and Completeness Lemmas}\label{The Method Section}
In this section we describe the method sketched above.\index{method, the} 
Given a sentence $\CAPTHETA$, the Method either produces a derivation of $\CAPTHETA$ or indicates a model that makes it false:

\begin{description}
\item[Step 0:] Write $\negation{\CAPTHETA}$ on line 1 as an assumption.
Then first apply the prenex steps to put $\negation{\CAPTHETA}$ in Prenex Normal Form (\CAPS{pnf}). 
Next, apply the disjunctive normal form steps to the inner, quantifier-free part of the sentence until it's in \CAPS{dnf}. 
At this point we'll have a sentence $(\negation{\CAPTHETA})^*$ that's in what we'll call \idf{prenex disjunctive normal form}\index{disjunctive normal form!prenex} (\CAPS{pdnf}).  

\item[Step 1:] We continue the derivation operating on $(\negation{\CAPTHETA})^*$, the \CAPS{pdnf} of the sentence we're concerned with. 
If this \CAPS{pdnf} is a universal statement and contains no constants we write as the next line the instance of it we obtain by eliminating the quantifier and substituting the constant $\constant{a}$ for the previously bound variable;
these steps are sanctioned by \Rule{$\forall$-Elim}.
This step is only done once, whereas the next three steps generally require repeated recursive applications. 

\item[Step 2:] For every universal sentence that appears thus far in the derivation, we add \emph{all new instances} that can be formed with constants that occur earlier in the derivation;
these steps are sanctioned by \Rule{$\forall$-Elim}.
E.g., if $\universal{\variable{x}}\CAPPHI$ appears on a line and the constant $\constant{c}$ appears anywhere (earlier) in the derivation, then if we have not taken an instance of $\CAPPHI$ with $\constant{c}$ yet (i.e., $\CAPPHI\constant{c}/\variable{x}$), we do so.
As a practical matter, this means that it is useful in following the method to keep track somewhere of the constants used at each stage and of which constants have been used to instantiate which universal statements.
Note that in this step we are taking new instances with old constants and that we are not adding any new assumptions. 
We may, however, be adding new existentials. 

\item[Step 3:] For every existential sentence that appears in the derivation for which no instance has been added yet, add an instance using the first constant which \emph{does not occur in any previous assumption}. 
Note that \mention{instance} is to be taken very strictly here. 
The fact that we instantiated $\existential{\variable{x}}\Kpp{\variable{x}}{\constant{a}}$ with $\Kpp{\constant{b}}{\constant{a}}$ takes care of that existential, but if we later add the sentence $\existential{\variable{x}}\Kpp{\variable{x}}{\constant{b}}$ then we must add an instance of it. 
The rule which sanctions these steps will be \Rule{Assumption}. 
We will eventually discharge these premises by \Rule{$\HORSESHOE$-Elim} and \Rule{$\exists$-Elim} if we get to a contradiction.
It is in anticipation of this eventuality that we carefully chose a constant which does not occur in any previous assumption.
Note that with this step we are adding new instances with new constants in new assumptions. 

\item[Step 4:] Determine whether the conjunction of the \emph{instances} of the matrix in the derivation thus far are contradictory. 
Officially, the way to do this is to take the conjunction of them all by \Rule{$\WEDGE\!$-Intro}, use \Rule{Distribution} to get the conjunction into \CAPS{dnf} and check whether every disjunct contains a contradiction. 
If so, then by a process of \Rule{$\VEE$-Elim} and \Rule{Any Contradiction} we can eventually produce the line $\conjunction{\Al}{\negation{\Al}}$. 

This step can be cumbersome in practice. We will give some unofficial short cuts to make this more manageable soon.

\item[Step 5:] \hfill
\begin{cenumerate}
\item If the matrices are contradictory (see step 4) we stop.
\item Or if the conjunction of the matrix instances is consistent and the last applications of Steps 2 and 3 produce no new sentences, we stop.
\item Or if the conjunction of the matrix instance is consistent and the last applications of Steps 2 and 3 produced new sentences, then we return to Step 2 and reapply those steps.
\end{cenumerate}
\end{description}
There are three possible outcomes of applying this method to a sentence:
\begin{cenumerate}
\item The method might reach a contradiction.
\item The method might stop without a contradiction.
\item The method might generate new sentences perpetually without contradiction.
\end{cenumerate}
We will show first that if a contradiction is reached we can construct a derivation of $\CAPTHETA$. 

\begin{THEOREM}{\LnpTC{Derivational Lemma} Derivational Lemma:}
If the Method starts with $\negation{\CAPTHETA}$ and produces a contradiction, then there is a derivation of $\CAPTHETA$.
\end{THEOREM}
\begin{PROOF}
Step 3 left us with $\conjunction{\Al}{\negation{\Al}}$ on a line with its assumptions being those of the matrices. 
We want to shift those assumptions so that we end up with the contradiction from the first assumption, $\negation{\CAPTHETA}$, alone.
We know by considering our method that other assumptions entered only by Step 2, where we added instances of existentials using new constants. 
We eliminate the last assumption by a \Rule{$\HORSESHOE$-Intro}. 
We know that this eliminated assumption introduced a \emph{new} constant from an existential. 
Therefore we know that this constant did not appear in any earlier assumption or in the existential of which we are taking an instance.
It also (obviously) does not occur in $\conjunction{\Al}{\negation{\Al}}$.
Thus we are allowed to use the rule \Rule{$\exists$-Elim} on the existential claim from which we assumed that instance. 

So we derive the contradiction $\conjunction{\Al}{\negation{\Al}}$ again, by \Rule{$\exists$-Elim}. 
We continue this process, repeating $\conjunction{\Al}{\negation{\Al}}$ as often as necessary to shift the dependence back to the assumption on line 1. 
This gives us a derivation of $\conjunction{\Al}{\negation{\Al}}$ from the first assumption, $\negation{\CAPTHETA}$, only. 

We then add two more lines: $\horseshoe{\negation{\CAPTHETA}}{\parconjunction{\Al}{\negation{\Al}}}$, sanctioned by \Rule{$\HORSESHOE$-Intro}, and $\CAPTHETA$, sanctioned by \Rule{$\NEGATION$-Elim}. 
Thus we have a derivation of $\CAPTHETA$ from no assumptions. 
\end{PROOF}

We have shown that if we obtain a contradiction in the derivation process we can derive the original sentence that interests us.


We must now show that if we do not obtain a contradiction (whether or not the method stops), then there is a model that makes $\negation{\CAPTHETA}$ true (and hence makes $\CAPTHETA$ false).

Before giving the rigorous version of the construction of the model, we will present some of the ideas in a more concrete context. 
If we consider a sentence such as $\disjunction{\parconjunction{\Kp{\constant{a}}}{\negation{\Gp{\constant{b}}}}}{\parconjunction{\negation{\Kp{\constant{b}}}}{\Hp{\constant{c}}}}$ we can observe several things. 
First, each disjunct is satisfiable \Iff no ion occurs both positively and negatively in it. 
It is obvious that a conjunction that includes a sentence and its negation cannot be satisfied, but we can show for a conjunction of ions that that is the only way in which it can fail to be satisfiable. 
For example, we can make $\Kp{\constant{a}}$ and $\negation{\Gp{\constant{b}}}$ true by letting $\KK$ be interpreted as the set of even numbers, $\GG$ the set of numbers divisible by $10$ and letting \mention{$\constant{a}$} be assigned $2$ and \mention{$\constant{b}$} be assigned $7$. 

Of course several such sentences taken together produce different results. E.g., as we saw above $\disjunction{\parconjunction{\Kp{\constant{a}}}{\conjunction{\negation{\Kp{\constant{b}}}}{\negation{\Gp{\constant{c}}}}}}{\parconjunction{\Gp{\constant{b}}}{\negation{\Gp{\constant{c}}}}}$ is satisfiable, as is $\disjunction{\parconjunction{\Kp{\constant{b}}}{\conjunction{\negation{\Kp{\constant{c}}}}{\negation{\Gp{\constant{b}}}}}}{\parconjunction{\Gp{\constant{c}}}{\negation{\Gp{\constant{d}}}}}$, but the two together (taken as a conjunction) are not.
The reason is that while each disjunct of the first sentence is self-consistent, it cannot be true simultaneously with either of the disjuncts of the second sentence. If we have a series of disjunctions then they are simultaneously satisfiable only if we can find a way of picking a disjunct from each one in such a way that all the chosen disjuncts can be true together.

This is relevant to the task at hand because we know that all of the non-quantified sentences in our derivation are in \CAPS{dnf} and are thus disjunctions of conjunctions of ions.
We are calling the quantifier free part of the original sentence the matrix.
It is usually not a sentence since it may have free variables. 
The sentences that are obtained from the matrix by substitution in the process of constructing the derivation are the matrix instances. 
We will use the notation $M_{i,j}$ for the disjuncts of the matrix instances, specifically the disjuncts of the first matrix instance will be $M_{1,1},M_{1,2},\ldots,M_{1,m}$.
Thus the first matrix instance is $\disjunction{M_{1,1}}{\disjunction{M_{1,2}}{\disjunction{\ldots}{M_{1,m}}}}$.
The matrix instances that appear in the derivation can be listed in an array:
\begin{center}
\begin{tabular}{ c }
$\disjunction{M_{1,1}}{\disjunction{M_{1,2}}{\disjunction{\ldots}{M_{1,m}}}}$ \\
$\disjunction{M_{2,1}}{\disjunction{M_{2,2}}{\disjunction{\ldots}{M_{2,m}}}}$ \\
\\
\hspace{.5in} $\vdots$ \\
\\
$\disjunction{M_{n,1}}{\disjunction{M_{n,2}}{\disjunction{\ldots}{M_{n,m}}}}$ \\
\end{tabular}
\end{center}
Note that if the the method never stops, then this array will be infinitely long. 

The matrices are jointly consistent \Iff there is a way of picking an $M_{i,j}$ from each matrix instance so that the conjunction of those $M_{i,j}$ contains no atomic sentence and its negation. 
In one direction this is easy to see: if there is no way of choosing a disjunct from each matrix instance that does not end up with an atomic sentence and its negation among the chosen sentences then the set of instances is inconsistent. 

To show that all instances are satisfiable when such a selection can be made without choosing a sentence and its negation will take some proving.
In order to do this we will need to define the \idf{master matrix list} $M$.\index{matrix!master list} 
We will first choose (if there is more than one) a set of disjuncts $M_{i,j}$ (including one from each matrix instance $M_i$) that does not contain any atomic sentence and its negation.
This will be a set of conjunctions of atomic sentences and negations of atomic sentences.
Our master matrix list $M$ simply consists of all these atomic sentences and negated atomic sentences.
Note that since the $M_{i,j}$ selections must be consistent no atomic sentence that appears unnegated also appears negated.
\begin{majorILnc}{\LnpDC{MatrixModel}}
Given a master matrix list $M$, the \nidf{matrix model of $M$}\index{matrix!model} is the model $\IntA_M$ such that:
\begin{cenumerate}
\item The universe of $\IntA_M$ contains one natural number for each constant that appears in $M$, and $\IntA_M(\constant{a})=1$, $\IntA_M(\constant{b})=2$, $\IntA_M(\constant{c})=3$, $\IntA_M(\constant{d})=4$, and so on; $\IntA_M(\variable{t})=1$ for any constant $\variable{t}$ that doesn't appear in $M$. 
\item For each $\integer{m}$-place predicate $\PP$, $\IntA_M(\PP)$ is the set of $\integer{m}$-tuples of natural numbers $\langle\integer{n}_1,\ldots,\integer{n}_\integer{m}\rangle$ such that $\IntA_M(\variable{t}_1)=\integer{n}_1,\ldots,\IntA_M(\variable{t}_\integer{m})=\integer{n}_\integer{m}$ and $\Pp{\variable{t}_1\ldots\variable{t}_\integer{m}}$ appears on the list $M$.
\item Assignments are only made if justified by these principles.
\end{cenumerate}
\end{majorILnc}
A bit more informally, we list the constants that occur on the master matrix list. $\IntA_M$ has a universe that contains as many natural numbers as constants used.
We assign to each constant that occurs on the list the natural number that indicates its place in the order, i.e. $1$ to $\constant{a}$, $2$ to $\constant{b}$, and so on.
%Any constants not occurring on the master matrix list $M$ will be assigned $1$. 
Note that this produces a \idf{census}.
Each $1$-place predicate is assigned the set of numbers associated with the constants such that an instance of the predicate followed by that constant appears on the master matrix list $M$. 
Each $2$-place predicate is assigned the set of pairs of numbers associated with constants such that an instance of the predicate followed by that pair of constants appears on the master matrix list $M$. 
E.g., if $\Kpp{\constant{a}}{\constant{b}}$, $\Kpp{\constant{b}}{\constant{c}}$, and $\Kpp{\constant{d}}{\constant{e}}$ appear on $M$, then $\IntA_M(\KK)$ is assigned $\{\langle1,2\rangle,\langle2,3\rangle,\langle4,5\rangle\}$.
Assignments are made in a similar fashion for $\integer{n}$-placed predicates for $\integer{n}>2$.\footnote{Note 
that if the method never stops, then the master matrix list $M$ will be infinite and we won't actually be able to write down the matrix model $\IntA_M$. 
But this isn't a problem, the matrix model $\IntA_M$ still exists, even if we can't write it down.} 
\begin{THEOREM}{\LnpTC{MethodLemmaA} The Method Lemma 1:}
The matrix model $\IntA_M$ makes true all sentences on the master matrix list $M$.
\end{THEOREM}
\begin{PROOF}
By construction, if an atomic sentence appears on the list we decided to put the relevant pair, triple, or whatever, of numbers in the set assigned to the predicate letter. 
For each negated atomic sentence on the list we know that we would not put the relevant pair, triple, or whatever, in the model of the predicate letter unless the atomic sentence which is being negated also appeared. 
But that never happened because $M$ is consistent by hypothesis.
\end{PROOF}
\begin{THEOREM}{\LnpTC{MethodLemmaB} The Method Lemma 2:}
All matrix instances in the derivation are true in the matrix model $\IntA_M$.
\end{THEOREM}
\begin{PROOF}
By Lemma 1 (Thm. \ref{MethodLemmaA}), all sentences on the master matrix list $M$ are true, and we included all the conjuncts of at least one disjunct $M_{i,j}$ from each matrix instance in forming the master list.
\end{PROOF}
\begin{THEOREM}{\LnpTC{MethodLemmaC} The Method Lemma 3:}
All quantified sentences in the derivation are true in the matrix model $\IntA_M$.
\end{THEOREM}

Informally, every universal has been instantiated with all the relevant constants, and every existential by at least one. 
Since $\IntA_M$ is a census, it follows that all the sentences in the derivation are true.

More rigorously...

\begin{PROOF}
We prove this lemma using a recursive proof on the number of quantifiers in each sentence.
\begin{description}
\item[Base Case:] 
The base case is the case of the sentences with $\integer{k}=0$ quantifiers. 
But these sentences are just the matrix instances in the derivation. 
We already proved in The Method Lemma 2 (Thm. \ref{MethodLemmaB}) that all these sentences are true the matrix model $\IntA_M$, so the base case is complete.

\item[Inheritance Step:] \hfill
\begin{description}
\item[Recursive Assumption:]
Our recursive assumption is that all sentences in the derivation with less than $\integer{k}$ quantifiers are true in the matrix model $\IntA_M$.

\item[Existential Quantifier:]
Say $\CAPTHETA$ is a sentence appearing in the derivation of the form $\existential{\ALPHA}\CAPPHI$, where $\CAPPHI$ is a formula with $\integer{k}-1$ quantifiers. 
Step 3 of the method guarantees that the sentence $\CAPPHI\variable{t}/\ALPHA$, for some constant $\variable{t}$, appears somewhere in the derivation. 
This sentence $\CAPPHI\variable{t}/\ALPHA$ has $\integer{k}-1$ quantifiers, so by the recursive assumption it is true in the matrix model $\IntA_M$. 
But then the existentially quantified sentence $\existential{\ALPHA}\CAPPHI$ is true in $\IntA_M$ as well.
(To show this rigorously, consider the $\variable{s}$-variant of $\IntA_M$, $\As{\variable{s}_1}{}$, that assigns the same element of the universe of $\IntA_M$ to $\variable{s}$ as $\IntA_M$ assigns to the constant $\variable{t}$.
Now by the Dragnet Theorem, Thm. \pncmvref{The Dragnet Theorem}, since $\IntA_M$ makes $\CAPPHI\variable{t}/\ALPHA$ true, the sentence $\CAPPHI\variable{s}/\ALPHA$ is true on $\As{\variable{s}_1}{}$.
It follows from this that $\existential{\ALPHA}\CAPPHI$ is true on $\IntA_M$.)

\item[Universal Quantifier:]
Say $\CAPTHETA$ is a sentence appearing in the derivation of the form $\universal{\ALPHA}\CAPPHI$, where $\CAPPHI$ is a formula with $\integer{k}-1$ quantifiers. 
All instances $\CAPPHI\variable{t}/\ALPHA$ which appear in the derivation have $\integer{k}-1$ quantifiers, and so by the recursive hypothesis are true in the matrix model $\IntA_M$. 
If we consider any $\variable{s}$-variant of $\IntA_M$, we know that what it assigns to $\variable{s}$ must be a number from the universe of $\IntA_M$;
we also know from the way that we constructed the matrix model $\IntA_M$ that a number was included in the universe of $\IntA_M$ only if it was assigned to some constant that occurred in the derivation.  
Let what's assigned to $\variable{s}$ by some $\variable{s}$-variant, $\As{\variable{s}_2}{}$, be the number associated with the constant $\constant{c}$.
Because the universal statement $\universal{\ALPHA}\CAPPHI$ occurred in the derivation, we know that we took all instances of it, including $\CAPPHI\constant{c}/\ALPHA$. 
As already stated, all instances $\CAPPHI\variable{t}/\ALPHA$ are true in $\IntA_M$, including $\CAPPHI\constant{c}/\ALPHA$.
By the Dragnet Theorem (Thm. \pncmvref{The Dragnet Theorem}), since $\CAPPHI\constant{c}/\ALPHA$ is true in $\IntA_M$ it follows that $\CAPPHI\variable{s}/\ALPHA$ is true on $\As{\variable{s}_2}{}$. 
But the exact same argument will work for every $\variable{s}$-variant of $\IntA_M$; so $\CAPPHI\variable{s}/\ALPHA$ is true on every $\variable{s}$-variant of $\IntA_M$. 
It follows that $\universal{\ALPHA}\CAPPHI$ is true on the matrix model $\IntA_M$. 
\end{description}

\item[Closure Step:]
Every sentence in the derivation is true in the matrix model $\IntA_M$, which is what was to be shown. 
\end{description}
\end{PROOF}

\subsection{Proving Completeness}
In\index{completeness!weak \GQD{}} this section we put together all the pieces from the last section to prove that \GQD{} is complete. 
\begin{THEOREM}{\LnpTC{MainGQDWCompletenessLemma} Main Weak \GQD{} Completeness Lemma:}
For all sentences $\CAPTHETA$ of \GQL{}, if the method is applied to $\negation{\CAPTHETA}$ then either: (a) the method produces a derivation of $\CAPTHETA$ in \GQDP{}, or (b) there is some model $\IntA$ that makes $\CAPTHETA$ false.
\end{THEOREM}
\begin{PROOF}
If the method is applied to $\negation{\CAPTHETA}$, then either (1) it will produce a contradiction $\conjunction{\Al}{\negation{\Al}}$, (2) the method halts without a contradiction, or (3) the method never halts (and hence never halts in a contradiction). 
If (1), then by the Derivational Lemma (Thm. \pmvref{Derivational Lemma}) there is a derivation of $\CAPTHETA$. 

If either (2) or (3) is the case, then by the Method Lemma 3 (Thm. \pmvref{MethodLemmaC}) we know that all sentences in the derivation starting with $(\negation{\CAPTHETA})^*$, the prenex disjunctive normal form sentence produced in Step 0 of the method from the sentence $\negation{\CAPTHETA}$ on line 1, are true in the matrix model $\IntA_M$. 
Note that all the steps in the derivation of $(\negation{\CAPTHETA})^*$ from $\negation{\CAPTHETA}$ are sanctioned by exchange rules;
therefore those steps can be turned upside down to produce a derivation in \GQDP{} of $\negation{\CAPTHETA}$ from $(\negation{\CAPTHETA})^*$. 
So by theorem \mvref{GQD Shortcut Theorem3} there's a derivation in \GQD{} of $\negation{\CAPTHETA}$ from $(\negation{\CAPTHETA})^*$.
Since \GQD{} is sound (Thm. \pmvref{Soundness of Quantifier Logic}), it follows that $(\negation{\CAPTHETA})^*\sdtstile{}{}\;\negation{\CAPTHETA}$.
Since we know that $(\negation{\CAPTHETA})^*$ is true in the matrix model $\IntA_M$, it follows that $\negation{\CAPTHETA}$ is true in $\IntA_M$ too.
So it follows that $\CAPTHETA$ is false in $\IntA_M$. 
\end{PROOF}
\begin{THEOREM}{\LnpTC{GQDWeakCompletenessTheorem} Weak \GQD{} Completeness Theorem:}
For all sentences $\CAPTHETA$ of \GQL{}, if $\sdtstile{}{}\CAPTHETA$, then $\sststile{}{}\CAPTHETA$ in \GQD{}.
\end{THEOREM}
\begin{PROOF}
Assume that $\sdtstile{}{}\CAPTHETA$. Then there are no models $\IntA$ which makes $\CAPTHETA$ false. 
Thus, if the method is applied to $\negation{\CAPTHETA}$ it can't be that some model $\IntA$ makes $\CAPTHETA$ false. 
By the Main \GQD{} Weak Completeness Lemma (Thm. \ref{MainGQDWCompletenessLemma}), it follows that when the method is applied to $\negation{\CAPTHETA}$ it produces a derivation of $\CAPTHETA$ in \GQDP{}. 
Hence there is a derivation of $\CAPTHETA$ in \GQD{}.
\end{PROOF}
\begin{THEOREM}{\LnpTC{GQDCompletenessTheorem} \GQD{} Completeness Theorem:}
For all finite sets $\Delta$ of \GQL{} sentences and \GQL{} sentence $\CAPPHI$, if $\Delta\sdtstile{}{}\CAPPHI$, then $\Delta\sststile{}{}\CAPPHI$.
\end{THEOREM}
\begin{PROOF}
The theorem follows immediately from the Weak \GQD{} Completeness Theorem and theorem \mvref{RegWeakCompletenessEquiv}.
\end{PROOF}

A consequence of our Strong Method  is that if $\CAPPHI$ is  entailed by an infinite set of sentences $\Delta$, it is entailed by and derivable from  a finite subset of $\Delta$.  If the Strong Method does not go on forever, then we get a contradiction at a finite stage and we have only assumed a finite subset of $\Delta$.

\subsection{Shortcut Rules for the Method}
The method discussed in section \ref{The Method Section} becomes practically unwieldy.
For example, if the matrix has three disjuncts with two sentences each, then combining two instances gives 9 disjuncts with 4 elements each, and combining three gives 27 disjuncts with 8 elements each. 
Thus we will use some additional short cut rules to speed the process of detecting contradictions. 
(But note that \emph{two} of the shortcut Rules we add here are not \emph{exchange} shortcut rules.  Greg's rule is the exception.)

Our first shortcut rule is \Rule{Greg's Rule}. We\index{Greg's Rule} know that if a conjunction contains an atomic formula and the negation of that atomic formula then we can derive the negation of the conjunction.
E.g., we can derive the negation of $\parconjunction{\Kp{\constant{a}}}{\conjunction{\Gp{\constant{b}}}{\conjunction{\Kp{\constant{c}}}{\negation{\Gp{\constant{b}}}}}}$.
So if we have a disjunction, one disjunct of which contains a contradiction of this kind, we can derive the negation of that disjunct and use disjunctive syllogism to prune that disjunct.
For example, assume the Method ends at the sentence $\disjunction{\parconjunction{\Kp{\constant{a}}}{\conjunction{\Gp{\constant{b}}}{\conjunction{\Kp{\constant{c}}}{\negation{\Gp{\constant{b}}}}}}}{\parconjunction{\Kp{\constant{a}}}{\conjunction{\Gp{\constant{b}}}{\conjunction{\Kp{\constant{c}}}{\negation{\Gp{\constant{a}}}}}}}$. We can independently derive $\negation{\parconjunction{\Kp{\constant{a}}}{\conjunction{\Gp{\constant{b}}}{\conjunction{\Kp{\constant{c}}}{\negation{\Gp{\constant{b}}}}}}}$. From these two we derive  $\parconjunction{\Kp{\constant{a}}}{\conjunction{\Gp{\constant{b}}}{\conjunction{\Kp{\constant{c}}}{\negation{\Gp{\constant{a}}}}}}$. 
Greg's Rule lets us accomplish those steps by crossing out the contradictory part and writing down the remaining ones.

It\index{$\VEE$/$\WEDGE$-Elim} is helpful to have a short cut rule which combines \Rule{$\WEDGE\!$-Elim} steps with \Rule{$\VEE$-Elim} steps to go from a disjunction of which each disjunct contains a particular sentence to that sentence itself on a later line;
we will call it \Rule{$\VEE$/$\WEDGE\!$-Elim} and it sanctions the step from $\disjunction{\parconjunction{\Kp{\constant{a}}}{\conjunction{\Gp{\constant{b}}}{\conjunction{\Kp{\constant{c}}}{\negation{\Gp{\constant{c}}}}}}}{\parconjunction{\Kp{\constant{a}}}{\conjunction{\Gp{\constant{b}}}{\conjunction{\Kp{\constant{c}}}{\negation{\Gp{\constant{a}}}}}}}$ to $\Gp{\constant{b}}$. 
In addition to citing the justification, if the sentence is at all complex you should circle the repeated subsentence.

Finally,\index{One Bad Apple} given the opposite of even one conjunct in a conjunction, we can derive the negation of the conjunction.
E.g., from $\Gp{\constant{a}}$ we can derive $\negation{\parconjunction{\Kp{\constant{a}}}{\conjunction{\Gp{\constant{b}}}{\conjunction{\Kp{\constant{c}}}{\negation{\Gp{\constant{a}}}}}}}$.
We will call this rule \Rule{One Bad Apple}, or \Rule{OBA}.
%\begin{table}[!ht]
%\renewcommand{\arraystretch}{1.5}
%\begin{center}
%\begin{tabular}{ p{1in} l l } %p{2.2in} p{2in}
%\toprule
%\textbf{Name} & \textbf{Given} & \textbf{May Add} \\ 
%\midrule
\renewcommand{\arraystretch}{1.5}
\begin{longtable}[c]{ p{1in} l l } %p{2.2in} p{2in}
\toprule
\textbf{Name} & \textbf{Given} & \textbf{May Add} \\ 
\midrule
\endfirsthead
\multicolumn{3}{c}{\emph{Continued from Previous Page}}\\
\toprule
\textbf{Name} & \textbf{Given} & \textbf{May Add} \\ 
\midrule
\endhead
\bottomrule
\caption{Short-Cut Rules for the Method}\\[-.15in]
\multicolumn{3}{c}{\emph{Continued next Page}}\\
\endfoot
\bottomrule
\caption{Short-Cut Rules for the Method}\\
\endlastfoot
\label{GSDplusMethod}\Rule{Greg's Rule} & $\disjunction{\CAPPSI_1}{\disjunction{\ldots}{\CAPPSI_{\integer{n}}}}$, where some & $\disjunction{\CAPPSI_1}{\disjunction{\ldots}{\disjunction{\CAPPSI_{\integer{i}-1}}{\disjunction{\CAPPSI_{\integer{i}+1}}{\disjunction{\ldots}{\CAPPSI_{\integer{n}}}}}}}$ \\[-.25cm]
 & $\CAPPSI_{\integer{i}}=\conjunction{\CAPPHI_1}{\conjunction{\ldots}{\conjunction{\CAPPHI_{\integer{j}}}{\ldots}}}$ & \\[-.25cm]
\nopagebreak
 & $\WEDGE\conjunction{\negation{\CAPPHI_{\integer{j}}}}{\conjunction{\ldots}{\CAPPHI_{\integer{m}}}}$ & \\
 
\Rule{$\VEE$/$\WEDGE\!$-Elim} & $\disjunction{\CAPPSI_{1}}{\disjunction{\ldots}{\CAPPSI_{\integer{n}}}}$, where & $\CAPPHI$ \\[-.25cm]
 & each $\CAPPSI_{\integer{i}}$ contains $\CAPPHI$ & \\
 
\Rule{OBA} &  $\conjunction{\CAPPHI_1}{\conjunction{\ldots}{\conjunction{\CAPPHI_i}{\conjunction{\ldots}{\CAPPHI_{\integer{n}}}}}}$, $\negation{\CAPPHI_i}$ & $\negation{\parconjunction{\CAPPHI_1}{\conjunction{\ldots}{\conjunction{\CAPPHI_i}{\conjunction{\ldots}{\CAPPHI_{\integer{n}}}}}}}$ \\
\nopagebreak
 & $\conjunction{\CAPPHI_1}{\conjunction{\ldots}{\conjunction{\negation{\CAPPHI_i}}{\conjunction{\ldots}{\CAPPHI_{\integer{n}}}}}}$, ${\CAPPHI_i}$ & $\negation{\parconjunction{\CAPPHI_1}{\conjunction{\ldots}{\conjunction{\negation{\CAPPHI_i}}{\conjunction{\ldots}{\CAPPHI_{\integer{n}}}}}}}$ \\
\end{longtable}
%\bottomrule
%\end{tabular}
%\end{center}
%\caption{Exchange Short-Cut Rules for \GSD{} (\GSDP{})}
%\label{GSDplus2}
%\end{table}

\section{Strong Completeness and Other Results}\label{Sec:Proving Strong Completeness}
In this section we want to extend our results and show that \GQD{} is strongly complete.
Note that the method we used to extend the Weak Completeness Theorem to the Completeness Theorem will not work here.
To do that, we used theorem \mvref{RegWeakCompletenessEquiv}, the proof of which depending on $\Delta$ being finite.  
To show that \GQD{} is strongly complete, we have modify the method we used to show that it's weakly complete.
\begin{THEOREM}{\LnpTC{GQDStrongCompletenessTheorem} Strong \GQD{} Completeness Theorem:}
For any set $\Delta$ of \GSL{} sentences and any \GSL{} sentence $\CAPPHI$, if $\Delta\sdtstile{}{}\CAPPHI$, then $\Delta\sststile{}{}\CAPPHI$.
\end{THEOREM}
\noindent{}To show that \GQD{} is weakly complete, we gave a method that, given a sentence $\CAPTHETA$, either produces a derivation of $\CAPTHETA$ or produces a model $\IntA$ which makes $\CAPTHETA$ false. 
To show that \GQD{} is strongly complete, what we want is a method that, given a (possibly infinite) set $\Delta$ of sentences and another sentence $\CAPPHI$, either produces a derivation of a contradiction $\conjunction{\Al}{\negation{\Al}}$ from $\negation{\CAPPHI}$ and some finite subset of $\Delta$ or produces a model $\IntA$ that makes $\negation{\CAPPHI}$ and every sentence in $\Delta$ true.

The method we'll give is a modification of the original method given in section \ref{The Method Section}. 
Since it is just a modification of the the original method, we'll only sketch the changes needed. 
We'll call the modified method the \niidf{strong method}\index{strong method, the}\index{method, the!strong}. 
Given some possibly countably infinite set $\Delta$ and sentence $\CAPPHI$, the strong method is:
\begin{description}
\item[Step 1:] Let $\Delta^*=\Delta\cup\{\negation{\CAPPHI}\}$. 
Then pair each sentence of $\Delta^*$ with a natural number and use that to determine the order in which they are assumed. 
The only constraint on this ordering is that $\negation{\CAPPHI}$ should be first.

\item[Step 2:] Put the first sentence of $\Delta^*$ on line 1 and put it in \CAPS{pdnf}, just as was done in Step 0 of the method.

\item[Step 3:] Apply Step 1 of the method.

\item[Step 4:] Apply Steps 2 and 3 of the method to the whole derivation thus far.

\item[Step 5:] Check for contradictions, just as in Step 4 of the method.

\item[Step 6:] \hfill
\begin{cenumerate}
\item If there's a contradiction, stop.
\item If there's no contradiction, write the next sentence of $\Delta^*$ on the next line of the derivation, put that sentence in \CAPS{pdnf}, and go back into Step 4. 
\end{cenumerate}
\end{description}
The strong method will either halt in a contradiction, or not. 
\begin{THEOREM}{\LnpTC{DerivationalLemmaS} Strong Derivational Lemma:}
If the strong method halts in a contradiction, then $\Delta\sststile{}{}\CAPPHI$.
\end{THEOREM}
\begin{PROOF}
If the strong method halts in a contradiction, then it will have produced a derivation of a contradiction $\conjunction{\Al}{\negation{\Al}}$ from $\negation{\CAPPHI}$ and some subset $\Delta'$ of $\Delta$.  We want to show that $\Delta\sststile{}{}\CAPPHI$; to do this, we first want to show that $\Delta'\sststile{}{}\CAPPHI$.

We might think we already know that $\negation{\CAPPHI},\Delta'\sststile{}{}\conjunction{\Al}{\negation{\Al}}$, but in fact, there are additional open assumptions that we made; the Strong Method tells us to make an additional assumption for each line containing an existentially quantified sentence. For these lines, we assume an instance of the existentially quantified sentence.  We want to discharge these assumptions by using \Rule{$\exists$-Elim}, but these assumptions may come prior to some of the assumptions we made from $\Delta'$.  We want to keep the sentences of $\Delta'$ as assumptions, because the contradiction we reached depends on them. 

We can discharge the assumed instances of existentially quantified sentences only by additionally discharging all the assumptions that come later in the derivation.  Accordingly, we want to extend our derivation so that we discharge \emph{all} our assumptions and then repeat all the assumptions \emph{except} for the instances of existentially quantified sentences. 

Let $\CAPTHETA_1, \CAPTHETA_2, \CAPTHETA_3, \ldots, \CAPTHETA_{\integer{n}}$ be the sentences of $\Delta'$ assumed in our derivation.  The last open assumption is either (a) the last sentence of $\Delta'$, i.e., $\CAPTHETA_{\integer{n}}$; (b) an instance of an existentially quantified sentence on an earlier line of the derivation which we'll call $\CAPPSI$; or (c) $\negation{\CAPPHI}$.  If (a) is the case, then discharge the assumption by applying the rule \Rule{$\HORSESHOE$-Intro} to get $\horseshoe{\CAPTHETA_{\integer{n}}}{\parconjunction{\Al}{\negation{\Al}}}$.  If (b) is the case, then first apply \Rule{$\HORSESHOE$-Intro} to get $\horseshoe{\CAPPSI}{\parconjunction{\Al}{\negation{\Al}}}$, and then apply \Rule{$\exists$-Elim} to get $\conjunction{\Al}{\negation{\Al}}$.  When (b) is the case, we know that the assumption introduced a \emph{new} constant, and therefore we know that that constant did not appear in any earlier assumption or in the existential of which we are taking an instance.
It also (obviously) does not occur in $\conjunction{\Al}{\negation{\Al}}$.
Thus the \Rule{$\exists$-Elim} step is legitimate. If (c) is the case then we don't have to worry about instances of existentially quantified sentences, and we can skip this part of the process.  Let us disregard case (c) for now.

Now we must continue to discharge the rest of the assumptions.  For any assumption of an instance of an existentially quantified sentence, we may do the same thing we did in (b) above---first use \Rule{$\HORSESHOE$-Intro} to derive a conditional, and then use \Rule{$\exists$-Elim} to derive the RHS of that conditional.  For any assumption that is a sentence of $\Delta'$ (i.e., $\CAPTHETA_{\integer{i}}$), we will use \Rule{$\HORSESHOE$-Intro} to derive a conditional, as in (a) above.  So, after discharging the assumption with the second to last sentence from $\Delta'$ we get $\horseshoe{\CAPTHETA_{\integer{n-1}}}{\parhorseshoe{\CAPTHETA_{\integer{n}}}{\parconjunction{\Al}{\negation{\Al}}}}$.  After the third instance, we derive $\horseshoe{\CAPTHETA_{\integer{n-2}}}{\parhorseshoe{\CAPTHETA_{\integer{n-1}}}{\parhorseshoe{\CAPTHETA_{\integer{n}}}{\parconjunction{\Al}{\negation{\Al}}}}}$.  And so on, so that we eventually get a sentence of the form $\horseshoe{\CAPTHETA_1}{\parhorseshoe{\CAPTHETA_2}{\parhorseshoe{\CAPTHETA_3}{\ldots \parhorseshoe{\CAPTHETA_{\integer{n}}}{\parconjunction{\Al}{\negation{\Al}}}}}}$.

After discharging all such assumptions, the only open assumption left is $\negation{\CAPPHI}$.  Apply \Rule{$\HORSESHOE$-Intro} once more to get something like the following: $\horseshoe{\negation{\CAPPHI}}{\parhorseshoe{\CAPTHETA_1}{\parhorseshoe{\CAPTHETA_2}{\ldots \parhorseshoe{\CAPTHETA_{\integer{n}}}{\parconjunction{\Al}{\negation{\Al}}}}}}$.  Now we have no open assumptions remaining.  (Note that we have effectively covered case (c) above, since applying \Rule{$\HORSESHOE$-Intro} in this case would give us $\horseshoe{\negation{\CAPPHI}}{\parconjunction{\Al}{\negation{\Al}}}$ and no open assumptions.  In this case, we didn't have to assume any of the sentences of $\Delta$ to get a contradiction, so $\Delta'$ is the empty set.)

At this point we have a conditional, possibly a very long one.  The RHS of the last conditional (possibly embedded in several conditionals) is our contradiction, $\parconjunction{\Al}{\negation{\Al}}$.  We want to show that $\Delta'\sststile{}{}\CAPPHI$, so let us make a series of assumptions from the sentences of $\Delta'$.  That is, let us assume each of $\CAPTHETA_1, \CAPTHETA_2, \CAPTHETA_3, \ldots, \CAPTHETA_{\integer{n}}$.  Now that we've assumed all the sentences of $\Delta'$, let us assume $\negation{\CAPPHI}$.

Given our earlier conditional of the form $\horseshoe{\negation{\CAPPHI}}{\parhorseshoe{\CAPTHETA_1}{\parhorseshoe{\CAPTHETA_2}{\ldots, \parhorseshoe{\CAPTHETA_{\integer{n}}}{\parconjunction{\Al}{\negation{\Al}}}}}}$ and the assumption $\negation{\CAPPHI}$, we may now apply \Rule{$\HORSESHOE$-Elim} to get $\horseshoe{\CAPTHETA_1}{\parhorseshoe{\CAPTHETA_2}{\parhorseshoe{\CAPTHETA_3}{\ldots, \parhorseshoe{\CAPTHETA_{\integer{n}}}{\parconjunction{\Al}{\negation{\Al}}}}}}$.  Because we also have all the sentences of $\Delta'$ as assumptions (i.e., all of $\CAPTHETA_1, \CAPTHETA_2, \CAPTHETA_3, \ldots, \CAPTHETA_{\integer{n}}$), we may apply a series of \Rule{$\HORSESHOE$-Elim} steps until we eventually derive $\conjunction{\Al}{\negation{\Al}}$.  That is, given our earlier assumption $\CAPTHETA_1$ and the conditional $\horseshoe{\CAPTHETA_1}{\parhorseshoe{\CAPTHETA_2}{\parhorseshoe{\CAPTHETA_3}{\ldots, \parhorseshoe{\CAPTHETA_{\integer{n}}}{\parconjunction{\Al}{\negation{\Al}}}}}}$, we may apply \Rule{$\HORSESHOE$-Elim} to derive $\horseshoe{\CAPTHETA_2}{\parhorseshoe{\CAPTHETA_3}{\ldots, \parhorseshoe{\CAPTHETA_{\integer{n}}}{\parconjunction{\Al}{\negation{\Al}}}}}$.  And then because we have $\CAPTHETA_2$ as an assumption we may again apply \Rule{$\HORSESHOE$-Elim} to get $\horseshoe{\CAPTHETA_3}{\ldots, \parhorseshoe{\CAPTHETA_{\integer{n}}}{\parconjunction{\Al}{\negation{\Al}}}}$.  And so on, until we derive $\conjunction{\Al}{\negation{\Al}}$.

Remember that our last open assumption is $\negation{\CAPPHI}$.  We may now discharge that assumption and apply \Rule{$\HORSESHOE$-Intro} to get $\horseshoe{\negation{\CAPPHI}}{\parconjunction{\Al}{\negation{\Al}}}$.  Then we apply \Rule{$\NEGATION$-Elim} to derive $\CAPPHI$.

Now we have as our open assumptions only the sentences of $\Delta'$ and we have derived $\CAPPHI$.  We have thus shown that $\Delta'\sststile{}{}\CAPPHI$.  And because $\Delta'$ is a subset of $\Delta$, it follows that $\Delta\sststile{}{}\CAPPHI$.
\end{PROOF}
\noindent{}Next, note that if the strong method doesn't halt in a contradiction, then we will have a list of matrix instances from which we can construct a matrix model $\IntA_M$ in just the same way we did for the method (Def. \pmvref{MatrixModel}).
Similar to the method, we have the following three theorems.
\begin{THEOREM}{\LnpTC{MethodSLemmaA} The Strong Method Lemma 1:}
The matrix model $\IntA_M$ makes true all sentences on the master matrix list $M$.
\end{THEOREM}
\begin{PROOF}
The same proof used for the method (Thm. \pmvref{MethodLemmaA}) applies here too.
\end{PROOF}
\begin{THEOREM}{\LnpTC{MethodSLemmaB} The Strong Method Lemma 2:}
All matrix instances in the derivation are true in the matrix model $\IntA_M$.
\end{THEOREM}
\begin{PROOF}
The same proof used for the method (Thm. \pmvref{MethodLemmaB}) applies here too.
\end{PROOF}
\begin{THEOREM}{\LnpTC{MethodSLemmaC} The Strong Method Lemma 3:}
All quantified sentences in the derivation are true in the matrix model $\IntA_M$.
\end{THEOREM}
\begin{PROOF}
The same proof used for the method (Thm. \pmvref{MethodLemmaC}) applies here too, so long as we can show that if an existential $\existential{\ALPHA}\CAPPSI$ appears on a line, at least one instance $\CAPPSI\variable{t}/\ALPHA$ does, and if a universal $\universal{\ALPHA}\CAPPSI$ appears on a line, then every instance $\CAPPSI\variable{t}/\ALPHA$ of it with a constant $\variable{t}$ appearing somewhere in the derivation appears somewhere in the derivation. So, all we need to show is that the strong method will derive the appropriate instances of all quantified sentences that appear in our derivation. Let $\Delta'$ be those sentences of $\Delta^*$ that appear in our derivation as a result of the application of the strong method.

\begin{description}
\item[Existential Quantifier:] 
Say $\CAPTHETA$ is some sentence in the derivation of the form $\existential{\ALPHA}\CAPPHI$. 
Step 4 of the strong method uses step 3 of the method on $\CAPTHETA$, which guarantees that the sentence $\CAPPHI\variable{t}/\ALPHA$, for some constant $\variable{t}$, appears somewhere in the derivation.
By hypothesis, step 4 of the strong method is applied to all sentences in $\Delta'$, so we know that it will derive the appropriate instances of all existentially quantified statements in in $\Delta'$.

\item[Universal Quantifier:]
Say $\CAPTHETA$ is a sentence appearing in the derivation of the form $\universal{\ALPHA}\CAPPHI$. 
Step 4 of the strong method uses step 2 of the method on $\CAPTHETA$, which, for every constant $\variable{t}$ in the derivation, guarantees that the sentence $\CAPPHI\variable{t}/\ALPHA$ is derived.
By hypothesis, step 4 of the strong method is applied to all sentences in $\Delta'$, so we know that it will derive the appropriate instances of all universally quantified statements in $\Delta'$.
\end{description}

\noindent{}So, the strong method derives the appropriate instances of all quantified sentences in $\Delta'$.

%(Why? assume not. Then say that universal is PHI and that constant is b. Since the derivation is never ending, there must have been a pass of steps 1 and 2 that comes after both the pass that introduced b and the pass that introduced PHI. So contra assumption, this pass put that instance of PHI in the sequence.).
\end{PROOF}
\noindent{}Finally, we have one last lemma:
\begin{THEOREM}{\LnpTC{MainGQDSCompletenessLemma} Main Strong \GQD{} Completeness Lemma:}
For all sets of \GQL{} sentences $\Delta$ and \GQL{} sentences $\CAPPHI$, if the strong method is applied to $\Delta^*=\Delta\cup\{\negation{\CAPPHI}\}$ then either: (a) the strong method produces a derivation of $\CAPPHI$ from $\Delta$ in \GQDP{}, or (b) there is a model $\IntA$ which makes every sentence in $\Delta$ true and $\CAPPHI$ false.
\end{THEOREM}
\begin{PROOF}
If the method is applied to $\Delta^*=\Delta\cup\{\negation{\CAPPHI}\}$, then either it halts in a contradiction or not. 
By the Strong Derivational Lemma (\pmvref{DerivationalLemmaS}), if the strong method halts in a contradiction, then $\Delta\sststile{}{}\CAPPHI$ in \GQDP{}.

If the method does not halt in a contradiction, then by the Strong Method Lemma 3 (Thm. \pmvref{MethodSLemmaC}) the matrix model $\IntA_M$  makes all the sentences in the derivation true. 
But since the strong method did not halt in a contradiction, for every sentence $\CAPPSI$ in $\Delta$ and $\negation{\CAPPHI}$, there's some sentence in \CAPS{pdnf} that's quantificationally equivalent to $\CAPPSI$ and appears in the derivation. 
So $\IntA_M$ makes all the sentences in $\Delta$ and the sentence $\negation{\CAPPHI}$ true; 
hence $\IntA_M$ makes all the sentences in $\Delta$ true and $\CAPPHI$ false. 
\end{PROOF}
\begin{PROOFOF}{Thm. \ref{GQDStrongCompletenessTheorem}, The Strong Completeness Theorem for GQD}
Assume that $\Delta\sdtstile{}{}\CAPPHI$. 
Then there can be no model $\IntA$ which makes all of the sentences in $\Delta$ true and $\CAPPHI$ false;
so, application of the method can't produce such a model.
Thus by the Main \GQD{} Strong Completeness Lemma (Thm. \ref{MainGQDSCompletenessLemma}), $\Delta\sststile{}{}\CAPPHI$ in \GQDP{}. 
It follows by theorem \mvref{GQD Shortcut Theorem3} that $\Delta\sststile{}{}\CAPPHI$ in \GQD{}.
\end{PROOFOF}

Our Method for proving completeness for QD is short of being a decision procedure.  If $\CAPPHI$ is a logical truth, then The Method will produce a derivation of it.  And if $\CAPPHI$ is not a logical truth, then in many cases it produces a model that makes $\CAPPHI$ false.  But sometimes it just doesn't stop.  We know that if it doesn't stop there is a model that makes the original sentence false, but at each stage we may not know whether it will stop (soon?) or not.   And we know that it can't stop at a finite stage for some sentences because those sentences are only false in an infinite model.

All we know from our work so far is that The Method works as described above. We don't know that there isn't a better method that provides a decision procedure.  Church's Theorem, proved by more advanced methods that involve clarifying what counts as a ``method'' or ``algorithm'' tells us that our result is as good as we can do for all of \GQL{}.

However, we can do better for the language \GQL{}1 and a little more.

%%%%%%%%%%%%%%%%%%%%%%%%%%%%%%%%%%%%%%%%%%%%%%%%%%
\section{Decidability and Church's Theorem}\label{Decidability and Churchs Theorem}
%%%%%%%%%%%%%%%%%%%%%%%%%%%%%%%%%%%%%%%%%%%%%%%%%%

Next\index{decidable}\index{undecidable} we turn to refinements of the method to obtain what are called \niidf{decision procedures} for logical truth in a language \Language{L}. %\index{decision procedure}
We first introduced the idea of a decision procedure in section \mvref{Section:Intro to Decidability}; here we shall fill things out a bit further. 
\begin{majorILnc}{\LnpDC{Def:DecisionProcedure}}
A \df{decision procedure} for logical truth in a language (or sublanguage) \Language{L} is a completely specified method which produces, for any sentence $\CAPPHI$ of \Language{L} and in a finite number of steps, the answer YES if $\CAPPHI$ is a logical truth and the answer NO otherwise.
\end{majorILnc}
\begin{majorILnc}{\LnpEC{TruthTableDecisionProcedure}}
We have already seen two decision procedures for \CAPS{tft} in \GSL{}: truth tables and the method discussed in the completeness proof of \GSD{}. 
(Others include truth trees and Quine's ``fell swoop'', \citealt{Quine1950}, \citealt[23]{Hodges2001}.)
The truth-table decision procedure is simple:\index{decision procedure!truth table} take a sentence $\CAPPHI$ of \GSL{} and construct a truth table for it. If you get all $\TrueB{}$ in the column under $\CAPPHI$; answer YES. If you don't, answer NO. 
Likewise for the method discussed in the completeness proof of \GSD{}:\index{decision procedure!for \CAPS{tft} in \GSL{}} take a sentence $\CAPPHI$, negate it to get $\negation{\CAPPHI}$, and apply the method. 
If it results in a contradiction $\conjunction{\Al}{\negation{\Al}}$, answer YES. 
If no contradiction is reached, answer NO. 
\end{majorILnc}
\noindent{}Our method of proving completeness for \GQD{} is a little short of being a decision procedure for quantificational truth in \GQL{} because it does not always produce an answer in a finite amount of time.
It can be shown that there is no decision procedure for the whole language \GQL{} \citetext{\citealp[83--86]{Hodges2001}, \citeyear[31]{Hodges2001b}, \citealp[486]{Bergmann2003}}.
\begin{THEOREM}{\LnpTC{ChurchsTheorem} Church's Theorem:}
If\index{Church's Theorem}\index{decision procedure!for \CAPS{qt} in \GQL{}|see{Church's Theorem}} \Language{L} is a sublanguage of \GQL{} with (1) the same logical connectives as \GQL{}, and (2) at least one 2-place predicate symbol, then there is no decision procedure for the set of logical truths of \Language{L}.\footnote{Actually, 
Church's Theorem also says that if we also consider languages with function symbols, then if \Language{L} has at least two 1-place function symbols there is no decision procedure for the set of logical truths of \Language{L}.}
\end{THEOREM}
Church's Theorem at once tells us that there is no decision procedure for quantificational truth in \GQL{}, but we can show that with some modifications our method from section \ref{The Method Section} can be turned into a decision procedure for certain sublanguages of \GQL{}. 
As the thesis suggests, one such sublanguage \Language{L} of \GQL{} is the language that consists of just 1-place predicate symbols. 
We've been calling this language \GQL{}1.\index{decision procedure!for \CAPS{qt} in monadic \GQL{}}\index{GQL!monadic}

The basic insight for modifying the method is that infinite loops are created by having an existential quantifier inside a universal quantifier; 
the existential requires a new constant to be instantiated, and that creates a new potential instance for the universal, which then creates a new existential, and so on. 

Put more positively, if the method produces a sentence in standardized form which has only existential, or only universal quantifiers, then the method will stop. 
Moreover, if in the standardized form the existentials all precede the universals the method will also stop, for we will first take instances of all the existentials using $\integer{n}$ constants if there are $\integer{n}$ existentials, and afterwards we will instantiate the $\integer{n}$ new constants in the $\integer{m}$ universals giving $\integer{n}^\integer{m}$ instances.
But, we will be done then since no more new constants will be added. 

The general principle is that if one quantifier occurs within the scope of another then it cannot be moved in front of the other quantifier. 
You may remember that the scope of a quantifier, say $\forall$ in $\universal{\variable{x}}\CAPPHI$, is the subformula $\CAPPHI$ of which it is the main connective. 
We can say two quantifiers are \niidf{independent}\index{quantifier!s, independent} if neither is in the scope of the other.
Let's say we have a sentence all of whose universal quantifiers are independent of each other.
Notice that these quantifiers can be brought forward in any order, and thus that we have a decision procedure for such sentences. 

The critical result to prove is that every sentence of \GQL{}1 is equivalent to a sentence whose quantifiers are independent. 
We can show this by appeal to the reverse of our procedure for putting sentences into prenex form, i.e. by moving quantifiers inward, but we use most of the same rules as for standard prenex (remember they are exchange rules).
\begin{THEOREM}{\LnpTC{MonadicGQLEquivTheorem}  \GQL{}1 Equivalence Theorem:}
Every sentence of \GQL{}1 is quantificationally equivalent to a sentence whose quantifiers are independent.
\end{THEOREM}
\begin{PROOF}
Proof here.
\end{PROOF}
\begin{THEOREM}{\LnpTC{MonadicDecisionTheorem} The \GQL{}1 Decision Theorem:}
The\index{Monadic Decision Theorem, The} modified method just described provides a decision procedure for quantificational truth in \GQL{}1.
\end{THEOREM}

%%%%%%%%%%%%%%%%%%%%%%%%%%%%%%%%%%%%%%%%%%%%%%%%%%
\section{L\"owenheim-Skolem and Compactness}
%%%%%%%%%%%%%%%%%%%%%%%%%%%%%%%%%%%%%%%%%%%%%%%%%%

A number of results follow directly from the completeness of \GQD{} or the method we used to prove completeness. 
\begin{THEOREM}{\LnpTC{LowenheimSkolemTheorem} The Downward L\"owenheim-Skolem Theorem:}
If a sentence of \GQL{} is true in any model, then it is true in one whose domain consists of all or some of the natural numbers.
\end{THEOREM}
\begin{PROOF}
If $\CAPPHI$ is true in some model, then $\negation{\CAPPHI}$ is not a quantificational truth.
Thus, applying the method to $\negation{\CAPPHI}$ will not produce a contradiction, but will produce a model of the natural numbers which falsifies $\negation{\CAPPHI}$ and hence makes $\CAPPHI$ true.
\end{PROOF}
It's important to note that there's really nothing special about the natural numbers.
When we devised the procedure for constructing a model of the ions in the master matrix list that results from the method (when no contradiction arises), we choose to use natural numbers for the universe. 
But it should be clear that we did this out of convenience (it's easy, after all, to associate constants with the natural numbers). 
We could have used any set of objects for the universe. 
What's important is that, whatever we used, the domain of the constructed model will at most be countably infinite (that is, it will at most be the size of the natural numbers and no larger). 
Hence a more abstract version of the downward Lowenheim-Skolem Theorem simply says: If a sentence of \GQL{} is true in any model, then either it's true in only models with finite domains, or, if it's true at all in models with infinite domains, then there's an model with a \emph{countably} infinite domain in which it's true. 

This thorem was proved in a weaker form originally by Leopold L\"owenheim \citeyearpar{Lowenheim1915}, and the proof was improved by Thoralf Skolem \citeyearpar{Skolem1920,Skolem1922}. 
Notice that the theorem talks only about models, and we have proved it via a detour through derivations. 
As you might imagine, there are more direct proofs, including Skolem's \citetext{\citealp{Tarski1956}, \citealp{Vaught1974}, \citealp[ch.~3.1]{Hodges1997}, \citeyear[63]{Hodges2001}}.

Notice also that after our work on \GQL{}1, we know that for monadic sentences the method will stop after a finite number of steps (if we arrange the prenex carefully) and so we can conclude that if a monadic sentence is true in any model then it is true in a finite one. 
\begin{THEOREM}{\LnpTC{MonadicIntSizeTheorem}}
If $\CAPPHI$ is a sentence of \GQL{}1 and has a model, then it has a finite model.
\end{THEOREM}

Our next corollary of completeness is the Compactness Theorem.
Although historically the completeness theorem was proved first and compactness followed as a corollary, today the compactness theorem takes center stage in many areas of logic (especially model theory). 
Like the L\"owenheim-Skolem Theorem, there are many different proofs of compactness that do not go through completeness or use any facts about derivations \citetext{see \citealt[321]{Kleene1967}, \citealt{Ebbinghaus1985}, \citealt[ch.~5.1]{Hodges1997}, \citealp[63]{Hodges2001}, \citeyear[29]{Hodges2001b}}. 
%answers the question of whether it's possible that an infinite set of sentences is intuitively contradictory but we cannot deduce the contradiction in our system because our derivations are finite?
%This is a specific version of the more general worry that if $\Delta$ is an infinite set then it might be that $\Delta\sdtstile{}{}\CAPPHI$ but not $\Delta\sststile{}{}\CAPPHI$. 
%We can prove that this doe not occur in our language by proving the following theorem.
\begin{THEOREM}{\LnpTC{Thm:CompactnessTheorem} The Compactness Theorem for \GQL{}:}
For all sets of sentences $\Delta$ of \GQL{}, if for every finite subset $\Delta'$ of $\Delta$ there exists a model $\IntA'$ that makes all the sentences in $\Delta'$ true, then there's some model $\IntA$ that makes all the sentences in $\Delta$ true. 
\end{THEOREM}
\begin{PROOF}
By the strong completeness theorem, for all sets $\Delta$ of \GQL{} sentences and \GQL{} sentence $\CAPPHI$, if $\Delta\sdtstile{}{}\CAPPHI$, then $\Delta\sststile{}{}\CAPPHI$.
Now assume that there's no model $\IntA$ that makes all the sentences in $\Delta$ true. 
Hence $\Delta\sdtstile{}{}\conjunction{\Al}{\negation{\Al}}$.
So by strong completeness, $\Delta\sststile{}{}\conjunction{\Al}{\negation{\Al}}$.
By definition, this implies that there's some finite subset $\Delta'$ of $\Delta$ such that $\conjunction{\Al}{\negation{\Al}}$ can be derived from $\Delta'$. 
Hence there is no model $\IntA$ that makes all the sentences in $\Delta'$ true. 
Hence it's not the case that for every finite subset $\Delta'$ of $\Delta$ there exists a model $\IntA'$ that makes all the sentences in $\Delta'$ true. 
\end{PROOF}

%%%%%%%%%%%%%%%%%%%%%%%%%%%%%%%%%%%%%%%%%%%%%%%%%%
\section{Exercises}
%%%%%%%%%%%%%%%%%%%%%%%%%%%%%%%%%%%%%%%%%%%%%%%%%%

\notocsubsection{Misc. Problems}{Misc Problems} 
\begin{enumerate}
\item Let's say that any derivation rule \Rule{R} that has the following property is \niidf{sound}:\index{derivation!rule!sound} if we add a line to a derivation $\Derivation{D}$ with sentence $\CAPPHI$ sanctioned by rule \Rule{R}, then $\Delta\sdtstile{}{}\CAPPHI$, where $\Delta$ is the set of unboxed assumptions for the new line. 
(Compare this with what it is for a rule to be truth-preserving, def. \pncmvref{Derivation Rule Soundness}, which is different.)
Then the proof of theorem \mvref{Main GSL Soundness Lemma} basically shows that \GSD{} is sound by showing that all the basic rules of \GSD{} are sound. 
We know that \GSDP{} is sound because \GSD{} is sound and (by theorem \pmvref{GSD Shortcut Theorem3}) anything you can derive in \GSDP{} can be derived in \GSD{}. 
But we could also show that \GSDP{} is sound directly (without appealing to theorem \ref{GSD Shortcut Theorem3}) by showing that the shortcut rules used in \GSDP{} themselves are sound. 
Of course, this follows from theorem \pmvref{GSD Shortcut Theorem2} and the fact that the basic rules are sound, but again we can show it directly. 
But again we can show it without going through the basic rules.
Show directly (without appealing to theorem \ref{GSD Shortcut Theorem2}) that the following rules are sound (see tables \pmvref{GSDplus1} and \pmvref{GSDplus2}): 
\begin{multicols}{2}
\begin{enumerate}
\item \Rule{M.T.}
\item \Rule{A.C.}
\item \Rule{$\HORSESHOE$/$\VEE$-Exch.}
\item \Rule{Contraposition}
\end{enumerate}
\end{multicols} 
\item Recall that $\HORSESHOE$ elimination can only be used on a conditional that is the main connective of a sentence. Show that if we do not make this restriction, then the rule is unsound. In other words, give a derivation which violates only that restriction (a derivation where you use $\HORSESHOE$ elimination on a horseshoe that's not the main connective) and which ends with a proof of a sentence that is \emph{not} a logical truth (not truth-functionally true) from the empty set of assumptions.
\end{enumerate}

%\theendnotes


%%%%%%%%%%%%%%%%%%%%%%%%%%%%%%%%%%%%%%%%%%%%%%%%%%
\chapter{Further Directions}\label{furtherdirections}
%%%%%%%%%%%%%%%%%%%%%%%%%%%%%%%%%%%%%%%%%%%%%%%%%%

%%%%%%%%%%%%%%%%%%%%%%%%%%%%%%%%%%%%%%%%%%%%%%%%%%
\section{Many-Valued Logic}
%%%%%%%%%%%%%%%%%%%%%%%%%%%%%%%%%%%%%%%%%%%%%%%%%%


In previous chapters we assumed that there are only two truth values. We now set aside that assumption. This allows us to explore many-valued logics. 
We explore the possibility that there are varying degrees of truth and falsity, such that there are sentences that are neither wholly true nor wholly false.
(Actually it turns out that there are many possibilities because there are many many-valued logics; in fact, there are infinitely many!) 
Among the reasons that have been given historically for rejecting the two-valuedness assumption are the beliefs that statements about the future, statements involving vague predicates, or statements about quantum-mechanical properties are not always either true or false. 
Most many-valued logics begin by rejecting the law of excluded middle $\disjunction{\Al}{\negation{\Al}}$, though there are exceptions.\footnote{For example, the most generally accepted logic for quantum mechanics keeps	$\disjunction{\Al}{\negation{\Al}}$ but rejects one of the distribution principles.} 
The number of values ranges from three to various infinite sets. 
The interpretations of the further values vary widely from author to author, as do the motivations for introducing the additional values. 

Emil Post was one of the first to study many-valued logics, but his motivation seems to have been entirely formal. 
The other major founder of many-valued logic was \L{}ukasiewicz. 
He sketched the idea of a many-valued logic in 1920 and published a systematic account in 1930. 
(Both are reprinted in \citealp{Borkowski1970}.) 
Unlike Post, \L{}ukasiewicz introduced three-valued logic for a philosophical reason: to provide a more appropriate representation for the indeterminacy of the future. 
He apparently was led to this both by a historical concern---studying Aristotle's discussion of necessity, particularly
his sea battle example---and by a quite contemporary concern about how to accommodate the indeterminism of modern physics within logic. 
Aristotle's sea battle argument is as follows: 
\begin{cenumerate} 
	\item If there will be a sea battle tomorrow, then necessarily there will be a sea battle tomorrow.
	\item If there will not be a sea battle tomorrow, then necessarily there will not be a sea battle tomorrow.
	\item Either there will or there will not be a sea battle tomorrow.
	\item Therefore, either there will necessarily be a sea battle tomorrow or there will necessarily not be a sea battle tomorrow.
\end{cenumerate}
Aristotle suggested that the third premise---the law of excluded middle, $\disjunction{\Al}{\negation{\Al}}$---should be rejected when $\Al$ is a statement about a future contingency. 
Thus the motivation, if not the details, of many-valued logic is as ancient as the study of logic itself. 
\L{}ukasiewicz developed this idea into a systematic logic. 

In all of his later work, \L{}ukasiewicz used 1 for truth, 0 for falsity, and intermediate values for other truth values. 
Most but not all writers use this convention. 
Of course it is one thing to decide that $\rfrac{1}{2}$ is your third truth value and another thing to give a philosophical explanation of it. 
For \L{}ukasiewicz the intermediate value is ``indeterminate''. 
Given this understanding, the most natural three-valued generalization of the two-valued truth tables is the following, in which negation reverses the truth value.
\begin{center}
	\begin{tabular}{ | c | c |}
		\hline
		$\Al$ & $\negation{\Al}$ \\ \hline
		1 & 0 \\ \hline
		\textonehalf{} & \textonehalf{} \\ \hline
		0 & 1 \\ \hline
	\end{tabular}		
\end{center}
Conjunction takes the minimum value of the conjuncts, while disjunction takes the maximum value of the disjuncts.
\begin{center}
	\begin{tabular}{ | c | c | c | c |}
		\hline
		$\conjunction{\Al}{\Bl}$ & 1 & \textonehalf{} & 0 \\ \hline
		1 & 1 & \textonehalf{} & 0 \\ \hline
		\textonehalf{} & \textonehalf{} & \textonehalf{} & 0 \\ \hline
		0 & 0 & 0 & 0 \\ \hline
	\end{tabular}		
\end{center}
\begin{center}
	\begin{tabular}{ | c | c | c | c |}
		\hline
		$\disjunction{\Al}{\Bl}$ & 1 & \textonehalf{} & 0 \\ \hline
		1 & 1 & 1 & 1 \\ \hline
		\textonehalf{} & 1 & \textonehalf{} & \textonehalf{} \\ \hline
		0 & 1 & \textonehalf{} & 0 \\ \hline
	\end{tabular}		
\end{center}
For example, the conjunction of a true sentence and an indeterminate sentence would seem to be indeterminate. 
It could become true if the indeterminacy were resolved in favor of truth or false if the indeterminacy were resolved in favor of falsity. 

Note that when all the components of a sentence formed from these connectives are assigned value $\rfrac{1}{2}$ the entire sentence has value $\rfrac{1}{2}$. 
If we introduce the conditional as $\disjunction{\negation{\Al}}{\Bl}$, as is often done in two-valued logic, then the conditionals would also have this property and there would be no sentences which are logical truths. 
More specifically, since the identification of the conditional with $\disjunction{\negation{\Al}}{\Bl}$ makes $\horseshoe{\Al}{\Al}$ equivalent to the law of excluded middle, $\horseshoe{\Al}{\Al}$ would not be a logical truth. 

Instead of using that traditional, oft questioned equivalence, \L{}ukasiewicz defined the conditional thus:
\begin{center}
	\begin{tabular}{ | c | c | c | c |}
		\hline
		$\horseshoe{\Al}{\Bl}$ & 1 & \textonehalf{} & 0 \\ \hline
		1 & 1 & \textonehalf{} & 0 \\ \hline
		\textonehalf{} & 1 & 1 & \textonehalf{} \\ \hline
		0 & 1 & 1 & 1 \\ \hline
	\end{tabular}		
\end{center}
One way of describing this table is that the conditional is false only in the case of $\horseshoe{T}{F}$ and is indeterminate only in two cases: $\horseshoe{T}{I}$ and $\horseshoe{I}{F}$. 
A rationale for these choices is that if $\Al$ were true and $\Bl$ indeterminate, then the conditional $\horseshoe{\Al}{\Bl}$ could be true if $\Bl$ were to be true and false if $\Bl$ were to be false. 
The choice of the value 1 when both constituents have value $\rfrac{1}{2}$ is required if $\horseshoe{\Al}{\Al}$ is to be logically true. 

Equivalence can be defined as usual: $\triplebar{\Al}{\Bl}$ iff $\horseshoe{\Al}{\Bl}$ and $\horseshoe{\Bl}{\Al}$.  
The truth table for the biconditional is specified by this definition above. 
We leave it to the reader to give a simpler, more direct explanation of the truth table.
In \L{}ukasiewicz' presentation of his system, he used only negation and the conditional, having noted that $\disjunction{\Al}{\Bl}$ can be defined as $\horseshoe{\parhorseshoe{\Al}{\Bl}}{\Bl}$ and then $\conjunction{\Al}{\Bl}$ can be defined by using the usual DeMorgan's principle.

In two-valued logic we define a sentence to be logically true iff it is true in all models. 
When we have more than two truth values, we must indicate which subset of the values are the designated values, i.e., those that are truth-like. 
Our definition now becomes: $\Al$ is a \idf{logical truth} iff it has a designated value in all models. 

Since \L{}ukasiewicz' motivation was to deny excluded middle, he chose only 1 as a designated value. 
This achieves the purpose of rendering excluded middle not a logical truth. 
It has one somewhat counterintuitive consequence though, which is that under a model in which both constituents are assigned value $\rfrac{1}{2}$, $\conjunction{\Al}{\negation{\Al}}$ has the same truth value as $\disjunction{\Al}{\negation{\Al}}$. 
Issues of the indeterminacy of the future are now generally studied within the framework of tense logic. 
Aristotle's argument is generally regarded as fallacious, but \L{}ukasiewicz' innovations have opened the possibilities for a variety of other systems and ideas. 

\subsection{Finite-valued systems with more than three values}

The \L{}ukasiewicz three-valued generalization can be systematically carried further. 
The $\integer{n}$-valued generalization consists of taking the values $\rfrac{i}{n-1}$ for $0 \leq i \leq n-1$. 
For example, four-valued logic would have the values 0, $\rfrac{1}{3}$, $\rfrac{2}{3}$, and 1. 
Conjunction will take the minimum value of the conjuncts, and disjunction the maximum value of the disjuncts. 
The value of a negation is 1 minus the value of the sentence being negated. 
For the conditional $\horseshoe{\Al}{\Bl}$ we have two cases, using $V(\Al)$ for the value of $\Al$:
\[V(\horseshoe{\Al}{\Bl}) = 
\begin{cases}
1 & \text{ if } V(\Al) \leq V(\Bl) \\
[1 – V(\Al)]  + V($\Bl$) & \text{ otherwise}
\end{cases}\]
In all of the \L{}ukasiewicz systems the only designated value is 1. 
Excluded middle will not be logically true in any of these systems, though in the even-valued systems excluded middle is always truer than the contradiction $\conjunction{\Al}{\negation{\Al}}$. 
Systems with more than one designated value were mentioned by Post, and this variation on \L{}ukasiewicz systems was studied by Slupecki and others. 

Four-valued logic was proposed for modal logic, the values being ``necessarily true'', ``contingently true'', ``contingently false'', and ``necessarily false''.
The \L{}ukasiewicz definitions of the usual connectives can be used and a modal operator added. 
While these truth tables have some uses, they have been superseded by the possible worlds approach to modal logic. We discuss possible world semantics later in this chapter.

\subsection{Infinite-valued systems}

The \L{}ukasiewicz $\integer{n}$-valued generalization can be systematically carried further still.
\L{}ukasiewicz also studied the cases where the set of truth values consists of all rational numbers (fractions) in the interval $[0,1]$ and also where the values consist of all real numbers (infinite decimal expansions) in the same interval. 
Conjunction, disjunction, negation, and the conditional (with the two cases) are the same as for finite-valued systems with more than three values. 
The set of logical truths in the three-valued logic is a subset of those in our traditional logic; if a sentence can be shown to be false in a two-valued model, then that model also works in the three-valued system. 

One question to ask is whether all these values make a difference. 
We already know part of the answer. 
$\disjunction{\Al}{\negation{\Al}}$ is a two-valued logical truth but not a three-valued one. 
But this does not give us an answer for the other systems. 
Before addressing this question, let's generalize our observation about excluded middle. 
If we think about the truth tables for $\NEGATION$, $\WEDGE$, and $\VEE$, we can see that when the input value(s) are $\rfrac{1}{2}$, the output value is always $\rfrac{1}{2}$. 
Thus a simple recursive proof (that we won't bother giving) shows that, in a model in which all atoms receive value $\rfrac{1}{2}$, any sentence without conditionals and biconditionals is not a logical truth, because it receives value $\rfrac{1}{2}$ on that model. 
\begin{THEOREM}{\LnpTC{Lukasiewicz systems}}
	All logical truths of \L{}ukasiewicz systems contain conditionals or biconditionals.
\end{THEOREM}
\noindent{}This means that if we are looking for sentences that discriminate between the $\integer{n}$-valued systems then we need to look at conditionals or biconditionals. 
One candidate for a form of sentence is $\disjunction{\parhorseshoe{\Al}{\Bl}}{\parhorseshoe{\Bl}{\Cl}}$. 
This is a two-valued logical truth but not a three-valued one, because we can assign the values 1, $\rfrac{1}{2}$, and 0 to the respective atoms and give
the sentence a value of $\rfrac{1}{2}$. 
Generalizing, sentences of the form
\begin{center} $\disjunction{\parhorseshoe{\Al_1}{\Al_2}}{\disjunction{\parhorseshoe{\Al_2}{\Al_3}}{\disjunction{\parhorseshoe{\Al_3}{\Al_4}}{\disjunction{\ldots}{\parhorseshoe{\Al _{n}}{\Al _{n+1}}}}}}$ 
\end{center}	
\noindent{}will be logical truths in an n-valued system but not in a system with at least $n+1$ values. 

%\noindent \textbf{Example.} 
%	Show whether the following sentences are three-valued truths. 
%	If so, are they n-valued truths for all finite n? \\

\begin{majorILnc}{\LnpEC{MVLTruthFunctionallyTrue}}
Is the sentence  $\horseshoe{\parconjunction{\Al}{\negation{\Bl}}}{\negation{\parhorseshoe{\Al}{\Bl}}}$ a three-valued truth? 
If so, are they $\integer{n}$-valued truths for all finite $\integer{n}$? 

\noindent{}\emph{Solution.} This is not a logical truth in any system with more than two values. 
If we assign both $\Al$ and $\Bl$ the same intermediate value, then $\negation{\Bl}$ will have an intermediate value and so $\conjunction{\Al}{\negation{\Bl}}$ will have an intermediate value. 
But since $\Al$ and $\Bl$ have the same value, $\horseshoe{\Al}{\Bl}$ will have the value 1 and $\negation{\parhorseshoe{\Al}{\Bl}}$ will have the value 0. 
So, the whole conditional will have an intermediate value, not 1.
\end{majorILnc}
\begin{majorILnc}{\LnpEC{MVLTruthFunctionallyTrueB}}
Is the sentence  $\horseshoe{\parconjunction{\parhorseshoe{\Al}{\Cl}}{\parhorseshoe{\Bl}{\Cl}}}{\parhorseshoe{\pardisjunction{\Al}{\Bl}}{\Cl}}$ a three-valued truth? 
If so, are they $\integer{n}$-valued truths for all finite $\integer{n}$? 

\noindent{}\emph{Solution.} We will give an argument that this sentence is true for all finite-valued logics. 
If the values of $\Al$ and $\Bl$ are both less than or equal to that of $\Cl$, then the right conditional has value 1 and consequently the whole sentence does too. 
Thus it can only be less than 1 if $\Al$ or $\Bl$ has value greater than $\Cl$. 
Suppose $\Al$ is greater than $\Cl$ and greater than or equal to $\Bl$. 
Then Max($\Al$, $\Bl$) = $\Al$ and the right conditional has the same value as $\horseshoe{\Al}{\Cl}$.  
But since $\Al > \Bl$, $1-\Al < 1-\Bl$, and so the value of the left conditional will also be the value of $\horseshoe{\Al}{\Cl}$ because that is the minimum value on that side.
\end{majorILnc}

%%%%%%%%%%%%%%%%%%%%%%%%%%%%%%%%%%%%%%%%%%%%%%%%%%
\section{Modal Sentential Logic}\label{modal sentential logic}
%%%%%%%%%%%%%%%%%%%%%%%%%%%%%%%%%%%%%%%%%%%%%%%%%%

We can think of sentential logic as the logic of truth functional operators and quantificational logic (at least, that part which extends sentential logic) as the logic of quantifiers and predicates. 
But there is a lot that's left out by this: possibility and necessity, deontic concepts like permissibility and obligation, doxastic concepts like belief and knowledge, and temporal concepts like past and future, just to name a few. 
Logics which attempt to capture these concepts are called modal logics. 
Sometimes the term \mention{modal logic} is just meant to pick out the logic of possibility and necessity, other times it's meant to pick out the broader class of modal logics just mentioned.
In this section we give a very brief introduction to the logic of possibility and necessity. 

\subsection{The Language \MGSL{}}
The language of modal sentential logic, \MGSL{}, is an extension of \GSL{}. 
We add to \GSL{} two 1-place logical connectives: $\BOX$ and $\DIAMOND$.  
Intuitively, $\BOX\CAPPHI$ says that $\CAPPHI$ is necessarily true, while $\DIAMOND\CAPPHI$ says that $\CAPPHI$ is possibly true. 
\begin{majorILnc}{\LnpDC{Recursive definition of Sentences of MGSL}} The \nidf{sentences} \underdf{of \MGSL{}}{sentence} are given by the following recursive definition:
\begin{description}
\item[Base Clause:] Every sentence letter (def. \ref{Basic Symbols of GSL}) is a sentence.
\item[Generating Clauses:] \hfill
\begin{cenumerate}
\item If $\CAPPHI$ is a sentence, then so are $\negation{\CAPPHI}$, $\BOX\CAPPHI$ and $\DIAMOND\CAPPHI$.
\item If $\CAPPHI$ and $\CAPTHETA$ are sentences, then so are $\parhorseshoe{\CAPPHI}{\CAPTHETA}$ and $\partriplebar{\CAPPHI}{\CAPTHETA}$.
\item If all of $\CAPPHI_1,\CAPPHI_2,\CAPPHI_3,\CAPPHI_4,\ldots,\CAPPHI_{\integer{n}}$ are sentences (the list must include at least two sentences and be finite), then so are $\parconjunction{\CAPPHI_1}{\conjunction{\CAPPHI_2}{\conjunction{\CAPPHI_3}{\conjunction{\CAPPHI_4}{\conjunction{\ldots}{\CAPPHI_{\integer{n}}}}}}}$ and $\pardisjunction{\CAPPHI_1}{\disjunction{\CAPPHI_2}{\disjunction{\CAPPHI_3}{\disjunction{\CAPPHI_4}{\disjunction{\ldots}{\CAPPHI_{\integer{n}}}}}}}$.
\end{cenumerate}
\item[Closure Clause:] A sequence of symbols is a \MGSL{} sentence if and only if its being a sentence follows from the previous two clauses.
\end{description}
\end{majorILnc}
\noindent{}Of course, as with \GSL{}, sentences of \MGSL{} only count as either true or false relative to a model.

%There are various deduction systems of modal logic (i.e., of \MGSL{}) and various distinct concepts of interpretations (for \MGSL{}). 
%The deductive system we will study is called \SF{}, and we'll use interpretations so that \SF{} is complete. (To be clear: S5 is a derivation system for MGSL, just like \GSD{} is a derivation system for GSL.) Strictly speaking what we are going to define are (a) interpretations for MGSL (just like we defined interpretations for \GSL{} and \GQL{}) and (b) a definition of truth for when a sentence of MGSL is true in one of these interpretations, and (c) we want to make sure that a sentence of MGSL is derivable in S5 iff it's true in all such interpretations. For brevity we will just speak of ``interpretations'' and ``truth'' with it understood that we mean interpretations for MGSL and truth in such interpretations. 

\subsection{Truth, Logical Truth, and Entailment in \MGSL{}}

Like the models for \GQL{} (and unlike the many-valued models for \GSL{}), models for \MGSL{} can be thought of as extensions of the (2-valued) models of \GSL{} (recall definition \pmvref{Definition of GSL interpretation}).
\begin{majorILnc}{\LnpDC{MGSLInterpretation}}
A model $\IntA$ for \MGSL{} is an ordered triple $\langle\world{w}_0,\integer{W},\integer{V}\rangle$ where:
\begin{cenumerate}
\item $\integer{W}$ is a set, the elements of which are thought of as ``possible worlds''
\item $\world{w}_0\in\integer{W}$, where $\world{w}_0$ plays the role of ``the actual world''
\item $\integer{V}$ is a function that assigns a subset of $\integer{W}$ to each sentence letter of \MGSL{}.
\end{cenumerate}
\end{majorILnc}
\noindent{}Elements of $\integer{W}$, ``possible worlds'', will be denoted by $\world{w},\world{w}_0,\world{w}_1,\world{w}_2$ and so on. 
The subset $\integer{V}(\CAPPHI)$ of $\integer{W}$ assigned to a sentence letter $\CAPPHI$ can be the empty set $\emptyset$, a non-empty proper subset of $\integer{W}$, or all of $\integer{W}$ itself. 
Intuitively, the set $\integer{V}(\CAPPHI)$ is the set of possible worlds in which $\CAPPHI$ is true.  And $\integer{W}-\integer{V}(\CAPPHI)$, those worlds not in $\integer{V}(\CAPPHI)$, are worlds in which $\CAPPHI$ is false.  (We use the \mention{$-$} symbol to indicate the \mention{complement} of a set.)\footnote{For example, relative to some domain, $\{ 1, 2, 3, 4\}$, the complement of the set $\{1, 2\}$ is $\{3, 4\}$.  We can indicate the latter with the following notation: $\{ 1, 2, 3, 4\}-\{1, 2\}$.}
Thus the function $\integer{V}$ in a \MGSL{} model is like a \GSL{} model, except that it assigns truth values to sentence letters in a possible world. 
Just as before, we can extend the notion of truth in a possible world $\world{w}$ on a model $\IntA$.
\begin{majorILnc}{\LnpDC{MGSLTruth}}
The following clauses define a function $\integer{V}^*$ which extends $\integer{V}$ to all sentences of \MGSL{}.
\begin{cenumerate}
\item If $\CAPPHI$ is a sentence letter, $\integer{V}^*(\CAPPHI)=\integer{V}(\CAPPHI)$.
\item If $\CAPPHI$ is $\negation{\CAPPSI}$, then $\integer{V}^*(\CAPPHI)=W-\integer{V}(\CAPPSI)$, i.e. $\integer{V}^*(\CAPPHI)$ is the set of worlds in $\integer{W}$ not in $\integer{V}(\CAPPSI)$.
\item If $\CAPPHI$ is $\parconjunction{\CAPPSI_1}{\conjunction{\ldots}{\CAPPSI_{\integer{n}}}}$, then $\integer{V}^*(\CAPPHI)=\integer{V}^*(\CAPPSI_1)\cap\ldots\cap\integer{V}^*(\CAPPSI_{\integer{n}})$.
\item If $\CAPPHI$ is $\pardisjunction{\CAPPSI_1}{\disjunction{\ldots}{\CAPPSI_{\integer{n}}}}$, then $\integer{V}^*(\CAPPHI)=\integer{V}^*(\CAPPSI_1)\cup\ldots\cup\integer{V}^*(\CAPPSI_{\integer{n}})$.
\item If $\CAPPHI$ is $\parhorseshoe{\CAPPSI}{\CAPTHETA}$, then $\integer{V}^*(\CAPPHI)=\integer{V}^*(\negation{\CAPPSI})\cup\integer{V}^*(\CAPTHETA)$.
\item If $\CAPPHI$ is $\partriplebar{\CAPPSI}{\CAPTHETA}$, then $\integer{V}^*(\CAPPHI)=(\integer{V}^*(\CAPPSI)\cap\integer{V}^*(\CAPTHETA))\cup(\integer{V}^*(\negation{\CAPPSI})\cap\integer{V}^*(\negation{\CAPTHETA}))$.
\item If $\CAPPHI$ is $\BOX\CAPPSI$, then $\integer{V}^*(\CAPPHI)=W$ \Iff $\integer{V}^*(\CAPPSI)=W$, otherwise $\integer{V}^*(\CAPPHI)=\emptyset$.
\item If $\CAPPHI$ is $\DIAMOND\CAPPSI$, then $\integer{V}^*(\CAPPHI)=W$ \Iff $\integer{V}^*(\CAPPSI)\neq\emptyset$, otherwise $\integer{V}^*(\CAPPHI)=\emptyset$.
\end{cenumerate}
\end{majorILnc}
\noindent{}For a given model $\IntA$, we call this function $\integer{V}^*$ the \idf{valuation function} of $\IntA$. 
From here there is a straightforward way to define truth in a world in a model:
\begin{majorILnc}{\LnpDC{MGSLTruthB}}
A sentence $\CAPPHI$ of \MGSL{} is \nidf{true in world $\world{w}$} in model $\IntA=\langle\world{w}_0,\integer{W},\integer{V}\rangle$ (where $\world{w}\in\integer{W}$) \Iff $\world{w}\in\integer{V}^*(\CAPPHI)$.
\end{majorILnc}
\noindent{}Although there isn't as much intuitive significance to it, we can define truth in a model:
\begin{majorILnc}{\LnpDC{MGSLTruthC}}
A sentence $\CAPPHI$ of \MGSL{} is \nidf{true} in model $\IntA=\langle\world{w}_0,\integer{W},\integer{V}\rangle$ \Iff $\integer{V}^*(\CAPPHI)=\integer{W}$.
\end{majorILnc}
\noindent{}Perhaps it's best not to use ``truth'' here, but instead something like ``truth everywhere''; so we might say that $\CAPPHI$ is true everywhere in $\IntA$ \Iff $\integer{V}^*(\CAPPHI)=\integer{W}$, for $\integer{V}^*$ the valuation function of $\IntA$.
The significant concepts are (1) truth in a world in a model (def. \ref{MGSLTruthB}), and (2) truth at all worlds in a model (def. \ref{MGSLTruthC}).

The concepts of logical truth, logical falsity, and logical indeterminacy can be adapted to \MGSL{} as well:
\begin{majorILnc}{\LnpDC{MGSLLogicalTruth}}
$\CAPPHI$ is a \nidf{modal truth}\index{truth!modal|textbf} \Iff for all models $\IntA=\langle\world{w}_0,\integer{W},\integer{V}\rangle$, $\integer{V}^*(\CAPPHI)=\integer{W}$.
\end{majorILnc}
\begin{majorILnc}{\LnpDC{MGSLLogicalFalsehood}}
$\CAPPHI$ is a \nidf{modal falsehood}\index{falsehood!modal|textbf} \Iff for all models $\IntA=\langle\world{w}_0,\integer{W},\integer{V}\rangle$, $\integer{V}^*(\CAPPHI)=\emptyset$.
\end{majorILnc}
\begin{majorILnc}{\LnpDC{MGSLLogicalIndeterminant}}
$\CAPPHI$ is \nidf{modally contingent}\index{indeterminate!modal|textbf} \Iff there's a model $\IntA=\langle\world{w}_0,\integer{W},\integer{V}\rangle$ such that $\integer{V}^*(\CAPPHI)\neq\emptyset$ and there's a model $\IntA=\langle\world{w}_0,\integer{W},\integer{V}\rangle$ such that $\integer{V}^*(\CAPPHI)\neq\integer{W}$.  (Note: It can be the same model for both.)
\end{majorILnc}
\noindent{}Alternatively (but equivalently) put, a sentence $\CAPPHI$ is a modal truth \Iff it's true everywhere on all models $\IntA$; i.e. \Iff for every world $\world{w}$ in every model $\IntA$, $\CAPPHI$ is true in $\world{w}$ in $\IntA$. 
Similarly, we can alternatively say that a sentence $\CAPPHI$ is a modal falsity \Iff for every world $\world{w}$ in every model $\IntA$, $\CAPPHI$ is false in $\world{w}$ in $\IntA$.

Finally, we also can define when a set $\Delta$ of \MGSL{} sentences entails a sentence $\CAPPHI$.
\begin{majorILnc}{\LnpDC{MGSLEntailment}}
A sentence $\CAPPHI$ is \nidf{entailed} by\index{entailment} a set of sentence $\Delta$ iff, for every model $\IntA=\langle\world{w}_0,\integer{W},\integer{V}\rangle$ and world $\world{w}\in\integer{W}$, whenever every sentence in $\Delta$ is true in $\world{w}$ in $\IntA$, then $\CAPPHI$ is also true in $\world{w}$ in $\IntA$; i.e. iff, for every model $\IntA=\langle\world{w}_0,\integer{W},\integer{V}\rangle$ and world $\world{w}\in\integer{W}$,  $\bigcap_{\CAPPSI\in\Delta}\integer{V}^*(\CAPPSI)\subseteq\integer{V}^*(\CAPPHI)$.
\end{majorILnc}

\begin{majorILnc}{\LnpEC{MGSLTruthExampleA}}
$\horseshoe{\DIAMOND\Al}{\BOX\DIAMOND\Al}$ is a modal truth. 
\end{majorILnc}
\begin{PROOF}
Note that $\world{w}\in\integer{V}^*(\negation{\CAPPSI})\cup\integer{V}^*(\CAPTHETA)$, for $\integer{V}^*$ the valuation function of some model $\IntA$ and $\world{w}$ a world in the set $\integer{W}$ of $\IntA$, iff if $\world{w}\in\integer{V}^*(\CAPPSI)$, then $\world{w}\in\integer{V}^*(\CAPTHETA)$. 
So, suppose $\DIAMOND\Al$ is true at some $\world{w}$ in some model $\IntA$, i.e. suppose that $\world{w}\in\integer{V}^*(\DIAMOND\Al)$ for $\integer{V}^*$ the valuation function of $\IntA$. 
By (8) of definition \ref{MGSLTruth}, either $\integer{V}^*(\DIAMOND\Al)=\integer{W}$ or $\integer{V}^*(\DIAMOND\Al)=\emptyset$. 
Since $\world{w}\in\integer{V}^*(\DIAMOND\Al)$, $\integer{V}^*(\DIAMOND\Al)\neq\emptyset$. 
So, $\integer{V}^*(\DIAMOND\Al)=\integer{W}$. 
Now Consider $\BOX\DIAMOND\Al$. 
Since $\integer{V}^*(\DIAMOND\Al)=\integer{W}$, by (7) of definition \ref{MGSLTruth} it follows that $\integer{V}^*(\BOX\DIAMOND\Al)=\integer{W}$. 
So, $\BOX\DIAMOND\Al$ is true at every element of $\integer{W}$, including $\world{w}$. 
Therefore, $\world{w}\in\integer{V}^*(\BOX\DIAMOND\Al)$. 
Since $\world{w}$ was arbitrary, this means that for $\IntA$, $\integer{V}^*(\negation{\CAPPSI})\cup\integer{V}^*(\CAPTHETA)=\integer{W}$. 
Since $\IntA$ was arbitrary, this means that $\integer{V}^*(\negation{\CAPPSI})\cup\integer{V}^*(\CAPTHETA)=\integer{W}$ for all $\IntA$.
\end{PROOF}

\begin{majorILnc}{\LnpEC{MGSLTruthExampleB}}
There are sentences $\CAPPHI$ and $\CAPPSI$ for which $\horseshoe{\BOX\pardisjunction{\CAPPHI}{\CAPPSI}}{\pardisjunction{\BOX\CAPPHI}{\BOX\CAPPSI}}$ is not a modal truth. 
\end{majorILnc}
\begin{PROOF}
Consider a model $\IntA=\langle\world{w}_0,\integer{W},\integer{V}\rangle$ with the set $W=\{\world{w}_0,\world{w}_1\}$. 
Let $\CAPPHI=\Bl$ and let $\CAPPSI=\;\negation{\Bl}$. 
Let $\integer{V}(\Bl)=\{w_0\}$. 
By (2) in definition \ref{MGSLTruth}, $\integer{V}^*(\negation{\Bl})=\{w_1\}$. 
By (4) in definition \ref{MGSLTruth}, $\integer{V}^*(\disjunction{\Bl}{\negation{\Bl}})=\{w_0,w_1\}$; hence by (7) of definition \ref{MGSLTruth} $\integer{V}^*(\BOX\pardisjunction{\Bl}{\negation{\Bl}})=\{\world{w}_0,\world{w}_1\}$. 
Thus, $\BOX\pardisjunction{\Bl}{\negation{\Bl}}$ is true at each $\world{w}$ in $W$, including world $\world{w}_0$. 
Consider then $\pardisjunction{\BOX\Bl}{\BOX\negation{\Bl}}$. 
By (7) of definition \ref{MGSLTruth}, $\integer{V}^*(\BOX\Bl)=\emptyset$ since $\integer{V}^*(\Bl)=\{w_0\}\neq\integer{W}$. 
Again, $\integer{V}^*(\BOX\negation{\Bl})=\emptyset$ since $\integer{V}^*(\negation{\Bl})=\{\world{w}_1\}\neq\integer{W}$. 
So, by (4) in definition \ref{MGSLTruth}, $\integer{V}^*(\disjunction{\BOX\Bl}{\BOX\negation{\Bl}})=\emptyset$. Thus, $\pardisjunction{\BOX\Bl}{\BOX\negation{\Bl}}$ is false at each $\world{w}$ in $\integer{W}$, including world $\world{w}_0$. 
Therefore, $\horseshoe{\BOX\pardisjunction{\Bl}{\negation{\Bl}}}{\pardisjunction{\BOX\Bl}{\BOX\negation{\Bl}}}$ is not true in $\world{w}_1$ in $\IntA$; so it is not a modal truth. 
\end{PROOF}

You should see a strong similarity here with the fact that $\horseshoe{\universal{\variable{x}}\pardisjunction{\Bp{\variable{x}}}{\negation{\Bp{\variable{x}}}}}{\pardisjunction{\universal{\variable{x}}\Bp{\variable{x}}}{\universal{\variable{x}}\negation{\Bp{\variable{x}}}}}$ is not a logical truth.

\begin{majorILnc}{\LnpEC{MGSLTruthExampleC}}
All of the following are modal truths, for any substitution of sentences $\CAPPHI$ and $\CAPPSI$. 
We leave the proofs as exercises to the reader.
\begin{multicols}{2}
\begin{cenumerate}
\item $\horseshoe{\BOX\BOX\CAPPHI}{\BOX\CAPPHI}$
\item $\horseshoe{\BOX\DIAMOND\CAPPHI}{\DIAMOND\CAPPHI}$
\item $\horseshoe{\DIAMOND\CAPPHI}{\DIAMOND\DIAMOND\CAPPHI}$
\item $\horseshoe{\BOX\CAPPHI}{\DIAMOND\BOX\CAPPHI}$
\item $\triplebar{\negation{\DIAMOND\CAPPHI}}{\BOX\negation{\CAPPHI}}$
\item $\triplebar{\DIAMOND\negation{\CAPPHI}}{\negation{\BOX\CAPPHI}}$
\item $\horseshoe{\BOX\parhorseshoe{\CAPPHI}{\CAPPSI}}{\parhorseshoe{\BOX\CAPPHI}{\BOX\CAPPSI}}$
\item $\horseshoe{\pardisjunction{\BOX\CAPPHI}{\BOX\CAPPSI}}{\BOX\pardisjunction{\CAPPHI}{\CAPPSI}}$
\item $\triplebar{\pardisjunction{\DIAMOND\CAPPHI}{\DIAMOND\CAPPSI}}{\DIAMOND\pardisjunction{\CAPPHI}{\CAPPSI}}$
\item $\triplebar{\BOX\parconjunction{\CAPPHI}{\CAPPSI}}{\parconjunction{\BOX\CAPPHI}{\BOX\CAPPSI}}$
\item $\horseshoe{\DIAMOND\parconjunction{\CAPPHI}{\CAPPSI}}{\parconjunction{\DIAMOND\CAPPHI}{\DIAMOND\CAPPSI}}$
\item $\triplebar{\negation{\DIAMOND\negation{\CAPPHI}}}{\BOX\CAPPHI}$
\item $\horseshoe{\BOX\parhorseshoe{\CAPPHI}{\CAPPSI}}{\parhorseshoe{\DIAMOND\CAPPHI}{\DIAMOND\CAPPSI}}$
\item $\horseshoe{\DIAMOND\parhorseshoe{\CAPPHI}{\CAPPSI}}{\parhorseshoe{\BOX\CAPPHI}{\DIAMOND\CAPPSI}}$
\item $\horseshoe{\negation{\DIAMOND\CAPPHI}}{\BOX\parhorseshoe{\CAPPHI}{\CAPPSI}}$
\item $\horseshoe{\BOX\partriplebar{\CAPPHI}{\CAPPSI}}{\partriplebar{\BOX\CAPPHI}{\BOX\CAPPSI}}$
\end{cenumerate}
\end{multicols}
\end{majorILnc}
One of the most distinctive features of the models we're using here (def. \ref{MGSLInterpretation}) and the definitions of truth (def. \ref{MGSLTruth}) and modal truth (def. \ref{MGSLLogicalTruth}) that come with them is that only the innermost modality of a string of modalities matters. 
This follows from the iteration of the following basic facts.
\begin{majorILnc}{\LnpTC{MGSLSFiveTheorem}} The following are modal truths, for every sentence $\CAPPHI$.
\begin{multicols}{2}
\begin{cenumerate}
\item $\triplebar{\BOX\BOX\CAPPHI}{\BOX\CAPPHI}$
\item $\triplebar{\BOX\DIAMOND\CAPPHI}{\DIAMOND\CAPPHI}$
\item $\triplebar{\DIAMOND\CAPPHI}{\DIAMOND\DIAMOND\CAPPHI}$
\item $\triplebar{\BOX\CAPPHI}{\DIAMOND\BOX\CAPPHI}$
\end{cenumerate}
\end{multicols}
\end{majorILnc}
\noindent{}Again we leave the rigorous proofs to the reader.
But, from the intuitive semantic point of view this is because the modalities $\BOX$ and $\DIAMOND$ behave like quantifiers over a fixed domain with only one variable that they can quantify. 
Thus any quantification after the first is vacuous. 
There are other ways of defining modal models of in which (roughly) the domain of quantification over worlds isn't always the same. The schemas in theorem \ref{MGSLSFiveTheorem} don't represent modal truths in these modal logics. We do not discuss alternative modal model definitions in this text.

\subsection{Derivations in \SF{}}
We would like a derivation system which is sound and complete for the semantics just defined in the last section, i.e. so that a sentence of \MGSL{} is a modal truth (def. \ref{MGSLLogicalTruth}) \Iff it's derivable in that system. 
We get such a system, usually called \SF{}, by adding introduction and elimination rules for $\BOX$ and $\DIAMOND$ to \GSD{}. 
%\begin{table}[!ht]
\renewcommand{\arraystretch}{1.5}
%\begin{center}
\begin{longtable}[c]{ p{1in} l l } %p{2.2in} p{2in}
\toprule
\textbf{Name} & \textbf{Given} & \textbf{May Add} \\ 
\midrule
\endfirsthead
\multicolumn{3}{c}{\emph{Continued from Previous Page}}\\
\toprule
\textbf{Name} & \textbf{Given} & \textbf{May Add} \\ 
\midrule
\endhead
\bottomrule
\caption{Basic Rules of \SF{}}\\[-.15in]
\multicolumn{3}{c}{\emph{Continued next Page}}\\
\endfoot
\bottomrule
\caption{Basic Rules of \SF{}}\\%
\endlastfoot%
\label{SF}%
\Rule{$\BOX\!$-Elim} & $\BOX\CAPPHI$ & $\CAPPHI$ \\
\Rule{$\BOX\!$-Intro} & $\CAPPHI$ (*) & $\BOX\CAPPHI$ \\
\Rule{$\DIAMOND\!$-Elim} & $\DIAMOND\CAPPHI$, $\horseshoe{\CAPPHI}{\CAPPSI}$ (*), (**) & $\CAPPSI$ \\
\Rule{$\DIAMOND\!$-Intro} &  $\CAPPHI$ & $\DIAMOND\CAPPHI$ \\
%\bottomrule
\end{longtable}
\index{derivation!rule!$S5$}\index{$S5$}
\index{derivation!rule!introduction}\index{introduction rule}
\index{derivation!rule!elimination}\index{elimination rule}
\noindent{}There are two restrictions to the rules: (*) in \Rule{$\BOX\!$-Intro} and \Rule{$\DIAMOND\!$-Elim}, the rule can only be applied if all the open assumptions have modal prefixes.
(**) In \Rule{$\DIAMOND\!$-Elim}, the rule can only be applied if $\CAPPSI$ has a modal prefix.
\begin{majorILnc}{\LnpDC{ModalPreflixDef}}
A sentence $\CAPPHI$ has a \df{modal prefix} \Iff
\begin{cenumerate}
\item it begins with $\BOX$;
\item it begins with $\DIAMOND$; or
\item it is of the form $\negation{\CAPPHI}$, $\negation{\negation{\CAPPHI}}$, $\ldots$ where $\CAPPHI$ is type (1) or (2).
\end{cenumerate}
\end{majorILnc}
\begin{majorILnc}{\LnpEC{S5ExampleA}}
As they are all modal truths, every instance of the schemas in example \mvref{MGSLTruthExampleC} and theorem \mvref{MGSLSFiveTheorem} can be derived in \SF{}.%\footnote{5--16 are theorems of all normal modal deduction systems (where ``normal'' is a technical term), while 1--4 are not theorems in all normal modal deduction systems, but only those in which the ``T'' axiom ($\horseshoe{\BOX\CAPPHI}{\CAPPHI}$) holds, i.e. only in those systems that can be characterized by reflective frames.}  
\end{majorILnc}

Just as before we can add shortcut rules to \SF{}. 
These we call \niidf{modal negation} rules, similar to the quantification negation rules (table \pmvref{GQDplus}).
It can be shown that anything we can prove using \SF{} and the following shortcut rules can be proved in \SF{} alone.
%\begin{table}[!ht]
\renewcommand{\arraystretch}{1.5}
%\begin{center}
\begin{longtable}[c]{ p{1in} l l } %p{2.2in} p{2in}
\toprule
\textbf{Name} & \textbf{Given} & \textbf{May Add} \\ 
\midrule
\endfirsthead
\multicolumn{3}{c}{\emph{Continued from Previous Page}}\\
\toprule
\textbf{Name} & \textbf{Given} & \textbf{May Add} \\ 
\midrule
\endhead
\bottomrule
\caption{Modal Negation Shortcut Rules}\\[-.15in]
\multicolumn{3}{c}{\emph{Continued next Page}}\\
\endfoot
\bottomrule
\caption{Modal Negation Shortcut Rules}\\%
\endlastfoot%
\label{SFMN}%
\Rule{MN} & $\negation{\BOX\CAPPHI}$ & $\DIAMOND\negation{\CAPPHI}$ \\
 & $\negation{\DIAMOND\CAPPHI}$ & $\BOX\negation{\CAPPHI}$ \\
 & $\negation{\DIAMOND\negation{\CAPPHI}}$ & $\BOX\CAPPHI$ \\
 &  $\negation{\BOX\negation{\CAPPHI}}$ & $\DIAMOND\CAPPHI$ \\
%\bottomrule
\end{longtable}
\index{derivation!rule!shortcut}

Proving soundness for \SF{} is very similar to proving it for \GSD{}. 
In fact, we start with the proof for \GSD{} and add four more parts: one establishing the soundness of each new modal rule. 
Proving completeness for \SF{} requires a more complicated modification of the methods for \GSD{}. 
For logically valid sentences we still need only provide a derivation, but for invalid sentences we must construct a model with various worlds and a model falsifying the sentence.

\subsection{History of Modal Logic}

It's worth discussing some historical background. 
(Part of this overview is drawn from Roberta Ballarin's SEP entry \citeyearpar{Ballarin2010}; the reader should also consult \citealp{Goldblatt2006}.)
Just as with sentential and quantificational logic, work on modal logic started as work on deduction systems without any semantics, and started with sentential logic without quantification.
C.I. Lewis is well known for his early work in the 1910's on derivation systems for sentential modal logic. 
In his \citeyearpar{Lewis1932}, coauthored with Cooper Langford, Lewis lays out his famous systems \SO{}--\SF{}.
(\SF{}, still of much interest today, is the system we will focus on here.)
It wasn't until Ruth Barcan Marcus \citeyearpar{Marcus1946,Marcus1946b,Marcus1947} and to a lesser extent Rudolf Carnap \citeyearpar{Carnap1946,Carnap1947} that systems were developed for quantificational modal logic (see also Marcus's \citeyearpar{Marcus1993}).

While Marcus and Carnap did work on semantics for quantified modal logic (with Carnap drawing on Leibniz's conception of a ``possible world''), it wasn't until the work of Stig Kanger \citeyearpar{Kanger1957}, Richard Montague \citeyearpar{Montague1960}, Jaakko Hintikka \citeyearpar{Hintikka1961}, A.N. Prior \citeyearpar{Prior1957}, and (especially) Saul Kripke \citeyearpar{Kripke1959,Kripke1963,Kripke1963b,Kripke1965} that modern looking semantics (for both sentential and quantificational modal logic) were devised.\footnote{This 
	claim isn't quite true: matrix, or algebraic, semantics for modal logics have a long history, and they surely count as ``modern looking''.} 
These are often called relational, or Kripke, semantics\index{Kripke semantics|see{relational semantics}}\index{relational semantics} and are what we'll look at here (at first in simplified form), although only for sentential modal logic.
Like quantificational logic, there are alternative semantics. 
Richard Montague \citeyearpar{Montague1970} and Dana Scott \citeyearpar{Scott1970} independently developed what's typically called neighborhood semantics\index{neighborhood semantics} \citep{ArloCosta2006}, while matrix (or algebraic) semantics\index{matrix semantics}\index{algebraic semantics|see{matrix semantics}} have a long history that goes back before the development of relational semantics (see \citealp[ch.~3]{Cocchiarella2008modal} for a textbook treatment).
We will not bring these up at all, instead focusing on the basics of relational semantics.

For those readers looking to pursue modal logic further, we recommend the following three textbooks (listed from most basic to most advanced): \citep{Beall2003}, \citep{Hughes1996}, and \citep{Cocchiarella2008modal}. 
For handbook-style treatments, see \citep{Cresswell2001}, \citep{Bull2001}, \citep{Zakharyaschev2001}, and \citep{Garson2001}.

%%%%%%%%%%%%%%%%%%%%%%%%%%%%%%%%%%%%%%%%%%%%%%%%%%
\section{Quantifier Logic with Identity}\label{Sec:Quantifier Logic with Identity}
%%%%%%%%%%%%%%%%%%%%%%%%%%%%%%%%%%%%%%%%%%%%%%%%%%

\subsection{Introduction}
%As the following theorem says, it is a fact that 
We aren't able to say in \GQL{} that one object is identical to another, which is often an important fact. We address this limitation of \GSL{} by extending the language. 

%\begin{THEOREM}{\LnpTC{GQLIdentityTheorem}}
%There exists no (possibly countably infinite) set of formulas $\Delta$ of \GQL{} each of which at most has the variables $\ALPHA$ and $\BETA$ free such that:
%\begin{quote}
%For any two constants $\variable{t}$ and $\variable{s}$, if $\Delta^*$ is the set of sentences got by substituting $\variable{t}$ for all occurrences of $\ALPHA$ and $\variable{s}$ for all occurrences of $\BETA$ in every sentence of $\Delta$, then: for all models $\IntA$, $\IntA$ makes every sentence of $\Delta^*$ true iff $\IntA(\variable{t})=\IntA(\variable{s})$.
%\end{quote}
%\end{THEOREM}
%\noindent{}This theorem is actually another result that follows from the Strong Method discussed in section \mvref{Sec:Proving Strong Completeness}. 
%Specifically, it follows from the the basic result:
%\begin{THEOREM}{\LnpTC{IdentityLemma}}
%If $\Delta$ is (at most) a countably infinite set of \GQL{} sentences that's consistent (i.e., there's at least one model $\IntA$ that makes every sentence in $\Delta$ true) and the constants $\variable{t}$ and $\variable{s}$ each appear at least once in one of the sentences of $\Delta$, then there's a model $\IntA$ which makes every sentence in $\Delta$ true such that $\IntA(\variable{t})\neq\IntA(\variable{s})$.
%\end{THEOREM}
%\begin{PROOF}
%Assume that $\Delta$ is (at most) a countably infinite set of \GQL{} sentences that's consistent and the constants $\variable{t}$ and $\variable{s}$ each appear at least once in one of the sentences of $\Delta$.
%Since $\Delta$ is at most countably infinite, we can apply the strong method to it. 
%Since $\Delta$ is consistent, the strong method will not halt in a contradiction; so (by theorem \pmvref{MethodSLemmaC}) we will be able to construct a model $\IntA$ which makes all the sentences in the sequence produced by the strong method (which includes all of $\Delta$) true. 
%By examining the method of constructing this model given in section \ref{The Method Section}, it is clear that $\variable{t}$ and $\variable{s}$ are each assigned to different objects in the universe. 
%(In fact, every constant that appears is assigned to a distinct object from the universe.)
%\end{PROOF}
%\begin{PROOFOF}{Thm. \ref{GQLIdentityTheorem}}
%Consider any (possibly countably infinite) set of formulas $\Delta$ of \GQL{} each of which at most has the variables $\ALPHA$ and $\BETA$ free and which satisfies.
%By theorem \ref{IdentityLemma}, for any constants $\variable{t}$ and $\variable{s}$, the set $\Delta^*$ got by replacing $\ALPHA$ and $\BETA$ with $\variable{t}$ and $\variable{s}$, respectively, has some model such that $\IntA(\variable{t})\neq\IntA(\variable{s})$. 
%Hence, for each of these sets $\Delta^*$, it's not the case that for all models $\IntA$, $\IntA$ makes every sentence of $\Delta^*$ true iff $\IntA(\variable{t})=\IntA(\variable{s})$.
%\end{PROOFOF}


%(The reader might note that the proof of theorem \ref{IdentityLemma} appealed to the method used to construct interpretations in the proof of completeness.
%Hence if there's some way to extend \GQD{} to a derivation system for the extension of \GQL{} which fixes that problem, any )

Before we give this extension, it's worth mentioning a few motivations. 
First, say we have two constants $\variable{t}$ and $\variable{s}$. 
It turns out that there's no sentence $\CAPPHI$ of \GQL{} (or even set of sentences of \GQL{}) which is true on all and only the models $\IntA$ where $\IntA(\variable{t})=\IntA(\variable{s})$. 
If we translate English names as constants in \GQL{}, then translations of the following sentences
\begin{menumerate}
\item\label{CiceroTully} Cicero is Tully.
\item\label{Superman} Clark Kent is Superman.
\item\label{MTSC} Mark Twain is Samuel Clemens.
\end{menumerate} 
will allow for models where (say) Cicero is not Tully, or Clark Kent is not Superman.
Perhaps more importantly, this means that there's no such translation of \ref{MTSC}, along with a translation of 
\begin{menumerate}
\item Mark Twain wrote \emph{The Adventures of Tom Sawyer}.
\end{menumerate}
which entails a translation of 
\begin{menumerate}
\item Samuel Clemens wrote \emph{The Adventures of Tom Sawyer}.
\end{menumerate}
(We can get around this difficulty if we instead translate names in English as predicates of \GQL{} that have only one member.  This is a somewhat unintuitive work-around, however.)

Next, say we want a sentence $\CAPPHI$ (or set of sentences $\Delta$) which contains a predicate $\PP$ and which is satisfied by all and only models $\IntA$ such that $\IntA(\PP)$ contains only a single element. 
We leave it to readers to convince themselves that there are no sentences like this in \GQL{}.
The point generalizes to any size set: for any number $\integer{n}$, there is no sentence (or set of sentences) of \GQL{} which is satisfied by all and only models $\IntA$ such that $\IntA(\PP)$ contains exactly $\integer{n}$ elements. 
(This holds when $\integer{n}$ is infinite too.)
But, if we extend \GQL{} to capture identity claims, then we will be able to find such sentences. 

Thinking about translations of English sentences again, this inability to capture cardinality claims suggests that \GQL{} can't provide satisfactory translations for English sentences like:
\begin{menumerate}
\item There exists exactly one bad apple. 
\item At least five different philosophers tried to change that light bulb.
\item There exists an infinite number of things. 
\end{menumerate}
So by extending \GQL{} to capture identity claims we'll also be able to capture cardinality claims, and translate sentences like these. 

\subsection{The Language \GQLI{}}
To get the language \GQLI{}, we add to the basic symbols of \GQL{} (def. \pmvref{Symbols of GQL}) the identity symbol \mention{$=$} and add a new base clause to the recursive definition of a \GQL{} formula (def. \pmvref{Definition of Formula of GQL}). 
\begin{majorILnc}{\LnpDC{Definition of Formula of GQLI}} The \nidf{formulas} \underdf{of \GQLI{}}{formulas} are given by the following recursive definition:
\begin{description}
\item[Base Clauses:] \hfill{}
\begin{cenumerate}
\item A sentence letter (atomic sentence of \GSL{}) is an atomic formula.
\item An $\integer{n}$-place predicate followed by $\integer{n}$ occurrences (tokens) of individual constants or variables is an atomic formula.
\item Given two individual terms $\variable{t}$ and $\variable{s}$ (see Sec. \ref{GQL Truth in an Interpretation}), $\variable{t}=\variable{s}$ is an atomic formula.
\end{cenumerate}

So $\variable{b}=\variable{d}$, $\variable{b}=\variable{x}$, $\variable{y}=\variable{d}$, and $\variable{x}=\variable{x}$ are all examples of atomic formulas.

\item[Generating Clauses:] \hfill{}
\begin{cenumerate}
\item If $\CAPPHI$ is a formula, then so is $\negation{\CAPPHI}$.
\item If $\CAPPHI$ and $\CAPTHETA$ are formulas, then so are $\parhorseshoe{\CAPPHI}{\CAPTHETA}$ and $\partriplebar{\CAPPHI}{\CAPTHETA}$.
\item If all of $\CAPPHI_1,\CAPPHI_2,\CAPPHI_3,\CAPPHI_4,\ldots,\CAPPHI_{\integer{n}}$ are formulas (the list must include at least two formulas and be finite), then so are $\parconjunction{\CAPPHI_1}{\conjunction{\CAPPHI_2}{\conjunction{\CAPPHI_3}{\conjunction{\CAPPHI_4}{\conjunction{\ldots}{\CAPPHI_{\integer{n}}}}}}}$ and $\pardisjunction{\CAPPHI_1}{\disjunction{\CAPPHI_2}{\disjunction{\CAPPHI_3}{\disjunction{\CAPPHI_4}{\disjunction{\ldots}{\CAPPHI_{\integer{n}}}}}}}$.
\item If $\CAPPHI$ is a formula and it does not contain an expression of the form $\universal{\ALPHA}$ or $\existential{\ALPHA}$ for some \GQL{} variable $\ALPHA$, then $\universal{\ALPHA}\CAPPHI$ and $\existential{\ALPHA}\CAPPHI$ are formulas.
\end{cenumerate}
\item[Closure Clause:] A string of symbols is a formula \Iff it can be generated by the clauses above.
\end{description}
\end{majorILnc}
\noindent{}Note that this definition of formulas of \GQLI{} is exactly the same as that for formulas of \GQL{} (def. \pmvref{Definition of Formula of GQL}), except for the new third base clause. 

\subsection{Truth, Logical Truth, and Entailment in GQLI}
The models for \GQLI{} are just the models for \GQL{} (recall def. \pmvref{GQL Interpretation}), but obviously we have to extend the definition of truth to account for sentences with the identity symbol, \mention{$=$}.
\begin{majorILnc}{\LnpDC{Truth for GQLI Formula}}
The following clauses fix when a sentence of \GQLI{} is \nidf{$\True$} (or \nidf{$\False$}) on a model $\IntA$:
\begin{cenumerate}
	\item A sentence letter $\CAPPHI$ is $\True$ on $\IntA$ \Iff $\As{}{}$ assigns $\True$ to it, i.e. \Iff $\As{}{}(\CAPPHI)=\TrueB$.
	\item An atomic sentence $\Pp{\variable{t}}$ with a 1-place predicate $\PP$ and an individual term $\variable{t}$ is $\True$ on $\IntA$ \Iff what $\IntA$ assigns to the individual term $\variable{t}$ is in the set $\IntA$ assigns to the predicate, i.e. \Iff $\IntA(\variable{t})\in\IntA(\PP)$.
	\item An atomic sentence $\Pp{\variable{t}_1\ldots\variable{t}_{\integer{n}}}$ with an $\integer{n}$-place predicate $\PP$ is $\True$ on $\IntA$ \Iff $\langle \As{}{}(\variable{t}_1),\As{}{}(\variable{t}_2),\ldots,\As{}{}(\variable{t}_{\integer{n}}) \rangle \in \As{}{}(\PP)$.
	\item An atomic sentence $\variable{t}=\variable{s}$ is $\True$ on $\IntA$ \Iff $\As{}{}(\variable{t})=\As{}{}(\variable{s})$. 
	\item A negation $\negation{\CAPPHI}$ is $\True$ on $\IntA$ \Iff the unnegated formula $\CAPPHI$ is $\False$ on $\IntA$.
	\item A conjunction $\parconjunction{\CAPPHI_1}{\conjunction{\ldots}{\CAPPHI_{\integer{n}}}}$ is $\True$ on $\IntA$ \Iff all conjuncts $\CAPPHI_1,\ldots,\CAPPHI_{\integer{n}}$ are $\True$ on $\IntA$.
	\item A disjunction $\pardisjunction{\CAPPHI_1}{\disjunction{\ldots}{\CAPPHI_{\integer{n}}}}$ is $\True$ on $\IntA$ \Iff at least one disjunct $\CAPPHI_1,\ldots,\CAPPHI_{\integer{n}}$ is $\True$ on $\IntA$.
	\item A conditional $\parhorseshoe{\CAPPSI}{\CAPPHI}$ is $\True$ on $\IntA$ \Iff the \CAPS{lhs} $\CAPPSI$ is $\False$ or the \CAPS{rhs} $\CAPPHI$ is $\True$ on $\IntA$.
	\item A biconditional $\partriplebar{\CAPPSI}{\CAPPHI}$ is $\True$ on $\IntA$ \Iff both sides, $\CAPPSI$ and $\CAPPHI$, have the same truth value on $\IntA$.
	\item A universal quantification $\universal{\ALPHA}\CAPPHI$ is $\True$ on $\IntA$ \Iff $\CAPPHI\variable{t}/\ALPHA$ is $\True$ on \emph{all} $\variable{t}$-variants of $\IntA$ (where $\variable{t}$ is the first \emph{constant} not contained in $\CAPPHI$).
	\item An existential quantification $\existential{\ALPHA}\CAPPHI$ is $\True$ on $\IntA$ \Iff $\CAPPHI\variable{t}/\ALPHA$ is $\True$ on \emph{some} $\variable{t}$-variant of $\IntA$ (where $\variable{t}$ is the first \emph{constant} not contained in $\CAPPHI$).
	\item A sentence $\CAPPHI$ is $\False$ on $\IntA$ \Iff $\CAPPHI$ is not $\True$ on $\IntA$.

\end{cenumerate}
\end{majorILnc}
\noindent{}This definition is exactly the same as the one given for \GQL{}, except for the added clause (4) which covers the new atomic sentence added in definition \ref{Definition of Formula of GQLI}. 

The definitions of quantificational truth (def. \ref{QT}), falsehood (def. \ref{QF}), and contingency (def. \ref{QI}) are the same as they were for \GQL{}. 
The definition of entailment is also the same, and likewise for the other relations defined in section \ref{GQL Entailment and other Relations}.

\begin{majorILnc}{\LnpEC{IdentityExampleA}}
Consider any model $\IntA$ which includes Cicero and Tully in its universe and assigns $\constant{c}$ to Cicero and $\constant{d}$ to Tully. 
(Cicero was a great Roman orator from the first century BC, and \mention{Tully} is an anglicized version of his name.)
Then there exists a sentence $\CAPPHI$ of \GQLI{} such that all and only those models $\IntA$ in which Cicero is Tully make $\CAPPHI$ true.
One such sentence is $\text{c}=\text{d}$.
Hence, $\text{c}=\text{d}$ is a reasonable translation of the English sentence \mention{Cicero is Tully.}
\end{majorILnc}
\begin{majorILnc}{\LnpEC{IdentityExampleB}}
Consider the sentence $\horseshoe{\parconjunction{\text{c}=\text{d}}{\Rp{\text{c}}}}{\Rp{\text{d}}}$ and consider just those models $\IntA$ which include Cicero and Tully in the universe, assign $\constant{c}$ to Cicero and $\constant{d}$ to Tully, and assign to $\RR$ the set of all Romans. 
Then any of those models which make the sentence true are such that either $\IntA(\constant{d})\in\IntA(\RR)$, or either $\IntA(\constant{c})\neq\IntA(\constant{d})$ or $\IntA(\constant{c})\notin\IntA(\RR)$.
Hence the sentence is a reasonable translation of the English sentence \mention{If Cicero is Tully and Cicero is Roman, then Tully is Roman.}
\end{majorILnc}
\begin{majorILnc}{\LnpEC{IdentityExampleC}}
Consider the sentence $\existential{\x}\bparconjunction{\Ap{\x}}{\universal{\y}\parhorseshoe{\Ap{\y}}{\y=\x}}$.
A model $\IntA$ makes the sentence true \Iff $\IntA(\AA)$ contains exactly one element.
Hence the sentence is a reasonable translation of the English sentence \mention{There exists exactly one apple}, or any other English sentence that differs from this one only in a change of the predicate \mention{apple}.
\end{majorILnc}
\begin{majorILnc}{\LnpEC{IdentityExampleD}}
Consider the sentence
\begin{quote} $\existential{\variable{w}}\existential{\variable{z}}\bparconjunction{\Ap{\variable{w}}}{\conjunction{\Ap{\variable{z}}}{\conjunction{\variable{w}\neq\variable{z}}{\universal{\variable{v}}\parhorseshoe{\Ap{\variable{v}}}{\cpardisjunction{\variable{v}=\variable{w}}{\variable{v}=\variable{z}}}}}}$.
\end{quote}
The two existential quantifiers with the negative identity guarantee that there are at least two instances of \Al, and the universal quantifier guarantees that these are the only ones. 
So a model $\IntA$ makes the sentence true \Iff $\IntA(\AA)$ contains exactly two elements.
Hence the sentence is a reasonable translation of the English sentence \mention{There exists (exactly) two apples.}
\end{majorILnc}
\begin{majorILnc}{\LnpEC{IdentityExampleE}}
Consider the sentence 
\begin{quote}
$\existential{\x}\bparconjunction{\Ap{\x}}{\universal{\y}\parhorseshoe{\Ap{\y}}{\y=\x}}\HORSESHOE{}$\\$\negation{\existential{\variable{w}}\existential{\variable{z}}\bparconjunction{\Ap{\variable{w}}}{\conjunction{\Ap{\variable{z}}}{\conjunction{\variable{w}\neq\variable{z}}{\universal{\variable{v}}\parhorseshoe{\Ap{\variable{v}}}{\cpardisjunction{\variable{v}=\variable{w}}{\variable{v}=\variable{z}}}}}}}$.
\end{quote}
This sentence is a reasonable translation of the English sentence \mention{If there exists exactly one apple, then there doesn't exist exactly two apples.}
\end{majorILnc}
\begin{majorILnc}{\LnpEC{IdentityExampleF}}
Consider the sentence $\conjunction{\conjunction{\Cpp{\text{b}}{\text{a}}}{\Np{\text{b}}}}{\universal{x}\parhorseshoe{\parconjunction{\Cpp{\x}{\text{a}}}{\Np{\x}}}{\x=\text{b}}}$.
This sentence is true in those models $\IntA$ where only one element of the domain is both in the set $\IntA(\NN)$ and stands in relation $\CC$ to $\IntA(\constant{a})$. 
So, if $\Cpp{\text{b}}{\text{a}}$ means ``$\text{b}$ is a child of $\text{a}$'', $\Np{\text{b}}$ means ``$\text{b}$ is male'', $\text{a}$ is Arnold, and $\text{b}$ is Bob, this sentence would make a reasonable translation of the English sentence \mention{Bob is Arnold's only son.}
\end{majorILnc}
\begin{majorILnc}{\LnpEC{IdentityExampleG}}
Consider the sentence $\existential{\y_1}\existential{\y_2}\universal{\x}\pardisjunction{\x=\y_1}{\x=\y_2}$.
It is true in all and only those models $\IntA$ that have two or less elements in their universe.
So, the sentence would make a reasonable translation of the English sentence \mention{At most only two things exist.}
\end{majorILnc}
\begin{majorILnc}{\LnpEC{IdentityExampleH}}
Consider the sentence $\existential{\x}\existential{\y}\x\neq\y$.
It is true in all and only those models $\IntA$ with two or more elements in their domain.
So, it would make a reasonable translation of the English sentence \mention{At least two things exist.}
\end{majorILnc}
\begin{majorILnc}{\LnpEC{IdentityExampleI}}
Consider the sentence $\existential{\x}\existential{\y}\existential{\variable{z}}\parconjunction{\x\neq\y}{\conjunction{\x\neq\variable{z}}{\y\neq\variable{z}}}$.
It's true in all and only those models $\IntA$ with three or more elements in their domain.
So, it would make a reasonable translation of the English sentence \mention{At least three things exist.}
\end{majorILnc}
\begin{majorILnc}{\LnpEC{IdentityExampleJ}}
Consider the sentence 
\begin{quote}
$\universal{\x}\{\Gp{\x}\HORSESHOE\universal{\y}[(\Pp{\y}\WEDGE\Rpp{\y}{\x})\HORSESHOE\existential{\variable{z}_1}\existential{\variable{z}_2}\existential{\variable{z}_3}$\\
\hspace*{.5in}$(\Ip{\variable{z}_1}\WEDGE\Ip{\variable{z}_2}\WEDGE\Ip{\variable{z}_3}\WEDGE\Gppp{\x}{\variable{z}_1}{\y}\WEDGE\Gppp{\x}{\variable{z}_2}{\y}\WEDGE\Gppp{\x}{\variable{z}_3}{\y}\WEDGE$\\
\hspace*{1in}$\conjunction{\variable{z}_1\neq\variable{z}_2}{\conjunction{\variable{z}_1\neq\variable{z}_3}{\variable{z}_2\neq\variable{z}_3}})]\}$.
\end{quote}
It would make a reasonable translation of the English sentence \mention{Every genie grants anyone who releases them at least three wishes.}
\end{majorILnc}
\begin{majorILnc}{\LnpEC{IdentityExampleK}}
Even with identity, there's no one sentence of GQLI which is true in all and only those models with infinite domains.\footnote{There are some single sentences of GQLI which are such that the only models that make them true have infinite domains, but not every model with an infinite domain makes these true. So, these sentences can't really be thought of as saying that there exists an infinite number of things. They say more than that.} But, there's an infinite set of sentences $\Delta$ such that a model makes every sentence of $\Delta$ true iff it has an infinite domain. Define: %\[\underset{i\neq j}{\overset{n}{\BWEDGE}}\x_i\neq\x_j\Leftrightarrow \parconjunction{\x_1\neq\x_2}{\conjunction{\x_1\neq\x_3}{\conjunction{\ldots}{\conjunction{\x_1\neq\x_{\integer{n}}}{\conjunction{\x_2\neq\x_3}{\conjunction{\x_2\neq\x_4}{\conjunction{\ldots}{\conjunction{\x_2\neq\x_{\integer{n}}}{\conjunction{\ldots\ldots}{\x_{n-1}\neq\x_{\integer{n}}}}}}}}}}}\]

\begin{center}
\begin{tabular}{ r c l }
	$\underset{i\neq j}{\overset{n}{\BWEDGE}}\x_i\neq\x_j$ & $\Leftrightarrow$ & ($\x_1\neq\x_2\WEDGE\x_1\neq\x_3\WEDGE\x_1\neq\x_4\WEDGE\ldots\WEDGE\x_1\neq\x_{\integer{n}}\WEDGE$ \\
	& & $\x_2\neq\x_3\WEDGE\x_2\neq\x_4\WEDGE\x_2\neq\x_5\WEDGE\ldots\WEDGE\x_2\neq\x_{\integer{n}}\WEDGE$ \\
	& & $\x_3\neq\x_4\WEDGE\x_3\neq\x_5\WEDGE\x_3\neq\x_6\WEDGE\ldots\WEDGE\x_3\neq\x_{\integer{n}}\WEDGE$ \\
	& & $\vdots$ \\
	& & $\x_{n-2}\neq\x_{n-1}\WEDGE\x_{n-2}\neq\x_{\integer{n}}\WEDGE$ \\
	& & $\x_{n-1}\neq\x_{\integer{n}}$) \\
\end{tabular}
\end{center}
Then a model makes every sentence in the following list true iff its universe is infinite:
\[\existential{\x_1}\existential{\x_2}\underset{i\neq j}{\overset{2}{\BWEDGE}}\x_i\neq\x_j\]
\[\existential{\x_1}\existential{\x_2}\existential{\x_3}\underset{i\neq j}{\overset{3}{\BWEDGE}}\x_i\neq\x_j\]
\[\existential{\x_1}\existential{\x_2}\existential{\x_3}\existential{\x_4}\underset{i\neq j}{\overset{4}{\BWEDGE}}\x_i\neq\x_j\]
\[\existential{\x_1}\existential{\x_2}\existential{\x_3}\existential{\x_4}\existential{\x_5}\underset{i\neq j}{\overset{5}{\BWEDGE}}\x_i\neq\x_j\]
\[\vdots\]
We leave it to the reader to show that this is true.
\end{majorILnc}

\subsection{Derivations in GQDI}
We get a sound and complete derivation system for \GQDI{} by adding an introduction and elimination rule for identity. 
%We get the system GQDI (Grandy Quantifier $+$ Identity Derivations) by adding two new basic rules to GQD:
%\begin{table}[!ht]
\renewcommand{\arraystretch}{1.5}
%\begin{center}
\begin{longtable}[c]{ p{1in} l l } %p{2.2in} p{2in}
\toprule
\textbf{Name} & \textbf{Given} & \textbf{May Add} \\ 
\midrule
\endfirsthead
\multicolumn{3}{c}{\emph{Continued from Previous Page}}\\
\toprule
\textbf{Name} & \textbf{Given} & \textbf{May Add} \\ 
\midrule
\endhead
\bottomrule
\caption{Basic Rules of \GQDI{}}\\[-.15in]
\multicolumn{3}{c}{\emph{Continued next Page}}\\
\endfoot
\bottomrule
\caption{Basic Rules of \GQDI{}}\\%
\endlastfoot%
\label{GQDI}%
\Rule{$=$-Intro} &  & $\variable{t}=\variable{t}$ \\
\Rule{$=$-Elim} & $\CAPPHI$, $\variable{t}=\variable{s}$ & $\CAPPHI\variable{t}/\variable{s}$ \\
%\bottomrule
\end{longtable}
\index{derivation!rule!\GQDI{}}\index{GQDI}
\index{derivation!rule!introduction}\index{introduction rule}
\index{derivation!rule!elimination}\index{elimination rule}
\noindent{}Note that since only sentences can appear on lines of derivations, $\variable{t}$ and $\variable{s}$ must be constants. 
Also, note that \Rule{$=$-Intro} is like \Rule{Assume} insofar as neither is ``applied'' to previous lines; both allow you to write a sentence that fits the may-add schema at any point in a derivation.
%All of the short cut rules in \GQDP{} (including prenex and reverse prenex) are still legal in GQDI; all of the good strategies still are good strategies.

Like \GQD{}, \GQDI{} is both sound and strongly complete.
\begin{THEOREM}{\LnpTC{GQDISoundness} \GQDI{} Soundness Theorem:}
For\index{soundness!of \GQDI{}} all sentences $\CAPPHI$ in \GQLI{} and sets of sentences $\Delta$, if $\Delta\sststile{}{}\CAPPHI$ in \GQDI{}, then $\Delta\sdtstile{}{}\CAPPHI$.
\end{THEOREM}
\begin{THEOREM}{\LnpTC{GQDIStrongCompleteness} \GQDI{} Strong Completeness Theorem:}
For\index{completeness!of \GQDI{}} all sentences $\CAPPHI$ in \GQLI{} and sets of sentences $\Delta$, if $\Delta\sdtstile{}{}\CAPPHI$ in \GQDI{}, then $\Delta\sststile{}{}\CAPPHI$.
\end{THEOREM}
\noindent{}We won't prove either the soundness or completeness of \GQDI{}, but a few remarks are in order.
There is nothing essentially different about the soundness proof of \GQDI{}, compared with the proof for \GQD{}.
We can still use the general approach from section \mvref{Sec:Completeness of GQD} to prove the (strong) completeness of \GQDI{}, but there are two important differences.
Here we briefly mention where changes need to be made without describing how those changes can be made.

The first difference concerns our search for contradictions (recall Step 4 from Sec. \pmvref{The Method Section}). 
It's not enough to take the conjunction of all the matrix instances (by \Rule{$\WEDGE\!$-Intro}), use \Rule{Distribution} to get the conjunction into \CAPS{dnf} and check whether every disjunct contains a contradiction. 
The reason is that there might be some disjuncts that don't contain a contradiction, but do contain an ion $\CAPPHI$, an ion $\negation{\CAPPSI}$ where $\CAPPHI=\CAPPHI\variable{t}/\variable{s}$, and the ion $\variable{t}=\variable{s}$. 
For example, consider the disjunct $\conjunction{\Gp{\constant{a}}}{\conjunction{\negation{\Gp{\constant{b}}}}{\constant{a}=\constant{b}}}$.
A contradiction can be derived from these disjuncts using \Rule{$=$-Elim}, so we need to adjust our procedure accordingly. 

The second difference concerns how we construct the model when The Method (or The Strong Method) doesn't produce a contradiction. 
The old procedure works by assigning distinct elements to each constant that appears in the list of sentences produced by The Method. 
So, if we could construct a model that made all the sentences produced by The Method (when it doesn't end in a contradiction) true by the old procedure, that model would assign distinct elements to each constant.
But there are sets $\Delta$ of \GQLI{} sentences that both contain ions of the form $\variable{t}=\variable{s}$ and are consistent. 
Clearly no model that assigns distinct elements to the constants $\variable{t}$ and $\variable{s}$ could make all the sentences in such a set $\Delta$ true. 

Our procedure is to take the model generated by the previous method and modify it by systematically changing the constant assignments so that if $\constant{s}=\constant{t}$ is in the Master Matrix list
$\constant{s}$ and $\constant{t}$ are both assigned the same constant.

%%%%%%%%%%%%%%%%%%%%%%%%%%%%%%%%%%%%%%%%%%%%%%%%%%
%\section{G\"odel's Theorem}
%%%%%%%%%%%%%%%%%%%%%%%%%%%%%%%%%%%%%%%%%%%%%%%%%%

%%%%%%%%%%%%%%%%%%%%%%%%%%%%%%%%%%%%%%%%%%%%%%%%%%
\section{Exercises}
%%%%%%%%%%%%%%%%%%%%%%%%%%%%%%%%%%%%%%%%%%%%%%%%%%

\notocsubsection{\CAPS{tft} in Many-Valued Logic}{ex:LT in MVL}
Each of the following sentences are 2-valued \CAPS{TFT}. 
Which are also 3-valued \CAPS{TFT}?
Which of the 3-valued \CAPS{TFT} are $\integer{n}$-valued \CAPS{TFT} for all finite $\integer{n}$? 
\begin{multicols}{2}
\begin{enumerate}
\item $\triplebar{\negation{\negation{\Al}}}{\Al}$
\item $\horseshoe{\parhorseshoe{\Al}{\Bl}}{\pardisjunction{\negation{\Al}}{\Bl}}$
\item $\horseshoe{\pardisjunction{\negation{\Al}}{\Bl}}{\parhorseshoe{\Al}{\Bl}}$
\item $\triplebar{\negation{\pardisjunction{\Al}{\Bl}}}{\parconjunction{\negation{\Al}}{\negation{\Bl}}}$
\item $\horseshoe{\Al}{\parhorseshoe{\negation{\Al}}{\Bl}}$
\item $\horseshoe{\Al}{\parhorseshoe{\parhorseshoe{\Al}{\Bl}}{\Bl}}$
\item $\horseshoe{\parconjunction{\Al}{\negation{\Al}}}{\Bl}$
\item $\horseshoe{\parhorseshoe{\Al}{\pardisjunction{\Bl}{\Cl}}}{\pardisjunction{\parhorseshoe{\Al}{\Bl}}{\Cl}}$
\end{enumerate}
\end{multicols}
\begin{enumerate}[start=9]
\item $\disjunction{\partriplebar{\Al}{\Bl}}{\disjunction{\partriplebar{\Al}{\Cl}}{\disjunction{\partriplebar{\Al}{\Dl}}{\disjunction{\partriplebar{\Bl}{\Cl}}{\disjunction{\partriplebar{\Bl}{\Dl}}{\partriplebar{\Cl}{\Dl}}}}}}$
\item $\horseshoe{\parconjunction{\pardisjunction{\Al}{\Bl}}{\negation{\Al}}}{\Bl}$
\end{enumerate}

\notocsubsection{Modal Truths in \MGSL{} \#1}{ex:LT in ML1}
For each schema below, show that it's a modal truth for all \MGSL{} sentences $\CAPPHI$ and $\CAPPSI$. 
These are from example \mvref{MGSLTruthExampleC}.
\begin{multicols}{2}
\begin{enumerate}
\item $\horseshoe{\BOX\BOX\CAPPHI}{\BOX\CAPPHI}$
\item $\horseshoe{\BOX\DIAMOND\CAPPHI}{\DIAMOND\CAPPHI}$
\item $\horseshoe{\DIAMOND\CAPPHI}{\DIAMOND\DIAMOND\CAPPHI}$
\item $\horseshoe{\BOX\CAPPHI}{\DIAMOND\BOX\CAPPHI}$
\item $\triplebar{\negation{\DIAMOND\CAPPHI}}{\BOX\negation{\CAPPHI}}$
\item $\triplebar{\DIAMOND\negation{\CAPPHI}}{\negation{\BOX\CAPPHI}}$
\item $\horseshoe{\BOX\parhorseshoe{\CAPPHI}{\CAPPSI}}{\parhorseshoe{\BOX\CAPPHI}{\BOX\CAPPSI}}$
\item $\horseshoe{\pardisjunction{\BOX\CAPPHI}{\BOX\CAPPSI}}{\BOX\pardisjunction{\CAPPHI}{\CAPPSI}}$
\item $\triplebar{\pardisjunction{\DIAMOND\CAPPHI}{\DIAMOND\CAPPSI}}{\DIAMOND\pardisjunction{\CAPPHI}{\CAPPSI}}$
\item $\triplebar{\BOX\parconjunction{\CAPPHI}{\CAPPSI}}{\parconjunction{\BOX\CAPPHI}{\BOX\CAPPSI}}$
\item $\horseshoe{\DIAMOND\parconjunction{\CAPPHI}{\CAPPSI}}{\parconjunction{\DIAMOND\CAPPHI}{\DIAMOND\CAPPSI}}$
\item $\triplebar{\negation{\DIAMOND\negation{\CAPPHI}}}{\BOX\CAPPHI}$
\item $\horseshoe{\BOX\parhorseshoe{\CAPPHI}{\CAPPSI}}{\parhorseshoe{\DIAMOND\CAPPHI}{\DIAMOND\CAPPSI}}$
\item $\horseshoe{\DIAMOND\parhorseshoe{\CAPPHI}{\CAPPSI}}{\parhorseshoe{\BOX\CAPPHI}{\DIAMOND\CAPPSI}}$
\item $\horseshoe{\negation{\DIAMOND\CAPPHI}}{\BOX\parhorseshoe{\CAPPHI}{\CAPPSI}}$
\item $\horseshoe{\BOX\partriplebar{\CAPPHI}{\CAPPSI}}{\partriplebar{\BOX\CAPPHI}{\BOX\CAPPSI}}$
\end{enumerate}
\end{multicols}

\notocsubsection{Modal Truths in \MGSL{} \#2}{ex:LT in ML2}
For each schema below, show that it's a modal truth for all \MGSL{} sentences $\CAPPHI$. 
These are from theorem \pmvref{MGSLSFiveTheorem}.
\begin{multicols}{2}
\begin{enumerate}
\item $\triplebar{\BOX\BOX\CAPPHI}{\BOX\CAPPHI}$
\item $\triplebar{\BOX\DIAMOND\CAPPHI}{\DIAMOND\CAPPHI}$
\item $\triplebar{\DIAMOND\CAPPHI}{\DIAMOND\DIAMOND\CAPPHI}$
\item $\triplebar{\BOX\CAPPHI}{\DIAMOND\BOX\CAPPHI}$
\end{enumerate}
\end{multicols}

\notocsubsection{Modal Truths in \MGSL{} \#3}{ex:LT in ML3}
Show whether each of the following is a modal truth. 
\begin{multicols}{2}
\begin{enumerate}
\item $\sststile{}{}\triplebar{\DIAMOND\Al}{\negation{\BOX\negation{\Al}}}$
\item $\sststile{}{}\horseshoe{\BOX\Al}{\BOX\BOX\Al}$
\item $\sststile{}{}\horseshoe{\BOX\parhorseshoe{\Al}{\Bl}}{\parhorseshoe{\BOX\Al}{\BOX\Bl}}$
\item $\sststile{}{}\triplebar{\DIAMOND\pardisjunction{\Al}{\Bl}}{\pardisjunction{\DIAMOND\Al}{\DIAMOND\Bl}}$
\item $\sststile{}{}\horseshoe{\parconjunction{\DIAMOND\Al}{\DIAMOND\Bl}}{\DIAMOND\parconjunction{\Al}{\Bl}}$
\item $\sststile{}{}\horseshoe{\BOX\parhorseshoe{\Al}{\Bl}}{\parhorseshoe{\Al}{\BOX\Bl}}$
\item $\sststile{}{}\horseshoe{\Al}{\BOX\Al}$
\item $\sststile{}{}\horseshoe{\DIAMOND\parhorseshoe{\Al}{\Bl}}{\parhorseshoe{\BOX\Al}{\DIAMOND\Bl}}$
\item $\sststile{}{}\horseshoe{\BOX\parhorseshoe{\Al}{\Bl}}{\parhorseshoe{\DIAMOND\Al}{\DIAMOND\Bl}}$
\item $\sststile{}{}\triplebar{\BOX\parconjunction{\Al}{\Bl}}{\parconjunction{\BOX\Al}{\BOX\Bl}}$
\end{enumerate}
\end{multicols}
\begin{enumerate}[start=11]
\item $\sststile{}{}\horseshoe{\BOX\DIAMOND\Al}{\DIAMOND\BOX\Al}$
\end{enumerate}

\notocsubsection{Derivations in \MGSL{}}{ex:Derivations in ML}
Write derivations for each schema below in \SF{}. 
Try finding derivations both with and without the modal negation rules. 
Again, these are from example \mvref{MGSLTruthExampleC}.
\begin{multicols}{2}
	\begin{enumerate}
		\item $\horseshoe{\BOX\BOX\CAPPHI}{\BOX\CAPPHI}$
		\item $\horseshoe{\BOX\DIAMOND\CAPPHI}{\DIAMOND\CAPPHI}$
		\item $\horseshoe{\DIAMOND\CAPPHI}{\DIAMOND\DIAMOND\CAPPHI}$
		\item $\horseshoe{\BOX\CAPPHI}{\DIAMOND\BOX\CAPPHI}$
		\item $\triplebar{\negation{\DIAMOND\CAPPHI}}{\BOX\negation{\CAPPHI}}$
		\item $\triplebar{\DIAMOND\negation{\CAPPHI}}{\negation{\BOX\CAPPHI}}$
		\item $\horseshoe{\BOX\parhorseshoe{\CAPPHI}{\CAPPSI}}{\parhorseshoe{\BOX\CAPPHI}{\BOX\CAPPSI}}$
		\item $\horseshoe{\pardisjunction{\BOX\CAPPHI}{\BOX\CAPPSI}}{\BOX\pardisjunction{\CAPPHI}{\CAPPSI}}$
		\item $\triplebar{\pardisjunction{\DIAMOND\CAPPHI}{\DIAMOND\CAPPSI}}{\DIAMOND\pardisjunction{\CAPPHI}{\CAPPSI}}$
		\item $\triplebar{\BOX\parconjunction{\CAPPHI}{\CAPPSI}}{\parconjunction{\BOX\CAPPHI}{\BOX\CAPPSI}}$
		\item $\horseshoe{\DIAMOND\parconjunction{\CAPPHI}{\CAPPSI}}{\parconjunction{\DIAMOND\CAPPHI}{\DIAMOND\CAPPSI}}$
		\item $\triplebar{\negation{\DIAMOND\negation{\CAPPHI}}}{\BOX\CAPPHI}$
		\item $\horseshoe{\BOX\parhorseshoe{\CAPPHI}{\CAPPSI}}{\parhorseshoe{\DIAMOND\CAPPHI}{\DIAMOND\CAPPSI}}$
		\item $\horseshoe{\DIAMOND\parhorseshoe{\CAPPHI}{\CAPPSI}}{\parhorseshoe{\BOX\CAPPHI}{\DIAMOND\CAPPSI}}$
		\item $\horseshoe{\negation{\DIAMOND\CAPPHI}}{\BOX\parhorseshoe{\CAPPHI}{\CAPPSI}}$
		\item $\horseshoe{\BOX\partriplebar{\CAPPHI}{\CAPPSI}}{\partriplebar{\BOX\CAPPHI}{\BOX\CAPPSI}}$
	\end{enumerate}
\end{multicols}


\notocsubsection{Derivations in \GQLI{}}{ex:Derivations in QLI}
Write derivations for each of the following in \GQDI{}.  
%\begin{multicols}{2}
\begin{enumerate}
\item $\sststile{}{}\universal{\variable{x}}\variable{x}=\variable{x}$
\item $\sststile{}{}\universal{\variable{x}}\universal{\variable{y}}\parhorseshoe{\variable{x}=\variable{y}}{\variable{y}=\variable{x}}$
\item $\sststile{}{}\universal{\variable{x}}\universal{\variable{y}}\universal{\variable{z}}\parhorseshoe{\parconjunction{\variable{x}=\variable{y}}{\variable{y}=\variable{z}}}{\variable{x}=\variable{z}}$
\item $\sststile{}{}\universal{\variable{x}}\partriplebar{\Gp{\variable{x}}}{\existential{\variable{y}}\parconjunction{\variable{x}=\variable{y}}{\Gp{\variable{y}}}}$
\item $\sststile{}{}\universal{\variable{x}}\partriplebar{\Gp{\variable{x}}}{\universal{\variable{y}}\parhorseshoe{\variable{x}=\variable{y}}{\Gp{\variable{y}}}}$
\item $\sststile{}{}\universal{\variable{x}}\universal{\variable{y}}\parhorseshoe{\variable{x}=\variable{y}}{\partriplebar{\Gp{\variable{x}}}{\Gp{\variable{y}}}}$
\item $\sststile{}{}\horseshoe{\existential{\variable{x}}\universal{\variable{y}}\partriplebar{\Gp{\variable{y}}}{\variable{y}=\variable{x}}}{\parconjunction{\existential{\variable{x}}\Gp{\variable{x}}}{\universal{\variable{x}}\universal{\variable{y}}\parhorseshoe{\parconjunction{\Gp{\variable{x}}}{\Gp{\variable{y}}}}{\variable{x}=\variable{y}}}}$
\item $\sststile{}{}\horseshoe{\parconjunction{\existential{\variable{x}}\Gp{\variable{x}}}{\universal{\variable{x}}\universal{\variable{y}}\parhorseshoe{\parconjunction{\Gp{\variable{x}}}{\Gp{\variable{y}}}}{\variable{x}=\variable{y}}}}{\existential{\variable{x}}\universal{\variable{y}}\partriplebar{\Gp{\variable{y}}}{\variable{y}=\variable{x}}}$
\item $\sststile{}{}\horseshoe{\universal{\variable{x}}\existential{\variable{y}}\parconjunction{\variable{y}\neq\variable{x}}{\Gp{\variable{y}}}}{\existential{\variable{x}}\existential{\variable{y}}\parconjunction{\variable{x}\neq\variable{y}}{\parconjunction{\Gp{\variable{x}}}{\Gp{\variable{y}}}}}$
\item $\sststile{}{}\horseshoe{\existential{\variable{x}}\existential{\variable{y}}\parconjunction{\variable{x}\neq\variable{y}}{\parconjunction{\Gp{\variable{x}}}{\Gp{\variable{y}}}}}{\universal{\variable{x}}\existential{\variable{y}}\parconjunction{\variable{y}\neq\variable{x}}{\Gp{\variable{y}}}}$
\item $\sststile{}{}\horseshoe{\parconjunction{\universal{\variable{x}}\existential{\variable{y}}\Gpp{\variable{x}}{\variable{y}}}{\universal{\variable{x}}\negation{\Gpp{\variable{x}}{\variable{x}}}}}{\universal{\variable{x}}\existential{\variable{y}}\parconjunction{\variable{x}\neq\variable{y}}{\Gpp{\variable{x}}{\variable{y}}}}$
\item $\sststile{}{}\horseshoe{\parconjunction{\Gp{\constant{a}}}{\negation{\Gp{\constant{b}}}}}{\existential{\variable{x}}\existential{\variable{y}}\variable{x}\neq\variable{y}}$
\item $\sststile{}{}\triplebar{\parconjunction{\Gp{\constant{a}}}{\universal{\variable{x}}\parhorseshoe{\variable{x}\neq\constant{a}}{\Gp{\variable{x}}}}}{\universal{\variable{x}}\Gp{\variable{x}}}$
\item $\sststile{}{}\horseshoe{\universal{\variable{x}}\parhorseshoe{\variable{x}\neq\constant{a}}{\Gp{\variable{x}}}}{\universal{\variable{x}}\universal{\variable{y}}\parhorseshoe{\variable{x}\neq\variable{y}}{\pardisjunction{\Gp{\variable{x}}}{\Gp{\variable{y}}}}}$
\item $\sststile{}{}\horseshoe{\existential{\variable{x}}\universal{\variable{y}}\parhorseshoe{\variable{y}\neq\variable{x}}{\Gp{\variable{y}}}}{\universal{\variable{x}}\universal{\variable{y}}\parhorseshoe{\variable{x}\neq\variable{y}}{\pardisjunction{\Gp{\variable{x}}}{\Gp{\variable{y}}}}}$
\item $\sststile{}{}\horseshoe{\universal{\variable{x}}\universal{\variable{y}}\parhorseshoe{\variable{x}\neq\variable{y}}{\pardisjunction{\Gp{\variable{x}}}{\Gp{\variable{y}}}}}{\existential{\variable{x}}\universal{\variable{y}}\parhorseshoe{\variable{y}\neq\variable{x}}{\Gp{\variable{y}}}}$
\item $\sststile{}{}\horseshoe{\existential{\variable{y}}\universal{\variable{x}}\variable{x}=\variable{y}}{\pardisjunction{\universal{\variable{x}}\Gp{\variable{x}}}{\universal{\variable{x}}\negation{\Gp{\variable{x}}}}}$
\item $\sststile{}{}\horseshoe{\universal{\variable{x}}\universal{\variable{y}}\universal{\variable{z}}\pardisjunction{\variable{x}=\variable{y}}{\disjunction{\variable{x}=\variable{z}}{\variable{y}=\variable{z}}}}{\pardisjunction{\universal{\variable{x}}\Gp{\variable{x}}}{\disjunction{\universal{\variable{x}}\parhorseshoe{\Gp{\variable{x}}}{\Hp{\variable{x}}}}{\universal{\variable{x}}\parhorseshoe{\Gp{\variable{x}}}{\negation{\Hp{\variable{x}}}}}}}$
\end{enumerate}
%\end{multicols}

\notocsubsection{Translations into \GQLI{}}{Translation Problems QLI} Translate each of the following English sentences into \GQLI{} sentences about the model $\IntA$ given in table \mvref{Trans Int Table QLI}.
\begin{multicols}{2}
\begin{enumerate}
\item Bob is Arnold's only son.
\item Arnold has at least two children.
\item Arnold has at least two sons.
\item Arnold has at most two sons.
\item Arnold has exactly two sons.
\item Bob is Carol's only brother.
\item Bob is an only child.
\item Bob is an only son.
\item Bob has at least two sisters.
\item Bob has just one sister.
\item Everyone has a sister.
\item Carol's mother is Bob's only sister.
\item Bob has at least two grandchildren.
\item Diane only dates Bob.
\item Bob only dates daughters.
\item Bob dates only daughters.
\item Do (16) another way.
\item Bob only dates only daughters.
\item Bob dates only only daughters.
\item Only Bob only dates only daughters.
\end{enumerate}
\end{multicols}
%\begin{table}[!ht]
%\renewcommand{\arraystretch}{1.5}
%\begin{center}
\begin{longtable}[c]{ l l l } %p{2.2in} p{2in}
\toprule
&\textbf{Symbol} & \textbf{Assignment} \\
\midrule 
\endfirsthead
\multicolumn{3}{c}{\emph{Continued from Previous Page}}\\
\toprule
&\textbf{Symbol} & \textbf{Assignment} \\
\midrule 
\endhead
\bottomrule
\caption{Interpretation for Translations in Section \ref{Translation Problems QLI}}\\ %[-.15in]
\multicolumn{3}{c}{\emph{Continued next Page}}\\
\endfoot
\bottomrule
\caption{Interpretation for Translations in Section \ref{Translation Problems QLI}}\\%
\endlastfoot%
\label{Trans Int Table QLI}% 
Universe:& & All people \\ \addlinespace[.25cm]
Constants:& $\constant{a}$& Arnold\\
& $\constant{b}$& Bob\\
& $\constant{c}$& Carol\\
& $\constant{d}$& Diane\\ \addlinespace[.25cm]
1 place predicates: &$\Mp{'}$& Male\\
&$\Ep{'}$& Female\\ \addlinespace[.25cm]
2 place predicates:&$\Pp{''}$& is the parent of\\
&$\Dp{''}$& dates\\
\end{longtable}

%\theendnotes


\clearemptydoublepage
%%%%%%%%%%%%%%%%%%%%%%%%%%%%%%%%%%%%%%%%%%%%%%%%%%
\chapter*{Appendix A: List of Theorems}
\addcontentsline{toc}{chapter}{Appendix A: List of Theorems}
%%%%%%%%%%%%%%%%%%%%%%%%%%%%%%%%%%%%%%%%%%%%%%%%%%
%\fancyhead[RE,LO]{\textsf{Appendix A: List of Theorems}}
\chead{\textsf{Appendix A: List of Theorems}}
\fancyhead[LE,RO]{\textsf{\thepage}}
\setcounter{section}{0}

\section*{Sentences}

\begin{majorILnc}{\textbf{Theorem \mvref{Number of sentences}.}}
The number of \GSL{} sentences is equal to the number of natural numbers.
\end{majorILnc}

\begin{majorILnc}{\textbf{Theorem \mvref{Recur Main Connective}.} Main Connective Theorem:} 
For\index{main connective!theorem} any non-atomic official \GSL{} sentence $\CAPPHI$ (any sentence with order 2 or greater), either $\CAPPHI$ has no token logical connectives other than negation, or there is one and only one token truth connective (or string of tokens) that's not a negation in $\CAPPHI$ that has one more token of `(' than `)' to the left of it.
\end{majorILnc}

\begin{majorILnc}{\textbf{Theorem \mvref{Disjunctivie Normal Form Theorem}.} The Disjunctive Normal Form Theorem:}
Every sentence of \GSL{} is truth functionally equivalent to a \GSL{} sentence which is in \CAPS{dnf}.
\end{majorILnc}

\begin{majorILnc}{\textbf{Theorem \mvref{Truth-functional Expressive Completeness of GSL}.} The Truth-functional Expressive Completeness Theorem:}
Any truth-functional connective of any fixed number of arguments (ternary, quadernary, etc) is already expressible in \GSL{}.
\end{majorILnc}

\begin{majorILnc}{\textbf{Theorem \mvref{PrenexNFTheorem}.} Prenex Normal Form Theorem:}
For all sentence $\CAPTHETA$ of \GQL{}, there is a provably equivalent sentence $\CAPTHETA^*$ in prenex normal form; that is, $\CAPTHETA^*$ is in prenex normal form and $\sststile{}{}\triplebar{\CAPTHETA}{\CAPTHETA^*}$ in \GQD{}.
\end{majorILnc}

\section*{Theorems about Truth, Logical Truth, and Entailment}

\begin{majorILnc}{\textbf{Theorem \mvref{GSL compositionality}.}}
For every interpretation $\IntA$, definition \mvref{True on a GSL interpretation} fixes a unique truth value on $\IntA$ for every sentence $\CAPPHI$ of \GSL{}. %; i.e. whether a sentence is $\True$ on some interpretation $\IntA$ depends only on what $\IntA$ assigns to the atomic sentences in that sentence.
\end{majorILnc}

%\begin{majorILnc}{\textbf{Theorem \ref{GSL compositionality}$^*$ (on page \pageref{GSL compositionality Star}).}}
%For any interpretation $\IntA$, the clauses in definition \mvref{True on a GSL interpretation} fix a unique truth value ($\True$ or $\False$), on $\IntA$, for every official \GSL{} sentence.
%\end{majorILnc}

\begin{majorILnc}{\textbf{Theorem \mvref{thm:localityoftruth}.}}
If two interpretations $\IntA$ and $\IntA'$ agree on all of the sentence letters of $\CAPPHI$, then $\CAPPHI$ is $\True$ in $\IntA$ \Iff $\CAPPHI$ is $\True$ in $\IntA'$.
\end{majorILnc}

\begin{majorILnc}{\textbf{Theorem \mvref{entailmentTFT theorem}.}}
For all \GSL{} sentences $\CAPTHETA$, $\:\sdtstile{}{}\CAPTHETA$ \Iff $\CAPTHETA$ is \CAPS{tft}.
\end{majorILnc} 

\begin{majorILnc}{\textbf{Theorem \mvref{Exponentiation of Entailment}.} \GSL{} Exportation Theorem:} 
For all \GSL{} sentences $\CAPPHI$ and $\CAPTHETA$, $\CAPPHI\sdtstile{}{}\CAPTHETA$ \Iff $\:\sdtstile{}{}\parhorseshoe{\CAPPHI}{\CAPTHETA}$.
\end{majorILnc}

\begin{majorILnc}{\textbf{Theorem \mvref{Exponentiation of Entailment GQL}.} \GQL{} Exportation Theorem:} 
For all \GQL{} sentences $\CAPPHI$ and $\CAPTHETA$, $\CAPPHI\sdtstile{}{}\CAPTHETA$ \Iff $\:\sdtstile{}{}\parhorseshoe{\CAPPHI}{\CAPTHETA}$.
\end{majorILnc}

\noindent{}See also theorem \mvref{expo generalizations}.

\begin{majorILnc}{\textbf{Theorem \mvref{TFE Replacement}.} Truth Functional Equivalence Replacement:}
If $\CAPPHI$ and $\CAPPHI^*$ are truth functionally equivalent, and $\CAPTHETA^*$ is the result of replacing one occurrence of $\CAPPHI$ by $\CAPPHI^*$ in $\CAPTHETA$, then $\CAPTHETA$ and $\CAPTHETA^*$ are truth functionally equivalent.
\end{majorILnc}

\begin{majorILnc}{\textbf{Theorem \mvref{The Assignment Theorem}.}}
For all sentences $\CAPPHI$ of \GQL{} and interpretations $\IntA$, $\CAPPHI$ is true on $\IntA$ relative to at least one assignment $\As{}{}$ \Iff it's true on $\IntA$ relative to all assignments $\As{}{}$.
\end{majorILnc}

\begin{majorILnc}{\textbf{Theorem \mvref{The Generalized Assignment Theorem}.}}
\begin{cenumerate}
\item For all formulas $\CAPPHI$ of \GQL{} with at most variable $\ALPHA$ free and for all interpretations $\IntA$ and assignments $\As{}{}$, $\CAPPHI$ is true on $\IntA$ relative to the assignment $\As{}{}$ \Iff it's true on $\IntA$ relative to every interpretation $\As{}{}\'$ that matches $\As{}{}$ on $\ALPHA$.
\item For all formulas $\CAPPHI$ of \GQL{} that have at most variables $\ALPHA_1,\ldots,\ALPHA_{\integer{n}}$ free and for all interpretations $\IntA$ and assignments $\As{}{}$, $\CAPPHI$ is true on $\IntA$ relative to the assignment $\As{}{}$ \Iff it's true on $\IntA$ relative to all assignments $\As{}{}\'$ that match $\As{}{}$ on all of $\ALPHA_1,\ldots,\ALPHA_{\integer{n}}$.
\end{cenumerate}
\end{majorILnc}

\begin{majorILnc}{\textbf{Theorem \mvref{Partial Assignment Corollary}.}}
If $\ALPHA$ is the only free variable of a formula $\CAPPHI$, then a partial $\ALPHA$-assignment on an interpretation $\IntA$ is sufficient, given definition \mvref{Truth for GQL Formula}, to fix a truth value for $\CAPPHI$ on $\IntA$.
\end{majorILnc}

\begin{majorILnc}{\textbf{Theorem \mvref{GQL Truth Corollary}.}} For all \GQL{} sentences $\CAPPHI,\CAPPHI_1,\ldots,\CAPPHI_{\integer{n}},\CAPPSI$ and for all interpretations $\IntA$,
\begin{cenumerate}
\item $\parhorseshoe{\CAPPHI}{\CAPPSI}$ is true in $\IntA$ iff: if $\CAPPHI$ is true in $\IntA$, then $\CAPPSI$ is true in $\IntA$.
\item $\partriplebar{\CAPPHI}{\CAPPSI}$ is true in $\IntA$ iff: $\CAPPHI$ is true in $\IntA$ \Iff $\CAPPSI$ is true in $\IntA$.
\item $\parconjunction{\CAPPHI_1}{\conjunction{\CAPPHI_2}{\conjunction{\ldots}{\CAPPHI_{\integer{n}}}}}$ is true in $\IntA$ \Iff all of the conjuncts $\CAPPHI_1,\CAPPHI_2,\ldots,\CAPPHI_{\integer{n}}$ are true in $\IntA$.
\item $\pardisjunction{\CAPPHI_1}{\disjunction{\CAPPHI_2}{\disjunction{\ldots}{\CAPPHI_{\integer{n}}}}}$ is true in $\IntA$ \Iff at least one disjunct $\CAPPHI_1,\CAPPHI_2,\ldots,\CAPPHI_{\integer{n}}$ is true in $\IntA$.
\item $\negation{\CAPPHI}$ is true in $\IntA$ \Iff $\CAPPHI$ is not true in $\IntA$.
\end{cenumerate}
\end{majorILnc}

\begin{majorILnc}{\textbf{Theorem \mvref{Quantifier Corollary}.}} For all \GQL{} formulas $\CAPPHI$ that at most have $\ALPHA$ free and for all interpretations $\IntA$,
\begin{cenumerate}
\item $\universal{\ALPHA}\CAPPHI$ is true in $\IntA$ \Iff one of the following two equivalent conditions holds:
\begin{enumerate}
\item $\CAPPHI$ is true in $\IntA$ on every partial $\ALPHA$-assignment $\As{}{}$.
\item there is no partial $\ALPHA$-assignment $\As{}{}$ on which $\CAPPHI$ is false in $\IntA$.
\end{enumerate} 
\item $\existential{\ALPHA}\CAPPHI$ is true in $\IntA$ \Iff there is at least one partial $\ALPHA$-assignment $\As{}{}$ on which $\CAPPHI$ is true.
\end{cenumerate}
\end{majorILnc}

\begin{majorILnc}{\textbf{Theorem \mvref{Quantifier Corollary B}.}}
For all \GQL{} formulas $\CAPPHI$ that at most have $\ALPHA_1,\ldots,\ALPHA_{\integer{n}}$ free and for all interpretations $\IntA$, 
\begin{cenumerate}
\item $\universal{\ALPHA_1}\ldots\universal{\ALPHA_{\integer{n}}}\CAPPHI$ is true in $\IntA$ \Iff one of the following two equivalent conditions holds:
\begin{enumerate}
\item $\CAPPHI$ is true in $\IntA$ on every partial $(\ALPHA_1,\ldots,\ALPHA_{\integer{n}})$-assignment $\As{}{}$.
\item there is no partial $(\ALPHA_1,\ldots,\ALPHA_{\integer{n}})$-assignment $\As{}{}$ on which $\CAPPHI$ is false in $\IntA$.
\end{enumerate}
\item $\existential{\ALPHA_1}\ldots\existential{\ALPHA_{\integer{n}}}\CAPPHI$ is true in $\IntA$ \Iff there's at least one partial $(\ALPHA_1,\ldots,\ALPHA_{\integer{n}})$-assignment $\As{}{}$ on which $\CAPPHI$ is true.
\end{cenumerate}
\end{majorILnc}


\begin{majorILnc}{\textbf{Theorem \mvref{The Dragnet Theorem}.} The Dragnet Theorem:}
If 
\begin{cenumerate}
\item $\CAPPHI$ contains no occurrences, whether bound or unbound, of term $\variable{s}$, and $\CAPPHI^*=\CAPPHI\variable{s}/\variable{t}$, and
\item $\As{}{}$ and $\As{*}{}$ are variable assignments with respect to interpretation $\IntA$ and $\IntA^*$ that differ only in that what $\As{}{}$ assigns to $\variable{t}$, $\As{*}{}$ assigns to $\variable{s}$,
\end{cenumerate} 
then: $\As{}{}$ makes $\CAPPHI$ true on $\IntA$ iff $\As{*}{}$ makes $\CAPPHI^*$ true on $\IntA^*$.
\end{majorILnc}

\begin{majorILnc}{\textbf{Theorem \mvref{Monotonicity of Entailment}.} Monotonicity of Entailment:}
For all \GSL{} sentences $\CAPPHI_1,\ldots,\CAPPHI_{\integer{n}},\CAPTHETA,\CAPPSI$:
\begin{center}
If $\CAPPHI_1,\CAPPHI_2,\ldots,\CAPPHI_{\integer{n}}\sdtstile{}{}\CAPPSI$, then $\CAPPHI_1,\CAPPHI_2,\ldots,\CAPPHI_{\integer{n}},\CAPTHETA\sdtstile{}{}\CAPPSI$
\end{center}
\end{majorILnc}

\begin{majorILnc}{\textbf{Theorem \mvref{Transitivity of Entailment}.} Transitivity of Entailment:}
For all \GSL{} sentences $\CAPPHI_1,\ldots,\CAPPHI_{\integer{n}}$, $\CAPTHETA$, and $\CAPPSI_1,\ldots,\CAPPSI_{\integer{k}}$:
\begin{center}
\begin{tabular}{ l@{\hspace{.25em}}l@{\hspace{.25em}}l }
If & $\CAPPHI_1,\CAPPHI_2,\ldots,\CAPPHI_{\integer{n}}\sdtstile{}{}\CAPPSI_1$ & and \\
   & $\CAPPHI_1,\CAPPHI_2,\ldots,\CAPPHI_{\integer{n}}\sdtstile{}{}\CAPPSI_2$ & and \\
   & \hspace{.5in} $\vdots$ &  \\
   & $\CAPPHI_1,\CAPPHI_2,\ldots,\CAPPHI_{\integer{n}}\sdtstile{}{}\CAPPSI_{\integer{k}}$ & and \\
   & $\CAPPSI_1,\CAPPSI_2,\ldots,\CAPPSI_{\integer{k}}\sdtstile{}{}\CAPTHETA$ & then: \\
   & & $\CAPPHI_1,\CAPPHI_2,\ldots,\CAPPHI_{\integer{n}}\sdtstile{}{}\CAPTHETA$   \\
\end{tabular}
\end{center}
\end{majorILnc}

\section*{Derivations}

\begin{majorILnc}{\textbf{Theorem \mvref{Soundess of Basic GSD Rules}.}}
Every application of every basic rule of GSD is truth-preserving, or sound.
\end{majorILnc}

\begin{majorILnc}{\textbf{Theorem \mvref{Soundness of Std Shortcut Applications}.}}
Every application of every shortcut rule from \GSDP{}, including both standard and exchange shortcut rules, is truth-preserving.
\end{majorILnc}

\begin{majorILnc}{\textbf{Theorem \mvref{ExchangeRuleGSDSoundness}.}}
Every application of every exchange shortcut rule from \GSDP{} is truth-preserving, even if we extend the notion of sanctioning for them with definition \ref{ExchangeRuleSanctioning}. 
\end{majorILnc} 

\begin{majorILnc}{\textbf{Theorem \mvref{GSD Shortcut Theorem}.}}
For all \GSL{} sentences $\CAPTHETA_1,\ldots,\CAPTHETA_{\integer{n}},\DELTA$ and rules \Rule{R$_1$}$,\ldots,$\Rule{R$_\integer{p}$}, if
\begin{cenumerate}
\item $\DELTA$ can be derived from $\CAPTHETA_1,\ldots,\CAPTHETA_{\integer{n}}$ using rules \Rule{R$_1$}$,\ldots,$\Rule{R$_\integer{p}$} and the basic rules of \GSD{}, and
\item every application of a rule \Rule{R$_1$} is derivable using the rules \Rule{R$_2$}, $\ldots$, \Rule{R$_\integer{p}$} and the basic rules of \GSD{},
\end{cenumerate}
then $\DELTA$ can be derived from $\CAPTHETA_1,\ldots,\CAPTHETA_{\integer{n}}$ using only rules \Rule{R$_2$}$,\ldots,$\Rule{R$_\integer{p}$} and the basic rules of \GSD{}.
%Any sentence $\CAPPHI$ that can be derived from the sentences $\CAPPSI_1,\ldots,\CAPPSI_{\integer{n}}$ using the basic rules plus some of the shortcut rules in tables \ref{GSDplus1} and \ref{GSDplus2} can be derived from $\CAPPSI_1,\ldots,\CAPPSI_{\integer{n}}$ using the basic rules alone. 
\end{majorILnc}

\begin{majorILnc}{\textbf{Theorem \mvref{GQD Shortcut Theorem}.}}
For all \GQL{} sentences $\CAPTHETA_1,\ldots,\CAPTHETA_{\integer{n}},\DELTA$ and rules \Rule{R$_1$}$,\ldots,$\Rule{R$_\integer{p}$}, if
\begin{cenumerate}
\item $\DELTA$ can be derived from $\CAPTHETA_1,\ldots,\CAPTHETA_{\integer{n}}$ using rules \Rule{R$_1$}$,\ldots,$\Rule{R$_\integer{p}$} and the basic rules of \GSD{}, and
\item every application of a rule \Rule{R$_1$} is derivable using the rules \Rule{R$_2$}, $\ldots$, \Rule{R$_\integer{p}$} and the basic rules of \GQD{},
\end{cenumerate}
then $\DELTA$ can be derived from $\CAPTHETA_1,\ldots,\CAPTHETA_{\integer{n}}$ using only rules \Rule{R$_2$}$,\ldots,$\Rule{R$_\integer{p}$} and the basic rules of \GQD{}.
\end{majorILnc}

\begin{majorILnc}{\textbf{Theorem \mvref{GSD Shortcut Theorem2}.}}
For all standard and exchange shortcut rules \Rule{R} (see tables \ref{GSDplus1} and \ref{GSDplus2}), every application of \Rule{R} is derivable using the basic rules of \GSD{}.
\end{majorILnc}

\begin{majorILnc}{\textbf{Theorem \mvref{GQD Shortcut Theorem2}.}}
For all standard and exchange shortcut rules \Rule{R} (see tables \ref{GSDplus1}, \ref{GSDplus2}, and \ref{GQDplus}), every application of \Rule{R} is derivable using the basic rules of \GQD{} (see tables \ref{GSD} and \ref{GQD}).
\end{majorILnc}

\begin{majorILnc}{\textbf{Theorem \mvref{GSD Shortcut Theorem3}.} Shortcut Rule Elimination Theorem:}
For all \GSL{} sentences $\CAPPHI_1,\ldots,\CAPPHI_{\integer{m}}$ and $\CAPPSI$, if $\CAPPSI$ can be derived from $\CAPPHI_1,\ldots,\CAPPHI_{\integer{m}}$ in \GSDP{} (that is, using the basic rules of \GSD{} and any of the standard and exchange shortcut rules), then $\CAPPSI$ can be derived from $\CAPPHI_1,\ldots,\CAPPHI_{\integer{m}}$ in \GSD{} (that is, using only the basic rules).
\end{majorILnc}

\begin{majorILnc}{\textbf{Theorem \mvref{GQD Shortcut Theorem3}.} Shortcut Rule Elimination Theorem for \GQDP{}:}
For all \GQL{} sentences $\CAPPHI_1,\ldots,\CAPPHI_{\integer{m}}$ and $\CAPPSI$, if $\CAPPSI$ can be derived from $\CAPPHI_1,\ldots,\CAPPHI_{\integer{m}}$ in \GQDP{}, then $\CAPPSI$ can be derived from $\CAPPHI_1,\ldots,\CAPPHI_{\integer{m}}$ in \GQD{}.
\end{majorILnc}

\begin{majorILnc}{\textbf{Theorem \mvref{ExchangeRuleGSDSoundnessLemma}.}}
For all exchange shortcut rules \Rule{R} from \GSDP{}, if $\CAPPHI$ and $\CAPPHI^*$ are the sentences you get after substituting \GSL{} sentences into the given and may-add schemas of \Rule{R}, respectively, then $\CAPPHI$ and $\CAPPHI^*$ are truth functionally equivalent. 
\end{majorILnc}

\begin{majorILnc}{\textbf{Theorem \mvref{ExchangeRuleTheorem}.} Restricted Replacement Theorem for \GSD{}:}
For all sentences $\CAPPSI$ of \GSL{}: if
\begin{cenumerate}
\item $\CAPPHI$ and $\CAPPHI^*$ are \GSL{} sentences such that $\CAPPHI\sststile{}{}\CAPPHI^*$ and $\CAPPHI^*\sststile{}{}\CAPPHI$, and
\item if $\CAPPHI$ is a subsentence of $\CAPPSI$, then $\CAPPSI^*$ is the \GSL{} sentence you get when you replace one instance (token) of $\CAPPHI$ with an instance (token) of $\CAPPHI^*$, and $\CAPPSI^*$ is $\CAPPSI$ if not, 
\end{cenumerate}
then $\CAPPSI^*$ can be derived from $\CAPPSI$ using only the basic rules of \GSD{}, i.e. $\CAPPSI\sststile{}{}\CAPPSI^*$.
\end{majorILnc}

\begin{majorILnc}{\textbf{Theorem \mvref{GQD Replacement Theorem}.} The Replacement Theorem for \GQD{}:}
If $\CAPPHI$ and $\CAPPHI^*$ are provably equivalent formulas of \GQL{}, and $\CAPTHETA$ and $\CAPTHETA^*$ differ only in that $\CAPTHETA$ contains the subformula $\CAPPHI$ in one place where $\CAPTHETA^*$ contains the subformula $\CAPPHI^*$, then $\CAPTHETA$ and $\CAPTHETA^*$ are provably equivalent.
\end{majorILnc}

\noindent{}Recall that we used the One-step Replacement Lemmas (theorem \ref{OneStepReplacementLemmas}) to prove this.

%\begin{majorILnc}{\textbf{Theorem \ref{OneStepReplacementLemmas}.} One-step Replacement Lemmas:}
%If $\sststile{}{}\forall\partriplebar{\CAPPHI}{\CAPPHI^*}$, then:
%\begin{cenumerate}
%\item $\sststile{}{}\forall\partriplebar{\negation{\CAPPHI}}{\negation{\CAPPHI^*}}$
%\item\label{exampleonesteplemma}
%$\sststile{}{}\forall\partriplebar{\parconjunction{\CAPPHI}{\conjunction{\CAPPHI_1}{\conjunction{\ldots}{\CAPPHI_{\integer{p}}}}}}{\parconjunction{\CAPPHI^*}{\conjunction{\CAPPHI_1}{\conjunction{\ldots}{\CAPPHI_{\integer{p}}}}}}$
%\item[] \hspace{1in} $\vdots$
%\item $\sststile{}{}\forall\partriplebar{\parconjunction{\CAPPHI_1}{\conjunction{\ldots}{\conjunction{\CAPPHI_{\integer{p}}}{\CAPPHI}}}}{\parconjunction{\CAPPHI_1}{\conjunction{\ldots}{\conjunction{\CAPPHI_{\integer{p}}}{\CAPPHI^*}}}}$
%\item 
%$\sststile{}{}\forall\partriplebar{\pardisjunction{\CAPPHI}{\disjunction{\CAPPHI_1}{\disjunction{\ldots}{\CAPPHI_{\integer{p}}}}}}{\pardisjunction{\CAPPHI^*}{\disjunction{\CAPPHI_1}{\disjunction{\ldots}{\CAPPHI_{\integer{p}}}}}}$
%\item[] \hspace{1in} $\vdots$
%\item $\sststile{}{}\forall\partriplebar{\pardisjunction{\CAPPHI_1}{\disjunction{\ldots}{\disjunction{\CAPPHI_{\integer{p}}}{\CAPPHI}}}}{\pardisjunction{\CAPPHI_1}{\disjunction{\ldots}{\disjunction{\CAPPHI_{\integer{p}}}{\CAPPHI^*}}}}$
%\item $\sststile{}{}\forall\bpartriplebar{\parhorseshoe{\CAPPHI}{\CAPPSI}}{\parhorseshoe{\CAPPHI^*}{\CAPPSI}}$
%\item $\sststile{}{}\forall\bpartriplebar{\parhorseshoe{\CAPPSI}{\CAPPHI}}{\parhorseshoe{\CAPPSI}{\CAPPHI^*}}$
%\item $\sststile{}{}\forall\bpartriplebar{\partriplebar{\CAPPHI}{\CAPPSI}}{\partriplebar{\CAPPHI^*}{\CAPPSI}}$
%\item $\sststile{}{}\forall\bpartriplebar{\partriplebar{\CAPPSI}{\CAPPHI}}{\partriplebar{\CAPPSI}{\CAPPHI^*}}$
%\end{cenumerate}
%And, if $\sststile{}{}\forall\universal{\BETA}\bpartriplebar{\CAPPHI}{\CAPPHI^*}$, then:
%\begin{enumerate}[label=(\arabic*), leftmargin=1.85\parindent,
%labelindent=.35\parindent, labelsep=*, itemsep=0pt, start=10]%
%\item $\sststile{}{}\forall\bpartriplebar{\universal{\BETA}\CAPPHI}{\universal{\BETA}\CAPPHI^*}$
%\item $\sststile{}{}\forall\bpartriplebar{\existential{\BETA}\CAPPHI}{\existential{\BETA}\CAPPHI^*}$
%\end{enumerate}
%\end{majorILnc}

\begin{majorILnc}{\textbf{Theorem \mvref{Non-decreasing Assumption Principle}.} Non-decreasing Assumption Principle (NDAP):}
If $\Delta_1$ is the set of assumptions of an unboxed line and $\Delta_2$ is the set of assumptions of a later unboxed line, then $\Delta_1$ is a subset of $\Delta_2$, i.e., $\Delta_1\subseteq\Delta_2$.
\end{majorILnc}

\begin{majorILnc}{\textbf{Theorem \mvref{GQD NDF Rule}.}}
Any two \GQL{} formulas got by substituting other \GQL{} formulas into the may-add and given schemas of \Rule{$\TRIPLEBAR$-Exchange} are provably equivalent; that is, $\sststile{}{}\forall\bpartriplebar{\partriplebar{\CAPTHETA}{\CAPPSI}}{\pardisjunction{\parconjunction{\CAPTHETA}{\CAPPSI}}{\parconjunction{\negation{\CAPTHETA}}{\negation{\CAPPSI}}}}$.
\end{majorILnc}

\begin{majorILnc}{\textbf{Theorem \mvref{GQD NDF Rule2}.}}
For all Prenex Exchange Rules \Rule{R}, any two \GQL{} formulas got by substituting other \GQL{} formulas into the may-add and given schemas of \Rule{R} are provably equivalent.
\end{majorILnc}

\section*{Soundness and Completeness}

\begin{majorILnc}{\textbf{Theorem \mvref{RegWeakCompletenessEquiv}.}}
\GSD{} is weakly complete \Iff it's complete; and likewise for \GQD{}.
\end{majorILnc}

\begin{majorILnc}{\textbf{Theorem \mvref{Soundness of Sentential Logic}.} \GSD{} Soundness Theorem:}
\GSD{} is sound; i.e., for every set $\Delta$ of sentences of \GSL{} and every sentence $\CAPPHI$ of \GSL{}, if $\Delta\sststile{}{}\CAPPHI$ in \GSD{}, then $\Delta\sdtstile{}{}\CAPPHI$.
\end{majorILnc}

\begin{majorILnc}{\textbf{Theorem \mvref{Main GSL Soundness Lemma}.} Soundness Lemma:}
For any sequence of derivation lines that is a derivation, the sentence $\CAPPHI$ on the last line is entailed by the set $\Delta$ of sentences that are on unboxed lines and are sanctioned by \Rule{Assumption}. 
\end{majorILnc}

\noindent{}Recall that we used theorems \ref{Monotonicity of Entailment}, \ref{Transitivity of Entailment}, and \ref{Non-decreasing Assumption Principle} to prove the Soundness Lemma.

\begin{majorILnc}{\textbf{Theorem \mvref{Soundness of Quantifier Logic}.} \GQD{} Soundness Theorem:}
\GQD{} is sound; i.e., for every set $\Delta$ of sentences of \GQL{} and every sentence $\CAPPHI$ of \GQL{}, if $\Delta\sststile{}{}\CAPPHI$ in \GSD{}, then $\Delta\sdtstile{}{}\CAPPHI$.
\end{majorILnc}

\begin{majorILnc}{\textbf{Theorem \mvref{GSDCompletenessLemma}.} The \GSD{} Weak Completeness Lemma:}
For\index{completeness!weak \GSD{}} any sentence $\CAPPHI$ of \GSD{}, either $\CAPPHI\sststile{}{}\conjunction{\Al}{\negation{\Al}}$, or $\CAPPHI$ is true in some interpretation $\IntA$.
\end{majorILnc}

\begin{majorILnc}{\textbf{Theorem \mvref{GSDWCompleteness}.} Weak \GSD{} Completeness Theorem:}
For all \GSL{} sentences $\CAPPHI$: if $\sdtstile{}{}\CAPPHI$, then $\sststile{}{}\CAPPHI$ in \GSD{}.
\end{majorILnc}

\begin{majorILnc}{\textbf{Theorem \mvref{GSDCompleteness}.} \GSD{} Completeness Theorem:}
For every finite set $\Delta$ of sentences of \GSL{} and every sentence $\CAPPHI$ of \GSL{}, if $\Delta\sdtstile{}{}\CAPPHI$, then $\Delta\sststile{}{}\CAPPHI$ in \GSD{}.
\end{majorILnc}

\begin{majorILnc}{\textbf{Theorem \mvref{Derivational Lemma}.} Derivational Lemma:}
If the Method starts with $\negation{\CAPTHETA}$ and produces a contradiction, then there is a derivation of $\CAPTHETA$.
\end{majorILnc}

\begin{majorILnc}{\textbf{Theorem \mvref{DerivationalLemmaS}.} Strong Derivational Lemma:}
If the strong method halts in a contradiction, then $\Delta\sststile{}{}\CAPPHI$.
\end{majorILnc}

\begin{majorILnc}{\textbf{Theorem \mvref{MethodLemmaA}.} The Method Lemma 1:}
The matrix model $\IntA_M$ makes true all sentences on the master matrix list $M$ generated by the method when it doesn't halt in contradiction.
\end{majorILnc}

\begin{majorILnc}{\textbf{Theorem \mvref{MethodSLemmaA}.} The Strong Method Lemma 1:}
The matrix model $\IntA_M$ makes true all sentences on the master matrix list $M$ generated by the strong method when it doesn't halt in contradiction.
\end{majorILnc}

\begin{majorILnc}{\textbf{Theorem \mvref{MethodLemmaB}.} The Method Lemma 2:}
All matrix instances in the derivation generated by the method are true in the matrix model $\IntA_M$.
\end{majorILnc}

\begin{majorILnc}{\textbf{Theorem \mvref{MethodSLemmaB}.} The Strong Method Lemma 2:}
All matrix instances in the derivation generated by the strong method are true in the matrix model $\IntA_M$.
\end{majorILnc}

\begin{majorILnc}{\textbf{Theorem \mvref{MethodLemmaC}.} The Method Lemma 3:}
All quantified sentences in the derivation generated by the method are true in the matrix model $\IntA_M$.
\end{majorILnc}

\begin{majorILnc}{\textbf{Theorem \mvref{MethodSLemmaC}.} The Strong Method Lemma 3:}
All quantified sentences in the derivation generated by the strong method are true in the matrix model $\IntA_M$.
\end{majorILnc}

\begin{majorILnc}{\textbf{Theorem \mvref{MainGQDWCompletenessLemma}.} Main Weak \GQD{} Completeness Lemma:}
For all sentences $\CAPTHETA$ of \GQL{}, if the method is applied to $\negation{\CAPTHETA}$ then either: (a) the method produces a derivation of $\CAPTHETA$ in \GQDP{}, or (b) an interpretation $\IntA$ can be read off which makes $\CAPTHETA$ false.
\end{majorILnc}

\begin{majorILnc}{\textbf{Theorem \mvref{MainGQDSCompletenessLemma}.} Main Strong \GQD{} Completeness Lemma:}
For all sets of sentences $\Delta$ of sentences of \GQL{} and \GQL{} sentences $\CAPPHI$, if the strong method is applied to $\Delta^*=\Delta\cup\{\negation{\CAPPHI}\}$ then either: (a) the strong method produces a derivation of $\CAPPHI$ from $\Delta$ in \GQDP{}, or (b) an interpretation $\IntA$ can be read off which makes every sentence in $\Delta$ true and $\CAPPHI$ false.
\end{majorILnc}

\begin{majorILnc}{\textbf{Theorem \mvref{GQDWeakCompletenessTheorem}.} Weak \GQD{} Completeness Theorem:}
For all sentences $\CAPTHETA$ of \GQL{}, if $\sdtstile{}{}\CAPTHETA$, then $\sststile{}{}\CAPTHETA$ in \GQD{}.
\end{majorILnc}

\begin{majorILnc}{\textbf{Theorem \mvref{GQDCompletenessTheorem}.} \GQD{} Completeness Theorem:}
For all finite sets $\Delta$ of \GQL{} sentences and \GQL{} sentence $\CAPPHI$, if $\Delta\sdtstile{}{}\CAPPHI$, then $\Delta\sststile{}{}\CAPPHI$.
\end{majorILnc}

\begin{majorILnc}{\textbf{Theorem \mvref{GQDStrongCompletenessTheorem}.} Strong \GQD{} Completeness Theorem:}
For any set $\Delta$ of \GSL{} sentences and any \GSL{} sentence $\CAPPHI$, if $\Delta\sdtstile{}{}\CAPPHI$, then $\Delta\sststile{}{}\CAPPHI$.
\end{majorILnc}

\section*{Decidability, Compactness, and L\"owenheim-Skolem}

\begin{majorILnc}{\textbf{Theorem \mvref{ChurchsTheorem}.} Church's Theorem:}
If \Language{L} is a sublanguage of \GQL{} with at least one 2-place predicate symbol, then there is no decision procedure for the set of logical truths of \Language{L}.
\end{majorILnc}

\begin{majorILnc}{\textbf{Theorem \mvref{MonadicGQLEquivTheorem}.} Monadic \GQL{} Equivalence Theorem:}
Every sentence of monadic \GQL{} is quantificationally equivalent to a sentence whose quantifiers are independent.
\end{majorILnc}

\begin{majorILnc}{\textbf{Theorem \mvref{MonadicDecisionTheorem}.} The Monadic Decision Theorem:}
The\index{Monadic Decision Theorem, The} modified method just described provides a decision procedure for quantificational truth in monadic \GQL{}.
\end{majorILnc}

\begin{majorILnc}{\textbf{Theorem \mvref{LowenheimSkolemTheorem}.} The Downward L\"owenheim-Skolem Theorem:}
If a sentence of \GQL{} is true in any interpretation, then it is true in one whose domain consists of all of some of the natural numbers.
\end{majorILnc}

\begin{majorILnc}{\textbf{Theorem \mvref{MonadicIntSizeTheorem}.}}
If $\CAPPHI$ is a sentence of monadic \GQL{} and has an interpretation, then it has a finite interpretation.
\end{majorILnc}

\begin{majorILnc}{\textbf{Theorem \mvref{Thm:CompactnessTheorem}.} The Compactness Theorem for \GQL{}:}
For all sets of sentences $\Delta$ of \GQL{}, if for every finite subset $\Delta'$ of $\Delta$ there exists an interpretation $\IntA'$ that makes all the sentences in $\Delta'$ true, then there's some interpretation $\IntA$ that makes all the sentences in $\Delta$ true. 
\end{majorILnc}

\section*{Many-valued, Modal, and Quantifier Logic with Identity}

\begin{majorILnc}{\textbf{Theorem \mvref{GQLIdentityTheorem}.}}
There exists no (possibly countably infinite) set of formulas $\Delta$ of \GQL{} each of which at most has the variables $\ALPHA$ and $\BETA$ free such that:
\begin{quote}
For any two constants $\variable{t}$ and $\variable{s}$, if $\Delta^*$ is the set of sentences got by substituting $\variable{t}$ for all occurrences of $\ALPHA$ and $\variable{s}$ for all occurrences of $\BETA$ in every sentence of $\Delta$, then: for all interpretations $\IntA$, $\IntA$ makes every sentence of $\Delta^*$ true iff $\IntA(\variable{t})=\IntA(\variable{s})$.
\end{quote}
\end{majorILnc}

\begin{majorILnc}{\textbf{Theorem \mvref{IdentityLemma}.}}
If $\Delta$ is (at most) a countably infinite set of \GQL{} sentences that's consistent (i.e., there's at least one interpretation $\IntA$ that makes every sentence in $\Delta$ true) and the constants $\variable{t}$ and $\variable{s}$ each appear at least once in one of the sentences of $\Delta$, then there's an interpretation $\IntA$ which makes every sentence in $\Delta$ true such that $\IntA(\variable{t})\neq\IntA(\variable{s})$.
\end{majorILnc}

\begin{majorILnc}{\textbf{Theorem \mvref{GQDISoundness}.} \GQDI{} Soundness Theorem:}
For\index{soundness!of \GQDI{}} all sentences $\CAPPHI$ in \GQLI{} and sets of sentences $\Delta$, if $\Delta\sststile{}{}\CAPPHI$ in \GQDI{}, then $\Delta\sdtstile{}{}\CAPPHI$.
\end{majorILnc}

\begin{majorILnc}{\textbf{Theorem \mvref{GQDIStrongCompleteness}.} \GQDI{} Strong Completeness Theorem:}
For\index{completeness!of \GQDI{}} all sentences $\CAPPHI$ in \GQLI{} and sets of sentences $\Delta$, if $\Delta\sdtstile{}{}\CAPPHI$ in \GQDI{}, then $\Delta\sststile{}{}\CAPPHI$.
\end{majorILnc}

%\theendnotes


%%%%%%%%%%%%%%%%%%%%%%%%%%%%%%%%%%%%%%%%%%%%%%%%%%
\chapter*{Appendix B: List of Derivation Rules}
\addcontentsline{toc}{chapter}{Appendix B: List of Derivation Rules}
%%%%%%%%%%%%%%%%%%%%%%%%%%%%%%%%%%%%%%%%%%%%%%%%%%
%\fancyhead[RE,LO]{\textsf{Appendix B: List of Derivation Rules}}
\chead{\textsf{Appendix B: List of Derivation Rules}}
\fancyhead[LE,RO]{\textsf{\thepage}}
\setcounter{section}{0}

\begin{tabular}{ l l }
\multicolumn{2}{l}{\textbf{Derivation Systems:}}\\
\GSD{}& Set 1\\
\GSDP{}& Sets 1,2,3\\
\GQD{}& Sets 1,4\\
\GQDP{}& Sets 1,2,3,4,5\\
\GQDPP{}& Sets 1,2,3,4,5,6,7,8\\
\SF{}& Sets 1,9\\
\SFP{}& Sets 1,2,3,9,10\\
\GQDI{}& Sets 1,4,11\\
\GQDIP{}& Sets 1,2,3,4,5,11\\
\end{tabular}

%\begin{table}[!ht]
\renewcommand{\arraystretch}{1.5}
%\begin{center}
\begin{longtable}[c]{ p{1in} l l } %p{2.2in} p{2in}
\toprule
\textbf{Name} & \textbf{Given} & \textbf{May Add} \\ 
\midrule
\endfirsthead
%\multicolumn{3}{c}{\emph{Continued from Previous Page}}\\
\toprule
\textbf{Name} & \textbf{Given} & \textbf{May Add} \\ 
\midrule
\endhead
%\bottomrule
%\caption{Basic Rules of \GSD{}}\\[-.15in]
%\multicolumn{3}{c}{\emph{Continued next Page}}\\
\endfoot
\bottomrule
%\caption{Basic Rules of \GSD{}}\\%
\endlastfoot%
%\label{GSD}%
%\midrule
\multicolumn{3}{l}{\textbf{Set 1: Basic Rules of GSD, table \mvref{GSD}}}\\
%\midrule
\Rule{Ass.} & & | $\CAPPHI$ \\
\Rule{Rep.} & $\CAPPHI$ & $\CAPPHI$ \\
\Rule{$\HORSESHOE$-Elim} & $\horseshoe{\CAPTHETA}{\CAPPSI}$, $\CAPTHETA$ & $\CAPPSI$ \\
\Rule{$\HORSESHOE$-Intro} &  | $\CAPTHETA\Rightarrow\CAPPSI$ & $\horseshoe{\CAPTHETA}{\CAPPSI}$, Box $|\CAPTHETA\Rightarrow\CAPPSI$ \\
\Rule{$\!\WEDGE\!$-Elim} &{}$\conjunction{\CAPTHETA_1}{\conjunction{\CAPTHETA_2}{\conjunction{\ldots}{\CAPTHETA_{\integer{n}}}}}$&{}Conjunction of any proper\\[-.25cm]
 & &{}subset of the conjuncts\\
\Rule{$\!\WEDGE\!$-Intro} & $\CAPTHETA_1$, $\CAPTHETA_2$, $\ldots$ $\CAPTHETA_{\integer{n}}$ & $\conjunction{\CAPTHETA_1}{\conjunction{\CAPTHETA_2}{\conjunction{\ldots}{\CAPTHETA_{\integer{n}}}}}$ \\
\Rule{$\VEE$-Elim} & $\disjunction{\CAPTHETA_1}{\disjunction{\CAPTHETA_2}{\disjunction{\ldots}{\CAPTHETA_{\integer{n}}}}}$, &  \\
 &  $\horseshoe{\CAPTHETA_1}{\CAPPSI}$,  &  \\
 &  $\horseshoe{\CAPTHETA_2}{\CAPPSI}$,  &  \\
 &  $\vdots$  &  \\
 &  $\horseshoe{\CAPTHETA_{\integer{n}}}{\CAPPSI}$ & $\CAPPSI$ \\
\Rule{$\VEE$-Intro} & $\CAPTHETA$ & $\disjunction{\CAPPSI_1}{\disjunction{\CAPPSI_2}{\disjunction{\ldots}{\CAPPSI_{\integer{n}}}}}$, \\[-.25cm]
\nopagebreak
 &  & where $\CAPTHETA=\CAPPSI_i$ for some $i$. \\
\Rule{$\NEGATION$-Intro} & $\horseshoe{\CAPTHETA}{\parconjunction{\CAPPSI}{\negation{\CAPPSI}}}$ & $\negation{\CAPTHETA}$ \\
\Rule{$\NEGATION$-Elim} & $\horseshoe{\negation{\CAPTHETA}}{\parconjunction{\CAPPSI}{\negation{\CAPPSI}}}$ & $\CAPTHETA$ \\
\Rule{$\TRIPLEBAR$-Intro} & $\horseshoe{\CAPTHETA}{\CAPPSI}$, $\horseshoe{\CAPPSI}{\CAPTHETA}$ & $\triplebar{\CAPTHETA}{\CAPPSI}$ \\
\Rule{$\TRIPLEBAR$-Elim} & $\triplebar{\CAPTHETA}{\CAPPSI}$, $\CAPPSI$ & $\CAPTHETA$ \\
\Rule{$\TRIPLEBAR$-Elim} & $\triplebar{\CAPTHETA}{\CAPPSI}$, $\CAPTHETA$ & $\CAPPSI$ \\
%\bottomrule
%\midrule
\multicolumn{3}{l}{\textbf{Set 2: Standard Shortcut Rules of GSD, table \mvref{GSDplus1}}}\\
\nopagebreak
%\midrule
\Rule{M.T.} & $\horseshoe{\CAPPHI}{\CAPTHETA}$, $\negation{\CAPTHETA}$ & $\negation{\CAPPHI}$ \\
\Rule{D.S.} & $\disjunction{\CAPPHI_1}{\disjunction{\ldots}{\disjunction{\CAPPHI_i}{\disjunction{\ldots}{\CAPPHI_{\integer{n}}}}}}$, $\negation{\CAPPHI_i}$ & $\disjunction{\CAPPHI_1}{\disjunction{\ldots}{\disjunction{\CAPPHI_{i-1}}{\disjunction{\CAPPHI_{i+1}}{\disjunction{\ldots}{\CAPPHI_{\integer{n}}}}}}}$ \\
\nopagebreak
 & $\disjunction{\CAPPHI_1}{\disjunction{\ldots}{\disjunction{\negation{\CAPPHI_i}}{\disjunction{\ldots}{\CAPPHI_{\integer{n}}}}}}$, ${\CAPPHI_i}$ & $\disjunction{\CAPPHI_1}{\disjunction{\ldots}{\disjunction{\CAPPHI_{i-1}}{\disjunction{\CAPPHI_{i+1}}{\disjunction{\ldots}{\CAPPHI_{\integer{n}}}}}}}$ \\
\Rule{A.C.} & ${\CAPPHI},{\negation{\CAPPHI}}$ & $\CAPPSI$ \\
\Rule{$\NEGATION$/$\TRIPLEBAR$-Intro} & $\triplebar{\CAPPHI}{\CAPPSI}$ & $\triplebar{\negation{\CAPPHI}}{\negation{\CAPPSI}}$ \\
%\midrule
\multicolumn{3}{l}{\textbf{Set 3: Exchange Shortcut Rules of GSD, table \mvref{GSDplus2}}}\\
\nopagebreak
%\midrule
\Rule{DeM} & $\negation{\parconjunction{\CAPPHI_1}{\conjunction{\ldots}{\CAPPHI_{\integer{n}}}}}$ & $\disjunction{\negation{\CAPPHI_1}}{\disjunction{\ldots}{\negation{\CAPPHI_{\integer{n}}}}}$\\
 & $\disjunction{\negation{\CAPPHI_1}}{\disjunction{\ldots}{\negation{\CAPPHI_{\integer{n}}}}}$ & $\negation{\parconjunction{\CAPPHI_1}{\conjunction{\ldots}{\CAPPHI_{\integer{n}}}}}$\\
 & $\negation{\pardisjunction{\CAPPHI_1}{\disjunction{\ldots}{\CAPPHI_{\integer{n}}}}}$ & $\conjunction{\negation{\CAPPHI_1}}{\conjunction{\ldots}{\negation{\CAPPHI_{\integer{n}}}}}$ \\
 & $\conjunction{\negation{\CAPPHI_1}}{\conjunction{\ldots}{\negation{\CAPPHI_{\integer{n}}}}}$ & $\negation{\pardisjunction{\CAPPHI_1}{\disjunction{\ldots}{\CAPPHI_{\integer{n}}}}}$ \\
\Rule{$\NEGATION\NEGATION$-Elim} & $\negation{\negation{\CAPPHI}}$ & $\CAPPHI$ \\
\Rule{$\NEGATION\NEGATION$-Intro} & $\CAPPHI$ & $\negation{\negation{\CAPPHI}}$ \\
\Rule{$\HORSESHOE$/$\VEE$-Exchange} & $\horseshoe{\CAPPHI}{\CAPTHETA}$ & $\disjunction{\negation{\CAPPHI}}{\CAPTHETA}$ \\
\nopagebreak
 & $\disjunction{\negation{\CAPPHI}}{\CAPTHETA}$ & $\horseshoe{\CAPPHI}{\CAPTHETA}$  \\
\Rule{Contraposition} & $\horseshoe{\CAPPHI}{\CAPTHETA}$ & $\horseshoe{\negation{\CAPTHETA}}{\negation{\CAPPHI}}$ \\
 & $\horseshoe{\negation{\CAPTHETA}}{\negation{\CAPPHI}}$ & $\horseshoe{\CAPPHI}{\CAPTHETA}$ \\
\Rule{$\NEGATION$/$\HORSESHOE$-Exchange} & $\negation{\parhorseshoe{\CAPPHI}{\CAPTHETA}}$ & $\conjunction{\CAPPHI}{\negation{\CAPTHETA}}$ \\
\nopagebreak
 & $\conjunction{\CAPPHI}{\negation{\CAPTHETA}}$ & $\negation{\parhorseshoe{\CAPPHI}{\CAPTHETA}}$ \\
\Rule{Distribution} & $\conjunction{\CAPTHETA}{\pardisjunction{\CAPPHI_1}{\disjunction{\ldots}{\CAPPHI_{\integer{n}}}}}$ & $\disjunction{\parconjunction{\CAPTHETA}{\CAPPHI_1}}{\disjunction{\ldots}{\parconjunction{\CAPTHETA}{\CAPPHI_{\integer{n}}}}}$\\
\nopagebreak
 & $\disjunction{\parconjunction{\CAPTHETA}{\CAPPHI_1}}{\disjunction{\ldots}{\parconjunction{\CAPTHETA}{\CAPPHI_{\integer{n}}}}}$ & $\conjunction{\CAPTHETA}{\pardisjunction{\CAPPHI_1}{\disjunction{\ldots}{\CAPPHI_{\integer{n}}}}}$\\
\nopagebreak 
 & $\conjunction{\pardisjunction{\CAPPHI_1}{\disjunction{\ldots}{\CAPPHI_{\integer{n}}}}}{\CAPTHETA}$ & $\disjunction{\parconjunction{\CAPPHI_1}{\CAPTHETA}}{\disjunction{\ldots}{\parconjunction{\CAPPHI_{\integer{n}}}{\CAPTHETA}}}$\\
\nopagebreak 
 & $\disjunction{\parconjunction{\CAPPHI_1}{\CAPTHETA}}{\disjunction{\ldots}{\parconjunction{\CAPPHI_{\integer{n}}}{\CAPTHETA}}}$  & $\conjunction{\pardisjunction{\CAPPHI_1}{\disjunction{\ldots}{\CAPPHI_{\integer{n}}}}}{\CAPTHETA}$\\
\nopagebreak 
 & $\disjunction{\CAPTHETA}{\parconjunction{\CAPPHI_1}{\conjunction{\ldots}{\CAPPHI_{\integer{n}}}}}$ & $\conjunction{\pardisjunction{\CAPTHETA}{\CAPPHI_1}}{\conjunction{\ldots}{\pardisjunction{\CAPTHETA}{\CAPPHI_{\integer{n}}}}}$\\
\nopagebreak 
 & $\conjunction{\pardisjunction{\CAPTHETA}{\CAPPHI_1}}{\conjunction{\ldots}{\pardisjunction{\CAPTHETA}{\CAPPHI_{\integer{n}}}}}$ & $\disjunction{\CAPTHETA}{\parconjunction{\CAPPHI_1}{\conjunction{\ldots}{\CAPPHI_{\integer{n}}}}}$\\
\nopagebreak 
 & $\disjunction{\parconjunction{\CAPPHI_1}{\conjunction{\ldots}{\CAPPHI_{\integer{n}}}}}{\CAPTHETA}$ & $\conjunction{\pardisjunction{\CAPPHI_1}{\CAPTHETA}}{\conjunction{\ldots}{\pardisjunction{\CAPPHI_{\integer{n}}}{\CAPTHETA}}}$\\
\nopagebreak
 & $\conjunction{\pardisjunction{\CAPPHI_1}{\CAPTHETA}}{\conjunction{\ldots}{\pardisjunction{\CAPPHI_{\integer{n}}}{\CAPTHETA}}}$ & $\disjunction{\parconjunction{\CAPPHI_1}{\conjunction{\ldots}{\CAPPHI_{\integer{n}}}}}{\CAPTHETA}$\\
%\midrule
\multicolumn{3}{l}{\textbf{Set 4: Basic Rules of GQD, table \mvref{GQD}}}\\
\nopagebreak
%\midrule
\Rule{$\forall$-Elim} & $\universal{\BETA}\CAPPHI$ & $\CAPPHI\constant{a}/\BETA$, for \mention{a} any  \\[-.25cm]
\nopagebreak
 &   &   individual constant \\
\Rule{$\forall$-Intro} & $\CAPPHI\constant{a}/\BETA$ & $\universal{\BETA}\CAPPHI$, iff \mention{a} does  \\[-.25cm]
 &  &  not occur in $\CAPPHI$  \\[-.25cm]
 &  & nor in any unboxed assumption \\
\Rule{$\exists$-Intro} & $\CAPPHI\constant{a}/\BETA$ & $\existential{\BETA}\CAPPHI$ \\
\Rule{$\exists$-Elim} & $\existential{\BETA}\CAPPHI$, $\horseshoe{\CAPPHI{\constant{a}/\BETA}}{\CAPTHETA}$ & $\CAPTHETA$, \Iff \mention{a} does \\[-.25cm]
\nopagebreak
 &  &  not occur in $\CAPPHI$ or $\CAPTHETA$, \\[-.25cm]
\nopagebreak
 & &  nor in any unboxed assumption\\
%\midrule
\multicolumn{3}{l}{\textbf{Set 5: Exchange Shortcut Rules of GQD, table \mvref{GQDplus}}}\\
\nopagebreak
%\midrule
\Rule{QN} & $\negation{\universal{\BETA}{\CAPPHI}}$ & $\existential{\BETA}\negation{{\CAPPHI}}$ \\
 & $\existential{\BETA}\negation{{\CAPPHI}}$ & $\negation{\universal{\BETA}{\CAPPHI}}$  \\
 & $\negation{\existential{\BETA}{\CAPPHI}}$ & $\universal{\BETA}\negation{{\CAPPHI}}$ \\
 &  $\universal{\BETA}\negation{{\CAPPHI}}$ & $\negation{\existential{\BETA}{\CAPPHI}}$ \\
%\midrule
\multicolumn{3}{l}{\textbf{Set 6: DNF Exchange Shortcut Rules for GSD, table \mvref{GSDplusDNF}}}\\
\nopagebreak
%\midrule
\Rule{$\TRIPLEBAR$-Exchange} &  $\triplebar{\CAPTHETA}{\CAPPSI}$ & $\disjunction{\parconjunction{\CAPTHETA}{\CAPPSI}}{\parconjunction{\negation{\CAPTHETA}}{\negation{\CAPPSI}}}$ \\
\nopagebreak
 & $\disjunction{\parconjunction{\CAPTHETA}{\CAPPSI}}{\parconjunction{\negation{\CAPTHETA}}{\negation{\CAPPSI}}}$ &  $\triplebar{\CAPTHETA}{\CAPPSI}$ \\
%\midrule
\multicolumn{3}{l}{\textbf{Set 7: Prenex Exchange Shortcut Rules for GQD, table \mvref{GSDplusPrenex}}}\\
\nopagebreak
%\midrule
\Rule{$\ALPHA$/$\BETA$-Exch} & $(\#\ALPHA)\CAPPHI$ & $(\#\BETA)\CAPPHI\BETA/\ALPHA$ \\
\Rule{Q Shuffling} & $\parconjunction{(\#\variable{x})\CAPTHETA}{\CAPPSI}$ & $(\#\variable{x})\parconjunction{\CAPTHETA}{\CAPPSI}$ \\
& $\parconjunction{\CAPTHETA}{(\#\variable{x})\CAPPSI}$ & $(\#\variable{x})\parconjunction{\CAPTHETA}{\CAPPSI}$ \\

& $\pardisjunction{(\#\variable{x})\CAPTHETA}{\CAPPSI}$ & $(\#\variable{x})\pardisjunction{\CAPTHETA}{\CAPPSI}$ \\
& $\pardisjunction{\CAPTHETA}{(\#\variable{x})\CAPPSI}$ & $(\#\variable{x})\pardisjunction{\CAPTHETA}{\CAPPSI}$ \\

& $\parhorseshoe{\CAPTHETA}{(\#\variable{x})\CAPPSI}$ & $(\#\variable{x})\parhorseshoe{\CAPTHETA}{\CAPPSI}$ \\

& $\parhorseshoe{\existential{\variable{x}}\CAPTHETA}{\CAPPSI}$ & $\universal{\variable{x}}\parhorseshoe{\CAPTHETA}{\CAPPSI}$ \\
& $\parhorseshoe{\universal{\variable{x}}\CAPTHETA}{\CAPPSI}$ & $\existential{\variable{x}}\parhorseshoe{\CAPTHETA}{\CAPPSI}$ \\
%\midrule
\multicolumn{3}{l}{\textbf{Set 8: The Method Shortcut Rules for GQD, table \mvref{GSDplusMethod}}}\\
\nopagebreak
%\midrule
\Rule{Greg's Rule} & $\disjunction{\CAPPSI_1}{\disjunction{\ldots}{\CAPPSI_{\integer{n}}}}$, where some & $\disjunction{\CAPPSI_1}{\disjunction{\ldots}{\disjunction{\CAPPSI_{\integer{i}-1}}{\disjunction{\CAPPSI_{\integer{i}+1}}{\disjunction{\ldots}{\CAPPSI_{\integer{n}}}}}}}$ \\[-.25cm]
 & $\CAPPSI_{\integer{i}}=\conjunction{\CAPPHI_1}{\conjunction{\ldots}{\conjunction{\CAPPHI_{\integer{j}}}{\ldots}}}$ & \\[-.25cm]
 & $\WEDGE\conjunction{\negation{\CAPPHI_{\integer{j}}}}{\conjunction{\ldots}{\CAPPHI_{\integer{m}}}}$ & \\
 
\Rule{$\VEE$/$\WEDGE\!$-Elim} & $\disjunction{\CAPPSI_{1}}{\disjunction{\ldots}{\CAPPSI_{\integer{n}}}}$, where & $\CAPPHI$ \\[-.25cm]
 & each $\CAPPSI_{\integer{i}}$ contains $\CAPPHI$ & \\
 
\Rule{OBA} &  $\conjunction{\CAPPHI_1}{\conjunction{\ldots}{\conjunction{\CAPPHI_i}{\conjunction{\ldots}{\CAPPHI_{\integer{n}}}}}}$, $\negation{\CAPPHI_i}$ & $\negation{\parconjunction{\CAPPHI_1}{\conjunction{\ldots}{\conjunction{\CAPPHI_i}{\conjunction{\ldots}{\CAPPHI_{\integer{n}}}}}}}$ \\
\nopagebreak
 & $\conjunction{\CAPPHI_1}{\conjunction{\ldots}{\conjunction{\negation{\CAPPHI_i}}{\conjunction{\ldots}{\CAPPHI_{\integer{n}}}}}}$, ${\CAPPHI_i}$ & $\negation{\parconjunction{\CAPPHI_1}{\conjunction{\ldots}{\conjunction{\negation{\CAPPHI_i}}{\conjunction{\ldots}{\CAPPHI_{\integer{n}}}}}}}$ \\
%\midrule
\multicolumn{3}{l}{\textbf{Set 9: Basic Rules for S5, table \mvref{SF}}}\\
\nopagebreak
%\midrule
\Rule{$\BOX\!$-Elim} & $\BOX\CAPPHI$ & $\CAPPHI$ \\
\Rule{$\BOX\!$-Intro} & $\CAPPHI$ (*) & $\BOX\CAPPHI$ \\
\Rule{$\DIAMOND\!$-Elim} & $\DIAMOND\CAPPHI$, $\horseshoe{\CAPPHI}{\CAPPSI}$ (*), (**) & $\CAPPSI$ \\
\Rule{$\DIAMOND\!$-Intro} &  $\CAPPHI$ & $\DIAMOND\CAPPHI$ \\
%\midrule
\multicolumn{3}{l}{\textbf{Set 10: Modal Negation Exchange Shortcut Rules for S5, table \mvref{SFMN}}}\\
\nopagebreak
%\midrule
\Rule{MN} & $\negation{\BOX\CAPPHI}$ & $\DIAMOND\negation{\CAPPHI}$ \\
 & $\negation{\DIAMOND\CAPPHI}$ & $\BOX\negation{\CAPPHI}$ \\
 & $\negation{\DIAMOND\negation{\CAPPHI}}$ & $\BOX\CAPPHI$ \\
 &  $\negation{\BOX\negation{\CAPPHI}}$ & $\DIAMOND\CAPPHI$ \\ 
%\midrule
\multicolumn{3}{l}{\textbf{Set 11: Basic Rules for GQDI, \mvref{GQDI}}}\\
\nopagebreak
%\midrule
\Rule{$=$-Intro} &  & $\variable{t}=\variable{t}$ \\
\Rule{$=$-Elim} & $\CAPPHI$, $\variable{t}=\variable{s}$ & $\CAPPHI\variable{t}/\variable{s}$ \\
\end{longtable}
\noindent{}(*) in \Rule{$\BOX\!$-Intro} and \Rule{$\DIAMOND\!$-Elim}, the rule can only be applied if all the open assumptions have modal prefixes.

\noindent{}(**) In \Rule{$\DIAMOND\!$-Elim}, the rule can only be applied if $\CAPPSI$ has a modal prefix.

%\theendnotes


\clearemptydoublepage
\addcontentsline{toc}{chapter}{Works Cited}
%\fancyhead[RE,LO]{\sffamily{}Works Cited}
\chead{\sffamily{}Works Cited}
\fancyhead[LE,RO]{\sffamily{}\thepage}
\setcounter{section}{0}
%\bibliography{C:/Users/Michael/Documents/writings/papers_and_books/annotated_bib/Bibliography}{}
\bibliography{Bibliography}{}
\bibliographystyle{philreview-m}
%bibstyle options (in usual directory): kluwer, agms, dcu, plainnat, chicago, astron
%my bibstyle options (must be placed in directory with tex file): 
    %analysis (styled used in journal analysis)
    %philreview (style used in journal phil review)
    %philreview-m (my modified version, this is prefered)
    %plainnat2 (my modified version of plainnat, this is 2nd choice)

\clearemptydoublepage
\addcontentsline{toc}{chapter}{Index}
%\fancyhead[RE,LO]{\sffamily{}Index}
\chead{\sffamily{}Index}
\fancyhead[LE,RO]{\sffamily{}\thepage}
\setcounter{section}{0}
\printindex

\end{document}

